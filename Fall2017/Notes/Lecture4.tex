 \documentclass[12pt,a4paper]{article}
%\usepackage{hyperref} % Use the Charter font for the document text
%\usepackage[UTF8]{ctex}
\usepackage{jheppub}

\usepackage{amsfonts,amssymb,amsmath}
\usepackage{mathtools}
\usepackage{tikz-cd}
\usepackage{tikz}
\usepackage{alltt}
\usepackage{amsfonts}
\usepackage{amsmath}
\usepackage{amssymb}
\usepackage{amsthm}
\usepackage{booktabs}
\usepackage{caption}
\usepackage{enumitem}
\usepackage{fancyhdr}
\usepackage{graphicx}
\usepackage{mathdots}
\usepackage{mathtools}
\usepackage{microtype}
\usepackage{multirow}
\usepackage{pdflscape}
\usepackage{pgfplots}
\usepackage{siunitx}
\usepackage{slashed}
\usepackage{tabularx}
\usepackage{tikz}
\usepackage{tkz-euclide}
\usepackage[normalem]{ulem}
\usepackage[all]{xy}
\usepackage{imakeidx}

\usepackage{wrapfig}


%%%今村セッテッティング%%%%%%%%%%%%%%%%%%%%%%%%%%%%%
\newcommand{\CC}{\mathbb{C}}
\newcommand{\ZZ}{\mathbb{Z}}
\newcommand{\RR}{\mathbb{R}}
\newcommand{\HH}{\mathbb{H}}

\newcommand{\hf}{\frac{1}{2}}
\newcommand{\tr}{{\rm tr}}
\newcommand{\ind}{{\rm ind}}
\newcommand{\ol}{\overline}
\newcommand{\ul}{\underline}
\newcommand{\up}{\uparrow}
\newcommand{\dn}{\downarrow}
\newcommand{\wt}{\widetilde}
\newcommand{\ra}{\rightarrow}
\newcommand{\wh}{\widehat}


%%%横山セッティング%%%%%%%%%%%%%%%%%%%%%%%%%%%%%%%%%
\newcommand{\NN}{\mathcal{N}\!}
\newcommand{\DD}{\mathcal{D}}
\newcommand{\UU}{U(1)}
\newcommand{\dd}{\mathrm{d}}
\renewcommand{\SS}{\mathbf{S}}
\renewcommand{\Im}{\mathrm{Im}}
\renewcommand{\Re}{\mathrm{Re}}
%\renewcommand{\<}{\langle}
\renewcommand{\>}{\rangle}
\newcommand{\Tr}{{\rm Tr}}

\renewcommand{\r}{\mathrm}

\newcommand{\sign}{\mathrm{sign}}

\newcommand{\lra}{\leftrightarrow}
\newcommand{\LL}{\mathcal{L}}
\newcommand{\la}{\leftarrow}
\newcommand{\ro}{\sqrt}
\newcommand{\Ra}{\Rightarrow}
\newcommand{\Pexp}{\mathrm{Pexp}}

\newcommand{\nn}{\nonumber \cr}
%\newcommand{\1}{\mbox{1}\hspace{-0.25em}\mbox{l}}

%数字のみ対応
\newcommand{\Maru}[1]{\ooalign{
\ifnum#1<10 \hfil\resizebox{.9\width}{.85\height}{#1}\hfil
\else
\hfil\resizebox{.6\width}{.8\height}{#1}\hfil
\fi
\crcr
\raise.1ex\hbox{$\bigcirc$}}}

%全文字対応
\newcommand{\maru}[1]{\ooalign{
\hfil\resizebox{.8\width}{\height}{#1}\hfil
\crcr
\raise.1ex\hbox{\large$\bigcirc$}}}


\newcommand{\nord}[1]{\vcentcolon\mathrel{#1}\vcentcolon}
\providecommand{\vcentcolon}{\mathrel{\mathop{:}}}


\def\P{\mathop{\cal P}}
\def\diag{\mathop{\rm diag}}


\def\Re{\mathop{\rm Re}\nolimits}
\def\Im{\mathop{\rm Im}\nolimits}
\def\Det{\mathop{\rm Det}\nolimits}
\def\sign{\mathop{\rm sign}\nolimits}


%%% rap %%% - make two letters overlap
\newcommand{\rap}[2]
{\setbox1=\hbox{#1}%
\setbox2=\hbox to\wd1{\hss #2\hss}%
\mbox{\rlap{\box1}\box2}}

%\newcommand{\sla}[1]{\rap{$#1$}{/}}
\newcommand{\sla}[1]{\rap{$#1$}{$\backslash$}}


\def\DY#1{{\MyGreen [DY: #1]}}
\newcommand{\MyGreen}{\color [rgb]{0,0.7,0}}

\usepackage[vcentermath]{youngtab}
% \Yboxdim4pt
\newcommand{\Y}{\yng}
\newcommand{\Young}{}


%%%title def%%%%%%%%%%%%%%%%%%%%%%%%%%%%%%%%%%%%%%%%%%%%%%%

\makeatletter
\def\maketitle{
\noindent
{\Large \@title \par\vskip 2pt}
\noindent
{\large \@date \hspace{4pt} \@author}
%\cr[-2pt]
%\noindent------------------------------------------------------------------------------------------
\par\vskip 1.5em
}

\author{横山 大輔}
\date{\today}
%%%本文%%%%%%%%%%%%%%%%%%%%%%%%%%%%%%%%%%%%%%%%%%%%%%%%%%%%%%%%

% \title{\centerline{Lecture 2}}
\begin{document}
% \maketitle
% \begin{abstract}

% \end{abstract}
% \tableofcontents

  \pagestyle{fancy}
  \renewcommand{\headrulewidth}{0.0pt}
  \rhead{}
  \lhead{}
  \cfoot{[\ \scshape\oldstylenums{\thepage}\ / %
    \scshape\oldstylenums{\pageref{lastpage}} ]}
%  \rfoot{\@author}

% \setcounter{section}{}
% \setcounter{section}{}
% \setcounter{subsection}{}

\centerline{\Large \bf  Lecture 4}

\vspace{12pt}
Since we learnt conformal field theories and techniques in the theories,
we would like to connect them to string theory, especially by considering vertex operators.

In conformal field theory the conformal symmetry is essential.
However, classical conformal symmetry, which is related to WS diff and Weyl sym, breaks at quantum level.
We regard this anomaly Weyl anomaly. We will look into important examples of the anomaly.

% \textbf{Comment:} we could not talk about Weyl anomaly from curved WS and the reason why dilaton vacuum expectation value
% plays the role of string in the actual 4th lecture.
\vspace{-12pt}




\section{Vertex operators}

% \DY{Vertex operator が適切な field op で表せることを書いて、
% 次節へのつなぎとしたい。}


Let us consider the vertex operators, which has been postponed to do from the second lecture.
We learnt the state-operator correspondence, which tells us a state corresponds to a certain local operator (vertex operator).
We also learnt that a closed string spectrum includes tachyon, graviton, etc.
Why not consider the corresponding local operators of the states ?


Recall a few spectra in a closed string:
\begin{align}\nonumber
\renewcommand\arraystretch{1.2}
 \begin{array}{ll}
  \textrm{Tachyon}\ \phi & |0;k \rangle \ , \\[2pt]
  \textrm{Graviton}\  G^{\mu\nu} & \alpha^\mu_{-1}\wt\alpha^\nu_{-1} |0;k \rangle  \quad (\textrm{symmetric in $\mu$ and $\nu$, and traceless} ) \ , \\[2pt]
  \textrm{B-field}\ B^{\mu\nu} & \alpha^\mu_{-1}\wt\alpha^\nu_{-1} |0;k \rangle  \quad (\textrm{anti-symmetric in $\mu$ and $\nu$} ) \ , \\[2pt]
  \textrm{Dilaton}\ \Phi & \alpha^\mu_{-1}\wt\alpha_{-1,\mu} |0;k \rangle \ .
 \end{array}
\end{align}
Note that in order to extract physical states we need to contract the polarization tensor $\zeta_{\mu\nu}$,
which satisfies $k^\mu \zeta_{\mu\nu} = 0$ ($\zeta_{\mu\nu}^{G} = \zeta_{\nu\mu}^{G}$ and $\zeta_{\mu}^{G,\mu} = 0$ etc.).
Let us first focus on the tachyon state, which is nothing but a vacuum state with a certain momentum $k^\mu$.
It was defined by $p |0;k \rangle = k |0;k \rangle$, namely,
\begin{align}\nonumber
 |0;k \rangle = e^{ik\cdot x} |0;0 \rangle
 \quad \ra \quad
 e^{ik\cdot X(0,0)} \ .
\end{align}
The final replacement is by the state-operator correspondence.


Next we consider the first excited states.
Remind ourselves that the closed string mode expansion is given by
\begin{align}\nonumber
 X^\mu(z,\ol z) &= X_R^\mu (z) +X_L^\mu (\ol z)   ,   \cr
 X_R^\mu (z) &= \frac{1}{2}x^\mu -i \frac{\alpha'}{2} p^\mu \log z +i \sqrt{ \frac{\alpha'}{2} } \sum_{n \neq 0} \frac{1}{n} \frac{\alpha_n^\mu}{z^n} \ .
\end{align}
Therefore, each mode can be expressed as follows.
\begin{align}\nonumber
 \alpha^\mu_{-m} = i\sqrt{\frac{2}{\alpha'}} \oint \frac{dz}{2\pi i} z^{-m} \partial X^\mu(z)
 \quad \ra \quad
 i\sqrt{\frac{2}{\alpha'}} \frac{1}{(m-1)!} \partial^m X(0)  .
\end{align}
At the last replacement we forgot the ``classical'' mode expansion and regarded $\partial X(z)$ as a local operator.
Now we have the correspondence of tachyon, graviton etc. as follows.
\begin{align}\nonumber
 &|0;k\rangle  \quad \ra \quad  e^{ikX}  \\[2pt]
%  &\alpha^\mu_{-n} |0;k\rangle  \ra   \partial^n X^\mu e^{ik\cdot X}  \cr
 &\zeta_{\mu\nu}\alpha^\mu_{-1}\wt\alpha^\nu_{-1} |0;k\rangle \quad \ra \quad \zeta_{\mu\nu} \partial X^\mu \ol\partial X^\nu e^{ikX}
\end{align}


The string amplitude is
\begin{align}\nonumber
 A_n = \sum_g \int \left(\DD h_{ab}\right)_{g,n} \int \DD X^\mu e^{-S_\sigma [X^\mu,h_{ab}]} \wh V_1 \cdots \wh V_n \ .
\end{align}
Note that, naively say, 
\begin{align}\nonumber
 \left(\DD h_{ab}\right)_{g,n} = \left(\DD h_{ab}\right)_{g,0} d^2z_1 \cdots d^2z_n
\end{align}
(naively say, this is because Weyl rescaling can move points on the Riemann surface to anywhere),
so we can re-write the amplitude as
\begin{align}\nonumber
 A_n = \sum_g \int \left(\DD h_{ab}\right)_{g,n} \int \DD X^\mu e^{-S_\sigma [X^\mu,h_{ab}]} \prod_{i=1}^n \int d^2z \sqrt h V_i  \ .
\end{align}
Here $\int d^2z \sqrt h V_i$ is an operator of the CFT, and hence, it must be Weyl \& WS diff $\sim$ conformal invariant.

\subsection{Mass from vertex operator}
Now, let us consider a constant scaling $z \ra \lambda z$ and $\ol z \ra \ol\lambda \ol z$.
Under the scaling, a field transforms as $\phi(z,\ol z) \ra \lambda^{-h}\ol\lambda^{-\ol h} \phi(z,\ol z)$,
which should compensate the scaling of the measure $dzd\ol z \ra \lambda\ol\lambda dzd\ol z$.
Namely, $h=\ol h=1$.
Conformal dimension of the vertex operators are
\begin{align}\nonumber
\renewcommand\arraystretch{1.2}
 \begin{array}{lll}
  \textrm{Name} & \mathcal O & (h, \ol h) \\\hline
  \textrm{Tachyon} & e^{ik\cdot X} & (\frac{\alpha'k^2}{4},\frac{\alpha'k^2}{4}) \\
  \textrm{1st excited states} & \zeta_{\mu\nu} \partial X^\mu \ol\partial X^\nu e^{ikX} & (1+\frac{\alpha'k^2}{4},1+\frac{\alpha'k^2}{4}) \\
 \end{array}
\end{align}
The consistency condition for the tachyon leads that
\begin{align}\nonumber
 M^2 = -k^2_\textrm{Tachyon} = -\frac{4}{\alpha'} \ .
\end{align}
Similarly, that for the first excited states leads
\begin{align}\nonumber
 M^2 = -k^2_\textrm{1st} = 0 \ .
\end{align}
Both results are consistent with the analysis from the first (and the second) lecture.




\section{Weyl anomaly}

As we have seen the Weyl symmetry ($T^a_a = 0$) is crucial.
However, in general matter theories, the symmetry has anomaly.
An important example is a string sigma model that is extended to curved space-time(ST):
\begin{align}\nonumber
 S = \frac{1}{4\pi \alpha'}\int \sqrt h d^2z\ h^{ab} \partial_a X^\mu \partial_b X^\nu g_{\mu\nu}(X) \ .
\end{align}
This is called (string) non-linear sigma model(NLSM).
The anomaly is characterized by $\beta$-functions.

Even if the matter theory has no anomaly on flat world-sheet(WS), $\RR^2$,
the same theory has anomaly on curved WS.
As we will see it is characterized by the central charge $T^a_a = -\frac{c}{12} R^{(2)}$.
Therefore, the string sigma model has anomaly.
However, the path integral for the metric $h_{ab}$ turns into
that of ``ghost CFT'' due to gauge-fixing,
and it leads $c=-26$.
This means that the matter theory has to have $c=26$ !


In total we have two types of anomalies:
\begin{enumerate}
 \item Anomaly from curved WS $T^a_a = -\frac{c}{12} R^{(2)}$ ,
 \item Anomaly from curved ST $\beta[G_{\mu\nu}] \neq 0$ .
\end{enumerate}
We will learn these anomalies in this order
(ghost CFT will be covered in the next lecture).



\section{Weyl anomaly from curved WS}

As is stated $T^a_a = A \neq 0$.
The form of $A$ is highly restricted by symmetries: $A$ should be
\begin{itemize}
 \item WS diff invariant ,
 \item zero on $\RR^2$ ,
 \item WS mass dimension two .
\end{itemize}
These restrictions lead to
\begin{align}\nonumber
 T^a_a = a R^{(2)} \ ,
\end{align}
where $R^{(2)}$ is a WS Ricci scalar.

Let us derive $T^a_a = -\frac{c}{12} R^{(2)}$.
Take the WS to be conformally flat (which is always possible using WS diff)
\begin{align}\nonumber
 ds^2 = e^{2\Omega(x,y)} (dxdx+dydy) = e^{2\Omega(z,\ol z)} dzd\ol z \ .
\end{align}
For the metric we have
\begin{align}\nonumber
 R^{(2)} = -2 \partial^a \partial_a \Omega \ ,
\end{align}
therefore, the diagonal element of the EM tensor becomes
\begin{align}\nonumber
 T_{z\ol z} = \frac{1}{4} e^{2\Omega}\ T^a_a = -2a \partial\ol\partial \Omega \ .
\end{align}
The other components can be derived from $\nabla_a T^a_b = 0$.
\begin{align}\nonumber
 &0 = \nabla^z T_{zz} +\nabla^{\ol z} T_{\ol zz} = \nabla_{\ol z} T_{zz} +\nabla_{z} T_{\ol zz}
 = \ol\partial \left( T_{zz} -2a\left( \partial\partial\Omega -\partial\Omega\partial\Omega \right) \right)  \cr
 &\therefore \quad T_{zz} = 2a\left( \partial\partial\Omega -\partial\Omega\partial\Omega \right)+T(z) \qquad (T(z): \textrm{ holomorphic})
\end{align}



Let us recall that the conformal Ward-Takahashi identity \textit{in flat WS}:
\begin{align}\nonumber
 \delta_{\epsilon,\ol\epsilon} \mathcal O (w,\ol w) &= \frac{1}{2\pi i} \oint_{\partial M}
 \left\{ dz\ \epsilon(z)T(z) - d\ol z\ \ol\epsilon(\ol z) \ol T(\ol z) \right\} \mathcal O (w,\ol w) \nonumber\cr
 &= \int_{M}\frac{d^2z}{2\pi }
 \left\{ \ol\partial\left(\epsilon(z)T(z)\right) +\partial\left(\ol\epsilon(\ol z) \ol T(\ol z)\right) \right\} \mathcal O (w,\ol w) \ .
\end{align}
(Notice that the convention is different from the last lecture.)
Note that EM tensor is conserved, $\partial^a T_{ab} = 0$, equivalently,
$\ol\partial T(z) = 0 = \partial \ol T(\ol z)$ with
$T(z) = T_{zz}$, $\ol T(\ol z) = T_{\ol z\ol z}$, and $T_{z\ol z} = 0$.
On the other hand, \textit{in the curved WS},
the current conservation is expressed with a covariant derivative
$\nabla^a T_{ab} = 0$.
For example,
\begin{align}
 &\nabla^a T_{az} = \nabla^z T_{zz} +\nabla^{\ol z} T_{\ol zz} = \ol\partial T(z) = 0  \cr
 &\textrm{with} \quad T(z) = T_{zz} - 2a\left( \partial\partial\Omega -\partial\Omega\partial\Omega \right) \ .
 \label{eq:curvedEM}
\end{align}
Similar for $\ol T(\ol z)$.
Hence, we have the same form of conformal Ward-Takahashi identity.
Especially,
\begin{align}\nonumber
 \delta_\epsilon T(z) = \epsilon(z)\partial T(z) +2\partial\epsilon(z) T(z) +\frac{c}{12} \partial^3 \epsilon(z) \ .
\end{align}
On the other hand, from the expression (\ref{eq:curvedEM}) we have
\begin{align}
 \delta_\epsilon T(z) = \delta_\epsilon T_{zz}(z) -2a \left( \partial\partial\delta_\epsilon\Omega -2\partial\Omega\partial\delta_\epsilon\Omega \right) \ .
 \label{eq:curvedTransfT}
\end{align}
Using (finite) transformations:
\begin{align}\nonumber
 &z \to \wt z = z -\epsilon(z) \ , \cr
 &T_{zz}(z) \to \wt T_{\wt z\wt z}(\wt z) = \left( \partial_{z} \wt z \right)^{-2} T_{zz}(z) \ , \cr
 &\Omega(z) \to \wt \Omega (\wt z) = \Omega(z) -\frac{1}{2} \log \left| \partial_z \wt z \right|^2 \ ,
\end{align}
the transformation of (\ref{eq:curvedTransfT}) becomes
\begin{align}\nonumber
 \delta_\epsilon T(z) = \epsilon(z)\partial T(z) +2\partial\epsilon(z) T(z) -a \partial^3 \epsilon(z) \ .
\end{align}
Therefore,
\begin{align}\nonumber
 T^a_a = aR^{(2)} = -\frac{1}{12} c R^{(2)}  .
\end{align}





\section{Non-linear sigma model (Graviton included)}

% \DY{ST metric を X dependent にすることがなぜ正当化されるのか書いて、
% その一般化を書く。}


So far we only consider a flat space-time(ST).
However, if we expect that the string theory
describe general relativity
the action should contain general metric $G_{\mu\nu}(X)$:
\begin{align}\nonumber
 S[X^\mu, h_{ab}] = \frac{1}{4\pi \alpha'} \int d^2x \left( \sqrt h h^{ab} \partial_a X^\mu \partial_b X^\nu G_{\mu\nu}(X)
 \right) \ .
\end{align}
An action of this shape is called non-linear sigma model(NLSM).
Now let us consider an almost flat metric:
\begin{align}\nonumber
 G_{\mu\nu}(X) = \eta_{\mu\nu} + f_{\mu\nu} (X) \ .
\end{align}
Then, the partition function becomes
\begin{align}\nonumber
 Z &= \int \DD h_{ab} \DD X^\mu e^{-S} \nonumber\cr
 &= \int \DD h_{ab} \DD X^\mu e^{-S_0} \left(
 1+\frac{1}{4\pi \alpha'} \int d^2x \left( \sqrt h h^{ab} \partial_a X^\mu \partial_b X^\nu f_{\mu\nu}(X) \right) +\cdots
 \right) \ .
\end{align}
Notice that the perturbative part is nothing but the graviton operator
with wave function $f_{\mu\nu}(X) = \zeta_{\mu\nu} e^{ik\cdot X}$.


Let us again consider general case $G_{\mu\nu}(X)$.
As a 2d field theory we can consider the vacuum expectation value(vev) for $X$,
which we set to $X_0$:
\begin{align}\nonumber
 \hat X(\sigma,\tau) = X_0 +X(\sigma,\tau) \ .
\end{align}
On the othat hand, $X_0$ is a certain point in ST and
we will expand the metric around this point.
If you choose the coordinate nicely (Riemann normal coordinate)
the expansion of the metric can be written as follows.
\begin{align}\nonumber
 G_{\mu\nu} (X) = G_{\mu\nu} -\frac{1}{3} R_{\mu\lambda\nu\rho} X^\lambda X^\rho +\mathcal O(X^3) \ ,
\end{align}
where $G_{\mu\nu}$ and $R_{\mu\lambda\nu\rho}$ are a metric and a Rimann tensor at $X_0$, respectively.
In a field theory sense those are coupling constants:
\begin{align}\nonumber
 S &= \frac{1}{4\pi \alpha'} \int d^2x \ \partial^a X^\mu \partial_a X^\nu
 \left( G_{\mu\nu} -\frac{1}{3} R_{\mu\lambda\nu\rho} X^\lambda X^\rho +\cdots
 \right)  \nonumber\cr
 &= \frac{1}{2} \int d^2x \ \partial^a X^\mu \partial_a X^\nu
 \left( G_{\mu\nu} -\frac{2\pi \alpha'}{3} R_{\mu\lambda\nu\rho} X^\lambda X^\rho +\mathcal O(\alpha'^2)
 \right) \ .
\end{align}
We rescaled the field $X \to \sqrt{2\pi \alpha'} X$ so that
the expansion looks like ``stringy expansion''.






\section{Perturbation theory for NLSM (Weyl anomaly from curved ST)}

% \DY{Perturbation を考える。}
% \DY{Beta-functions がどうして Weyl anomaly と関係するのかを書く。}

Note that discussion here is very naive and for the details one should
consult \cite{Callan:1988xx}.


\vspace{12pt}
We want to check if the theory (NLSM) has Weyl anomaly.
As we briefly saw it is an intracting theory, and in general,
intracting theories have non-trivial $\beta$-functions:
\begin{align}\nonumber
 \beta[\lambda] \equiv E \frac{\partial}{\partial E} \lambda(E) = \frac{\partial}{\partial (\log E)} \lambda(E) \ ,
\end{align}
where $\lambda$ is a coupling constant and $E$ is a characteristic energy scale.
When we consider a global scaling of coordinate:$z \to \wt z = s z = (1-\epsilon) z = e^{-\epsilon} z$,
enegy scales oppositely: $E \to \wt E = \frac{1}{s} E = e^{\epsilon} E$.
So the $\beta$-function can be written as
\begin{align}\nonumber
 \beta[\lambda] = \frac{\partial}{\partial \epsilon} \lambda(\epsilon) \ .
\end{align}
The variation of the action is expressed in two ways:
\begin{align}\nonumber
 &\delta_\epsilon S =
 \begin{cases}
  \int \frac{d^2x}{2\pi} \sqrt{h} \delta_\epsilon h^{ab}  T_{ba} = -\epsilon \int \frac{d^2x}{2\pi} T^a_a \ , \\[2pt]
  \frac{1}{4\pi \alpha'} \int d^2x \ \partial^a X^\mu \partial_a X^\nu
  \left( \epsilon \frac{\partial}{\partial \epsilon} G_{\mu\nu}(\epsilon) +\cdots \right) \ ,
 \end{cases}
\end{align}
where the first variation is a formal transformation of the theory, and in quantum regime,
it should be proportional to the trace part of the EM tensor.
On the other hand, the second variation is the actual theory with an assumption that
$\epsilon$ dependence of the theory is only in the coupling constants.
Identifying them we have
\begin{align}\nonumber
  &\therefore \qquad T^a_a = -\frac{1}{2\alpha'} \beta[G_{\mu\nu}] \partial^b X^\mu \partial_b X^\nu +\cdots \ .
\end{align}
This shows that the anomaly is parametrized by $\beta$-functions.


Let us consider perturbation theory, namely
loop corrections to two-point function etc, so that we can see if the theory is anomalous.
\begin{align}\nonumber
 &\langle X^\mu (x_1) X^\nu (x_2) \rangle \nonumber\cr & \qquad =
 \int \frac{d^2k}{(2\pi)^2} \frac{2\pi \alpha'}{k^2} e^{ik\cdot (x_1-x_2)}
 \left\{ G^{\mu\nu} +\frac{2\pi \alpha'}{3} R^{\mu\nu} \left(
 \int \frac{d^2p}{(2\pi)^2} \frac{1}{p^2} + \frac{1}{k^2}\int \frac{d^2p}{(2\pi)^2}
 \right)+\cdots \right\} \ .
\end{align}
We further focus on the logarithmic divergence and introduce regularization parameters:
\begin{align}\nonumber
 \int_E^\Lambda \frac{d^2p}{(2\pi)^2} \frac{1}{p^2} = \frac{1}{2\pi} \log
 \left( \frac{\Lambda}{E} \right) \ ,
\end{align}
where $\Lambda$ is an ultra-violet(UV) energy scale supposed to be $\infty$,
and $E$ is an infra-red(IR) energy scale supposed to be our life energy scale, which is very law ($\sim 0$).

The divergence can be subtracted by counter terms as follows.
Define $\hat S = S +S_\mathrm{ct}$, which is called bare action:
\begin{align}\nonumber
 \hat S = \frac{1}{4\pi \alpha'} \int d^2x \ \partial^a \hat X^\mu \partial_a \hat X^\nu
 \left( \hat G_{\mu\nu} -\frac{1}{3} \hat R_{\mu\lambda\nu\rho} \hat X^\lambda \hat X^\rho +\cdots
 \right)
\end{align}
with
\begin{align}\nonumber
 &\hat X^\mu = Z^\mu_\nu X^\nu \ , \quad Z^\mu_\nu = \delta^\mu_\nu +
 \sum_{n=1}^\infty \alpha'^n Z_{(n), \nu}^\mu (\Lambda/E) \ , \cr
 &\hat G_{\mu\nu} = G_{\mu\nu} +\sum_{n=1}^\infty \alpha'^n G^{(n)}_{\mu\nu} (\Lambda/E) \ ,
 \textrm{etc.}
\end{align}
Physical action is $S$, which describes IR physics of energy scale $E$, on the other hand, $\hat S$ is called
bare action, which describes UV physics of energy scale $\Lambda$.
Note that the bare action only depends on $\Lambda$ (not on $E$), and hence,
the bare coupling constants ($\hat G_{\mu\nu}$ etc) only depends on high energy $\Lambda$.
The counter terms leads other contributions
\begin{align}
 &\langle X^\mu (x_1) X^\nu (x_2) \rangle \nonumber\\ & \qquad \sim
 \left\{ G^{\mu\nu} +\frac{\alpha'}{3} R^{\mu\nu} \log \left( \frac{\Lambda}{E} \right)
 -\alpha' \left(G_{(1)}^{\mu\nu} +Z_{(1)}^{\mu\nu} +Z_{(1)}^{\nu\mu}
 \right)+\cdots \right\} \ .
 \label{eq:renom2pt}
\end{align}
Unfortunately, the equation above cannot fix the ration between $G_{(1)}$ and $Z_{(1)}$.
We need further information like 4-pt function etc.
to determine the ratio.
We simply list the result:
\begin{align}\nonumber
 &G_{\mu\nu}^{(1)} = R_{\mu\nu} \log \left( \frac{\Lambda}{E} \right) \ , \quad
 Z_{\mu\nu}^{(1)} =  -\frac{1}{3} R_{\mu\nu} \log \left( \frac{\Lambda}{E} \right) \ ,
\end{align}
which does cancel the divergent term in (\ref{eq:renom2pt}).
From the result we can derive the $\beta$-function:
\begin{align}\nonumber
 &G_{\mu\nu} (E,\Lambda) = \hat G_{\mu\nu} (\Lambda) - \alpha' R_{\mu\nu} \log \left( \frac{\Lambda}{E} \right) \ , \cr
 &\beta[G_{\mu\nu}] = \frac{\partial}{\partial (\log E)} G_{\mu\nu} (E,\Lambda) = \alpha' R_{\mu\nu} \ .
\end{align}
Therefore, in order for the theory to be anomaly free we need the space-time to be Ricci flat ($R_{\mu\nu}=0$).
This is equivalent for Hilbert-Einstein action to satisfy its E.O.M.



\section{NLSM (general)}

So far only the graviton has been included but not B-field or dilaton.
Here, we simply give the action and their $\beta$-functions.
\begin{align}\nonumber
 &S = \frac{1}{4\pi \alpha'} \int d^2x \left( \sqrt h h^{ab} \partial_a X^\mu \partial_b X^\nu G_{\mu\nu}(X)
 +i \varepsilon^{ab} \partial_a X^\mu \partial_b X^\nu B_{\mu\nu}(X)
 +\alpha' \sqrt h R^{(2)} \Phi(X)
 \right) \ .
\end{align}
Weyl anomaly:
\begin{align}\nonumber
 T^a_a = -\frac{1}{2\alpha'} \beta[G_{\mu\nu}] \partial_a X^\mu \partial_b X^\nu
 -\frac{i}{2\alpha'} \beta[B_{\mu\nu}] \varepsilon^{ab} \partial_a X^\mu \partial_b X^\nu
 -\frac{1}{2} \beta [\Phi] R^{(2)}
\end{align}
$\beta$-functions:
\begin{align}\nonumber
 &\beta[G_{\mu\nu}] = \alpha' R_{\mu\nu} +2\alpha' \nabla_\mu \nabla_\nu \Phi
 -\frac{\alpha'}{4}H_{\mu\lambda\rho}H_\nu^{\ \lambda\rho} +\mathcal O(\alpha'^2) \ , \cr
 &\beta[B_{\mu\nu}] = -\frac{\alpha'}{2} \nabla^\lambda H_{\lambda\mu\nu}
 +\alpha' \nabla^\lambda \Phi H_{\lambda\mu\nu} +\mathcal O(\alpha'^2) \ , \cr
 &\beta[\Phi] = \frac{D-26}{6} -\frac{\alpha'}{2} \nabla^2\Phi
 +\alpha' \nabla^\lambda\Phi \nabla_\lambda\Phi
 -\frac{\alpha'}{24}H_{\mu\lambda\rho}H^{\mu\lambda\rho} +\mathcal O(\alpha'^2) \ .
\end{align}
There are a few remarks on them:
\begin{itemize}

 \item String theory to be consistent requires all the $\beta$-functions to be zero.

 \item Note that 2d gravity has no physical D.O.F, and the Einstein-Hilbert action gives a topological number(Euler characteristic).
       So if the dilaton is constant we have
  \begin{align}\nonumber
   S_\mathrm{dilaton} &= \frac{1}{4\pi \alpha'} \int d^2x \left( \alpha' \sqrt{h} R^{(2)} \Phi(X) \right)  \nonumber\cr
   &\to \frac{1}{4\pi} \int d^2x \left( \sqrt{h} R^{(2)} \Phi \right)
   = \Phi (2-2g)
  \end{align}
       Define $g_\mathrm{str} = e^\Phi$ and the action leads
  \begin{align}\nonumber
   e^{-S_\mathrm{dilaton}} = g_\mathrm{str}^{2g-2} \ .
  \end{align}
       $g_\mathrm{str}$ can be understood as a string coupling as follows.
       See Fig.~\ref{stringCoupling2.eps}. $n$-point string tree amplitude can be understood $n-2$ cylinders attaching to a cylinder.
\begin{figure}[htb]
\centerline{\includegraphics[width=400pt]{stringCoupling2.eps}}
 \caption{4-pt amplitude example. The upper one is a construction of 4pt tree amplitude from cylinders. The lower one is a construction of 4-pt 1-loop amplitude from the tree amplitude.}
\label{stringCoupling2.eps}
\end{figure}
       So the amplitude should proportional to $g_\mathrm{str}^{n-2}$.
       Higher loop (higher genus) amplitude can be derived by attaching g cylinders to the tree amplitude,
       and the amplitude should be $\hat A_{n,g} \propto g_\mathrm{str}^{n-2+2g}$.
       Usually, vertex operators are re-normalized so that $g_\mathrm{str}^n$ is included in the definition of $V_1 \cdots V_n$.
       Therefore, the re-normalized amplitude should be
  \begin{align}\nonumber
   A_{n,g} \propto g_\mathrm{str}^{2g-2} \ ,
  \end{align}
       which coincide with the dilaton action.

 \item Trivial $\beta$-function ($\beta=0$) is equivalent to the E.O.M of the following ST action.
  \begin{align}\nonumber
   S_\mathrm{eff} = \frac{1}{2\kappa_D^2} \int d^D X \sqrt{-G} e^{-2\Phi} \left[
   \frac{2(D-26)}{3\alpha'} +R -\frac{1}{12} H_{\mu\lambda\rho}H^{\mu\lambda\rho}
   +4 \nabla^\lambda\Phi \nabla_\lambda\Phi +\mathcal O(\alpha'^2)
   \right] \ .
  \end{align}

 \item B-field is a higher dimensional analogy of gauge fields.
       It has gauge transformation
  \begin{align}\nonumber
   \delta B_{\mu\nu} = \partial_\mu \Lambda_\nu -\partial_\nu \Lambda_\mu \ ,
  \end{align}
       and field strength $H_{\mu\nu\lambda}$:
   \begin{align}\nonumber
    H_{\mu\nu\lambda} = \partial_\mu B_{\nu\lambda}
    +\partial_\nu B_{\lambda\mu} +\partial_\lambda B_{\mu\nu} \ .
   \end{align}
       B-field plays an important role with open string.

\end{itemize}




\begin{thebibliography}{CDLOGP91}

% %\cite{AlvarezGaume:1981hn}
% \bibitem[AFM81]{AlvarezGaume:1981hn}
%   L.~Alvarez-Gaume, D.~Z.~Freedman and S.~Mukhi,
%   ``The Background Field Method and the Ultraviolet Structure of the Supersymmetric Nonlinear Sigma Model,''
%   Annals Phys.\  {\bf 134}, 85 (1981).
%   %doi:10.1016/0003-4916(81)90006-3
%   %%CITATION = doi:10.1016/0003-4916(81)90006-3;%%
%   %525 citations counted in INSPIRE as of 09 Oct 2017

 \bibitem[CT88]{Callan:1988xx}
               Curt Callan and Lárus Thorlacius.
               \textit{SIGMA MODELS AND STRING THEORY.}
               TASI Lecture, 1988.
               (The link below is a direct link to the pdf file of 45MB )
               \href{http://www.damtp.cam.ac.uk/user/tong/string/sigma.pdf}
               {http://www.damtp.cam.ac.uk/user/tong/string/sigma.pdf}.




\end{thebibliography}


\label{lastpage}

% \begin{tikzpicture}[>=stealth,scale=1]
%  \draw[->] (0,0)--(1,0);
%  \draw[latex-stealth] (0,0.5)--(1,0.5);
%  \draw[latex-stealthnew,arrowhead=2mm] (0,1)--(1,1);
% \end{tikzpicture}

% \begin{figure}[htb]
% \centerline{\includegraphics[width=250pt]{.eps}}
% \caption{}
% \label{.eps}
% \end{figure}



% \begin{table}[htbp]
%  \begin{center}
%   \caption{}
%   \vspace{4pt}
%   \label{table:001}
% \begin{tabular}{|c|c|c|c|c|}
% \hline
% \hline
%   Category & Sector & $(h_A,h_B,h_T)$ & Mirror theory & ABJM model \cr
% \hline
%  1 & & \parbox{40pt}{$(0,0,0)$ $(1,1,1)$} & $1.11906$ & $1.13290$ \cr
% \hline
%  2 & & \parbox{40pt}{$(0,0,1)$ $(1,1,0)$} & $-0.10861$ & $-0.10861$ \cr
% \hline
%  \multirow{2}{*}{\vspace{-15pt}3} & 3-1 & \parbox{40pt}{$(0,1,0)$ $(1,0,1)$} & $0.176777$ &
%  \multirow{2}{*}{\vspace{-15pt}$0.176577$} \cr
% \cline{2-4}
%  & 3-2 & \parbox{40pt}{$(1,0,0)$ $(0,1,1)$} & $0.176777$ & \cr
% \hline
% \hline
% \end{tabular}
% \end{center}
% \end{table}


% \begin{thebibliography}{99}

% % \cite{Imamura:2012rq}
% \bibitem{Imamura:2012rq}
%   Y.~Imamura and D.~Yokoyama,
%   %``S^3/Z_n partition function and dualities,''
%   JHEP {\bf 1211}, 122 (2012)
%   [arXiv:1208.1404 [hep-th]].
%   %%CITATION = ARXIV:1208.1404;%%

 % \bibitem{fnorio:legendre}
 %         fnorio
 %         ``ルジャンドル変換とは何か''
 %         \url{http://fnorio.com/0146Legendre_transformation/Legendre_transformation.html}


 % \bibitem{EMAN:dynamics}
 %         EMAN物理学
 %         ``ハミルトニアン''
 %         \url{http://eman-physics.net/analytic/hamilton.html}


 % \bibitem{Wiki:legendre}
 %         Wiki
 %         ``ルジャンドル変換''
 %         \url{https://ja.wikipedia.org/wiki/ルジャンドル変換}

 % \bibitem{mathtrain:legendre}
 %         高校数学の美しい物語
 %         ``ルジャンドル変換の意味と具体例''
 %         \url{http://mathtrain.jp/legendrehenkan}

% \end{thebibliography}


\bibliography{dd}

\end{document}
