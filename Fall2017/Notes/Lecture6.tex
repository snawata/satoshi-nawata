 \documentclass[12pt,a4paper]{article}
%\usepackage{hyperref} % Use the Charter font for the document text
%\usepackage[UTF8]{ctex}
\usepackage{jheppub}

\usepackage{amsfonts,amssymb,amsmath}
\usepackage{mathtools}
\usepackage{tikz-cd}
\usepackage{tikz}
\usepackage{alltt}
\usepackage{amsfonts}
\usepackage{amsmath}
\usepackage{amssymb}
\usepackage{amsthm}
\usepackage{booktabs}
\usepackage{caption}
\usepackage{enumitem}
\usepackage{fancyhdr}
\usepackage{graphicx}
\usepackage{mathdots}
\usepackage{mathtools}
\usepackage{microtype}
\usepackage{multirow}
\usepackage{pdflscape}
\usepackage{pgfplots}
\usepackage{siunitx}
\usepackage{slashed}
\usepackage{tabularx}
\usepackage{tikz}
\usepackage{tkz-euclide}
\usepackage[normalem]{ulem}
\usepackage[all]{xy}
\usepackage{imakeidx}

\usepackage{wrapfig}

%%%今村セッテッティング%%%%%%%%%%%%%%%%%%%%%%%%%%%%%
\newcommand{\CC}{\mathbb{C}}
\newcommand{\ZZ}{\mathbb{Z}}
\newcommand{\RR}{\mathbb{R}}
\newcommand{\HH}{\mathbb{H}}

\newcommand{\hf}{\frac{1}{2}}
\newcommand{\tr}{{\rm tr}}
\newcommand{\ind}{{\rm ind}}
\newcommand{\ol}{\overline}
\newcommand{\ul}{\underline}
\newcommand{\up}{\uparrow}
\newcommand{\dn}{\downarrow}
\newcommand{\wt}{\widetilde}
\newcommand{\ra}{\rightarrow}
\newcommand{\wh}{\widehat}


%%%横山セッティング%%%%%%%%%%%%%%%%%%%%%%%%%%%%%%%%%
\newcommand{\NN}{\mathcal{N}\!}
\newcommand{\DD}{\mathcal{D}}
\newcommand{\UU}{U(1)}
\newcommand{\dd}{\mathrm{d}}
\renewcommand{\SS}{\mathbf{S}}
\renewcommand{\Im}{\mathrm{Im}}
\renewcommand{\Re}{\mathrm{Re}}
%\renewcommand{\<}{\langle}
\renewcommand{\>}{\rangle}
\newcommand{\Tr}{{\rm Tr}}

\renewcommand{\r}{\mathrm}

\newcommand{\sign}{\mathrm{sign}}

\newcommand{\lra}{\leftrightarrow}
\newcommand{\LL}{\mathcal{L}}
\newcommand{\la}{\leftarrow}
\newcommand{\ro}{\sqrt}
\newcommand{\Ra}{\Rightarrow}
\newcommand{\Pexp}{\mathrm{Pexp}}

\newcommand{\nn}{\nonumber \cr}
%\newcommand{\1}{\mbox{1}\hspace{-0.25em}\mbox{l}}

%数字のみ対応
\newcommand{\Maru}[1]{\ooalign{
\ifnum#1<10 \hfil\resizebox{.9\width}{.85\height}{#1}\hfil
\else
\hfil\resizebox{.6\width}{.8\height}{#1}\hfil
\fi
\crcr
\raise.1ex\hbox{$\bigcirc$}}}

%全文字対応
\newcommand{\maru}[1]{\ooalign{
\hfil\resizebox{.8\width}{\height}{#1}\hfil
\crcr
\raise.1ex\hbox{\large$\bigcirc$}}}


\newcommand{\nord}[1]{\vcentcolon\mathrel{#1}\vcentcolon}
\providecommand{\vcentcolon}{\mathrel{\mathop{:}}}


\def\P{\mathop{\cal P}}
\def\diag{\mathop{\rm diag}}


\def\Re{\mathop{\rm Re}\nolimits}
\def\Im{\mathop{\rm Im}\nolimits}
\def\Det{\mathop{\rm Det}\nolimits}
\def\sign{\mathop{\rm sign}\nolimits}


%%% rap %%% - make two letters overlap
\newcommand{\rap}[2]
{\setbox1=\hbox{#1}%
\setbox2=\hbox to\wd1{\hss #2\hss}%
\mbox{\rlap{\box1}\box2}}

%\newcommand{\sla}[1]{\rap{$#1$}{/}}
\newcommand{\sla}[1]{\rap{$#1$}{$\backslash$}}


\def\DY#1{{\MyGreen [DY: #1]}}
\newcommand{\MyGreen}{\color [rgb]{0,0.7,0}}

\usepackage[vcentermath]{youngtab}
% \Yboxdim4pt
\newcommand{\Y}{\yng}
\newcommand{\Young}{}


%%%title def%%%%%%%%%%%%%%%%%%%%%%%%%%%%%%%%%%%%%%%%%%%%%%%

\makeatletter
\def\maketitle{
\noindent
{\Large \@title \par\vskip 2pt}
\noindent
{\large \@date \hspace{4pt} \@author}
%\cr[-2pt]
%\noindent------------------------------------------------------------------------------------------
\par\vskip 1.5em
}

\author{横山 大輔}
\date{\today}
%%%本文%%%%%%%%%%%%%%%%%%%%%%%%%%%%%%%%%%%%%%%%%%%%%%%%%%%%%%%%

% \title{\centerline{Lecture 2}}
\begin{document}
% \maketitle
% \begin{abstract}

% \end{abstract}
% \tableofcontents

%  \pagestyle{fancy}
%  \renewcommand{\headrulewidth}{0.0pt}
%  \rhead{}
%  \lhead{}
%  \cfoot{[\ \scshape\oldstylenums{\thepage}\ / %
%    \scshape\oldstylenums{\pageref{lastpage}} ]}
%%  \rfoot{\@author}

% \setcounter{section}{}
% \setcounter{subsection}{}
% \setcounter{subsubsection}{}

\centerline{\Large \bf  Lecture 6}

\vspace{12pt}
% \DY{なにかコメント}
Finally, we will discuss string amplitude.
The amplitude tell us about the feature of the string theory.
\vspace{-12pt}




\section{Global part of the gauge fixing}
% \DY{$6g-6+2n$を説明する。}


As we naive showed the string amplitude is written as follows.
\begin{align*}
 A_n = \sum_g \int \frac{[\DD h_{ab}]_{g,0}}{(\textrm{Weyl $\times$ diff})} \int \DD X^\mu e^{-S_\sigma [X^\mu,h_{ab}]} \prod_{i=1}^n \int d^2z \sqrt h V_i  \ .
\end{align*}
In this section we will learn the degrees of the metric integration
in the string amplitude after the (Weyl $\times$ diff) gauge fixing.
Namely,
\begin{align*}
 \frac{[\DD h_{ab}]_{g,n}}{(\textrm{Weyl $\times$ diff})} = 6g -6 +2n \ .
\end{align*}

\subsection{Naive explanation}

The $2n$ is simply coming from the integration of the vertex operators,
which is needed for the vertex operator to be Weyl \& diff-invariant.
The $6g$ can be understood in a naive way as follows.
Even after gauge fixing (say fixing metric locally)
there is a freedom to change the ``shape'' of world-sheet(WS),
which is called metric moduli (we simply denote moduli).
Let us consider to increase the number of genus $g$.
As we did in the lecture 4 we can do that by attaching a cylinder
to a certain Riemann surface $\Sigma_{g}$.
When this is done we need to specify where the end points of the cylinder are
and its ``shape'' of the cylinder.
The end points are denoted by two complex points $p_1,\ p_2$.
Around the points we introduce complex coordinate $z_1,\ z_2$
and impose one condition $z_1 z_2 = c$,
where $c$ is a constant whose phase specifies twist and magnitude specifies
length (see Fig.~\ref{g-increase.eps}).
In total there are another 6 parameters to describe the shape of WS.

\begin{figure}[htb]
\centerline{\includegraphics[width=300pt]{g-increase.eps}}
\caption{Attaching a pair of cylinders to $\Sigma_g$, and patching the pair of cylinders by $z_1z_2=c$.}
\label{g-increase.eps}
\end{figure}

Let us see some concrete examples.
As we saw in the Problem 5 sphere has 6 symmetries(Conformal Killing Group(CKG))
so that we can move three points on sphere to arbitrary locations.
Therefore, sphere amplitude has $2n-6$ integrations.
For torus there are 2 moduli and 2 CKG, thus torus amplitude has
$2n$ integrations.
From the discussion above for genus $g$ Riemann surface
its amplitude has $6g-6+2n$ integrations.





\subsection{Conformal Killing Group \& Metric moduli of space}

From the previous lecture we know the variation of the metric:
\begin{align*}
 \delta h_{ab} = \left(P \cdot \epsilon \right)_{ab} +2\wt \omega h_{ab} \ ,
\end{align*}
where
\begin{align*}
 &\left(P \cdot \epsilon \right)_{ab} = \nabla_a \epsilon_b +\nabla_b \epsilon_a
 -h_{ab} (\nabla \cdot \epsilon) \ , \cr
 &\wt \omega = \omega +\frac{1}{2} (\nabla \cdot \epsilon) \ .
\end{align*}
Conformal Killing Vectors are transformations that do not change metric, namely,
$\delta h_{ab} = 0$. $\omega$ is totally fix by $\nabla\cdot\epsilon$.
Therefore, they are the solutions of
\begin{align*}
 \left(P\cdot\epsilon\right)_{ab} = 0 \ .
\end{align*}
On the other hand, moduli is given by all the transformations
perpendicular to the transformation $\delta h_{ab}$, which we will denote
$\delta^\perp h_{ab}$.
It can be derived as follows.
\begin{align*}
 0 &= \int \sqrt h d^2\sigma\ \delta^\perp h_{ab} \left[
 \left(P\cdot \epsilon\right)^{ab} +2\wt\omega h^{ab} \right] \cr
 &= \int \sqrt h d^2\sigma\ \left[
 \left(P^T \cdot \delta^\perp h \right)_a \epsilon^a +2\wt\omega h^{ab}\delta^\perp h_{ab} \right] \ .
\end{align*}
To satisfy the equation for arbitrary $\epsilon$ and $\omega$
it is required that
\begin{align*}
 &h^{ab} \delta^\perp h_{ab} = 0 \ , \cr
 &\left(P^T \cdot \delta^\perp h \right)_a = \nabla^b \delta^\perp h_{ba} = 0 \ .
\end{align*}
CK-eq and moduli eq becomes simpler in conformal gauge in complex coordinates:
\begin{align*}
 &\partial \ol\epsilon = \ol\partial \epsilon = 0 \ , \cr
 &\partial \delta^\perp h_{\ol z\ol z} = \ol\partial \delta^\perp h_{zz} = 0 \ .
\end{align*}

\subsubsection*{Examples}
For sphere we have
\begin{align*}
 &\delta^\perp h_{zz} = \delta^\perp h_{\ol z\ol z} = 0 \ , \cr
 &\epsilon = a_0 +a_1 z + a_2 z^2 \ , \cr
 &\ol\epsilon = a_0^* +a_1^* z + a_2^* z^2 \ .
\end{align*}
Therefore, there are 6 CKVs and no modulus.

For torus we have
\begin{align*}
 &\delta^\perp h_{zz} = a \ , \cr
 &\epsilon = b \ .
\end{align*}
Therefore, there are 2 CKVs and 2 moduli.




\subsubsection*{Relation to ghost zero modes}


Action for ghost is
\begin{align*}
 S_\textrm{gh} = \frac{1}{2\pi} \int \sqrt{h} d^2z \ b_{ab} \nabla^a c^b \ .
\end{align*}
As we saw in the previous lecture
the nonzero ghost modes cancel non-physical modes in $X^\mu$.
On the other hand, its zero mode
\begin{align*}
 &P\cdot c = \nabla_a c_b +\nabla_b c_a
 -h_{ab} (\nabla \cdot c) = 0 \ , \cr
 &\nabla^a b_{ab} = 0 \ .
\end{align*}
corresponds to CKG and moduli.
Since these modes are absent in the action
we need to insert appropriate zero modes to derive a non-trivial amplitude
(because $\int d\theta \cdot 1 = 0$).
\begin{table}[htbp]
 \begin{center}
  \caption{The number of zero modes of $b$ and $c$.}
  \vspace{0pt}
  \label{table:001}
\begin{tabular}{c|c|c|c}
 & $g=0$ & $g=1$ & $g \ge 2$ \cr\hline
 $c$ & $6$ & $2$ & $0$ \cr
 $b$ & $0$ & $2$ & $6g-6$
\end{tabular}
\end{center}
\end{table}





\subsection{Ghost number anomaly}


Ghost fields have so called ghost number $[c]=1$ and $[b]=-1$,
and other fields have zero ghost number.
Correspondingly, there is a ghost number current $j = cb$
for the transformation $\delta c = \epsilon c$ and $\delta b = -\epsilon b$.
The current satisfies the conservation law $\ol\partial j = 0$.
However, it has anomaly in curved WS:
\begin{align*}
 \nabla^z j_z = \kappa \cdot R^{(2)} \ .
\end{align*}

%
%OPE with EM tensor leads
%\begin{align*}
% T(z)j(w) &= -\nord{(2b\partial c + \partial b c) (z)} \nord{cb(w)} \cr
% &\sim \frac{-3}{(z-w)^3} +\frac{ j(w)}{(z-w)^2} +\frac{\partial j(w)}{z-w} \ .
%\end{align*}
%From this the infinitesimal %\& finite
%transformations can be derived as follows.
%\begin{align*}
% &\delta j = \epsilon \partial j +\partial\epsilon j -\frac{3}{2}\partial^2 \epsilon \ . % \cr
%% &j(z) = \frac{\wt z}{\partial z} \wt j (\wt z) -\frac{3}{2} \partial_z \log \left(\frac{\partial\wt z}{\partial z}\right) \ .
%\end{align*}
%
%On the other hand, for the conformal gauge $ds^2 = e^{2\omega} dzd\ol z$,
%the WS curvature is $R = -8e^{-2\omega} \partial\ol\partial\omega$.
%Therefore, we can derive
%\begin{align*}
% j_z = -4\kappa \partial\omega -j(z) \ ,
%\end{align*}
%where $j(z)$ is the holomorphic current.
%As we did before (for EM tensor anomaly)
%we deduce that
%\begin{align*}
% &\delta j(z) = \epsilon \partial j(z) +\partial\epsilon j(z) -2\kappa\partial^2 \epsilon \ ,  \cr
% &\therefore \qquad \kappa = \frac{3}{4} \ .
%\end{align*}
%
%
%This anomaly puts constraint on non-vanishing ghost correlators:
%\begin{align*}
% &\wh\delta \left\langle \prod_{i=1}^m c(z_i) \prod_{j=1}^n b(z_j) \right\rangle
% = \left\langle \left(\int \frac{d^2x}{2\pi}\sqrt h \nabla^a j_a \right)\prod_{i=1}^m c(z_i) \prod_{j=1}^n b(z_j) \right\rangle  \cr
% &\quad= (m-n) \left\langle \prod_{i=1}^m c(z_i) \prod_{j=1}^n b(z_j) \right\rangle
% = \left\langle \left(\int \frac{3d^2x}{8\pi}\sqrt h R^{(2)} \right)\prod_{i=1}^m c(z_i) \prod_{j=1}^n b(z_j) \right\rangle  \cr
% &\quad= (3-3g) \left\langle \prod_{i=1}^m c(z_i) \prod_{j=1}^n b(z_j) \right\rangle \ .
%\end{align*}
%Namely, $\# c - \# b = 3-3g$, and similarly, $\#\ol c - \#\ol b = 3-3g$.
%


OPE with EM tensor leads
\begin{align*}
 T(z)j(w) &= -\nord{(2b\partial c + \partial b c) (z)} \nord{cb(w)} \nonumber\\
 &\sim \frac{-3}{(z-w)^3} +\frac{ j(w)}{(z-w)^2} +\frac{\partial j(w)}{z-w} \ .
\end{align*}
From this the infinitesimal %\& finite
transformations can be derived as follows.
\begin{align*}
 &\delta j = \epsilon \partial j +\partial\epsilon j -\frac{3}{2}\partial^2 \epsilon \ . % \\
% &j(z) = \frac{\wt z}{\partial z} \wt j (\wt z) -\frac{3}{2} \partial_z \log \left(\frac{\partial\wt z}{\partial z}\right) \ .
\end{align*}

On the other hand, for the conformal gauge $ds^2 = e^{2\omega} dzd\ol z$,
the WS curvature is $R = -8e^{-2\omega} \partial\ol\partial\omega$.
Therefore, we can derive
\begin{align*}
 j_z = -4\kappa \partial\omega +j(z) \ ,
\end{align*}
where $j(z)$ is the holomorphic current.
As we did before (for EM tensor anomaly)
we deduce that
\begin{align*}
 &\delta j(z) = \epsilon \partial j(z) +\partial\epsilon j(z) -2\kappa\partial^2 \epsilon \ ,  \\
 &\therefore \qquad \kappa = -\frac{3}{4} \ .
\end{align*}




This anomaly puts constraint on non-vanishing ghost correlators.
Let us first consider classical action of $\delta$:
\begin{align*}
 &\frac{1}{\epsilon} \delta \left\langle \prod_{i=1}^m c(z_i) \prod_{j=1}^n b(z_j) \right\rangle
 = (m-n) \left\langle \prod_{i=1}^m c(z_i) \prod_{j=1}^n b(z_j) \right\rangle \ .
\end{align*}
On the other hand,
using holomorphic current equation
\begin{align*}
 0 = \ol\partial J(z) = \nabla^z j_z -\kappa \cdot R^{(2)} \ ,
\end{align*}
the Ward-Takahashi indentity becomes
\begin{align*}
 &\frac{1}{\epsilon} \wh\delta \left\langle \prod_{i=1}^m c(z_i) \prod_{j=1}^n b(z_j) \right\rangle
 = \left\langle \left(\int \frac{d^2x}{2\pi}\sqrt h \ol\partial J(z) \right)\prod_{i=1}^m c(z_i) \prod_{j=1}^n b(z_j) \right\rangle  \nonumber\\
 &\quad
 = \left\langle \left(\int \frac{d^2x}{2\pi}\sqrt h \left(\nabla^z j_z +\frac{3}{4} R^{(2)}\right) \right)\prod_{i=1}^m c(z_i) \prod_{j=1}^n b(z_j) \right\rangle  \nonumber\\
 &\quad
 = \left\langle \left(\int \frac{3d^2x}{8\pi}\sqrt h R^{(2)} \right)\prod_{i=1}^m c(z_i) \prod_{j=1}^n b(z_j) \right\rangle
 = (3-3g) \left\langle \prod_{i=1}^m c(z_i) \prod_{j=1}^n b(z_j) \right\rangle \ .
\end{align*}
Therefore, we can conclude that $\# c - \# b = 3-3g$, and similarly, $\#\ol c - \#\ol b = 3-3g$.





\subsection{Ghost action and the zero modes}

Keeping the zero modes in mind we redo the BRST quantization.
\begin{align*}
 S_\textrm{gh} &= Q_B \int \frac{d^2 x}{4\pi} \sqrt{\wh h} \left( h^{ab} -\wh h^{ab}(t) \right)  \nonumber\\
 &\simeq \int \frac{d^2 x}{4\pi} \left\{
 B_{ab} \left( h^{ab} -\wh h^{ab}(t) \right)
 -b_{ab} \left( \left(P\cdot c \right)^{ab} -\frac{\partial}{\partial t_k} \wh h^{ab}(t) \xi_k \right)
 +4\pi \eta_{ai} c^a(\wh \sigma_i)+\cdots
 \right\} \ ,
\end{align*}
where $\cdots$ is an abbreviation of vanishing terms when $h_{ab}=\wh h_{ab}(t)$,
$i$ runs for $\# c +\# \ol c \equiv \mu$ and $k$ runs for $\# b +\# \ol b \equiv \nu$.
With this gauge fixing the string amplitude is written as follows.
\begin{align*}
 A_n &= \sum_g \int d^{\mu}t \int \DD b \DD c \DD X  e^{-S_\sigma [X,h] -S_\textrm{gh}[b,c] -\lambda\chi}  \nonumber\\
 &\quad \times \prod_{l=\mu/2+1}^{n}\int d^2z_l \prod_{i=1}^{\mu/2}c\ol c(\sigma_i)
 \prod_{k=1}^\nu (b, \partial_k h) \prod_{j=1}^n g_\textrm{st} \sqrt h V_j  \ .
\end{align*}
where $\chi$ is the Euler number, $\lambda$ is a vacuum expectation value of a dilaton, $g_\textrm{st} =e^\lambda$, and
\begin{align*}
 (b, \partial_k h) = \int \frac{d^2x}{4\pi} \sqrt{\wh h}\ b_{ab} \frac{\partial}{\partial t_k} \wh h^{ab}(t) \ .
\end{align*}
This is \textbf{the most general amplitude formula} we will consider in the remaining part.

\section{Tree amplitude}

% \DY{簡単な例を見せたい。Pole から intermediate states についての情報}


Let us consider $g=0$ case, namely, sphere case.
In this case we have $\mu =6$ and $\nu = 0$.
Then, the amplitude formula becomes
\begin{align*}
 A_{g=0,n} &= g_\textrm{st}^{-2} \int \DD b \DD c \DD X  e^{-S_\sigma [X,h] -S_\textrm{gh}[b,c]}
 \prod_{l=4}^{n}\int d^2z_l \prod_{i=1}^{3}c\ol c(\sigma_i)
 \prod_{j=1}^n g_\textrm{st} \sqrt h V_j  \cr
 &= g_\textrm{st}^{n-2} \prod_{l=4}^{n}\int d^2z_l  \left\langle \prod_{j=1}^n \sqrt h V_j \right\rangle_X
 \left\langle c\ol c(z_1) c\ol c(z_2) c\ol c(z_3) \right\rangle_{bc} \ ,
\end{align*}
where we divided the amplitude to the matter sector and the ghost sector.
Note that there is no easy definition for less than 3pt amplitude.

Though we can consider any vertex operators we will focus on the simplest example,
which is tachyon vertex operators:
\begin{align*}
 V_j = \nord{e^{ik_j\cdot X(z_j,\ol z_j)}} \ .
\end{align*}



\subsection{Matter sector}

In the following argument we will focus on the important part.
To be precise we need subtle argument for metric and zero modes.
However, all of the subtleties vanish in the end.
(For the interested students see \cite[\S6.2]{Polchinski})


Let us consider a generating functional
\begin{align*}
 Z[J] = \left\langle \exp\left( i\int \sqrt h d^2z J(z,\ol z) '\cdot X(z,\ol z) \right) \right\rangle \ .
\end{align*}
Expand $X^\mu$ in a complete set $X_I$:
\begin{align*}
 &X^\mu = \sum_I x_I^\mu X_I \ ,  \cr
 &\Delta X_I = -\omega_I^2 X_I \ ,  \cr
 &\int \sqrt h d^2z X_I X_{I'} = \delta_{II'} \ .
\end{align*}
Then,
\begin{align*}
 Z[J] = \prod_{I,\mu} \int dx_I^\mu \ \exp \left( -\frac{\omega_I^2 X_I\cdot X_I}{4\pi \alpha'} +ix_I \cdot J_I \right) \ ,
\end{align*}
where
\begin{align*}
 J^\mu_I = \int \sqrt h d^2 z J^\mu X_I \ .
\end{align*}
Note that for the zero mode $\omega_I = 0$ and hence the integral leads delta function.
So the result is
\begin{align*}
 Z[J] &= C (2\pi)^D \delta^D(J_0) \prod_{I\neq 0}\exp \left( -\frac{\pi\alpha' J_I\cdot J_I}{\omega_I} \right)  \cr
 &= C (2\pi)^D \delta^D(J_0) \prod_{I\neq 0}\exp \left( -\frac{1}{2} \int h d^2z d^2w J(z)\cdot J(w) G(z,w) \right) \ ,
\end{align*}
where $G(z,w) = -\frac{\alpha'}{2} \log|z-w|^2$.



When we consider tachyon vertex operators it corresponds to
\begin{align*}
 J(z) = \sum_{i=1}^n k_i \delta^2(z-z_i) \ .
\end{align*}
Hence, the amplitude is
\begin{align*}
 \left\langle \prod_{j=1}^n \sqrt h V_j \right\rangle_X = C_X (2\pi)^D \delta^D(\sum k_i) \prod_{i < j}^n |z_{ij}|^{\alpha'k_i\cdot k_j} \ .
\end{align*}
Note that we ignored the divergent part that is coming from $i=j$ because the vertex operators are normal ordered ones.




\subsection{Ghost sector}

For the ghost sector
\begin{align*}
 \left\langle c\ol c(z_1) c\ol c(z_2) c\ol c(z_3) \right\rangle_{bc} \ ,
\end{align*}
we only need to consider the zero modes, which is
\begin{align*}
 c(z) = c_{0} + c_{1} z +c_{2} z^2 \ .
\end{align*}
Therefore,
\begin{align*}
 \left\langle c\ol c(z_1) c\ol c(z_2) c\ol c(z_3) \right\rangle_{bc}
 = C_{bc} \int \prod_{i=0}^2 d\ol c_i dc_i \ c\ol c(z_1) c\ol c(z_2) c\ol c(z_3)
 = C_{bc} |z_{12}|^2 |z_{23}|^2 |z_{31}|^2 \ .
\end{align*}



\subsection{Shapiro-Virasoro amplitude}

Let us see 4pt amplitude of tachyons.
\begin{align*}
 A_{0,4} = g_\textrm{st}^{2} C_\textrm{4pt} (2\pi)^D \delta^D(\sum k_i) \int d^2z_4
 \prod_{i < j}^4 |z_{ij}|^{\alpha'k_i\cdot k_j} \prod_{i < j}^3 |z_{ij}|^2 \ .
\end{align*}
Since we have 6 CKG we can set $(z_1,z_2,z_3)$ to $(0,1,\infty)$,
then the expression reduces to
\begin{align*}
 A_{0,4} = g_\textrm{st}^{2} C_\textrm{4pt} (2\pi)^D \delta^D(\sum k_i)
 B\left( -\frac{\alpha' s}{4}-1,  -\frac{\alpha' t}{4}-1,  -\frac{\alpha' u}{4}-1 \right) \ ,
\end{align*}
where
\begin{align*}
 &B(a,b,c) = \int d^2z |z|^{2a-2} |1-z|^{2b-2} = \pi \frac{\Gamma(a)\Gamma(b)\Gamma(c)}{\Gamma(a+b)\Gamma(b+c)\Gamma(c+a)} \ , \cr
 &s = -k_{1+2}^2 = -k_{3+4}^2 = -2 k_1\cdot k_2 -\frac{8}{\alpha'} \ , \cr
 &t = -k_{1+3}^2 = -k_{2+4}^2 = -2 k_1\cdot k_3 -\frac{8}{\alpha'} \ , \cr
 &s = -k_{1+4}^2 = -k_{2+3}^2 = -2 k_1\cdot k_4 -\frac{8}{\alpha'} \ , \qquad (s+t+u = -\frac{16}{\alpha'}) \ . \nonumber
\end{align*}
This is called Shapiro-Virasoro amplitude.
For the derivation of the $B$ function see \cite[\S3.2]{GSW} Vol.1 pp.386 and pp.373.


Let us discuss the poles in the amplitude so that we can see what kind of states propagate.
The s-channel poles (see Fig.~\ref{s-channel.eps}) are
\begin{align*}
 -\frac{\alpha's}{4}-1 \in \ZZ_{\le 0} , \qquad s = -k_{1+2}^2 = m^2 = \frac{4}{\alpha'} (n-1) \quad (n \in \ZZ_{\ge 0}) \ ,
\end{align*}
where $m$ is a mass of intermediate states.
As it shows the intermediate states could be tachyon, graviton etc.
\begin{figure}[htb]
\centerline{\includegraphics[width=150pt]{s-channel.eps}}
\caption{s-channel.}
\label{s-channel.eps}
\end{figure}

We can fix the overall constant from unitarity (see Fig.~\ref{unitarity.eps}).
\begin{figure}[htb]
\centerline{\includegraphics[width=250pt]{unitarity.eps}}
\caption{Unitarity requirement.}
\label{unitarity.eps}
\end{figure}
The 4pt amplitude can be divided into two 3pt amplitude as in Fig.
At s-channel tachyon pole it becomes
\begin{align*}
 A_{0,3} &= a_3 (2\pi)^D \delta^D (\sum k_i) \ , \qquad a_3 \simeq C g_\textrm{st} \ , \cr
 A_{0,4} &= a_4 (2\pi)^D \delta^D (\sum k_i) \ , \qquad a_4 \simeq -\frac{4\pi}{\alpha'} \frac{1}{s+\frac{4}{\alpha'}} C g_\textrm{st}^2
 \ .
\end{align*}
Unitarity requires
\begin{align*}
 &a_4 = \frac{(a_3)^2}{s+\frac{4}{\alpha'}} \ , \qquad
 \therefore \quad C = -\frac{4\pi}{\alpha'} \ .
\end{align*}


\section{1-loop amplitude}

% \DY{UV finite であることを説明したい。}

Here, we will briefly see $g=1$ case, which is nothing but torus.
Torus can be constructed from 1d flat complex space $\CC$ by identification (see Fig.~\ref{torus.eps})
\begin{figure}[htb]
\centerline{\includegraphics[width=150pt]{torus.eps}}
\caption{Torus from $\CC$ by the identification.}
\label{torus.eps}
\end{figure}
\begin{align*}
 z \sim z+2\pi \sim z+2\pi \tau \ ,
\end{align*}
where $\tau = \tau_1 i\tau_2$ and assume $\tau_2 > 0$.
Since the metric is flat the $\tau$ is a complex modulus, which describe the ``shape'' of the torus.
There is also a complex translation invariance, which is 2 CKG.

Torus has so called a modular invariance.
Let us define two generators:
\begin{align*}
 T: \tau \to \tau+1 \ , \quad S: \tau \to -\frac{1}{\tau} \ .
\end{align*}
These $T,\ S$ generates modular transformation
\begin{align*}
 \tau \to \frac{a\tau+b}{c\tau+d} \ , \quad
 \begin{pmatrix}
  a & b \cr c & d
 \end{pmatrix} \in PSL(2,\ZZ) \ ,
\end{align*}
The ``shape'' of torus is conformally invariant under the modular transformation.
Therefore, the parameter space of $\tau$ is limited to the fundamental region, see Fig.~\ref{fundamental.eps}.
\begin{figure}[htb]
\centerline{\includegraphics[width=120pt]{fundamental.eps}}
\caption{Fundamental region of $\tau$.}
\label{fundamental.eps}
\end{figure}

As we did for sphere case we will encounter ghost expectation,
whose contribution is only from the zero modes.
In the torus case since the space is flat and finite
we can insert a WS integral as follows
\begin{align*}
 \langle c(0) \rangle_{bc} = \langle C \rangle_{bc} \ , \quad C \equiv \int \frac{d^2z}{4\pi^2\tau_2}c(z) \ .
\end{align*}
($\ol C$ and $B, \ol B$ are similarly defined.)


The string amplitude for torus case is
\begin{align*}
 A_{1,n}
 &= g_\textrm{st}^{n} \frac{1}{2} \int_F d^2\tau   \left\langle (b, \partial_\tau h) (b, \partial_{\ol\tau} h)
 c\ol c\ \sqrt h V_1 \prod_{j=2}^{n}\int d^2z_j \sqrt h V_j \right\rangle \ .
\end{align*}
As the same logic as before we can insert the integration even for $z_1$ since the amplitude should not depend on $z_1$:
\begin{align*}
 A_{1,n}
 &= g_\textrm{st}^{n} \frac{1}{2} \frac{1}{4\pi^2\tau_2} \int_F d^2\tau
 \left\langle (b, \partial_\tau h) (b, \partial_{\ol\tau} h) C\ol C\ \right\rangle_{bc}
 \left\langle \prod_{j=1}^{n}\int d^2z_j \sqrt h V_j \right\rangle_X \ .
\end{align*}


Though seemingly the metric would not depend on $\tau$, it does indeed depend on $\tau$ as follows.
Let us consider small deformation of the metric
\begin{align*}
 &ds^2 = dzd\ol z \to d(z+\epsilon\ol z) d(\ol z+\ol\epsilon z) \ , \cr
 &\Leftrightarrow \quad \delta h_{zz} = \ol\epsilon \ , \quad \delta h_{\ol z\ol z} = \epsilon \ .
\end{align*}
This deformation changes the period to $(2\pi (1+\epsilon),2\pi(\tau+\epsilon\ol\tau))$.
Therefore the modulus becomes $\wt \tau = \frac{\tau+\epsilon\ol\tau}{1+\epsilon} \simeq \tau -2i\epsilon\tau_2$
$\Leftrightarrow\ \delta \tau = -2i\epsilon\tau_2$.
Now the $\tau$ derivative is
\begin{align*}
 \partial_\tau h_{\ol z\ol z} = \lim_{\epsilon \to 0} \frac{\delta h_{\ol z\ol z}}{\delta \tau} = \frac{i}{2\tau_2} \ ,
 % \quad \partial_{\ol \tau} h_{zz} = -\frac{i}{2\tau_2} \ ,
 \quad\Rightarrow\quad (b, \partial_\tau h) = 2\pi i B \quad \textrm{etc}.
\end{align*}


One may think the simplest amplitude would be tachyon amplitude,
however, it is actually vacuum amplitude $n=0$, which can be calculated in torus case opposed to sphere case.
\begin{align*}
 A_{1,n}
 &= \frac{1}{2} \int_F \frac{d^2\tau}{\tau_2}
 \left\langle B\ol B C\ol C\ \right\rangle_{bc}
 \left\langle 1 \right\rangle_X \ .
\end{align*}


\subsubsection*{Matter sector}

Using operator formalism
\begin{align*}
 \left\langle 1 \right\rangle_X = \tr \exp \left[ 2\pi i \left( \tau_1 P +i\tau_2 H \right)\right]
 = \tr \left[ q^{L_0 -\frac{c}{24}} \ol q^{\ol L_0 -\frac{c}{24}} \right] \ ,
\end{align*}
where $q = e^{2\pi i\tau}$, $P = L_0 -\ol L_0$, $H = L_0 +\ol L_0 -\frac{c}{12}$, and
\begin{align*}
 L_0 = \frac{\alpha'}{2} p^2 +\sum_{n > 0} \alpha_{-n} \cdot \alpha_n \ .
\end{align*}
\begin{align*}
 \left\langle 1 \right\rangle_X &=
 \int \frac{d^Dxd^Dp}{(2\pi)^D} \exp \left( -4\pi\tau_2 \frac{\alpha'}{4} p^2 \right)
 \left| \frac{q^{-\frac{1}{24}}}{\prod_{n \ge 1} (1-q^n)} \right|^{2D}  \cr
 &= i \frac{V}{(2\pi l_s)^{26}} \left(\tau_2\right)^{-13} \left|\eta(\tau)\right|^{-52} \ ,
\end{align*}
where $l_s = \sqrt{\alpha'}$.


\subsubsection*{Ghost sector}


Similarly,
\begin{align*}
 \left\langle B\ol B C\ol C\ \right\rangle_{bc}
 = \tr \left[ (-1)^F b_0 \ol b_0 c_0 \ol c_0 q^{L_0 -\frac{c}{24}} \ol q^{\ol L_0 -\frac{c}{24}}  \right]
\end{align*}
with
$c=-26$ and $L_0 = \sum n \nord{b_{-n}c_n} -1$. Hence
\begin{align*}
 \left\langle B\ol B C\ol C\ \right\rangle_{bc}
 = q^{-1-\frac{c}{24}} \ol q^{-1-\frac{c}{24}} \prod_{n \ge 1} (1-q^n) (1-\ol q^n) = |\eta(\tau)|^4 \ .
\end{align*}



So, the final result is
\begin{align*}
 A_{1,n}
 &= \frac{iV}{(2\pi l_s)^{26}} \int_F \frac{d^2\tau}{2\tau_2}  \left(\tau_2\right)^{-13} \left|\eta(\tau)\right|^{-48} \ .
\end{align*}

\textbf{Comments:}
\begin{itemize}
 \item There is no UV divergence (see Fig. and recall the naive argument).
 \item $\left|\eta(\tau)\right|^{-48} \simeq \left|q^{-1} +24 +\mathcal O(q)\right|^2$,
       which shows correct spectrum of a closed string.
\end{itemize}
\begin{figure}[htb]
\centerline{\includegraphics[width=350pt]{noDivergence.eps}}
\caption{Naive explanation of no UV divergence in string theory.}
\label{noDivergence.eps}
\end{figure}

\bibliography{string-lecture}
\bibliographystyle{halpha}
% \bibliographystyle{JHEP}

% \begin{thebibliography}{CDLOGP91}

% %\cite{AlvarezGaume:1981hn}
% \bibitem[AFM81]{AlvarezGaume:1981hn}
%   L.~Alvarez-Gaume, D.~Z.~Freedman and S.~Mukhi,
%   ``The Background Field Method and the Ultraviolet Structure of the Supersymmetric Nonlinear Sigma Model,''
%   Annals Phys.\  {\bf 134}, 85 (1981).
%   %doi:10.1016/0003-4916(81)90006-3
%   %%CITATION = doi:10.1016/0003-4916(81)90006-3;%%
%   %525 citations counted in INSPIRE as of 09 Oct 2017


 
% \bibitem[CT88]{Callan:1988xx}
%  Curt Callan and Lárus Thorlacius.
%  \textit{SIGMA MODELS AND STRING THEORY.}
%  TASI Lecture, 1988.
%  (The link below is a direct link to the pdf file of 45MB )
%  \href{http://www.damtp.cam.ac.uk/user/tong/string/sigma.pdf}
%  {http://www.damtp.cam.ac.uk/user/tong/string/sigma.pdf}.




% \end{thebibliography}


\label{lastpage}

% \begin{tikzpicture}[>=stealth,scale=1]
%  \draw[->] (0,0)--(1,0);
%  \draw[latex-stealth] (0,0.5)--(1,0.5);
%  \draw[latex-stealthnew,arrowhead=2mm] (0,1)--(1,1);
% \end{tikzpicture}

% \begin{figure}[htb]
% \centerline{\includegraphics[width=250pt]{.eps}}
% \caption{}
% \label{.eps}
% \end{figure}



% \begin{table}[htbp]
%  \begin{center}
%   \caption{}
%   \vspace{4pt}
%   \label{table:001}
% \begin{tabular}{|c|c|c|c|c|}
% \hline
% \hline
%   Category & Sector & $(h_A,h_B,h_T)$ & Mirror theory & ABJM model \cr
% \hline
%  1 & & \parbox{40pt}{$(0,0,0)$ $(1,1,1)$} & $1.11906$ & $1.13290$ \cr
% \hline
%  2 & & \parbox{40pt}{$(0,0,1)$ $(1,1,0)$} & $-0.10861$ & $-0.10861$ \cr
% \hline
%  \multirow{2}{*}{\vspace{-15pt}3} & 3-1 & \parbox{40pt}{$(0,1,0)$ $(1,0,1)$} & $0.176777$ &
%  \multirow{2}{*}{\vspace{-15pt}$0.176577$} \cr
% \cline{2-4}
%  & 3-2 & \parbox{40pt}{$(1,0,0)$ $(0,1,1)$} & $0.176777$ & \cr
% \hline
% \hline
% \end{tabular}
% \end{center}
% \end{table}


% \begin{thebibliography}{99}

% % \cite{Imamura:2012rq}
% \bibitem{Imamura:2012rq}
%   Y.~Imamura and D.~Yokoyama,
%   %``S^3/Z_n partition function and dualities,''
%   JHEP {\bf 1211}, 122 (2012)
%   [arXiv:1208.1404 [hep-th]].
%   %%CITATION = ARXIV:1208.1404;%%

 % \bibitem{fnorio:legendre}
 %         fnorio
 %         ``ルジャンドル変換とは何か''
 %         \url{http://fnorio.com/0146Legendre_transformation/Legendre_transformation.html}


 % \bibitem{EMAN:dynamics}
 %         EMAN物理学
 %         ``ハミルトニアン''
 %         \url{http://eman-physics.net/analytic/hamilton.html}


 % \bibitem{Wiki:legendre}
 %         Wiki
 %         ``ルジャンドル変換''
 %         \url{https://ja.wikipedia.org/wiki/ルジャンドル変換}

 % \bibitem{mathtrain:legendre}
 %         高校数学の美しい物語
 %         ``ルジャンドル変換の意味と具体例''
 %         \url{http://mathtrain.jp/legendrehenkan}

% \end{thebibliography}


% \bibliography{dd}

\end{document}
