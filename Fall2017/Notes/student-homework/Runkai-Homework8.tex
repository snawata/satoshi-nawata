 \documentclass[12pt,a4paper]{article}
%\usepackage{hyperref} % Use the Charter font for the document text
%\usepackage[UTF8]{ctex}
\usepackage{jheppub}

\usepackage{amsfonts,amssymb,amsmath}
\usepackage{mathtools}
\usepackage{tikz-cd}
\usepackage{tikz}
\usepackage{alltt}
\usepackage{amsfonts}
\usepackage{amsmath}
\usepackage{amssymb}
\usepackage{amsthm}
\usepackage{booktabs}
\usepackage{caption}
\usepackage{enumitem}
\usepackage{fancyhdr}
\usepackage{graphicx}
\usepackage{mathdots}
\usepackage{mathtools}
\usepackage{microtype}
\usepackage{multirow}
\usepackage{pdflscape}
\usepackage{pgfplots}
\usepackage{siunitx}
\usepackage{slashed}
\usepackage{tabularx}
\usepackage{tikz}
\usepackage{tkz-euclide}
\usepackage[normalem]{ulem}
\usepackage[all]{xy}
\usepackage{imakeidx}
\usepackage{gensymb}
\usepackage{simplewick}
\usepackage{feynmp-auto}
\usepackage{wrapfig}



%%%%%%%  Greek letters %%%%%%%%%%%%%%%%%%
\def\a{\alpha}
\def\b{\beta}
\def\c{\gamma} \def\g{\gamma}
\def\d{\delta}
\def\e{\epsilon}
\def\f{\phi}
\def\vf{\varphi}  \def\tvf{\tilde{\varphi}}
\def\vp{\varphi}
\def\h{\eta}
\def\i{\iota}
\def\j{\psi}
\def\k{\kappa}
\def\m{\mu}
\def\n{\nu}
\def\o{\omega}  \def\w{\omega}
\def\q{\theta}  \def\th{\theta}
\def\r{\rho}
\def\s{\sigma}
\def\t{\tau}
\def\u{\upsilon}
\def\x{\xi}
\def\z{\zeta}

\def\A{\Alpha}
\def\B{\Beta}
\def\G{\Gamma}
\def\D{\Delta}
\def\E{\Epsilon}
\def\F{Phi}
\def\h{\eta}
\def\I{\Iota}
\def\J{Psi}
\def\K{\Kappa}
\def\L{\lambdabda}
\def\M{\Mu}
\def\N{\Nu}
\def\O{\Omega}  \def\w{\omega}
\def\Q{\Theta}  \def\Th{\Theta}
\def\R{\Rho}
\def\Si{\Sigma}
\def\T{\Tau}
\def\Up{\Upsilon}
\def\X{\Xi}
\def\Z{\Zeta}








%%%%%%%%%%%% math fonts %%%%%%%%%%%%%%%%%%%%%%%%%%%%%%%%%%%%%
%
%---------- mathbb font --------------------------------
%

\newcommand{\bA}{\ensuremath{\mathbb{A}}}
\newcommand{\bB}{\ensuremath{\mathbb{B}}}
\newcommand{\bC}{\ensuremath{\mathbb{C}}}
\newcommand{\bD}{\ensuremath{\mathbb{D}}}
\newcommand{\bE}{\ensuremath{\mathbb{E}}}
\newcommand{\bF}{\ensuremath{\mathbb{F}}}
\newcommand{\bG}{\ensuremath{\mathbb{G}}}
\newcommand{\bH}{\ensuremath{\mathbb{H}}}
\newcommand{\bI}{\ensuremath{\mathbb{I}}}
\newcommand{\bJ}{\ensuremath{\mathbb{J}}}
\newcommand{\bK}{\ensuremath{\mathbb{K}}}
\newcommand{\bL}{\ensuremath{\mathbb{L}}}
\newcommand{\bM}{\ensuremath{\mathbb{M}}}
\newcommand{\bN}{\ensuremath{\mathbb{N}}}
\newcommand{\bO}{\ensuremath{\mathbb{O}}}
\newcommand{\bP}{\ensuremath{\mathbb{P}}}
\newcommand{\bQ}{\ensuremath{\mathbb{Q}}}
\newcommand{\bR}{\ensuremath{\mathbb{R}}}
\newcommand{\bS}{\ensuremath{\mathbb{S}}}
\newcommand{\bT}{\ensuremath{\mathbb{T}}}
\newcommand{\bU}{\ensuremath{\mathbb{U}}}
\newcommand{\bV}{\ensuremath{\mathbb{V}}}
\newcommand{\bW}{\ensuremath{\mathbb{W}}}
\newcommand{\bX}{\ensuremath{\mathbb{X}}}
\newcommand{\bY}{\ensuremath{\mathbb{Y}}}
\newcommand{\bZ}{\ensuremath{\mathbb{Z}}}


%
%
%---------- mathbf font --------------------------------
%


%
%---------- mathbf font --------------------------------
%


\newcommand{\bfA}{\ensuremath{\mathbf{A}}}
\newcommand{\bfB}{\ensuremath{\mathbf{B}}}
\newcommand{\bfC}{\ensuremath{\mathbf{C}}}
\newcommand{\bfD}{\ensuremath{\mathbf{D}}}
\newcommand{\bfE}{\ensuremath{\mathbf{E}}}
\newcommand{\bfF}{\ensuremath{\mathbf{F}}}
\newcommand{\bfG}{\ensuremath{\mathbf{G}}}
\newcommand{\bfH}{\ensuremath{\mathbf{H}}}
\newcommand{\bfI}{\ensuremath{\mathbf{I}}}
\newcommand{\bfJ}{\ensuremath{\mathbf{J}}}
\newcommand{\bfK}{\ensuremath{\mathbf{K}}}
\newcommand{\bfL}{\ensuremath{\mathbf{L}}}
\newcommand{\bfM}{\ensuremath{\mathbf{M}}}
\newcommand{\bfN}{\ensuremath{\mathbf{N}}}
\newcommand{\bfO}{\ensuremath{\mathbf{O}}}
\newcommand{\bfP}{\ensuremath{\mathbf{P}}}
\newcommand{\bfQ}{\ensuremath{\mathbf{Q}}}
\newcommand{\bfR}{\ensuremath{\mathbf{R}}}
\newcommand{\bfS}{\ensuremath{\mathbf{S}}}
\newcommand{\bfT}{\ensuremath{\mathbf{T}}}
\newcommand{\bfU}{\ensuremath{\mathbf{U}}}
\newcommand{\bfV}{\ensuremath{\mathbf{V}}}
\newcommand{\bfW}{\ensuremath{\mathbf{W}}}
\newcommand{\bfX}{\ensuremath{\mathbf{X}}}
\newcommand{\bfY}{\ensuremath{\mathbf{Y}}}
\newcommand{\bfZ}{\ensuremath{\mathbf{Z}}}
\newcommand{\bfa}{\ensuremath{\mathbf{a}}}
\newcommand{\bfb}{\ensuremath{\mathbf{b}}}
\newcommand{\bfc}{\ensuremath{\mathbf{c}}}
\newcommand{\bfd}{\ensuremath{\mathbf{d}}}
\newcommand{\bfe}{\ensuremath{\mathbf{e}}}
\newcommand{\bff}{\ensuremath{\mathbf{f}}}
\newcommand{\bfg}{\ensuremath{\mathbf{g}}}
\newcommand{\bfh}{\ensuremath{\mathbf{h}}}
\newcommand{\bfi}{\ensuremath{\mathbf{i}}}
\newcommand{\bfj}{\ensuremath{\mathbf{j}}}
\newcommand{\bfk}{\ensuremath{\mathbf{k}}}
\newcommand{\bfl}{\ensuremath{\mathbf{l}}}
\newcommand{\bfm}{\ensuremath{\mathbf{m}}}
\newcommand{\bfn}{\ensuremath{\mathbf{n}}}
\newcommand{\bfo}{\ensuremath{\mathbf{o}}}
\newcommand{\bfp}{\ensuremath{\mathbf{p}}}
\newcommand{\bfq}{\ensuremath{\mathbf{q}}}
\newcommand{\bfr}{\ensuremath{\mathbf{r}}}
\newcommand{\bfs}{\ensuremath{\mathbf{s}}}
\newcommand{\bft}{\ensuremath{\mathbf{t}}}
\newcommand{\bfu}{\ensuremath{\mathbf{u}}}
\newcommand{\bfv}{\ensuremath{\mathbf{v}}}
\newcommand{\bfw}{\ensuremath{\mathbf{w}}}
\newcommand{\bfx}{\ensuremath{\mathbf{x}}}
\newcommand{\bfy}{\ensuremath{\mathbf{y}}}
\newcommand{\bfz}{\ensuremath{\mathbf{z}}}




%---------- mathscript font -----------------------------
%

\newcommand{\scA}{\ensuremath{\mathscr{A}}}
\newcommand{\scB}{\ensuremath{\mathscr{B}}}
\newcommand{\scC}{\ensuremath{\mathscr{C}}}
\newcommand{\scD}{\ensuremath{\mathscr{D}}}
\newcommand{\scE}{\ensuremath{\mathscr{E}}}
\newcommand{\scF}{\ensuremath{\mathscr{F}}}
\newcommand{\scG}{\ensuremath{\mathscr{G}}}
\newcommand{\scH}{\ensuremath{\mathscr{H}}}
\newcommand{\scI}{\ensuremath{\mathscr{I}}}
\newcommand{\scJ}{\ensuremath{\mathscr{J}}}
\newcommand{\scK}{\ensuremath{\mathscr{K}}}
\newcommand{\scL}{\ensuremath{\mathscr{L}}}
\newcommand{\scM}{\ensuremath{\mathscr{M}}}
\newcommand{\scN}{\ensuremath{\mathscr{N}}}
\newcommand{\scO}{\ensuremath{\mathscr{O}}}
\newcommand{\scP}{\ensuremath{\mathscr{P}}}
\newcommand{\scQ}{\ensuremath{\mathscr{Q}}}
\newcommand{\scR}{\ensuremath{\mathscr{R}}}
\newcommand{\scS}{\ensuremath{\mathscr{S}}}
\newcommand{\scT}{\ensuremath{\mathscr{T}}}
\newcommand{\scU}{\ensuremath{\mathscr{U}}}
\newcommand{\scV}{\ensuremath{\mathscr{V}}}
\newcommand{\scW}{\ensuremath{\mathscr{W}}}
\newcommand{\scX}{\ensuremath{\mathscr{X}}}
\newcommand{\scY}{\ensuremath{\mathscr{Y}}}
\newcommand{\scZ}{\ensuremath{\mathscr{Z}}}
\newcommand{\scAH}{\ensuremath{\mathscr{A}\!\!\scH}}

%
%---------- mathfrak font -----------------------------
%

\newcommand{\frakA}{\ensuremath{\mathfrak{A}}}
\newcommand{\frakB}{\ensuremath{\mathfrak{B}}}
\newcommand{\frakC}{\ensuremath{\mathfrak{C}}}
\newcommand{\frakD}{\ensuremath{\mathfrak{D}}}
\newcommand{\frakE}{\ensuremath{\mathfrak{E}}}
\newcommand{\frakF}{\ensuremath{\mathfrak{F}}}
\newcommand{\frakG}{\ensuremath{\mathfrak{G}}}
\newcommand{\frakH}{\ensuremath{\mathfrak{H}}}
\newcommand{\frakI}{\ensuremath{\mathfrak{I}}}
\newcommand{\frakJ}{\ensuremath{\mathfrak{J}}}
\newcommand{\frakK}{\ensuremath{\mathfrak{K}}}
\newcommand{\frakL}{\ensuremath{\mathfrak{L}}}
\newcommand{\frakM}{\ensuremath{\mathfrak{M}}}
\newcommand{\frakN}{\ensuremath{\mathfrak{N}}}
\newcommand{\frakO}{\ensuremath{\mathfrak{O}}}
\newcommand{\frakP}{\ensuremath{\mathfrak{P}}}
\newcommand{\frakQ}{\ensuremath{\mathfrak{Q}}}
\newcommand{\frakR}{\ensuremath{\mathfrak{R}}}
\newcommand{\frakS}{\ensuremath{\mathfrak{S}}}
\newcommand{\frakT}{\ensuremath{\mathfrak{T}}}
\newcommand{\frakU}{\ensuremath{\mathfrak{U}}}
\newcommand{\frakV}{\ensuremath{\mathfrak{V}}}
\newcommand{\frakW}{\ensuremath{\mathfrak{W}}}
\newcommand{\frakX}{\ensuremath{\mathfrak{X}}}
\newcommand{\frakY}{\ensuremath{\mathfrak{Y}}}
\newcommand{\frakZ}{\ensuremath{\mathfrak{Z}}}
\newcommand{\fraka}{\ensuremath{\mathfrak{a}}}
\newcommand{\frakb}{\ensuremath{\mathfrak{b}}}
\newcommand{\frakc}{\ensuremath{\mathfrak{c}}}
\newcommand{\frakd}{\ensuremath{\mathfrak{d}}}
\newcommand{\frake}{\ensuremath{\mathfrak{e}}}
\newcommand{\frakf}{\ensuremath{\mathfrak{f}}}
\newcommand{\frakg}{\ensuremath{\mathfrak{g}}}
\newcommand{\frakh}{\ensuremath{\mathfrak{h}}}
\newcommand{\fraki}{\ensuremath{\mathfrak{i}}}
\newcommand{\frakj}{\ensuremath{\mathfrak{j}}}
\newcommand{\frakk}{\ensuremath{\mathfrak{k}}}
\newcommand{\frakl}{\ensuremath{\mathfrak{l}}}
\newcommand{\frakm}{\ensuremath{\mathfrak{m}}}
\newcommand{\frakn}{\ensuremath{\mathfrak{n}}}
\newcommand{\frako}{\ensuremath{\mathfrak{o}}}
\newcommand{\frakp}{\ensuremath{\mathfrak{p}}}
\newcommand{\frakq}{\ensuremath{\mathfrak{q}}}
\newcommand{\frakr}{\ensuremath{\mathfrak{r}}}
\newcommand{\fraks}{\ensuremath{\mathfrak{s}}}
\newcommand{\frakt}{\ensuremath{\mathfrak{t}}}
\newcommand{\fraku}{\ensuremath{\mathfrak{u}}}
\newcommand{\frakv}{\ensuremath{\mathfrak{v}}}
\newcommand{\frakw}{\ensuremath{\mathfrak{w}}}
\newcommand{\frakx}{\ensuremath{\mathfrak{x}}}
\newcommand{\fraky}{\ensuremath{\mathfrak{y}}}
\newcommand{\frakz}{\ensuremath{\mathfrak{z}}}
\newcommand{\fraksl}{\ensuremath{\mathfrak{sl}}}
\newcommand{\frakso}{\ensuremath{\mathfrak{so}}}
\newcommand{\fraksp}{\ensuremath{\mathfrak{sp}}}

%%%%%%%%%%%%  Calligraphic, Roman and Maths integers %%%%%%%%%%%%%%%%%%

\newcommand{\cA}{\mathcal{A}}
\newcommand{\cB}{\mathcal{B}}
\newcommand{\cC}{\mathcal{C}}
\newcommand{\cD}{\mathcal{D}}
\newcommand{\cE}{\mathcal{E}}
\newcommand{\cF}{\mathcal{F}}
\newcommand{\cG}{\mathcal{G}}
\newcommand{\cH}{\mathcal{H}}
\newcommand{\cI}{\mathcal{I}}
\newcommand{\cJ}{\mathcal{J}}
\newcommand{\cK}{\mathcal{K}}
\newcommand{\cL}{\mathcal{L}}
\newcommand{\cM}{\mathcal{M}}
\newcommand{\cN}{\mathcal{N}}
\newcommand{\cO}{\mathcal{O}}
\newcommand{\cQ}{\mathcal{Q}}
\newcommand{\cS}{\mathcal{S}}
\newcommand{\cX}{\mathcal{X}}
\newcommand{\cY}{\mathcal{Y}}
\newcommand{\cW}{\mathcal{W}}
\newcommand{\cR}{\mathcal{R}}
\newcommand{\cT}{\mathcal{T}}
\newcommand{\cZ}{\mathcal{Z}}

%%%%%%%%%%%%%%%%%%%%%%%%%%%%%%%%%%%%%%%%%%%%%%%%%%%%%%%%%%%%%%%%
\newcommand{\SU}{\mathrm{SU}}
\newcommand{\SO}{\mathrm{SO}}
\newcommand{\SL}{\mathrm{SL}}
\newcommand{\Sp}{\mathrm{Sp}}
\newcommand{\su}{\mathrm{su}}
\newcommand{\so}{\mathrm{so}}
\newcommand{\spl}{\mathrm{sp}}
\newcommand{\gl}{\mathrm{gl}}
\newcommand{\sll}{\mathrm{sl}}
\newcommand{\U}{\mathrm{U}}
\newcommand{\ul}{\mathrm{u}}
\newcommand{\Spin}{\mathrm{Spin}}
\newcommand{\Pin}{\mathrm{Pin}}
%%%%%%%%%%%%%%%%%%%%%%%%%%%%%%%%%%%%%%%%%%%%%%%%%%%%%%%%%%%%%%%%
\renewcommand{\Im}{{\rm Im}}
\renewcommand{\Re}{{\rm Re}}
\newcommand{\Tr}{\mbox{Tr}}
\newcommand{\Pf}{\mbox{Pf}}
\newcommand{\sgn}{\mbox{sgn}}
\newcommand{\Vir}{{\rm Vir}}
\newcommand{\Li}{{\rm Li}}

\def\tl{\tilde}
\def\wt{\widetilde}
\def\wh{\widehat}
\def\bar{\overline}
\newcommand\bz{{\bar{z}}}


\newcommand {\comm} [2] {\mbox {$\left[ #1, #2 \right]$}}
\newcommand {\acomm} [2] {\mbox {$\left\{ #1, #2 \right\}$}}

\newtheorem{lemma}{Lemma}[section]
\newtheorem{conjecture}[lemma]{Conjecture} 
\newtheorem{corollary}[lemma]{Corollary} 
\newtheorem{theorem}[lemma]{Theorem} 
\newtheorem{definition}[lemma]{Definition} 
\newtheorem{question}[lemma]{Question} 
\newtheorem{proposition}[lemma]{Proposition} 

\newcommand {\nod} [1] {\mbox {$:#1\!:$}}
\newcommand	{\abs}	[1] {{\left| #1 \right|}}
\newcommand {\brac} [1]	{{\left\{	#1 \right\}}}

\def\ap{{\alpha^\prime}}
\def\zb{\bar{z}}


\newcommand{\Gam}[1]{{\Gamma^{#1}}}

\title{Homework Eight}

\author{R.K.Tao}

\begin{document}
	\date{}
	
%	\maketitle
	
	\section{Bosonic string on a torus}
	
	\subsection{}
	
	The torus partition function on a circle $S^1$ of Radius R is given by
	\begin{equation}
	Z^{25}=\Tr q^{L_0-\frac{1}{24}} \bar{q}^{\bar{L}_0-\frac{1}{24}}
	\end{equation}
    $L_0$ and $\bar{L}_0$ are zero mode virasoro generators
    \begin{align}
    L_0 &= \frac{\alpha'{p_L}^2}{4}+\sum_{n=1}^{\infty}  \hat{\alpha}_{-n}\hat{\alpha}_{n}\\
    \bar{L}_0 &= \frac{\alpha'{p_R}^2}{4}+\sum_{n=1}^{\infty}  \hat{\tilde{\alpha}}_{-n}\hat{\tilde{\alpha}}_{n}
    \end{align} 
    with right and left momentum
    \begin{align}
    p_L &= \frac{n}{R}+\frac{wR}{\alpha'}\\
    p_R &= \frac{n}{R}-\frac{wR}{\alpha'}
    \end{align}
    Then we can express $Z^{25}$ in terms of Dedekind $\eta$ function
    \begin{align}
    Z^{25}&=(q\bar{q})^{-\frac{1}{24}}
      \Tr q^{\frac{\alpha'{p_L}^2}{4}+\sum_{n=1}^{+\infty}  \hat{\alpha}_{-n}\hat{\alpha}_{n}} \bar{q}^{\frac{\alpha'{p_R}^2}{4}+\sum_{n=1}^{+\infty}  \hat{\tilde{\alpha}}_{-n}\hat{\tilde{\alpha}}_{n}}\notag\\
    &=(q\bar{q})^{-\frac{1}{24}} 
      \sum_{n,w=-\infty}^{+\infty}q^{\frac{\alpha'{p_L}^2}{4}}\bar{q}^{\frac{\alpha'{p_R}^2}{4}}
      \sum_{N_n,\widetilde{N}_n=0}^{+\infty}q^{nN_n}\bar{q}^{n\widetilde{N}_n}\notag\\
    &=(q\bar{q})^{-\frac{1}{24}} 
      \sum_{n,w=-\infty}^{+\infty}q^{\frac{\alpha'{p_L}^2}{4}}\bar{q}^{\frac{\alpha'{p_R}^2}{4}}
      \prod_{n=1}^{+\infty}\abs{1-q^n}^{-2}\notag\\
    &= \abs{\eta(q)}^{-2}
      \sum_{n,w=-\infty}^{+\infty}q^{\frac{\alpha'{p_L}^2}{4}}\bar{q}^{\frac{\alpha'{p_R}^2}{4}}
    \end{align}
    The partition function for the non-compact space $\mathbb{R}^{1,24}$
    \begin{equation}
    Z^{1,24} = \text{const} \times \abs{\eta(q)}^{-46}
    \end{equation}
    The matter part contributes $\abs{\eta(q)}^{-50}$(25 dimension), and the ghost part contribute $\abs{\eta(q)}^4$ as usual. 
    
    To recover each mass formula term. We need expand the whole partion function.
    \begin{align}
    Z^{1,24}Z^{25} & = \abs{\eta(q)}^{-46}
      \abs{\eta(q)}^{-2}
      \sum_{n,w=-\infty}^{+\infty}q^{\frac{\alpha'{p_L}^2}{4}}\bar{q}^{\frac{\alpha'{p_R}^2}{4}}\notag\\
    &=\big[(q\bar{q})^{-\frac{1}{24}}
      \prod_{n=1}^{+\infty}(1-q^n)^{-1}
      (1-\bar{q  }^n)^{-1}\big]^{24}
      \sum_{n,w=-\infty}^{+\infty}q^{\frac{\alpha'{p_L}^2}{4}}\bar{q}^{\frac{\alpha'{p_R}^2}{4}}\notag\\
    &=\big[(q\bar{q})^{-\frac{1}{24}}
      \prod_{n=1}^{+\infty}
      \sum_{N_n=0}^{+\infty}q^{nN_n}
      \sum_{\widetilde{N}_n=0}^{+\infty}\bar{q}^{n\widetilde{N}_n}
      \big]^{24}
      \sum_{n,w=-\infty}^{+\infty}q^{\frac{\alpha'{p_L}^2}{4}}\bar{q}^{\frac{\alpha'{p_R}^2}{4}}\notag\\
    &=\sum_{N_n=0}^{+\infty}
      \sum_{\tilde{N}_n=0}^{+\infty}
      \sum_{n,w=-\infty}^{+\infty}
      q^{\frac{\alpha'{p_L}^2}{4}-1+\sum_{n=1}^{\infty}24nN_n}
      \bar{q}^{\frac{\alpha'{p_R}^2}{4}-1+\sum_{n=1}^{\infty}24n\widetilde{N}_n}
    \end{align}
    The number of state: Pochinski (1.3.37)
    \begin{equation}
    N = \sum_{2}^{D-1}\sum_{n=1}^{+\infty}nN_{im}
    \end{equation}
    Where D is dimension. Rewrite the partition function
    \begin{equation}
    Z^{1,24}Z^{25}=
      \sum_{N_n=0}^{+\infty}
      \sum_{\widetilde{N}_n=0}^{+\infty}
      \sum_{n,w=-\infty}^{+\infty}
      q^{\frac{\alpha'{p_L}^2}{4}-1+N}
      \bar{q}^{\frac{\alpha'{p_R}^2}{4}-1+\widetilde{N}}
    \end{equation} 
    For each term, which means for each fixed $N_n$,$\widetilde{N}_n$,$n$ and $w$, it should be real which gives the level matching conditon
    \begin{align}
    0 &=\frac{\alpha'p_L^2}{4}+N - \frac{\alpha'p_R^2}{4}-\tilde{N}\notag\\
      &=\frac{\alpha'}{4}\times\frac{4nw}{\alpha'}+N-\widetilde{N}\notag\\
      &= nw+N-\widetilde{N}
    \end{align}
    Which is the desired level matching condition.
    The partition function now writes
    \begin{align}
    Z^{1,24}Z^{25}&=
      \sum_{N_n=0}^{+\infty}
      \sum_{\widetilde{N}_n=0}^{+\infty}
      \sum_{n,w=-\infty}^{+\infty}
      \abs{q}^{\frac{\alpha'{p_L}^2}{4}-1+N+\frac{\alpha'{p_R}^2}{4}-1+\widetilde{N}}\notag\\
    &=\sum_{N_n=0}^{+\infty}
      \sum_{\widetilde{N}_n=0}^{+\infty}
      \sum_{n,w=-\infty}^{+\infty}
      \abs{q}^{\frac{\alpha'}{2}\big[\frac{n^2}{R^2}+\frac{\omega^2R^2}{\alpha'^2}+\frac{2}{\alpha'}(N+\widetilde{N}-2)\big]}
    \end{align}
    So each term means the right hand sides of the mass formula $M^2 = \frac{n^2}{R^2}+\frac{\omega^2R^2}{\alpha'^2}+\frac{2}{\alpha'}(N+\widetilde{N}-2)$.
    
\subsection{}
    Write the more explicit form of partition funciton $Z^{25}$
    \begin{align}
    Z^{25} =&
      \abs{\eta(q)}^{-2}
      \sum_{n,w=-\infty}^{+\infty}
      q^{\frac{\alpha'{p_L}^2}{4}}
      \bar{q}^{\frac{\alpha'{p_R}^2}{4}}\notag\\
    =&\abs{\eta(q)}^{-2}
      \sum_{n,w=-\infty}^{+\infty}
      \exp{2\pi i(\tau_1 + i \tau_2)\times \frac{\alpha'}{4}\big(\frac{n}{R}+\frac{wR}{\alpha'}\big)^2}\times\\
      &\times\exp{-2\pi i(\tau_1 - i \tau_2)\times \frac{\alpha'}{4}\big(\frac{n}{R}-\frac{wR}{\alpha'}\big)^2}
    \end{align} 
    Calculate the factor in exponential
    \begin{align}
    &2\pi i(\tau_1 + i \tau_2)\times \frac{\alpha'}{4}\big(\frac{n}{R}+\frac{wR}{\alpha'}\big)^2-2\pi i(\tau_1 - i \tau_2)\times \frac{\alpha'}{4}\big(\frac{n}{R}-\frac{wR}{\alpha'}\big)^2\notag\\
    =& 2\pi i \tau_1 \times \frac{\alpha'}{4}\times \frac{4nw}{\alpha'}
       -2\pi\tau_2 \times \frac{\alpha'}{4}\times 2\big(\frac{n^2}{R^2}+\frac{w^2R^2}{\alpha'^2}\big)\notag\\
    =& 2\pi i \tau_1 nw - \pi  \tau_2\big(\frac{\alpha'n^2}{R^2}+\frac{w^2R^2}{\alpha'}\big)
    \end{align}
    The partition function now is
    \begin{equation}
    Z^{25}=\abs{\eta(q)}^{-2}
    \sum_{n,w=-\infty}^{+\infty}
    \exp{ 2\pi i \tau_1 nw - \pi  \tau_2\big(\frac{\alpha'n^2}{R^2}+\frac{w^2R^2}{\alpha'}\big)}
    \end{equation}
    Using the Poisson resummation formula.
    \begin{equation}
    \sum_{n=-\infty}^{\infty} \exp(-\pi a n^2 +2\pi i bn ) = a^{-1/2}\sum_{m=-\infty}^{\infty}\exp[-\frac{\pi(m-b)^2}{a}]
    \end{equation} 
    By replacement $a\rightarrow \frac{\tau_2\alpha'}{R^2}$ and $b \rightarrow \tau_1 w$ the partition formula is
    \begin{align}
    Z^{25} &= \abs{\eta(q)}^{-2}
      \big(\frac{\tau_2\alpha'}{R^2}\big)^{-1/2}
      \sum_{w=-\infty}^{+\infty}
      \exp(-\pi\tau_2\frac{w^2R^2}{\alpha'})
      \sum_{m=-\infty}^{\infty}
      \exp[-\frac{\pi R^2(m-\tau_1 w)^2}{\tau_2 \alpha'}]\notag\\
    &=\abs{\eta(q)}^{-2}
      \big(\frac{\tau_2\alpha'}{R^2}\big)^{-1/2}
      \sum_{w,m=-\infty}^{+\infty}
      \exp[-\frac{\pi R^2}{\tau_2\alpha'}(\tau_2^2w^2 + m^2 -2m\tau_1w +\tau^2_1w^2)]\notag\\
    &= \abs{\eta(q)}^{-2}
      \big(\frac{\tau_2\alpha'}{R^2}\big)^{-1/2}
      \sum_{w,m=-\infty}^{+\infty}
      \exp[-\frac{\pi R^2}{\tau_2\alpha'}\abs{m-w\tau}^2]
    \end{align}
    Now illustrate the modular invariance of the partition function. Under the transformation $\tau \rightarrow \tau+1 $,
    $\tau_2 \rightarrow \tau_2$,
    $\eta(\tau) \rightarrow e^{i\pi/12}\eta(\tau)$. Thus the partition funciton changes
    \begin{align}
    Z^{25}=&\abs{\eta(q)}^{-2}
      \big(\frac{\tau_2\alpha'}{R^2}\big)^{-1/2}
      \sum_{w,m=-\infty}^{+\infty}
      \exp[-\frac{\pi R^2}{\tau_2\alpha'}\abs{m-w\tau}^2]\notag\\
      \rightarrow &\abs{\eta(q)}^{-2}
      \big(\frac{\tau_2\alpha'}{R^2}\big)^{-1/2}
      \sum_{w,m=-\infty}^{+\infty}
      \exp[-\frac{\pi R^2}{\tau_2\alpha'}\abs{m-w(\tau+1)}^2]
    \end{align}
    is invariant if we replace the sum index $m$ by $m+w$.
    
    Under the transformation $\tau\rightarrow -1/\tau$, we have $\tau_2 \rightarrow \tau_2/\abs{\tau}^2$, $\eta(\tau) \rightarrow \sqrt{-i\tau}\eta(\tau)$.
    \begin{align}
    Z^{25}=&\abs{\eta(q)}^{-2}
    \big(\frac{\tau_2\alpha'}{R^2}\big)^{-1/2}
    \sum_{w,m=-\infty}^{+\infty}
    \exp[-\frac{\pi R^2}{\tau_2\alpha'}\abs{m-w\tau}^2]\notag\\
    \rightarrow& \abs{\eta(q)}^{-2}
    \sqrt{\tau}
    \big(\frac{\tau_2\alpha'}{R^2}\big)^{-1/2}
    \sqrt{\tau}^{-1}
    \sum_{w,m=-\infty}^{+\infty}
    \exp[-\frac{\pi R^2}{\tau_2\alpha'}\abs{\tau}^2\abs{m-w(-1/\tau)}^2]\notag\\
    =& \abs{\eta(q)}^{-2}
    \big(\frac{\tau_2\alpha'}{R^2}\big)^{-1/2}
    \sum_{w,m=-\infty}^{+\infty}
    \exp[-\frac{\pi R^2}{\tau_2\alpha'}\abs{m\tau+w}^2]
    \end{align}
    is invariant if we replace the sum index $m\rightarrow-w$ and $w\rightarrow m$.
    
\subsection{}
    In the case of self-dual $R = \sqrt{\alpha'}$. The partition function $Z^{25}$ now is
    \begin{align}
    Z^{25}=&\abs{\eta(q)}^{-2}
      \sum_{n,w=-\infty}^{+\infty}
      q^{\frac{\alpha'{p_L}^2}{4}}
      \bar{q}^{\frac{\alpha'{p_R}^2}{4}}\notag\\
    =&\abs{\eta(q)}^{-2}
      \sum_{n,w=-\infty}^{+\infty}
      q^{\frac{(n+w)^2}{4}}
      \bar{q}^{\frac{(n-w)^2}{4}}\notag\\
    \end{align}
    Do the replacement $n \rightarrow n-w $ 
    \begin{align}
    Z^{25} =& \abs{\eta(q)}^{-2}
       \sum_{n,w=-\infty}^{+\infty}
       q^{\frac{n^2}{4}}
       \bar{q}^{\frac{(n-2w)^2}{4}}\notag\\
    =& \abs{\eta(q)}^{-2}
       \sum_{w=-\infty}^{+\infty}
       \bigg(\sum_{2k+1 ,k\in\mathbb{Z}}
       q^{\frac{(2k+1)^2}{4}}
       \bar{q}^{\frac{(2k+1-2w)^2}{4}}
       +\sum_{2k,k\in \mathbb{Z}}
       q^{\frac{(2k)^2}{4}}
       \bar{q}^{\frac{(2k-2w)^2}{4}}\bigg)\notag\\
    =& \abs{\eta(q)}^{-2}
       \sum_{w,k=-\infty}^{+\infty}
        q^{(k+1/2)^2}
        \bar{q}^{(k-w+1/2)^2}
        +q^{k^2}
        \bar{q}^{(k-w)^2}
    \end{align}
    With the replacement $w \rightarrow -w+k$, we have
    \begin{align}
    Z^{25} =& \abs{\eta(q)}^{-2}
      \sum_{w,k=-\infty}^{+\infty}
      q^{(k+1/2)^2}
      \bar{q}^{(w+1/2)^2}
      +q^{k^2}
      \bar{q}^{w^2}\notag\\
    &=\abs{\eta^{-1}\sum_{n\in\mathbb{Z}}q^{n^2}}^2
      +\abs{\eta^{-1}\sum_{n\in\mathbb{Z}}q^{n^2+1/2}}^2\notag\\
    &=\abs{\chi_1(q)}^2+\abs{\chi_2(q)}^2
    \end{align}
    The $\chi_i$ are the characters of the SU(2) affine Lie algebra with level k = 1.
    Now the whole partition function is
    \begin{equation}
    Z^{1,24}Z^{25} =\text{const}\times
      \abs{\eta(q)}^{-48}
      \sum_{w,k=-\infty}^{+\infty}
      q^{(k+1/2)^2}
      \bar{q}^{(w+1/2)^2}
      +q^{k^2}
      \bar{q}^{w^2}
    \end{equation}
    According to lecture note 8, first factor $\abs{\eta(q)}^{-48} \simeq \abs{q^{-1}+24+O(q)}^2$. Its easy to see that the terms with $k = w = 0$ in the second factor contribute to usual 24 bosonic massless states. However $\abs{q}^2$, $q$, $\bar{q}$ terms in the  the second factor gives contributions to additional massless states, which corresponds to eight cases $k,w = \pm 1$ , $k = 0 , w = \pm 1$ and $k = \pm 1 , w = 0$. 
    
    We now briefly conclude these cases. We denote $w , n$ in original partition function as $w_o, n_o$
    \begin{align}
    & k = w = \pm 1 \qquad n_o = \pm 2 \qquad w_o= 0 \qquad N = \widetilde{N} =0\notag\\
    & k = -w = \pm 1 \qquad  n_o = 0 \qquad w_o= \pm 2 \qquad N = \widetilde{N} =0\notag\\
    & k = \pm 1 \qquad w=0 \qquad n_o = w_o = \pm 1 \qquad N =0\quad \widetilde{N} = 1\notag\\
    & k = 0 \qquad w= \pm 1 \qquad n_o = -w_o = \pm 1 \qquad N =1\quad \widetilde{N} = 0\notag
    \end{align} 
    The last four states are physical additional massless states with nonzero state number. 
    
    \subsection{}
    The current in the bosonic string theory is defined by
    \begin{align}
    j^{\pm}(z) &= j^{1}(z)\pm ij^2(z)\coloneq e^{\pm2iX^{25}(z)/\sqrt{\alpha'}}\notag\\
    j^{3}(z) &\coloneq i\partial X^{25}(z)/\sqrt{\alpha'}
    \end{align}
    so we can write $j^1(z)$ and $j^2(z)$
    \begin{align}
    j^1(z)&=\frac{1}{2}(j^{+}(z)+j^{-}(z)) = \frac{1}{2}(e^{2iX^{25}(z)/\sqrt{\alpha'}}+e^{-2iX^{25}(z)/\sqrt{\alpha'}})\\
    j^2(z)&=\frac{1}{2i}(j^{+}(z)+j^{-}(z)) = \frac{1}{2i}(e^{2iX^{25}(z)/\sqrt{\alpha'}}-e^{-2iX^{25}(z)/\sqrt{\alpha'}})
    \end{align}
    Useful formula which have proved in previous notes or exercises:
    \begin{align}
    X^{25}(z)X^{25}(0) &\sim -\frac{\alpha'}{2}\ln z\\
    :e^{ik_1X^{25}(z)}::e^{ik_2X^{25}(0)}: &= z^{\frac{\alpha'}{2}k_1k_2}:e^{ik_1X^{25}(z)}e^{ik_2X^{25}(0)}:\\
    :\partial X^{25}(z)::e^{ikX^{25}(0)}: &\sim \partial(-\frac{\alpha'}{2}\ln z) ik :e^{ikX^{25}(0)}:
    \end{align}
    Calculate OPE $j^3j^3$
    \begin{align}
    j^3(z)j^3(0) &= -\frac{1}{\alpha'}
      \partial X^{25}(z)
      \partial X^{25}(0)
      \sim \frac{1}{\alpha'}
      \partial^2(-\frac{\alpha'}{2}\ln z)
      = \frac{1}{2z^2}
    \end{align}
    OPE $j^1j^1$
    \begin{align}
    j^1(z)j^1(0)=&\frac{1}{4}:
      (e^{2iX^{25}(z)/\sqrt{\alpha'}}+
      e^{-2iX^{25}(z)/\sqrt{\alpha'}}):
      :(e^{2iX^{25}(0)/\sqrt{\alpha'}}+
      e^{-2iX^{25}(0)/\sqrt{\alpha'}}):\notag\\
    =& \frac{1}{4}
      \bigg(
      z^{2}
      :e^{2iX^{25}(z)/\sqrt{\alpha'}}
      e^{2iX^{25}(0)/\sqrt{\alpha'}}:+
      z^{-2}
      :e^{2iX^{25}(z)/\sqrt{\alpha'}}
      e^{-2iX^{25}(0)/\sqrt{\alpha'}}:+\notag\\
      &+z^{-2}
      :e^{-2iX^{25}(z)/\sqrt{\alpha'}}
      e^{2iX^{25}(0)/\sqrt{\alpha'}}:+
      z^{2}
      :e^{-2iX^{25}(z)/\sqrt{\alpha'}}
      e^{-2iX^{25}(0)/\sqrt{\alpha'}}:
      \bigg)\notag\\
    =&\frac{1}{4}
      \bigg[
      z^{-2}
      (1+\frac{2i}{\sqrt{\alpha'}}\partial X^{25}(0)z+O(z^2))+
      z^{-2}
      (1-\frac{2i}{\sqrt{\alpha'}}\partial X^{25}(0)z+O(z^2))+
      O(z^2)
      \bigg]\notag\\
    =& \frac{1}{2z^2} + O(z^2)\notag\\
    \sim& \frac{1}{2z^2}
    \end{align}
    Similar for OPE $j^2j^2$
    \begin{align}
    j^2(z)j^2(0)=&-\frac{1}{4}:
    (e^{2iX^{25}(z)/\sqrt{\alpha'}}-
    e^{-2iX^{25}(z)/\sqrt{\alpha'}}):
    :(e^{2iX^{25}(0)/\sqrt{\alpha'}}-
    e^{-2iX^{25}(0)/\sqrt{\alpha'}}):\notag\\
    =& -\frac{1}{4}
    \bigg(
    z^{2}
    :e^{2iX^{25}(z)/\sqrt{\alpha'}}
    e^{2iX^{25}(0)/\sqrt{\alpha'}}:-
    z^{-2}
    :e^{2iX^{25}(z)/\sqrt{\alpha'}}
    e^{-2iX^{25}(0)/\sqrt{\alpha'}}:+\notag\\
    &-z^{-2}
    :e^{-2iX^{25}(z)/\sqrt{\alpha'}}
    e^{2iX^{25}(0)/\sqrt{\alpha'}}:+
    z^{2}
    :e^{-2iX^{25}(z)/\sqrt{\alpha'}}
    e^{-2iX^{25}(0)/\sqrt{\alpha'}}:
    \bigg)\notag\\
    =&\frac{1}{4}
    \bigg[
    z^{-2}
    (1+\frac{2i}{\sqrt{\alpha'}}\partial X^{25}(0)z+O(z^2))+
    z^{-2}
    (1-\frac{2i}{\sqrt{\alpha'}}\partial X^{25}(0)z+O(z^2))+
    O(z^2)
    \bigg]\notag\\
    =& \frac{1}{2z^2} + O(z^2)\notag\\
    \sim& \frac{1}{2z^2}
    \end{align}
    Calculate OPE $j^3j^1$
    \begin{align}
    j^3(z)j^1(0) =& 
      \frac{i}{2\sqrt{\alpha'}}
      \partial X^{25}(z)
      :(
      e^{2iX^{25}(0)/\sqrt{\alpha'}}+
      e^{-2iX^{25}(0)/\sqrt{\alpha'}}
      ):\notag \\
    \sim&\frac{i}{2\sqrt{\alpha'}}
    \bigg(
    -\frac{\alpha'}{2}\times \frac{2i}{\sqrt{\alpha'}}
    z^{-1}
    :e^{2iX^{25}(0)/\sqrt{\alpha'}}:
    -\frac{\alpha'}{2}\times \frac{-2i}{\sqrt{\alpha'}}
    z^{-1}
    :e^{-2iX^{25}(0)/\sqrt{\alpha'}}:
    \bigg)\notag\\
    =&\frac{i}{z}
    \frac{1}{2i}
    \bigg(
    :e^{2iX^{25}(0)/\sqrt{\alpha'}}:-
    :e^{-2iX^{25}(0)/\sqrt{\alpha'}}:
    \bigg)\notag\\
    =&\frac{i}{z}j^2(0)   
    \end{align}
    When calculate OPE $j^1j^3$ , the $\partial$ in $j^3$ act on $"0"$ offer an extra minus sign, so we have
    \begin{equation}
    j^1(z)j^3(0) \sim -\frac{i}{z}j^2(0)
    \end{equation}
    Similar calculation for OPE $j^3j^2$
    \begin{align}
    j^3(z)j^2(0) =& 
    \frac{1}{2\sqrt{\alpha'}}
    \partial X^{25}(z)
    :(
    e^{2iX^{25}(0)/\sqrt{\alpha'}}-
    e^{-2iX^{25}(0)/\sqrt{\alpha'}}
    ):\notag \\
    \sim&\frac{1}{2\sqrt{\alpha'}}
    \bigg(
    -\frac{\alpha'}{2}\times \frac{2i}{\sqrt{\alpha'}}
    z^{-1}
    :e^{2iX^{25}(0)/\sqrt{\alpha'}}:
    +\frac{\alpha'}{2}\times \frac{-2i}{\sqrt{\alpha'}}
    z^{-1}
    :e^{-2iX^{25}(0)/\sqrt{\alpha'}}:
    \bigg)\notag\\
    =&-\frac{i}{z}
    \frac{1}{2}
    \bigg(
    :e^{2iX^{25}(0)/\sqrt{\alpha'}}:+
    :e^{-2iX^{25}(0)/\sqrt{\alpha'}}:
    \bigg)\notag\\
    =&-\frac{i}{z}j^1(0)\\
     j^2(z)j^3(0) =& \frac{i}{z}j^1(0)   
    \end{align}
    Finally, we goes to OPE $j^1j^2$
    \begin{align}
    j^1(z)j^2(0) =&\frac{1}{4i}:
    (e^{2iX^{25}(z)/\sqrt{\alpha'}}+
    e^{-2iX^{25}(z)/\sqrt{\alpha'}}):
    :(e^{2iX^{25}(0)/\sqrt{\alpha'}}-
    e^{-2iX^{25}(0)/\sqrt{\alpha'}}):\notag\\
    =& \frac{1}{4i}
    \bigg(
    z^{2}
    :e^{2iX^{25}(z)/\sqrt{\alpha'}}
    e^{2iX^{25}(0)/\sqrt{\alpha'}}:-
    z^{-2}
    :e^{2iX^{25}(z)/\sqrt{\alpha'}}
    e^{-2iX^{25}(0)/\sqrt{\alpha'}}:+\notag\\
    &+z^{-2}
    :e^{-2iX^{25}(z)/\sqrt{\alpha'}}
    e^{2iX^{25}(0)/\sqrt{\alpha'}}:-
    z^{2}
    :e^{-2iX^{25}(z)/\sqrt{\alpha'}}
    e^{-2iX^{25}(0)/\sqrt{\alpha'}}:
    \bigg)\notag\\
    =&\frac{1}{4i}
    \bigg[
    -z^{-2}
    (1+\frac{2i}{\sqrt{\alpha'}}\partial X^{25}(0)z+O(z^2))+
    z^{-2}
    (1-\frac{2i}{\sqrt{\alpha'}}\partial X^{25}(0)z+O(z^2))+
    O(z^2)
    \bigg]\notag\\
    \sim&\frac{i}{z}\frac{i}{\sqrt{\alpha'}}\partial X^{25}(0)
    =\frac{i}{z}j^3(0)
    \end{align}
    Follow the similar pattern we have 
    \begin{equation}
    j^2(z)j^1(0) \sim -\frac{i}{z}\frac{i}{\sqrt{\alpha'}}\partial X^{25}(0)=-\frac{i}{z}j^3(0)
    \end{equation}
    Thus the OPE can be concluded
    \begin{equation}
    j^a(z)j^b(0) \sim \frac{\delta^{ab}}{2z^2}+\frac{i\epsilon^{abc}j^c(0)}{z}
    \end{equation}
    The oscillator modes of the currents
    \begin{equation}
    j^a(z)=\sum_{m\in \mathbb{Z}}\frac{j^a_m}{z^{m+1}}
    \end{equation}
    So we have
    \begin{equation}
    j^a_m = \oint_{c}\frac{dz}{2\pi i} z^m j^a(z)
    \end{equation}
    The commutator $\comm{j^a_m}{j^b_n}$ is
    \begin{align}
    \comm{j^a_m}{j^b_n}&=
      \bigg(\oint_{c_1}\oint_{c_2}-
      \oint_{c_2}\oint_{c_3}\bigg)
      \frac{dz}{2\pi i}\frac{dw}{2\pi i}
      z^m w^n
      j^a(z)j^b(z) \notag\\
    & =\oint_{c_2}
      \frac{dw}{2\pi i}
      \oint_{c_W}
      \frac{dz}{2\pi i}
      z^m w^n
      \bigg[
      \frac{\delta^{ab}}{2(z-w)^2}+\frac{i\epsilon^{abc}j^c(w)}{z-w}
      \bigg]\notag\\
    & = \oint_{c_2}
      \frac{dw}{2\pi i}
      \bigg[
      \frac{1}{2}m\omega^{n+m-1}\delta^{ab}+i\epsilon^{abc}j^{c}(\omega)\omega^{m+n}
      \bigg]\notag\\
    & = \delta_{n+m,0}\frac{1}{2}m\delta^{ab}+i\epsilon^{abc}j^{c}_{m+n}
    \end{align}
    
    
    
    
    \section{RR field strengths and T-duality in Type II}
    The Clifford algebra of SO(1,9) gamma matrices defined as
    \begin{equation}
    \acomm {\Gamma^{\mu}}{\Gamma^{\nu}} = 2 \eta^{\mu\nu} 
      \qquad \mu,\nu =0,1 \dots 9
    \end{equation} 
    The chirality matrix is $\Gam{11}=\Gam{0}\Gam{1}\cdots\Gam{9}$, calculate $\Gam{11}\Gam{11}$:
    \begin{equation}
    \Gam{11}\Gam{11} = \Gam{0}\Gam{1}\cdots\Gam{9}\Gam{0}\Gam{1}\cdots\Gam{9} = (-1)^{9+8+\cdots+1}(\Gam{0})^2(\Gam{1})^2\cdots(\Gam{9})^2 =(-1)^{46} = 1
    \end{equation} 
    The anticommutator of $\Gam{11}$ and $\Gam{\mu}$ vanishes
    \begin{equation}
    \acomm{\Gam{11}}{\Gam{\mu}} 
    =\Gam{11}\Gam{\mu}+\Gam{\mu}\Gam{11}
    =\Gam{11}\Gam{\mu}+(-1)^{9}\Gam{11}\Gam{\mu}
    =0
    \end{equation}
    Chiral spinors are now defined by 
    $\Gam{11}\psi_{\pm} = \pm \psi_{\pm}$ , then we have
    \begin{align}
    \bar{\psi}_{\pm}\Gam{11}
    =\psi_{\pm}^{\dagger}\Gam{0}\Gam{11}
    =(\Gam{11}^{\dagger}\Gam{0}^{\dagger}\psi_{\pm})^{\dagger}
    \end{align}
    Notice that
    \begin{align}
    \Gam{11}^{\dagger}
    =\Gam{9}^{\dagger}\cdots \Gam{0}^{\dagger}
    =\Gam{0}\Gam{9}\Gam{8}\cdots\Gam{0}(\Gam{0})^{-1}
    =(-1)^{45}\Gam{0}\Gam{0}\Gam{1}\cdots\Gam{9}(\Gam{0})^{-1}
    =-\Gam{0}\Gam{11}(\Gam{0})^{-1}
    \end{align}
    So, we have
    \begin{align}
    \bar{\psi}_{\pm}\Gam{11}
    =(\Gam{0}\Gam{11}\Gam{0}(\Gam{0})^{-1}\psi_{\pm})^{\dagger}
    =\pm(\Gam{0}\psi_{\pm})^{\dagger}
    =\pm\psi_{\pm}^{\dagger}(-\Gam{0})
    =\mp\bar{\psi}_{\pm}
    \end{align} 
    The R R field strengths $G_{\mu_1\dots\mu_{p+1}}$ as bilinears
    \begin{align}
    \text{II A}: \quad \bar{\psi}_{-}^{L}\Gam{\mu_1\cdots\mu_{p+1}}\psi_{+}^R
    \qquad
    \text{II B}: \quad \bar{\psi}_{+}^{L}\Gam{\mu_1\cdots\mu_{p+1}}\psi_{+}^R
    \end{align}
    Since the spacetime indexes are antisymmetric  $p+1 \leq 10 $ .
    Consider Type II A
    \begin{align}
    \bar{\psi}_{-}^{L}\Gam{\mu_1\cdots\mu_{p+1}}\psi_{+}^R
    =\bar{\psi}_{-}^{L}\Gam{11}\Gam{\mu_1\cdots\mu_{p+1}}\Gam{11}\psi_{+}^R
    =\bar{\psi}_{-}^{L}(\Gam{11})^{2}(-1)^{p+1}\Gam{\mu_1\cdots\mu_{p+1}}\psi_{+}^R
    =(-1)^{p+1}\bar{\psi}_{-}^{L}\Gam{\mu_1\cdots\mu_{p+1}}\psi_{+}^R
    \end{align}
    So, for type II A, the spinor bilinears will vanish unless p is odd. $p$ can be $1,3,5,7,9$ .
    Similar for Type II B
    \begin{align}
    \bar{\psi}_{+}^{L}\Gam{\mu_1\cdots\mu_{p+1}}\psi_{+}^R
    =-\bar{\psi}_{+}^{L}\Gam{11}\Gam{\mu_1\cdots\mu_{p+1}}\Gam{11}\psi_{+}^R
    =(-1)^{p+2}\bar{\psi}_{+}^{L}\Gam{\mu_1\cdots\mu_{p+1}}\psi_{+}^R
    \end{align}
    For type II B, the spinor bilinears will vanish unless p is even. So $p$ can be $0,2,4,6,8$ 
    
    Since $\beta_9 = \Gam{9}\Gam{11}$, we have 
    \begin{align}
    \acomm{\beta_9}{\Gam{9}} = \acomm{\Gam{9}\Gam{11}}{\Gam{9}}
    =\Gam{9}\acomm{\Gam{11}}{\Gam{9}}-\comm{\Gam{9}}{\Gam{9}}\Gam{11}=0
    \end{align}
    For $\mu \neq 9$
    \begin{align}
    \comm{\beta_9}{\Gam{\mu}} = \comm{\Gam{9}\Gam{11}}{\Gam{\mu}}
    =\Gam{9}\acomm{\Gam{11}}{\Gam{\mu}}-\acomm{\Gam{9}}{\Gam{\mu}}\Gam{11}
    =0
    \end{align}
    Under  T-duality transformation $\psi^{L}\rightarrow\psi^{L}$, $\psi^{R}\rightarrow \beta_9 \psi^{R}$
    \begin{align}
    \bar{\psi}_{-}^{L}\Gam{\mu_1\cdots\mu_{p+1}}\psi_{+}^R
    \rightarrow
    \bar{\psi}_{-}^{L}\Gam{\mu_1\cdots\mu_{p+1}}\beta_{9}\psi_{+}^R
    \end{align}
    Follow the same step, we see that
    \begin{align}
    \bar{\psi}_{-}^{L}\Gam{\mu_1\cdots\mu_{p+1}}\beta_{9}\psi_{+}^R
    =\bar{\psi}_{-}^{L}\Gam{11}\Gam{\mu_1\cdots\mu_{p+1}}\beta_{9}\Gam{11}\psi_{+}^R=(-1)^{p+2}\bar{\psi}_{-}^{L}\Gam{\mu_1\cdots\mu_{p+1}}\beta_{9}\psi_{+}^R
    \end{align}
    The transformed spinor bilinear vanishes unless $p$ is even which is the case of type II B. And we find $\bar{\psi}_{-}^{L}\Gam{\mu_1\cdots\mu_{p+1}}\beta_{9}\psi_{+}^R \sim \bar{\psi}_{-}^{L}\beta_{9}\Gam{\mu_1\cdots\mu_{p+1}}\psi_{+}^R$ up to a sign factor. And 
    \begin{align}
    \bar{\psi}_{-}^{L}\beta_{9}\Gam{11} = -\bar{\psi}_{-}^{L}\Gam{11}\beta_{9}
    =+ \bar{\psi}_{-}^{L}\beta_{9}
    \end{align}
    So $\bar{\psi}_{-}^{L}\beta_{9}\sim \bar{\psi}_{+}^{L}$. Thus T duality transform RR field strenghths in II A to II B, and the same reason vise versa.
    \begin{align}
    &\bar{\psi}_{+}^{L}\Gam{\mu_1\cdots\mu_{p+1}}\psi_{+}^R
     \rightarrow
     \bar{\psi}_{+}^{L}\Gam{\mu_1\cdots\mu_{p+1}}\beta_{9}\psi_{+}^R\\
    &\bar{\psi}_{+}^{L}\Gam{\mu_1\cdots\mu_{p+1}}\beta_{9}\psi_{+}^R
     =\bar{\psi}_{+}^{L}\Gam{11}\Gam{\mu_1\cdots\mu_{p+1}}\beta_{9}\Gam{11}\psi_{+}^R=(-1)^{p+2}\bar{\psi}_{+}^{L}\Gam{\mu_1\cdots\mu_{p+1}}\beta_{9}\psi_{+}^R
    \end{align}
    The transformed spinor bilinear vanishes unless $p$ is odd which is the case of type II A. And we find $\bar{\psi}_{+}^{L}\Gam{\mu_1\cdots\mu_{p+1}}\beta_{9}\psi_{+}^R \sim \bar{\psi}_{+}^{L}\beta_{9}\Gam{\mu_1\cdots\mu_{p+1}}\psi_{+}^R$ up to a sign factor. And 
    \begin{align}
    \bar{\psi}_{+}^{L}\beta_{9}\Gam{11} = -\bar{\psi}_{+}^{L}\Gam{11}\beta_{9}
    =- \bar{\psi}_{+}^{L}\beta_{9}
    \end{align}
     So $\bar{\psi}_{+}^{L}\beta_{9}\sim \bar{\psi}_{-}^{L}$.
     
     \section{D-branes in Type II and T- duality}
     \subsection{}
     Since  T-duality exchange NN to DD and ND to DN in corresponding spacetime direction, the number (\# NN + \# DD) and (\# ND + \# DN) are invariant.
     \subsection{}
     Without loss of generality, we place all the spacetime dimension as small as possible. For II B theory p must be odd .Above tables list all the possible configuration, and all these configurations are dual to the D1-D5 configuration.
     \begin{table}[!hbp]
     	\centering
     	\begin{tabular}{lclclclclclclclclclclcl}
     	         &0       &1       &2       &3       &4      &5  
     	         &6       &7       &8       &9  \\
     		\midrule
     		D3   &$\times$&$\times$&$\times$&$\times$&        &   
     		     &        &        &        &   \\
     		D1   &$\times$&        &        &        &$\times$&   
     		     &        &        &        &   \\
     	\end{tabular}
     	\caption{ transform to D1-D5 by $T_4,T_5$}
     \end{table}
     \begin{table}[!hbp]
     	\centering
     	\begin{tabular}{lclclclclclclclclclclcl}
     		     &0       &1       &2       &3       &4      &5  
     		     &6       &7       &8       &9           \\
     		\midrule
     		D3   &$\times$&$\times$&$\times$&$\times$&        &   
     		     &        &        &        &            \\
     		D3   &$\times$&$\times$&        &        &$\times$&   
     		      $\times$&        &        &        &   \\
     	\end{tabular}
     	\caption{ transform to D1-D5 by $T_2,T_3$}
     \end{table}
     \begin{table}[!hbp]
     	\centering
     	\begin{tabular}{lclclclclclclclclclclcl}
     		     &0       &1       &2       &3       &4      &5  
     		     &6       &7       &8       &9           \\
     		\midrule
     		D3   &$\times$&$\times$&$\times$&$\times$&        &   
     		     &        &        &        &            \\
     		D5   &$\times$&$\times$&$\times$&        &$\times$&   
     		      $\times$&$\times$&        &        &   \\
     	\end{tabular}
     	\caption{ transform to D1-D5 by $T_1,T_3$}
     \end{table}
     \begin{table}[!hbp]
     	\centering
     	\begin{tabular}{lclclclclclclclclclclcl}
     		    &0       &1       &2       &3       &4      &5  
     	     	&6       &7       &8       &9           \\
     		\midrule
     		D3   &$\times$&$\times$&$\times$&$\times$&        &   
     		     &        &        &        &            \\
     		D5   &$\times$&$\times$&$\times$&$\times$&$\times$&   
     		      $\times$&$\times$&$\times$&        &   \\
     	\end{tabular}
     	\caption{ transform to D1-D5 by $T_1,T_2$}
     \end{table}
\end{document}