\documentclass[a4j,12pt]{jarticle}


% \usepackage{a4j}
\usepackage{fancyhdr}
\usepackage{lastpage}

% \usepackage{showkeys}
% \usepackage[dvips]{graphicx}

\usepackage{bm}

\usepackage{amsmath}
\usepackage{amsfonts}
\usepackage{amssymb}
\usepackage{slashed}
% \usepackage{mathbbol}  % 数字も白抜きにしてくれる
\usepackage{multirow}  % Tableで複数のセルにまたがるセルを作れる

\usepackage[a4paper,top=2.5cm,bottom=2.5cm,left=2.5cm,right=2.5cm,headsep=10pt]{geometry}
\usepackage[compact]{titlesec}
\titlespacing*{\section}{0pt}{3ex}{2ex}     % * を付けると続く文章が indent されない。
\titlespacing*{\subsection}{0pt}{2ex}{1ex}
\titlespacing*{\subsubsection}{0pt}{1ex}{1ex}
% \usepackage{graphicx}           % tikzで使う graphicxと競合するので排除
% \usepackage[usenames]{color}
% \usepackage[usenames,dvipdfmx]{color}    % optionは同時に指定出来る。
\usepackage[dvipsnames,dvipdfmx]{xcolor}    % optionは同時に指定出来る。


\usepackage[british]{babel}
\input{colordvi.tex}

\usepackage[dvipdfm,colorlinks,pagebackref,pdfusetitle,urlcolor=blue,citecolor=MidnightBlue,linkcolor=MidnightBlue,bookmarksnumbered,plainpages=false]{hyperref}



\input{dummy.tex}

%%% tikzセッティング %%%%%%%%%%%%%%%%%%%%%%%%%%%%%%%
\usepackage[dvipdfmx]{graphicx}
\usepackage{tikz}
\usetikzlibrary{arrows,shapes,patterns,snakes,calc}
\input{arrowsnew}
\usetikzlibrary{decorations.markings}
\usetikzlibrary{positioning}
%%% end of tikz %%%%%%%%%%%%%%%%%%%%%%%%%%%%%%%%%%%%

\usepackage[hang,bf,figurename=Fig.\ , tablename=Table\ ,margin=1cm]{caption}
\renewcommand{\captionfont}{\footnotesize}

%%% listingsセッティング %%%%%%%%%%%%%%%%%%%%%%%%%%%
\usepackage{listings, jlisting}
\renewcommand{\lstlistingname}{Code}
\definecolor{mygreen}{rgb}{0,0.6,0}
\definecolor{mygray}{rgb}{0.5,0.5,0.5}
\definecolor{mymauve}{rgb}{0.58,0,0.82}

\lstset{ %
  % language=Octave,                 % the language of the code
  backgroundcolor=\color{white},   % choose the background color; you must add \usepackage{color} or \usepackage{xcolor}
  basicstyle=\footnotesize,        % the size of the fonts that are used for the code
  breakatwhitespace=false,         % sets if automatic breaks should only happen at whitespace
  breaklines=true,                 % sets automatic line breaking
  captionpos=t,                    % sets the caption-position to bottom
  commentstyle=\color{mygreen},    % comment style
  deletekeywords={...},            % if you want to delete keywords from the given language
  escapeinside={\%*}{*)},          % if you want to add LaTeX within your code
  extendedchars=true,              % lets you use non-ASCII characters; for 8-bits encodings only, does not work with UTF-8
  frame=single,                    % adds a frame around the code
  keepspaces=true,                 % keeps spaces in text, useful for keeping indentation of code (possibly needs columns=flexible)
  keywordstyle=\color{blue},       % keyword style
  morekeywords={*,...},            % if you want to add more keywords to the set
  numbers=left,                    % where to put the line-numbers; possible values are (none, left, right)
  numbersep=5pt,                   % how far the line-numbers are from the code
  numberstyle=\tiny\color{mygray}, % the style that is used for the line-numbers
  rulecolor=\color{black},         % if not set, the frame-color may be changed on line-breaks within not-black text (e.g. comments (green here))
  showspaces=false,                % show spaces everywhere adding particular underscores; it overrides 'showstringspaces'
  showstringspaces=false,          % underline spaces within strings only
  showtabs=false,                  % show tabs within strings adding particular underscores
  stepnumber=1,                    % the step between two line-numbers. If it's 1, each line will be numbered
  stringstyle=\color{mymauve},     % string literal style
  tabsize=2,                       % sets default tabsize to 2 spaces
  title=\lstname                   % show the filename of files included with \lstinputlisting; also try caption instead of title
}
%%% ned of listings セッティング %%%%%%%%%%%%%%%%%%%%%%%%%%%


\pdfstringdefDisableCommands{%
    \renewcommand*{\bm}[1]{#1}%
    % any other necessary redefinitions 
}

%%%今村セッテッティング%%%%%%%%%%%%%%%%%%%%%%%%%%%%%
\newcommand{\CC}{\mathbb{C}}
\newcommand{\ZZ}{\mathbb{Z}}
\newcommand{\RR}{\mathbb{R}}
\newcommand{\HH}{\mathbb{H}}

\newcommand{\hf}{\frac{1}{2}}
\newcommand{\tr}{{\rm tr}}
\newcommand{\ind}{{\rm ind}}
\newcommand{\ol}{\overline}
\newcommand{\ul}{\underline}
\newcommand{\up}{\uparrow}
\newcommand{\dn}{\downarrow}
\newcommand{\wt}{\widetilde}
\newcommand{\ra}{\rightarrow}
\newcommand{\wh}{\widehat}


%%%横山セッティング%%%%%%%%%%%%%%%%%%%%%%%%%%%%%%%%%
\newcommand{\NN}{\mathcal{N}\!}
\newcommand{\DD}{\mathcal{D}}
\newcommand{\UU}{U(1)}
\newcommand{\dd}{\mathrm{d}}
\renewcommand{\SS}{\mathbf{S}}
\renewcommand{\Im}{\mathrm{Im}}
\renewcommand{\Re}{\mathrm{Re}}
\renewcommand{\<}{\langle}
\renewcommand{\>}{\rangle}
\newcommand{\Tr}{{\rm Tr}}

\renewcommand{\r}{\mathrm}

\newcommand{\sign}{\mathrm{sign}}

\newcommand{\stimes}{\hspace{-1pt}\times\hspace{-1pt}}

\newcommand{\lra}{\leftrightarrow}
\newcommand{\LL}{\mathcal{L}}
\newcommand{\la}{\leftarrow}
\newcommand{\ro}{\sqrt}
\newcommand{\Ra}{\Rightarrow}
\newcommand{\Pexp}{\mathrm{Pexp}}

\newcommand{\nn}{\nonumber \\}
\newcommand{\1}{\mbox{1}\hspace{-0.25em}\mbox{l}}

%数字のみ対応
\newcommand{\Maru}[1]{\ooalign{
\ifnum#1<10 \hfil\resizebox{.9\width}{.85\height}{#1}\hfil
\else
\hfil\resizebox{.6\width}{.8\height}{#1}\hfil
\fi
\crcr
\raise.1ex\hbox{$\bigcirc$}}}

%全文字対応
\newcommand{\maru}[1]{\ooalign{
\hfil\resizebox{.8\width}{\height}{#1}\hfil
\crcr
\raise.1ex\hbox{\large$\bigcirc$}}}


\newcommand{\nord}[1]{\vcentcolon\mathrel{#1}\vcentcolon}
\providecommand{\vcentcolon}{\mathrel{\mathop{:}}}


\def\P{\mathop{\cal P}}
\def\diag{\mathop{\rm diag}}


\def\Re{\mathop{\rm Re}\nolimits}
\def\Im{\mathop{\rm Im}\nolimits}
\def\Det{\mathop{\rm Det}\nolimits}
\def\sign{\mathop{\rm sign}\nolimits}


%%% rap %%% - make two letters overlap
\newcommand{\rap}[2]
{\setbox1=\hbox{#1}%
\setbox2=\hbox to\wd1{\hss #2\hss}%
\mbox{\rlap{\box1}\box2}}

%\newcommand{\sla}[1]{\rap{$#1$}{/}}
\newcommand{\sla}[1]{\rap{$#1$}{$\backslash$}}


\def\DY#1{{\MyGreen [DY: #1]}}
\newcommand{\MyGreen}{\color [rgb]{0,0.7,0}}

\usepackage[vcentermath]{youngtab}
% \Yboxdim4pt
\newcommand{\Y}{\yng}
\newcommand{\Young}{}


%%%title def%%%%%%%%%%%%%%%%%%%%%%%%%%%%%%%%%%%%%%%%%%%%%%%

\makeatletter
\def\maketitle{
\noindent
{\Large \@title \par\vskip 2pt}
\noindent
{\large \@date \hspace{4pt} \@author}
%\\[-2pt]
%\noindent------------------------------------------------------------------------------------------
\par\vskip 1.5em
}

\author{横山 大輔}
\date{\today}
%%%本文%%%%%%%%%%%%%%%%%%%%%%%%%%%%%%%%%%%%%%%%%%%%%%%%%%%%%%%%

% \title{\centerline{Lecture 2}}
\begin{document}
% \maketitle
% \begin{abstract}

% \end{abstract}
% \tableofcontents

  \pagestyle{fancy}
  \renewcommand{\headrulewidth}{0.0pt}
  \rhead{}
  \lhead{}
  \cfoot{[\ \scshape\oldstylenums{\thepage}\ / %
    \scshape\oldstylenums{\pageref{lastpage}} ]}
%  \rfoot{\@author}

% \setcounter{section}{}
% \setcounter{subsection}{}
% \setcounter{subsubsection}{}

\centerline{\Large \bf  Lecture 14}

% \vspace{12pt}
% \DY{なにかコメント}

% \vspace{-12pt}

% \vspace{-4pt}
% \begin{itemize}
%  \setlength{\itemsep}{0pt}
%  \item Type I superstring theory.
%  \item In type I theory, we only have D$1$-, D$5$-, and D$9$-, branes.
%  \item O$9^-$-plane is needed for type I to be consistent theory.
%  \item T-duality of type I theory.
% \end{itemize}
% \vspace{-4pt}



\section{D-brane dynamics}


So far we treated D-branes as a static solid object that is associated to boundary conditions.
On the other hand, it actually has dynamics as like a fundamental string.
We will learn the dynamics mainly through their action.
Therefore, the first half is devoted to derive the action for D$p$-branes.
Later half is devoted for connection between D-branes, F1-string, and NS5-brane.


\subsection{D-brane action}

We assume D-branes action is something similar to Nambu-Goto action.
Note that when we quantize the string action we used a string sigma action instead of Nambu-Goto action,
so that we can evade complexity of the square root.
There, the dimension of the world-sheet to be $2$ was crucial for quantization.
Therefore, we cannot follow the quantization procedure for the fundamental string to quantize D$p$-brane, in general.
Namely, the action we will learn here is an effective action of D-branes.


What we expect for the action is that
\vspace{-4pt}
\begin{itemize}
 \setlength{\itemsep}{0pt}
 \item it contains scalars that is a map from the world-sheet to space-time,
 \item it also contains vectors living on the D-branes, which arose from an open string massless spectrum,
 \item it involves B-field because open strings end on D-branes,
 \item it has supersymmetry (though we only talk about bosonic part in this lecture).
\end{itemize}
\vspace{-4pt}
The proposed effective action is called \textbf{Dirac-Born-Infeld(DBI) action}.
There are several approach to the action.
We assume Nambu-Goto action for D-branes and generalize it by utilizing T-duality.
Hence, \textit{the action is T-duality manifest}.

We will describe a world-sheet of D$p$-branes by $\sigma^a (a=0,1,\cdots,p)$.
$X^\mu(\sigma^a)$ are the scalars.
% Gauge field and its field strength are described by $A_\mu$ and $F_{\mu\nu}$, respectively.
Then, the Nambu-Goto action is
\begin{align}
 S_\textrm{D$p$} &= -T_\textrm{D$p$} \int d^{p+1}\sigma
 \sqrt{-\det \left( G_{\mu\nu} \frac{\partial X^\mu}{\partial \sigma^a} \frac{\partial X^\nu}{\partial \sigma^b} \right)} \ ,
\end{align}
where $T_\textrm{D$p$}$ is a D$p$-brane tension, which is discussed later.
$x^\mu$, which will appear later, are used for space-time coordinate.



Let us consider a simple set up (see Fig.~\ref{D1D2.eps}).
\begin{figure}[htb]
\centerline{\includegraphics[width=250pt]{D1D2.eps}}
\caption{D$1$-brane and its T-dual D$2$-brane on $\RR \stimes S^1$.}
\label{D1D2.eps}
\end{figure}
The space-time is $\RR_t \stimes \RR \stimes S^1$
and the metric is $ds^2 = \eta_{\mu\nu} dx^\mu dx^\nu$ (i.e. $G_{\mu\nu} = \eta_{\mu\nu}$).
D$1$-brane locates at $X_2$ ($\sim X_2 +2\pi R$),
and D$2$-brane has Wilson line $A_2$ ($\sim A_2 +\frac{1}{\wt R}$).
Note that here $A_2$ is not a 2-form but $A_{\mu=2}$.
T-duality relates these quantities (see Lecture note 9):
\begin{align}
 X_2 = 2\pi \alpha' A_2 \ .
\end{align}
Now consider vibrating D$1$-brane $X^2 = X^2(X^1)$ (see Fig.~\ref{vibD1.eps}).
\begin{figure}[htb]
\centerline{\includegraphics[width=120pt]{vibD1.eps}}
\caption{Vibrating D$1$-brane.}
\label{vibD1.eps}
\end{figure}
It maps to a field strength $F_{12} = \partial_1 X_2(X^1) \neq 0$ on D2-brane.
Suppose the vibrating D$1$-brane is described by the Nambu-Goto action
\begin{align}
 S_\textrm{D1} &= -T_\textrm{D1} \int d^2\sigma \sqrt{-\det \left( G_{\mu\nu} \frac{\partial X^\mu}{\partial \sigma^a}
 \frac{\partial X^\nu}{\partial \sigma^b} \right)} \quad\textrm{with}\quad
 X^0 = \sigma^0 \ , \quad X^1 = \sigma^1 \ , \nn
 &= -T_\textrm{D1} \int d\sigma^0 d\sigma^1 \sqrt{1 +\left(\frac{\partial X^2}{\partial \sigma^1} \right)^2} \ .
\end{align}
This expression seems to coincide with
\begin{align}
 S_\textrm{D2} &= -T_\textrm{D2} \int d^3\sigma \sqrt{-\det \left( G_{\mu\nu} \frac{\partial X^\mu}{\partial \sigma^a}
 \frac{\partial X^\nu}{\partial \sigma^b} +2\pi \alpha' F_{ab} \right)}  \nn
 &= -T_\textrm{D2} \cdot 2\pi \wt R\cdot \int d\sigma^0 d\sigma^1 \sqrt{1 +\left(2\pi\alpha'F_{12} \right)^2} \ .
\end{align}


\subsubsection*{D-brane tension}

From the dimension analysis D-brane tension should be the following form.
\begin{align}
 T_{\mathrm Dp} \sim \frac{\mathrm{mass}}{p\textrm{-dim vol}} \quad\Rightarrow\quad
 T_{\mathrm Dp} \sim \frac{1}{l_s^{p+1}} \ .
\end{align}
From the argument above in order for the two action to coincide
we need $T_\textrm{D1} = 2\pi \wt R T_\textrm{D2}$.
On the other hand, we do not want $T_{\textrm{D}p}$ to depend on $R$ because
$T_{\textrm{D}p}$ should be independent of space-time geometry.
Note that D-brane effective theory is supposed to reproduce open string amplitude,
whose leading contribution is the disk amplitude $\sim e^{-\langle \Phi \rangle}$. % \DY{homework ? see Lec. 4}
Thus, we reach the following form
\begin{align}
 S_{\textrm{D}p} &= -T_{\textrm{D}p} \int d^{p+1}\sigma\, e^{-\Phi(X)} \sqrt{-\det \left( G_{\mu\nu}
 \partial_a X^\mu \partial_b X^\nu +2\pi \alpha' F_{ab} \right)} \ .
\end{align}
The ratio of effective tensions of D1 and D2 branes is
\begin{align}
 \frac{T_{\textrm{D}1}^\mathrm{eff}}{T_{\textrm{D}2}^\mathrm{eff}}
 = \frac{T_{\textrm{D}1} e^{-\Phi}}{T_{\textrm{D}2} e^{-\wt\Phi}}
 =  \frac{T_{\textrm{D}1}}{T_{\textrm{D}2}} \cdot \frac{\wt R}{l_s} = 2\pi \wt R
 \quad\Rightarrow\quad T_{\textrm{D}1} = 2\pi l_s \cdot T_{\textrm{D}2} \ .
\end{align}
Note that the dilation field transforms under T-duality as $e^{-\wt\Phi} = e^{-\Phi} \frac{\wt R}{l_s}$ (see Homework 11 Prob. 4).
For D-branes in superstring theory the correct normalization is
$T_{\textrm{D}p} = \frac{2\pi}{(2\pi l_s)^{p+1}}$. % \DY{Homework ? heavy calculation}



\subsubsection*{RR charge and its normalization}

We heavily used the fact that D$p$-branes couples to $C_{p+1}$ in previous lectures.
Let us consider a concrete coupling.
It should be the following form.
\begin{align}
 S_{\textrm{D}p} = \cdots + q_{\textrm{D}p} \cdot \int d^{p+1}\sigma\, e^{-\Phi}
 C_{\mu_1\cdots\mu_{p+1}}(X) \frac{\partial X^{\mu_1}}{\partial \sigma^1 } \cdots
 \frac{\partial X^{\mu_{p+1}}}{\partial \sigma^{p+1} }
 = \cdots +q_{\textrm{D}p} \cdot \int_{\textrm{D}p} e^{-\Phi} C_{(p+1)} \ .
\end{align}
By considering T-duality we can conclude that
$q_{\textrm{D}p} = \frac{2\pi}{(2\pi l_s)^{p+1}} = T_{\textrm{D}p}$ (up to a constant). % \DY{Homework ?}
Note that the equality of the charge and the tension is crucial for multiple D$p$-branes to co-exist statically.
This is because the tension induce gravitational force (graviton \& dilaton) between D$p$-branes, which is attractive,
on the other hand, the RR charge induce repulsive force for positively(negatively) charged objects, which are D$p$-branes.


Note that we have to use proper normalization for RR-fields.
The convention used above is called \textbf{string normalization},
and the one used in SUGRA is called \textbf{canonical normalization}.
Let us recall the IIA SUGRA. Appropriate part is, for example,
\begin{align}
 S_{A,R} = \frac{1}{2\kappa_{10}^2} \int d^{10}x \sqrt{-G} \left[ -\frac{1}{2} G_{(2)}^2 \right]
 +q_{\textrm{D}0} \cdot \int_{\textrm{D}0} C_{(1)} \ .
\end{align}
The same expression in the string normalization is
\begin{align}
 S_{A,R} = \frac{1}{2\kappa_{10}^2} \int d^{10}x \sqrt{-G} e^{-2\Phi} \left[ -\frac{1}{2} G_{(2)}^2 \right]
 +q_{\textrm{D}0} \cdot \int_{\textrm{D}0} e^{-\Phi} C_{(1)} \ .
\end{align}
The Dirac quantization conditions(for the details see homework 11 Prob. 2.2) for these two expressions are the same:
\begin{align}
 2\kappa_{10}^2 q_e q_m \in 2\pi \ZZ \ .
\end{align}
($q_e = q_{\mathrm{D}0}$ and $q_m = q_{\mathrm{D}(6)}$ for the above case.)
However, we also define $q^\mathrm{eff} = q e^{-\Phi} = q g_s$ in the string normalization,
then, the quantization condition is
\begin{align}
 2\kappa_{10}^2 q_e^\mathrm{eff} q_m^\mathrm{eff} g_s^2 \in 2\pi \ZZ \ .
\end{align}
In this convention we always assume that $\Phi$ is non-dynamical.
Confirm that $q_{\textrm{D}p}^\mathrm{eff} = T_{\textrm{D}p}^\mathrm{eff} = \frac{2\pi}{(2\pi l_s)^{p+1}g_s}$
satisfies the quantization condition.
Note that some references define $q e^{-\Phi}$ as $q$ (similarly $T_{\textrm{D}p}e^{-\Phi}$ as $T_{\textrm{D}p}$).
The reason we used canonical normalization is simply that the expression is much simpler (especially kinetic terms).


For the B-field, the normalization is different and the quantization condition is
\begin{align}
 T_\mathrm{F1} \cdot T_\mathrm{NS5} \cdot 2\kappa_{10}^2 g_s^2 \in 2\pi \ZZ \ .
\end{align}
Since $T_\mathrm{F1} = \frac{2\pi}{(2\pi l_s)^2}$,
\begin{align}
 T_\mathrm{NS5} = \frac{2\pi}{T_\mathrm{F1} \cdot 2\kappa_{10}^2 g_s^2}
 = \frac{2\pi}{(2\pi l_s)^6 g_s^2} \ .
\end{align}



% \textbf{Dirac quantization condition}


% % Let us consider D$p$-brane located at $\bm x_{\perp} = (x_{p+1},\cdots,x_9) = \bm 0$.
% % Then, the RR-coupling becomes
% % \begin{align}
% %  \mu_{\textrm{D}p} \cdot \int_{\bm x_{\perp} = \bm 0} C_{p+1}
% %  = \mu_{\textrm{D}p} \cdot \int_{\bm x_{\perp} = \bm 0} C_{p+1} \wedge \delta(\bm x_{\perp})
% %  dx^{p+1} \cdots dx^9 \ .
% % \end{align}
% % With the kinetic term the action is
% % \begin{align}
% %  S = -\frac{1}{4\kappa_{10}^2} \int F_{(p+2)} \wedge * F_{(p+2)}
% %  +\mu_{\textrm{D}p} \int C_{p+1} \wedge \delta(\bm x_{\perp})
% %  dx^{p+1} \cdots dx^9 \ ,
% % \end{align}
% % and the E.O.M is
% % \begin{align}
% %  d*F_{(p+2)} = 2\kappa_{10}^2 \mu_{\textrm{D}p} \delta(\bm x_{\perp})
% %  dx^{p+1} \cdots dx^9 \ .
% % \end{align}
% % Integration over ball that perpendicular to the world-line of the D$p$-brane leads
% % \begin{align}
% %  2\kappa_{10}^2 \mu_{\textrm{D}p} = 
% % \end{align}


% Let us consider a $(p+1)$-form RR field.
% The electrically coupled object is D$p$-brane and the magnetic one is D$(6-p)$-brane.
% The magnetic charge of the D$(6-p)$-brane is
% \begin{align}
%  \mu_{6-p} = \frac{1}{2\kappa_{10}^2} \int_{B^{p+3}} d F_{(p+2)} =
%  \frac{1}{2\kappa_{10}^2} \int_{\partial B^{p+3}} F_{(p+2)} \ .
% \end{align}
% Now consider D$p$-brane moving around D$(6-p)$-brane.
% \begin{align}
%  S \sim \mu_{p} \int_M C_{(p+1)} = \mu_{p} \int_{\partial^{-1}M} F_{(p+2)} \ ,
% \end{align}
% where $\partial^{-1}M$ is manifolds whose boundary is $M$.
% The choice of $\partial^{-1}M$ is arbitrary, and does not affect the result.
% Consider $\partial^{-1}M_N -\partial^{-1}M_S$ surround the D$(6-p)$-brane.
% \begin{align}
%  \mu_{p} \int_{\partial^{-1}M_N} F_{(p+2)} -\mu_{p} \int_{\partial^{-1}M_S} F_{(p+2)}
%  = \mu_{p} \mu_{6-p} 2\kappa_{10}^2 \in 2\pi \ZZ \ .
% \end{align}
% The values $\mu_{p} = \frac{2\pi}{(2\pi l_s)^{p+1}}$,
% $2\kappa_{10}^2 = \frac{(2\pi l_s)^{8}}{2\pi}$ satisfy the quantization condition.



% Assume the condition works also for F$1$-string and NS$5$-brane,
% \begin{align}
%  T_\mathrm{F1} \cdot T_\mathrm{NS5} \cdot 2\kappa_{10}^2 g_s^2 = 2\pi \ ,
% \end{align}
% where we are in the string frame.
% Since $T_\mathrm{F1} = \frac{2\pi}{(2\pi l_s)^2}$,
% \begin{align}
%  T_\mathrm{NS5} = \frac{2\pi}{T_\mathrm{F1} \cdot 2\kappa_{10}^2 g_s^2}
%  = \frac{2\pi}{(2\pi l_s)^6 g_s^2} \ .
% \end{align}




\subsubsection*{Generalization of RR coupling and the DBI action}

Let us again consider vibrating D$1$-brane with the RR-coupling.
T-duality connects the following two expression.
\begin{align}
 &S_{\textrm{D}1} =
 \cdots +q_{\textrm{D}1} \cdot \int dx^0 dx^1 e^{-\Phi}
 \left( C_{01} +C_{02} \frac{\partial X^2}{\partial \sigma^1}\right) \ , \\
 &S_{\textrm{D}2} =
 \cdots +q_{\textrm{D}2} \cdot \int dx^0 dx^1 dx^2 e^{-\wt \Phi}
 \left( \wt C_{012} +\wt C_{0} \cdot 2\pi\alpha' F_{12} \right) \ ,
\end{align}
where $C_{01} \lra \wt C_{012}$, $C_{02} \lra \wt C_{0}$, and $X^2 \lra 2\pi\alpha' A_2$.
This can be understood as follows (see Fig.~\ref{D0D2.eps}).
\begin{figure}[htb]
\centerline{\includegraphics[width=250pt]{D0D2.eps}}
\caption{D$0$-D$2$ bound state.}
\label{D0D2.eps}
\end{figure}
The vibrating D$1$-brane consists of straight D$1$-brane along $x^1$
and local vibration along $x^2$.
After T-duality along $x^2$ the vibration part becomes D$0$-brane and gives
$F_{12}\neq 0$.
Generalization of the RR-coupling is
\begin{align}
 S_{\textrm{D}p} =
 \cdots +q_{\textrm{D}p} \cdot \int C_\mathrm{RR} \wedge \exp(2\pi\alpha' F_{(2)}) \ ,
\end{align}
where $C_{RR} = \sum_{n} C_{(n)}$.


Finally, full general form of D$p$-brane action is given as follows.
\begin{align}
 S_{\textrm{D}p} &= -T_{\textrm{D}p} \int d^{p+1}\sigma\, e^{-\Phi(X)} \sqrt{-\det \left( G_{ab}
 +2\pi \alpha' F_{ab} -B_{ab} \right)} \nn
 &\qquad +q_{\textrm{D}p} \cdot \int C_{RR} \wedge \exp(2\pi\alpha' F_{(2)}-B_{(2)}) \ ,
\end{align}
where $G_{ab} = G_{\mu\nu} \partial_a X^\mu \partial_b X^\nu$ and
$B_{ab} = B_{\mu\nu} \partial_a X^\mu \partial_b X^\nu$.
The first line is called \textbf{DBI action}.
Note that we used canonical normalization for RR-fields.
B-field should appear with $F_{(2)}$ due to the gauge invariance.


Let us consider a fundamental string action that is coupled to D$p$-branes.
\begin{align}
 S &= -\frac{1}{4\pi \alpha'} \int d^2\sigma \sqrt{-h} h^{ab} \partial_a X^\mu \partial_b X^\nu G_{\mu\nu}
 +\epsilon^{ab} \partial_a X^\mu \partial_b X^\nu B_{\mu\nu} +\cdots \nn
 &\qquad +\int_{\partial\Sigma} d\sigma^0 \partial_0 X^\mu A_\mu  \nn
 &= \cdots -\frac{1}{2\pi\alpha'} \int_{\Sigma} B_{(2)} +\int_{\partial\Sigma} A_{(1)} \ .
\end{align}
Important point here is that the gauge transformation of B-field $\delta_B B_{(2)} = d \lambda_{(1)}$ in the action is NOT invariant
if there are boundaries, which is exactly the situation we consider now:
\begin{align}
 \delta_B \int_{\Sigma} B_{(2)} = \int_{\Sigma} d \lambda_{(1)} = \int_{\partial\Sigma} \lambda_{(1)} \neq 0 \ .
\end{align}
This is compensated if $A_{(1)}$ transform as follows:
\begin{align}
 \delta_B A_{(1)} = \frac{\lambda_{(1)}}{2\pi\alpha'} \ .
\end{align}



Though we focused on the bosonic part so far, there is a fermionic part
so that they form space-time supersymmetry.
Here we only write down the leading fluctuation:
\begin{align}
 -i\int d^{p+1}\sigma\, \tr \left( \ol\psi \Gamma^a D_a \psi \right) \ .
\end{align}
For the full nonlinear supersymmetric form one should consult with, for example \cite{Tseytlin:1999dj}.


\subsection{Branes, Strings ending on Branes}

We will look into the RR coupling further from a different view point.
Maxwell equation leads charge conservation law as follows.
\begin{align}
 \begin{array}{l}
  d *\!F_{(2)} = J_e \\
  d F_{(2)} = J_m
 \end{array} \qquad\Rightarrow\qquad
 \begin{array}{l}
  d J_e = 0 \\
  d J_m = 0
 \end{array} \ .
\end{align}
Charge conservation assure that a world-line of the charged particle does not end (closed path or infinitely long).
If we apply this logic to branes we may find the same result for branes.
However, charge conservation for SUGRA is quite non-trivial due to the non-linearity of the E.O.Ms.
We will see the case of generalized type IIA SUGRA.


\subsubsection*{Massive IIA SUGRA}

When we saw the IIA SUGRA action
you may wonder why there is no RR-field corresponding to D$8$-brane,
which is $9$-form and its field strength is $10$-form $G_{(10)}$.
Since it is non-dynamical ($d *\! G_{(10)} = 0$ leads $*G_{(10)} = G_{(0)} \equiv m$) it has constant contributions
to the action, called \textbf{massive IIA SUGRA}.
$m$ is called \textbf{Romans mass} because it is a constant and partly contributes as a mass term in the action.


Let us see the massive IIA SUGRA action (we omit wedge product $\wedge$ in this lecture):
\begin{align}
 &S_\mathrm{A,NS} = \frac{1}{2\kappa_{10}^2} \int d^{10}x \sqrt{ -G} e^{-2\Phi} \left[
 R +4 \partial_\mu \Phi \partial^\mu \Phi -\frac{1}{2} H_{(3)}^{2} \right] \ ,  \\
 &S_\mathrm{A,R} = \frac{1}{2\kappa_{10}^2} \int d^{10}x \sqrt{ -G} \left[
 -\frac{1}{2} m^{2} -\frac{1}{2} G_{(2)}^{2} -\frac{1}{2} G_{(4)}^{2} \right] \ ,  \\
 &S_\mathrm{A,CS} = \frac{1}{2\kappa_{10}^2} \int \left[
 -\frac{1}{2} B_{(2)} G_{(4)} G_{(4)}
 +\frac{1}{2} B_{(2)}^2 G_{(2)} G_{(4)}
 -\frac{1}{6} B_{(2)}^3 G_{(2)}^2 \right. \nn
 &\hspace{100pt} \left.
 -\frac{m}{6} B_{(2)}^3 G_{(4)}
 +\frac{m}{8} B_{(2)}^4 G_{(2)}
 -\frac{m^2}{40} B_{(2)}^5
 \right] \ ,
\end{align}
where
\begin{align}
 \begin{array}{l}
  H_{(3)} = dB_{(2)} \ , \\
  G_{(2)} = dC_{(1)} + m B_{(2)} \ , \\
  % &\wt G_{(4)} = G_{(4)} +C_{(1)} \wedge H_{(3)} \ , \\
  % &G_{(4)} =dC_{(3)} +\frac{1}{2} m B_{(2)} \wedge B_{(2)} \ .
  G_{(4)} =dC_{(3)} +dC_{(1)} B_{(2)} +\frac{1}{2} m B_{(2)}^2 \ .
 \end{array} \label{eq:fsGf}
\end{align}


% \subsection{Branes that have boundaries}

\subsubsection*{RR- field}

From the massive IIA action we have following equation of motion for $C_{(1)}$ and $C_{(3)}$:
E.O.M
\begin{align}
 -d*\!G_{(2)} &= H_{(3)} *\!G_{(4)} \ , \\
 d*\!G_{(4)} &= H_{(3)} G_{(4)} \ .
\end{align}
Since we know the relation between the field strengths and their gauge fields (\ref{eq:fsGf})
we have following Bianchi identities:
\begin{align}
 d G_{(2)} &= m H_{(3)} \ , \\
 d G_{(4)} &= H_{(3)} G_{(2)} \ .
\end{align}
Now we relabel $m$ by $G_{(0)}$ and define the dual field strengths:
\begin{align}
 G_{(10)} = * G_{(0)} \ , \qquad
 G_{(8)} = -*\! G_{(2)} \ , \qquad
 G_{(6)} = * G_{(4)} \ .
\end{align}
Then, the E.O.M and Bianchi ids are re-written as
\begin{align}
 d G_{(2n)} = G_{(2n-2)} H_{(3)} \ .
 \label{eq:eom1}
\end{align}
Let us define following formal sum of RR-fields
\begin{align}
 G_\mathrm{even} = G_{(0)} +G_{(2)} +G_{(4)} +G_{(6)} +G_{(8)} +G_{(10)} \ .
\end{align}
Using the formal sum we can express (\ref{eq:eom1}) by single expression
\begin{align}
 d G_\mathrm{even} =  H_{(3)} G_\mathrm{even} \ .
\end{align}
This equation can be solved as follows.
\begin{align}
 &G_\mathrm{even} = e^{B_{(2)}} \left( m +dC_\mathrm{odd} \right) \ , \\
 &C_\mathrm{odd} = C_{(1)} +C_{(3)} +C_{(5)} +C_{(7)} +C_{(9)} \ .
\end{align}



\subsubsection*{B-field}

Field strength of the B-field is defined by
\begin{align}
 H_{(3)} = d  B_{(2)} \ ,
\end{align}
hence, the Bianchi id is $d  H_{(3)} = 0$.
E.O.M is given as follows.
\begin{align}
 d \left(e^{-2\Phi}*\!  H_{(3)} \right) = m *\!G_{(2)} +*G_{(4)}G_{(2)} -\frac{1}{2} G_{(4)}^2 \ .
\end{align}
If we define the dual field strength $H_{(7)} = e^{-2\Phi}*\!  H_{(3)}$,
then, the E.O.M becomes
\begin{align}
 dH_{(7)} = -\frac{1}{2} \left[ (\mathcal T G_\mathrm{even}) G_\mathrm{even} \right]_{(8)}
\end{align}
where
\begin{align}
 \mathcal T (dx^{i_1} \cdots dx^{i_n}) = (dx^{i_n} \cdots dx^{i_1}) \ ,
\end{align}
which is a ``transpose'' of differential forms.


Note that those field strengths are invariant under the gauge transformations of B-field as well as RR-fields:
\begin{align}
 &\delta_B B_{(2)} = d \lambda_{(1)} \ , \qquad \delta_B C_\mathrm{odd} = -\lambda_{(1)} \left(m+dC_\mathrm{odd}\right) \ , \\
 &\delta_C B_{(2)} = 0 \ , \qquad \delta_C C_\mathrm{odd} = d \lambda_\mathrm{even} \ ,
\end{align}
where we introduced a formal sum of gauge parameters
\begin{align}
 \lambda_\mathrm{even} = \lambda_{(0)} +\lambda_{(2)} +\lambda_{(4)} +\lambda_{(6)} +\lambda_{(8)} \ .
\end{align}

\subsubsection*{Brane currents}

Let us introduce brane currents $J_{(8)}^\mathrm{F1}$, $J_{(4)}^\mathrm{NS5}$, and
\begin{align}
 J_\mathrm{odd} = J_{(1)}^\mathrm{D8} +J_{(3)}^\mathrm{D6} +J_{(5)}^\mathrm{D4} +J_{(7)}^\mathrm{D2} +J_{(9)}^\mathrm{D0} \ ,
\end{align}
and add to the E.O.Ms:
\begin{align}
 d H_{(3)} &= J_{(4)}^\mathrm{NS5} \ , \\
 d H_{(7)} &= J_{(8)}^\mathrm{F1} -\frac{1}{2} \mathcal (T G_\mathrm{even}) G_\mathrm{even} \ , \\
 d G_\mathrm{even} &= J_\mathrm{odd} +H_{(3)} G_\mathrm{even} \ .
\end{align}
From these equations we can derive following ``charge conservation'' law:
\begin{align}
 d J_{(4)}^\mathrm{NS5} &= 0 \ , \\
 d J_{(8)}^\mathrm{F1} &= \left[ J_\mathrm{odd} (\mathcal T G_\mathrm{even}) \right]_{(9)} \ , \\
 d J_\mathrm{odd} &= -J_{(4)}^\mathrm{NS5} G_\mathrm{even} -J_\mathrm{odd} H_{(3)} \ .
\end{align}
From the laws we can deduce several facts (see Table~\ref{table:001}):
\vspace{-4pt}
\begin{itemize}
 \setlength{\itemsep}{0pt}
 \item NS$5$-brane cannot have boundaries,
 \item F$1$ string can end on any D-branes,
 \item D$p$-brane can end on NS$5$-brane up to $p=6$ (D$8$ cannot),
 \item D$p$-brane can end on D$(p+2)$-brane.
\end{itemize}
\vspace{-4pt}
\begin{table}[htbp]
 \begin{center}
  \caption{Branes on which brane ends and branes that end on.}
  \vspace{4pt}
  \label{table:001}
\begin{tabular}{c|c}
 Brane & Branes end on \\\hline
 F1 & nothing \\
 NS5-brane & D0, D2, D4, D6 \\
 D0-brane & F1 \\
 D2-brane & F1, D0 \\
 D4-brane & F1, D2 \\
 D6-brane & F1, D4 \\
 D8-brane & F1, D6
\end{tabular}
\end{center}
\end{table}

The fact that D-brane is coupled to RR-field requires that
the brane action should include $S = \int C_\mathrm{odd}$.
However, it is invariant under the gauge transformations.
The invariant form is
\begin{align}
 S = \int \left( e^{2\pi\alpha' F_{(2)} -B_{(2)}} C_\mathrm{odd} +m\omega\right) \ ,
\end{align}
where
\begin{align}
 \omega = \sum_{n} \frac{1}{(n+1)!} A_{(1)} F_{(2)}^n \ .
\end{align}
This is consistent with the previous analysis.


\subsection{Bound states of D-branes}

As we saw D$p$-brane action has following term
\begin{align}
 S \sim \int e^{2\pi\alpha' F_{(2)}} C_\mathrm{RR} = \int \delta_{D-p-1} (\mathrm Dp) e^{2\pi\alpha' F_{(2)}} C_\mathrm{RR} \ ,
\end{align}
where we set $m=0=B_{(2)}$.
The fact that the action includes not only $C_{(p+1)}$ but also $C_{(p+1-2n)}$ means that
D$p$-brane can have D$(p-2n)$-brane charges for $n \in \ZZ_{+}$.

\subsubsection*{D0-D2 bound state}

Let us consider a concrete example of D$2$-brane case.
The action include following term
\begin{align}
 S \sim \frac{1}{2\pi} \int_{\RR_t \stimes \Sigma} \left( C_{(3)} + F_{(2)} C_{(1)}\right) \ ,
\end{align}
where $\Sigma$ is an image of world-sheet space (not time).
Note that the flux is quantized;
\begin{align}
 \int_{\Sigma} F_{(2)} = 2\pi n \qquad n \in \ZZ \ .
\end{align}


Now consider a process the $\Sigma$ shrinks to zero.
In this process $C_{(3)}$ part becomes zero as the volume becomes zero.
On the other hand, $C_{(1)}$ part remains finite because the flux $F_{(2)}$ is quantized and gives
\begin{align}
 S \sim \frac{1}{2\pi} \int_{\RR_t \stimes \Sigma} \left( F_{(2)} C_{(1)}\right) \to n \int_{\RR_t} C_{(1)} \ .
\end{align}
This is nothing but $n$ D$0$-branes.
Namely, when D$2$-brane with $n$ flux shrinks to a point, $n$ D$0$-branes remains.
The former state can be understood that it is a bound state of D$2$- and D$0$-branes (see Fig.~\ref{D0D2Bound.eps}).
\begin{figure}[htb]
\centerline{\includegraphics[width=300pt]{D0D2Bound.eps}}
\caption{Transition of D$0$-D$2$ bound state and D$0$ + D$2$.}
\label{D0D2Bound.eps}
\end{figure}


\subsubsection*{Myers effect}

Let us consider the opposite process of the previous argument.
When there are $n$ D$0$-branes they can become D$2$-brane.
This situation can be accelerated by inducing background $C_{(3)}$ flux.
If there is $C_{(3)}$ flux, then, being D$2$-brane is a lower energy state than being D$0$-branes (see Fig.~\ref{Myers.eps}).
\begin{figure}[htb]
\centerline{\includegraphics[width=200pt]{Myers.eps}}
\caption{Myers effect: transition from D$0$-branes to D$0$-D$2$ bound state.}
\label{Myers.eps}
\end{figure}
This is something similar to polarization phenomenon in electro-magnetism,
and in this case, it is called \textbf{Myers effect}.


\subsubsection*{Other bound states}

Left it for homework.



\subsubsection*{Hanany-Witten effect}

There is so called Hanany-Witten effect,
which is a brane creation/annihilation process when some branes cross each other.
Typical example is a cross of NS$5$ and D$5$ create/annihilate D$3$-brane (see Fig.~\ref{HW.eps}).
\begin{figure}[htb]
\centerline{\includegraphics[width=250pt]{HW.eps}}
\caption{Hanany-Witten effect: crossing of D$5$- and NS$5$-branes.}
\label{HW.eps}
\end{figure}
Another example is that D$p$ and D$p'$ for $p+p'=8$ create/annihilate F$1$-string.
Crossing of two M$5$-branes create/annihilate M$2$-brane etc (look it up in the web if you are interested in).



% \subsubsection*{Supersymmetric theories from branes}






% \begin{thebibliography}{CDLOGP91}

%  \bibitem[Pol98]{Pol98}
%                 J. Polchinski.
%                 String theory. Vol. 1.
%                 Cambridge University Press, 1998.

%  \bibitem[BP09]{Blumenhagen:2009zz}
%                R.~Blumenhagen and E.~Plauschinn.
%                \newblock {Introduction to conformal field theory}.
%                \newblock {\em Lect. Notes Phys.}, 779:1--256, 2009.

% \end{thebibliography}


% \bibliography{string-lecture}
% \bibliographystyle{halpha}
% \bibliographystyle{JHEP}

\begin{thebibliography}{CDLOGP91}

% %\cite{AlvarezGaume:1981hn}
% \bibitem[AFM81]{AlvarezGaume:1981hn}
%   L.~Alvarez-Gaume, D.~Z.~Freedman and S.~Mukhi,
%   ``The Background Field Method and the Ultraviolet Structure of the Supersymmetric Nonlinear Sigma Model,''
%   Annals Phys.\  {\bf 134}, 85 (1981).
%   %doi:10.1016/0003-4916(81)90006-3
%   %%CITATION = doi:10.1016/0003-4916(81)90006-3;%%
%   %525 citations counted in INSPIRE as of 09 Oct 2017



% \bibitem[CT88]{Callan:1988xx}
%  Curt Callan and Lárus Thorlacius.
%  \textit{SIGMA MODELS AND STRING THEORY.}
%  TASI Lecture, 1988.
%  (The link below is a direct link to the pdf file of 45MB )
%  \href{http://www.damtp.cam.ac.uk/user/tong/string/sigma.pdf}
%  {http://www.damtp.cam.ac.uk/user/tong/string/sigma.pdf}.


 %\cite{Tseytlin:1999dj}
\bibitem{Tseytlin:1999dj} 
  A.~A.~Tseytlin,
  ``Born-Infeld action, supersymmetry and string theory,''
  In *Shifman, M.A. (ed.): The many faces of the superworld* 417-452
  % doi:10.1142/9789812793850_0025
  [hep-th/9908105].
  %%CITATION = doi:10.1142/9789812793850_0025;%%
  %392 citations counted in INSPIRE as of 14 Dec 2017



\end{thebibliography}


\label{lastpage}

% \begin{tikzpicture}[>=stealth,scale=1]
%  \draw[->] (0,0)--(1,0);
%  \draw[latex-stealth] (0,0.5)--(1,0.5);
%  \draw[latex-stealthnew,arrowhead=2mm] (0,1)--(1,1);
% \end{tikzpicture}

% \begin{figure}[htb]
% \centerline{\includegraphics[width=250pt]{.eps}}
% \caption{}
% \label{.eps}
% \end{figure}



% \begin{table}[htbp]
%  \begin{center}
%   \caption{}
%   \vspace{4pt}
%   \label{table:001}
% \begin{tabular}{|c|c|c|c|c|}
% \hline
% \hline
%   Category & Sector & $(h_A,h_B,h_T)$ & Mirror theory & ABJM model \\
% \hline
%  1 & & \parbox{40pt}{$(0,0,0)$ $(1,1,1)$} & $1.11906$ & $1.13290$ \\
% \hline
%  2 & & \parbox{40pt}{$(0,0,1)$ $(1,1,0)$} & $-0.10861$ & $-0.10861$ \\
% \hline
%  \multirow{2}{*}{\vspace{-15pt}3} & 3-1 & \parbox{40pt}{$(0,1,0)$ $(1,0,1)$} & $0.176777$ &
%  \multirow{2}{*}{\vspace{-15pt}$0.176577$} \\
% \cline{2-4}
%  & 3-2 & \parbox{40pt}{$(1,0,0)$ $(0,1,1)$} & $0.176777$ & \\
% \hline
% \hline
% \end{tabular}
% \end{center}
% \end{table}


% \begin{thebibliography}{99}

% % \cite{Imamura:2012rq}
% \bibitem{Imamura:2012rq}
%   Y.~Imamura and D.~Yokoyama,
%   %``S^3/Z_n partition function and dualities,''
%   JHEP {\bf 1211}, 122 (2012)
%   [arXiv:1208.1404 [hep-th]].
%   %%CITATION = ARXIV:1208.1404;%%

 % \bibitem{fnorio:legendre}
 %         fnorio
 %         ``ルジャンドル変換とは何か''
 %         \url{http://fnorio.com/0146Legendre_transformation/Legendre_transformation.html}


 % \bibitem{EMAN:dynamics}
 %         EMAN物理学
 %         ``ハミルトニアン''
 %         \url{http://eman-physics.net/analytic/hamilton.html}


 % \bibitem{Wiki:legendre}
 %         Wiki
 %         ``ルジャンドル変換''
 %         \url{https://ja.wikipedia.org/wiki/ルジャンドル変換}

 % \bibitem{mathtrain:legendre}
 %         高校数学の美しい物語
 %         ``ルジャンドル変換の意味と具体例''
 %         \url{http://mathtrain.jp/legendrehenkan}

% \end{thebibliography}


% \bibliography{dd}

\end{document}
