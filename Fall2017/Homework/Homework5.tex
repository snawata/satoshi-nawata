\documentclass[12pt,a4paper]{article}
%\usepackage{hyperref} % Use the Charter font for the document text
%\usepackage[UTF8]{ctex}


\usepackage{amsfonts,amssymb,amsmath}
\usepackage{mathtools}
\usepackage{tikz-cd}
\usepackage{fullpage}
\usepackage{tikz}
\usepackage{alltt}
\usepackage{amsfonts}
\usepackage{amsmath}
\usepackage{amssymb}
\usepackage{amsthm}
\usepackage{booktabs}
\usepackage{caption}
\usepackage{enumitem}
\usepackage{fancyhdr}
\usepackage{graphicx}
% \usepackage{mathdots}
\usepackage{mathtools}
\usepackage{microtype}
\usepackage{multirow}
\usepackage{pdflscape}
\usepackage{pgfplots}
\usepackage{siunitx}
\usepackage{slashed}
\usepackage{tabularx}
\usepackage{tikz}
\usepackage{tkz-euclide}
% \usepackage[normalem]{ulem}
\usepackage[all]{xy}
\usepackage{imakeidx}

\usepackage{wrapfig}



%%%%%%%  Greek letters %%%%%%%%%%%%%%%%%%
\def\a{\alpha}
\def\b{\beta}
\def\c{\gamma} \def\g{\gamma}
\def\d{\delta}
\def\e{\epsilon}
\def\f{\phi}
\def\vf{\varphi}  \def\tvf{\tilde{\varphi}}
\def\vp{\varphi}
\def\h{\eta}
\def\i{\iota}
\def\j{\psi}
\def\k{\kappa}
\def\m{\mu}
\def\n{\nu}
\def\o{\omega}  \def\w{\omega}
\def\q{\theta}  \def\th{\theta}
\def\r{\rho}
\def\s{\sigma}
\def\t{\tau}
\def\u{\upsilon}
\def\x{\xi}
\def\z{\zeta}

\def\A{\Alpha}
\def\B{\Beta}
\def\G{\Gamma}
\def\D{\Delta}
\def\E{\Epsilon}
\def\F{Phi}
\def\h{\eta}
\def\I{\Iota}
\def\J{Psi}
\def\K{\Kappa}
\def\L{\Lambda}
\def\M{\Mu}
\def\N{\Nu}
\def\O{\Omega}  \def\w{\omega}
\def\Q{\Theta}  \def\Th{\Theta}
\def\R{\Rho}
\def\Si{\Sigma}
\def\T{\Tau}
\def\Up{\Upsilon}
\def\X{\Xi}
\def\Z{\Zeta}








%%%%%%%%%%%% math fonts %%%%%%%%%%%%%%%%%%%%%%%%%%%%%%%%%%%%%
%
%---------- mathbb font --------------------------------
%

\newcommand{\bA}{\ensuremath{\mathbb{A}}}
\newcommand{\bB}{\ensuremath{\mathbb{B}}}
\newcommand{\bC}{\ensuremath{\mathbb{C}}}
\newcommand{\bD}{\ensuremath{\mathbb{D}}}
\newcommand{\bE}{\ensuremath{\mathbb{E}}}
\newcommand{\bF}{\ensuremath{\mathbb{F}}}
\newcommand{\bG}{\ensuremath{\mathbb{G}}}
\newcommand{\bH}{\ensuremath{\mathbb{H}}}
\newcommand{\bI}{\ensuremath{\mathbb{I}}}
\newcommand{\bJ}{\ensuremath{\mathbb{J}}}
\newcommand{\bK}{\ensuremath{\mathbb{K}}}
\newcommand{\bL}{\ensuremath{\mathbb{L}}}
\newcommand{\bM}{\ensuremath{\mathbb{M}}}
\newcommand{\bN}{\ensuremath{\mathbb{N}}}
\newcommand{\bO}{\ensuremath{\mathbb{O}}}
\newcommand{\bP}{\ensuremath{\mathbb{P}}}
\newcommand{\bQ}{\ensuremath{\mathbb{Q}}}
\newcommand{\bR}{\ensuremath{\mathbb{R}}}
\newcommand{\bS}{\ensuremath{\mathbb{S}}}
\newcommand{\bT}{\ensuremath{\mathbb{T}}}
\newcommand{\bU}{\ensuremath{\mathbb{U}}}
\newcommand{\bV}{\ensuremath{\mathbb{V}}}
\newcommand{\bW}{\ensuremath{\mathbb{W}}}
\newcommand{\bX}{\ensuremath{\mathbb{X}}}
\newcommand{\bY}{\ensuremath{\mathbb{Y}}}
\newcommand{\bZ}{\ensuremath{\mathbb{Z}}}


%
%\parskip=1em
%\parindent=0.3in
%\setlength\oddsidemargin{0.5in} \setlength\evensidemargin{0.5in}
%\setlength\textwidth{5.5in}
%
%\hfuzz6pt % Don't bother to report over-full boxes if over-edge is < 6pt
%
%\newlength{\defbaselineskip}
%\setlength{\defbaselineskip}{\baselineskip}
%\newcommand{\setlinespacing}[1]%
%           {\setlength{\baselineskip}{#1 \defbaselineskip}}
%\newcommand{\doublespacing}{\setlength{\baselineskip}%
%                           {2.0 \defbaselineskip}}
%\newcommand{\singlespacing}{\setlength{\baselineskip}{\defbaselineskip}}
%
%\newcommand{\properpagestyle}{\pagestyle{myheadings}\markboth{}{}\markright{}}


%---------- mathscript font -----------------------------
%

\newcommand{\scA}{\ensuremath{\mathscr{A}}}
\newcommand{\scB}{\ensuremath{\mathscr{B}}}
\newcommand{\scC}{\ensuremath{\mathscr{C}}}
\newcommand{\scD}{\ensuremath{\mathscr{D}}}
\newcommand{\scE}{\ensuremath{\mathscr{E}}}
\newcommand{\scF}{\ensuremath{\mathscr{F}}}
\newcommand{\scG}{\ensuremath{\mathscr{G}}}
\newcommand{\scH}{\ensuremath{\mathscr{H}}}
\newcommand{\scI}{\ensuremath{\mathscr{I}}}
\newcommand{\scJ}{\ensuremath{\mathscr{J}}}
\newcommand{\scK}{\ensuremath{\mathscr{K}}}
\newcommand{\scL}{\ensuremath{\mathscr{L}}}
\newcommand{\scM}{\ensuremath{\mathscr{M}}}
\newcommand{\scN}{\ensuremath{\mathscr{N}}}
\newcommand{\scO}{\ensuremath{\mathscr{O}}}
\newcommand{\scP}{\ensuremath{\mathscr{P}}}
\newcommand{\scQ}{\ensuremath{\mathscr{Q}}}
\newcommand{\scR}{\ensuremath{\mathscr{R}}}
\newcommand{\scS}{\ensuremath{\mathscr{S}}}
\newcommand{\scT}{\ensuremath{\mathscr{T}}}
\newcommand{\scU}{\ensuremath{\mathscr{U}}}
\newcommand{\scV}{\ensuremath{\mathscr{V}}}
\newcommand{\scW}{\ensuremath{\mathscr{W}}}
\newcommand{\scX}{\ensuremath{\mathscr{X}}}
\newcommand{\scY}{\ensuremath{\mathscr{Y}}}
\newcommand{\scZ}{\ensuremath{\mathscr{Z}}}
\newcommand{\scAH}{\ensuremath{\mathscr{A}\!\!\scH}}

%
%---------- mathfrak font -----------------------------
%

\newcommand{\frakA}{\ensuremath{\mathfrak{A}}}
\newcommand{\frakB}{\ensuremath{\mathfrak{B}}}
\newcommand{\frakC}{\ensuremath{\mathfrak{C}}}
\newcommand{\frakD}{\ensuremath{\mathfrak{D}}}
\newcommand{\frakE}{\ensuremath{\mathfrak{E}}}
\newcommand{\frakF}{\ensuremath{\mathfrak{F}}}
\newcommand{\frakG}{\ensuremath{\mathfrak{G}}}
\newcommand{\frakH}{\ensuremath{\mathfrak{H}}}
\newcommand{\frakI}{\ensuremath{\mathfrak{I}}}
\newcommand{\frakJ}{\ensuremath{\mathfrak{J}}}
\newcommand{\frakK}{\ensuremath{\mathfrak{K}}}
\newcommand{\frakL}{\ensuremath{\mathfrak{L}}}
\newcommand{\frakM}{\ensuremath{\mathfrak{M}}}
\newcommand{\frakN}{\ensuremath{\mathfrak{N}}}
\newcommand{\frakO}{\ensuremath{\mathfrak{O}}}
\newcommand{\frakP}{\ensuremath{\mathfrak{P}}}
\newcommand{\frakQ}{\ensuremath{\mathfrak{Q}}}
\newcommand{\frakR}{\ensuremath{\mathfrak{R}}}
\newcommand{\frakS}{\ensuremath{\mathfrak{S}}}
\newcommand{\frakT}{\ensuremath{\mathfrak{T}}}
\newcommand{\frakU}{\ensuremath{\mathfrak{U}}}
\newcommand{\frakV}{\ensuremath{\mathfrak{V}}}
\newcommand{\frakW}{\ensuremath{\mathfrak{W}}}
\newcommand{\frakX}{\ensuremath{\mathfrak{X}}}
\newcommand{\frakY}{\ensuremath{\mathfrak{Y}}}
\newcommand{\frakZ}{\ensuremath{\mathfrak{Z}}}
\newcommand{\fraka}{\ensuremath{\mathfrak{a}}}
\newcommand{\frakb}{\ensuremath{\mathfrak{b}}}
\newcommand{\frakc}{\ensuremath{\mathfrak{c}}}
\newcommand{\frakd}{\ensuremath{\mathfrak{d}}}
\newcommand{\frake}{\ensuremath{\mathfrak{e}}}
\newcommand{\frakf}{\ensuremath{\mathfrak{f}}}
\newcommand{\frakg}{\ensuremath{\mathfrak{g}}}
\newcommand{\frakh}{\ensuremath{\mathfrak{h}}}
\newcommand{\fraki}{\ensuremath{\mathfrak{i}}}
\newcommand{\frakj}{\ensuremath{\mathfrak{j}}}
\newcommand{\frakk}{\ensuremath{\mathfrak{k}}}
\newcommand{\frakl}{\ensuremath{\mathfrak{l}}}
\newcommand{\frakm}{\ensuremath{\mathfrak{m}}}
\newcommand{\frakn}{\ensuremath{\mathfrak{n}}}
\newcommand{\frako}{\ensuremath{\mathfrak{o}}}
\newcommand{\frakp}{\ensuremath{\mathfrak{p}}}
\newcommand{\frakq}{\ensuremath{\mathfrak{q}}}
\newcommand{\frakr}{\ensuremath{\mathfrak{r}}}
\newcommand{\fraks}{\ensuremath{\mathfrak{s}}}
\newcommand{\frakt}{\ensuremath{\mathfrak{t}}}
\newcommand{\fraku}{\ensuremath{\mathfrak{u}}}
\newcommand{\frakv}{\ensuremath{\mathfrak{v}}}
\newcommand{\frakw}{\ensuremath{\mathfrak{w}}}
\newcommand{\frakx}{\ensuremath{\mathfrak{x}}}
\newcommand{\fraky}{\ensuremath{\mathfrak{y}}}
\newcommand{\frakz}{\ensuremath{\mathfrak{z}}}
\newcommand{\fraksl}{\ensuremath{\mathfrak{sl}}}
\newcommand{\frakso}{\ensuremath{\mathfrak{so}}}
\newcommand{\fraksp}{\ensuremath{\mathfrak{sp}}}

%%%%%%%%%%%%  Calligraphic, Roman and Maths integers %%%%%%%%%%%%%%%%%%

\newcommand{\cA}{\mathcal{A}}
\newcommand{\cB}{\mathcal{B}}
\newcommand{\cC}{\mathcal{C}}
\newcommand{\cD}{\mathcal{D}}
\newcommand{\cE}{\mathcal{E}}
\newcommand{\cF}{\mathcal{F}}
\newcommand{\cG}{\mathcal{G}}
\newcommand{\cH}{\mathcal{H}}
\newcommand{\cI}{\mathcal{I}}
\newcommand{\cJ}{\mathcal{J}}
\newcommand{\cK}{\mathcal{K}}
\newcommand{\cL}{\mathcal{L}}
\newcommand{\cM}{\mathcal{M}}
\newcommand{\cN}{\mathcal{N}}
\newcommand{\cO}{\mathcal{O}}
\newcommand{\cQ}{\mathcal{Q}}
\newcommand{\cS}{\mathcal{S}}
\newcommand{\cX}{\mathcal{X}}
\newcommand{\cY}{\mathcal{Y}}
\newcommand{\cW}{\mathcal{W}}
\newcommand{\cR}{\mathcal{R}}
\newcommand{\cT}{\mathcal{T}}
\newcommand{\cZ}{\mathcal{Z}}

%%%%%%%%%%%%%%%%%%%%%%%%%%%%%%%%%%%%%%%%%%%%%%%%%%%%%%%%%%%%%%%%
\newcommand{\SU}{\mathrm{SU}}
\newcommand{\SO}{\mathrm{SO}}
\newcommand{\SL}{\mathrm{SL}}
\newcommand{\Sp}{\mathrm{Sp}}
\newcommand{\su}{\mathrm{su}}
\newcommand{\so}{\mathrm{so}}
\newcommand{\spl}{\mathrm{sp}}
\newcommand{\gl}{\mathrm{gl}}
\newcommand{\sll}{\mathrm{sl}}
\newcommand{\U}{\mathrm{U}}
\newcommand{\ul}{\mathrm{u}}
\newcommand{\Spin}{\mathrm{Spin}}
\newcommand{\Pin}{\mathrm{Pin}}
%%%%%%%%%%%%%%%%%%%%%%%%%%%%%%%%%%%%%%%%%%%%%%%%%%%%%%%%%%%%%%%%
\renewcommand{\Im}{{\rm Im}}
\renewcommand{\Re}{{\rm Re}}
\newcommand{\Tr}{\mbox{Tr}}
\newcommand{\Pf}{\mbox{Pf}}
\newcommand{\sgn}{\mbox{sgn}}
\newcommand{\Vir}{{\rm Vir}}
\newcommand{\Li}{{\rm Li}}

\def\tl{\tilde}
\def\wt{\widetilde}
\def\wh{\widehat}
\def\bar{\overline}





\def\bea{\begin{align}}
\def\eea{\end{align}}
\def\be{\begin{equation}}
\def\ee{\end{equation}}
\def\ba{\begin{align}}
\def\ea{\end{align}}


\usepackage{graphicx}


%%% Yokoyama def %%%
\newcommand{\ol}{\overline}
\newcommand{\nord}[1]{\vcentcolon\mathrel{#1}\vcentcolon}
\providecommand{\vcentcolon}{\mathrel{\mathop{:}}}
%%%%%%%%%%%%%%%%%%%%


\begin{document}\thispagestyle{empty}

\centerline{\Large \bf Homework 5: Due at class on Oct 27}


\subsection*{Prob. 1 (Ghost number anomaly)}


\subsubsection*{1.1 Ghost number current}

Derive the conserved current for the transformation
\begin{align}
 \delta_{g} c = \epsilon_g c \ , \quad \delta_{g} b = -\epsilon_g b \ .
\end{align}
Holomorphic part of the ghost action is given by
\begin{align}
 S_{gh} = \frac{1}{2\pi} \int d^2z b \ol\partial c \ .
\end{align}
(You can forget about the anti-holomorphic part for the problem.)



\subsubsection*{1.2 OPE of EM tensor and the current}

Calculate the OPE between ghost EM tensor and the current derived above.
The EM tensor is given as follows.
\begin{align}
 T(z) = -\nord{(2b\partial c +\partial bc)(z)} \ .
\end{align}

Furthermore, write down an infinitesimal conformal transformation (with parameter $\epsilon(z)$) of the ghost number current
from the OPE result. (Only holomorphic part is enough.)


\subsubsection*{1.3 Ghost number anomaly from curved WS}


Use the assumption $\nabla^a j_a = \kappa R^{(2)}$ derive the current $j_z = -4\kappa \partial \omega -j(z)$.
(You can assume that $j_{\ol z}=0$.)
Metric is given by
\begin{align}
 ds^2 = e^{2\omega} dzd\ol z \ .
\end{align}

Conformal transformation laws for $j_z$ and $\omega$ are given as follows.
\begin{align}
 \wt{j_z} (\wt z) = \left(\frac{\partial \wt z}{\partial z}\right)^{-1} j_z(z) \ ,\\
 \wt \omega (\wt z) = \omega(z) -\frac{1}{2} \log \left| \frac{\partial \wt z}{\partial z} \right|^2 \ ,
\end{align}
where $\wt z = z -\epsilon(z)$.
Using the transformations, derive the infinitesimal transformation for $j(z) = -j_z(z) -4\kappa \partial \omega(z)$,
and confirm $\kappa = \frac{3}{4}$ by comparing the infinitesimal transformation derived here
and the one derived from the OPE calculation.




\subsection*{Prob. 2 (Tree amplitude)}

There are a few approach for derivation of the string amplitude.
One is path integral method we discussed in the lecture,
Here we use holomorphicity and operator method to derive the string amplitude.


\subsubsection*{2.1 Conformal Killing Vectors}

Show that
\begin{align}
 \left(P\cdot\epsilon\right)_{ab} = 0 \ , \quad \nabla^a \theta_{ab} = 0 \ ,
\end{align}
reduces to
\begin{align}
 \partial \ol\epsilon = \ol\partial\epsilon = 0 \ , \quad
 \partial \ol\theta = \ol\partial \theta = 0 \ ,
\end{align}
for conformal gauge $ds^2 = e^{2\omega}dzd\ol z$,
where $\epsilon = \epsilon^z$, $\theta = \theta_{zz}$, and similar for their
anti-holomorphic part.
Note that the position of the index is determined so because
$\epsilon^a$ is a vector and $\theta_{ab}$ is a 2-form.

Let us consider sphere case.
On sphere we can consider 2 patches,
on which the coordinates are $z$ and $u$, respectively,
and the metrics are flat.
Their transition is defined as
\begin{align}
 u = \frac{1}{z} \ .
\end{align}
Derive $\epsilon$ and $\theta$ by requiring that they are globally defined
holomorphic vector and holomorphic 2-form, respectively.



\subsubsection*{2.2 Ghost sector}


Let us consider the ghost number current $j = \nord{cb}$ on sphere.
Finite conformal transformation of $j$ is given by
\begin{align}
 j(z) = \left(\frac{\partial \wt z}{\partial z}\right) \wt j (\wt z)
 -\frac{3}{2}\partial_z \log \left(\frac{\partial \wt z}{\partial z}\right) \ .
\end{align}
Show that ghost number operator
\begin{align}
 \int_C \frac{dz}{2\pi i} j(z)
\end{align}
should be equal to $3$ by considering the other patch $\wt z=u$.


Derive the expression for $\langle c(z_1) c(z_2) c(z_3) \rangle$ on sphere up to
over all constant, from the features:
$c(z)$ is fermionic so $\lim_{z\to w} c(z)c(w) = 0$,
and $c(z)$ has weight $-1$.
(forget about the anomaly for the anti-holomorphic part otherwise this is zero).





\subsubsection*{2.3 Matter sector}

Here we use operator method to derive tachyon 4pt expectation value:
\begin{align}
 \left\langle \nord{e^{ik_1\cdot X(z_1)}} \nord{e^{ik_2\cdot X(z_2)}} \nord{e^{ik_3\cdot X(z_3)}} \nord{e^{ik_4\cdot X(z_4)}} \right\rangle \ .
\end{align}
Set $(z_1,z_2,z_3)$ to $(0,1,\infty)$ and then the expectation value simplifies to
\begin{align}
 &\left\langle \nord{e^{ik_3\cdot X(\infty)}} T\left[\nord{e^{ik_4\cdot X(z_4)}} \nord{e^{ik_2\cdot X(1)}}\right] \nord{e^{ik_1\cdot X(0)}} \right\rangle  \nonumber\\
 &\quad = \left\langle 0,k_3 \left| T\left[\nord{e^{ik_4\cdot X(z_4)}} \nord{e^{ik_2\cdot X(1)}}\right] \right|0;k_1\right\rangle \ ,
\end{align}
where $T[]$ is a radial ordering, namely,
\begin{align}
 T\left[\nord{e^{ik_4\cdot X(z_4)}} \nord{e^{ik_2\cdot X(1)}}\right]
 = \nord{e^{ik_4\cdot X(z_4)}} \nord{e^{ik_2\cdot X(1)}} \quad \textrm{if}\ |z_4| > 1 \ .
\end{align}
Note that $\nord{}$ can be understood as the normal ordering of the operators,
namely,
\begin{align}
 \nord{e^{ik\cdot X(z)}} = e^{ik\cdot X_+(z)} e^{ik\cdot X_-(z)} \ ,
\end{align}
where
\begin{align}
 &X^\mu_+(z) = x^\mu -i\sqrt{\frac{\alpha'}{2}} \sum_{n=1}^\infty \frac{1}{n} \left( \alpha^\mu_{-n} z^n +\wt \alpha^\mu_{-n} \ol z^n\right) \ , \\
 &X^\mu_-(z) = -i\frac{\alpha'}{2} p^\mu \log|z|^2  +i\sqrt{\frac{\alpha'}{2}} \sum_{n=1}^\infty \frac{1}{n} \left( \alpha^\mu_{n} z^{-n} +\wt \alpha^\mu_{n} \ol z^{-n}\right) \ .
\end{align}
Assume $|z_4|>1$ and compute
\begin{align}
 \left\langle 0,k_3 \left| \nord{e^{ik_4\cdot X(z_4)}} \nord{e^{ik_2\cdot X(1)}} \right|0;k_1\right\rangle \ .
\end{align}
Guess what is the result for $|z_4|<1$ case.
Campbell-Baker-Hausdorff formula is useful.
\begin{align}
 e^A e^B  = e^B e^A e^{\left[A,B\right]} \ .
\end{align}



\subsubsection*{2.4 Another info from the amplitude}

Show that residue of the Shapiro-Virasoro amplitude at $s$-channel pole
can be written as a polynomial of $t-u$.
What is the relation between the order of the polynomial and
the maximum spin of the $s$-channel pole state?
What is the slope of the spin versus mass-squared plot?



% \subsection*{Prob. 3 (Towards 1-loop string amplitude)}

% \subsubsection*{3.1 Matter partition function}

% \subsubsection*{3.2 Ghost partition function}

% \subsubsection*{3.3 Something}




\end{document}

