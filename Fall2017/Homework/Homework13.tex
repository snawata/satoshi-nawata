\documentclass[a4j,12pt]{jarticle}


% \usepackage{a4j}
\usepackage{fancyhdr}
\usepackage{lastpage}

% \usepackage{showkeys}
% \usepackage[dvips]{graphicx}

\usepackage{bm}

\usepackage{amsmath}
\usepackage{amsfonts}
\usepackage{amssymb}
\usepackage{slashed}
% \usepackage{mathbbol}  % 数字も白抜きにしてくれる
\usepackage{multirow}  % Tableで複数のセルにまたがるセルを作れる

\usepackage[a4paper,top=2.5cm,bottom=2.5cm,left=2.5cm,right=2.5cm,headsep=10pt]{geometry}
\usepackage[compact]{titlesec}
\titlespacing*{\section}{0pt}{3ex}{2ex}
\titlespacing*{\subsection}{0pt}{2ex}{1ex}
\titlespacing*{\subsubsection}{0pt}{1ex}{1ex}

% \usepackage{graphicx}           % tikzで使う graphicxと競合するので排除
% \usepackage[usenames]{color}
% \usepackage[usenames,dvipdfmx]{color}    % optionは同時に指定出来る。
\usepackage[dvipsnames,dvipdfmx]{xcolor}    % optionは同時に指定出来る。


\usepackage[british]{babel}
\input{colordvi.tex}

\usepackage[dvipdfm,colorlinks,pagebackref,pdfusetitle,urlcolor=blue,citecolor=MidnightBlue,linkcolor=MidnightBlue,bookmarksnumbered,plainpages=false]{hyperref}



\input{dummy.tex}

%%% tikzセッティング %%%%%%%%%%%%%%%%%%%%%%%%%%%%%%%
\usepackage[dvipdfmx]{graphicx}
\usepackage{tikz}
\usetikzlibrary{arrows,shapes,patterns,snakes,calc}
\input{arrowsnew}
\usetikzlibrary{decorations.markings}
\usetikzlibrary{positioning}
%%% end of tikz %%%%%%%%%%%%%%%%%%%%%%%%%%%%%%%%%%%%

\usepackage[hang,bf,figurename=Fig.\ , tablename=Table\ ,margin=1cm]{caption}
\renewcommand{\captionfont}{\footnotesize}

%%% listingsセッティング %%%%%%%%%%%%%%%%%%%%%%%%%%%
\usepackage{listings, jlisting}
\renewcommand{\lstlistingname}{Code}
\definecolor{mygreen}{rgb}{0,0.6,0}
\definecolor{mygray}{rgb}{0.5,0.5,0.5}
\definecolor{mymauve}{rgb}{0.58,0,0.82}

\lstset{ %
  % language=Octave,                 % the language of the code
  backgroundcolor=\color{white},   % choose the background color; you must add \usepackage{color} or \usepackage{xcolor}
  basicstyle=\footnotesize,        % the size of the fonts that are used for the code
  breakatwhitespace=false,         % sets if automatic breaks should only happen at whitespace
  breaklines=true,                 % sets automatic line breaking
  captionpos=t,                    % sets the caption-position to bottom
  commentstyle=\color{mygreen},    % comment style
  deletekeywords={...},            % if you want to delete keywords from the given language
  escapeinside={\%*}{*)},          % if you want to add LaTeX within your code
  extendedchars=true,              % lets you use non-ASCII characters; for 8-bits encodings only, does not work with UTF-8
  frame=single,                    % adds a frame around the code
  keepspaces=true,                 % keeps spaces in text, useful for keeping indentation of code (possibly needs columns=flexible)
  keywordstyle=\color{blue},       % keyword style
  morekeywords={*,...},            % if you want to add more keywords to the set
  numbers=left,                    % where to put the line-numbers; possible values are (none, left, right)
  numbersep=5pt,                   % how far the line-numbers are from the code
  numberstyle=\tiny\color{mygray}, % the style that is used for the line-numbers
  rulecolor=\color{black},         % if not set, the frame-color may be changed on line-breaks within not-black text (e.g. comments (green here))
  showspaces=false,                % show spaces everywhere adding particular underscores; it overrides 'showstringspaces'
  showstringspaces=false,          % underline spaces within strings only
  showtabs=false,                  % show tabs within strings adding particular underscores
  stepnumber=1,                    % the step between two line-numbers. If it's 1, each line will be numbered
  stringstyle=\color{mymauve},     % string literal style
  tabsize=2,                       % sets default tabsize to 2 spaces
  title=\lstname                   % show the filename of files included with \lstinputlisting; also try caption instead of title
}
%%% ned of listings セッティング %%%%%%%%%%%%%%%%%%%%%%%%%%%


\pdfstringdefDisableCommands{%
    \renewcommand*{\bm}[1]{#1}%
    % any other necessary redefinitions 
}

%%%今村セッテッティング%%%%%%%%%%%%%%%%%%%%%%%%%%%%%
\newcommand{\CC}{\mathbb{C}}
\newcommand{\ZZ}{\mathbb{Z}}
\newcommand{\RR}{\mathbb{R}}
\newcommand{\HH}{\mathbb{H}}

\newcommand{\hf}{\frac{1}{2}}
\newcommand{\tr}{{\rm tr}}
\newcommand{\ind}{{\rm ind}}
\newcommand{\ol}{\overline}
\newcommand{\ul}{\underline}
\newcommand{\up}{\uparrow}
\newcommand{\dn}{\downarrow}
\newcommand{\wt}{\widetilde}
\newcommand{\ra}{\rightarrow}
\newcommand{\wh}{\widehat}


%%%横山セッティング%%%%%%%%%%%%%%%%%%%%%%%%%%%%%%%%%
\newcommand{\NN}{\mathcal{N}\!}
\newcommand{\DD}{\mathcal{D}}
\newcommand{\UU}{U(1)}
\newcommand{\dd}{\mathrm{d}}
\renewcommand{\SS}{\mathbf{S}}
\renewcommand{\Im}{\mathrm{Im}}
\renewcommand{\Re}{\mathrm{Re}}
\renewcommand{\<}{\langle}
\renewcommand{\>}{\rangle}
\newcommand{\Tr}{{\rm Tr}}

\renewcommand{\r}{\mathrm}

\newcommand{\sign}{\mathrm{sign}}

\newcommand{\lra}{\leftrightarrow}
\newcommand{\LL}{\mathcal{L}}
\newcommand{\la}{\leftarrow}
\newcommand{\ro}{\sqrt}
\newcommand{\Ra}{\Rightarrow}
\newcommand{\Pexp}{\mathrm{Pexp}}

\newcommand{\nn}{\nonumber \\}
\newcommand{\1}{\mbox{1}\hspace{-0.25em}\mbox{l}}

%数字のみ対応
\newcommand{\Maru}[1]{\ooalign{
\ifnum#1<10 \hfil\resizebox{.9\width}{.85\height}{#1}\hfil
\else
\hfil\resizebox{.6\width}{.8\height}{#1}\hfil
\fi
\crcr
\raise.1ex\hbox{$\bigcirc$}}}

%全文字対応
\newcommand{\maru}[1]{\ooalign{
\hfil\resizebox{.8\width}{\height}{#1}\hfil
\crcr
\raise.1ex\hbox{\large$\bigcirc$}}}


\newcommand{\nord}[1]{\vcentcolon\mathrel{#1}\vcentcolon}
\providecommand{\vcentcolon}{\mathrel{\mathop{:}}}


\def\P{\mathop{\cal P}}
\def\diag{\mathop{\rm diag}}


\def\Re{\mathop{\rm Re}\nolimits}
\def\Im{\mathop{\rm Im}\nolimits}
\def\Det{\mathop{\rm Det}\nolimits}
\def\sign{\mathop{\rm sign}\nolimits}


%%% rap %%% - make two letters overlap
\newcommand{\rap}[2]
{\setbox1=\hbox{#1}%
\setbox2=\hbox to\wd1{\hss #2\hss}%
\mbox{\rlap{\box1}\box2}}

%\newcommand{\sla}[1]{\rap{$#1$}{/}}
\newcommand{\sla}[1]{\rap{$#1$}{$\backslash$}}


\def\DY#1{{\MyGreen [DY: #1]}}
\newcommand{\MyGreen}{\color [rgb]{0,0.7,0}}

\usepackage[vcentermath]{youngtab}
% \Yboxdim4pt
\newcommand{\Y}{\yng}
\newcommand{\Young}{}


%%%title def%%%%%%%%%%%%%%%%%%%%%%%%%%%%%%%%%%%%%%%%%%%%%%%

\makeatletter
\def\maketitle{
\noindent
{\Large \@title \par\vskip 2pt}
\noindent
{\large \@date \hspace{4pt} \@author}
%\\[-2pt]
%\noindent------------------------------------------------------------------------------------------
\par\vskip 1.5em
}

\author{横山 大輔}
\date{\today}
%%%本文%%%%%%%%%%%%%%%%%%%%%%%%%%%%%%%%%%%%%%%%%%%%%%%%%%%%%%%%

% \title{\centerline{Lecture 2}}
\begin{document}
% \maketitle
% \begin{abstract}

% \end{abstract}
% \tableofcontents

% \parindent=0pt

  \pagestyle{fancy}
  \renewcommand{\headrulewidth}{0.0pt}
  \rhead{}
  \lhead{}
  \cfoot{[\ \scshape\oldstylenums{\thepage}\ / %
    \scshape\oldstylenums{\pageref{lastpage}} ]}
%  \rfoot{\@author}

% \setcounter{section}{}
% \setcounter{subsection}{}
% \setcounter{subsubsection}{}

\centerline{\Large \bf  Homework 13: Due at Dec 25}

% \vspace{12pt}
% \DY{なにかコメント}

% \vspace{-12pt}

% \vspace{-4pt}
% \begin{itemize}
%  \setlength{\itemsep}{0pt}
%  \item Type I superstring theory.
%  \item In type I theory, we only have D$1$-, D$5$-, and D$9$-, branes.
%  \item O$9^-$-plane is needed for type I to be consistent theory.
%  \item T-duality of type I theory.
% \end{itemize}
% \vspace{-4pt}




\subsection*{Prob. 1 DBI action}

DBI action for D$p$-brane is given by
\begin{align}
 &S_\mathrm{DBI} = \int d^{p+1} \sigma \LL_\mathrm{DBI} \ , \\
 &\LL_\mathrm{DBI} = T_{\mathrm Dp} -T_{\mathrm Dp}
 \sqrt{-\det \left(G_{ab} +k F_{ab} -B_{ab}\right)} \ ,
 \label{eq:lag}
\end{align}
where we assumed the world-sheet metric $h_{ab}$ is flat (i.e. $h_{ab} = \eta_{ab} = \diag(-1,+1,\ldots,+1)$),
the first constant term is added so that the vacuum energy becomes zero,
$k$ is $2\pi\alpha' = 2\pi l_s^2$,
$T_{\mathrm Dp} = \frac{2\pi}{g_s (2\pi l_s)^{p+1}}$,
and
\begin{align}
 G_{ab} = \partial_a X^\mu \partial_b X^\nu \eta_{\mu\nu} \ ,
\end{align}
namely, we assume the space-time is flat.
In the following we always assume $B_{ab} = 0$.

Static gauge is defined by $X^i = \sigma^i$ for $(i=0,1,\cdots,p)$,
and rewrite $X^s = \phi^s(\sigma^i)$ for $(s=p+1,\cdots,9)$.
Displacement field is defined by $\bm D = \frac{\partial \LL}{\partial \bm E}$, where $\bm E = F_{i0}\ (i=1,\cdots,p)$.

\subsubsection*{Prob. 1.1 Gauge theory on D-brane}
Consider D$3$-brane and take the static gauge.
\begin{itemize}
 \item Expand the Lagrangian~(\ref{eq:lag}) up to a quadratic order of fields
       and write down a canonical Lagrangian.
 \item Compare the coupling constant derived from the expansion and the canonical one $g$:
       \begin{align}
        \LL_\mathrm{gauge} = -\frac{1}{4g^2} F_{ab} F^{ab} \ .
       \end{align}
\end{itemize}


\subsubsection*{Prob. 1.2 Features of the DBI action 1: Schwinger effect}
Consider D$p$-brane, take the static gauge, set $\phi^s = 0$,
and also set the magnetic field to zero; $\bm B = \bm 0$.
\begin{itemize}
 \item Calculate the displacement field $\bm D = \frac{\partial \LL_\mathrm{DBI}}{\partial \bm E}$.
 \item Assume only $F_{10} = E$ has non-zero value and
       plot $E$ in terms of $D$ that is a corresponding non-zero component
       of the displacement field.
       Explain physically why there are bounds for $E$.
       (This is called \textbf{Schwinger effect}.)
\end{itemize}

\subsubsection*{Prob. 1.3 Features of the DBI action 2: Self-energy}
Consider the same setting as Prob. 1.2 with $p=3$.
\begin{itemize}
 \item Calculate the self-energy $U_Q$ of a charged particle:
       \begin{align}
        U_Q = \int d^3\sigma \mathcal H(\bm D) \ , \qquad
        \mathcal H(\bm D) = \bm E \cdot \bm D -\LL_\mathrm{DBI} \ ,
       \end{align}
       where $d^3\sigma$ is a spacial(not time) volume form.
       The charged particle is characterized by
       \begin{align}
        \bm\nabla \cdot \bm D = Q \delta^3(\bm \sigma) \ ,
       \end{align}
       where $\bm \sigma$ is a world-sheet space(not time) vector.
       You can use the following constant:
       \begin{align}
        c \equiv \int_0^\infty dx \left( \sqrt{1+x^4} -x^2 \right) \simeq 1.236 \ .
       \end{align}
 \item Compare the self-energy with that of a normal electromagnetism:
       \begin{align}
        \LL_\mathrm{EM} = \frac{\varepsilon}{2} \bm E^2 \ .
       \end{align}
       Discuss also $l_s \to 0$ limit in the self-energy.
\end{itemize}


\subsubsection*{Prob. 1.4 Brane bending}

Consider D$p$-brane, take the static gauge, set $\bm B = \bm 0$.
Assume only one of $\phi^s$ has non-zero value, say $\phi^9 \equiv X$.
Then, the Lagrangian becomes
\begin{align}
 \wt \LL_\mathrm{DBI} = T_{\mathrm Dp} -T_{\mathrm Dp}
 \sqrt{ \left( 1 -\bm{\mathcal E}^2 \right) \left( 1 +(\bm\nabla X)^2 \right)
 +\left( \bm{\mathcal E}\cdot \bm\nabla X \right)^2 -\dot X^2
 } \ ,
\end{align}
where $\bm{\mathcal E} = k \bm E$, $\dot X = \frac{\partial X}{\partial \sigma^0}$.
\begin{itemize}
 \item[a)] Show $\wt \LL_\mathrm{DBI} = \LL_\mathrm{DBI}$ in the case that
         only one of components of $\bm E$ is non-zero, say $E_1$,
         and $X$ only depends on $\sigma^0,\ \sigma^1$, and $\sigma^2$.
         (This special case implies full identity $\wt \LL_\mathrm{DBI} = \LL_\mathrm{DBI}$
         thanks to rotation symmetry.)
\end{itemize}
Let us look for a static solution, namely, all the fields are independent of $\sigma^0$.
Then, we can express $\bm E$ by $\bm E = \bm \nabla A_0$.
\begin{itemize}
 \item[b)] Derive equations of motion of $X$ and $A_0$.
         Confirm that both equations are satisfied if
         \begin{align}
          k \bm E = k \bm\nabla A_0 = \pm \bm\nabla X \ , \qquad
          k\bm\nabla^2 A_0 = \bm\nabla^2 X = 0 \ .
          \label{eq:eom}
         \end{align}
 \item[c)] Show that when (\ref{eq:eom}) holds we have
         \begin{align}
          \bm D = k^2 T_{\mathrm Dp} \bm E \ ,
         \end{align}
         and the energy
         \begin{align}
          U = \int d^p\sigma \left( \bm E \cdot \bm D -\wt \LL_\mathrm{DBI}  \right)
          = k^2 T_{\mathrm Dp} \int d^p\sigma \bm E^2 \ .
         \end{align}
\end{itemize}
Consider a string is ending on the D$p$-brane (stretching to the 9th direction),
which is a source of the electric (displacement) field
\begin{align}
 \bm\nabla \cdot \bm D = \delta^p (\bm\sigma) \ .
\end{align}
\begin{itemize}
 \item[d)] Derive the solution for $\bm D$ as well as $X$ and show that the energy $U$ is diverging.
\end{itemize}
Let us regularize the energy by a cut off $\delta$:
\begin{align}
 U(\delta) = k^2 T_{\mathrm Dp} \int_{r>\delta} d^p\sigma \bm E^2 \ .
\end{align}
\begin{itemize}
 \item[e)] Show that $U(\delta) = T_\mathrm{str} |X(\delta)| = \frac{1}{k} |X(\delta)|$.
         Draw the shape of the D$p$-brane, and discuss the meaning of $U(\delta)$.
\end{itemize}



\subsection*{Prob. 2 Bound states and supersymmetry}

In type II theories we have left $16$ supercharges $Q_L$
and right $16$ supercharges $Q_R$.
Existence of string or branes break some of supersymmetry.
Conditions are summarized in Table~\ref{table:001}.
\begin{table}[htbp]
 \begin{center}
  \caption{Supersymmetry conditions for branes.}
  \vspace{0pt}
  \label{table:001}
\begin{tabular}{l|l}
 IIA branes & Conditions \\\hline
 F$1$-String(01) & $\Gamma^{01} Q_L = Q_L\ , \quad \Gamma^{01} Q_R = -Q_R$ \\[3pt]
 NS$5$-brane(012345) & $\Gamma^{012345} Q_L = Q_L\ , \quad \Gamma^{012345} Q_R = Q_R$ \\[3pt]
 D$p$-brane($p$: even) & $\Gamma^{01\cdots p} Q_R = Q_L$
\end{tabular}
\end{center}
\begin{center}
\begin{tabular}{l|l}
 IIB branes & Conditions \\\hline
 F$1$-String(01) & $\Gamma^{01} Q_L = Q_L\ , \quad \Gamma^{01} Q_R = -Q_R$ \\[3pt]
 NS$5$-brane(012345) & $\Gamma^{012345} Q_L = Q_L\ , \quad \Gamma^{012345} Q_R = -Q_R$ \\[3pt]
 D$p$-brane($p$: odd) & $\Gamma^{01\cdots p} Q_R = Q_L$
\end{tabular}
\end{center}
\end{table}
\begin{itemize}
 \item How many supersymmetry remain if there exist D$p$-brane ? (We call this D$p$ system and similar for multi branes.)
 \item How many supersymmetry remain for D$0$-D$2$ system ?
       D$0$ lies on $(0)$ direction and D$2$ lies on $(012)$ directions.
 \item How many supersymmetry remain for F$1$-D$1$ system ?
       Both string lie on $(01)$ directions.
 \item How many supersymmetry remain for D$0$-D$4$ system ?
       D$0$ lies on $(0)$ direction and D$4$ lies on $(01234)$ directions.
 \item Show that D$4$-brane carries D$0$-brane RR-charge when
       there exist non-trivial instanton configuration ($n\neq 0$):
       \begin{align}
        \frac{1}{8\pi^2} \int F_{(2)} \wedge F_{(2)} = n \in \ZZ \ .
       \end{align}
 \end{itemize}



\subsection*{Prob. 3 Branes ending on a brane in IIB superstring}

Start from equations of motions of IIB SUGRA:
\begin{align}
 d G_\mathrm{odd} &= H_{(3)} G_\mathrm{odd} \ , \\
 d H_{(3)} &= 0 \ , \\
 d H_{(7)} &= -\frac{1}{2} \left[ (\mathcal T G_\mathrm{odd}) G_\mathrm{odd} \right]_{(8)} \ ,
\end{align}
introduce brane currents and construct current conservation equations.
From the equations make a table of branes, which can end on some branes.
Is the table compatible with S-duality ?

% \subsection*{Prob. 4 SUSY preserved by D-branes}



% \begin{thebibliography}{CDLOGP91}

%  \bibitem[Pol98]{Pol98}
%                 J. Polchinski.
%                 String theory. Vol. 1.
%                 Cambridge University Press, 1998.

%  \bibitem[BP09]{Blumenhagen:2009zz}
%                R.~Blumenhagen and E.~Plauschinn.
%                \newblock {Introduction to conformal field theory}.
%                \newblock {\em Lect. Notes Phys.}, 779:1--256, 2009.

% \end{thebibliography}


% \bibliography{string-lecture}
% \bibliographystyle{halpha}
% \bibliographystyle{JHEP}

% \begin{thebibliography}{CDLOGP91}

% %\cite{AlvarezGaume:1981hn}
% \bibitem[AFM81]{AlvarezGaume:1981hn}
%   L.~Alvarez-Gaume, D.~Z.~Freedman and S.~Mukhi,
%   ``The Background Field Method and the Ultraviolet Structure of the Supersymmetric Nonlinear Sigma Model,''
%   Annals Phys.\  {\bf 134}, 85 (1981).
%   %doi:10.1016/0003-4916(81)90006-3
%   %%CITATION = doi:10.1016/0003-4916(81)90006-3;%%
%   %525 citations counted in INSPIRE as of 09 Oct 2017



% \bibitem[CT88]{Callan:1988xx}
%  Curt Callan and Lárus Thorlacius.
%  \textit{SIGMA MODELS AND STRING THEORY.}
%  TASI Lecture, 1988.
%  (The link below is a direct link to the pdf file of 45MB )
%  \href{http://www.damtp.cam.ac.uk/user/tong/string/sigma.pdf}
%  {http://www.damtp.cam.ac.uk/user/tong/string/sigma.pdf}.




% \end{thebibliography}


\label{lastpage}

% \begin{tikzpicture}[>=stealth,scale=1]
%  \draw[->] (0,0)--(1,0);
%  \draw[latex-stealth] (0,0.5)--(1,0.5);
%  \draw[latex-stealthnew,arrowhead=2mm] (0,1)--(1,1);
% \end{tikzpicture}

% \begin{figure}[htb]
% \centerline{\includegraphics[width=250pt]{.eps}}
% \caption{}
% \label{.eps}
% \end{figure}



% \begin{table}[htbp]
%  \begin{center}
%   \caption{}
%   \vspace{4pt}
%   \label{table:001}
% \begin{tabular}{|c|c|c|c|c|}
% \hline
% \hline
%   Category & Sector & $(h_A,h_B,h_T)$ & Mirror theory & ABJM model \\
% \hline
%  1 & & \parbox{40pt}{$(0,0,0)$ $(1,1,1)$} & $1.11906$ & $1.13290$ \\
% \hline
%  2 & & \parbox{40pt}{$(0,0,1)$ $(1,1,0)$} & $-0.10861$ & $-0.10861$ \\
% \hline
%  \multirow{2}{*}{\vspace{-15pt}3} & 3-1 & \parbox{40pt}{$(0,1,0)$ $(1,0,1)$} & $0.176777$ &
%  \multirow{2}{*}{\vspace{-15pt}$0.176577$} \\
% \cline{2-4}
%  & 3-2 & \parbox{40pt}{$(1,0,0)$ $(0,1,1)$} & $0.176777$ & \\
% \hline
% \hline
% \end{tabular}
% \end{center}
% \end{table}


% \begin{thebibliography}{99}

% % \cite{Imamura:2012rq}
% \bibitem{Imamura:2012rq}
%   Y.~Imamura and D.~Yokoyama,
%   %``S^3/Z_n partition function and dualities,''
%   JHEP {\bf 1211}, 122 (2012)
%   [arXiv:1208.1404 [hep-th]].
%   %%CITATION = ARXIV:1208.1404;%%

 % \bibitem{fnorio:legendre}
 %         fnorio
 %         ``ルジャンドル変換とは何か''
 %         \url{http://fnorio.com/0146Legendre_transformation/Legendre_transformation.html}


 % \bibitem{EMAN:dynamics}
 %         EMAN物理学
 %         ``ハミルトニアン''
 %         \url{http://eman-physics.net/analytic/hamilton.html}


 % \bibitem{Wiki:legendre}
 %         Wiki
 %         ``ルジャンドル変換''
 %         \url{https://ja.wikipedia.org/wiki/ルジャンドル変換}

 % \bibitem{mathtrain:legendre}
 %         高校数学の美しい物語
 %         ``ルジャンドル変換の意味と具体例''
 %         \url{http://mathtrain.jp/legendrehenkan}

% \end{thebibliography}


% \bibliography{dd}

\end{document}
