\documentclass[12pt,a4paper]{article}
%\usepackage{hyperref} % Use the Charter font for the document text
%\usepackage[UTF8]{ctex}


\usepackage{amsfonts,amssymb,amsmath}
\usepackage{mathtools}
\usepackage{tikz-cd}
\usepackage{fullpage}
\usepackage{tikz}
\usepackage{alltt}
\usepackage{amsfonts}
\usepackage{amsmath}
\usepackage{amssymb}
\usepackage{amsthm}
\usepackage{booktabs}
\usepackage{caption}
\usepackage{enumitem}
\usepackage{fancyhdr}
\usepackage{graphicx}
% \usepackage{mathdots}
\usepackage{mathtools}
\usepackage{microtype}
\usepackage{multirow}
\usepackage{pdflscape}
\usepackage{pgfplots}
\usepackage{siunitx}
\usepackage{slashed}
\usepackage{tabularx}
\usepackage{tikz}
\usepackage{tkz-euclide}
% \usepackage[normalem]{ulem}
\usepackage[all]{xy}
\usepackage{imakeidx}

\usepackage{wrapfig}



%%%%%%%  Greek letters %%%%%%%%%%%%%%%%%%
\def\a{\alpha}
\def\b{\beta}
\def\c{\gamma} \def\g{\gamma}
\def\d{\delta}
\def\e{\epsilon}
\def\f{\phi}
\def\vf{\varphi}  \def\tvf{\tilde{\varphi}}
\def\vp{\varphi}
\def\h{\eta}
\def\i{\iota}
\def\j{\psi}
\def\k{\kappa}
\def\m{\mu}
\def\n{\nu}
\def\o{\omega}  \def\w{\omega}
\def\q{\theta}  \def\th{\theta}
\def\r{\rho}
\def\s{\sigma}
\def\t{\tau}
\def\u{\upsilon}
\def\x{\xi}
\def\z{\zeta}

\def\A{\Alpha}
\def\B{\Beta}
\def\G{\Gamma}
\def\D{\Delta}
\def\E{\Epsilon}
\def\F{Phi}
\def\h{\eta}
\def\I{\Iota}
\def\J{Psi}
\def\K{\Kappa}
\def\L{\Lambda}
\def\M{\Mu}
\def\N{\Nu}
\def\O{\Omega}  \def\w{\omega}
\def\Q{\Theta}  \def\Th{\Theta}
\def\R{\Rho}
\def\Si{\Sigma}
\def\T{\Tau}
\def\Up{\Upsilon}
\def\X{\Xi}
\def\Z{\Zeta}








%%%%%%%%%%%% math fonts %%%%%%%%%%%%%%%%%%%%%%%%%%%%%%%%%%%%%
%
%---------- mathbb font --------------------------------
%

\newcommand{\bA}{\ensuremath{\mathbb{A}}}
\newcommand{\bB}{\ensuremath{\mathbb{B}}}
\newcommand{\bC}{\ensuremath{\mathbb{C}}}
\newcommand{\bD}{\ensuremath{\mathbb{D}}}
\newcommand{\bE}{\ensuremath{\mathbb{E}}}
\newcommand{\bF}{\ensuremath{\mathbb{F}}}
\newcommand{\bG}{\ensuremath{\mathbb{G}}}
\newcommand{\bH}{\ensuremath{\mathbb{H}}}
\newcommand{\bI}{\ensuremath{\mathbb{I}}}
\newcommand{\bJ}{\ensuremath{\mathbb{J}}}
\newcommand{\bK}{\ensuremath{\mathbb{K}}}
\newcommand{\bL}{\ensuremath{\mathbb{L}}}
\newcommand{\bM}{\ensuremath{\mathbb{M}}}
\newcommand{\bN}{\ensuremath{\mathbb{N}}}
\newcommand{\bO}{\ensuremath{\mathbb{O}}}
\newcommand{\bP}{\ensuremath{\mathbb{P}}}
\newcommand{\bQ}{\ensuremath{\mathbb{Q}}}
\newcommand{\bR}{\ensuremath{\mathbb{R}}}
\newcommand{\bS}{\ensuremath{\mathbb{S}}}
\newcommand{\bT}{\ensuremath{\mathbb{T}}}
\newcommand{\bU}{\ensuremath{\mathbb{U}}}
\newcommand{\bV}{\ensuremath{\mathbb{V}}}
\newcommand{\bW}{\ensuremath{\mathbb{W}}}
\newcommand{\bX}{\ensuremath{\mathbb{X}}}
\newcommand{\bY}{\ensuremath{\mathbb{Y}}}
\newcommand{\bZ}{\ensuremath{\mathbb{Z}}}


%
%\parskip=1em
%\parindent=0.3in
%\setlength\oddsidemargin{0.5in} \setlength\evensidemargin{0.5in}
%\setlength\textwidth{5.5in}
%
%\hfuzz6pt % Don't bother to report over-full boxes if over-edge is < 6pt
%
%\newlength{\defbaselineskip}
%\setlength{\defbaselineskip}{\baselineskip}
%\newcommand{\setlinespacing}[1]%
%           {\setlength{\baselineskip}{#1 \defbaselineskip}}
%\newcommand{\doublespacing}{\setlength{\baselineskip}%
%                           {2.0 \defbaselineskip}}
%\newcommand{\singlespacing}{\setlength{\baselineskip}{\defbaselineskip}}
%
%\newcommand{\properpagestyle}{\pagestyle{myheadings}\markboth{}{}\markright{}}


%---------- mathscript font -----------------------------
%

\newcommand{\scA}{\ensuremath{\mathscr{A}}}
\newcommand{\scB}{\ensuremath{\mathscr{B}}}
\newcommand{\scC}{\ensuremath{\mathscr{C}}}
\newcommand{\scD}{\ensuremath{\mathscr{D}}}
\newcommand{\scE}{\ensuremath{\mathscr{E}}}
\newcommand{\scF}{\ensuremath{\mathscr{F}}}
\newcommand{\scG}{\ensuremath{\mathscr{G}}}
\newcommand{\scH}{\ensuremath{\mathscr{H}}}
\newcommand{\scI}{\ensuremath{\mathscr{I}}}
\newcommand{\scJ}{\ensuremath{\mathscr{J}}}
\newcommand{\scK}{\ensuremath{\mathscr{K}}}
\newcommand{\scL}{\ensuremath{\mathscr{L}}}
\newcommand{\scM}{\ensuremath{\mathscr{M}}}
\newcommand{\scN}{\ensuremath{\mathscr{N}}}
\newcommand{\scO}{\ensuremath{\mathscr{O}}}
\newcommand{\scP}{\ensuremath{\mathscr{P}}}
\newcommand{\scQ}{\ensuremath{\mathscr{Q}}}
\newcommand{\scR}{\ensuremath{\mathscr{R}}}
\newcommand{\scS}{\ensuremath{\mathscr{S}}}
\newcommand{\scT}{\ensuremath{\mathscr{T}}}
\newcommand{\scU}{\ensuremath{\mathscr{U}}}
\newcommand{\scV}{\ensuremath{\mathscr{V}}}
\newcommand{\scW}{\ensuremath{\mathscr{W}}}
\newcommand{\scX}{\ensuremath{\mathscr{X}}}
\newcommand{\scY}{\ensuremath{\mathscr{Y}}}
\newcommand{\scZ}{\ensuremath{\mathscr{Z}}}
\newcommand{\scAH}{\ensuremath{\mathscr{A}\!\!\scH}}

%
%---------- mathfrak font -----------------------------
%

\newcommand{\frakA}{\ensuremath{\mathfrak{A}}}
\newcommand{\frakB}{\ensuremath{\mathfrak{B}}}
\newcommand{\frakC}{\ensuremath{\mathfrak{C}}}
\newcommand{\frakD}{\ensuremath{\mathfrak{D}}}
\newcommand{\frakE}{\ensuremath{\mathfrak{E}}}
\newcommand{\frakF}{\ensuremath{\mathfrak{F}}}
\newcommand{\frakG}{\ensuremath{\mathfrak{G}}}
\newcommand{\frakH}{\ensuremath{\mathfrak{H}}}
\newcommand{\frakI}{\ensuremath{\mathfrak{I}}}
\newcommand{\frakJ}{\ensuremath{\mathfrak{J}}}
\newcommand{\frakK}{\ensuremath{\mathfrak{K}}}
\newcommand{\frakL}{\ensuremath{\mathfrak{L}}}
\newcommand{\frakM}{\ensuremath{\mathfrak{M}}}
\newcommand{\frakN}{\ensuremath{\mathfrak{N}}}
\newcommand{\frakO}{\ensuremath{\mathfrak{O}}}
\newcommand{\frakP}{\ensuremath{\mathfrak{P}}}
\newcommand{\frakQ}{\ensuremath{\mathfrak{Q}}}
\newcommand{\frakR}{\ensuremath{\mathfrak{R}}}
\newcommand{\frakS}{\ensuremath{\mathfrak{S}}}
\newcommand{\frakT}{\ensuremath{\mathfrak{T}}}
\newcommand{\frakU}{\ensuremath{\mathfrak{U}}}
\newcommand{\frakV}{\ensuremath{\mathfrak{V}}}
\newcommand{\frakW}{\ensuremath{\mathfrak{W}}}
\newcommand{\frakX}{\ensuremath{\mathfrak{X}}}
\newcommand{\frakY}{\ensuremath{\mathfrak{Y}}}
\newcommand{\frakZ}{\ensuremath{\mathfrak{Z}}}
\newcommand{\fraka}{\ensuremath{\mathfrak{a}}}
\newcommand{\frakb}{\ensuremath{\mathfrak{b}}}
\newcommand{\frakc}{\ensuremath{\mathfrak{c}}}
\newcommand{\frakd}{\ensuremath{\mathfrak{d}}}
\newcommand{\frake}{\ensuremath{\mathfrak{e}}}
\newcommand{\frakf}{\ensuremath{\mathfrak{f}}}
\newcommand{\frakg}{\ensuremath{\mathfrak{g}}}
\newcommand{\frakh}{\ensuremath{\mathfrak{h}}}
\newcommand{\fraki}{\ensuremath{\mathfrak{i}}}
\newcommand{\frakj}{\ensuremath{\mathfrak{j}}}
\newcommand{\frakk}{\ensuremath{\mathfrak{k}}}
\newcommand{\frakl}{\ensuremath{\mathfrak{l}}}
\newcommand{\frakm}{\ensuremath{\mathfrak{m}}}
\newcommand{\frakn}{\ensuremath{\mathfrak{n}}}
\newcommand{\frako}{\ensuremath{\mathfrak{o}}}
\newcommand{\frakp}{\ensuremath{\mathfrak{p}}}
\newcommand{\frakq}{\ensuremath{\mathfrak{q}}}
\newcommand{\frakr}{\ensuremath{\mathfrak{r}}}
\newcommand{\fraks}{\ensuremath{\mathfrak{s}}}
\newcommand{\frakt}{\ensuremath{\mathfrak{t}}}
\newcommand{\fraku}{\ensuremath{\mathfrak{u}}}
\newcommand{\frakv}{\ensuremath{\mathfrak{v}}}
\newcommand{\frakw}{\ensuremath{\mathfrak{w}}}
\newcommand{\frakx}{\ensuremath{\mathfrak{x}}}
\newcommand{\fraky}{\ensuremath{\mathfrak{y}}}
\newcommand{\frakz}{\ensuremath{\mathfrak{z}}}
\newcommand{\fraksl}{\ensuremath{\mathfrak{sl}}}
\newcommand{\frakso}{\ensuremath{\mathfrak{so}}}
\newcommand{\fraksp}{\ensuremath{\mathfrak{sp}}}

%%%%%%%%%%%%  Calligraphic, Roman and Maths integers %%%%%%%%%%%%%%%%%%

\newcommand{\cA}{\mathcal{A}}
\newcommand{\cB}{\mathcal{B}}
\newcommand{\cC}{\mathcal{C}}
\newcommand{\cD}{\mathcal{D}}
\newcommand{\cE}{\mathcal{E}}
\newcommand{\cF}{\mathcal{F}}
\newcommand{\cG}{\mathcal{G}}
\newcommand{\cH}{\mathcal{H}}
\newcommand{\cI}{\mathcal{I}}
\newcommand{\cJ}{\mathcal{J}}
\newcommand{\cK}{\mathcal{K}}
\newcommand{\cL}{\mathcal{L}}
\newcommand{\cM}{\mathcal{M}}
\newcommand{\cN}{\mathcal{N}}
\newcommand{\cO}{\mathcal{O}}
\newcommand{\cQ}{\mathcal{Q}}
\newcommand{\cS}{\mathcal{S}}
\newcommand{\cX}{\mathcal{X}}
\newcommand{\cY}{\mathcal{Y}}
\newcommand{\cW}{\mathcal{W}}
\newcommand{\cR}{\mathcal{R}}
\newcommand{\cT}{\mathcal{T}}
\newcommand{\cZ}{\mathcal{Z}}

%%%%%%%%%%%%%%%%%%%%%%%%%%%%%%%%%%%%%%%%%%%%%%%%%%%%%%%%%%%%%%%%
\newcommand{\SU}{\mathrm{SU}}
\newcommand{\SO}{\mathrm{SO}}
\newcommand{\SL}{\mathrm{SL}}
\newcommand{\Sp}{\mathrm{Sp}}
\newcommand{\su}{\mathrm{su}}
\newcommand{\so}{\mathrm{so}}
\newcommand{\spl}{\mathrm{sp}}
\newcommand{\gl}{\mathrm{gl}}
\newcommand{\sll}{\mathrm{sl}}
\newcommand{\U}{\mathrm{U}}
\newcommand{\ul}{\mathrm{u}}
\newcommand{\Spin}{\mathrm{Spin}}
\newcommand{\Pin}{\mathrm{Pin}}
%%%%%%%%%%%%%%%%%%%%%%%%%%%%%%%%%%%%%%%%%%%%%%%%%%%%%%%%%%%%%%%%
\renewcommand{\Im}{{\rm Im}}
\renewcommand{\Re}{{\rm Re}}
\newcommand{\Tr}{\mbox{Tr}}
\newcommand{\Pf}{\mbox{Pf}}
\newcommand{\sgn}{\mbox{sgn}}
\newcommand{\Vir}{{\rm Vir}}
\newcommand{\Li}{{\rm Li}}

\def\tl{\tilde}
\def\wt{\widetilde}
\def\wh{\widehat}
\def\bar{\overline}





\def\bea{\begin{align}}
\def\eea{\end{align}}
\def\be{\begin{equation}}
\def\ee{\end{equation}}
\def\ba{\begin{align}}
\def\ea{\end{align}}


\usepackage{graphicx}


%%% Yokoyama def %%%
\newcommand{\ol}{\overline}
\newcommand{\nord}[1]{\vcentcolon\mathrel{#1}\vcentcolon}
\providecommand{\vcentcolon}{\mathrel{\mathop{:}}}
%%%%%%%%%%%%%%%%%%%%


\begin{document}\thispagestyle{empty}

\centerline{\Large \bf Homework 8 (Due at class on Nov 17)}







\subsection*{Prob. 1 (Bosonic string on a circle)}


\subsubsection*{1.1}
The torus partition function of the bosonic string on a circle $S^1$ of radius $R$ is given by
\begin{align}	
Z^{25} &= \Tr~ q^{L_0 -  1 / 24} \bar q^{\wt L_0 - 1 / 24} ,	\cr 
	&=  \left| \eta(q)\right| ^ {-2}
		\sum_{n, m} q^{\frac {\a'}{4} p_R^2}\bar q ^ {\frac {\a'}{4} p_L^2}~. \label{torus-PF}
\end{align}
If we include the non-compact space $\bR^{1,24}$, we have to multiply the partition function of  the non-compact direction 
$$
Z^{1,24} =\textrm{const}\times |\eta(q)|^{-46}
$$
By expanding out the Dedekind $\eta$-functions in $Z^{1,24}Z^{25}$, show that each term means the right hand sides of the mass formula and the level matching condition:
\begin{align}
M^2&=\frac{n^2}{R^2}+\frac{w^2R^2}{\a'^2}+\frac2{\a'}(N+\wt N-2)\cr
0&=nw+N-\wt N~.\nonumber
\end{align}

\subsubsection*{1.2}
By using the Poisson resummation formula,
$$
\sum_{n\in \bZ}\exp(-\pi a n^2+2\pi i b n)=a^{-1/2}\sum_{m\in\bZ}\exp\Big[-\frac{\pi(m-b)^2}{a}\Big]~,
$$
show that \eqref{torus-PF} is modular-invariant.

\subsubsection*{1.3}
Show that the partition function \eqref{torus-PF} of the theory at the self-dual radius $R=\sqrt{\a'}$ can be written as
$$
Z^{25}=|\chi_1(q)|^2+|\chi_2(q)|^2\,, \qquad  \textrm{where} \quad \chi_1= \frac 1 {\eta} \sum_n q^{n^2} 
\quad \chi_2=\frac 1 {\eta} \sum_n q^{(n + 1 / 2)^2}
$$
The $\chi_i$ are the characters of the $\SU(2)$ affine Lie algebra with level $k=1$. By expanding this expression out find the massless states from above.

\subsubsection*{1.4}
Show that the currents in the bosonic string theory defined by
$$
j^\pm(z)=j^1(z)\pm i \, j^2(z):=e^{\pm 2iX^{25}(z)/\sqrt{\a'}} \qquad j^3(z):=i\, \partial X^{25}(z)/\sqrt{\a'}~,
$$
satisfy the OPEs
$$
j^a(z) j^b (0) \sim \frac { \delta^{ab} } {2z^2}  
		+ \frac {i {\epsilon^{abc}} j^c(0)} {z}~.
$$
From the OPEs, 
show that the oscillator modes of the currents 
$$
j^a(z)=\sum_{m\in\bZ}\frac{j^a_m}{z^{m+1}}~,
$$
satisfy
$$
[j_m^a,j^b_n]=\frac  m2\delta_{m+n,0}\delta^{ab} + i \e^{abc} j^c_{m+n}~.
$$
This infinite-dimensional algebra is called the \textbf{$\SU(2)$ affine Lie 
algebra with level $k=1$}. (Check that the zero modes satisfy the $\SU(2)$ Lie algebra.)

\subsection*{Prob. 2 (RR field strengths and T-duality in Type II)}
Let
$$\{\G^\m,\G^\n\} = 2\eta^{\m\n}\qquad  \m= 0,\cdots,9 $$ 
be the Clifford algebra of SO(1,9) gamma matrices.  The gamma matrices have the
following hermiticity property,
$$
(\G^\mu)^\dagger=-\G^0\G^\m(\G^0)^{-1}~.
$$
Verify that the chirality matrix
$$
\G^{11}=\G^0\G^1\cdots \G^9~,
$$
satisfies
$$
\G^2=1~,\qquad \{\G^{11},\G^\m\}=0~.
$$
Chiral spinors are now defined by
\be\label{chirality}
\G^{11} \psi_\pm= \pm \psi_{\pm}~,
\ee
Show that
$$ \overline \psi_\pm\G^{11}=\mp \overline \psi_\pm$$
where  $\overline \psi_\pm = \psi_\pm^\dagger \G^0 $.

We define the RR field strengths $G^{\m_1\cdots\m_{p+1}}$ as spinor bilinears
\be\label{FS}
\textrm{IIA}:~ \overline\psi_-^R\G^{\m_1\cdots\m_{p+1}}\psi_+^L~,\qquad\qquad \textrm{IIB}:~ \overline\psi_+^R\G^{\m_1\cdots\m_{p+1}}\psi_+^L~,
\ee
 where $\psi^R$ ($\psi^L$) comes from the right (left) movers
and
 $$\G^{\m_1\cdots\m_{p+1}}=\G^{[\m_1}\cdots\G^{\m_{p+1}]}  $$
 is the antisymmetric product of ($p+1$) gamma matrices.
Using the chirality \eqref{chirality} of the spinors, determine for which values of $p$ the RR field strengths \eqref{FS} are non-zero.



In the lecture, we learn that T-duality of the 9th direction in Type II theory acts the left-moving fermion mode 
 $$\wt\psi_n^9 \to -\wt\psi^9_n~,\qquad n\in\bZ~.$$
The action of duality on the spinor fields is of the form
\be\label{t-dual}
\psi^R \to \psi^R ~,\qquad
\psi^L  \to \beta_9\psi^L 
\ee
where $\beta_9 = \Gamma^9\Gamma^{11}$. Show that
$$
\{\beta_9, \G^9\}=0~, \qquad [\beta_9,\G^\mu]=0~, \quad \textrm{for}\quad \mu\neq9~.
$$
Using the effect of \eqref{t-dual} on the RR field strengths \eqref{FS}, show that T-duality transforms the RR field strengths in IIA to those in IIB, and vice versa.

\subsection*{Prob. 3 (D-branes in Type II and T-duality)}
Two D-branes intersect orthogonally over a $p$-brane if they share $p$ directions with the remaining directions wrapping different directions. For example a D5-brane extending in the directions $x^0,x^1,\cdots,x^5$ and a D3-brane extending in $x^0,x^1,x^2,x^6$ intersect orthogonally over an 2-brane $x^0,x^1,x^2$. 
\begin{table}[h]\centering
\begin{tabular}{c|ccccccccccc}
&0&1&2&3&4&5&6&7&8&9\\\hline
D3&$\times$&$\times$&$\times$&&&&$\times$&&&\\
D5&$\times$&$\times$&$\times$&$\times$&$\times$&$\times$
\end{tabular}
\end{table}

\noindent In such cases we can divide the spacetime directions into 4 sets, {NN, ND, DN, DD} according to whether the coordinate $X^\mu$ has Neumann (N) or Dirichlet (D) boundary conditions on the first or second brane. In the example of the D3-D5 system for a string stretching from the D3-brane to the D5-brane: NN = $\{x^0,x^1,x^2\}$, ND=$\{x^6\}$, DN = $\{x^3,x^4,x^5\}$,  DD = $\{x^7,x^8,x^9\}$.


\subsubsection*{3.1} Show that the numbers (\#NN+\#DD) and (\#ND+\#DN) are invariant under T-duality, where \#NN is the number of NN directions, etc.

\subsubsection*{3.2} List all orthogonal intersections in IIB string theory that have (\#ND+\#DN) = 4 and contain at least one D3-brane. Show that all these configurations are T-dual to the following D1-D5 configuration:
\begin{table}[h]\centering
\begin{tabular}{c|ccccccccccc}
&0&1&2&3&4&5&6&7&8&9\\\hline
D1&$\times$&$\times$&&&&\\
D5&$\times$&$\times$&$\times$&$\times$&$\times$&$\times$
\end{tabular}
\end{table}

\end{document}

