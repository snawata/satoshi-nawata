\documentclass[a4j,12pt]{jarticle}


% \usepackage{a4j}
\usepackage{fancyhdr}
\usepackage{lastpage}

% \usepackage{showkeys}
% \usepackage[dvips]{graphicx}

\usepackage{bm}

\usepackage{amsmath}
\usepackage{amsfonts}
\usepackage{amssymb}
\usepackage{slashed}
% \usepackage{mathbbol}  % 数字も白抜きにしてくれる
\usepackage{multirow}  % Tableで複数のセルにまたがるセルを作れる

\usepackage[a4paper,top=2.5cm,bottom=2.5cm,left=2.5cm,right=2.5cm,headsep=10pt]{geometry}
\usepackage[compact]{titlesec}
\titlespacing*{\section}{0pt}{3ex}{2ex}
\titlespacing*{\subsection}{0pt}{2ex}{1ex}
\titlespacing*{\subsubsection}{0pt}{1ex}{1ex}

% \usepackage{graphicx}           % tikzで使う graphicxと競合するので排除
% \usepackage[usenames]{color}
% \usepackage[usenames,dvipdfmx]{color}    % optionは同時に指定出来る。
\usepackage[dvipsnames,dvipdfmx]{xcolor}    % optionは同時に指定出来る。


\usepackage[british]{babel}
\input{colordvi.tex}

\usepackage[dvipdfm,colorlinks,pagebackref,pdfusetitle,urlcolor=blue,citecolor=MidnightBlue,linkcolor=MidnightBlue,bookmarksnumbered,plainpages=false]{hyperref}



\input{dummy.tex}

%%% tikzセッティング %%%%%%%%%%%%%%%%%%%%%%%%%%%%%%%
\usepackage[dvipdfmx]{graphicx}
\usepackage{tikz}
\usetikzlibrary{arrows,shapes,patterns,snakes,calc}
\input{arrowsnew}
\usetikzlibrary{decorations.markings}
\usetikzlibrary{positioning}
%%% end of tikz %%%%%%%%%%%%%%%%%%%%%%%%%%%%%%%%%%%%

\usepackage[hang,bf,figurename=Fig.\ , tablename=Table\ ,margin=1cm]{caption}
\renewcommand{\captionfont}{\footnotesize}

%%% listingsセッティング %%%%%%%%%%%%%%%%%%%%%%%%%%%
\usepackage{listings, jlisting}
\renewcommand{\lstlistingname}{Code}
\definecolor{mygreen}{rgb}{0,0.6,0}
\definecolor{mygray}{rgb}{0.5,0.5,0.5}
\definecolor{mymauve}{rgb}{0.58,0,0.82}

\lstset{ %
  % language=Octave,                 % the language of the code
  backgroundcolor=\color{white},   % choose the background color; you must add \usepackage{color} or \usepackage{xcolor}
  basicstyle=\footnotesize,        % the size of the fonts that are used for the code
  breakatwhitespace=false,         % sets if automatic breaks should only happen at whitespace
  breaklines=true,                 % sets automatic line breaking
  captionpos=t,                    % sets the caption-position to bottom
  commentstyle=\color{mygreen},    % comment style
  deletekeywords={...},            % if you want to delete keywords from the given language
  escapeinside={\%*}{*)},          % if you want to add LaTeX within your code
  extendedchars=true,              % lets you use non-ASCII characters; for 8-bits encodings only, does not work with UTF-8
  frame=single,                    % adds a frame around the code
  keepspaces=true,                 % keeps spaces in text, useful for keeping indentation of code (possibly needs columns=flexible)
  keywordstyle=\color{blue},       % keyword style
  morekeywords={*,...},            % if you want to add more keywords to the set
  numbers=left,                    % where to put the line-numbers; possible values are (none, left, right)
  numbersep=5pt,                   % how far the line-numbers are from the code
  numberstyle=\tiny\color{mygray}, % the style that is used for the line-numbers
  rulecolor=\color{black},         % if not set, the frame-color may be changed on line-breaks within not-black text (e.g. comments (green here))
  showspaces=false,                % show spaces everywhere adding particular underscores; it overrides 'showstringspaces'
  showstringspaces=false,          % underline spaces within strings only
  showtabs=false,                  % show tabs within strings adding particular underscores
  stepnumber=1,                    % the step between two line-numbers. If it's 1, each line will be numbered
  stringstyle=\color{mymauve},     % string literal style
  tabsize=2,                       % sets default tabsize to 2 spaces
  title=\lstname                   % show the filename of files included with \lstinputlisting; also try caption instead of title
}
%%% ned of listings セッティング %%%%%%%%%%%%%%%%%%%%%%%%%%%


\pdfstringdefDisableCommands{%
    \renewcommand*{\bm}[1]{#1}%
    % any other necessary redefinitions 
}

%%%今村セッテッティング%%%%%%%%%%%%%%%%%%%%%%%%%%%%%
\newcommand{\CC}{\mathbb{C}}
\newcommand{\ZZ}{\mathbb{Z}}
\newcommand{\RR}{\mathbb{R}}
\newcommand{\HH}{\mathbb{H}}

\newcommand{\hf}{\frac{1}{2}}
\newcommand{\tr}{{\rm tr}}
\newcommand{\ind}{{\rm ind}}
\newcommand{\ol}{\overline}
\newcommand{\ul}{\underline}
\newcommand{\up}{\uparrow}
\newcommand{\dn}{\downarrow}
\newcommand{\wt}{\widetilde}
\newcommand{\ra}{\rightarrow}
\newcommand{\wh}{\widehat}


%%%横山セッティング%%%%%%%%%%%%%%%%%%%%%%%%%%%%%%%%%
\newcommand{\NN}{\mathcal{N}\!}
\newcommand{\DD}{\mathcal{D}}
\newcommand{\UU}{U(1)}
\newcommand{\dd}{\mathrm{d}}
\renewcommand{\SS}{\mathbf{S}}
\renewcommand{\Im}{\mathrm{Im}}
\renewcommand{\Re}{\mathrm{Re}}
\renewcommand{\<}{\langle}
\renewcommand{\>}{\rangle}
\newcommand{\Tr}{{\rm Tr}}

\renewcommand{\r}{\mathrm}

\newcommand{\sign}{\mathrm{sign}}

\newcommand{\lra}{\leftrightarrow}
\newcommand{\LL}{\mathcal{L}}
\newcommand{\la}{\leftarrow}
\newcommand{\ro}{\sqrt}
\newcommand{\Ra}{\Rightarrow}
\newcommand{\Pexp}{\mathrm{Pexp}}

\newcommand{\nn}{\nonumber \\}
\newcommand{\1}{\mbox{1}\hspace{-0.25em}\mbox{l}}

%数字のみ対応
\newcommand{\Maru}[1]{\ooalign{
\ifnum#1<10 \hfil\resizebox{.9\width}{.85\height}{#1}\hfil
\else
\hfil\resizebox{.6\width}{.8\height}{#1}\hfil
\fi
\crcr
\raise.1ex\hbox{$\bigcirc$}}}

%全文字対応
\newcommand{\maru}[1]{\ooalign{
\hfil\resizebox{.8\width}{\height}{#1}\hfil
\crcr
\raise.1ex\hbox{\large$\bigcirc$}}}


\newcommand{\nord}[1]{\vcentcolon\mathrel{#1}\vcentcolon}
\providecommand{\vcentcolon}{\mathrel{\mathop{:}}}


\def\P{\mathop{\cal P}}
\def\diag{\mathop{\rm diag}}


\def\Re{\mathop{\rm Re}\nolimits}
\def\Im{\mathop{\rm Im}\nolimits}
\def\Det{\mathop{\rm Det}\nolimits}
\def\sign{\mathop{\rm sign}\nolimits}


%%% rap %%% - make two letters overlap
\newcommand{\rap}[2]
{\setbox1=\hbox{#1}%
\setbox2=\hbox to\wd1{\hss #2\hss}%
\mbox{\rlap{\box1}\box2}}

%\newcommand{\sla}[1]{\rap{$#1$}{/}}
\newcommand{\sla}[1]{\rap{$#1$}{$\backslash$}}


\def\DY#1{{\MyGreen [DY: #1]}}
\newcommand{\MyGreen}{\color [rgb]{0,0.7,0}}

\usepackage[vcentermath]{youngtab}
% \Yboxdim4pt
\newcommand{\Y}{\yng}
\newcommand{\Young}{}


%%%title def%%%%%%%%%%%%%%%%%%%%%%%%%%%%%%%%%%%%%%%%%%%%%%%

\makeatletter
\def\maketitle{
\noindent
{\Large \@title \par\vskip 2pt}
\noindent
{\large \@date \hspace{4pt} \@author}
%\\[-2pt]
%\noindent------------------------------------------------------------------------------------------
\par\vskip 1.5em
}

\author{横山 大輔}
\date{\today}
%%%本文%%%%%%%%%%%%%%%%%%%%%%%%%%%%%%%%%%%%%%%%%%%%%%%%%%%%%%%%

% \title{\centerline{Lecture 2}}
\begin{document}
% \maketitle
% \begin{abstract}

% \end{abstract}
% \tableofcontents

% \parindent=0pt

  \pagestyle{fancy}
  \renewcommand{\headrulewidth}{0.0pt}
  \rhead{}
  \lhead{}
  \cfoot{[\ \scshape\oldstylenums{\thepage}\ / %
    \scshape\oldstylenums{\pageref{lastpage}} ]}
%  \rfoot{\@author}

% \setcounter{section}{}
% \setcounter{subsection}{}
% \setcounter{subsubsection}{}

\centerline{\Large \bf  Homework 11: Due at class on Dec 8}

% \vspace{12pt}
% \DY{なにかコメント}

% \vspace{-12pt}

% \vspace{-4pt}
% \begin{itemize}
%  \setlength{\itemsep}{0pt}
%  \item Type I superstring theory.
%  \item In type I theory, we only have D$1$-, D$5$-, and D$9$-, branes.
%  \item O$9^-$-plane is needed for type I to be consistent theory.
%  \item T-duality of type I theory.
% \end{itemize}
% \vspace{-4pt}




\subsection*{Prob. 1 Electromagnetic duality}

\subsubsection*{Prob. 1.1 Differential form}


An $n$-form field is defined by
\begin{align}
 A_n = \frac{1}{n!} A_{\mu_1,\mu_2,\cdots,\mu_n} dx^{\mu_1} \wedge dx^{\mu_2} \wedge \cdots \wedge dx^{\mu_n} \ .
\end{align}
Hodge dual in a $D$-dimensional curved space is defined by
\begin{align}
 * A_n = \frac{\sqrt{|G|}}{n! (D-n)!} {\epsilon_{\mu_1,\cdots,\mu_{D-n}}}^{\mu_{D-n+1},\cdots,\mu_{D}} A_{\mu_{D-n+1},\cdots,\mu_D} dx^{\mu_1} \wedge \cdots \wedge dx^{\mu_{D-n}} \ ,
\end{align}
where $G$ is the determinant of a metric $G_{MN}$, and
$\epsilon_{\mu_1,\cdots,\mu_D}$ is the totally anti-symmetric tensor and normalized as $\epsilon_{0,1,\cdots,D-1} = 1$.
\begin{itemize}
 \item Confirm that
\begin{align}
 \frac{1}{2g^2} \int {F_2}^2 \equiv \frac{1}{2g^2} \int F_2 \wedge * F_2
 = \frac{1}{4g^2} \int \sqrt{|G|} d^Dx\ F_{\mu\nu} F^{\mu\nu} \ .
\end{align}
(For example, compare the coefficients of $F_{01}F^{01}$ term in both side.
If it is still difficult consider $D=4$.)
\end{itemize}
Comment: this is the convention used in the lecture for kinetic terms of anti-symmetric tensors in SUGRA.


\subsubsection*{Prob. 1.2 Electromagnetic duality}

Let us consider a following action
\begin{align}
 S = -\frac{1}{2g^2} \int F_{n+1} \wedge * F_{n+1} - \int f_{D-n-1} \wedge \left( F_{n+1} -d A_n \right) \ ,
\end{align}
where $F_{n+1}$ and $A_n$ are independent each other.
\begin{itemize}
 \item Consider an equation of motion for $f_{D-n-1}$ and derive the solution for the E.O.M.
       Then, what is the physical meaning of $F_{n+1}$ and $S$ with the solution ?
 \item Consider an E.O.M. for $F_{n+1}$ and derive the solution.
       Rewrite the action $S$ in terms of $f_{D-n-1}$ using the solution.
 \item Consider an equation of motion for $A_n$ with the action $S(f_{D-n-1})$ and derive the solution for the E.O.M.
       Then, what is the physical meaning of $f_{D-n-1}$ and $S(f_{D-n-1})$ with the solution ?
 \item Define $\wt S = (\mathrm{sign})S$, where you should properly choose the $(\mathrm{sign})$ so that
       the kinetic term has the usual sign. If we write $\wt S \equiv -\frac{1}{2\wt g^2} \int \wt F_{D-n-1} \wedge * \wt F_{D-n-1}$,
       what is the value of $\wt g$ in terms of $g$ ?
\end{itemize}
(Again, if you are confused with the differential form convention get back to normal convention and consider $D=4$.)




\subsection*{Prob. 2 Dirac monopole}



\subsubsection*{Prob. 2.1 Wu-Yang monopole}

Let us consider so called Dirac monopole, which is a field configuration of $\bm B$
that satisfies
\begin{align}
 \nabla \cdot \bm B = q_m \delta^3 (\bm r) \ ,
 \label{eq:1}
\end{align}
where $\delta^3 (\bm r)$ is a three-dimensional delta function and $\bm r$ is a three-dimensional space point vector.
\begin{itemize}
 \item Derive a solution of $\bm B$.
 \item Explain that $\bm B$ cannot be expressed by space components of a gauge field $\bm A$ globally.
\end{itemize}
Consider the polar coordinate $(r,\theta,\phi)$ and a following gauge field
\begin{align}
 \bm A^N &= \frac{q_m(1-\cos\theta)}{4\pi r\sin\theta} \bm e_\phi  ,  \\
 \bm A^S &= -\frac{q_m(1+\cos\theta)}{4\pi r\sin\theta} \bm e_\phi  ,
\end{align}
where $\bm A^N$ is defined in a region $r\neq 0$ and $\theta\neq \pi$,
and $\bm A^S$ is defined in a region $r\neq 0$ and $\theta\neq 0$.
(Comments: The singular lines are called Dirac string. Those solutions are called Wu-Yang monopole.
The point is that a gauge field $A_\mu$ is a section of fiber bundle and is not necessarily defined globally.)
\begin{itemize}
 \item Show that $\bm A^N-\bm A^S$ is a gauge transformation in the region $r\neq 0$ and $\theta\neq 0, \pi$.
 \item Show that both $\bm A^N$ and $\bm A^S$ lead the $\bm B$ derived above in the region where they are defined.
 \item Show that magnetic flux $\Phi$ from the monopole is $q_m$, using $\bm A^N$ and $\bm A^S$.
\end{itemize}
The statements so far are based on vector analysis.
More properly, a gauge field is a 1-form $A_1$ and magnetic flux density is a 2-form $B_2$.
Let us rewrite the statements in terms of differential forms.
\begin{itemize}
 \item Explain that Eq.~(\ref{eq:1}) can be expressed as $dF_2 = q_m \delta_3(\bm r)$,
       where $\delta_3(\bm r)$ is a 3-form delta function:
       $\delta_3(\bm r) = \delta^3(\bm r) d^3\bm r \left(= \delta(x)\delta(y)\delta(z) dx\wedge dy\wedge dz \right)$.
 \item Explain the Wu-Yang monopole can be written by
 \begin{align}
  A_N = \frac{q_m}{4\pi}(1-\cos\theta) d\phi \ , \qquad A_S = -\frac{q_m}{4\pi}(1+\cos\theta) d\phi \ .
 \end{align}
 \item Show that $A_N-A_S$ is a gauge transformation.
 \item Show that magnetic flux $\Phi$ from the monopole is $q_m$, using $A_N$ and $A_S$ in terms of differential forms.
\end{itemize}


\subsubsection*{Prob. 2.2 Dirac quantization condition}
The previous argument in differential forms can be easily generalized to higher dimension.
Let us consider a $(p+1)$-form gauge field $C_{p+1}$, which satisfies following equation of motion.
\begin{align}
 \frac{1}{2\kappa^2} dG_{p+2} = q_m \delta_{p+3}(M_{D-p-3}) \ ,
\end{align}
where $G_{p+2}$ is a field strength of $C_{p+1}$, $\delta_{p+3}(M_{D-p-3})$ is a $(p+3)$-form delta function,
and $M_{D-p-3}$ is a world-sheet of an object that is magnetically coupled to $C_{p+1}$ (Consider $D=4$ and $p=0$, which is the monopole case).
\begin{itemize}
 \item (Optional) Explain the origin of $2\kappa^2$.
 \item Show that $G_{p+2}$ integrated over a sphere enclosing the magnetic object is $2\kappa^2 q_m$.
\end{itemize}


Let us consider an object that is electrically coupled to $C_{p+1}$, whose coupling is
\begin{align}
 S_E = q_e \int_{E_{p+1}} C_{p+1} = q_e \int_{\wh E_{p+2}} G_{p+2} \equiv S_E(\wh E_{p+2}) \ ,
\end{align}
where $E_{p+1}$ is a world-sheet that starts from $t=-\infty$ and ends at $t=\infty$
or starts at $\bm x_0$ and ends at $\bm x_0$ (closed path), and $\wh E_{p+2}$ is a manifold whose boundary is $E_{p+1}$.
\begin{itemize}
 \item Since a choice of $\wh E_{p+2}$ is arbitrary one can consider a deformation of $\delta \wh E_{p+2} = \wh E_{p+2}^N -\wh E_{p+2}^S$,
       and then, $e^{iS_E(\delta\wh E_{p+2})} =1$. Show that this leads Dirac quantization condition.
\end{itemize}
% Classically the world-sheet is fixed, however, quantum mechanically any world-sheet has to be considered.
% Therefore, if there exist the magnetic object we have to consider two world-sheet $E_{p+1}^N$ and $E_{p+1}^S$
% such that $E_{p+1}^N -E_{p+1}^S$ enclose the magnetic object, or a closed path enclosing the magnetic object.
% \begin{itemize}
%  \item Show that $e^{iS_E}$
% \end{itemize}
% \begin{itemize}
%  \item (Optional) Show that gauge invariance of the action assure that a deformation of the world-sheet $E_{p+1}$ does not change the action
%        up to $2\pi n$ ($n\in \ZZ$).
%  \item Assume that a deformation of $E_{p+1}$ enclose the magnetic object and the deformation does not change the action,
%        and then, show that it leads Dirac quantization condition $2\kappa^2 q_e q_m \in 2\pi \ZZ$.
% \end{itemize}





\subsection*{Prob. 3 $SL(2,\RR)$ invariance of type IIB SUGRA}


\begin{itemize}
 \item Show that the $SL(2,\RR)$ invariance of the type IIB SUGRA.
\end{itemize}




\subsection*{Prob. 4 Kaluza-Klein theory}

Let us consider a $D+1$-dimensional real scalar Kaluza-Klein theory:
\begin{align}
 \LL^{(D+1)} = -\frac{1}{2} G^{MN} \partial_M \phi \partial_N \phi \ .
\end{align}
The metric is given by
\begin{align}
 ds^2 = G_{MN} dx^M dx^N = g_{\mu\nu} dx^\mu dx^\nu + e^{2\sigma} \left( dy +A_\mu dx^\mu \right)^2 \ ,
 \label{eq:2}
\end{align}
where $M,N$ run from $0$ to $D$, $\mu,\nu$ run from $0$ to $D-1$,
and $y \equiv x^D$, which is compactified to $S^1$ with radius $r$.
\begin{itemize}
 \item Rewrite the Lagrangian in terms of $\mu,\nu$, and $y$, rather than $M,N$.
 \item Calculate the determinant of $G_{MN}$ (i.e. $G \equiv \det G_{MN}$) and express it in terms of $g \equiv \det g_{\mu\nu}$.
 \item Assume that the fields $(g_{\mu\nu}, A_\mu, \sigma)$ are independent of $y$, except $\phi$,
       which can be expanded as $\phi(\bm x,y) = \sum_{n \in \ZZ} \phi_n(\bm x)e^{iny/r},\ (\phi_n^*(\bm x)=\phi_{-n}(\bm x))$.
       Write down the effective Lagrangian of $D$-dimensional theory:
 \begin{align}
  \sqrt{-g}\LL^{(D)} = \int dy \sqrt{-G} \LL^{(D+1)} \ .
 \end{align}
 \item Derive the masses of the fields $\phi_n$, and their coupling constants to the KK-gauge field $A_\mu$.
\end{itemize}


Let us consider a gravi-dilaton KK-theory:
\begin{align}
 S_{EH}^{(D+1)} \simeq \frac{1}{2\kappa^2} \int d^{D+1}x \sqrt{-G} e^{-2\Phi} \Big(R +\cdots \Big) \ ,
\end{align}
which, with the same metric Eq.~(\ref{eq:2}), reduces to
\begin{align}
 S_{EH}^{(D)} \simeq \frac{2\pi L}{2\kappa^2} \int d^{D}x \sqrt{-g} e^{-2\Phi} \Big(R +\cdots \Big) \ ,
\end{align}
where we defined $L = r e^{-\sigma}$, and $\cdots$ includes the dilaton kinetic term and other terms that are not important here.
\begin{itemize}
 \item Define an effective coupling $\frac{1}{2\kappa_\mathrm{eff}^2} = \frac{2\pi L}{2\kappa^2}e^{-2\langle\Phi\rangle}$,
       and compare it with that of T-dual theory.
       $\frac{1}{2\kappa_\mathrm{eff}^2}$ should be the same after the T-dual.
       Derive the value of $\langle\wt \Phi\rangle$ in terms of $\langle\Phi\rangle$,
       where $\langle\wt \Phi\rangle$ is a dilaton vev of the T-dual theory.
\end{itemize}



% \begin{thebibliography}{CDLOGP91}

%  \bibitem[Pol98]{Pol98}
%                 J. Polchinski.
%                 String theory. Vol. 1.
%                 Cambridge University Press, 1998.

%  \bibitem[BP09]{Blumenhagen:2009zz}
%                R.~Blumenhagen and E.~Plauschinn.
%                \newblock {Introduction to conformal field theory}.
%                \newblock {\em Lect. Notes Phys.}, 779:1--256, 2009.

% \end{thebibliography}


% \bibliography{string-lecture}
% \bibliographystyle{halpha}
% \bibliographystyle{JHEP}

% \begin{thebibliography}{CDLOGP91}

% %\cite{AlvarezGaume:1981hn}
% \bibitem[AFM81]{AlvarezGaume:1981hn}
%   L.~Alvarez-Gaume, D.~Z.~Freedman and S.~Mukhi,
%   ``The Background Field Method and the Ultraviolet Structure of the Supersymmetric Nonlinear Sigma Model,''
%   Annals Phys.\  {\bf 134}, 85 (1981).
%   %doi:10.1016/0003-4916(81)90006-3
%   %%CITATION = doi:10.1016/0003-4916(81)90006-3;%%
%   %525 citations counted in INSPIRE as of 09 Oct 2017



% \bibitem[CT88]{Callan:1988xx}
%  Curt Callan and Lárus Thorlacius.
%  \textit{SIGMA MODELS AND STRING THEORY.}
%  TASI Lecture, 1988.
%  (The link below is a direct link to the pdf file of 45MB )
%  \href{http://www.damtp.cam.ac.uk/user/tong/string/sigma.pdf}
%  {http://www.damtp.cam.ac.uk/user/tong/string/sigma.pdf}.




% \end{thebibliography}


\label{lastpage}

% \begin{tikzpicture}[>=stealth,scale=1]
%  \draw[->] (0,0)--(1,0);
%  \draw[latex-stealth] (0,0.5)--(1,0.5);
%  \draw[latex-stealthnew,arrowhead=2mm] (0,1)--(1,1);
% \end{tikzpicture}

% \begin{figure}[htb]
% \centerline{\includegraphics[width=250pt]{.eps}}
% \caption{}
% \label{.eps}
% \end{figure}



% \begin{table}[htbp]
%  \begin{center}
%   \caption{}
%   \vspace{4pt}
%   \label{table:001}
% \begin{tabular}{|c|c|c|c|c|}
% \hline
% \hline
%   Category & Sector & $(h_A,h_B,h_T)$ & Mirror theory & ABJM model \\
% \hline
%  1 & & \parbox{40pt}{$(0,0,0)$ $(1,1,1)$} & $1.11906$ & $1.13290$ \\
% \hline
%  2 & & \parbox{40pt}{$(0,0,1)$ $(1,1,0)$} & $-0.10861$ & $-0.10861$ \\
% \hline
%  \multirow{2}{*}{\vspace{-15pt}3} & 3-1 & \parbox{40pt}{$(0,1,0)$ $(1,0,1)$} & $0.176777$ &
%  \multirow{2}{*}{\vspace{-15pt}$0.176577$} \\
% \cline{2-4}
%  & 3-2 & \parbox{40pt}{$(1,0,0)$ $(0,1,1)$} & $0.176777$ & \\
% \hline
% \hline
% \end{tabular}
% \end{center}
% \end{table}


% \begin{thebibliography}{99}

% % \cite{Imamura:2012rq}
% \bibitem{Imamura:2012rq}
%   Y.~Imamura and D.~Yokoyama,
%   %``S^3/Z_n partition function and dualities,''
%   JHEP {\bf 1211}, 122 (2012)
%   [arXiv:1208.1404 [hep-th]].
%   %%CITATION = ARXIV:1208.1404;%%

 % \bibitem{fnorio:legendre}
 %         fnorio
 %         ``ルジャンドル変換とは何か''
 %         \url{http://fnorio.com/0146Legendre_transformation/Legendre_transformation.html}


 % \bibitem{EMAN:dynamics}
 %         EMAN物理学
 %         ``ハミルトニアン''
 %         \url{http://eman-physics.net/analytic/hamilton.html}


 % \bibitem{Wiki:legendre}
 %         Wiki
 %         ``ルジャンドル変換''
 %         \url{https://ja.wikipedia.org/wiki/ルジャンドル変換}

 % \bibitem{mathtrain:legendre}
 %         高校数学の美しい物語
 %         ``ルジャンドル変換の意味と具体例''
 %         \url{http://mathtrain.jp/legendrehenkan}

% \end{thebibliography}


% \bibliography{dd}

\end{document}
