\documentclass[12pt,a4paper]{article}
%\usepackage{hyperref} % Use the Charter font for the document text
%\usepackage[UTF8]{ctex}
\usepackage{jheppub}

\usepackage{amsfonts,amssymb,amsmath}
\usepackage{mathtools}
\usepackage{tikz-cd}
\usepackage{tikz}
\usepackage{alltt}
\usepackage{amsfonts}
\usepackage{amsmath}
\usepackage{amssymb}
\usepackage{amsthm}
\usepackage{booktabs}
\usepackage{caption}
\usepackage{enumitem}
\usepackage{fancyhdr}
\usepackage{graphicx}
\usepackage{mathdots}
\usepackage{mathtools}
\usepackage{microtype}
\usepackage{multirow}
\usepackage{pdflscape}
\usepackage{pgfplots}
\usepackage{siunitx}
\usepackage{slashed}
\usepackage{tabularx}
\usepackage{tikz}
\usepackage{tkz-euclide}
\usepackage[normalem]{ulem}
\usepackage[all]{xy}
\usepackage{imakeidx}
\usepackage{gensymb}
\usepackage{simplewick}
\usepackage{feynmp-auto}
\usepackage{wrapfig}



%%%%%%%  Greek letters %%%%%%%%%%%%%%%%%%
\def\a{\alpha}
\def\b{\beta}
\def\c{\gamma} \def\g{\gamma}
\def\d{\delta}
\def\e{\epsilon}
\def\f{\phi}
\def\vf{\varphi}  \def\tvf{\tilde{\varphi}}
\def\vp{\varphi}
\def\h{\eta}
\def\i{\iota}
\def\j{\psi}
\def\k{\kappa}
\def\m{\mu}
\def\n{\nu}
\def\o{\omega}  \def\w{\omega}
\def\q{\theta}  \def\th{\theta}
\def\r{\rho}
\def\s{\sigma}
\def\t{\tau}
\def\u{\upsilon}
\def\x{\xi}
\def\z{\zeta}

\def\A{\Alpha}
\def\B{\Beta}
\def\G{\Gamma}
\def\D{\Delta}
\def\E{\Epsilon}
\def\F{Phi}
\def\h{\eta}
\def\I{\Iota}
\def\J{Psi}
\def\K{\Kappa}
\def\L{\lambdabda}
\def\M{\Mu}
\def\N{\Nu}
\def\O{\Omega}  \def\w{\omega}
\def\Q{\Theta}  \def\Th{\Theta}
\def\R{\Rho}
\def\Si{\Sigma}
\def\T{\Tau}
\def\Up{\Upsilon}
\def\X{\Xi}
\def\Z{\Zeta}








%%%%%%%%%%%% math fonts %%%%%%%%%%%%%%%%%%%%%%%%%%%%%%%%%%%%%
%
%---------- mathbb font --------------------------------
%

\newcommand{\bA}{\ensuremath{\mathbb{A}}}
\newcommand{\bB}{\ensuremath{\mathbb{B}}}
\newcommand{\bC}{\ensuremath{\mathbb{C}}}
\newcommand{\bD}{\ensuremath{\mathbb{D}}}
\newcommand{\bE}{\ensuremath{\mathbb{E}}}
\newcommand{\bF}{\ensuremath{\mathbb{F}}}
\newcommand{\bG}{\ensuremath{\mathbb{G}}}
\newcommand{\bH}{\ensuremath{\mathbb{H}}}
\newcommand{\bI}{\ensuremath{\mathbb{I}}}
\newcommand{\bJ}{\ensuremath{\mathbb{J}}}
\newcommand{\bK}{\ensuremath{\mathbb{K}}}
\newcommand{\bL}{\ensuremath{\mathbb{L}}}
\newcommand{\bM}{\ensuremath{\mathbb{M}}}
\newcommand{\bN}{\ensuremath{\mathbb{N}}}
\newcommand{\bO}{\ensuremath{\mathbb{O}}}
\newcommand{\bP}{\ensuremath{\mathbb{P}}}
\newcommand{\bQ}{\ensuremath{\mathbb{Q}}}
\newcommand{\bR}{\ensuremath{\mathbb{R}}}
\newcommand{\bS}{\ensuremath{\mathbb{S}}}
\newcommand{\bT}{\ensuremath{\mathbb{T}}}
\newcommand{\bU}{\ensuremath{\mathbb{U}}}
\newcommand{\bV}{\ensuremath{\mathbb{V}}}
\newcommand{\bW}{\ensuremath{\mathbb{W}}}
\newcommand{\bX}{\ensuremath{\mathbb{X}}}
\newcommand{\bY}{\ensuremath{\mathbb{Y}}}
\newcommand{\bZ}{\ensuremath{\mathbb{Z}}}



%
%---------- mathbf font --------------------------------
%


\newcommand{\bfA}{\ensuremath{\mathbf{A}}}
\newcommand{\bfB}{\ensuremath{\mathbf{B}}}
\newcommand{\bfC}{\ensuremath{\mathbf{C}}}
\newcommand{\bfD}{\ensuremath{\mathbf{D}}}
\newcommand{\bfE}{\ensuremath{\mathbf{E}}}
\newcommand{\bfF}{\ensuremath{\mathbf{F}}}
\newcommand{\bfG}{\ensuremath{\mathbf{G}}}
\newcommand{\bfH}{\ensuremath{\mathbf{H}}}
\newcommand{\bfI}{\ensuremath{\mathbf{I}}}
\newcommand{\bfJ}{\ensuremath{\mathbf{J}}}
\newcommand{\bfK}{\ensuremath{\mathbf{K}}}
\newcommand{\bfL}{\ensuremath{\mathbf{L}}}
\newcommand{\bfM}{\ensuremath{\mathbf{M}}}
\newcommand{\bfN}{\ensuremath{\mathbf{N}}}
\newcommand{\bfO}{\ensuremath{\mathbf{O}}}
\newcommand{\bfP}{\ensuremath{\mathbf{P}}}
\newcommand{\bfQ}{\ensuremath{\mathbf{Q}}}
\newcommand{\bfR}{\ensuremath{\mathbf{R}}}
\newcommand{\bfS}{\ensuremath{\mathbf{S}}}
\newcommand{\bfT}{\ensuremath{\mathbf{T}}}
\newcommand{\bfU}{\ensuremath{\mathbf{U}}}
\newcommand{\bfV}{\ensuremath{\mathbf{V}}}
\newcommand{\bfW}{\ensuremath{\mathbf{W}}}
\newcommand{\bfX}{\ensuremath{\mathbf{X}}}
\newcommand{\bfY}{\ensuremath{\mathbf{Y}}}
\newcommand{\bfZ}{\ensuremath{\mathbf{Z}}}
\newcommand{\bfa}{\ensuremath{\mathbf{a}}}
\newcommand{\bfb}{\ensuremath{\mathbf{b}}}
\newcommand{\bfc}{\ensuremath{\mathbf{c}}}
\newcommand{\bfd}{\ensuremath{\mathbf{d}}}
\newcommand{\bfe}{\ensuremath{\mathbf{e}}}
\newcommand{\bff}{\ensuremath{\mathbf{f}}}
\newcommand{\bfg}{\ensuremath{\mathbf{g}}}
\newcommand{\bfh}{\ensuremath{\mathbf{h}}}
\newcommand{\bfi}{\ensuremath{\mathbf{i}}}
\newcommand{\bfj}{\ensuremath{\mathbf{j}}}
\newcommand{\bfk}{\ensuremath{\mathbf{k}}}
\newcommand{\bfl}{\ensuremath{\mathbf{l}}}
\newcommand{\bfm}{\ensuremath{\mathbf{m}}}
\newcommand{\bfn}{\ensuremath{\mathbf{n}}}
\newcommand{\bfo}{\ensuremath{\mathbf{o}}}
\newcommand{\bfp}{\ensuremath{\mathbf{p}}}
\newcommand{\bfq}{\ensuremath{\mathbf{q}}}
\newcommand{\bfr}{\ensuremath{\mathbf{r}}}
\newcommand{\bfs}{\ensuremath{\mathbf{s}}}
\newcommand{\bft}{\ensuremath{\mathbf{t}}}
\newcommand{\bfu}{\ensuremath{\mathbf{u}}}
\newcommand{\bfv}{\ensuremath{\mathbf{v}}}
\newcommand{\bfw}{\ensuremath{\mathbf{w}}}
\newcommand{\bfx}{\ensuremath{\mathbf{x}}}
\newcommand{\bfy}{\ensuremath{\mathbf{y}}}
\newcommand{\bfz}{\ensuremath{\mathbf{z}}}



%
%\parskip=1em
%\parindent=0.3in
%\setlength\oddsidemargin{0.5in} \setlength\evensidemargin{0.5in}
%\setlength\textwidth{5.5in}
%
%\hfuzz6pt % Don't bother to report over-full boxes if over-edge is < 6pt
%
%\newlength{\defbaselineskip}
%\setlength{\defbaselineskip}{\baselineskip}
%\newcommand{\setlinespacing}[1]%
%           {\setlength{\baselineskip}{#1 \defbaselineskip}}
%\newcommand{\doublespacing}{\setlength{\baselineskip}%
%                           {2.0 \defbaselineskip}}
%\newcommand{\singlespacing}{\setlength{\baselineskip}{\defbaselineskip}}
%
%\newcommand{\properpagestyle}{\pagestyle{myheadings}\markboth{}{}\markright{}}


%---------- mathscript font -----------------------------
%

\newcommand{\scA}{\ensuremath{\mathscr{A}}}
\newcommand{\scB}{\ensuremath{\mathscr{B}}}
\newcommand{\scC}{\ensuremath{\mathscr{C}}}
\newcommand{\scD}{\ensuremath{\mathscr{D}}}
\newcommand{\scE}{\ensuremath{\mathscr{E}}}
\newcommand{\scF}{\ensuremath{\mathscr{F}}}
\newcommand{\scG}{\ensuremath{\mathscr{G}}}
\newcommand{\scH}{\ensuremath{\mathscr{H}}}
\newcommand{\scI}{\ensuremath{\mathscr{I}}}
\newcommand{\scJ}{\ensuremath{\mathscr{J}}}
\newcommand{\scK}{\ensuremath{\mathscr{K}}}
\newcommand{\scL}{\ensuremath{\mathscr{L}}}
\newcommand{\scM}{\ensuremath{\mathscr{M}}}
\newcommand{\scN}{\ensuremath{\mathscr{N}}}
\newcommand{\scO}{\ensuremath{\mathscr{O}}}
\newcommand{\scP}{\ensuremath{\mathscr{P}}}
\newcommand{\scQ}{\ensuremath{\mathscr{Q}}}
\newcommand{\scR}{\ensuremath{\mathscr{R}}}
\newcommand{\scS}{\ensuremath{\mathscr{S}}}
\newcommand{\scT}{\ensuremath{\mathscr{T}}}
\newcommand{\scU}{\ensuremath{\mathscr{U}}}
\newcommand{\scV}{\ensuremath{\mathscr{V}}}
\newcommand{\scW}{\ensuremath{\mathscr{W}}}
\newcommand{\scX}{\ensuremath{\mathscr{X}}}
\newcommand{\scY}{\ensuremath{\mathscr{Y}}}
\newcommand{\scZ}{\ensuremath{\mathscr{Z}}}
\newcommand{\scAH}{\ensuremath{\mathscr{A}\!\!\scH}}

%
%---------- mathfrak font -----------------------------
%

\newcommand{\frakA}{\ensuremath{\mathfrak{A}}}
\newcommand{\frakB}{\ensuremath{\mathfrak{B}}}
\newcommand{\frakC}{\ensuremath{\mathfrak{C}}}
\newcommand{\frakD}{\ensuremath{\mathfrak{D}}}
\newcommand{\frakE}{\ensuremath{\mathfrak{E}}}
\newcommand{\frakF}{\ensuremath{\mathfrak{F}}}
\newcommand{\frakG}{\ensuremath{\mathfrak{G}}}
\newcommand{\frakH}{\ensuremath{\mathfrak{H}}}
\newcommand{\frakI}{\ensuremath{\mathfrak{I}}}
\newcommand{\frakJ}{\ensuremath{\mathfrak{J}}}
\newcommand{\frakK}{\ensuremath{\mathfrak{K}}}
\newcommand{\frakL}{\ensuremath{\mathfrak{L}}}
\newcommand{\frakM}{\ensuremath{\mathfrak{M}}}
\newcommand{\frakN}{\ensuremath{\mathfrak{N}}}
\newcommand{\frakO}{\ensuremath{\mathfrak{O}}}
\newcommand{\frakP}{\ensuremath{\mathfrak{P}}}
\newcommand{\frakQ}{\ensuremath{\mathfrak{Q}}}
\newcommand{\frakR}{\ensuremath{\mathfrak{R}}}
\newcommand{\frakS}{\ensuremath{\mathfrak{S}}}
\newcommand{\frakT}{\ensuremath{\mathfrak{T}}}
\newcommand{\frakU}{\ensuremath{\mathfrak{U}}}
\newcommand{\frakV}{\ensuremath{\mathfrak{V}}}
\newcommand{\frakW}{\ensuremath{\mathfrak{W}}}
\newcommand{\frakX}{\ensuremath{\mathfrak{X}}}
\newcommand{\frakY}{\ensuremath{\mathfrak{Y}}}
\newcommand{\frakZ}{\ensuremath{\mathfrak{Z}}}
\newcommand{\fraka}{\ensuremath{\mathfrak{a}}}
\newcommand{\frakb}{\ensuremath{\mathfrak{b}}}
\newcommand{\frakc}{\ensuremath{\mathfrak{c}}}
\newcommand{\frakd}{\ensuremath{\mathfrak{d}}}
\newcommand{\frake}{\ensuremath{\mathfrak{e}}}
\newcommand{\frakf}{\ensuremath{\mathfrak{f}}}
\newcommand{\frakg}{\ensuremath{\mathfrak{g}}}
\newcommand{\frakh}{\ensuremath{\mathfrak{h}}}
\newcommand{\fraki}{\ensuremath{\mathfrak{i}}}
\newcommand{\frakj}{\ensuremath{\mathfrak{j}}}
\newcommand{\frakk}{\ensuremath{\mathfrak{k}}}
\newcommand{\frakl}{\ensuremath{\mathfrak{l}}}
\newcommand{\frakm}{\ensuremath{\mathfrak{m}}}
\newcommand{\frakn}{\ensuremath{\mathfrak{n}}}
\newcommand{\frako}{\ensuremath{\mathfrak{o}}}
\newcommand{\frakp}{\ensuremath{\mathfrak{p}}}
\newcommand{\frakq}{\ensuremath{\mathfrak{q}}}
\newcommand{\frakr}{\ensuremath{\mathfrak{r}}}
\newcommand{\fraks}{\ensuremath{\mathfrak{s}}}
\newcommand{\frakt}{\ensuremath{\mathfrak{t}}}
\newcommand{\fraku}{\ensuremath{\mathfrak{u}}}
\newcommand{\frakv}{\ensuremath{\mathfrak{v}}}
\newcommand{\frakw}{\ensuremath{\mathfrak{w}}}
\newcommand{\frakx}{\ensuremath{\mathfrak{x}}}
\newcommand{\fraky}{\ensuremath{\mathfrak{y}}}
\newcommand{\frakz}{\ensuremath{\mathfrak{z}}}
\newcommand{\fraksl}{\ensuremath{\mathfrak{sl}}}
\newcommand{\frakso}{\ensuremath{\mathfrak{so}}}
\newcommand{\fraksp}{\ensuremath{\mathfrak{sp}}}

%%%%%%%%%%%%  Calligraphic, Roman and Maths integers %%%%%%%%%%%%%%%%%%

\newcommand{\cA}{\mathcal{A}}
\newcommand{\cB}{\mathcal{B}}
\newcommand{\cC}{\mathcal{C}}
\newcommand{\cD}{\mathcal{D}}
\newcommand{\cE}{\mathcal{E}}
\newcommand{\cF}{\mathcal{F}}
\newcommand{\cG}{\mathcal{G}}
\newcommand{\cH}{\mathcal{H}}
\newcommand{\cI}{\mathcal{I}}
\newcommand{\cJ}{\mathcal{J}}
\newcommand{\cK}{\mathcal{K}}
\newcommand{\cL}{\mathcal{L}}
\newcommand{\cM}{\mathcal{M}}
\newcommand{\cN}{\mathcal{N}}
\newcommand{\cO}{\mathcal{O}}
\newcommand{\cQ}{\mathcal{Q}}
\newcommand{\cS}{\mathcal{S}}
\newcommand{\cX}{\mathcal{X}}
\newcommand{\cY}{\mathcal{Y}}
\newcommand{\cW}{\mathcal{W}}
\newcommand{\cR}{\mathcal{R}}
\newcommand{\cT}{\mathcal{T}}
\newcommand{\cZ}{\mathcal{Z}}

%%%%%%%%%%%%%%%%%%%%%%%%%%%%%%%%%%%%%%%%%%%%%%%%%%%%%%%%%%%%%%%%
\newcommand{\SU}{\mathrm{SU}}
\newcommand{\SO}{\mathrm{SO}}
\newcommand{\SL}{\mathrm{SL}}
\newcommand{\Sp}{\mathrm{Sp}}
\newcommand{\su}{\mathrm{su}}
\newcommand{\so}{\mathrm{so}}
\newcommand{\spl}{\mathrm{sp}}
\newcommand{\gl}{\mathrm{gl}}
\newcommand{\sll}{\mathrm{sl}}
\newcommand{\U}{\mathrm{U}}
\newcommand{\ul}{\mathrm{u}}
\newcommand{\Spin}{\mathrm{Spin}}
\newcommand{\Pin}{\mathrm{Pin}}
%%%%%%%%%%%%%%%%%%%%%%%%%%%%%%%%%%%%%%%%%%%%%%%%%%%%%%%%%%%%%%%%
\renewcommand{\Im}{{\rm Im}}
\renewcommand{\Re}{{\rm Re}}
\newcommand{\Tr}{\mbox{Tr}}
\newcommand{\Pf}{\mbox{Pf}}
\newcommand{\sgn}{\mbox{sgn}}
\newcommand{\Vir}{{\rm Vir}}
\newcommand{\Li}{{\rm Li}}

\def\tl{\tilde}
\def\wt{\widetilde}
\def\wh{\widehat}
\def\bar{\overline}
\newcommand\bz{{\bar{z}}}



\newtheorem{lemma}{Lemma}[section]
\newtheorem{conjecture}[lemma]{Conjecture} 
\newtheorem{corollary}[lemma]{Corollary} 
\newtheorem{theorem}[lemma]{Theorem} 
\newtheorem{definition}[lemma]{Definition} 
\newtheorem{question}[lemma]{Question} 
\newtheorem{proposition}[lemma]{Proposition} 





\def\bea{\begin{align}}
\def\eea{\end{align}}
\def\be{\begin{equation}}
\def\ee{\end{equation}}
\def\ba{\begin{align}}
\def\ea{\end{align}}
%%% Yokoyama def %%%
\newcommand{\ol}{\overline}
\newcommand{\nord}[1]{\vcentcolon\mathrel{#1}\vcentcolon}
\providecommand{\vcentcolon}{\mathrel{\mathop{:}}}
%%%%%%%%%%%%%%%%%%%%


\begin{document}\thispagestyle{empty}

\centerline{\Large \bf Homework 10 (Due at class on Dec 1)}


\section{ Fermionization}

\subsection{}
At the special radius $R =\sqrt{\frac{\a'}{2}}$ of the circle compactification, one can redefine a bosonic field $H(z):=\sqrt{\frac{2}{\a'}}X(z)$ so that the OPE are given by
$$H(z)H(0) \sim - \ln z~.$$
Let us also consider two Majorana-Weyl fermions $\psi^1, \psi^2$ with OPE 
$$\psi^i(z)\psi^j(0) \sim \frac{\d^{ij}}{z}~.$$
We can define the complex fermion $$\psi(z) = 2^{-1/2} (\psi^1(z) +i\psi^2(z))\,,\qquad \overline\psi(z) = 2^{-1/2} (\psi^1(z) -i\psi^2(z))~.$$
Show the equivalence of operators in boson and fermion
$$:e^{iH} :\cong\psi\,,\quad :e^{-iH} :\cong\overline \psi\,,\quad i\partial H \cong \psi\overline \psi\,,\quad T_H \cong T_\psi~,$$
by calculating the OPEs of operators in both theories and comparing. 





\subsection{}
At the special radius $R =\sqrt{\frac{\a'}{2}}$, show that the torus partition function can be written as
$$
Z^{25}=\frac{1}{2|\eta(q)|^2}\left[\Big| \sum_n q^{n^2/2} \Big|^2+\Big| \sum_n (-1)^n q^{n^2/2} \Big|^2+\Big|\sum_n q^{\frac12(n + 1 / 2)^2}\Big|^2\right]\,.
$$
    Verify that this is the torus partition function of a free complex fermion with all the possible boundary conditions (AA, AP, PA, PP).  This is called the \textbf{diagonal modular invariant} partition function.


\section{Check the central charge}
The world-sheet action of Heterotic string theory is given by
\begin{align}
 S^{\textrm{m}} &= \frac{1}{4\pi} \int d^2 z\ \Big( \frac{2}{\alpha'} \partial X^\mu  \overline\partial X_\mu+\psi^\mu\overline\partial\psi_\mu+\wt \lambda^A\partial\wt\lambda_A\Big)\cr
S^{\textrm{gh}}&=\frac{1}{2\pi}\int d^2z \ (b\overline \partial c+\bar b \partial \bar c+\beta\overline \partial \g)\nonumber
\end{align}
where $\mu$ are 10-dimensional indices and the right-moving sector is supersymmetric. In the lecture, I have explained that if there are 32 left-moving fermions $\wt\lambda^A$, there is no Weyl-anomaly $c^{\textrm{tot}}=0$. Explain in detail the value of each contribution $c^X,c^\psi,c^{bc},c^{\beta\gamma},c^{\lambda}$ in the lecture note.





\section{Toroidal compactifications}

Let us consider the toroidal compactification in the presence of $B$-field
$$
S=-\frac{1}{4\pi\a'}\int d^2\sigma(\d_{IJ}\eta^{\a\b}+ B_{IJ}\e^{\a\b})\partial_\a X^I\partial_\b X^J~,
$$
where $\e^{01} =-1$ and $\eta^{\a\b}=\textrm{diag}(-1,1)$. The bosonic field is quantized as in the lecture note 
\begin{align}
X^I(z)&=x^I-i\sqrt{\a'\over2}p_R^I
\ln z+i\sqrt{\a'\over2}\sum_{m\neq0}{\alpha^I_m\over mz^m}, \nonumber\\
\overline X^I(\bz)&={\wt x}^I-i\sqrt{\a'\over2}{p^I_L}
\ln \bz+i\sqrt{\a'\over2}\sum_{m\neq0}{{\wt\alpha}^I_m\over m\bz^m}.\nonumber
\end{align}
where the bosonic field is periodically identified on the lattice $W^I\in \Lambda$
\be\label{bc}
X ^I( \s+2\pi,\t)=X ^I( \s,\t)+2\pi W^I~.
\ee




\subsection{} Show that  the center of mass momentum
$$
\pi_I:=\int_0^{2\pi} d\sigma ~ \Pi_I~,\qquad  \Pi_I=\frac{\d S}{\d \dot{X}^I}~,
$$
together with \eqref{bc} imply that
$$
(p_I)_{L,R} :=\d_{IJ}p^J_{L,R} = \sqrt{\frac{\a'}{2}}\Big( \pi_I  \pm \frac1{\a'}(\d_{IJ} \mp B_{IJ})W^J \Big)~.
$$
Here left-modes get the upper sign and right-modes get the lower sign whenever you find the notation $\pm$ and $\mp$.



\subsection{}  It is $\pi_I$ that generates translations and it must therefore lie on the lattice $\Lambda^*_D$, $\pi_I=e^{*i}_Im_i$ for $m_i\in\bZ$. Writing $\mathbf{g}=g_{ij} = e^I_i \d_{IJ }e^J_j $, $\mathbf{b} = b_{ij} = e^I_i B_{IJ} e^J_j$, 
show that the mass formula and the level-matching condition  for closed strings are modified to (a matrix notation is employed in the expression below)
\begin{gather}\label{mass-level}
\a' M^2={\a'}\bfm^T\bfg^{-1}\bfm+\frac{1}{\a'}\bfn^T(\bfg-\bfb\bfg^{-1}\bfb)\bfn+2\bfn^T\bfb\bfg^{-1}\bfm+2(N +\wt N -2)\cr
N-\wt N=\boldsymbol{\pi}\cdot \bfW=\bfm^T \bfn~.
\end{gather}


\subsection{}
Use \eqref{mass-level} to show that the spectrum is invariant under the map
$$
\bfm\leftrightarrow\bfn~,\quad \a' \bfg^{-1}\leftrightarrow \frac1{\a'} (\bfg-\bfb\bfg^{-1}\bfb)~, \quad  \bfb\bfg^{-1}\leftrightarrow -\bfg^{-1} \bfb~.
$$
The second and the third are indeed equivalent to
$$
\frac{1}{\a'}(\bfg+\bfb)\leftrightarrow \a'(\bfg+\bfb)^{-1}~,
$$
and this is the T-duality in the presence of $B$-field.


\subsection{} A non-trivial instructive example is the 2-torus $T^2= \bR^2/\Lambda$ where $\bfe_1,\bfe_2$ are the two generators of $\Lambda$ and its metric is $g_{ij}=\bfe_i\cdot \bfe_j$.
The torus has one K\"ahler modulus $\sqrt{\det g_{ij}}$ and one complex structure modulus
$$
\tau=\frac{|\bfe_2|}{|\bfe_1|}e^{i \,\textrm{angle}(\bfe_1,\bfe_2)}=\frac{g_{12}+i \sqrt{\det g_{ij} }}{g_{11}}=\tau_1+i\tau_2~.
$$
In the presence of $B$-field, the K\"ahler modulus  is complexified
$$
\omega=\frac{1}{\a'}(B+i \sqrt{\det g_{ij}})=\omega_1+i\omega_2~.
$$
This leads to the expression
$$
g_{ij}=\a'\frac{\omega_2}{\tau_2}\begin{pmatrix} 1&\tau_1\\ \tau_1 & |\tau|^2\end{pmatrix}~.
$$
Using these expressions, show that the momentum contribution to the mass formula is given by
\begin{align}
\bfp_L^2&=\frac{1}{2\omega_2\tau_2}|m_2-\t m_1+\overline{\omega}(n^1 + \t n^2)|^2  \cr
\bfp_R^2&=\frac{1}{2\omega_2\tau_2}|m_2-\t m_1+{\omega}(n^1 + \t n^2)|^2~.\nonumber
\end{align}
Show that, if we transform the momentum and winding numbers appropriately, the string spectrum is then invariant under the following two $\SL(2,\bZ)$ transformations


\noindent $\bullet$ T-dualities
$$
\omega\to\frac{a\omega+b}{c\omega+d}
$$
$\bullet$ Torus modular transformations
$$
\tau\to\frac{a\tau+b}{c\tau+d}
$$
$\bullet$  \textbf{mirror symmetry} (symmetry between complexified K\"ahler modulus and complex structure modulus)
$$
\omega\leftrightarrow \tau
$$
How do the momentum and winding numbers transform?


%Let us now figure out what is the moduli space of such
%a compactification.  
%Any nontrivial $O(D,D)$ rotation 
%would map one even self-dual Lorentzian 
%lattice into a different one.  
%The converse is a mathematical fact:  
%any two even self-dual Lorentzian 
%lattices are related by some $O(D,D)$ 
%rotation.  Therefore the space of such lattices is simply $O(D,D)$.  
%However, not all of them correspond to different compactifications.  
%The spectrum for the $(26-D)$-dimensional theory is determined 
%by $p_L^2$ and $p_R^2$.  They are left invariant by  
%$O(D)\times O(D)$, the maximal compact subgroup of $O(D,D)$, 
%acting independently on the left and right momenta respectively.  
%Therefore 
%the space of vacua is \textbf{locally} $O(D,D) / (O(D)\times O(D))$, 
%of dimension $D^2$.  



\end{document}

