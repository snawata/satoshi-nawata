\documentclass[12pt,a4paper]{article}
%\usepackage{hyperref} % Use the Charter font for the document text
%\usepackage[UTF8]{ctex}

\usepackage{simplewick}
\usepackage{feynmp-auto}
\usepackage{wrapfig}

\usepackage{macros}


\begin{document}\thispagestyle{empty}

\centerline{\Large \bf Homework 6 (Due at class on Nov 3)}

\section{Wold-sheet supersymmetry}
\subsection{Supersymmetric transformation}
Show that the action
\be\label{action}
 S^{\textrm{m}} = \frac{1}{4\pi} \int d^2 z\ \Big( \frac{2}{\alpha'} \partial X  \overline\partial X+\psi^\mu\overline\partial\psi_\mu+\wt \psi^\mu\partial\wt\psi_\mu\Big)
\ee
is invariant under the supersymmetric transformation
$$
\delta_{\e,\bar\e} X^\mu=-\sqrt{\frac{\a'}{2}}\,(\e \psi^\mu+\overline \e \wt\psi^\mu)~,\qquad
\delta_\e \psi^\mu =\sqrt{\frac{2}{\a'}}\, \e \partial X^\mu~, \qquad \delta_{\bar\e} \wt\psi^\mu =\sqrt{\frac{2}{\a'}}\,\overline\e \overline \partial X^\mu~.
$$
Also, show that the combination of supersymmetric transformations provides derivatives
$$
[\delta_{\e_1,\bar \e_1},\delta_{\e_2,\bar \e_2}]\cO=2\e_1\e_2\partial  \cO+2\overline \e_1\overline \e_2\overline \partial  \cO~.
$$

\subsection{ Superspace formalism}
This theory can be formulated in terms of superspace $(z,\bar z,\theta, \bar \theta)$ where $(\theta, \bar \theta)$ are anti-commuting Grassmann coordinates. We can introduce the superfield
$$
Y^\mu(z,\bar z,\theta, \bar \theta) = X^\mu(z,\bar z) + i \theta  \psi^\mu(z,\bar z) +i \bar \theta \wt \psi^\mu(z,\bar z) + \frac12 \bar\theta \theta F^\mu(z,\bar z)~,
$$
as well as the superderivative
$$
D = \frac{\partial}{\partial \theta} + \theta \partial_z~, \quad \overline D = \frac{\partial}{\partial  \bar\theta} +\bar \theta \overline\partial_{\bar z}~.
$$
The field $F^\mu$ is called an \textbf{auxiliary field}. Then, show that the action \eqref{action} at $\a'=2$ is equivalent to
$$
S=\frac1{4\pi}\int d^2z d^2\theta \  \overline D Y^\mu DY_\mu~,
$$
where the superspace integral is defined as $d^2\theta=d\theta d\bar\theta$, and
$$
\int d\theta d\bar\theta ~ \bar\theta\theta=1~.
$$


\subsection{Supersymmetric ghost}
Let us define the ghost superfields as
$$
B=\b+\theta b~, \qquad C=c+\theta \g ~.
$$
Show that the supersymmetric ghost action can be written as
$$
S^{\textrm{gh}} = \frac{1}{2\pi} \int d^2z  d^2\theta ~ B \overline D C~.
$$

\section{$\cN=1$ superconformal  algebra}
In general, a 2d superconformal field theory has the following OPEs for the stress-energy tensor $T(z)$ and the supercurrent $G(z)$
\begin{align}\label{OPE}
T(z)T(w) &\sim {c/2\over (z-w)^4}+{2\over (z-w)^2}
  T(w)+{1\over z-w}\partial_w T(w)\cr
T(z) G(w) &\sim {3/2\over (z-w)^2}G(w)+{1\over
  z-w}\partial_w G(w)\cr
G(z)G(w) &\sim {2c/3\over (z-w)^3}+{2\over
z-w}T(w)\,\,.
\end{align}
As in the lecture, we carry out the mode expansion
$$
T_B(z)=\sum\limits_{m\in\bZ}\frac{L_m}{z^{m+2}}~,\qquad G(z)=\sum_{r\in \bZ+\nu}\frac{G_r}{ z^{r+3/2}}
$$
where $\nu=0$ and $\nu=\frac12$ correspond to Ramond sector and
Neveu-Schwarz sector, respectively.
From the OPEs \eqref{OPE}, derive $\cN=1$ superconformal
algebra
\begin{align*}
[L_m,L_n] &=(m-n)L_{m+n}+{c\over
12}\,m(m^2-1)\delta_{m+n,0}\cr
[L_m,G_r] &=\left(\frac12\,m-r\right)G_{r+m}\cr
\{G_r,G_s\} &=2 L_{r+s}+{c\over 3}(r^2-\nu^2)
  \delta_{r+s,0}~.
\end{align*}


\section{First massive states}
Find the number of physical degrees of freedom at the first massive level after the GSO projection both in the NS and R sectors.


\end{document}
