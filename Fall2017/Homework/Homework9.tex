\documentclass[12pt,a4paper]{article}
%\usepackage{hyperref} % Use the Charter font for the document text
%\usepackage[UTF8]{ctex}


\usepackage{amsfonts,amssymb,amsmath}
\usepackage{mathtools}
\usepackage{tikz-cd}
\usepackage{fullpage}
\usepackage{tikz}
\usepackage{alltt}
\usepackage{amsfonts}
\usepackage{amsmath}
\usepackage{amssymb}
\usepackage{amsthm}
\usepackage{booktabs}
\usepackage{caption}
\usepackage{enumitem}
\usepackage{fancyhdr}
\usepackage{graphicx}
% \usepackage{mathdots}
\usepackage{mathtools}
\usepackage{microtype}
\usepackage{multirow}
\usepackage{pdflscape}
\usepackage{pgfplots}
\usepackage{siunitx}
\usepackage{slashed}
\usepackage{tabularx}
\usepackage{tikz}
\usepackage{tkz-euclide}
% \usepackage[normalem]{ulem}
\usepackage[all]{xy}
\usepackage{imakeidx}

\usepackage{wrapfig}
\usepackage{bm}


%%%%%%%  Greek letters %%%%%%%%%%%%%%%%%%
\def\a{\alpha}
\def\b{\beta}
\def\c{\gamma} \def\g{\gamma}
\def\d{\delta}
\def\e{\epsilon}
\def\f{\phi}
\def\vf{\varphi}  \def\tvf{\tilde{\varphi}}
\def\vp{\varphi}
\def\h{\eta}
\def\i{\iota}
\def\j{\psi}
\def\k{\kappa}
\def\m{\mu}
\def\n{\nu}
\def\o{\omega}  \def\w{\omega}
\def\q{\theta}  \def\th{\theta}
\def\r{\rho}
\def\s{\sigma}
\def\t{\tau}
\def\u{\upsilon}
\def\x{\xi}
\def\z{\zeta}

\def\A{\Alpha}
\def\B{\Beta}
\def\G{\Gamma}
\def\D{\Delta}
\def\E{\Epsilon}
\def\F{Phi}
\def\h{\eta}
\def\I{\Iota}
\def\J{Psi}
\def\K{\Kappa}
\def\L{\Lambda}
\def\M{\Mu}
\def\N{\Nu}
\def\O{\Omega}  \def\w{\omega}
\def\Q{\Theta}  \def\Th{\Theta}
\def\R{\Rho}
\def\Si{\Sigma}
\def\T{\Tau}
\def\Up{\Upsilon}
\def\X{\Xi}
\def\Z{\Zeta}








%%%%%%%%%%%% math fonts %%%%%%%%%%%%%%%%%%%%%%%%%%%%%%%%%%%%%
%
%---------- mathbb font --------------------------------
%

\newcommand{\bA}{\ensuremath{\mathbb{A}}}
\newcommand{\bB}{\ensuremath{\mathbb{B}}}
\newcommand{\bC}{\ensuremath{\mathbb{C}}}
\newcommand{\bD}{\ensuremath{\mathbb{D}}}
\newcommand{\bE}{\ensuremath{\mathbb{E}}}
\newcommand{\bF}{\ensuremath{\mathbb{F}}}
\newcommand{\bG}{\ensuremath{\mathbb{G}}}
\newcommand{\bH}{\ensuremath{\mathbb{H}}}
\newcommand{\bI}{\ensuremath{\mathbb{I}}}
\newcommand{\bJ}{\ensuremath{\mathbb{J}}}
\newcommand{\bK}{\ensuremath{\mathbb{K}}}
\newcommand{\bL}{\ensuremath{\mathbb{L}}}
\newcommand{\bM}{\ensuremath{\mathbb{M}}}
\newcommand{\bN}{\ensuremath{\mathbb{N}}}
\newcommand{\bO}{\ensuremath{\mathbb{O}}}
\newcommand{\bP}{\ensuremath{\mathbb{P}}}
\newcommand{\bQ}{\ensuremath{\mathbb{Q}}}
\newcommand{\bR}{\ensuremath{\mathbb{R}}}
\newcommand{\bS}{\ensuremath{\mathbb{S}}}
\newcommand{\bT}{\ensuremath{\mathbb{T}}}
\newcommand{\bU}{\ensuremath{\mathbb{U}}}
\newcommand{\bV}{\ensuremath{\mathbb{V}}}
\newcommand{\bW}{\ensuremath{\mathbb{W}}}
\newcommand{\bX}{\ensuremath{\mathbb{X}}}
\newcommand{\bY}{\ensuremath{\mathbb{Y}}}
\newcommand{\bZ}{\ensuremath{\mathbb{Z}}}


%
%\parskip=1em
%\parindent=0.3in
%\setlength\oddsidemargin{0.5in} \setlength\evensidemargin{0.5in}
%\setlength\textwidth{5.5in}
%
%\hfuzz6pt % Don't bother to report over-full boxes if over-edge is < 6pt
%
%\newlength{\defbaselineskip}
%\setlength{\defbaselineskip}{\baselineskip}
%\newcommand{\setlinespacing}[1]%
%           {\setlength{\baselineskip}{#1 \defbaselineskip}}
%\newcommand{\doublespacing}{\setlength{\baselineskip}%
%                           {2.0 \defbaselineskip}}
%\newcommand{\singlespacing}{\setlength{\baselineskip}{\defbaselineskip}}
%
%\newcommand{\properpagestyle}{\pagestyle{myheadings}\markboth{}{}\markright{}}


%---------- mathscript font -----------------------------
%

\newcommand{\scA}{\ensuremath{\mathscr{A}}}
\newcommand{\scB}{\ensuremath{\mathscr{B}}}
\newcommand{\scC}{\ensuremath{\mathscr{C}}}
\newcommand{\scD}{\ensuremath{\mathscr{D}}}
\newcommand{\scE}{\ensuremath{\mathscr{E}}}
\newcommand{\scF}{\ensuremath{\mathscr{F}}}
\newcommand{\scG}{\ensuremath{\mathscr{G}}}
\newcommand{\scH}{\ensuremath{\mathscr{H}}}
\newcommand{\scI}{\ensuremath{\mathscr{I}}}
\newcommand{\scJ}{\ensuremath{\mathscr{J}}}
\newcommand{\scK}{\ensuremath{\mathscr{K}}}
\newcommand{\scL}{\ensuremath{\mathscr{L}}}
\newcommand{\scM}{\ensuremath{\mathscr{M}}}
\newcommand{\scN}{\ensuremath{\mathscr{N}}}
\newcommand{\scO}{\ensuremath{\mathscr{O}}}
\newcommand{\scP}{\ensuremath{\mathscr{P}}}
\newcommand{\scQ}{\ensuremath{\mathscr{Q}}}
\newcommand{\scR}{\ensuremath{\mathscr{R}}}
\newcommand{\scS}{\ensuremath{\mathscr{S}}}
\newcommand{\scT}{\ensuremath{\mathscr{T}}}
\newcommand{\scU}{\ensuremath{\mathscr{U}}}
\newcommand{\scV}{\ensuremath{\mathscr{V}}}
\newcommand{\scW}{\ensuremath{\mathscr{W}}}
\newcommand{\scX}{\ensuremath{\mathscr{X}}}
\newcommand{\scY}{\ensuremath{\mathscr{Y}}}
\newcommand{\scZ}{\ensuremath{\mathscr{Z}}}
\newcommand{\scAH}{\ensuremath{\mathscr{A}\!\!\scH}}

%
%---------- mathfrak font -----------------------------
%

\newcommand{\frakA}{\ensuremath{\mathfrak{A}}}
\newcommand{\frakB}{\ensuremath{\mathfrak{B}}}
\newcommand{\frakC}{\ensuremath{\mathfrak{C}}}
\newcommand{\frakD}{\ensuremath{\mathfrak{D}}}
\newcommand{\frakE}{\ensuremath{\mathfrak{E}}}
\newcommand{\frakF}{\ensuremath{\mathfrak{F}}}
\newcommand{\frakG}{\ensuremath{\mathfrak{G}}}
\newcommand{\frakH}{\ensuremath{\mathfrak{H}}}
\newcommand{\frakI}{\ensuremath{\mathfrak{I}}}
\newcommand{\frakJ}{\ensuremath{\mathfrak{J}}}
\newcommand{\frakK}{\ensuremath{\mathfrak{K}}}
\newcommand{\frakL}{\ensuremath{\mathfrak{L}}}
\newcommand{\frakM}{\ensuremath{\mathfrak{M}}}
\newcommand{\frakN}{\ensuremath{\mathfrak{N}}}
\newcommand{\frakO}{\ensuremath{\mathfrak{O}}}
\newcommand{\frakP}{\ensuremath{\mathfrak{P}}}
\newcommand{\frakQ}{\ensuremath{\mathfrak{Q}}}
\newcommand{\frakR}{\ensuremath{\mathfrak{R}}}
\newcommand{\frakS}{\ensuremath{\mathfrak{S}}}
\newcommand{\frakT}{\ensuremath{\mathfrak{T}}}
\newcommand{\frakU}{\ensuremath{\mathfrak{U}}}
\newcommand{\frakV}{\ensuremath{\mathfrak{V}}}
\newcommand{\frakW}{\ensuremath{\mathfrak{W}}}
\newcommand{\frakX}{\ensuremath{\mathfrak{X}}}
\newcommand{\frakY}{\ensuremath{\mathfrak{Y}}}
\newcommand{\frakZ}{\ensuremath{\mathfrak{Z}}}
\newcommand{\fraka}{\ensuremath{\mathfrak{a}}}
\newcommand{\frakb}{\ensuremath{\mathfrak{b}}}
\newcommand{\frakc}{\ensuremath{\mathfrak{c}}}
\newcommand{\frakd}{\ensuremath{\mathfrak{d}}}
\newcommand{\frake}{\ensuremath{\mathfrak{e}}}
\newcommand{\frakf}{\ensuremath{\mathfrak{f}}}
\newcommand{\frakg}{\ensuremath{\mathfrak{g}}}
\newcommand{\frakh}{\ensuremath{\mathfrak{h}}}
\newcommand{\fraki}{\ensuremath{\mathfrak{i}}}
\newcommand{\frakj}{\ensuremath{\mathfrak{j}}}
\newcommand{\frakk}{\ensuremath{\mathfrak{k}}}
\newcommand{\frakl}{\ensuremath{\mathfrak{l}}}
\newcommand{\frakm}{\ensuremath{\mathfrak{m}}}
\newcommand{\frakn}{\ensuremath{\mathfrak{n}}}
\newcommand{\frako}{\ensuremath{\mathfrak{o}}}
\newcommand{\frakp}{\ensuremath{\mathfrak{p}}}
\newcommand{\frakq}{\ensuremath{\mathfrak{q}}}
\newcommand{\frakr}{\ensuremath{\mathfrak{r}}}
\newcommand{\fraks}{\ensuremath{\mathfrak{s}}}
\newcommand{\frakt}{\ensuremath{\mathfrak{t}}}
\newcommand{\fraku}{\ensuremath{\mathfrak{u}}}
\newcommand{\frakv}{\ensuremath{\mathfrak{v}}}
\newcommand{\frakw}{\ensuremath{\mathfrak{w}}}
\newcommand{\frakx}{\ensuremath{\mathfrak{x}}}
\newcommand{\fraky}{\ensuremath{\mathfrak{y}}}
\newcommand{\frakz}{\ensuremath{\mathfrak{z}}}
\newcommand{\fraksl}{\ensuremath{\mathfrak{sl}}}
\newcommand{\frakso}{\ensuremath{\mathfrak{so}}}
\newcommand{\fraksp}{\ensuremath{\mathfrak{sp}}}

%%%%%%%%%%%%  Calligraphic, Roman and Maths integers %%%%%%%%%%%%%%%%%%

\newcommand{\cA}{\mathcal{A}}
\newcommand{\cB}{\mathcal{B}}
\newcommand{\cC}{\mathcal{C}}
\newcommand{\cD}{\mathcal{D}}
\newcommand{\cE}{\mathcal{E}}
\newcommand{\cF}{\mathcal{F}}
\newcommand{\cG}{\mathcal{G}}
\newcommand{\cH}{\mathcal{H}}
\newcommand{\cI}{\mathcal{I}}
\newcommand{\cJ}{\mathcal{J}}
\newcommand{\cK}{\mathcal{K}}
\newcommand{\cL}{\mathcal{L}}
\newcommand{\cM}{\mathcal{M}}
\newcommand{\cN}{\mathcal{N}}
\newcommand{\cO}{\mathcal{O}}
\newcommand{\cQ}{\mathcal{Q}}
\newcommand{\cS}{\mathcal{S}}
\newcommand{\cX}{\mathcal{X}}
\newcommand{\cY}{\mathcal{Y}}
\newcommand{\cW}{\mathcal{W}}
\newcommand{\cR}{\mathcal{R}}
\newcommand{\cT}{\mathcal{T}}
\newcommand{\cZ}{\mathcal{Z}}

%%%%%%%%%%%%%%%%%%%%%%%%%%%%%%%%%%%%%%%%%%%%%%%%%%%%%%%%%%%%%%%%
\newcommand{\SU}{\mathrm{SU}}
\newcommand{\SO}{\mathrm{SO}}
\newcommand{\SL}{\mathrm{SL}}
\newcommand{\Sp}{\mathrm{Sp}}
\newcommand{\su}{\mathrm{su}}
\newcommand{\so}{\mathrm{so}}
\newcommand{\spl}{\mathrm{sp}}
\newcommand{\gl}{\mathrm{gl}}
\newcommand{\sll}{\mathrm{sl}}
\newcommand{\U}{\mathrm{U}}
\newcommand{\ul}{\mathrm{u}}
\newcommand{\Spin}{\mathrm{Spin}}
\newcommand{\Pin}{\mathrm{Pin}}
%%%%%%%%%%%%%%%%%%%%%%%%%%%%%%%%%%%%%%%%%%%%%%%%%%%%%%%%%%%%%%%%
\renewcommand{\Im}{{\rm Im}}
\renewcommand{\Re}{{\rm Re}}
\newcommand{\Tr}{\mbox{Tr}}
\newcommand{\Pf}{\mbox{Pf}}
\newcommand{\sgn}{\mbox{sgn}}
\newcommand{\Vir}{{\rm Vir}}
\newcommand{\Li}{{\rm Li}}

\def\tl{\tilde}
\def\wt{\widetilde}
\def\wh{\widehat}
\def\bar{\overline}





\def\bea{\begin{align}}
\def\eea{\end{align}}
\def\be{\begin{equation}}
\def\ee{\end{equation}}
\def\ba{\begin{align}}
\def\ea{\end{align}}


\usepackage{graphicx}


%%% Yokoyama def %%%
\newcommand{\ol}{\overline}
\newcommand{\tr}{\mathrm{tr}}
\newcommand{\nord}[1]{\vcentcolon\mathrel{#1}\vcentcolon}
\providecommand{\vcentcolon}{\mathrel{\mathop{:}}}
%%%%%%%%%%%%%%%%%%%%


\begin{document}\thispagestyle{empty}

\centerline{\Large \bf Homework 9: Due at class on Nov 24}




% \subsection*{Prob. 1 Cylinder partition function and torus partition function}

% The torus partition function of bosonic string is given by
% \begin{align}
%  A_{0,T} = \frac{iV_{26}}{(2\pi l_s)^{26}} \int_F \frac{d^2\tau}{2\tau_2^2}
%  \tau_2^{-12} |\eta(\tau)|^{-48} \ ,
% \end{align}
% where $F$ is the fundamental region of the torus moduli space.
% \begin{itemize}
%  \item Consider low energy region in $F$ and show that $\tau_1$ integration ($-\frac{1}{2} \le \tau_1 \le \frac{1}{2}$) leads level matching condition.
%  \item Again, consider low energy region.
%        Show that
%        it can be identified with the cylinder partition function up to constant.
%  \item Explain why an open string short 1-loop corresponds to
%        a closed string long propergation.
% \end{itemize}


\subsection*{Prob. 1 Contribution from symplectic representation}


Consider
\begin{align}
 P \Lambda^T P = \pm \Lambda
\end{align}
for $P = i
\begin{pmatrix}
 0 & -\bm{1}_{k\times k} \\
 \bm{1}_{k\times k} & 0
\end{pmatrix}$
where $n=2k$.
Derive the dimension(degrees of freedom) of $\Lambda$ for $+$ and $-$, respectively,
and show that $\tr[\Omega_\Lambda] = -n$.


\subsection*{Prob. 2 Orientation flip in superstring}


Consider the orientation flip operator $\Omega$ for world-sheet fermions of closed string.
\begin{itemize}
 \item Define the action of the orientation operator on the fermions
       $\psi^\mu(t,\sigma)$, $\wt \psi^\mu(t,\sigma)$ in NS- and R-sector properly,
       and derive the action on their modes.
       (Hint: the orientation flip is define for $c$ as $\Omega: c(t,\sigma) \to -\ol c(t,2\pi-\sigma)$.
       The minus sign in from of $\ol c$ is coming from relative phase of overall coefficient of the mode expansion in cylinder:
       $c = i \sum c_n e^{in(it+\sigma)}$ and $\ol c = -i \sum \ol c_n e^{in(it+\sigma)}$.
 \item Consider IIB RR-fields as in Prob. 2 of Homework 8, and show that only the 2-form
       RR-field survives under the $\Omega$ projection $\frac{1+\Omega}{2}$.
       (Hint: you have to consider field strength of $n$-form RR-fields because RR-fields themselves
       are not physical degrees of freedom.
       Note that RR-fields are real valued object so its complex conjugate is itself.
       Also note that the complex conjugate of fermions gives minus sign
       $(\psi^\dagger \chi)^\dagger =-\chi^\dagger\psi$ due to their statistics.
       You can assume that gamma matrices are invariant under $\Omega$.
       This is because zero modes are identical in L and R so $\Gamma$ is actually sum of L and R, i.e. $\Gamma = \Gamma^L +\Gamma^R$,
       which is manifestly invariant under $\Omega$.)
\end{itemize}





\subsection*{Prob. 3 O$p$-plane}


Let us consider so called orientifold action.
It is a combination of the orientation flip $\Omega$ as well as
a space-time parity ($\mathbb Z_2$-orbifold) $R_p$:
\begin{align}
 R_p :
 \begin{cases}
  X^i(t,\sigma) \quad\leftrightarrow\quad X^i(t,\sigma) \qquad &(i=0,1,\cdots,p) \ ,\\
  X^a(t,\sigma) \quad\leftrightarrow\quad -X^a(t,\sigma) \qquad &(a=p+1,\cdots,D-1) \ .
 \end{cases}
\end{align}
Note that $\Omega_{D-1}$ is $\Omega$ itself.
The orientifold action $\Omega_p = \Omega \cdot R_p$ is associated to O$p$-plane
(this is why we called the $\Omega$ action
in superstring theory by O$9$-plane).
\begin{itemize}
 \item Write down the orientifold action for $X^i(t,\sigma)$ and $X^a(t,\sigma)$
       of closed bosonic string, as well as their modes (do not forget $x$ and $p$).
 \item Consider bosonic closed string ($D=26$) in an existence of O$23$-plane
       located at $(X_{24}, X_{25})=(0,0)$.
       Illustrate a closed string in $(X_{24}, X_{25})$-plane, as well as its mirror image.
       Note that the closed string is either oriented, or unoriented.
       State which is correct and explain why so.
 \item Massless states of the oriented closed string are given by
  \begin{align}
   |\Phi\rangle = \int \prod_{i,a} dp^i dp^a \Phi^\pm_{IJ} (\tau,p^i,p^a)
   \left( \alpha_{-1}^I \wt \alpha_{-1}^J \pm \alpha_{-1}^J \wt \alpha_{-1}^I \right)
   |p^i,p^a\rangle \ ,
  \end{align}
       where $\Phi^\pm_{IJ} (\tau,p^i,p^a)$ are wavefunctions,
       and $I,J$ runs over the values of both $a$ and $i$.
       Find the conditions on $\Phi^\pm_{ab}, \Phi^\pm_{ia}, \Phi^\pm_{ij}$
       so that they guarantee $\Omega_p$ invariance.
       The result should be the form of
  \begin{align}
   \Phi^\pm_{IJ} (\tau,p^i,p^a) = (+ \textrm{ or } -)\cdot \Phi^\pm_{IJ} (\tau,p^i,-p^a) \ .
  \end{align}
\end{itemize}




\subsection*{Prob. 4 Partition function on $S^1/\mathbb Z_2$}

Let us consider a partition function on $S^1/\mathbb Z_2$ of closed bosonic string.
We only consider the direction along the $S^1/\mathbb Z_2$.
The partition function should naively be given by $\mathbb Z_2$-orbifold projection $\frac{R+1}{2}$
\begin{align}
 Z_\mathrm{orb} = \mathrm{tr}_\mathrm{circ} \left[ \frac{1+R}{2} q^{L_0-\frac{1}{24}} \ol q^{\ol L_0-\frac{1}{24}} \right] \ ,
\end{align}
where $R$ is defined
\begin{align}
 R:\ X(z,\ol z)\ \leftrightarrow\ R X(z,\ol z) R = -X(z,\ol z) \ .
\end{align}
$X$ on $S^1$ is given by
\begin{align}
 X(z) = x +i\sqrt{\frac{\alpha'}{2}} \left( -\alpha_0 \log z +\sum_{n\neq 0} \frac{1}{n} \frac{\alpha_n}{z^n} \right) \ , \\
 \ol X(\ol z) = \wt x +i\sqrt{\frac{\alpha'}{2}} \left( -\wt \alpha_0 \log \ol z +\sum_{n\neq 0} \frac{1}{n} \frac{\wt \alpha_n}{\ol z^n} \right) \ ,
\end{align}
with
\begin{align}
 \alpha_0 = \sqrt{\frac{\alpha'}{2}}\left( \frac{n}{R} +\frac{wR}{\alpha'} \right) \ ,
 \quad \wt \alpha_0 = \sqrt{\frac{\alpha'}{2}}\left( \frac{n}{R} -\frac{wR}{\alpha'} \right) \ ,
\end{align}

\begin{itemize}
 \item Derive $S^1$ partition function: $Z_\mathrm{circ} = \mathrm{tr}_\mathrm{circ} \left[ q^{L_0-\frac{1}{24}} \ol q^{\ol L_0-\frac{1}{24}} \right]$,
       where $L_0 = \frac{1}{2} \sum_{n\in \mathbb Z} \nord{\alpha_{-n} \alpha_{n}}$.
 \item Write down the action of $R$ on $x$ and $\alpha_n$.
       Show that the ground state transforms under $R$ as
       $R |n,w\rangle = |-n,-w\rangle$.
 \item Derive $\mathrm{tr}_\mathrm{circ} \left[ \frac{R}{2} q^{L_0-\frac{1}{24}} \ol q^{\ol L_0-\frac{1}{24}} \right]$, and express it using $\eta(\tau)$ and $\vartheta_2(\tau)$.
       Confirm that it is modular non-invariant.
\end{itemize}

In order to get modular-invariant partition function on $S^1/\mathbb Z_2$,
we need so called twisted sector, which satisfies following boundary condition.
\begin{align}
 X(e^{2\pi i}z) = R X(z) R = -X(z) \ ,
\end{align}
and similar for $\ol X(\ol z)$.
\begin{itemize}
 \item Derive mode expansion for $X$, which is similar to WS fermion in R-sector.
       However note that $X$ has $0$-level ($h=0$).
 \item Derive the partition function in twisted sector $Z_\mathrm{tw} = \mathrm{tr}_\mathrm{tw} \left[ \frac{1+R}{2} q^{L_0-\frac{1}{24}} \ol q^{\ol L_0-\frac{1}{24}} \right]$
       using $\eta(\tau)$, $\vartheta_3(\tau)$, and $\vartheta_4(\tau)$,
       where $L_0 = \frac{1}{2} (\sum_{n\in \mathbb Z +\frac{1}{2}} \nord{\alpha_{-n} \alpha_{n}}) +\frac{1}{16}$.
 \item Confirm that the sum $Z_\mathrm{orb} +Z_\mathrm{tw}$ is modular invariant.
       (You do not have to show $Z_\mathrm{circ}$ is modular invariant.)
\end{itemize}

\end{document}

