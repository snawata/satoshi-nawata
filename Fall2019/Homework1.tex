\documentclass[12pt,a4paper]{article}
\usepackage{hyperref} % Use the Charter font for the document text
%\usepackage[UTF8]{ctex}
\usepackage{fullpage}
\usepackage{amsfonts,amssymb,amsmath}

\usepackage{physics}
\usepackage{epsfig}
\usepackage{amsmath}
\usepackage{amssymb}
\usepackage{amsthm}
\usepackage{indentfirst}
\usepackage{xspace}
\usepackage{multirow}
\usepackage{hyperref}
\usepackage{xcolor}
\usepackage{verbatim}
\usepackage{subfigure}
\usepackage{mathrsfs}
\usepackage{bbm}


%\hypersetup{colorlinks=true,urlcolor=darkred,linkcolor=darkred,citecolor=darkred}
%\usepackage{verbatim}
\usepackage[letterpaper,margin=0.9in,headheight=15pt]{geometry}
\usepackage{mathpazo}
\usepackage{authblk}
\usepackage{empheq}
\usepackage{feynmp}
\usepackage{graphicx}
\usepackage[matrix,arrow]{xy}
\usepackage{young}
\usepackage[vcentermath]{youngtab}
\usepackage{slashed}
%\usepackage{fontds}
%
\usepackage{bbm}
\usepackage{youngtab}
\usepackage{rotfloat}
\usepackage{stmaryrd}
\usepackage{amsfonts,amssymb,amsmath}
\usepackage{tikz-cd}
\usepackage{thmtools}
\usepackage{dashrule}
\usepackage[missing=]{gitinfo2}
\usepackage{fancyhdr}
\usepackage{mdframed}
\usepackage{booktabs}
\usepackage{subfiles}
\usepackage{simplewick}

\usepackage[utf8]{inputenc}


%%%%%%%%%%%% math fonts %%%%%%%%%%%%%%%%%%%%%%%%%%%%%%%%%%%%%
%
%---------- mathbb font --------------------------------
%

\newcommand{\bA}{\ensuremath{\mathbb{A}}}
\newcommand{\bB}{\ensuremath{\mathbb{B}}}
\newcommand{\bC}{\ensuremath{\mathbb{C}}}
\newcommand{\bD}{\ensuremath{\mathbb{D}}}
\newcommand{\bE}{\ensuremath{\mathbb{E}}}
\newcommand{\bF}{\ensuremath{\mathbb{F}}}
\newcommand{\bG}{\ensuremath{\mathbb{G}}}
\newcommand{\bH}{\ensuremath{\mathbb{H}}}
\newcommand{\bI}{\ensuremath{\mathbb{I}}}
\newcommand{\bJ}{\ensuremath{\mathbb{J}}}
\newcommand{\bK}{\ensuremath{\mathbb{K}}}
\newcommand{\bL}{\ensuremath{\mathbb{L}}}
\newcommand{\bM}{\ensuremath{\mathbb{M}}}
\newcommand{\bN}{\ensuremath{\mathbb{N}}}
\newcommand{\bO}{\ensuremath{\mathbb{O}}}
\newcommand{\bP}{\ensuremath{\mathbb{P}}}
\newcommand{\bQ}{\ensuremath{\mathbb{Q}}}
\newcommand{\bR}{\ensuremath{\mathbb{R}}}
\newcommand{\bS}{\ensuremath{\mathbb{S}}}
\newcommand{\bT}{\ensuremath{\mathbb{T}}}
\newcommand{\bU}{\ensuremath{\mathbb{U}}}
\newcommand{\bV}{\ensuremath{\mathbb{V}}}
\newcommand{\bW}{\ensuremath{\mathbb{W}}}
\newcommand{\bX}{\ensuremath{\mathbb{X}}}
\newcommand{\bY}{\ensuremath{\mathbb{Y}}}
\newcommand{\bZ}{\ensuremath{\mathbb{Z}}}



%
%---------- mathbf font --------------------------------
%


\newcommand{\bfA}{\ensuremath{\mathbf{A}}}
\newcommand{\bfB}{\ensuremath{\mathbf{B}}}
\newcommand{\bfC}{\ensuremath{\mathbf{C}}}
\newcommand{\bfD}{\ensuremath{\mathbf{D}}}
\newcommand{\bfE}{\ensuremath{\mathbf{E}}}
\newcommand{\bfF}{\ensuremath{\mathbf{F}}}
\newcommand{\bfG}{\ensuremath{\mathbf{G}}}
\newcommand{\bfH}{\ensuremath{\mathbf{H}}}
\newcommand{\bfI}{\ensuremath{\mathbf{I}}}
\newcommand{\bfJ}{\ensuremath{\mathbf{J}}}
\newcommand{\bfK}{\ensuremath{\mathbf{K}}}
\newcommand{\bfL}{\ensuremath{\mathbf{L}}}
\newcommand{\bfM}{\ensuremath{\mathbf{M}}}
\newcommand{\bfN}{\ensuremath{\mathbf{N}}}
\newcommand{\bfO}{\ensuremath{\mathbf{O}}}
\newcommand{\bfP}{\ensuremath{\mathbf{P}}}
\newcommand{\bfQ}{\ensuremath{\mathbf{Q}}}
\newcommand{\bfR}{\ensuremath{\mathbf{R}}}
\newcommand{\bfS}{\ensuremath{\mathbf{S}}}
\newcommand{\bfT}{\ensuremath{\mathbf{T}}}
\newcommand{\bfU}{\ensuremath{\mathbf{U}}}
\newcommand{\bfV}{\ensuremath{\mathbf{V}}}
\newcommand{\bfW}{\ensuremath{\mathbf{W}}}
\newcommand{\bfX}{\ensuremath{\mathbf{X}}}
\newcommand{\bfY}{\ensuremath{\mathbf{Y}}}
\newcommand{\bfZ}{\ensuremath{\mathbf{Z}}}




%
%---------- mathcal font -----------------------------
%

\newcommand{\scA}{\ensuremath{\mathscr{A}}}
\newcommand{\scB}{\ensuremath{\mathscr{B}}}
\newcommand{\scC}{\ensuremath{\mathscr{C}}}
\newcommand{\scD}{\ensuremath{\mathscr{D}}}
\newcommand{\scE}{\ensuremath{\mathscr{E}}}
\newcommand{\scF}{\ensuremath{\mathscr{F}}}
\newcommand{\scG}{\ensuremath{\mathscr{G}}}
\newcommand{\scH}{\ensuremath{\mathscr{H}}}
\newcommand{\scI}{\ensuremath{\mathscr{I}}}
\newcommand{\scJ}{\ensuremath{\mathscr{J}}}
\newcommand{\scK}{\ensuremath{\mathscr{K}}}
\newcommand{\scL}{\ensuremath{\mathscr{L}}}
\newcommand{\scM}{\ensuremath{\mathscr{M}}}
\newcommand{\scN}{\ensuremath{\mathscr{N}}}
\newcommand{\scO}{\ensuremath{\mathscr{O}}}
\newcommand{\scP}{\ensuremath{\mathscr{P}}}
\newcommand{\scQ}{\ensuremath{\mathscr{Q}}}
\newcommand{\scR}{\ensuremath{\mathscr{R}}}
\newcommand{\scS}{\ensuremath{\mathscr{S}}}
\newcommand{\scT}{\ensuremath{\mathscr{T}}}
\newcommand{\scU}{\ensuremath{\mathscr{U}}}
\newcommand{\scV}{\ensuremath{\mathscr{V}}}
\newcommand{\scW}{\ensuremath{\mathscr{W}}}
\newcommand{\scX}{\ensuremath{\mathscr{X}}}
\newcommand{\scY}{\ensuremath{\mathscr{Y}}}
\newcommand{\scZ}{\ensuremath{\mathscr{Z}}}

%
%---------- mathfrak font -----------------------------
%

\newcommand{\frakA}{\ensuremath{\mathfrak{A}}}
\newcommand{\frakB}{\ensuremath{\mathfrak{B}}}
\newcommand{\frakC}{\ensuremath{\mathfrak{C}}}
\newcommand{\frakD}{\ensuremath{\mathfrak{D}}}
\newcommand{\frakE}{\ensuremath{\mathfrak{E}}}
\newcommand{\frakF}{\ensuremath{\mathfrak{F}}}
\newcommand{\frakG}{\ensuremath{\mathfrak{G}}}
\newcommand{\frakH}{\ensuremath{\mathfrak{H}}}
\newcommand{\frakI}{\ensuremath{\mathfrak{I}}}
\newcommand{\frakJ}{\ensuremath{\mathfrak{J}}}
\newcommand{\frakK}{\ensuremath{\mathfrak{K}}}
\newcommand{\frakL}{\ensuremath{\mathfrak{L}}}
\newcommand{\frakM}{\ensuremath{\mathfrak{M}}}
\newcommand{\frakN}{\ensuremath{\mathfrak{N}}}
\newcommand{\frakO}{\ensuremath{\mathfrak{O}}}
\newcommand{\frakP}{\ensuremath{\mathfrak{P}}}
\newcommand{\frakQ}{\ensuremath{\mathfrak{Q}}}
\newcommand{\frakR}{\ensuremath{\mathfrak{R}}}
\newcommand{\frakS}{\ensuremath{\mathfrak{S}}}
\newcommand{\frakT}{\ensuremath{\mathfrak{T}}}
\newcommand{\frakU}{\ensuremath{\mathfrak{U}}}
\newcommand{\frakV}{\ensuremath{\mathfrak{V}}}
\newcommand{\frakW}{\ensuremath{\mathfrak{W}}}
\newcommand{\frakX}{\ensuremath{\mathfrak{X}}}
\newcommand{\frakY}{\ensuremath{\mathfrak{Y}}}
\newcommand{\frakZ}{\ensuremath{\mathfrak{Z}}}
\newcommand{\fraka}{\ensuremath{\mathfrak{a}}}
\newcommand{\frakb}{\ensuremath{\mathfrak{b}}}
\newcommand{\frakc}{\ensuremath{\mathfrak{c}}}
\newcommand{\frakd}{\ensuremath{\mathfrak{d}}}
\newcommand{\frake}{\ensuremath{\mathfrak{e}}}
\newcommand{\frakf}{\ensuremath{\mathfrak{f}}}
\newcommand{\frakg}{\ensuremath{\mathfrak{g}}}
\newcommand{\frakh}{\ensuremath{\mathfrak{h}}}
\newcommand{\fraki}{\ensuremath{\mathfrak{i}}}
\newcommand{\frakj}{\ensuremath{\mathfrak{j}}}
\newcommand{\frakk}{\ensuremath{\mathfrak{k}}}
\newcommand{\frakl}{\ensuremath{\mathfrak{l}}}
\newcommand{\frakm}{\ensuremath{\mathfrak{m}}}
\newcommand{\frakn}{\ensuremath{\mathfrak{n}}}
\newcommand{\frako}{\ensuremath{\mathfrak{o}}}
\newcommand{\frakp}{\ensuremath{\mathfrak{p}}}
\newcommand{\frakq}{\ensuremath{\mathfrak{q}}}
\newcommand{\frakr}{\ensuremath{\mathfrak{r}}}
\newcommand{\fraks}{\ensuremath{\mathfrak{s}}}
\newcommand{\frakt}{\ensuremath{\mathfrak{t}}}
\newcommand{\fraku}{\ensuremath{\mathfrak{u}}}
\newcommand{\frakv}{\ensuremath{\mathfrak{v}}}
\newcommand{\frakw}{\ensuremath{\mathfrak{w}}}
\newcommand{\frakx}{\ensuremath{\mathfrak{x}}}
\newcommand{\fraky}{\ensuremath{\mathfrak{y}}}
\newcommand{\frakz}{\ensuremath{\mathfrak{z}}}
\newcommand{\fraksl}{\ensuremath{\mathfrak{sl}}}
\newcommand{\frakso}{\ensuremath{\mathfrak{so}}}
\newcommand{\fraksp}{\ensuremath{\mathfrak{sp}}}

%%%%%%%%%%%%  Calligraphic, Roman and Maths integers %%%%%%%%%%%%%%%%%%

\newcommand{\cA}{\mathcal{A}}
\newcommand{\cB}{\mathcal{B}}
\newcommand{\cC}{\mathcal{C}}
\newcommand{\cD}{\mathcal{D}}
\newcommand{\cE}{\mathcal{E}}
\newcommand{\cF}{\mathcal{F}}
\newcommand{\cG}{\mathcal{G}}
\newcommand{\cH}{\mathcal{H}}
\newcommand{\cI}{\mathcal{I}}
\newcommand{\cJ}{\mathcal{J}}
\newcommand{\cK}{\mathcal{K}}
\newcommand{\cL}{\mathcal{L}}
\newcommand{\cM}{\mathcal{M}}
\newcommand{\cN}{\mathcal{N}}
\newcommand{\cO}{\mathcal{O}}
\newcommand{\cP}{\mathcal{P}}
\newcommand{\cQ}{\mathcal{Q}}
\newcommand{\cS}{\mathcal{S}}
\newcommand{\cU}{\mathcal{U}}
\newcommand{\cX}{\mathcal{X}}
\newcommand{\cY}{\mathcal{Y}}
\newcommand{\cV}{\mathcal{V}}
\newcommand{\cW}{\mathcal{W}}
\newcommand{\cR}{\mathcal{R}}
\newcommand{\cT}{\mathcal{T}}
\newcommand{\cZ}{\mathcal{Z}}


%%%%%%%%%%%% mathsf%%%%%%%%%%%%%%%%%%


\newcommand{\sfA}{\ensuremath{\mathsf{A}}}
\newcommand{\sfB}{\ensuremath{\mathsf{B}}}
\newcommand{\sfC}{\ensuremath{\mathsf{C}}}
\newcommand{\sfD}{\ensuremath{\mathsf{D}}}
\newcommand{\sfE}{\ensuremath{\mathsf{E}}}
\newcommand{\sfF}{\ensuremath{\mathsf{F}}}
\newcommand{\sfG}{\ensuremath{\mathsf{G}}}
\newcommand{\sfH}{\ensuremath{\mathsf{H}}}
\newcommand{\sfJ}{\ensuremath{\mathsf{J}}}
\newcommand{\sfK}{\ensuremath{\mathsf{K}}}
\newcommand{\sfL}{\ensuremath{\mathsf{L}}}
\newcommand{\sfM}{\ensuremath{\mathsf{M}}}
\newcommand{\sfN}{\ensuremath{\mathsf{N}}}
\newcommand{\sfO}{\ensuremath{\mathsf{O}}}
\newcommand{\sfP}{\ensuremath{\mathsf{P}}}
\newcommand{\sfQ}{\ensuremath{\mathsf{Q}}}
\newcommand{\sfS}{\ensuremath{\mathsf{S}}}
\newcommand{\sfU}{\ensuremath{\mathsf{U}}}
\newcommand{\sfX}{\ensuremath{\mathsf{X}}}
\newcommand{\sfY}{\ensuremath{\mathsf{Y}}}
\newcommand{\sfW}{\ensuremath{\mathsf{W}}}
\newcommand{\sfR}{\ensuremath{\mathsf{R}}}
\newcommand{\sfT}{\ensuremath{\mathsf{T}}}
\newcommand{\sfZ}{\ensuremath{\mathsf{Z}}}

%%%%%%%%%%%%  Special letters for Lie groups %%%%%%%%%%%%%%%%%%

\newcommand{\biA}{{\mathbi{A}}}
\newcommand{\biB}{{\mathbi{B}}}
\newcommand{\biC}{{\mathbi{C}}}
\newcommand{\biD}{{\mathbi{D}}}
\newcommand{\biE}{{\mathbi{E}}}
\newcommand{\biF}{{\mathbi{F}}}
\newcommand{\biG}{{\mathbi{G}}}
\newcommand{\biH}{{\mathbi{H}}}
\newcommand{\biI}{{\mathbi{I}}}
\newcommand{\biJ}{{\mathbi{J}}}
\newcommand{\biK}{{\mathbi{K}}}
\newcommand{\biL}{{\mathbi{L}}}
\newcommand{\biM}{{\mathbi{M}}}
\newcommand{\biN}{{\mathbi{N}}}
\newcommand{\biO}{{\mathbi{O}}}
\newcommand{\biP}{{\mathbi{P}}}
\newcommand{\biQ}{{\mathbi{Q}}}
\newcommand{\biS}{{\mathbi{S}}}
\newcommand{\biU}{{\mathbi{U}}}
\newcommand{\biX}{{\mathbi{X}}}
\newcommand{\biY}{{\mathbi{Y}}}
\newcommand{\biV}{{\mathbi{V}}}
\newcommand{\biW}{{\mathbi{W}}}
\newcommand{\biR}{{\mathbi{R}}}
\newcommand{\biT}{{\mathbi{T}}}
\newcommand{\biZ}{{\mathbi{Z}}}




%%%%%%%%%%%%%%%%%%%%%%%%%%%%%%%%%%%%%%%%%%%%%%%%%%%%%%%%%%%%%%%%
\newcommand{\SU}{\mathrm{SU}}
\newcommand{\SO}{\mathrm{SO}}
\newcommand{\SL}{\mathrm{SL}}
\newcommand{\Sp}{\mathrm{Sp}}
\newcommand{\U}{\mathrm{U}}
\newcommand{\ul}{\mathrm{u}}
\newcommand{\Spin}{\mathrm{Spin}}
\newcommand{\Pin}{\mathrm{Pin}}
\newcommand{\PSL}{\mathrm{PSL}}
%%%%%%%%%%%%%%%%%%%%%%%%%%%%%%%%%%%%%%%%%%%%%%%%%%%%%%%%%%%%%%%%




\def \be  {\begin{equation}}
\def \ee  {\end{equation}}
\def \bea {\begin{equation}\begin{aligned}}
\def \eea {\end{aligned}\end{equation}}
\def \ba  {\begin{eqnarray}}
\def \ea  {\end{eqnarray}}

\begin{document}\thispagestyle{empty}

\centerline{\Large \bf Homework 1: Due at class on Sep 27}
\section{Lagrangian and symmetry}
\subsection{}Let us consider the Lagrangian of the electromagnetic field
$$
\mathcal{L}_{\textrm{EM}}=-\frac{1}{4} F_{\mu \nu} F^{\mu \nu}-A_\mu J^\mu
$$
where $F_{\mu\nu}=\partial_\mu A_\nu-\partial_\nu A_\mu$ and $J_\mu$ is a 4-vector.  Derive the Euler-Lagrange equations for \(A_{\mu}(x)\). Write the equations in
standard form of the Maxwell's equations by identifying \(E^{i}=-F^{0 i}\) and \(\epsilon^{i j k} B^{k}=-F^{i j} .\)

\subsection{}\label{complex-scalar} Show that the Lagrangian of a complex scalar field
$$
\mathcal{L}_{\textrm{cpx scalar}}=\partial_{\mu} \phi^{*}(x) \partial^{\mu} \phi(x)-m^{2} \phi^{*}(x) \phi(x)
$$
is invariant under the global \(\U(1)\) symmetry
$$
\phi(x) \rightarrow e^{i \alpha} \phi(x), \quad \phi^{*}(x) \rightarrow e^{-i \alpha} \phi^{*}(x), \quad \alpha \in \mathbb{R}
$$
Find the corresponding Noether current.



\subsection{} Let us consider the Lagrangian of a complex scalar field coupled to electromagnetic field
$$
\mathcal{L}=\left(D_{\mu} \phi\right)^{*} D^{\mu} \phi-m^{2} \phi^{*} \phi-\frac{1}{4} F_{\mu \nu} F^{\mu \nu}
$$
 where \(D_{\mu}=\partial_{\mu}+i e A_{\mu} .\) Find the energy momentum \(T^{\mu \nu}\).

 Show that \(\mathcal{L}\) is invariant under the $\U(1)$ gauge (sometimes called \textbf{local}) symmetry
\begin{align}\nonumber
\phi(x) & \rightarrow e^{i \alpha(x)} \phi(x) \\ A_{\mu}(x) & \rightarrow A_{\mu}(x)-\frac1e \partial_{\mu} \alpha(x)\nonumber
\end{align}
Note that $\alpha$ for the \textbf{global} \(\U(1)\) symmetry in Problem \ref{complex-scalar} does not depend on \(x .\)





\section{Complex scalar field}

Let us consider quantum theory of a complex scalar field in Problem \ref{complex-scalar}.

\subsection{} The mode expansion of the complex scalar  field \(\phi(\mathbf{x})\) is
$$
\begin{aligned} \phi(\mathbf{x}) &=\int \frac{d^{3} p}{2 \pi^{3}} \frac{1}{\sqrt{2 E_{\mathbf{p}}}}\left(a_{\mathbf{p}} e^{i \mathbf{p} \cdot \mathbf{x}}+b_{\mathbf{p}}^{\dagger} e^{-i \mathbf{p} \cdot \mathbf{x}}\right) \\ \phi^*(\mathbf{x})&=\int \frac{d^{3} p}{2 \pi^{3}} \frac{1}{\sqrt{2 E_{\mathbf{p}}}}\left(b_{\mathbf{p}} e^{i \mathbf{p} \cdot \mathbf{x}}+a_{\mathbf{p}}^{\dagger} e^{-i \mathbf{p} \cdot \mathbf{x}}\right) \end{aligned}~.
$$
Imposing the canonical commutation relations
$$[\phi(\mathbf{x}), \pi(\mathbf{y})]=i \delta^{(3)}(\mathbf{x}-\mathbf{y})=[\phi^*(\mathbf{x}), \pi^*(\mathbf{y})]~,$$
and the others vanish, derive the commutation relations of the creation and annihilation operators \(a_{\mathbf{p}}, a_{\mathbf{p}}^{\dagger}, b_{\mathbf{p}}, b_{\mathbf{p}}^{\dagger}~.\)

\subsection{}
Show that the Noether charge \(Q\) for the global \(\U(1)\) symmetry in Problem \ref{complex-scalar} can be expressed as follows in terms of the
modes of \(\phi :\)
$$
Q=-\int \frac{d^{3} p}{(2 \pi)^{3}}\left(a^{\dagger}_{\mathbf{p}} a_{\mathbf{p}}-b^{\dagger}_{\mathbf{p}} b_{\mathbf{p}}\right)
$$
after dropping a constant term.
Show that the state \(a^{\dagger}_{\mathbf{p}}|0\rangle\) has the $\U(1)$ charge \(- 1\) and the state \(b^{\dagger}_{\mathbf{p}}|0\rangle\) has the $\U(1)$ charge \(+1\).



\subsection{}
Let us consider the operators in the Heisenberg picture
$$
\begin{aligned} \phi(x) &=\phi(\mathbf{x},t) =\int \frac{\mathrm{d}^{3} p}{(2 \pi)^{3}} \frac{1}{\sqrt{2 E_{\mathbf{p}}}}\left(a_{\mathbf{p}} e^{-i p \cdot x}+b_{\mathbf{p}}^{\dagger} e^{i p \cdot x}\right) \\ \phi^{*}(x) &=\phi^{*}(\mathbf{x},t)=\int \frac{\mathrm{d}^{3} p}{(2 \pi)^{3}} \frac{1}{\sqrt{2 E_{\mathbf{p}}}}\left(b_{\mathbf{p}} e^{-i p x}+a_{\mathbf{p}}^{\dagger} e^{i p \cdot x}\right) \end{aligned}
$$
Using the commutation relations of the creation and annihilation operators to calculate
\(\langle 0|T[\phi(x) \phi(y)]| 0\rangle,\left\langle 0\left|T\phi^{*}(x) \phi^{*}(y)\right| 0\right\rangle\), and \(\left\langle 0\left|T\phi(x) \phi^{*}(y)\right| 0\right\rangle\).
Compare with the Feynman propagator the real scalar field.


\section{Propagator}
The Feynman propagator of a real scalar field in momentum space is given by
$$
D_{F}(x-y)=\langle 0|T \phi(x) \phi(y)| 0\rangle=\int \frac{d^{4} p}{(2 \pi)^{4}} \frac{i}{p^{2}-m^{2}+i \epsilon} e^{-i p \cdot(x-y)}
$$
where the \(i \epsilon\) is there to ensure that the correct poles are included in the \(p_{0}\) integral.
Compute \(D_{F}\) explicitly and write it as a function of the Lorentz-invariant length \(s^{2}=\)
\((x-y)_{\mu}(x-y)^{\mu}\) in terms of the modified Hankel and
Bessel functions in the limit \(\epsilon \rightarrow 0\):
\bea
H_{n}^{(1)}(z)&= \frac{-2i}{\sqrt{\pi}\left(n-\frac{1}{2}\right) !} \left(-\frac{1}{2} z\right)^{n}\int_{1}^{\infty-i \varepsilon} e^{-i z x}\left(x^{2}-1\right)^{n-1 / 2} dx\cr
K_{n}(z)&=\frac{\sqrt{\pi}}{\left(n-\frac{1}{2}\right) !}\left(\frac{1}{2} z\right)^{n} \int_{1}^{\infty} e^{-z x}\left(x^{2}-1\right)^{n-1 / 2} d x \nonumber
\eea
where \(K_{n}(z)=i^{n+1}(\pi / 2) H_{n}^{(1)}(i z)\).

Discuss the behaviors of the propagator in the following regimes by using the properties of the modified Hankel and
Bessel functions. (See \href{https://en.wikipedia.org/wiki/Bessel_function}{Wikipedia}.)
\begin{itemize}
    \item  \(s \rightarrow 0 \)
    \item \(s^{2}\) is large and positive (timelike separation)
    \item  \(s^{2}\) is large and negative (spacelike separation)
    \item \(m^{2} \rightarrow 0\)
    \end{itemize}






\end{document}
