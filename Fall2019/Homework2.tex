\documentclass[12pt,a4paper]{article}
\usepackage{hyperref} % Use the Charter font for the document text
%\usepackage[UTF8]{ctex}
\usepackage{fullpage}
\usepackage{amsfonts,amssymb,amsmath}
\usepackage{slashed}
\usepackage{physics}
\usepackage{epsfig}
\usepackage{amsmath}
\usepackage{amssymb}
\usepackage{amsthm}
\usepackage{indentfirst}
\usepackage{xspace}
\usepackage{multirow}
\usepackage{hyperref}
\usepackage{xcolor}
\usepackage{verbatim}
\usepackage{subfigure}
\usepackage{mathrsfs}
\usepackage{bbm}


%\hypersetup{colorlinks=true,urlcolor=darkred,linkcolor=darkred,citecolor=darkred}
%\usepackage{verbatim}
\usepackage[letterpaper,margin=0.9in,headheight=15pt]{geometry}
\usepackage{mathpazo}
\usepackage{authblk}
\usepackage{empheq}
\usepackage{feynmp}
\usepackage{graphicx}
\usepackage[matrix,arrow]{xy}
\usepackage{young}
\usepackage[vcentermath]{youngtab}
\usepackage{slashed}
%\usepackage{fontds}
%
\usepackage{bbm}
\usepackage{youngtab}
\usepackage{rotfloat}
\usepackage{stmaryrd}
\usepackage{amsfonts,amssymb,amsmath}
\usepackage{tikz-cd}
\usepackage{thmtools}
\usepackage{dashrule}
\usepackage[missing=]{gitinfo2}
\usepackage{fancyhdr}
\usepackage{mdframed}
\usepackage{booktabs}
\usepackage{subfiles}
\usepackage{simplewick}

\usepackage[utf8]{inputenc}


%%%%%%%%%%%% math fonts %%%%%%%%%%%%%%%%%%%%%%%%%%%%%%%%%%%%%
%
%---------- mathbb font --------------------------------
%

\newcommand{\bA}{\ensuremath{\mathbb{A}}}
\newcommand{\bB}{\ensuremath{\mathbb{B}}}
\newcommand{\bC}{\ensuremath{\mathbb{C}}}
\newcommand{\bD}{\ensuremath{\mathbb{D}}}
\newcommand{\bE}{\ensuremath{\mathbb{E}}}
\newcommand{\bF}{\ensuremath{\mathbb{F}}}
\newcommand{\bG}{\ensuremath{\mathbb{G}}}
\newcommand{\bH}{\ensuremath{\mathbb{H}}}
\newcommand{\bI}{\ensuremath{\mathbb{I}}}
\newcommand{\bJ}{\ensuremath{\mathbb{J}}}
\newcommand{\bK}{\ensuremath{\mathbb{K}}}
\newcommand{\bL}{\ensuremath{\mathbb{L}}}
\newcommand{\bM}{\ensuremath{\mathbb{M}}}
\newcommand{\bN}{\ensuremath{\mathbb{N}}}
\newcommand{\bO}{\ensuremath{\mathbb{O}}}
\newcommand{\bP}{\ensuremath{\mathbb{P}}}
\newcommand{\bQ}{\ensuremath{\mathbb{Q}}}
\newcommand{\bR}{\ensuremath{\mathbb{R}}}
\newcommand{\bS}{\ensuremath{\mathbb{S}}}
\newcommand{\bT}{\ensuremath{\mathbb{T}}}
\newcommand{\bU}{\ensuremath{\mathbb{U}}}
\newcommand{\bV}{\ensuremath{\mathbb{V}}}
\newcommand{\bW}{\ensuremath{\mathbb{W}}}
\newcommand{\bX}{\ensuremath{\mathbb{X}}}
\newcommand{\bY}{\ensuremath{\mathbb{Y}}}
\newcommand{\bZ}{\ensuremath{\mathbb{Z}}}



%
%---------- mathbf font --------------------------------
%


\newcommand{\bfA}{\ensuremath{\mathbf{A}}}
\newcommand{\bfB}{\ensuremath{\mathbf{B}}}
\newcommand{\bfC}{\ensuremath{\mathbf{C}}}
\newcommand{\bfD}{\ensuremath{\mathbf{D}}}
\newcommand{\bfE}{\ensuremath{\mathbf{E}}}
\newcommand{\bfF}{\ensuremath{\mathbf{F}}}
\newcommand{\bfG}{\ensuremath{\mathbf{G}}}
\newcommand{\bfH}{\ensuremath{\mathbf{H}}}
\newcommand{\bfI}{\ensuremath{\mathbf{I}}}
\newcommand{\bfJ}{\ensuremath{\mathbf{J}}}
\newcommand{\bfK}{\ensuremath{\mathbf{K}}}
\newcommand{\bfL}{\ensuremath{\mathbf{L}}}
\newcommand{\bfM}{\ensuremath{\mathbf{M}}}
\newcommand{\bfN}{\ensuremath{\mathbf{N}}}
\newcommand{\bfO}{\ensuremath{\mathbf{O}}}
\newcommand{\bfP}{\ensuremath{\mathbf{P}}}
\newcommand{\bfQ}{\ensuremath{\mathbf{Q}}}
\newcommand{\bfR}{\ensuremath{\mathbf{R}}}
\newcommand{\bfS}{\ensuremath{\mathbf{S}}}
\newcommand{\bfT}{\ensuremath{\mathbf{T}}}
\newcommand{\bfU}{\ensuremath{\mathbf{U}}}
\newcommand{\bfV}{\ensuremath{\mathbf{V}}}
\newcommand{\bfW}{\ensuremath{\mathbf{W}}}
\newcommand{\bfX}{\ensuremath{\mathbf{X}}}
\newcommand{\bfY}{\ensuremath{\mathbf{Y}}}
\newcommand{\bfZ}{\ensuremath{\mathbf{Z}}}




%
%---------- mathcal font -----------------------------
%

\newcommand{\scA}{\ensuremath{\mathscr{A}}}
\newcommand{\scB}{\ensuremath{\mathscr{B}}}
\newcommand{\scC}{\ensuremath{\mathscr{C}}}
\newcommand{\scD}{\ensuremath{\mathscr{D}}}
\newcommand{\scE}{\ensuremath{\mathscr{E}}}
\newcommand{\scF}{\ensuremath{\mathscr{F}}}
\newcommand{\scG}{\ensuremath{\mathscr{G}}}
\newcommand{\scH}{\ensuremath{\mathscr{H}}}
\newcommand{\scI}{\ensuremath{\mathscr{I}}}
\newcommand{\scJ}{\ensuremath{\mathscr{J}}}
\newcommand{\scK}{\ensuremath{\mathscr{K}}}
\newcommand{\scL}{\ensuremath{\mathscr{L}}}
\newcommand{\scM}{\ensuremath{\mathscr{M}}}
\newcommand{\scN}{\ensuremath{\mathscr{N}}}
\newcommand{\scO}{\ensuremath{\mathscr{O}}}
\newcommand{\scP}{\ensuremath{\mathscr{P}}}
\newcommand{\scQ}{\ensuremath{\mathscr{Q}}}
\newcommand{\scR}{\ensuremath{\mathscr{R}}}
\newcommand{\scS}{\ensuremath{\mathscr{S}}}
\newcommand{\scT}{\ensuremath{\mathscr{T}}}
\newcommand{\scU}{\ensuremath{\mathscr{U}}}
\newcommand{\scV}{\ensuremath{\mathscr{V}}}
\newcommand{\scW}{\ensuremath{\mathscr{W}}}
\newcommand{\scX}{\ensuremath{\mathscr{X}}}
\newcommand{\scY}{\ensuremath{\mathscr{Y}}}
\newcommand{\scZ}{\ensuremath{\mathscr{Z}}}

%
%---------- mathfrak font -----------------------------
%

\newcommand{\frakA}{\ensuremath{\mathfrak{A}}}
\newcommand{\frakB}{\ensuremath{\mathfrak{B}}}
\newcommand{\frakC}{\ensuremath{\mathfrak{C}}}
\newcommand{\frakD}{\ensuremath{\mathfrak{D}}}
\newcommand{\frakE}{\ensuremath{\mathfrak{E}}}
\newcommand{\frakF}{\ensuremath{\mathfrak{F}}}
\newcommand{\frakG}{\ensuremath{\mathfrak{G}}}
\newcommand{\frakH}{\ensuremath{\mathfrak{H}}}
\newcommand{\frakI}{\ensuremath{\mathfrak{I}}}
\newcommand{\frakJ}{\ensuremath{\mathfrak{J}}}
\newcommand{\frakK}{\ensuremath{\mathfrak{K}}}
\newcommand{\frakL}{\ensuremath{\mathfrak{L}}}
\newcommand{\frakM}{\ensuremath{\mathfrak{M}}}
\newcommand{\frakN}{\ensuremath{\mathfrak{N}}}
\newcommand{\frakO}{\ensuremath{\mathfrak{O}}}
\newcommand{\frakP}{\ensuremath{\mathfrak{P}}}
\newcommand{\frakQ}{\ensuremath{\mathfrak{Q}}}
\newcommand{\frakR}{\ensuremath{\mathfrak{R}}}
\newcommand{\frakS}{\ensuremath{\mathfrak{S}}}
\newcommand{\frakT}{\ensuremath{\mathfrak{T}}}
\newcommand{\frakU}{\ensuremath{\mathfrak{U}}}
\newcommand{\frakV}{\ensuremath{\mathfrak{V}}}
\newcommand{\frakW}{\ensuremath{\mathfrak{W}}}
\newcommand{\frakX}{\ensuremath{\mathfrak{X}}}
\newcommand{\frakY}{\ensuremath{\mathfrak{Y}}}
\newcommand{\frakZ}{\ensuremath{\mathfrak{Z}}}
\newcommand{\fraka}{\ensuremath{\mathfrak{a}}}
\newcommand{\frakb}{\ensuremath{\mathfrak{b}}}
\newcommand{\frakc}{\ensuremath{\mathfrak{c}}}
\newcommand{\frakd}{\ensuremath{\mathfrak{d}}}
\newcommand{\frake}{\ensuremath{\mathfrak{e}}}
\newcommand{\frakf}{\ensuremath{\mathfrak{f}}}
\newcommand{\frakg}{\ensuremath{\mathfrak{g}}}
\newcommand{\frakh}{\ensuremath{\mathfrak{h}}}
\newcommand{\fraki}{\ensuremath{\mathfrak{i}}}
\newcommand{\frakj}{\ensuremath{\mathfrak{j}}}
\newcommand{\frakk}{\ensuremath{\mathfrak{k}}}
\newcommand{\frakl}{\ensuremath{\mathfrak{l}}}
\newcommand{\frakm}{\ensuremath{\mathfrak{m}}}
\newcommand{\frakn}{\ensuremath{\mathfrak{n}}}
\newcommand{\frako}{\ensuremath{\mathfrak{o}}}
\newcommand{\frakp}{\ensuremath{\mathfrak{p}}}
\newcommand{\frakq}{\ensuremath{\mathfrak{q}}}
\newcommand{\frakr}{\ensuremath{\mathfrak{r}}}
\newcommand{\fraks}{\ensuremath{\mathfrak{s}}}
\newcommand{\frakt}{\ensuremath{\mathfrak{t}}}
\newcommand{\fraku}{\ensuremath{\mathfrak{u}}}
\newcommand{\frakv}{\ensuremath{\mathfrak{v}}}
\newcommand{\frakw}{\ensuremath{\mathfrak{w}}}
\newcommand{\frakx}{\ensuremath{\mathfrak{x}}}
\newcommand{\fraky}{\ensuremath{\mathfrak{y}}}
\newcommand{\frakz}{\ensuremath{\mathfrak{z}}}
\newcommand{\fraksl}{\ensuremath{\mathfrak{sl}}}
\newcommand{\frakso}{\ensuremath{\mathfrak{so}}}
\newcommand{\fraksp}{\ensuremath{\mathfrak{sp}}}

%%%%%%%%%%%%  Calligraphic, Roman and Maths integers %%%%%%%%%%%%%%%%%%

\newcommand{\cA}{\mathcal{A}}
\newcommand{\cB}{\mathcal{B}}
\newcommand{\cC}{\mathcal{C}}
\newcommand{\cD}{\mathcal{D}}
\newcommand{\cE}{\mathcal{E}}
\newcommand{\cF}{\mathcal{F}}
\newcommand{\cG}{\mathcal{G}}
\newcommand{\cH}{\mathcal{H}}
\newcommand{\cI}{\mathcal{I}}
\newcommand{\cJ}{\mathcal{J}}
\newcommand{\cK}{\mathcal{K}}
\newcommand{\cL}{\mathcal{L}}
\newcommand{\cM}{\mathcal{M}}
\newcommand{\cN}{\mathcal{N}}
\newcommand{\cO}{\mathcal{O}}
\newcommand{\cP}{\mathcal{P}}
\newcommand{\cQ}{\mathcal{Q}}
\newcommand{\cS}{\mathcal{S}}
\newcommand{\cU}{\mathcal{U}}
\newcommand{\cX}{\mathcal{X}}
\newcommand{\cY}{\mathcal{Y}}
\newcommand{\cV}{\mathcal{V}}
\newcommand{\cW}{\mathcal{W}}
\newcommand{\cR}{\mathcal{R}}
\newcommand{\cT}{\mathcal{T}}
\newcommand{\cZ}{\mathcal{Z}}


%%%%%%%%%%%% mathsf%%%%%%%%%%%%%%%%%%


\newcommand{\sfA}{\ensuremath{\mathsf{A}}}
\newcommand{\sfB}{\ensuremath{\mathsf{B}}}
\newcommand{\sfC}{\ensuremath{\mathsf{C}}}
\newcommand{\sfD}{\ensuremath{\mathsf{D}}}
\newcommand{\sfE}{\ensuremath{\mathsf{E}}}
\newcommand{\sfF}{\ensuremath{\mathsf{F}}}
\newcommand{\sfG}{\ensuremath{\mathsf{G}}}
\newcommand{\sfH}{\ensuremath{\mathsf{H}}}
\newcommand{\sfJ}{\ensuremath{\mathsf{J}}}
\newcommand{\sfK}{\ensuremath{\mathsf{K}}}
\newcommand{\sfL}{\ensuremath{\mathsf{L}}}
\newcommand{\sfM}{\ensuremath{\mathsf{M}}}
\newcommand{\sfN}{\ensuremath{\mathsf{N}}}
\newcommand{\sfO}{\ensuremath{\mathsf{O}}}
\newcommand{\sfP}{\ensuremath{\mathsf{P}}}
\newcommand{\sfQ}{\ensuremath{\mathsf{Q}}}
\newcommand{\sfS}{\ensuremath{\mathsf{S}}}
\newcommand{\sfU}{\ensuremath{\mathsf{U}}}
\newcommand{\sfX}{\ensuremath{\mathsf{X}}}
\newcommand{\sfY}{\ensuremath{\mathsf{Y}}}
\newcommand{\sfW}{\ensuremath{\mathsf{W}}}
\newcommand{\sfR}{\ensuremath{\mathsf{R}}}
\newcommand{\sfT}{\ensuremath{\mathsf{T}}}
\newcommand{\sfZ}{\ensuremath{\mathsf{Z}}}

%%%%%%%%%%%%  Special letters for Lie groups %%%%%%%%%%%%%%%%%%

\newcommand{\biA}{{\mathbi{A}}}
\newcommand{\biB}{{\mathbi{B}}}
\newcommand{\biC}{{\mathbi{C}}}
\newcommand{\biD}{{\mathbi{D}}}
\newcommand{\biE}{{\mathbi{E}}}
\newcommand{\biF}{{\mathbi{F}}}
\newcommand{\biG}{{\mathbi{G}}}
\newcommand{\biH}{{\mathbi{H}}}
\newcommand{\biI}{{\mathbi{I}}}
\newcommand{\biJ}{{\mathbi{J}}}
\newcommand{\biK}{{\mathbi{K}}}
\newcommand{\biL}{{\mathbi{L}}}
\newcommand{\biM}{{\mathbi{M}}}
\newcommand{\biN}{{\mathbi{N}}}
\newcommand{\biO}{{\mathbi{O}}}
\newcommand{\biP}{{\mathbi{P}}}
\newcommand{\biQ}{{\mathbi{Q}}}
\newcommand{\biS}{{\mathbi{S}}}
\newcommand{\biU}{{\mathbi{U}}}
\newcommand{\biX}{{\mathbi{X}}}
\newcommand{\biY}{{\mathbi{Y}}}
\newcommand{\biV}{{\mathbi{V}}}
\newcommand{\biW}{{\mathbi{W}}}
\newcommand{\biR}{{\mathbi{R}}}
\newcommand{\biT}{{\mathbi{T}}}
\newcommand{\biZ}{{\mathbi{Z}}}




%%%%%%%%%%%%%%%%%%%%%%%%%%%%%%%%%%%%%%%%%%%%%%%%%%%%%%%%%%%%%%%%
\newcommand{\SU}{\mathrm{SU}}
\newcommand{\SO}{\mathrm{SO}}
\newcommand{\SL}{\mathrm{SL}}
\newcommand{\Sp}{\mathrm{Sp}}
\newcommand{\U}{\mathrm{U}}
\newcommand{\ul}{\mathrm{u}}
\newcommand{\Spin}{\mathrm{Spin}}
\newcommand{\Pin}{\mathrm{Pin}}
\newcommand{\PSL}{\mathrm{PSL}}
%%%%%%%%%%%%%%%%%%%%%%%%%%%%%%%%%%%%%%%%%%%%%%%%%%%%%%%%%%%%%%%%




\def \be  {\begin{equation}}
\def \ee  {\end{equation}}
\def \bea {\begin{equation}\begin{aligned}}
\def \eea {\end{aligned}\end{equation}}
\def \ba  {\begin{eqnarray}}
\def \ea  {\end{eqnarray}}

\begin{document}\thispagestyle{empty}

\centerline{\Large \bf Homework 2: Due at class on Oct 18}






\section{Lorentz group}


\subsection{}\label{1.1}



Show that the generators
$$
J^{\mu \nu}=i\left(x^{\mu} \partial^{\nu}-x^{\nu} \partial^{\mu}\right)
$$
obey the Lorentz algebra
\be\label{Lorentz-algebra}\left[J^{\mu \nu}, J^{\rho \sigma}\right]=i\left(g^{\nu \rho} J^{\mu \sigma}-g^{\mu \rho} J^{\nu \sigma}-g^{\nu \sigma} J^{\mu \rho}+g^{\mu \sigma} J^{\nu \rho}\right)~.\ee


Find the commutation relations of $J^{\mu \nu}$ with the generators of translations $P^\rho=-i\partial^\rho$. These define the Lie algebra of
the Poincare group (Lorentz $+$ translations).

\subsection{}
A 4-vector $V$ is transformed as $V\to \exp (-\frac{i\omega_{\mu\nu}J^{\mu\nu}}{2})V$ under a Lorentz transformation.
Show that an infinitesimal Lorentz transformation can be written as
$$
-\frac{i\omega_{\mu\nu}J^{\mu\nu}}{2}=-\sum_ai\left(\theta_a L_a+\beta_a K_a\right)=\left(\begin{array}{cccc}{0} & {-\beta_{1}} & {-\beta_{2}} & {-\beta_{3}} \\ {-\beta_{1}} & {0} & {-\theta_{3}} & {\theta_{2}} \\ {-\beta_{2}} & {\theta_{3}} & {0} & {-\theta_{1}} \\ {-\beta_{3}} & {-\theta_{2}} & {\theta_{1}} & {0}\end{array}\right)
$$
where $K_a$ and $L_a$ generate Lorentz boosts and space-rotations, respectively.
Rewrite the Lorentz algebra in terms of $K_a$ and $L_a$.

\subsection{} Show that the matrices
$$
J_{a}^{+} \equiv \frac{1}{2}\left(L_{a}+i K_{a}\right), \quad J_{a}^{-} \equiv \frac{1}{2}\left(L_{a}-i K_{a}\right)
$$
satisfy the commutation relations
$$
\left[J_{a}^{+}, J_{b}^{+}\right] =i \epsilon_{abc} J_{c}^{+}~, \quad \left[J_{a}^{-}, J_{b}^{-}\right] =i \epsilon_{abc} J_{c}^{-}~, \quad \left[J_{a}^{+}, J_{b}^{-}\right] =0~.
$$
Therefore, the Lorentz algebra can be understood as two copies of the $\mathfrak{su}(2)$ algebra
$$\left[s^{a}, s^{b}\right]=i \epsilon^{a b c} s^{c} $$
where \(s^{a}=\sigma^{a} / 2,\) $(a=1,2,3)$ are a half of the Pauli matrices.





\section{Clifford algebra}



\subsection{}
The gamma matrices form the Clifford algebra
\be\label{Clifford}\left\{\gamma^{\mu}, \gamma^{\nu}\right\}=2 g^{\mu \nu}~.\ee
Show that the generators of Lorentz transformations in the Dirac spinor representation
$$
S^{\mu \nu}=\frac{i}{4}\left[\gamma^{\mu}, \gamma^{\nu}\right]
$$
satisfy the Lorentz algebra \eqref{Lorentz-algebra}. Compute the commutation relation $[S^{\mu\nu},\gamma^\rho]$, and compare it with Problem \ref{1.1}.

\subsection{}
Use the Clifford algebra \eqref{Clifford} to show that \(\gamma^{5}=i \gamma^{0} \gamma^{1} \gamma^{2} \gamma^{3}\) obeys
$$
\left\{\gamma^{5}, \gamma^{\mu}\right\}=0, \quad\left(\gamma^{5}\right)^{2}=1~.
$$
Show that the chirality projection operators
$$P_{L}=\frac{1-\gamma^{5}}{2}, \quad P_{R}=\frac{1+\gamma^{5}}{2}$$
satisfy \(P_{L}^{2}=P_{L}, \quad P_{R}^{2}=P_{R}, \quad P_{L} P_{R}=P_{R} P_{L}=0\).

\subsection{}
Show that when $m=0$, $\psi \rightarrow e^{i \theta \gamma^{5}} \psi$ is a symmetry of the Dirac Lagrangian, which is called the \textbf{axial symmetry}. Find the corresponding Noether  current.


\subsection{}
Check that the basis of the gamma matrices chosen in Peskin-Schroeder
\be\label{gamma-convention}
\gamma^{0}=\left(\begin{array}{cc}0&1 \\ 1&0\end{array}\right) \quad \gamma^{i}=\left(\begin{array}{cc}{0} & {\sigma^{i}} \\ {-\sigma^{i}}&0\end{array}\right)
\ee
obey the Clifford algebra \eqref{Clifford}. Express $\gamma^5$ in this basis.

\section{Magnetic moment from Dirac equation}
The Dirac equation in the presence of the electromagnetic field is
$$
(i \slashed{D}-m) \psi=0
$$
where $D_\mu=\partial_\mu+ieA_\mu$. Using the convention \eqref{gamma-convention} of the gamma matrices in Peskin-Schroeder, show that non-relativistic limit of the equation for the plane wave
$$\psi(x)=\begin{pmatrix}u_L(p)\\ u_R(p)\end{pmatrix} e^{-ip\cdot x}$$
is
$$
H_{\textrm{nr}}=\frac{1}{2 m} \boldsymbol{\sigma} \cdot(\mathbf{p}-e \mathbf{A}) \boldsymbol{\sigma} \cdot(\mathbf{p}-e \mathbf{A})+e A^0
$$
Rewrite this Hamiltonian in the following form
$$
H_{\textrm{nr}}=\frac{1}{2 m}(\mathbf{p}-e \mathbf{A})^{2}-g \frac{e}{2 m} \mathbf{s} \cdot \mathbf{B}+e A^0
$$
and read off the $g$-factor. Here $\mathbf{s}=\boldsymbol{\sigma}/2$.



\section{Spinor identities}

\subsection{}
In the lecture, we see that the plane wave solutions to the Dirac equation are given by
$$
\psi=u_s(p) e^{-i p\cdot x}~, \quad \psi=v_s(p) e^{i p\cdot x}~,
$$
where
$$
u_s(p)=\begin{pmatrix}
\sqrt{ p \cdot \sigma} \xi_s \\ \sqrt{ p \cdot\overline  \sigma} \xi_s
\end{pmatrix}~, \quad  v_s(p)=\begin{pmatrix}
\sqrt{ p \cdot \sigma} \xi_s \\ -\sqrt{ p \cdot\overline  \sigma} \xi_s
\end{pmatrix}~.
$$
Taking the Weyl spinor basis as $\xi^\dagger_s \xi_{s'}=\delta_{ss'}$, show that if we sum over polarization states of the spinor, we have
$$
\sum_{s} u_{s}(p) \overline{u}_{s}(p)=\slashed{p}+m ~,\qquad \sum_{s} v_{s}(p) \overline{v}_{s}(p)=\slashed{p}-m ~.$$
Note that the formulas implicitly involves the spinor indices, which can be explict as
$$
\sum_{s} u_{s a}(p) \overline{u}_{s b}(p)=\gamma_{a b}^{\mu} p_{\mu}+m \delta_{ab}~.
$$

\subsection{}
Show
$\overline u_{s}(p) \gamma^{\mu} u_{s^{\prime}}(p)=2 p^{\mu} \delta_{s s^{\prime}}$.
This is a simple version of the \textbf{Gordon identity}.

\end{document}
