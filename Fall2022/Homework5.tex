\documentclass[12pt,a4paper]{article}
\usepackage{macros}

\begin{document}\thispagestyle{empty}

\centerline{\Large \bf Homework 5: Due at class on Nov 23}

\section{Derivation}
Derive the equation (7.36) and (7.37) of the lecture note.



\section{Bhabha scattering}
In this probelm, we consider the process of electron-positron \(\left(e^{+} e^{-} \rightarrow e^{+} e^{-}\right)\) scattering in QED.

\subsection{}
Compute the amplitude for this process by summing two Feynman diagrams. Your
answer will depend on the polarizations of the initial and final electrons and positrons.
Note that, unlike the \(e^{+}e^-\rightarrow \mu^{+} \mu^{-}\) process considered in class, your answer will be the
sum of two Feynman diagrams. The overall sign is not too important, but make sure to
get the relative sign between these two Feynman diagrams correct!
\subsection{}
Imagine that we do not measure the polarizations of the initial or final states. Compute
the scattering probability \(\frac{1}{4} \sum|\mathcal{M}|^{2},\) where we average over initial polarizations and sum
over final polarizations. Write your answer as a function of the momenta of the incoming
and outgoing particles. You may make the approximation where \(E_{c m}\gg m,\) so that we
can just ignore the electron mass.
\subsection{}

Now compute the differential cross section in centre of mass frame, and write it as a
function of the Mandelstam variables as
$$
\frac{d \sigma}{d \cos \theta}=\frac{\pi \alpha^{2}}{s}\left(u^{2}\left(\frac{1}{s}+\frac{1}{t}\right)^{2}+\left(\frac{t}{s}\right)^{2}+\left(\frac{s}{t}\right)^{2}\right)
$$
Take the reference frame as in Peskin-Schroeder above Eqn(5.11), write this formula as a function of \(E_{c m}\) and \(\theta\).
Note that the result diverges at \(\theta \rightarrow 0 .\) Can you explain why?





\section{Integral}
Perform the following integral and show the identity
$$ \int \frac{d^{4} k}{(2 \pi)^{4}} \frac{1}{\left[k^{2}-\Delta+i \epsilon\right]^{n}} =(-1)^{n} \frac{i}{16 \pi^{2}} \frac{1}{(n-1)(n-2)} \frac{1}{\Delta^{n-2}}~,$$
in the following two ways, and check both give the same answer:
\begin{itemize}
    \item Start with the calculation of the \(d k^{0}\) integral via Cauchy's theorem and then
    integrate over \(d^{3} k\).
    \item Alternatively the \(k^{0}\) integration can be done via a Wick
   rotation and afterwards substituting \(k^{0}=i k_{E}^{0},(\boldsymbol{k} \equiv \boldsymbol{k}_{E}) .\) Then the integral
   can be done in the four-dimensional Euclidean space (with \(\left.k^{2}=-((k_{E}^{0}\right)^{2}+\boldsymbol{k}^{2}))\)
   For that purpose you will need the integral over the four-dimensional solid angle
   \(\int d \Omega_{4}=2 \pi^{2} .\)
\end{itemize}




\section{1-loop correction in scalar QED}
The Lagrangian of the scalar QED is given by
\begin{equation}\label{lag2}
{\cal L}_{\rm scalar}=-\frac14F_{\mu\nu}F^{\mu\nu}+(D_\mu \phi)^*D^\mu \phi-m^2 \phi^*\phi
\end{equation}
where the covariant derivative is expressed by $D_\mu=\partial_\mu+ieA_\mu$.
%
% \subsection{}
% Find Feynman rules for the scalar QED.
% \subsection{}
Find the diagrams which contribute to the 1PI diagrams of the scalar vertex in  the order ${\cal O}(e^3)$  and write down the integral expression  for
\bea
-ie K^\mu(p,p')\equiv{\raisebox{-1.2cm}{\includegraphics[width=3cm]{vertex}}}=-ie (p+p')^\mu+e^3K_{\textrm{1-loop}}^\mu(p,p')+ {\cal O}(e^5)~.
\eea
Perform momentum integration by using the Pauli-Villars regularization.

% \subsection{}
% Find the diagrams which contribute to the 1PI diagrams of the scalar propagator in the order ${\cal O}(e^2)$  and write down the integral expression  for $\Pi_{\phi}^{\textrm{1-loop}}({k})$
% \bea
% i\Pi_{\phi}(k)\equiv{\raisebox{-.3cm}{\includegraphics[width=2.5cm]{phi-1PI}}}=e^2\Pi_{\phi}^{\textrm{1-loop}}({k}) + {\cal O}(e^4)~.
% \eea
% Perform momentum integration by using the Pauli-Villars regularization.
%
%
%
%
% \section{Bremsstrahlung}
% In the lecture, the IR divergence of the electron vertex is given by
% \begin{equation}
% \begin{aligned} \Delta F_{1}(q^{2}) &=F_{1}(q^{2})-F_{1}(0)=\frac{\alpha}{2 \pi} C_{\mathrm{IR}}(q^{2}) \times \log \left(\frac{-q^{2}}{\mu^{2}}\right)+\text { regular terms } \\ C_{\mathrm{IR}}(q^{2}) &=-1+\int_{0}^{1} d \xi\left(\frac{m^{2}-q^{2} / 2}{m^{2}-q^{2} \xi(1-\xi)}\right) ~.\end{aligned}
% \end{equation}
% This cancels with the infrared singularity in the \(\mathcal{O}(\alpha)\) cross-section for the single photon emission,
% $$
% d \sigma\left[e^{-}(p) \rightarrow e^{-}\left(p^{\prime}\right)+\gamma\right]=d \sigma_{0}\left[e^{-}(p) \rightarrow e^{-}\left(p^{\prime}\right)\right] \times \frac{\alpha}{\pi} I\left(\beta, \beta^{\prime}\right) \times \frac{d E_{\gamma}}{E_{\gamma}}
% $$
% where
% $$
% I\left(\beta, \beta^{\prime}\right)=\int \frac{d \Omega_{\boldsymbol{k}}}{4 \pi}|\boldsymbol{k}|^{2}\left[\frac{2 p \cdot p^{\prime}}{(k \cdot p)\left(k \cdot p^{\prime}\right)}-\frac{m^{2}}{(k \cdot p)^{2}}-\frac{m^{2}}{\left(k \cdot p^{\prime}\right)^{2}}\right]~.
% $$
% Show that \(2 C_{\mathrm{IR}}(q^{2})=I\left(\beta, \beta^{\prime}\right)\).
\end{document}
