

\documentclass[a4paper,11pt]{article}
\usepackage[top=3cm, bottom=3cm, left=2cm, right=2cm]{geometry}
\usepackage{CJK}

\usepackage{hyperref} % Use the Charter font for the document text
%\usepackage[UTF8]{ctex}
\usepackage{fullpage}
\usepackage{xcolor}
\usepackage{graphicx}



\definecolor{darkblue}{rgb}{0.1,0.1,0.7}
\definecolor{darkred}{rgb}{0.5,0.1,0.1}
\definecolor{darkgreen}{rgb}{0.0,0.42,0.06}
\hypersetup{colorlinks=true,urlcolor=darkblue,linkcolor=darkblue,citecolor=darkred}



\begin{document}\thispagestyle{empty}
\begin{CJK}{UTF8}{gbsn}

\centerline{\Large \bf Syllabus}

\begin{description}
\item{\bf Course name:} Quantum Field Theory I (PHYS130069.01)
\item{\bf Instructor:} Satoshi Nawata, Physics S422, Jiangwan \href{mailto:snawata@fudan.edu.cn}{snawata@fudan.edu.cn}
\item{\bf Teaching Assistant:} Jiaqun Jiang, \href{mailto:20210190013@fudan.edu.cn}{20210190013@fudan.edu.cn}\\ \hspace{6cm} Jiahao Zheng, \href{mailto:21210190042@m.fudan.edu.cn}{21210190042@m.fudan.edu.cn}
\item{\bf Hours:}  Wednesday 9:55 -- 12:30
\item{\bf Place:} H4205
\item{\bf Office hour:} Sunday 10am-12am, or question via Wechat.
\item{\bf Prerequisites:} Quantum mechanics, Electrodynamics, Classical Mechanics
\item{\bf About the course:}

Quantum field theory is just quantum mechanics for an infinite number of degrees of freedom. The Standard Model in particle physics was formulated in the framework of quantum field theory. In addition, it has been successfully applied to many-body systems in condensed matter physics.

In this course, I will introduce to some basics of quantum field theory. Because of infinite degrees of freedom, quantum field theory not only involves many new conceptual
points but also many technically complicated computations. Therefore, this course will guide you, step by step, to develop the crucial computational techniques and to learn important concepts. Starting from canonical quantization, we will learn perturbative quantum field theory and quantum electrodynamics.

The lecture note will be updated in \href{https://www.overleaf.com/read/jnjssfjftwmf}{https://www.overleaf.com/read/jnjssfjftwmf}. The course will continue to Spring 2023 during which we will cover $\beta$-functions, Yang-Mills theory, quantum chromodynamics, and hopefully Standard Model.

Any students are very welcome to audit this course.

\item{\bf Main content:}
\begin{itemize}
\item Free fields and canonical quantization
\item S-matrix and Feynman rules
\item Scattering amplitudes
\item Quantum Electrodynamics
\item Loop diagrams and quantum corrections
\item Path integrals
\end{itemize}

\item{\bf Main textbook:}

An Introduction to Quantum Field Theory. Michael E. Peskin, Daniel V. Schroeder

\item{\bf Supplementary textbooks:}

Quantum Field Theory, Mark Srednicki

Quantum field theory and the standard model, Matthew Dean Schwartz.

The Quantum Theory of Field, Steven Weinberg

\item{\bf Grade evaluation:} Grade will be determined based on homework sets (60\%) given every other week and the final test (40\%).


\end{description}


\end{CJK}


\end{document}
