\documentclass[12pt,a4paper]{article}
\usepackage{macros} 

\begin{document}\thispagestyle{empty}

\centerline{\Large \bf Homework 2: Due at class on Oct 12}






\section{Lorentz group}


\subsection{}\label{1.1}



Show that the generators
$$
J^{\mu \nu}=i\left(x^{\mu} \partial^{\nu}-x^{\nu} \partial^{\mu}\right)
$$
obey the Lorentz algebra
\be\label{Lorentz-algebra}\left[J^{\mu \nu}, J^{\rho \sigma}\right]=i\left(g^{\nu \rho} J^{\mu \sigma}-g^{\mu \rho} J^{\nu \sigma}-g^{\nu \sigma} J^{\mu \rho}+g^{\mu \sigma} J^{\nu \rho}\right)~.\ee


Find the commutation relations of $J^{\mu \nu}$ with the generators of translations $P^\rho=-i\partial^\rho$. These define the Lie algebra of
the Poincare group (Lorentz $+$ translations).

\subsection{}
A 4-vector $V$ is transformed as $V\to \exp (-\frac{i\omega_{\mu\nu}J^{\mu\nu}}{2})V$ under a Lorentz transformation.
Show that an infinitesimal Lorentz transformation can be written as
$$
-\frac{i\omega_{\mu\nu}J^{\mu\nu}}{2}=-\sum_ai\left(\theta_a L_a+\beta_a K_a\right)=\left(\begin{array}{cccc}{0} & {-\beta_{1}} & {-\beta_{2}} & {-\beta_{3}} \\ {-\beta_{1}} & {0} & {-\theta_{3}} & {\theta_{2}} \\ {-\beta_{2}} & {\theta_{3}} & {0} & {-\theta_{1}} \\ {-\beta_{3}} & {-\theta_{2}} & {\theta_{1}} & {0}\end{array}\right)
$$
where $K_a$ and $L_a$ generate Lorentz boosts and space-rotations, respectively.
Rewrite the Lorentz algebra in terms of $K_a$ and $L_a$.

\subsection{} Show that the matrices
$$
J_{a}^{+} \equiv \frac{1}{2}\left(L_{a}+i K_{a}\right), \quad J_{a}^{-} \equiv \frac{1}{2}\left(L_{a}-i K_{a}\right)
$$
satisfy the commutation relations
$$
\left[J_{a}^{+}, J_{b}^{+}\right] =i \epsilon_{abc} J_{c}^{+}~, \quad \left[J_{a}^{-}, J_{b}^{-}\right] =i \epsilon_{abc} J_{c}^{-}~, \quad \left[J_{a}^{+}, J_{b}^{-}\right] =0~.
$$
Therefore, the Lorentz algebra can be understood as two copies of the $\mathfrak{su}(2)$ algebra
$$\left[s^{a}, s^{b}\right]=i \epsilon^{a b c} s^{c} $$
where \(s^{a}=\sigma^{a} / 2,\) $(a=1,2,3)$ are a half of the Pauli matrices.





\section{Derivations}



\subsection{}
Derive the canonical anticommutation relations
\begin{equation}
\begin{gathered}
\left\{\psi_a(\boldsymbol{x}), \psi_b^{\dagger}(\boldsymbol{y})\right\}=\delta^{(3)}(\boldsymbol{x}-\boldsymbol{y}) \delta_{a b} \\
\left\{\psi_a(\boldsymbol{x}), \psi_b(\boldsymbol{y})\right\}=\left\{\psi_a^{\dagger}(\boldsymbol{x}), \psi_b^{\dagger}(\boldsymbol{y})\right\}=0
\end{gathered}
\end{equation}
from the following relations
\begin{equation}
\left\{a_{\boldsymbol{p}}^r, a_{\boldsymbol{q}}^{s \dagger}\right\}=\left\{b_{\boldsymbol{p}}^r, b_{\boldsymbol{q}}^{s \dagger}\right\}=(2 \pi)^3 \delta^{(3)}(\boldsymbol{p}-\boldsymbol{q}) \delta^{r s},
\end{equation}
and others anti-commutes. Or the other way around, namely derive from $\psi$-$\psi^\dagger$ anticommutation relations to $a$-$b$ anticommutation relations.


\subsection{}
Derive the expression of the Hamiltonian
$$
H=\int \frac{d^3 p}{(2 \pi)^3} E_p \sum_s\left[a_{\boldsymbol{p}}^{s \dagger} a_{\boldsymbol{p}}^s+b_{\boldsymbol{p}}^{s \dagger} b_{\boldsymbol{p}}^s-(2 \pi)^3 \delta(0)\right]
$$
from
\begin{equation}
\mathcal{H}=\overline{\psi}\left(-i \gamma^i \partial_i+m\right) \psi~.
\end{equation}


\subsection{}
Show that when $m=0$, $\psi \rightarrow e^{i \theta \gamma^{5}} \psi$ is a symmetry of the Dirac Lagrangian, which is called the \textbf{axial symmetry}. Find the corresponding Noether  current.


\section{Magnetic moment from Dirac equation}
The Dirac equation in the presence of the electromagnetic field is
$$
(i \slashed{D}-m) \psi=0
$$
where $D_\mu=\partial_\mu+ieA_\mu$. Using the convention of the gamma matrices in Peskin-Schroeder, show that non-relativistic limit of the equation for the plane wave
$$\psi(x)=\begin{pmatrix}u_L(p)\\ u_R(p)\end{pmatrix} e^{-ip\cdot x}$$
is
$$
H_{\textrm{nr}}=\frac{1}{2 m} \boldsymbol{\sigma} \cdot(\boldsymbol{p}-e \boldsymbol{A}) \boldsymbol{\sigma} \cdot(\boldsymbol{p}-e \boldsymbol{A})+e A^0
$$
Rewrite this Hamiltonian in the following form
$$
H_{\textrm{nr}}=\frac{1}{2 m}(\boldsymbol{p}-e \boldsymbol{A})^{2}-g \frac{e}{2 m} \boldsymbol{s} \cdot \boldsymbol{B}+e A^0
$$
and read off the $g$-factor. Here $\boldsymbol{s}=\boldsymbol{\sigma}/2$.



\section{Spinor identities}

\subsection{}
In the lecture, we see that the plane wave solutions to the Dirac equation are given by
$$
\psi=u_s(p) e^{-i p\cdot x}~, \quad \psi=v_s(p) e^{i p\cdot x}~,
$$
where
$$
u_s(p)=\begin{pmatrix}
\sqrt{ p \cdot \sigma} \xi_s \\ \sqrt{ p \cdot\overline  \sigma} \xi_s
\end{pmatrix}~, \quad  v_s(p)=\begin{pmatrix}
\sqrt{ p \cdot \sigma} \xi_s \\ -\sqrt{ p \cdot\overline  \sigma} \xi_s
\end{pmatrix}~.
$$
Taking the Weyl spinor basis as $\xi^\dagger_s \xi_{s'}=\delta_{ss'}$, show that if we sum over polarization states of the spinor, we have
$$
\sum_{s} u_{s}(p) \overline{u}_{s}(p)=\slashed{p}+m ~,\qquad \sum_{s} v_{s}(p) \overline{v}_{s}(p)=\slashed{p}-m ~.$$
Note that the formulas implicitly involves the spinor indices, which can be explict as
$$
\sum_{s} u_{s a}(p) \overline{u}_{s b}(p)=\gamma_{a b}^{\mu} p_{\mu}+m \delta_{ab}~.
$$

\subsection{}
Show
$\overline u_{s}(p) \gamma^{\mu} u_{s^{\prime}}(p)=2 p^{\mu} \delta_{s s^{\prime}}$.
This is a simple version of the \textbf{Gordon identity}.

\section{Bonus problems}
\subsection{}
The gamma matrices form the Clifford algebra
\be\label{Clifford}\left\{\gamma^{\mu}, \gamma^{\nu}\right\}=2 g^{\mu \nu}~.\ee
Show that the generators of Lorentz transformations in the Dirac spinor representation
$$
S^{\mu \nu}=\frac{i}{4}\left[\gamma^{\mu}, \gamma^{\nu}\right]
$$
satisfy the Lorentz algebra \eqref{Lorentz-algebra}. Compute the commutation relation $[S^{\mu\nu},\gamma^\rho]$, and compare it with Problem \ref{1.1}.

\subsection{}
Use the Clifford algebra \eqref{Clifford} to show that \(\gamma^{5}=i \gamma^{0} \gamma^{1} \gamma^{2} \gamma^{3}\) obeys
$$
\left\{\gamma^{5}, \gamma^{\mu}\right\}=0, \quad\left(\gamma^{5}\right)^{2}=1~.
$$
Show that the chirality projection operators
$$P_{-}=\frac{1-\gamma^{5}}{2}, \quad P_{+}=\frac{1+\gamma^{5}}{2}$$
satisfy \(P_{-}^{2}=P_{-}, \quad P_{+}^{2}=P_{+}, \quad P_{-} P_{+}=P_{+} P_{-}=0\).

\subsection{}
Check that the basis of the gamma matrices chosen in Peskin-Schroeder
\be\label{gamma-convention}
\gamma^{0}=\left(\begin{array}{cc}0&1 \\ 1&0\end{array}\right) \quad \gamma^{i}=\left(\begin{array}{cc}{0} & {\sigma^{i}} \\ {-\sigma^{i}}&0\end{array}\right)
\ee
obey the Clifford algebra \eqref{Clifford}. Express $\gamma^5$ in this basis.


\end{document}
