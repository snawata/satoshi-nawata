\documentclass[12pt,a4paper]{article}
\usepackage{macros}

\begin{document}\thispagestyle{empty}

\centerline{\Large \bf Homework 3: Due at class on Oct 26}


\section{More practice with gamma matrices}
Show the following identities without using explicit representation of the matrices:
$$
\operatorname{Tr}\left(\gamma^{\mu} \gamma^{\nu}\right)=4 g^{\mu \nu}
$$

$$
\operatorname{Tr}\left(\gamma^{\mu} \gamma^{\nu} \gamma^{\kappa} \gamma^{\lambda}\right)=4 (g^{\mu \nu} g^{\kappa \lambda}-g^{\mu \kappa} g^{\nu \lambda}+g^{\mu \lambda} g^{\nu \kappa})$$

$$ \operatorname{Tr}\left(\gamma^{5} \gamma^{\mu} \gamma^{\nu} \gamma^{\kappa} \gamma^{\lambda}\right)=-4 i \epsilon^{\mu \nu \kappa \lambda}
$$


$$
\operatorname{Tr}\left(\gamma^{\mu_{1}} \gamma^{\mu_{2}} \ldots \gamma^{\mu_{2 n+1}}\right)=0
$$




\section{Feymann diagrams for simple integrals (Alex Maloney's homework)}

In this problem we will consider a toy version of version of scalar $\phi^{4}$ theory, where instead of
computing a full path integral we compute a single integral:
\be \label{partition-fn}
Z(\lambda)=\frac{1}{\sqrt{2\pi}}\int_{-\infty}^{\infty} d \phi \exp \left\{-\frac{1}{2} \phi^{2}-\frac{\lambda}{4 !} \phi^{4}\right\}
\ee
where $\phi$ is just a real number, rather than a field. This is the path integral for ``0 -dimensional scalar $\phi^{4}$
field theory".

\subsection{}\label{a}
Compute $Z(0)$, $Z(0.01)$, $Z(0.1)$, $Z(0.4)$, $Z(1)$, $Z(5)$ numerically by using your favourite numerical computing program. Ask or collaborate with your friends if you don't have one.

\subsection{}
Replace $\exp \left(-\lambda \phi^{4} / 24\right)$
with its series expansion in $\lambda$. Find the complete series expansion
in $\lambda$ in closed form:
\be\label{eqn1}
Z(\lambda)=\sum_{n=0}^{\infty} c_{n} \lambda^{n}
\ee
and find an explicit expression for $c_{n}$. (Do this by expanding the exponent, exchanging
orders of summation and integration, and doing the integral for each term in the series.)

\subsection{}\label{b}
Let us now develop the Feynman diagram expansion for $Z(\lambda)$.
Explain how each $c_{n}$ could be computed by using (appropriately modified) Feynman
rules for our simple theory. Describe the Feynman rules for this theory and draw the
appropriate ``Feynman Diagrams" for the first few terms $c_{n}$. Check that this matches
the answer you got in Problem \ref{b} for the first few $c_{n}$. Note that we are computing
the analog of ``vacuum" diagrams in QFT.

\subsection{}
Evaluate the order $\lambda^{0}$ and $\lambda^{1}$ terms in this series numerically for the values of $\lambda$ given in
Problem \ref{a}. For which values of $\lambda$ does the order $\lambda^{1}$ term help improve the accuracy of our
perturbative expansion for $Z(\lambda) ?$



\subsection{}
What happens to the integral $Z(\lambda)$ when $\lambda$ is negative? Conclude from your answer that
the radius of convergence (in $\lambda$) of the series you developed in Problem \ref{b} must be zero.

\subsection{}\label{c}
Using Stirling's approximation, find the asymptotic form of the $c_{n}$ for large $n$.

\smallskip

Now consider the individual terms $\lambda^{n} c_{n}$ in the series for $Z(\lambda)$. Show that, for fixed $\lambda$,
these terms will decrease as function of $n$ until they reach a minimum at some critical
value of $n$  (call it $n_{o}(\lambda)$), after which they start increasing until they diverge at $n \rightarrow \infty$.
Use your approximate form for $c_{n}$ to estimate $n_{0}(\lambda)$ for small $\lambda$. This shows explicitly
that the radius of convergence of this series is zero.

\smallskip

Now, show that the smallest term in the series is
$$
\lambda^{n_{0}} c_{n_{0}} \sim e^{f(\lambda)}
$$
where $f(\lambda)$ is a function you should determine. You need only determine the leading
behaviour of $f(\lambda)$ at small $\lambda$.

\subsection{}
Based on this, you might conclude that the series in Problem \ref{b} is useless. But that is not
the case at all; it still contains lots of useful information about $Z(\lambda) !$
The expansion
$$
e^{-x}=\sum_{m=0}^{\infty} \frac{(-1)^{m}}{m !} x^{m}=1-x+\frac{x^{2}}{2}-\frac{x^{3}}{6}+\ldots
$$
has the following property for $x>0:$ the partial sums
$$
f_{n}(x) \equiv \sum_{m=0}^{n} \frac{(-1)^{m}}{m !} x^{m}
$$
are alternately strict over-estimates and strict under-estimates of the actual function;
that is, for $x>0, f_{0}(x)=1>e^{-x}, f_{1}(x)=(1-x)<e^{-x}, f_{2}(x)=\left(1-x+x^{2} / 2\right)>e^{-x}$
and so forth with the $<,>$ alternating.

\smallskip

Use this property to show that the partial sums found above (i.e. \eqref{eqn1} with $n$ cut off at
$0,1,2,3, \ldots),$ are alternately over-estimates and under-estimates of $Z(\lambda)$. Therefore, the
true answer $Z(\lambda)$ always lies between neighbouring terms in the series of partial sums.
Use this property to find a bound for $Z(\lambda)$ at $\lambda=1,$ by evaluating alternating terms
until they start to diverge. How tight is the bound? Repeat for $Z(0.4)$ and $Z(0.1)$. How
does this compare with your numerical answer for $Z$?

\smallskip


Argue that, as $\lambda$ gets smaller and smaller, one can use the Feynman diagram expansion
to place tighter and tighter bounds on $Z(\lambda)$. Estimate for how tight the bound will be
(i.e. how big the error terms will be) as a function of $\lambda,$ when $\lambda$ is small. You may find
the function $f(\lambda)$ in Problem \ref{c} useful.

\smallskip


We conclude that, while the series does not converge, it gives us very good information
about the value of $Z(\lambda)$. A series with this property - zero radius of convergence but the
ability to give good information near the origin - is called an \emph{asymptotic series}. Feynman
diagram expansions in QFT typically only give asymptotic series, rather than convergent
series.


\subsection{}
Show that the Borel transform $$\mathcal{B} \mathcal{Z}(\lambda)=\sum_{\ell=0}^{\infty} \frac{c_{\ell} \lambda^{\ell}}{\ell!} $$ converges provided $|\lambda|<\frac{3}{2}$ and that in this case
$$
\mathcal{Z}(\lambda)=\int_{0}^{\infty} \mathrm{d} z \mathrm{e}^{-z} \mathcal{B} Z(z \lambda)
$$
so that $\mathcal{Z}(\lambda)$ may be recovered from its Borel transform.

\subsection{}
By expanding $\mathrm{e}^{-\frac{1}{2} \phi^{2}}$ in the integral \eqref{partition-fn} obtain the strong coupling expansion
$$
\mathcal{Z}(\lambda)=\frac{1}{2 \sqrt{\pi}} \sum_{L=0}^{\infty} \frac{(-1)^{L}}{L !} \Gamma\left(\frac{L}{2}+\frac{1}{4}\right)\left(\frac{6}{\lambda}\right)^{\frac{L}{2}+\frac{1}{4}}
$$
for $\mathcal{Z}(\lambda)$ as a series in $1 / \sqrt{\lambda}$. For $\lambda=0.1$, how many terms does one need to obtain the value at which the weak coupling expansion appeared to converge?


\end{document}
