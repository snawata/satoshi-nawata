\documentclass[12pt,a4paper]{article}
\usepackage{macros}
\begin{document}\thispagestyle{empty}

\centerline{\Large \bf Homework 4: Due at class on Nov 9}


\section{Feynman rules in $\phi^{3}$ theory}
\subsection{}
Give the Feynman rules for correlation functions to the propagator, the vertex and the external points in position-
space and derive from these the Feynman rules in momentum-space for the \(\lambda \phi^{3}\) theory, i.e.
\(\mathcal{L}_{\text {int }}=-\frac{\lambda}{3 !} \phi^{3} .\)

\subsection{}
Calculate the symmetry factors for the following diagrams:
\begin{figure}[h]\centering
\includegraphics[width=3cm]{phi3-diag2}\qquad
\includegraphics[width=3cm]{phi3-diag3}\qquad
\includegraphics[width=5.5cm]{phi3-diag1}
\end{figure}
\subsection{}
Now let us consider the one-loop correction to the $\phi^3$ term, given by the following diagram.
\begin{figure}[h]\centering
\includegraphics[width=3cm]{phi3-diag4}
\end{figure}
Write down the corresponding amplitude using the Feynman rules
in terms of integrals over the intermediate points and Wick contractions,
represented with factors of \(D_{F}\).





\section{Peskin \& Schroeder Problem 4.3 (linear sigma model)}

\end{document}
