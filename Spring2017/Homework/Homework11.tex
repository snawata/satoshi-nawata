\documentclass[12pt,a4paper]{article}
\usepackage{hyperref} % Use the Charter font for the document text
%\usepackage[UTF8]{ctex}
\usepackage{fullpage}
\usepackage{tikz}
\usepackage{wrapfig}
\usepackage{bbm}
\usepackage{youngtab}
\usepackage{rotfloat}
\usepackage{stmaryrd}
\usepackage{amsfonts,amssymb,amsmath}
\usepackage{mathrsfs}
\usepackage{hyperref}
\usepackage{tikz-cd}
\usepackage{tcolorbox}




\def\bea{\begin{eqnarray}}
\def\eea{\end{eqnarray}}
\def\be{\begin{equation}}
\def\ee{\end{equation}}
\def\ba{\begin{align}}
\def\ea{\end{align}}

\renewcommand{\Im}{{\rm Im}}
\renewcommand{\Re}{{\rm Re}}
\newcommand{\Tr}{\mbox{Tr}}
\newcommand{\Pf}{\mbox{Pf}}
\newcommand{\sgn}{\mbox{sgn}}
\newcommand{\Vir}{{\rm Vir}}
\newcommand{\Li}{{\rm Li}}



\newcommand{\SU}{\mathrm{SU}}
\newcommand{\SO}{\mathrm{SO}}
\newcommand{\SL}{\mathrm{SL}}
\newcommand{\Sp}{\mathrm{Sp}}
\newcommand{\su}{\mathrm{su}}
\newcommand{\so}{\mathrm{so}}
\newcommand{\spl}{\mathrm{sp}}
\newcommand{\gl}{\mathrm{gl}}
\newcommand{\sll}{\mathrm{sl}}
\newcommand{\ul}{\mathrm{u}}
\newcommand{\GL}{\mathrm{GL}}




\newcommand{\bA}{\ensuremath{\mathbb{A}}}
\newcommand{\bB}{\ensuremath{\mathbb{B}}}
\newcommand{\bC}{\ensuremath{\mathbb{C}}}
\newcommand{\bD}{\ensuremath{\mathbb{D}}}
\newcommand{\bE}{\ensuremath{\mathbb{E}}}
\newcommand{\bF}{\ensuremath{\mathbb{F}}}
\newcommand{\bG}{\ensuremath{\mathbb{G}}}
\newcommand{\bH}{\ensuremath{\mathbb{H}}}
\newcommand{\bI}{\ensuremath{\mathbb{I}}}
\newcommand{\bJ}{\ensuremath{\mathbb{J}}}
\newcommand{\bK}{\ensuremath{\mathbb{K}}}
\newcommand{\bL}{\ensuremath{\mathbb{L}}}
\newcommand{\bM}{\ensuremath{\mathbb{M}}}
\newcommand{\bN}{\ensuremath{\mathbb{N}}}
\newcommand{\bO}{\ensuremath{\mathbb{O}}}
\newcommand{\bP}{\ensuremath{\mathbb{P}}}
\newcommand{\bQ}{\ensuremath{\mathbb{Q}}}
\newcommand{\bR}{\ensuremath{\mathbb{R}}}
\newcommand{\bS}{\ensuremath{\mathbb{S}}}
\newcommand{\bT}{\ensuremath{\mathbb{T}}}
\newcommand{\bU}{\ensuremath{\mathbb{U}}}
\newcommand{\bV}{\ensuremath{\mathbb{V}}}
\newcommand{\bW}{\ensuremath{\mathbb{W}}}
\newcommand{\bX}{\ensuremath{\mathbb{X}}}
\newcommand{\bY}{\ensuremath{\mathbb{Y}}}
\newcommand{\bZ}{\ensuremath{\mathbb{Z}}}

\renewcommand{\Im}{{\rm Im}}
\newcommand{\fraksp}{\ensuremath{\mathfrak{Z}}}


\newtheorem{lemma}{Lemma}[section]
\newtheorem{conjecture}[lemma]{Conjecture} 
\newtheorem{corollary}[lemma]{Corollary} 
\newtheorem{theorem}[lemma]{Theorem} 
\newtheorem{definition}[lemma]{Definition} 
\newtheorem{question}[lemma]{Question} 
\newtheorem{proposition}[lemma]{Proposition} 

\usepackage{graphicx}


\begin{document}\thispagestyle{empty}

\centerline{\Large \bf Homework 11: Due at class on May 19}

\subsection*{1. Pontryagin classes of sphere}
For the tangent bundle $TS^n$ of a sphere, show that its total Pontryagin class is trivial $p(TS^n)=1$.




\subsection*{2. Chern classes of $\bC P^n$}
An arbitrary point $\ell =[z_0;\cdots;z_n] \in\bC P^n$ is a complex line $\bC$ through the origin of $\bC^{n+1}$. So we can construct a complex line bundle $L$ over the complex projective space $\bC P^n$ as
\begin{align}
L&=\{ (\ell,w)\in \bC P^n\times \bC^{n+1}|\ w \in \ell\}~,\cr
&=\{ ([z_0;\cdots;z_n], w)\in \bC P^n\times \bC^{n+1}|\ w=\lambda \cdot (z_0,\cdots,z_n)  ~, \quad \lambda\in\bC\} ~.
\end{align}
(The notation is the same as in Problem 4 of Homework 9.) This is called the \textbf{Hopf line bundle}. If we consider points with unit length in each fiber of the Hopf line bundle, then we obtain a sphere
$$
S^{2n+1}=\{ (\ell,z)\in \bC P^n\times \bC^{n+1}| \ z \in \ell~, \ |z|=1\}~.
$$
Therefore, the Hope line bundle is the associated line bundle of the Hopf $S^1$-bundle. 

In fact, the Whiteney sum of the tangent bundle $T\bC P^n$ and the trivial bundle $\epsilon$ is the Whiteney sum of $(n+1)$ copies of $L^*$
$$
T\bC P^n \oplus \epsilon \cong L^*\oplus \cdots \oplus L^*~.
$$
where $L^*$ is the dual line bundle of $L$. (See Morita Proposition 5.13 for the derivation.)


Let us first consider  the case of $n=1$. We know that cohomology $H^2(\bC P^1;\bZ)\cong \bZ$ and its generator $x$ is the volume form. Show that $c_1(L)=-x$ and $c_1(T\bC P^1)=2x$. 
In addition, find the curvature of the Ehresmann connection 
$$A=i~ \textrm{Im}(\overline z_0dz_0 +\overline z_1dz_1 )$$ 
given in Problem 4 of Homework 9, and compare it with the volume form.
Actually, show that over $\bC P^1$ one can find an infinite family of non-equivalent complex line bundles. 


Find the total Chern class of $T\bC P^n$. In fact, the $2i$-th de Rham cohomology group of $\bC P^n$ is isomorphic to $\bR$ generated by $x^i=x\wedge \cdots \wedge x$: 
$$H^{2i}_{dR}(\bC P^n)\cong \bR\langle x^i \rangle  ~ \quad \textrm{for} \quad  i\le n~.$$
Moreover, the de Rham cohomology groups of $\bC P^n$ has a ring structure where the multiplication is given by the wedge product:
$$
H_{dR}^*(\bC P^n)\cong \bR[x]/(x^{n+1})~.
$$



\subsection*{3. Chern-Simons and Wess-Zumino-Witten}



Let us define the Chern-Simons action over a 3-manifold $M$
$$
S_{CS}=\frac{k}{4\pi} \int_M  \Tr\left(A\wedge dA +\frac23A\wedge A\wedge A\right)~.
$$
Under the gauge transformation 
$$
A \to  g^{-1}dg+g^{-1}Ag~,
$$
show that the action is transformed as
$$
S_{CS}\to S_{CS} +\frac{k}{4\pi}\int_Md\;\Tr(g^{-1}Ag\wedge g^{-1}dg)-\frac{k}{12\pi}\int_M\Tr((g^{-1}dg)^3)~.
$$
Let us consider the case $M=S^3$ and $G=\SU(2)$. Then, the second term vanishes because $S^3$ is closed. Show that the third term is
$$
\frac{k}{12\pi}\int_M\Tr((g^{-1}dg)^3)=2\pi k \deg(g) ~,
$$
where $\deg(g)$ is the mapping degree of
$$
g:S^3\to \SU(2)\cong S^3~.
$$
(Hint: $\mu=g^{-1}dg$ is the Maurer-Cartan form  and $\frac{1}{6}\Tr (\mu\wedge \mu\wedge \mu)$ is the left-invariant volume form of $\SU(2)$.) Argue that the level $k$ must be an integer in order for the partition function to be well-defined.


\subsection*{4. Chern-Simons from 4-dimension}

Show the following identity 
$$
\Tr(F\wedge F)=d\; \Tr\left(A\wedge dA +\frac23A\wedge A\wedge A\right)~.
$$
Therefore, if there exists a four-manifold $B$ whose boundary is $\partial B=M$, then we can indeed define Chern-Simons action by
$$
S_{CS}=\frac{k}{4\pi}\int_B \Tr(F\wedge F)~,
$$ 
when $G$ is compact and semi-simple.




















\end{document}

