\documentclass[12pt,a4paper]{article}
\usepackage{hyperref} % Use the Charter font for the document text
%\usepackage[UTF8]{ctex}
\usepackage{fullpage}
\usepackage{tikz}
\usepackage{wrapfig}
\usepackage{bbm}
\usepackage{youngtab}
\usepackage{rotfloat}
\usepackage{stmaryrd}
\usepackage{amsfonts,amssymb,amsmath}
\usepackage{mathrsfs}
\usepackage{hyperref}
\usepackage{tikz-cd}
\usepackage{tcolorbox}




\def\bea{\begin{eqnarray}}
\def\eea{\end{eqnarray}}
\def\be{\begin{equation}}
\def\ee{\end{equation}}
\def\ba{\begin{align}}
\def\ea{\end{align}}

\renewcommand{\Im}{{\rm Im}}
\renewcommand{\Re}{{\rm Re}}
\newcommand{\Tr}{\mbox{Tr}}
\newcommand{\Pf}{\mbox{Pf}}
\newcommand{\sgn}{\mbox{sgn}}
\newcommand{\Vir}{{\rm Vir}}
\newcommand{\Li}{{\rm Li}}



\newcommand{\SU}{\mathrm{SU}}
\newcommand{\SO}{\mathrm{SO}}
\newcommand{\SL}{\mathrm{SL}}
\newcommand{\Sp}{\mathrm{Sp}}
\newcommand{\su}{\mathrm{su}}
\newcommand{\so}{\mathrm{so}}
\newcommand{\spl}{\mathrm{sp}}
\newcommand{\gl}{\mathrm{gl}}
\newcommand{\sll}{\mathrm{sl}}
\newcommand{\ul}{\mathrm{u}}
\newcommand{\GL}{\mathrm{GL}}




\newcommand{\bA}{\ensuremath{\mathbb{A}}}
\newcommand{\bB}{\ensuremath{\mathbb{B}}}
\newcommand{\bC}{\ensuremath{\mathbb{C}}}
\newcommand{\bD}{\ensuremath{\mathbb{D}}}
\newcommand{\bE}{\ensuremath{\mathbb{E}}}
\newcommand{\bF}{\ensuremath{\mathbb{F}}}
\newcommand{\bG}{\ensuremath{\mathbb{G}}}
\newcommand{\bH}{\ensuremath{\mathbb{H}}}
\newcommand{\bI}{\ensuremath{\mathbb{I}}}
\newcommand{\bJ}{\ensuremath{\mathbb{J}}}
\newcommand{\bK}{\ensuremath{\mathbb{K}}}
\newcommand{\bL}{\ensuremath{\mathbb{L}}}
\newcommand{\bM}{\ensuremath{\mathbb{M}}}
\newcommand{\bN}{\ensuremath{\mathbb{N}}}
\newcommand{\bO}{\ensuremath{\mathbb{O}}}
\newcommand{\bP}{\ensuremath{\mathbb{P}}}
\newcommand{\bQ}{\ensuremath{\mathbb{Q}}}
\newcommand{\bR}{\ensuremath{\mathbb{R}}}
\newcommand{\bS}{\ensuremath{\mathbb{S}}}
\newcommand{\bT}{\ensuremath{\mathbb{T}}}
\newcommand{\bU}{\ensuremath{\mathbb{U}}}
\newcommand{\bV}{\ensuremath{\mathbb{V}}}
\newcommand{\bW}{\ensuremath{\mathbb{W}}}
\newcommand{\bX}{\ensuremath{\mathbb{X}}}
\newcommand{\bY}{\ensuremath{\mathbb{Y}}}
\newcommand{\bZ}{\ensuremath{\mathbb{Z}}}

\renewcommand{\Im}{{\rm Im}}
\newcommand{\fraksp}{\ensuremath{\mathfrak{Z}}}


\newtheorem{lemma}{Lemma}[section]
\newtheorem{conjecture}[lemma]{Conjecture} 
\newtheorem{corollary}[lemma]{Corollary} 
\newtheorem{theorem}[lemma]{Theorem} 
\newtheorem{definition}[lemma]{Definition} 
\newtheorem{question}[lemma]{Question} 
\newtheorem{proposition}[lemma]{Proposition} 

\usepackage{graphicx}


\begin{document}\thispagestyle{empty}

\centerline{\Large \bf Homework 10: Due at class on May 12}


\section{Instanton}
Let $(M,g)$ be an oriented closed 4-dimensional Riemannian manifold. We denote an orthonormal basis of one-forms $\Omega^1(M)$ by $e^1,\cdots,e^4$.
The Hodge star operator 
$$
\ast:\Omega^2(M)\to\Omega^2(M)
$$
is subject to $\ast \cdot\ast=1$. Therefore, the eigenvalues of the Hodge star operator $\ast$ on $\Omega^2(M)$ are $\pm1$ so that we denote the decomposition of $\Omega^2(M)$ by its eigenvalues 
\be\label{decomp}\Omega^2(M)=\Omega_+^2\oplus \Omega_-^2~.\ee
In fact, we can write a basis of $\Omega_+^2$ as
$$
e^1\wedge e^2+e^3\wedge e^4~,\quad e^1\wedge e^3+e^4\wedge e^2~,\quad e^1\wedge e^4+e^2\wedge e^3~, 
$$
whereas a basis of $\Omega_-^2$ is spanned by
$$
e^1\wedge e^2-e^3\wedge e^4~,\quad e^1\wedge e^3-e^4\wedge e^2~,\quad e^1\wedge e^4-e^2\wedge e^3~.
$$
Let $P$ be a principal $G$-bundle over $M$. In addition,  let $A$ be a connection and $F$ be its curvature. According to \eqref{decomp}, we can decompose
$$
F=F_++F_-~.
$$
If $F_-=0$, namely $\ast F=F$, then the connection $A$ is called \textbf{self-dual connection or instanton}. On the other hand, if $F_+=0$, the connection $A$ is called \textbf{anti-self-dual connection or anti-instanton}. Show that both self-dual and anti-self-dual connections satisfy the Yang-Mills equation $\ast d_A \ast F=0$.

By setting the Yang-Mills coupling $g_{YM}=1$, the Yang-Mills action is
$$
\mathscr{YM}(A)=\frac12\int_M\Tr(F\wedge \ast F)~.
$$
In addition, the integral of the first Pontryagin class $4 \pi^2 p_1(P)$ is called the first Pontryagin number
$$
p(P)= \frac12 \int_M\Tr (F\wedge F)~.
$$
Show the following equality 
$$
\mathscr{YM}(A)\ge  |p(P)|
$$
where the equality holds when $A$ is either self-dual or anti-self-dual connection.

\section{Belavin-Polyakov-Schwarz-Tyupkin instanton}

Let $\bH$ be the quaternion where an element $x\in\bH$ can be expressed as
$$x=x_1+x_2i+x_3j+x_4k$$ where $x_a\in\bR$ $(1\le a\le4)$ and
$$
i^2=j^2=k^2=-1~,\quad ij=-ji=k~, \quad jk=-kj=i~,\quad ki=-ik=j ~. 
$$
We define the imaginary part of $x$ as 
$$
\textrm{Im}\;x=x_2i+x_3j+x_4k
$$
so that the conjugate $\overline x$ is written as
$$
\overline x=x_1-x_2i-x_3j-x_4k
$$
Therefore, the multiplication becomes
$$
\overline{xy}=\overline y \cdot \overline x
$$
The norm of $x$ is 
$$
|x|^2=x\overline x=\overline x x=x_1^2+x_2^2+x_3^2+x_4^2
$$

The symplectic group $\Sp(1)=\{x\in\bH|~ |x|=1\} $ is  a Lie group, which is isomorphic to $\SU(2)$. Its Lie algebra $\mathfrak{sp}(1)$ can be written
$$
\mathfrak{sp}(1)=\Im \;\bH=\{x_2i+x_3j+x_4k|~x_2,x_3,x_4\in\bR\}
\quad 
\mathrm{with \ the \ Lie \ bracket }
\quad
[x,y]=\Im ~xy~.
$$

The connection $A$ of $\SU(2)$-principal bundle over $\bR^4$ can be understood as one-form taking value on $\Im \bH$
$$
A=\sum_{a=1}^4A_a(x)dx^a   ~,\qquad \mathrm{where} \quad A_a(x)\in \Im\; \bH~.
$$
Let us define the connection by 
$$
A=\Im \left\{  \frac{\overline  xdx}{1+|x|^2}\right\}
$$
Compute its curvature $F$ and show that it is anti-self-dual. This configuration is named as \textbf{Belavin-Polyakov-Schwarz-Tyupkin anti-instanton}. 
  
  \vspace{1cm}

Given a scale transformation $\lambda:\bH\to \bH; x\mapsto \frac x \lambda$ $(\lambda>0)$, the pull-back of the connection by this map 
$$
(\lambda^* A)(x)=\Im \left\{  \frac{\overline  xdx}{\lambda^2+|x|^2}\right\}
$$
is anti-self-dual. Furthermore, one can change the position of instanton by 
$$
A_{\lambda,x_0}(x)=\Im \left\{  \frac{\overline{(x-x_0)}dx}{\lambda^2+|x-x_0|^2}\right\}~.
$$
The 5 parameters $(\lambda,x_0)$ are called \textbf{moduli of instanton}.


\section{Prasad-Sommerfield monopole}




Let $P=\bR^3\times G$ be the trivial principal $G$-bundle over $\bR^3$ and $A$ be the connection of $P$. In addition, let $\Phi$ be a section of the adjoint bundle $P\times_{Ad} \mathfrak{g}$. Namely, the field $\Phi$ takes its value on the Lie algebra $ \mathfrak{g}$ and $D_A$ acts on it as $D_A\Phi=d \Phi + [A,\Phi]$. We define the action 
$$
S=\frac 12\int \textrm{Tr}( F\wedge \ast F) +\frac12 \int \Tr (D_A\Phi \wedge \ast D_A\Phi)~,
$$
called Yang-Mills-Higgs action. 
Show that the equation of motion for this action becomes
\begin{align}
&\ast d_A \ast F+[\Phi,D_A\Phi]=0\cr
& \ast d_A \ast D_A\Phi=0~.\nonumber
\end{align}
Show that these equations are equivalent to 
$$
\ast F=\pm D_A\Phi~,
$$
which is called \textbf{Bogomol'nyi equation or  non-Abelian monopole equation}.

For $G=\SU(2)$, show that the following configuration is the solution of  Bogomol'nyi equation
\begin{align}
A(x)&=\frac{i}{2}\left( \frac 1{\sinh r} -\frac 1r\right) \left(  \vec{r} \times \vec{\sigma}\right)\cdot d\vec{x}~,\cr
\Phi(x)&=\mp  \frac{i}{2}\left( \frac 1{\tanh r} -\frac 1r\right) \left(  \vec{r} \cdot \vec{\sigma}\right) ~,\nonumber
\end{align}
where $\vec{\sigma}=(\sigma_1,\sigma_2,\sigma_3)$ are Pauli matrices and $\vec{r}=(x_1,x_2,x_3)/r$ is the unit vector in $\bR^3$ with $r=|x|$. This is called  \textbf{Prasad-Sommerfield monopole}. Find the value of the action for this solution.








\end{document}

