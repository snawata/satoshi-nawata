\documentclass[12pt,a4paper]{article}
\usepackage{hyperref} % Use the Charter font for the document text
%\usepackage[UTF8]{ctex}
\usepackage{fullpage}
\usepackage{amsfonts,amssymb,amsmath}


\newcommand{\bA}{\ensuremath{\mathbb{A}}}
\newcommand{\bB}{\ensuremath{\mathbb{B}}}
\newcommand{\bC}{\ensuremath{\mathbb{C}}}
\newcommand{\bD}{\ensuremath{\mathbb{D}}}
\newcommand{\bE}{\ensuremath{\mathbb{E}}}
\newcommand{\bF}{\ensuremath{\mathbb{F}}}
\newcommand{\bG}{\ensuremath{\mathbb{G}}}
\newcommand{\bH}{\ensuremath{\mathbb{H}}}
\newcommand{\bI}{\ensuremath{\mathbb{I}}}
\newcommand{\bJ}{\ensuremath{\mathbb{J}}}
\newcommand{\bK}{\ensuremath{\mathbb{K}}}
\newcommand{\bL}{\ensuremath{\mathbb{L}}}
\newcommand{\bM}{\ensuremath{\mathbb{M}}}
\newcommand{\bN}{\ensuremath{\mathbb{N}}}
\newcommand{\bO}{\ensuremath{\mathbb{O}}}
\newcommand{\bP}{\ensuremath{\mathbb{P}}}
\newcommand{\bQ}{\ensuremath{\mathbb{Q}}}
\newcommand{\bR}{\ensuremath{\mathbb{R}}}
\newcommand{\bS}{\ensuremath{\mathbb{S}}}
\newcommand{\bT}{\ensuremath{\mathbb{T}}}
\newcommand{\bU}{\ensuremath{\mathbb{U}}}
\newcommand{\bV}{\ensuremath{\mathbb{V}}}
\newcommand{\bW}{\ensuremath{\mathbb{W}}}
\newcommand{\bX}{\ensuremath{\mathbb{X}}}
\newcommand{\bY}{\ensuremath{\mathbb{Y}}}
\newcommand{\bZ}{\ensuremath{\mathbb{Z}}}

\newcommand{\frakgl}{\ensuremath{\mathfrak{gl}}}
\newcommand{\fraksl}{\ensuremath{\mathfrak{sl}}}
\newcommand{\frakso}{\ensuremath{\mathfrak{so}}}
\newcommand{\fraksp}{\ensuremath{\mathfrak{sp}}}

\newcommand{\U}{\mathrm{U}}
\newcommand{\OO}{\mathrm{O}}

\newcommand{\SU}{\mathrm{SU}}
\newcommand{\SO}{\mathrm{SO}}
\newcommand{\SL}{\mathrm{SL}}
\newcommand{\Sp}{\mathrm{Sp}}
\newcommand{\su}{\mathrm{su}}
\newcommand{\so}{\mathrm{so}}
\newcommand{\spl}{\mathrm{sp}}
\newcommand{\gl}{\mathrm{gl}}
\newcommand{\sll}{\mathrm{sl}}
\newcommand{\ul}{\mathrm{u}}
\newcommand{\GL}{\mathrm{GL}}



\newtheorem{lemma}{Lemma}[section]
\newtheorem{conjecture}[lemma]{Conjecture} 
\newtheorem{corollary}[lemma]{Corollary} 
\newtheorem{theorem}[lemma]{Theorem} 
\newtheorem{definition}[lemma]{Definition} 
\newtheorem{question}[lemma]{Question} 
\newtheorem{proposition}[lemma]{Proposition} 

\usepackage{graphicx}


\begin{document}\thispagestyle{empty}

\centerline{\Large \bf Homework 8: Due at class on April 28}



 \vspace{.5cm}
\noindent 1. Let us identify $S^2 =\bC\cup \{\infty\}$. Then, a holomorphic map $g:\bC\to \bC;z\mapsto z^n$ can be extended to $g:S^2\to S^2$. Find the mapping degree $\deg g$ of $g$.



 
 \vspace{.5cm}
\noindent 2. \textbf{Fundamental theorem of algebra}

We define $f:\bC\to \bC$  by
$f(z) = z^n +a_1z^{n-1} +\cdots+a_n $ for $n\ge 1$.  In addition, by writing $z=x+iy$, we define one-form
$$
\omega=\textrm{Im}\frac{dz}{z}= \frac{-ydx}{x^2+y^2}+\frac{xdy}{x^2+y^2}~.
$$
Then, show that
$$
\frac{1}{2\pi} \int_{C_R} f^*\omega=n~,
$$
where $C_R$ is the circle with sufficiently large radius $R$. (Hint: construct homotopy between  $f$ and $g$ above.) If there were no zero points $f(z)=0$ inside $C_R$, show that 
$$
\frac{1}{2\pi} \int_{C_R} f^*\omega=0~
$$
by using the Stokes theorem.


 \vspace{.5cm}
\noindent 3. Show that Euler characteristics of a compact Lie group is zero.

 \vspace{.5cm}
\noindent 4.  Find real dimensions of the following Lie groups
\begin{itemize}
\item Complex general linear group: $\GL(n, \bC) = \{A \in \textrm{M}_n(\bC)| \det A \neq 0\}$
\item Complex special linear group: $\SL(n, \bC) = \{A \in \GL(n, \bC)| \det A = 1\}$
\item Unitary group $\U(n) = \{A \in \GL(n, \bC)| AA^\dagger = I\}$
\item Special unitary group $\SU(n) =\{A \in \U(n)| \det A = 1\}$
\item Real general linear group: $\GL(n, \bR) = \{A \in \textrm{M}_n(\bR)| \det A \neq 0\}$
\item Real special linear group: $\SL(n, \bR) = \{A \in \GL(n, \bR)| \det A = 1\}$
\item  Orthogonal group $\OO(n) =  \{A \in \GL(n, \bR)| AA^T = I\}$
\item Special orthogonal group $\SO(n) = \{A \in \OO(n)| \det A = 1\}$
\end{itemize}


 \vspace{.5cm}
\noindent 5.  Write down the definitions of the following Lie algebras:
$\frakgl(n,\bC)$, $\fraksl(n,\bC)$, $\mathfrak{su}(n)$ and $\frakso(n)$. 


 \vspace{.5cm}
\noindent 6. Show that the group of Lorentz transformations can be expressed $\SL(2,\bC)/\pm \textrm{Id}$. Hint: If we define
$$
A:=\left( \begin{array}{cc}t+x&z+yi\\ z-yi&t-x\end{array}\right)~
$$
where $(t,x,y,z)\in \bR^4$, then we have $t^2-x^2-y^2-z^2=\det A$.

 \vspace{.5cm}
\noindent 7. Show that $\SO(4)\cong  \{\SU(2) \times \SU(2)\}/\{\pm \textrm{Id}\}$, where $\textrm{Id} \hookrightarrow \SU(2) \times \SU(2)$ is the
diagonal embedding. The hint is given as follows.



Let $\bH$ be the quaternion in which an element $x\in\bH$ can be expressed as
$$x=x_1+x_2i+x_3j+x_4k$$ where $x_a\in\bR$ $( a=1,\cdots,4)$ and
$$
i^2=j^2=k^2=-1~,\quad ij=-ji=k~, \quad jk=-kj=i~,\quad ki=-ik=j ~. 
$$
We define the imaginary part of $x$ as 
$$
\textrm{Im}~x=x_2i+x_3j+x_4k
$$
so that the conjugate $\overline x$ is written as
$$
\overline x=x_1-x_2i-x_3j-x_4k
$$
Therefore, the multiplication becomes
$$
\overline{xy}=\overline y \cdot \overline x
$$
The norm of $x$ is 
$$
|x|^2=x\overline x=\overline x x=x_1^2+x_2^2+x_3^2+x_4^2
$$
From this viepoint, $\SU(2)$ can be considered as a group of unit quaternions $\SU(2)=\{x\in\bH|~ |x|=1\} $. Then $\SU(2)\times \SU(2)$ acts on $\mathbb{H}$ by rotations in the following way:
$$x\mapsto q_1 xq_2^{-1}$$
is a rotation of $\mathbb{R}^4=\mathbb{H}$ for $q_1$, $q_2\in \SU(2)$.
Then $(-q_1,-q_2)$ represents the same rotation as $(q_1,q_2)$. Show that these represent all the rotations of $\bR^4=\bH$ so that it is isomorphic to $\SO(4)$.


\end{document}

