\documentclass[12pt,a4paper]{article}
\usepackage{hyperref} % Use the Charter font for the document text
%\usepackage[UTF8]{ctex}
\usepackage{fullpage}
\usepackage{amsfonts,amssymb,amsmath}


\newcommand{\bA}{\ensuremath{\mathbb{A}}}
\newcommand{\bB}{\ensuremath{\mathbb{B}}}
\newcommand{\bC}{\ensuremath{\mathbb{C}}}
\newcommand{\bD}{\ensuremath{\mathbb{D}}}
\newcommand{\bE}{\ensuremath{\mathbb{E}}}
\newcommand{\bF}{\ensuremath{\mathbb{F}}}
\newcommand{\bG}{\ensuremath{\mathbb{G}}}
\newcommand{\bH}{\ensuremath{\mathbb{H}}}
\newcommand{\bI}{\ensuremath{\mathbb{I}}}
\newcommand{\bJ}{\ensuremath{\mathbb{J}}}
\newcommand{\bK}{\ensuremath{\mathbb{K}}}
\newcommand{\bL}{\ensuremath{\mathbb{L}}}
\newcommand{\bM}{\ensuremath{\mathbb{M}}}
\newcommand{\bN}{\ensuremath{\mathbb{N}}}
\newcommand{\bO}{\ensuremath{\mathbb{O}}}
\newcommand{\bP}{\ensuremath{\mathbb{P}}}
\newcommand{\bQ}{\ensuremath{\mathbb{Q}}}
\newcommand{\bR}{\ensuremath{\mathbb{R}}}
\newcommand{\bS}{\ensuremath{\mathbb{S}}}
\newcommand{\bT}{\ensuremath{\mathbb{T}}}
\newcommand{\bU}{\ensuremath{\mathbb{U}}}
\newcommand{\bV}{\ensuremath{\mathbb{V}}}
\newcommand{\bW}{\ensuremath{\mathbb{W}}}
\newcommand{\bX}{\ensuremath{\mathbb{X}}}
\newcommand{\bY}{\ensuremath{\mathbb{Y}}}
\newcommand{\bZ}{\ensuremath{\mathbb{Z}}}



\newtheorem{lemma}{Lemma}[section]
\newtheorem{conjecture}[lemma]{Conjecture} 
\newtheorem{corollary}[lemma]{Corollary} 
\newtheorem{theorem}[lemma]{Theorem} 
\newtheorem{definition}[lemma]{Definition} 
\newtheorem{question}[lemma]{Question} 
\newtheorem{proposition}[lemma]{Proposition} 

\usepackage{graphicx}


\begin{document}\thispagestyle{empty}

\centerline{\Large \bf Homework 3: Due at class on March 24}
\vspace{.5cm}

\noindent 1. Show that, for $v,w\in \Gamma(TM)$ and $f\in C^\infty(M)$, 
$$
[fv,w]=f[v,w]+w(f) v~,\qquad [v,fw]=f[v,w]+v(f) w~.
$$

\vspace{.5cm}
\noindent 2.  Let $\varphi_t:M\to M$ be the flow generated by a vector field $v\in \Gamma (TM)$. Then, show that for another vector field $w\in \Gamma(TM)$
$$
[v,w]=\lim_{t\to 0}\frac{(\varphi_{-t})*w-w}{t}
$$

\vspace{.5cm}
\noindent 3.  Given a vector field $v\in \Gamma(TM)$, the Lie derivative
$$
L_v:\Omega^k(M)\to \Omega^k(M)
$$
can be defined by
$$
L_v(\omega)=\lim_{t\to0}\frac{\varphi_t^*\omega-\omega}{t}
$$
where  $\varphi_t:M\to M$ be the flow  as above. Show that it satisfies
$$
L_v\omega(v_1,\cdots,v_k)=v\omega(v_1,\cdots,v_k)-\sum_{I=1}^k\omega(v_1,\cdots, [v,v_i],\cdots,v_k)
$$ 
for $v_1,\cdots,v_k\in \Gamma(TM)$. 


\vspace{.5cm}
\noindent 4. Define an $n$-form $\omega$ on the space $\bR^{n+1}\backslash\{0\}$, obtained from $\bR^{n+1}$ by removing the origin, by 
$$
\omega=\frac{1}{|x|^{n+1}}\sum_{i=1}^{n+1}(-1)^{i-1}x_idx_1\wedge \cdots \wedge \widehat{dx_i} \wedge \cdots \wedge dx_{n+1}~,
$$
where $|x|=(x_1^2+\cdots x_{n+1}^2)^{\frac12}$.
Prove that $d\omega=0$.

\vspace{.5cm}
\noindent 5. Let $S^n$ be the unit sphere in $\bR^{n+1}$. Find the value of the integral in the case of $n=2$
$$
\int_{S^2}\omega
$$
in the case of $n=2$ where $\omega$ is as above.

\newpage
\noindent 6. (Example 5.12. of Nakahara) 

A symplectic form on a smooth manifold $M$ is non-degenerate closed 2-form $\omega$. ``Non-degenerate'' means that the mapping $\omega : TM \to T^*M$; $X \mapsto \omega(X,-)$
 is an isomorphism. We denote the 1-form $\omega(X,-)$ by $i(X)\omega$.
 
 
 The couple ($M, \omega$) of a smooth manifold $M$ and a symplectic form $\omega$ is called a
symplectic manifold. Any symplectic manifold is even dimensional and if dim($M) =
2n$, $\omega^n$ is a volume-form. 
 
 
 
 
 Any smooth function $f\in C^\infty(M)$ gives rise to a vector field $X_f$ defined uniquely
by the equation
$$i(X_f )\omega = df~.$$
This vector field is called the ``Hamiltonian vector field'' with Hamiltonian $f$. Given a chart $(q_i,p_i)$ $(i=1,\cdots,n)$ of $M$ with a symplectic form 
\begin{equation}\label{symp}
\omega=\sum_{i=1}^n dq_i\wedge dp_i~,
\end{equation}
show that the flow of the Hamiltonian vector field $X_H$ associated to a Hamiltonian $H$ can be described by 
$$
\frac{dq_i}{dt}=\frac{\partial H}{\partial  p_i}~,\qquad \frac{dp_i}{dt}=-\frac{\partial H}{\partial  q_i}~,
$$
which are Hamilton's canonical equations. 


For $f,g\in C^\infty(M)$, we define the Poisson product by
$$
\{ f,g\}=\omega(X_f,X_g)~.
$$
 Show that given the symplectic form \eqref{symp}, the Poisson product can be written as the local coordinate
 $$
 \{ f,g\}=\sum_{i=1}^n \left(\frac{\partial f}{\partial  q_i}\frac{\partial g}{\partial  p_i}-\frac{\partial f}{\partial  p_i}\frac{\partial g}{\partial  q_i} \right)~.
 $$
 For $f\in C^\infty(M)$, show that the differentiation with respect to the Hamiltonian flow  can be expressed by
 $$
 \frac{df}{dt}=-\{H,f\}~.
 $$
 
 
Let $H=q^2+p^2$ be the Hamiltonian on the symplectic manifold ($\bR^{2},\omega=dq\wedge dp$). Write down the Hamiltonian vector field $X_H$ as well as the Hamilton's canonical equations in terms of the coordinate. Draw the flow generated by the Hamiltonian vector field and show that the Hamiltonian is constant along the flow.
 
 
 
 
 
\end{document}

