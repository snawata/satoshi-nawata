\documentclass[12pt,a4paper]{article}
\usepackage{hyperref} % Use the Charter font for the document text
%\usepackage[UTF8]{ctex}
\usepackage{fullpage}
\usepackage{amsfonts,amssymb,amsmath}


\newcommand{\bA}{\ensuremath{\mathbb{A}}}
\newcommand{\bB}{\ensuremath{\mathbb{B}}}
\newcommand{\bC}{\ensuremath{\mathbb{C}}}
\newcommand{\bD}{\ensuremath{\mathbb{D}}}
\newcommand{\bE}{\ensuremath{\mathbb{E}}}
\newcommand{\bF}{\ensuremath{\mathbb{F}}}
\newcommand{\bG}{\ensuremath{\mathbb{G}}}
\newcommand{\bH}{\ensuremath{\mathbb{H}}}
\newcommand{\bI}{\ensuremath{\mathbb{I}}}
\newcommand{\bJ}{\ensuremath{\mathbb{J}}}
\newcommand{\bK}{\ensuremath{\mathbb{K}}}
\newcommand{\bL}{\ensuremath{\mathbb{L}}}
\newcommand{\bM}{\ensuremath{\mathbb{M}}}
\newcommand{\bN}{\ensuremath{\mathbb{N}}}
\newcommand{\bO}{\ensuremath{\mathbb{O}}}
\newcommand{\bP}{\ensuremath{\mathbb{P}}}
\newcommand{\bQ}{\ensuremath{\mathbb{Q}}}
\newcommand{\bR}{\ensuremath{\mathbb{R}}}
\newcommand{\bS}{\ensuremath{\mathbb{S}}}
\newcommand{\bT}{\ensuremath{\mathbb{T}}}
\newcommand{\bU}{\ensuremath{\mathbb{U}}}
\newcommand{\bV}{\ensuremath{\mathbb{V}}}
\newcommand{\bW}{\ensuremath{\mathbb{W}}}
\newcommand{\bX}{\ensuremath{\mathbb{X}}}
\newcommand{\bY}{\ensuremath{\mathbb{Y}}}
\newcommand{\bZ}{\ensuremath{\mathbb{Z}}}



\newtheorem{lemma}{Lemma}[section]
\newtheorem{conjecture}[lemma]{Conjecture} 
\newtheorem{corollary}[lemma]{Corollary} 
\newtheorem{theorem}[lemma]{Theorem} 
\newtheorem{definition}[lemma]{Definition} 
\newtheorem{question}[lemma]{Question} 
\newtheorem{proposition}[lemma]{Proposition} 

\usepackage{graphicx}


\begin{document}\thispagestyle{empty}

\centerline{\Large \bf Homework 7: Due at class on April 21}

 \vspace{.5cm}
\noindent 1. Show that $\partial_{n-1}\cdot \partial_n=0$

 \vspace{.5cm}
\noindent 2. Show that the Euler characteristics is equal to
$$
\chi(X)=\sum_{i\ge 0 } (-1)^i \dim C_i(K,\bR)
$$
given a triangulation $|K|\to X$.





 \vspace{.5cm}
\noindent 3. Show that the Euler characteristics of an odd-dimensional compact oriented closed manifold is zero.



 \vspace{.5cm}
\noindent 4. Find the integer-valued homology group $H_\ell(\Sigma_g,\bZ)$ of a Riemann surface $\Sigma_g$ of genus $g$.


 \vspace{.5cm}
\noindent 5. Find both the integer-valued $H_\ell(\bR P^2,\bZ)$ and the real-valued $H_\ell(\bR P^2,\bR)$ homology groups of $\bR P^2$. Does the Poincar\'e duality hold? 



%
% \vspace{.5cm}
%\noindent 5.  \textbf{Fundamental theorem of algebra}
%
%We define $f:\bC\to \bC$  by
%$f(z) = z^n +a_1z^{n-1} +\cdots+a_n $ for $n\ge 1$. In addition, by writing $z=x+iy$, we define one-form
%$$
%\omega=\frac{-ydx}{x^2+y^2}+\frac{xdy}{x^2+y^2}~.
%$$
%Then, show that
%$$
%\frac{1}{2\pi} \int_{C_R} f^*\omega=n~,
%$$
%where $C_R$ is the circle with sufficiently large radius $R$. If there were no zero points $f(z)=0$ inside $C_R$, show that 
%$$
%\frac{1}{2\pi} \int_{C_R} f^*\omega=0~
%$$
%by using the Stokes theorem.
%

\end{document}

