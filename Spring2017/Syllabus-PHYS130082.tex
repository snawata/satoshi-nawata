\documentclass[12pt,a4paper]{article}
\usepackage{hyperref} % Use the Charter font for the document text
%\usepackage[UTF8]{ctex}
\usepackage{fullpage}

\usepackage{graphicx}


\begin{document}\thispagestyle{empty}

\centerline{\Large \bf Syllabus}

\begin{description}
\item{\bf Course name:} Differential geometry and topology in physics 
\item{\bf Instructor:} Satoshi Nawata, Science Building 624, \href{mailto:snawata@fudan.edu.cn}{snawata@fudan.edu.cn}
\item{\bf Hours:} 9:55-11:35 on Friday 
\item{\bf Place:} HGX 403
\item{\bf Office hour:} Tuesday/Thursday 15:00-16:00
\item{\bf Prerequisites:} Linear algebra, Calculus, Classical mechanics
\item{\bf About the course:}

In this course, I will explain basics of differential geometry and topology and their applications to physics. The theory of differential forms is indispensable for Maxwell's electro-magnetic theory. Einstein's general relativity has been established based on Riemannian geometry. Non-perturbative quantum effects like Aharonov-Bohm effect, Dirac monopoles, Berry phase, quantum Hall effects, instantons, and anomaly have deep connections to vector bundles and characteristic classes. Therefore, this course introduces to basic concepts of differential geometry and topology. In addition, I will explain many aspects of physics from the viewpoint of geometry and topology. The course will be taught in English.

\item{\bf Main content:}
\begin{itemize}
\item differential manifolds, Riemannian manifold, complex manifolds 
\item theory of differential forms, harmonic forms
\item vector bundles, connections, curvature, characteristic classes
\item homology, cohomology, fundamental groups, homotopy groups
\item relations to physics such as electromagnetism, general relativity, quantum physics,  condensed matter physics
\end{itemize}

\item{\bf Main textbook:} Mikio Nakahara, Geometry, Topology and Physics

\item{\bf Supplementary textbooks:} 

Theodore Frankel, The Geometry of Physics

Eguchi-Gilkey-Hannson, Gravitation, gauge theory and differential geometry 

Shigeyuki Morita, Geometry of differential forms

\begin{figure}[h]
  \centering
      \includegraphics[width=12cm]{Chinese-textbook}
\end{figure}
\item{\bf Grading:} Grade will be determined based on exercises and final exam.


\end{description}



\end{document}

