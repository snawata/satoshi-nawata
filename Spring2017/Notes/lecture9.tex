 \documentclass[12pt,a4paper]{article}
\usepackage{hyperref} % Use the Charter font for the document text
%\usepackage[UTF8]{ctex}
\usepackage{fullpage}
\usepackage{amsfonts,amssymb,amsmath}
\usepackage{mathtools}
\usepackage{tikz-cd}
\usepackage{tikz}
\usepackage{mathrsfs}

\usepackage{alltt}
\usepackage{amsfonts}
\usepackage{amsmath}
\usepackage{amssymb}
\usepackage{amsthm}
\usepackage{booktabs}
\usepackage{caption}
\usepackage{enumitem}
\usepackage{fancyhdr}
\usepackage{graphicx}
\usepackage{mathdots}
\usepackage{mathtools}
\usepackage{microtype}
\usepackage{multirow}
\usepackage{pdflscape}
\usepackage{pgfplots}
\usepackage{siunitx}
\usepackage{slashed}
\usepackage{tabularx}
\usepackage{tikz}
\usepackage{tkz-euclide}
\usepackage[normalem]{ulem}
\usepackage[all]{xy}
\usepackage{imakeidx}




\newcommand{\frakA}{\ensuremath{\mathfrak{A}}}
\newcommand{\frakB}{\ensuremath{\mathfrak{B}}}
\newcommand{\frakC}{\ensuremath{\mathfrak{C}}}
\newcommand{\frakD}{\ensuremath{\mathfrak{D}}}
\newcommand{\frakE}{\ensuremath{\mathfrak{E}}}
\newcommand{\frakF}{\ensuremath{\mathfrak{F}}}
\newcommand{\frakG}{\ensuremath{\mathfrak{G}}}
\newcommand{\frakH}{\ensuremath{\mathfrak{H}}}
\newcommand{\frakI}{\ensuremath{\mathfrak{I}}}
\newcommand{\frakJ}{\ensuremath{\mathfrak{J}}}
\newcommand{\frakK}{\ensuremath{\mathfrak{K}}}
\newcommand{\frakL}{\ensuremath{\mathfrak{L}}}
\newcommand{\frakM}{\ensuremath{\mathfrak{M}}}
\newcommand{\frakN}{\ensuremath{\mathfrak{N}}}
\newcommand{\frakO}{\ensuremath{\mathfrak{O}}}
\newcommand{\frakP}{\ensuremath{\mathfrak{P}}}
\newcommand{\frakQ}{\ensuremath{\mathfrak{Q}}}
\newcommand{\frakR}{\ensuremath{\mathfrak{R}}}
\newcommand{\frakS}{\ensuremath{\mathfrak{S}}}
\newcommand{\frakT}{\ensuremath{\mathfrak{T}}}
\newcommand{\frakU}{\ensuremath{\mathfrak{U}}}
\newcommand{\frakV}{\ensuremath{\mathfrak{V}}}
\newcommand{\frakW}{\ensuremath{\mathfrak{W}}}
\newcommand{\frakX}{\ensuremath{\mathfrak{X}}}
\newcommand{\frakY}{\ensuremath{\mathfrak{Y}}}
\newcommand{\frakZ}{\ensuremath{\mathfrak{Z}}}
\newcommand{\fraka}{\ensuremath{\mathfrak{a}}}
\newcommand{\frakb}{\ensuremath{\mathfrak{b}}}
\newcommand{\frakc}{\ensuremath{\mathfrak{c}}}
\newcommand{\frakd}{\ensuremath{\mathfrak{d}}}
\newcommand{\frake}{\ensuremath{\mathfrak{e}}}
\newcommand{\frakf}{\ensuremath{\mathfrak{f}}}
\newcommand{\frakg}{\ensuremath{\mathfrak{g}}}
\newcommand{\frakh}{\ensuremath{\mathfrak{h}}}
\newcommand{\fraki}{\ensuremath{\mathfrak{i}}}
\newcommand{\frakj}{\ensuremath{\mathfrak{j}}}
\newcommand{\frakk}{\ensuremath{\mathfrak{k}}}
\newcommand{\frakl}{\ensuremath{\mathfrak{l}}}
\newcommand{\frakm}{\ensuremath{\mathfrak{m}}}
\newcommand{\frakn}{\ensuremath{\mathfrak{n}}}
\newcommand{\frako}{\ensuremath{\mathfrak{o}}}
\newcommand{\frakp}{\ensuremath{\mathfrak{p}}}
\newcommand{\frakq}{\ensuremath{\mathfrak{q}}}
\newcommand{\frakr}{\ensuremath{\mathfrak{r}}}
\newcommand{\fraks}{\ensuremath{\mathfrak{s}}}
\newcommand{\frakt}{\ensuremath{\mathfrak{t}}}
\newcommand{\fraku}{\ensuremath{\mathfrak{u}}}
\newcommand{\frakv}{\ensuremath{\mathfrak{v}}}
\newcommand{\frakw}{\ensuremath{\mathfrak{w}}}
\newcommand{\frakx}{\ensuremath{\mathfrak{x}}}
\newcommand{\fraky}{\ensuremath{\mathfrak{y}}}
\newcommand{\frakz}{\ensuremath{\mathfrak{z}}}
\newcommand{\fraksl}{\ensuremath{\mathfrak{sl}}}
\newcommand{\frakso}{\ensuremath{\mathfrak{so}}}
\newcommand{\fraksp}{\ensuremath{\mathfrak{sp}}}





\newcommand{\bA}{\ensuremath{\mathbb{A}}}
\newcommand{\bB}{\ensuremath{\mathbb{B}}}
\newcommand{\bC}{\ensuremath{\mathbb{C}}}
\newcommand{\bD}{\ensuremath{\mathbb{D}}}
\newcommand{\bE}{\ensuremath{\mathbb{E}}}
\newcommand{\bF}{\ensuremath{\mathbb{F}}}
\newcommand{\bG}{\ensuremath{\mathbb{G}}}
\newcommand{\bH}{\ensuremath{\mathbb{H}}}
\newcommand{\bI}{\ensuremath{\mathbb{I}}}
\newcommand{\bJ}{\ensuremath{\mathbb{J}}}
\newcommand{\bK}{\ensuremath{\mathbb{K}}}
\newcommand{\bL}{\ensuremath{\mathbb{L}}}
\newcommand{\bM}{\ensuremath{\mathbb{M}}}
\newcommand{\bN}{\ensuremath{\mathbb{N}}}
\newcommand{\bO}{\ensuremath{\mathbb{O}}}
\newcommand{\bP}{\ensuremath{\mathbb{P}}}
\newcommand{\bQ}{\ensuremath{\mathbb{Q}}}
\newcommand{\bR}{\ensuremath{\mathbb{R}}}
\newcommand{\bS}{\ensuremath{\mathbb{S}}}
\newcommand{\bT}{\ensuremath{\mathbb{T}}}
\newcommand{\bU}{\ensuremath{\mathbb{U}}}
\newcommand{\bV}{\ensuremath{\mathbb{V}}}
\newcommand{\bW}{\ensuremath{\mathbb{W}}}
\newcommand{\bX}{\ensuremath{\mathbb{X}}}
\newcommand{\bY}{\ensuremath{\mathbb{Y}}}
\newcommand{\bZ}{\ensuremath{\mathbb{Z}}}


%
%\parskip=1em
%\parindent=0.3in
%\setlength\oddsidemargin{0.5in} \setlength\evensidemargin{0.5in}
%\setlength\textwidth{5.5in}
%
%\hfuzz6pt % Don't bother to report over-full boxes if over-edge is < 6pt
%
%\newlength{\defbaselineskip}
%\setlength{\defbaselineskip}{\baselineskip}
%\newcommand{\setlinespacing}[1]%
%           {\setlength{\baselineskip}{#1 \defbaselineskip}}
%\newcommand{\doublespacing}{\setlength{\baselineskip}%
%                           {2.0 \defbaselineskip}}
%\newcommand{\singlespacing}{\setlength{\baselineskip}{\defbaselineskip}}
%
%\newcommand{\properpagestyle}{\pagestyle{myheadings}\markboth{}{}\markright{}}




\def\Ric{\mathop{\rm Ric}}
\def\cRic{\mathop{\stackrel{\circ}{\Ric}}}
\def\Scal{\mathop{\rm R}}
\def\scL{\mathop{\mathcal L}}
\def\Hess{\mathop{\rm Hess}}
\def\bt{\mathop{\bar\tau}}
\def\dist{\mathop{\rm dist}}
\def\Cut{\mathop{\rm Cut}}
\def\Riem{\mathop{\rm Rm}}
\def\scal{\mathop{\rm scal}}
\def\Sec{\mathop{\rm Sec}}
\def\Diam{\mathop{\rm Diam}}
\def\CS{\mathop{\rm C_S}}
\def\V{\mathop{\rm V}}
\def\Vol{\mathop{\rm Vol}}
\def\Area{\mathop{\rm Area}}
\def\VR{\mathop{\rm VR}}
\def\supp{\mathop{\rm supp}}
\def\div{\mathop{\rm div}}
\def\inj{\mathop{\rm inj}}
\def\diam{\mathop{\rm diam}}
\def\Id{\mathop{\rm Id}}
\def\RRR{\mathop{\mathcal{R}}}
\def\MMM{\mathop{\mathcal{M}}}
\def\HHH{\mathop{\mathcal{H}}}
\def\VVV{\mathop{\mathcal{V}}}
\def\FF{\mathop{\mathbb{F}}}
\def\RR{\mathop{\mathbb{R}}}
\def\QQ{\mathop{\mathbb{Q}}}
\def\CC{\mathop{\mathbb{C}}}
\def\ZZ{\mathop{\mathbb{Z}}}
\def\SS{\mathop{\mathbb{S}}}
\def\SSS{\mathop{\mathcal{S}}}
\def\PP{\mathop{\mathbb{P}}}
\def\End{\mathop{\rm End}}
\def\Aut{\mathop{\rm Aut}}
\def\Ad{\mathop{\rm Ad}}
\def\ad{\mathop{\rm ad}}
\def\hht{\mathop{\rm ht}}
\def\gl{\mathop{\mathfrak{gl}}}
\def\ssl{\mathop{\mathfrak{sl}}}
\def\TP{\mathop{\mathcal{TP}}}
\def\PPP{\mathop{\mathcal{P}}}
\def\gggg{\mathop{\mathfrak{g}}}
\def\ffff{\mathop{\mathfrak{f}}}
\def\OO{\mathop{\mathcal{O}}}
\def\oo{\mathop{\mathfrak{o}}}
\def\GG{\mathop{\mathcal{G}}}
\def\WWW{\mathop{\mathcal{W}}}
\def\Rad{\mathop{\rm Rad}}
\def\Der{\mathop{\rm Der}}
\def\Ker{\mathop{\rm Ker}}
\def\Im{\mathop{\rm Im}}

\def\be{\begin{eqnarray}}
\def\ee{\end{eqnarray}}
\def\beg{\begin{eqnarray*}}
\def\ees{\end{eqnarray*}}


%\newcommand{\qed}{\hfill$\Box$}
\theoremstyle{definition}
\newtheorem*{aim}{Aim}
\newtheorem*{axiom}{Axiom}
\newtheorem*{claim}{Claim}
\newtheorem*{cor}{Corollary}
\newtheorem*{conjecture}{Conjecture}
\newtheorem*{defi}{Definition}
\newtheorem*{eg}{Example}
\newtheorem*{ex}{Exercise}
\newtheorem*{fact}{Fact}
\newtheorem*{law}{Law}
\newtheorem*{lemma}{Lemma}
\newtheorem*{notation}{Notation}
\newtheorem*{prop}{Proposition}
\newtheorem*{question}{Question}
\newtheorem*{thm}{Theorem}





% Maths symbols
\newcommand{\abs}[1]{\left\lvert #1\right\rvert}
%\newcommand\ad{\mathrm{ad}}
\newcommand\AND{\mathsf{AND}}
\newcommand\Art{\mathrm{Art}}
\newcommand{\Bilin}{\mathrm{Bilin}}
\newcommand{\bket}[1]{\left\lvert #1\right\rangle}
\newcommand{\B}{\mathcal{B}}
\newcommand{\bolds}[1]{{\bfseries #1}}
\newcommand{\brak}[1]{\left\langle #1 \right\rvert}
\newcommand{\braket}[2]{\left\langle #1\middle\vert #2 \right\rangle}
\newcommand{\bra}{\langle}
\newcommand{\cat}[1]{\mathsf{#1}}
\newcommand{\C}{\mathbb{C}}
\newcommand{\CP}{\mathbb{CP}}
\newcommand{\cU}{\mathcal{U}}
%\newcommand{\Der}{\mathrm{Der}}
\newcommand{\D}{\mathrm{D}}
\newcommand{\dR}{\mathrm{dR}}
\newcommand{\E}{\mathbb{E}}
\newcommand{\F}{\mathbb{F}}
\newcommand{\Frob}{\mathrm{Frob}}
%\newcommand{\GG}{\mathbb{G}}
%\newcommand{\gl}{\mathfrak{gl}}
\newcommand{\GL}{\mathrm{GL}}
\newcommand{\G}{\mathcal{G}}
\newcommand{\Gr}{\mathrm{Gr}}
\newcommand{\haut}{\mathrm{ht}}
\newcommand{\Hol}{\mathrm{Hol}}
\newcommand{\hol}{\mathfrak{hol}}
%\newcommand{\Id}{\mathrm{Id}}
\newcommand{\ket}{\rangle}
\newcommand{\lie}[1]{\mathfrak{#1}}
\newcommand{\Mat}{\mathrm{Mat}}
\newcommand{\N}{\mathbb{N}}
\newcommand{\norm}[1]{\left\lVert #1\right\rVert}
\newcommand{\normalorder}[1]{\mathop{:}\nolimits\!#1\!\mathop{:}\nolimits}
\newcommand{\NOT}{\mathsf{NOT}}
\newcommand{\op}{\mathrm{op}}
\newcommand{\Oc}{\mathcal{O}}
\newcommand{\Or}{\mathrm{O}}
\newcommand\OR{\mathsf{OR}}
\newcommand{\ort}{\mathfrak{o}}
\newcommand{\PGL}{\mathrm{PGL}}
\newcommand{\ph}{\,\cdot\,}
\newcommand{\pr}{\mathrm{pr}}
\newcommand{\Prob}{\mathbb{P}}
\newcommand{\PSL}{\mathrm{PSL}}
\newcommand{\Ps}{\mathcal{P}}
\newcommand{\PSU}{\mathrm{PSU}}
\newcommand{\pt}{\mathrm{pt}}
\newcommand{\qeq}{\mathrel{``{=}"}}
\newcommand{\Q}{\mathbb{Q}}
\newcommand{\R}{\mathbb{R}}
\newcommand{\RP}{\mathbb{RP}}
\newcommand{\Rs}{\mathcal{R}}
\newcommand{\SL}{\mathrm{SL}}
\newcommand{\so}{\mathfrak{so}}
\newcommand{\SO}{\mathrm{SO}}
\newcommand{\Spin}{\mathrm{Spin}}
\newcommand{\Sp}{\mathrm{Sp}}
\newcommand{\su}{\mathfrak{su}}
\newcommand{\SU}{\mathrm{SU}}
\newcommand{\term}[1]{\textbf{#1}\index{#1}}
\newcommand{\T}{\mathbb{T}}
\newcommand{\tv}[1]{|#1|}
\newcommand{\U}{\mathrm{U}}
\newcommand{\uu}{\mathfrak{u}}
\newcommand{\Vect}{\mathrm{Vect}}
\newcommand{\wsto}{\stackrel{\mathrm{w}^*}{\to}}
\newcommand{\wt}{\mathrm{wt}}
\newcommand{\wto}{\stackrel{\mathrm{w}}{\to}}
\newcommand{\Z}{\mathbb{Z}}
\renewcommand{\d}{\mathrm{d}}
\renewcommand{\H}{\mathbb{H}}
\renewcommand{\P}{\mathbb{P}}
\renewcommand{\sl}{\mathfrak{sl}}
\renewcommand{\vec}[1]{\boldsymbol{\mathbf{#1}}}
%\renewcommand{\F}{\mathcal{F}}

\let\Im\relax
\let\Re\relax

\DeclareMathOperator{\adj}{adj}
\DeclareMathOperator{\Ann}{Ann}
\DeclareMathOperator{\area}{area}
%\DeclareMathOperator{\Aut}{Aut}
\DeclareMathOperator{\Bernoulli}{Bernoulli}
\DeclareMathOperator{\betaD}{beta}
\DeclareMathOperator{\bias}{bias}
\DeclareMathOperator{\binomial}{binomial}
\DeclareMathOperator{\card}{card}
\DeclareMathOperator{\ccl}{ccl}
\DeclareMathOperator{\Char}{char}
\DeclareMathOperator{\ch}{ch}
\DeclareMathOperator{\cl}{cl}
\DeclareMathOperator{\cls}{\overline{\mathrm{span}}}
\DeclareMathOperator{\coker}{coker}
\DeclareMathOperator{\conv}{conv}
\DeclareMathOperator{\corr}{corr}
\DeclareMathOperator{\cosec}{cosec}
\DeclareMathOperator{\cosech}{cosech}
\DeclareMathOperator{\cov}{cov}
\DeclareMathOperator{\covol}{covol}
\DeclareMathOperator{\diag}{diag}
%\DeclareMathOperator{\diam}{diam}
\DeclareMathOperator{\Diff}{Diff}
\DeclareMathOperator{\disc}{disc}
\DeclareMathOperator{\dom}{dom}
%\DeclareMathOperator{\End}{End}
\DeclareMathOperator{\energy}{energy}
\DeclareMathOperator{\erfc}{erfc}
\DeclareMathOperator{\erf}{erf}
\DeclareMathOperator*{\esssup}{ess\,sup}
\DeclareMathOperator{\ev}{ev}
\DeclareMathOperator{\Ext}{Ext}
\DeclareMathOperator{\fst}{fst}
\DeclareMathOperator{\Fit}{Fit}
\DeclareMathOperator{\fix}{fix}
\DeclareMathOperator{\Frac}{Frac}
\DeclareMathOperator{\Gal}{Gal}
\DeclareMathOperator{\gammaD}{gamma}
\DeclareMathOperator{\gr}{gr}
\DeclareMathOperator{\hcf}{hcf}
\DeclareMathOperator{\Hom}{Hom}
\DeclareMathOperator{\id}{id}
\DeclareMathOperator{\Image}{image}
\DeclareMathOperator{\Im}{Im}
\DeclareMathOperator{\Ind}{Ind}
\DeclareMathOperator{\Int}{Int}
\DeclareMathOperator{\Isom}{Isom}
\DeclareMathOperator{\lcm}{lcm}
\DeclareMathOperator{\length}{length}
\DeclareMathOperator{\Lie}{Lie}
\DeclareMathOperator{\like}{like}
\DeclareMathOperator{\Lk}{Lk}
\DeclareMathOperator{\Maps}{Maps}
\DeclareMathOperator{\mse}{mse}
\DeclareMathOperator{\multinomial}{multinomial}
\DeclareMathOperator{\orb}{orb}
\DeclareMathOperator{\ord}{ord}
\DeclareMathOperator{\otp}{otp}
\DeclareMathOperator{\Poisson}{Poisson}
\DeclareMathOperator{\poly}{poly}
\DeclareMathOperator{\rank}{rank}
\DeclareMathOperator{\rel}{rel}
%\DeclareMathOperator{\Rad}{Rad}
\DeclareMathOperator{\Re}{Re}
\DeclareMathOperator*{\res}{res}
\DeclareMathOperator{\Res}{Res}
%\DeclareMathOperator{\Ric}{Ric}
\DeclareMathOperator{\rk}{rk}
\DeclareMathOperator{\Rees}{Rees}
\DeclareMathOperator{\Root}{Root}
\DeclareMathOperator{\sech}{sech}
\DeclareMathOperator{\sgn}{sgn}
\DeclareMathOperator{\snd}{snd}
\DeclareMathOperator{\Spec}{Spec}
\DeclareMathOperator{\spn}{span}
\DeclareMathOperator{\stab}{stab}
\DeclareMathOperator{\St}{St}
%\DeclareMathOperator{\supp}{supp}
\DeclareMathOperator{\Syl}{Syl}
\DeclareMathOperator{\Sym}{Sym}
\DeclareMathOperator{\tr}{tr}
\DeclareMathOperator{\Tr}{Tr}
\DeclareMathOperator{\var}{var}
\DeclareMathOperator{\vol}{vol}
\usetikzlibrary{knots}




\pgfarrowsdeclarecombine{twolatex'}{twolatex'}{latex'}{latex'}{latex'}{latex'}
\tikzset{->/.style = {decoration={markings,
                                  mark=at position 1 with {\arrow[scale=2]{latex'}}},
                      postaction={decorate}}}
\tikzset{<-/.style = {decoration={markings,
                                  mark=at position 0 with {\arrowreversed[scale=2]{latex'}}},
                      postaction={decorate}}}
\tikzset{<->/.style = {decoration={markings,
                                   mark=at position 0 with {\arrowreversed[scale=2]{latex'}},
                                   mark=at position 1 with {\arrow[scale=2]{latex'}}},
                       postaction={decorate}}}
\tikzset{->-/.style = {decoration={markings,
                                   mark=at position #1 with {\arrow[scale=2]{latex'}}},
                       postaction={decorate}}}
\tikzset{-<-/.style = {decoration={markings,
                                   mark=at position #1 with {\arrowreversed[scale=2]{latex'}}},
                       postaction={decorate}}}
\tikzset{->>/.style = {decoration={markings,
                                  mark=at position 1 with {\arrow[scale=2]{latex'}}},
                      postaction={decorate}}}
\tikzset{<<-/.style = {decoration={markings,
                                  mark=at position 0 with {\arrowreversed[scale=2]{twolatex'}}},
                      postaction={decorate}}}
\tikzset{<<->>/.style = {decoration={markings,
                                   mark=at position 0 with {\arrowreversed[scale=2]{twolatex'}},
                                   mark=at position 1 with {\arrow[scale=2]{twolatex'}}},
                       postaction={decorate}}}
\tikzset{->>-/.style = {decoration={markings,
                                   mark=at position #1 with {\arrow[scale=2]{twolatex'}}},
                       postaction={decorate}}}
\tikzset{-<<-/.style = {decoration={markings,
                                   mark=at position #1 with {\arrowreversed[scale=2]{twolatex'}}},
                       postaction={decorate}}}


\tikzset{circ/.style = {fill, circle, inner sep = 0, minimum size = 3}}
\tikzset{scirc/.style = {fill, circle, inner sep = 0, minimum size = 1.5}}
\tikzset{mstate/.style={circle, draw, blue, text=black, minimum width=0.7cm}}

\tikzset{eqpic/.style={baseline={([yshift=-.5ex]current bounding box.center)}}}
\tikzset{commutative diagrams/.cd,cdmap/.style={/tikz/column 1/.append style={anchor=base east},/tikz/column 2/.append style={anchor=base west},row sep=tiny}}


\definecolor{mblue}{rgb}{0.2, 0.3, 0.8}
\definecolor{morange}{rgb}{1, 0.5, 0}
\definecolor{mgreen}{rgb}{0.1, 0.4, 0.2}
\definecolor{mred}{rgb}{0.5, 0, 0}


%\title{ Lecture 4}
\begin{document}\thispagestyle{empty}

\centerline{\Large \bf Lecture 9}

\centerline{\Large \bf Nakahara section  9.4, 10}




\section{Principal \texorpdfstring{$G$}{G}-bundles}

Last time, we have studied vector bundles where a fiber is a vector space. We can further generalize it to a \textbf{fiber bundle} where a fiber is a general manifold $F$ and a transition function is given by a differomorphism of $F$. Among fiber bundles, principal $G$-bundles play an important role in physics.




\begin{defi}[Principal $G$-bundle]\index{principal $G$-bundle}
  Let $G$ be a Lie group, and $M$ a manifold. A \term{principal $G$-bundle} is a smooth manifold $P$ with a projection $\pi: P \to M$ such that a fiber is $\pi^{-1}(\{x\}) \cong G$ for each $x \in M$. 
  More precisely, we are given an open cover $\{U_\alpha\}$ of $M$ and diffeomorphisms
  \[
   \begin{tikzcd}
          \pi^{-1}(U) \ar[r, "t"] \ar[d, "\pi"] & U \times G \ar[dl, "p_1"]\\
          U
        \end{tikzcd}  \]
such that the transition functions
  \[
    t_\alpha \circ t_\beta^{-1}: (U_\alpha \cap U_\beta) \times G \to (U_\alpha \cap U_\beta) \times G
  \]
  is of the form
  \[
    (x, g) \mapsto (x, t_{\alpha\beta}(x) \cdot g)
  \]
  for some $t_{\alpha\beta}: U_\alpha \cap U_\beta \to G$ where $G$ is called the \term{structure group}.
\end{defi}

We cab define an action of $G$ on the total space $P$ to the right 
$$
P\times G \to P; (u,g)\mapsto ug
$$
where each fiber onto itself. 

Given a principle $G$-bundle and a representation $\rho:G\to \mathfrak{gl}(V)$ for a vector space $V$, we can construct \term{associated vector bundle}
$$
E=P\times_\rho V
$$  
as follow. Let us consider the direct product $P\times V$ and $G$ action 
$$
(u,y)\mapsto (us,\rho(s)^{-1}y)  \qquad \textrm{for}   \quad s\in G
$$
We define the associated bundle as the quotient space $P\times_\rho V:=(P\times V)/G$. Conversely, given a $G$-bundle $E\to M$ with fiber $V$, there is a canonical way of producing a principal $G$-bundle by using transition functions.


\section{Connections and curvatures}
In Riemannian geometry, we have learnt Levi-Civita connections and Riemann curvature. Even in vector bundles and principal $G$-bundles, we can introduce connections and curvatures.

\begin{defi}[Connection]\index{connection}
  A \term{connection} \index{$\nabla$} in a vector bundle $\pi:E\to M$  is a bilinear map 
  $$\nabla:\mathfrak{X}(M)\times \Gamma(E) \to \Gamma(E);(X,s)\mapsto \nabla_X s$$ 
  satisfying
  \begin{enumerate}
    \item $ \nabla_{fX}(s) = f\nabla_{X}(s) $
      
    \item Leibnitz property: $ \nabla_X (fs) = (X f) s + f (\nabla_X s)$
  \end{enumerate}
 for all $s \in \Gamma(E)$ and $f \in C^\infty(M)$.
\end{defi}

If a vector bundle $E$ has a metric $g$, a connection $\nabla$ is \term{compatible} with the metric if
$$
X g(s_1,s_2)= g(\nabla_X s_1,s_2)+g( s_1,\nabla_X s_2)
$$
for any $X\in \mathfrak{X}(M)$ and $s_1,s_2 \in \Gamma(E)$.

\begin{defi}[Curvature]\index{Curvature}
  A \term{curvature} \index{$\nabla$} in a vector bundle $\pi:E\to M$  is a trilinear map 
  $$F:\mathfrak{X}(M)\times \mathfrak{X}(M)\times \Gamma(E) \to \Gamma(E);(X,Y,s)\mapsto F(X,Y) s$$ 
  defined by
  $$
  F(X,Y)s=\frac12\left[ \nabla_X\nabla_Y-\nabla_Y\nabla_X-\nabla_{[X,Y]}\right]s~.
  $$
  It has the following properties
    \begin{enumerate}
    \item $ F(X,Y)s =-F(Y,X)s $
    \item  $F(fX,gY)(hs)  = fghF(X,Y)s $  for  $f,g,h\in C^\infty(M)$
  \end{enumerate}
 for all $s \in \Gamma(E)$.
\end{defi}


For a certain open subset of $M$, we can take a frame $s_1,\cdots,s_r \in \Gamma(\pi^{-1}(U))$. For any vector field $X$ on $U$, the connection can be locally written as 
$$
\nabla_X s_j=\sum_{i=1}^rA^i_j(X)s_i
$$ 
where $A^i_j\in \Omega^1(U,\mathfrak{gl}(r,\R))$ (1-form on $U$ taking its value on $\mathfrak{gl}(r,\R)$) is called \term{connection form}. We now look at the curvature $R$ from differential forms.
$$
F(X,Y)s_j=\sum_{i=1}^r F^i_j (X,Y)s_i
$$
where $F^i_j\in \Omega^2(U,\mathfrak{gl}(r,\R))$  (2-form on $U$ taking its value on $\mathfrak{gl}(r,\R)$)  is called \term{curvature form}. They are related by the following equation:
$$
F=dA+A\wedge A~.
$$
The curvature form satisfies the Bianchi identity
$$
dF-F\wedge A+A\wedge F=0~.
$$

More explicitly, in physics, we write a section $s=\sum_{j=1}^rv^j(x)s_j$ on $U$ so that
$$
\nabla_{\frac{\partial}{\partial x^\mu}} s=\sum_{i=1}^r\left[\frac{\partial}{\partial x^\mu} v^i(x) + (A_{\mu}){}^i{}_j v^j(x) \right]s_i
$$
In addition, the curvature can be written in terms of local coordinates
$$
F\left(\frac{\partial}{\partial x^\mu},\frac{\partial}{\partial x^\nu}\right)=F_{\mu\nu}=\frac{\partial A_\nu}{\partial x^\mu}-\frac{\partial A_\mu}{\partial x^\nu} +[ A_\mu, A_\nu]~.
$$
In the case of Maxwell $\U(1)$ theory, the last term vanish because it is a commutative group.



It is useful to know how the connection transforms under a change of local trivialization. Given a transition function $g_{\alpha\beta}: U_\alpha \cap U_\beta \to GL(r,\R)$, the gauge fields on $U_\alpha$ and $U_\beta$ are related by
\be\label{gauge-trans}
  A_\beta = g_{\alpha \beta}^{-1} A_\alpha g_{\alpha\beta} + g_{\alpha\beta}^{-1} d (g_{\alpha\beta})~,
\ee
and the curvature forms are related by
$$
F_\beta=g_{\alpha \beta}^{-1} F_\alpha g_{\alpha\beta}
$$

This construction makes sense for $G$-bundle, and so we can canonically identify both $G$ and $\mathfrak{g}$ as subsets of $\GL(n, \R)$ for some $n$. Then  we have to replace the first term of \eqref{gauge-trans} by the adjoint representation of $G$ on $\mathfrak{g}$, and the second of \eqref{gauge-trans} by the \term{Maurer--Cartan form}.

In particular, for the Maxwell theory, the gauge transformation can be written as $g_{\alpha\beta}=e^{i\lambda_{\alpha\beta}(x)}$ so that
$$
A_\beta=A_\alpha + i d\lambda_{\alpha\beta}
$$
and the curvature form stays invariant.



\subsection{Parallel transport and holonomy group}
Given a connection $\nabla$ on a vector bundle $E$, on can define \term{horizontal} directions in the space $\Gamma(E)$ of sections. A section $s\in \Gamma(E)$ is \term{parallel} along a path $\gamma:I\to (M)$
$$
\nabla_{\dot \gamma(t)} s=0  \quad \textrm{for} \quad t\in I~.
$$
In terms of local coordinates, it can be written as
$$
\frac{ds_i}{dt}+\sum_{j=1}^r (A_\mu){}^j{}_i \frac{dx^\mu}{dt} s_j=0~.
$$
A theorem of ordinary differential equations tells us that given an initial data $s(t=0)\in E_{\gamma(0)}$, one can do parallel transform along $\gamma(t)$ so that we have a map
$$
E_{\gamma(0)} \ni s(t=0) \to s(t=1)\in E_{\gamma(1)}~.
$$
In particular, if we consider a curve $p=\gamma(0)=\gamma(1)$, we obtain amap $\tau_\gamma:E_p\to E_p$. For given two curves $\gamma_1$ and $\gamma_2$, we can have a multiplication
$$
\tau_{\gamma_1\circ \gamma_2}=\tau_{\gamma_1}\circ \tau_{\gamma_2}
$$
and the inverse is defined by 
$$
\tau_{\gamma^{-1}}=\tau_{\gamma}^{-1}~,
$$
so that it forms a group called \term{holonomy group}. In the case of $\U(1)$, this is the origin of the \term{Aharonov-Bohm effect}.

\begin{figure}[h]\centering
\includegraphics[width=10cm]{abm}
\end{figure}
\begin{figure}[h]\centering
\includegraphics[width=7cm]{Aharonov-Bohm}
\caption{The Aharonov-Bohm solenoid effect takes place when the wave function of a charged particle passing around a long solenoid experiences a phase shift as a result of the enclosed magnetic field, despite the magnetic field being negligible in particle tragectories.}
\end{figure}

\subsection{Levi-Civita connections}
We can consider the tangent bundle $TM$ as a vector bundle and its metric $g$ is indeed a Riemannian metric. Then, we can uniquely determine the natural connection called \term{Levi-Civita connections}. Let $X_1, \cdots,  X_n$ be an orthonormal frame vector field on an open set $U\subset M$ and their dual $e^1,\cdots,e^n\in \Omega^1(U)$. For the Levi-Civita connection $\nabla$, we can write
\begin{align}
\nabla_{X_j}X_i&=\sum_k\mathbf{\Gamma}^k_{ij}X_k~,\cr
R(X_i,X_j)X_k&=\sum_{l}\mathbf{R}^{l}{}_{ijk}X_l~.\nonumber
\end{align}
  
We can define connection one-form and curvature two-form taking their values on $\mathfrak{so}(n,\R)$: 
$$
\omega^k{}_j=\sum_k\mathbf{\Gamma}^k_{ij} e^i~, \quad \Omega^l_i=\sum_{l}\mathbf{R}^{l}{}_{ijk} e^j\wedge e^k~.
$$
They satisfy the following conditions
  \begin{align}      
de^i &=-\sum_j \omega^{i}{}_j\wedge e^j~,\cr
\Omega^i{}_j&=d\omega^i{}_j+\omega^i{}_k\wedge \omega^k{}_j~.\nonumber
  \end{align}








\subsection{Ehresmann connections}
Similarly, one can construct theory of connections on principal $G$-bundles, which are called  \term{Ehresmann connections}. In principle, a Ehresmann connection determine the horizontal direction of a principal $G$-bundle.



\begin{defi}[Ehresmann connection]\index{Ehresmann}
  A \term{Ehresmann connection} $A$ on a principle $G$-bundle $\pi:P\to M$  is a one-form taking its value on $\frakg$, which satisfies the following conditions:
    \begin{enumerate}
    \item Given $X\in \frakg$, there is the corresponding vector field $\overline X$ on $P$
    $$
    \overline X_u=\frac{d}{dt}(u\cdot \exp t X)\Big|_u  \quad \textrm{for} \quad u\in P~.
    $$
  Then, $A$ is subject to
  $$
  A(\overline X)=X \in \frakg~.
  $$
    \item For ${}^\forall g\in G$, 
    $$ R^\ast_g(A)=Ad(g^{-1}) A~.$$
    In other words, 
    $$ A_{ug}(R_g(Y))=Ad(g^{-1})(A_u(Y)) ~, \quad \textrm{for} \quad Y\in \frakg ~.   $$
  \end{enumerate}
\end{defi}

For an open set $U_\alpha\subset M$,  there is a local trivialization $\psi_{U_\alpha}: \pi^{-1}(U_\alpha)\to U_\alpha\times G$. Then, we have a section on $\pi^{-1}(U_\alpha)$
$$ \sigma_{U_\alpha}:U_\alpha\to P; x\mapsto \psi_{U_\alpha}^{-1}(x,e)~.$$ 
Then, if we define $A_\alpha=\sigma^*_{U_\alpha}(A)$, we have cocycle condition
$$
  A_\beta = g_{\alpha \beta}^{-1} A_\alpha g_{\alpha\beta} + g_{\alpha\beta}^{-1} d (g_{\alpha\beta})~, \quad \textrm{for} \quad U_\alpha\cap U_\beta~,
$$
where $g_{\alpha\beta}$ is the transition function on $U_\alpha\cap U_\beta$.

We denote the space of Ehresmann connections on a principal $G$-bundle $P$ by $\mathscr{A}_P$. The space $\mathscr{G}_P$ of gauge transformations is the section $\Gamma(M,G_P)$ of the bundle $G_P:=P\times_{Ad}G$. A gauge transformation $g\in  \mathscr{G}_P$ acts on the space of connection $\mathscr{A}_P$ via
$$
\mathscr{G}_P\times \mathscr{A}_P\to \mathscr{A}_P; (g,A)\mapsto g^*(A)=\{g^*(A)_U,g^*(A)_V, \cdots\}
$$ 
where 
$$
g^*(A)_U=g_U^{-1}d(g_U)+g_U^{-1}Ag_U~.
$$
If $A'=g^*(A)$ for $A,A'\in \mathscr{A}_P$, they are physically identical. So we denote the space of physically different connections by
$$
\mathscr{B}_P:=\mathscr{A}_P/\mathscr{G}_P~.
$$






\section{Yang-Mills theory}
Now we can describe non-Abelian gauge theory called \textbf{Yang-Mills theory}.
Let us consider principal $G$-bundle or its associated $G$-bundle. In addition, let $A$ be a connection on it and $F$ be its curvature.
The classical \term{Yang-Mills action} can be written as 
\begin{align}
  S_{YM}[A] &= \frac{1}{2g^2_{YM}} \int_M \Tr F\wedge \ast F\cr
  &= \frac{1}{2g^2_{YM}} \int_M \Tr \left(F_{\mu\nu} F^{\mu\nu}\right) \;\sqrt{g} \d^d x,
\end{align}
where the curvature is 2-form taking its value on the Lie algebra $\mathfrak{g}$, and $\Tr$ is taking over Lie algebra $\mathfrak{g}$.
The parameter $g^2_{YM}$ is the Yang-Mills coupling constant. For the flat space, we have $\sqrt{g} = 1$ so that we will drop this term. This is also known as \term{non-Abelian gauge theory}.

For example, if $G = \SU(N)$, we can choose the basis of the Lie algebra $\mathfrak{su}(N)$
\[
  \Tr(T_a T_b) = \frac{1}{2} \delta_{ab},
\]
and on a local $U \subseteq M$, we have
\[
  S_{YM}[A] = \frac{1}{4 g^2_{YM}} \int F_{\mu\nu}^a F^{b, \mu\nu} \delta_{ab} \; \d^d x,
\]
with $F_{\mu\nu}=\sum_a F_{\mu\nu}^a T_a$ and
\[
  F_{\mu\nu}^a = \partial_\mu A^a_\nu - \partial_\nu A_\mu^a + f_{bc}{}^a A_\mu^b A_\nu^c~.
\]
Thus, Yang--Mills theory is the natural generalization of Maxwell theory to the non-Abelian case.



At the level of the classical field equations, if we vary our connection by $A_\mu \mapsto A_\mu + \delta a_\mu$, where $\delta a$ is a matrix-valued $1$-form, then we have
\[
  \delta F_{\mu\nu} = \partial_{[\mu} \delta a_{\nu]} + [A_\mu, \delta a_\nu].
\]
The equation of motion can be obtained by taking the variation of the Yang-Mills acion
\[
\delta S_{YM}[A] = \frac{1}{2g_{YM}^2} \int \Tr(\delta F_{\mu\nu}, F^{\mu\nu}) \;\d^d x = \frac{1}{2g_{YM}^2} \int \Tr(\nabla_\mu \delta a_\nu, F^{\mu\nu})\;\d^d x = 0.
\]
Therefore, the \term{Yang--Mills equation} is
\[
  \nabla^\mu F_{\mu\nu} = \partial^\mu F_{\mu\nu} + [A^\mu, F_{\mu\nu}] = 0~,
\]
or we can write it without coordinates
$$
\delta_A F:=\ast d_A\ast F=\ast (d+A)\ast F=0~.
$$
Recall we also had the Bianchi identity
\[
  \nabla_\mu F_{\nu\lambda} + \nabla_\nu F_{\lambda\mu} + \nabla_\lambda F_{\mu\nu} = 0~.
\]
\term{Unlike} Maxwell's equations, these are non-linear PDE's for $A$. We no longer have the principle of superposition. This is similar to general relativity. 

The Yang-Mills path integral is expressed by
$$
Z_{YM}=\int_{\mathscr{A}/\mathscr{G}} DA~ \exp({iS_{YM}[A]})~.
$$







\end{document}
