\documentclass[12pt,a4paper]{article}
\usepackage{hyperref} % Use the Charter font for the document text
%\usepackage[UTF8]{ctex}
\usepackage{fullpage}
\usepackage{amsfonts,amssymb,amsmath}
\usepackage{mathtools}
\usepackage{tikz-cd}

\newcommand{\bA}{\ensuremath{\mathbb{A}}}
\newcommand{\bB}{\ensuremath{\mathbb{B}}}
\newcommand{\bC}{\ensuremath{\mathbb{C}}}
\newcommand{\bD}{\ensuremath{\mathbb{D}}}
\newcommand{\bE}{\ensuremath{\mathbb{E}}}
\newcommand{\bF}{\ensuremath{\mathbb{F}}}
\newcommand{\bG}{\ensuremath{\mathbb{G}}}
\newcommand{\bH}{\ensuremath{\mathbb{H}}}
\newcommand{\bI}{\ensuremath{\mathbb{I}}}
\newcommand{\bJ}{\ensuremath{\mathbb{J}}}
\newcommand{\bK}{\ensuremath{\mathbb{K}}}
\newcommand{\bL}{\ensuremath{\mathbb{L}}}
\newcommand{\bM}{\ensuremath{\mathbb{M}}}
\newcommand{\bN}{\ensuremath{\mathbb{N}}}
\newcommand{\bO}{\ensuremath{\mathbb{O}}}
\newcommand{\bP}{\ensuremath{\mathbb{P}}}
\newcommand{\bQ}{\ensuremath{\mathbb{Q}}}
\newcommand{\bR}{\ensuremath{\mathbb{R}}}
\newcommand{\bS}{\ensuremath{\mathbb{S}}}
\newcommand{\bT}{\ensuremath{\mathbb{T}}}
\newcommand{\bU}{\ensuremath{\mathbb{U}}}
\newcommand{\bV}{\ensuremath{\mathbb{V}}}
\newcommand{\bW}{\ensuremath{\mathbb{W}}}
\newcommand{\bX}{\ensuremath{\mathbb{X}}}
\newcommand{\bY}{\ensuremath{\mathbb{Y}}}
\newcommand{\bZ}{\ensuremath{\mathbb{Z}}}



\newtheorem{lemma}{Lemma}[section]
\newtheorem{conjecture}[lemma]{Conjecture} 
\newtheorem{corollary}[lemma]{Corollary} 
\newtheorem{theorem}[lemma]{Theorem} 
\newtheorem{definition}[lemma]{Definition} 
\newtheorem{question}[lemma]{Question} 
\newtheorem{proposition}[lemma]{Proposition} 

\usepackage{graphicx}


\begin{document}\thispagestyle{empty}

\centerline{\Large \bf Lecture 4 }

\centerline{\Large \bf Nakahara \S 6.2, 6.3, 6.4, 7.1 and 7.9}


\section{de Rham cohomology}

Given a differential operator 
$$
d:\Omega^k(M)\to\Omega^{k+1}(M)~,
$$
$\omega\in\Omega^k(M)$ is called a \textbf{closed form} if $d\omega=0$, and   an \textbf{exact form} if there exists $(k-1)$-form such that $\omega=d\eta$. Let us denote the set of all closed $k$-forms on $M$ by $Z^k(M)$ and the set of all exact $k$-forms by $B^k(M)$. 
\begin{align}
Z^k(M)&=\textrm{Ker}(d:\Omega^k(M)\to\Omega^{k+1}(M))\cr
B^k(M)&=\textrm{Im}(d:\Omega^{k-1}(M)\to\Omega^{k}(M))\nonumber
\end{align}
The de Rham cohomology group of $M$ is defined as the quotient space 
$$
H^k_{dR}(M)=Z^k(M)/B^k(M)
$$
In other words, the de Rham cohomology group of $M$ is the cohomology of de Rahm complex
$$
0\to \Omega^0(M) \xrightarrow{d}\Omega^1(M) \xrightarrow{d}\cdots  \xrightarrow{d} \Omega^n(M)\to0
$$

\begin{lemma}[Poincar\'e Lemma]
The de Rham cohomology of $\bR^n$ is trivial. That is, 
$$
H^k(\bR^n)=\left\{\begin{array}{l} \bR \quad k=0 \\ 0 ~ \quad  k\neq 0\end{array}\right.
$$
\end{lemma}
This also holds when $M$ is contractible, namely when one can smoothly shrink $M$ to a point. This is not true on a general manifold. However, it is true on each coordinate chart. The issue is how these charts are patched together globally.


\section{Metric}


A metric $g$ on $M$ is symmetric and non-degenerate bilinear form
$$g_p:T_p M\times T_p M\to \bR$$
in such a way that $g_p$ is smooth with respect to $p$. Its components are given by $g_{ij}=g(\partial_i,\partial_j)$. If $g_{ij}$ is positive definite, $(M,g)$ is called a Riemannian manifold. In a chart $U,(x^1\cdots,x^n)$, it can be written locally as
$$
ds^2= g_{ij}(x)dx^i\otimes dx^j~.
$$
On an arbitrary smooth manifold, one can show there exists a Riemannian metric by using a partition of unity.
On a Riemannian manifold, one can introduce the notion of the length of a tangent vector $v\in T_pM$
$$
||v||=\sqrt{g(v,v)}~.
$$
For a curve $c:[a,b]\to M$, the length $L(c)$ of curve can be defined by
$$
L(c)=\int_{a}^b ||\dot c(t)||dt=\int_a^b \sqrt{g_{ij}\frac{dx^i}{dt}\frac{dx^j}{dt}}
$$

For a smooth map $f:M\to N$, the metric $g$ on a smooth manifold $N$ can be pull-back to the metric $f^*g$ on a smooth manifold $M$ in such a way that for $v,w\in T_pM$
$$
f^*g(v,w)=g(f_*v,f_*w)~.
$$

\begin{definition}[Isometry]\index{isometry}
  Let $(M, g)$ and $(N, h)$ be Riemannian manifolds. We say $f: M \to N$ is an \emph{isometry} if it is a diffeomorphism and $f^*h = g$. In other words, for any $p \in M$ and $u, v \in T_p M$, we need
  \[
    h\big(f_* u, f_* v\big) = g(u, v).
  \]
\end{definition}
In fact, given isometries $f_1,f_2$, its product $f_1\circ f_2$ is also an isometry so that  the set of isometries forms a group, called the \textbf{isometry group}. 
\vspace{.5cm}

For instance, any reflection, translation and rotation is a global isometry on Euclidean spaces. The isometry group $E(n)$ of $\bR^n$ is the Euclidian group, and  therefore it has as subgroups the translational group $T(n)$, and the orthogonal group $O(n)$. Any element of $E(n)$ is a translation followed by an orthogonal transformation (the linear part of the isometry), in a unique way:
$$
x \mapsto A (x + b)
$$
where $A\in O(n)$ is an orthogonal matrix.

\vspace{.5cm}

Using a metric $g$, one have an isomorphism between the vector field and one form 
$$
\hat g: \mathfrak{X}(M) \cong \Omega^1(M)
$$
in such a way that for vector fields $v,w\in  \mathfrak{X}(M)$ on $M$,
$$
\hat g(v)(w)=g(v,w)
$$
By using the isomorphism $T_pM\cong T^*_pM$, we can introduce an inner product on $T^*_pM$.


\section{Vielbeins, volume form, Hodge $\ast$ operator}

For simplicity, we will assume that the metric $g_{ij}$ is positive definite. For a metric with more general signature, we just have to introduce appropriate sign factors to some of the formulae below.
Since the metric $g_{ij}$ is symmetric, we can find a basis $\{e^a_i \}$ $(a=1,\cdots ,n)$ so that
$$
g_{ij} =
\sum^n_{a=1} e^a_ie^a_j~. 
$$
This basis is called \textbf{the orthonormal frame}. For a given metric, an orthonormal frame is defined modulo $O(n)$.


This can be done at each point $p$ on $M$. $e^a$'s are called \textbf{vielbeins} where viel means many in German, and bein is a leg. (In 4 dimensions, they are also called vierbeins or tetrads. In dimensions other than 4, words like f\"unfbein, etc. have been used. Vielbein covers all dimensions.)
Using the vierbeins, the volume form vol is defined by 
$$\textrm{vol} = e^1 \wedge e^2 \wedge \cdots \wedge e^n~.$$
Note that it may not be possible to define vol globally on $M$ since it is invariant under $SO(n)$ but not under $O(n)$. It may not be possible to choose a sign factor for vol (associated to $\bZ_2 = O(n)/SO(n)$) consistently over $M$. The volume form is well-defined if and only if $M$ is orientable.


Using coordinates, we can express the volume form as
$$\textrm{vol} = \sqrt{g}dx_1 \wedge dx_2 \wedge \cdots \wedge dx_n~,$$
where $g = \det g$ (we are assuming that the metric is positive definite). There is an isomorphism 
$$
\ast : \Omega^k(M)\to \Omega^{n-k}(M)
$$
is defined by
$$
\ast(e^1\wedge \cdots \wedge e^k)=e^{k+1}\wedge \cdots \wedge e^n
$$
For a $k$-form $\omega$, the \textbf{Hodge $\ast$ operator} is defined as 
$$
(\ast \omega)_{i_{k+1}\cdots i_n}=\frac{1}{k!} \frac{\epsilon^{j_1\cdots j_kj_{k+1}\cdots j_n}}{\sqrt{g}}\omega_{j_1\cdots j_k}g_{j_{k+1}i_{k+1}}\cdots g_{j_ni_n}~.
$$
Here I used the totally anti-symmetric tensor $\epsilon_{i_1\cdots i_n}$ and $\epsilon^{i_1\cdots i_n}$ normalized as 
$$
\epsilon_{12\cdots n}=\epsilon^{12\cdots n}=1
$$
The important point is that the Hodge star depends on the metric $g$.    Under coordinate transformations,  $\epsilon_{i_1\cdots i_n}$  does not transform as a tensor. However, we can remedy this by multiplying 
$\sqrt{g}$ to make it into the volume form. The volume form transforms as a tensor if coordinate transformations preserve the orientation. If we change the orientation, we get an extra $(-1)$. 

\section{Adjoint operator} 
The adjoint operator $\delta$ on $\Omega^k$ of the exterior differential $d$ is defined by
$$\delta\omega = (-1)^{nk+n+1} \ast d \ast \omega~.$$
We have commutative diagram
$$
\begin{tikzcd}
\Omega^k(M) \arrow[r, "\ast"] \arrow[d, "\delta"]
& \Omega^{n-k}(M) \arrow[d, "d" ] \\
\Omega^{k-1}(M)\arrow[r,  "(-1)^k\ast"]
& \Omega^{n-k+1}(M)
\end{tikzcd}
$$
We can easily verify the following properties,
$$\delta^2 =0~,\quad \ast\delta d=d\delta\ast~,\quad d\ast \delta=\delta 
\ast d=0~.$$
It is an adjoint operator in the sense that 
$$
(d\omega,\eta)=(\omega,\delta\eta)~.
$$
where we assume that $M$ is an oriented, compact, closed manifold.
where the inner product on $\Omega^k(M)$ is defined by
$$
(\rho,\sigma)=\int_M \rho\wedge \ast \sigma =\int_M\sigma\wedge \ast \rho
$$

On a Riemannian manifold $M$, an operator defined by
$$\Delta= \delta d + d\delta : \Omega^k(M) \to \Omega^k(M)~$$
is called \textbf{Laplace-Beltrami operator}. A form $\omega\in\Omega^*(M)$ such that $\Delta\omega=0$ is called a \textbf{harmonic form}. In particular, a function $f$ such that $\Delta f=0$ is called a \textbf{harmonic function}.
One can show that a necessary and sufficient condition of harmonic form: $\Delta\omega=0$ iff $d\omega=0=\delta\omega$.

\section{Hodge theorem and Hodge decomposition}
Let us denote the set of all harmonic $k$-forms on $M$
$$
\bH^k(M)=\{\omega\in\Omega^k(M)|\Delta\omega=0\}~.
$$

\begin{theorem}[Hodge theorem]
On an oriented compact Riemanninan manifold,  the natural map $\bH^k(M)\to H^k_{dR}(M)$ is isomorphism. 
\end{theorem}


\begin{theorem}[Hodge decomposition]
For an oriented compact Riemanninan manifold, we have the orthogonal decomposition
$$
\Omega^k(M)=\bH^k(M)\oplus d\Omega^{k-1}(M)\oplus \delta\Omega^{k+1}(M)
$$ 
\end{theorem}







Applications of the Hodge theorem is as follows.








\begin{theorem}[Poincare duality]
For an connected, oriented, compact $n$-dimensional Riemanninan manifold, the bilinear map 
$$
H_{dR}^k(M)\times H_{dR}^{n-k}(M)\to \bR; ~(\omega,\eta)\mapsto \int_M\omega\wedge \eta
$$
is  non-degenerate and hence induces an isomorphism
$$
H_{dR}^{n-k}(M) \cong  H_{dR}^k(M)^*
$$
\end{theorem}


The Hodge theorem tells us that $\omega\in H^*k(M)$ is a harmonic form. Due to $d\ast =\ast d$, $\eta=\ast \omega$ is also a harmonic form. If $\omega\neq 0$, we have 
$$
\int_M\omega\wedge \eta=||\omega ||^2\neq0
$$









\end{document}

