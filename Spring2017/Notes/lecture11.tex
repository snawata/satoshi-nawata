 \documentclass[12pt,a4paper]{article}
\usepackage{hyperref} % Use the Charter font for the document text
%\usepackage[UTF8]{ctex}
\usepackage{fullpage}
\usepackage{amsfonts,amssymb,amsmath}
\usepackage{mathtools}
\usepackage{tikz-cd}
\usepackage{tikz}
\usepackage{mathrsfs}

\usepackage{alltt}
\usepackage{amsfonts}
\usepackage{amsmath}
\usepackage{amssymb}
\usepackage{amsthm}
\usepackage{booktabs}
\usepackage{caption}
\usepackage{enumitem}
\usepackage{fancyhdr}
\usepackage{graphicx}
\usepackage{mathdots}
\usepackage{mathtools}
\usepackage{microtype}
\usepackage{multirow}
\usepackage{pdflscape}
\usepackage{pgfplots}
\usepackage{siunitx}
\usepackage{slashed}
\usepackage{tabularx}
\usepackage{tikz}
\usepackage{tkz-euclide}
\usepackage[normalem]{ulem}
\usepackage[all]{xy}
\usepackage{imakeidx}

\newcommand{\bA}{\ensuremath{\mathbb{A}}}
\newcommand{\bB}{\ensuremath{\mathbb{B}}}
\newcommand{\bC}{\ensuremath{\mathbb{C}}}
\newcommand{\bD}{\ensuremath{\mathbb{D}}}
\newcommand{\bE}{\ensuremath{\mathbb{E}}}
\newcommand{\bF}{\ensuremath{\mathbb{F}}}
\newcommand{\bG}{\ensuremath{\mathbb{G}}}
\newcommand{\bH}{\ensuremath{\mathbb{H}}}
\newcommand{\bI}{\ensuremath{\mathbb{I}}}
\newcommand{\bJ}{\ensuremath{\mathbb{J}}}
\newcommand{\bK}{\ensuremath{\mathbb{K}}}
\newcommand{\bL}{\ensuremath{\mathbb{L}}}
\newcommand{\bM}{\ensuremath{\mathbb{M}}}
\newcommand{\bN}{\ensuremath{\mathbb{N}}}
\newcommand{\bO}{\ensuremath{\mathbb{O}}}
\newcommand{\bP}{\ensuremath{\mathbb{P}}}
\newcommand{\bQ}{\ensuremath{\mathbb{Q}}}
\newcommand{\bR}{\ensuremath{\mathbb{R}}}
\newcommand{\bS}{\ensuremath{\mathbb{S}}}
\newcommand{\bT}{\ensuremath{\mathbb{T}}}
\newcommand{\bU}{\ensuremath{\mathbb{U}}}
\newcommand{\bV}{\ensuremath{\mathbb{V}}}
\newcommand{\bW}{\ensuremath{\mathbb{W}}}
\newcommand{\bX}{\ensuremath{\mathbb{X}}}
\newcommand{\bY}{\ensuremath{\mathbb{Y}}}
\newcommand{\bZ}{\ensuremath{\mathbb{Z}}}


%
%\parskip=1em
%\parindent=0.3in
%\setlength\oddsidemargin{0.5in} \setlength\evensidemargin{0.5in}
%\setlength\textwidth{5.5in}
%
%\hfuzz6pt % Don't bother to report over-full boxes if over-edge is < 6pt
%
%\newlength{\defbaselineskip}
%\setlength{\defbaselineskip}{\baselineskip}
%\newcommand{\setlinespacing}[1]%
%           {\setlength{\baselineskip}{#1 \defbaselineskip}}
%\newcommand{\doublespacing}{\setlength{\baselineskip}%
%                           {2.0 \defbaselineskip}}
%\newcommand{\singlespacing}{\setlength{\baselineskip}{\defbaselineskip}}
%
%\newcommand{\properpagestyle}{\pagestyle{myheadings}\markboth{}{}\markright{}}




\def\Ric{\mathop{\rm Ric}}
\def\cRic{\mathop{\stackrel{\circ}{\Ric}}}
\def\Scal{\mathop{\rm R}}
\def\scL{\mathop{\mathcal L}}
\def\Hess{\mathop{\rm Hess}}
\def\bt{\mathop{\bar\tau}}
\def\dist{\mathop{\rm dist}}
\def\Cut{\mathop{\rm Cut}}
\def\Riem{\mathop{\rm Rm}}
\def\scal{\mathop{\rm scal}}
\def\Sec{\mathop{\rm Sec}}
\def\Diam{\mathop{\rm Diam}}
\def\CS{\mathop{\rm C_S}}
\def\V{\mathop{\rm V}}
\def\Vol{\mathop{\rm Vol}}
\def\Area{\mathop{\rm Area}}
\def\VR{\mathop{\rm VR}}
\def\supp{\mathop{\rm supp}}
\def\div{\mathop{\rm div}}
\def\inj{\mathop{\rm inj}}
\def\diam{\mathop{\rm diam}}
\def\Id{\mathop{\rm Id}}
\def\RRR{\mathop{\mathcal{R}}}
\def\MMM{\mathop{\mathcal{M}}}
\def\HHH{\mathop{\mathcal{H}}}
\def\VVV{\mathop{\mathcal{V}}}
\def\FF{\mathop{\mathbb{F}}}
\def\RR{\mathop{\mathbb{R}}}
\def\QQ{\mathop{\mathbb{Q}}}
\def\CC{\mathop{\mathbb{C}}}
\def\ZZ{\mathop{\mathbb{Z}}}
\def\SS{\mathop{\mathbb{S}}}
\def\SSS{\mathop{\mathcal{S}}}
\def\PP{\mathop{\mathbb{P}}}
\def\End{\mathop{\rm End}}
\def\Aut{\mathop{\rm Aut}}
\def\Ad{\mathop{\rm Ad}}
\def\ad{\mathop{\rm ad}}
\def\hht{\mathop{\rm ht}}
\def\gl{\mathop{\mathfrak{gl}}}
\def\ssl{\mathop{\mathfrak{sl}}}
\def\TP{\mathop{\mathcal{TP}}}
\def\PPP{\mathop{\mathcal{P}}}
\def\gggg{\mathop{\mathfrak{g}}}
\def\ffff{\mathop{\mathfrak{f}}}
\def\OO{\mathop{\mathcal{O}}}
\def\oo{\mathop{\mathfrak{o}}}
\def\GG{\mathop{\mathcal{G}}}
\def\WWW{\mathop{\mathcal{W}}}
\def\Rad{\mathop{\rm Rad}}
\def\Der{\mathop{\rm Der}}
\def\Ker{\mathop{\rm Ker}}
\def\Im{\mathop{\rm Im}}

\def\be{\begin{eqnarray}}
\def\ee{\end{eqnarray}}
\def\beg{\begin{eqnarray*}}
\def\ees{\end{eqnarray*}}


%\newcommand{\qed}{\hfill$\Box$}
\theoremstyle{definition}
\newtheorem*{aim}{Aim}
\newtheorem*{axiom}{Axiom}
\newtheorem*{claim}{Claim}
\newtheorem*{cor}{Corollary}
\newtheorem*{conjecture}{Conjecture}
\newtheorem*{defi}{Definition}
\newtheorem*{eg}{Example}
\newtheorem*{ex}{Exercise}
\newtheorem*{fact}{Fact}
\newtheorem*{law}{Law}
\newtheorem*{lemma}{Lemma}
\newtheorem*{notation}{Notation}
\newtheorem*{prop}{Proposition}
\newtheorem*{question}{Question}
\newtheorem*{thm}{Theorem}





% Maths symbols
\newcommand{\abs}[1]{\left\lvert #1\right\rvert}
%\newcommand\ad{\mathrm{ad}}
\newcommand\AND{\mathsf{AND}}
\newcommand\Art{\mathrm{Art}}
\newcommand{\Bilin}{\mathrm{Bilin}}
\newcommand{\bket}[1]{\left\lvert #1\right\rangle}
\newcommand{\B}{\mathcal{B}}
\newcommand{\bolds}[1]{{\bfseries #1}}
\newcommand{\brak}[1]{\left\langle #1 \right\rvert}
\newcommand{\braket}[2]{\left\langle #1\middle\vert #2 \right\rangle}
\newcommand{\bra}{\langle}
\newcommand{\cat}[1]{\mathsf{#1}}
\newcommand{\C}{\mathbb{C}}
\newcommand{\CP}{\mathbb{CP}}
\newcommand{\cU}{\mathcal{U}}
%\newcommand{\Der}{\mathrm{Der}}
\newcommand{\D}{\mathrm{D}}
\newcommand{\dR}{\mathrm{dR}}
\newcommand{\E}{\mathbb{E}}
\newcommand{\F}{\mathbb{F}}
\newcommand{\Frob}{\mathrm{Frob}}
%\newcommand{\GG}{\mathbb{G}}
%\newcommand{\gl}{\mathfrak{gl}}
\newcommand{\GL}{\mathrm{GL}}
\newcommand{\G}{\mathcal{G}}
\newcommand{\Gr}{\mathrm{Gr}}
\newcommand{\haut}{\mathrm{ht}}
\newcommand{\Hol}{\mathrm{Hol}}
\newcommand{\hol}{\mathfrak{hol}}
%\newcommand{\Id}{\mathrm{Id}}
\newcommand{\ket}{\rangle}
\newcommand{\lie}[1]{\mathfrak{#1}}
\newcommand{\Mat}{\mathrm{Mat}}
\newcommand{\N}{\mathbb{N}}
\newcommand{\norm}[1]{\left\lVert #1\right\rVert}
\newcommand{\normalorder}[1]{\mathop{:}\nolimits\!#1\!\mathop{:}\nolimits}
\newcommand{\NOT}{\mathsf{NOT}}
\newcommand{\op}{\mathrm{op}}
\newcommand{\Oc}{\mathcal{O}}
\newcommand{\Or}{\mathrm{O}}
\newcommand\OR{\mathsf{OR}}
\newcommand{\ort}{\mathfrak{o}}
\newcommand{\PGL}{\mathrm{PGL}}
\newcommand{\ph}{\,\cdot\,}
\newcommand{\pr}{\mathrm{pr}}
\newcommand{\Prob}{\mathbb{P}}
\newcommand{\PSL}{\mathrm{PSL}}
\newcommand{\Ps}{\mathcal{P}}
\newcommand{\PSU}{\mathrm{PSU}}
\newcommand{\pt}{\mathrm{pt}}
\newcommand{\qeq}{\mathrel{``{=}"}}
\newcommand{\Q}{\mathbb{Q}}
\newcommand{\R}{\mathbb{R}}
\newcommand{\RP}{\mathbb{RP}}
\newcommand{\Rs}{\mathcal{R}}
\newcommand{\SL}{\mathrm{SL}}
\newcommand{\so}{\mathfrak{so}}
\newcommand{\SO}{\mathrm{SO}}
\newcommand{\Spin}{\mathrm{Spin}}
\newcommand{\Sp}{\mathrm{Sp}}
\newcommand{\su}{\mathfrak{su}}
\newcommand{\SU}{\mathrm{SU}}
\newcommand{\term}[1]{\textbf{#1}\index{#1}}
\newcommand{\T}{\mathbb{T}}
\newcommand{\tv}[1]{|#1|}
\newcommand{\U}{\mathrm{U}}
\newcommand{\uu}{\mathfrak{u}}
\newcommand{\Vect}{\mathrm{Vect}}
\newcommand{\wsto}{\stackrel{\mathrm{w}^*}{\to}}
\newcommand{\wt}{\mathrm{wt}}
\newcommand{\wto}{\stackrel{\mathrm{w}}{\to}}
\newcommand{\Z}{\mathbb{Z}}
\renewcommand{\d}{\mathrm{d}}
\renewcommand{\H}{\mathbb{H}}
\renewcommand{\P}{\mathbb{P}}
\renewcommand{\sl}{\mathfrak{sl}}
\renewcommand{\vec}[1]{\boldsymbol{\mathbf{#1}}}
%\renewcommand{\F}{\mathcal{F}}

\def\sm{\mathop{C^\infty}}


\let\Im\relax
\let\Re\relax

\DeclareMathOperator{\adj}{adj}
\DeclareMathOperator{\Ann}{Ann}
\DeclareMathOperator{\area}{area}
%\DeclareMathOperator{\Aut}{Aut}
\DeclareMathOperator{\Bernoulli}{Bernoulli}
\DeclareMathOperator{\betaD}{beta}
\DeclareMathOperator{\bias}{bias}
\DeclareMathOperator{\binomial}{binomial}
\DeclareMathOperator{\card}{card}
\DeclareMathOperator{\ccl}{ccl}
\DeclareMathOperator{\Char}{char}
\DeclareMathOperator{\ch}{ch}
\DeclareMathOperator{\cl}{cl}
\DeclareMathOperator{\cls}{\overline{\mathrm{span}}}
\DeclareMathOperator{\coker}{coker}
\DeclareMathOperator{\conv}{conv}
\DeclareMathOperator{\corr}{corr}
\DeclareMathOperator{\cosec}{cosec}
\DeclareMathOperator{\cosech}{cosech}
\DeclareMathOperator{\cov}{cov}
\DeclareMathOperator{\covol}{covol}
\DeclareMathOperator{\diag}{diag}
%\DeclareMathOperator{\diam}{diam}
\DeclareMathOperator{\Diff}{Diff}
\DeclareMathOperator{\disc}{disc}
\DeclareMathOperator{\dom}{dom}
%\DeclareMathOperator{\End}{End}
\DeclareMathOperator{\energy}{energy}
\DeclareMathOperator{\erfc}{erfc}
\DeclareMathOperator{\erf}{erf}
\DeclareMathOperator*{\esssup}{ess\,sup}
\DeclareMathOperator{\ev}{ev}
\DeclareMathOperator{\Ext}{Ext}
\DeclareMathOperator{\fst}{fst}
\DeclareMathOperator{\Fit}{Fit}
\DeclareMathOperator{\fix}{fix}
\DeclareMathOperator{\Frac}{Frac}
\DeclareMathOperator{\Gal}{Gal}
\DeclareMathOperator{\gammaD}{gamma}
\DeclareMathOperator{\gr}{gr}
\DeclareMathOperator{\hcf}{hcf}
\DeclareMathOperator{\Hom}{Hom}
\DeclareMathOperator{\id}{id}
\DeclareMathOperator{\Image}{image}
\DeclareMathOperator{\Im}{Im}
\DeclareMathOperator{\Ind}{Ind}
\DeclareMathOperator{\Int}{Int}
\DeclareMathOperator{\Isom}{Isom}
\DeclareMathOperator{\lcm}{lcm}
\DeclareMathOperator{\length}{length}
\DeclareMathOperator{\Lie}{Lie}
\DeclareMathOperator{\like}{like}
\DeclareMathOperator{\Lk}{Lk}
\DeclareMathOperator{\Maps}{Maps}
\DeclareMathOperator{\mse}{mse}
\DeclareMathOperator{\multinomial}{multinomial}
\DeclareMathOperator{\orb}{orb}
\DeclareMathOperator{\ord}{ord}
\DeclareMathOperator{\otp}{otp}
\DeclareMathOperator{\Poisson}{Poisson}
\DeclareMathOperator{\poly}{poly}
\DeclareMathOperator{\rank}{rank}
\DeclareMathOperator{\rel}{rel}
%\DeclareMathOperator{\Rad}{Rad}
\DeclareMathOperator{\Re}{Re}
\DeclareMathOperator*{\res}{res}
\DeclareMathOperator{\Res}{Res}
%\DeclareMathOperator{\Ric}{Ric}
\DeclareMathOperator{\rk}{rk}
\DeclareMathOperator{\Rees}{Rees}
\DeclareMathOperator{\Root}{Root}
\DeclareMathOperator{\sech}{sech}
\DeclareMathOperator{\sgn}{sgn}
\DeclareMathOperator{\snd}{snd}
\DeclareMathOperator{\Spec}{Spec}
\DeclareMathOperator{\spn}{span}
\DeclareMathOperator{\stab}{stab}
\DeclareMathOperator{\St}{St}
%\DeclareMathOperator{\supp}{supp}
\DeclareMathOperator{\Syl}{Syl}
\DeclareMathOperator{\Sym}{Sym}
\DeclareMathOperator{\tr}{tr}
\DeclareMathOperator{\Tr}{Tr}
\DeclareMathOperator{\var}{var}
\DeclareMathOperator{\vol}{vol}
\usetikzlibrary{knots}




\pgfarrowsdeclarecombine{twolatex'}{twolatex'}{latex'}{latex'}{latex'}{latex'}
\tikzset{->/.style = {decoration={markings,
                                  mark=at position 1 with {\arrow[scale=2]{latex'}}},
                      postaction={decorate}}}
\tikzset{<-/.style = {decoration={markings,
                                  mark=at position 0 with {\arrowreversed[scale=2]{latex'}}},
                      postaction={decorate}}}
\tikzset{<->/.style = {decoration={markings,
                                   mark=at position 0 with {\arrowreversed[scale=2]{latex'}},
                                   mark=at position 1 with {\arrow[scale=2]{latex'}}},
                       postaction={decorate}}}
\tikzset{->-/.style = {decoration={markings,
                                   mark=at position #1 with {\arrow[scale=2]{latex'}}},
                       postaction={decorate}}}
\tikzset{-<-/.style = {decoration={markings,
                                   mark=at position #1 with {\arrowreversed[scale=2]{latex'}}},
                       postaction={decorate}}}
\tikzset{->>/.style = {decoration={markings,
                                  mark=at position 1 with {\arrow[scale=2]{latex'}}},
                      postaction={decorate}}}
\tikzset{<<-/.style = {decoration={markings,
                                  mark=at position 0 with {\arrowreversed[scale=2]{twolatex'}}},
                      postaction={decorate}}}
\tikzset{<<->>/.style = {decoration={markings,
                                   mark=at position 0 with {\arrowreversed[scale=2]{twolatex'}},
                                   mark=at position 1 with {\arrow[scale=2]{twolatex'}}},
                       postaction={decorate}}}
\tikzset{->>-/.style = {decoration={markings,
                                   mark=at position #1 with {\arrow[scale=2]{twolatex'}}},
                       postaction={decorate}}}
\tikzset{-<<-/.style = {decoration={markings,
                                   mark=at position #1 with {\arrowreversed[scale=2]{twolatex'}}},
                       postaction={decorate}}}


\tikzset{circ/.style = {fill, circle, inner sep = 0, minimum size = 3}}
\tikzset{scirc/.style = {fill, circle, inner sep = 0, minimum size = 1.5}}
\tikzset{mstate/.style={circle, draw, blue, text=black, minimum width=0.7cm}}

\tikzset{eqpic/.style={baseline={([yshift=-.5ex]current bounding box.center)}}}
\tikzset{commutative diagrams/.cd,cdmap/.style={/tikz/column 1/.append style={anchor=base east},/tikz/column 2/.append style={anchor=base west},row sep=tiny}}


\definecolor{mblue}{rgb}{0.2, 0.3, 0.8}
\definecolor{morange}{rgb}{1, 0.5, 0}
\definecolor{mgreen}{rgb}{0.1, 0.4, 0.2}
\definecolor{mred}{rgb}{0.5, 0, 0}


%\title{ Lecture 4}
\begin{document}\thispagestyle{empty}

\centerline{\Large \bf Lecture 11}

\centerline{\Large \bf The end of Morita, and Nakahara section 11.5 and 12}

\section{Flat connections and holonomy homomorphisms}
Let $P\to M$ be a principal $G$-bundle over a manifold $M$. An Ehresmann connection $A$ on $P$ is called a \term{flat connection} If its curvature form $F$ is identically zero $F=0$.  A principal $G$-bundle equipped with a fiat connection is called a flat $G$-bundle. As an example of flat bundle, the produce bundle $M\times G$ has a trivial connection, which is flat. This example is too trivial, and there are much richer stories for flat $G$-bundles.

A connection $A$ determines the horizontal direction of a principal $G$-bundle. Namely at each point $u\in P$, we define
$$
\mathcal{H}_u = \{X\in T_uP|A(X) = 0 \}~.
$$
The collection $\mathcal{H}=\cup_u \mathcal{H}_u$ is called \term{distribution}. The distribution is called \term{completely integrable} if 
$$ 
\textrm{for any two vector fields } X,Y\in \mathcal{H} \longrightarrow [X,Y]\in \mathcal{H}~.
$$ 
Another way to describe completely integrable distribution $\mathcal{H}$ is that for any point on $M$ there exists an integral manifold containing it. (A submanifold $N$ of $M$ is called an \term{integral manifold} of $\mathcal{H}$ if $T_uN=\mathcal{H}_u$ for ${}^\forall u\in N$.)
 

For horizontal vector field $X,Y$, we have $A(X)=0=A(Y)$. 
$$
F(X,Y)=dA(X,Y)=\frac12\left\{X(A(Y))+Y(A(X))-A([X,Y])\right\}=-\frac12 A([X,Y])
$$
Therefore, $F=0 \leftrightarrow [X,Y]$ horizontal vector field. A connection $A$ on a principal $G$-bundle is fiat if and only if the corresponding distribution $\mathcal{H}$ is completely integrable. 


\begin{figure}[h]\centering
\includegraphics[scale =.8]{holonomy}
\end{figure}


Then, for each point $u \in P$ if we denote by $L_u$ the maximal integral manifold passing through it, then the projection $\pi : L_u \to  M$ becomes a covering map. Now let $p_0 \in M$ be a base point of the base space and choose $u_0 \in \pi^{-1}(p_0)$. Then a homomorphism
$$\rho : \pi_1(M) \to G~,$$
called the \term{holonomy homomorphism}, is defined as follows.  For each element $\alpha\in \pi_1(M)$  of the fundamental group, we choose a closed curve $\gamma$ with initial point $p_0$ which represents $\alpha$. Let $\tilde \gamma$ be the lift of $\gamma$ to $L_{u_0}$ with initial point $u_0$. Then we can write the end point of $\tilde \gamma=u_0\cdot g$. Since $\pi: L_{u_0} \to M$ is the covering map. we see that the end point of $\tilde\gamma$ is determined uniquely by $\alpha$, and it is independent of choice of $\tilde \gamma$. Then, we set
$$
\rho(\gamma)=g^{-1}~.
$$

\begin{thm}
Via the holonomy homomorphism, the set of flat connections on $P$ is in a one-to-one correspondence with the set of conjugacy classes of homomorphisms $\rho : \pi_1(M) \to G$.
\end{thm}

\section{Chern-Simons theory}
Let $M$ be a compact 3-manifold. We will consider a particular physical theory called \term{Chern-Simons} theory on 3-dimension. Let $P=M\times G$ be a trivial principal $G$-bundle and we denote a connection on $P$ by $A$.


The Chern-Simons action for $A$ can be written as
\begin{align}
S_{CS}[M,A] &= \frac{k}{4\pi} \int_M \operatorname{Tr}\left(A \wedge dA + \frac23 A \wedge A \wedge A\right)\cr
&=\frac{k}{8\pi}\int_M \epsilon^{\mu\nu\rho} \Tr \left( A_\mu(\partial_\nu A_\rho -\partial_\rho A_\nu)+ \frac23A_\mu [A_\nu,A_\rho] \right)\nonumber
\end{align}
The action is independent of metric of $M$ so it gives a topological invariant of $M$. In fact, the Chern-Simons 3-form is another kind of characteristic class of flat $G$-bundle.
The a parameter $k$ of the theory (inverse of the coupling constant) is called \term{level}. If $G$ is compact and semi-simple, the level $k$ has to be an integer in order for the action to be gauge invariant.
 Classically the equations of motion which are the extrema of the action are flat connections:
 $${\frac {\delta S}{\delta A}}={\frac {k}{2\pi }}F=0~.$$

\subsection{Abelian Chern-Simons theory}
Let us first consider the case when $G = \U(1)$, namely the Abelian Chern-Simons theory. It
has the action,
$$S_{U(1)}[M,A] = \frac{k}{4\pi}  \int_M A\wedge dA~,$$
Since $\U(1)$ is not a semi-simple group, the level $k$ is not necessarily an integer in this case. The Abelian Chern-Simons theory describes the fractional quantum Hall effect as we will see in the last lecture.

Given an orientable close 3-manifold $M$, we can consider the path integral of $\U(1)$ Chern-Simons theory
$$
Z[M]=\int_{\mathscr{A}/\mathscr{G}} \mathcal{D}A  e^{iS_{U(1)}(M,A)}~.
$$
This can be evaluated by so-called \term{one-loop determinent} and  $Z[M]$ turns out to be also topological invariant, called \term{analytic (Reidemeister) torsion}. This means that Chern-Simons theory provides topological invariants even at quantum level!
This was first shown by A. Schwarz in 1978, giving the first construction of what we now call a \term{topological quantum field theory} \cite{Schwarz:1978}.


Furthermore, we can consider holonomy group in Chern-Simons theory on $M=S^3$. Given a loop $\gamma:I\to S^3$ with $I(0)=I(1)=p_0$, the parallel transport along $K$ with respect to $A$ provides a holonomy group and it is expressed as 
$$
u_0\to u_0\cdot\exp(i\oint_K A)~.
$$
We denote it by
$$
W_K=\exp(i\oint_K A)
$$
This is called \term{Wilson loop operator}, which plays an important role in physics. The expectation value of the Wilson loop operator can be expressed by the Feynman path integral
$$
\langle W_K\rangle =\int_{\mathscr{A}/\mathscr{G}} \mathcal{D}A ~W_K(A)~ e^{iS_{U(1)}(A)}~.
$$
Let us evaluate the expectation value of two loops $K_1$ and $K_2$ in $\bR^3$.
\be\label{twopt}\langle W_{K_1}W_{K_2}\rangle=\left \langle \exp(\oint_{K_1}dx_1^\mu A_\mu\oint_{K_2}dx_2^\nu A_\nu ) \right\rangle~.\ee
Clearly, this expression
can be easily evaluated in terms of the two-point correlator (propagator) in $S^3$
$$\langle A_\mu(x) A_\nu (y)\rangle=\frac{i}{k}\epsilon_{\mu\nu\rho}\frac{(x-y)^\rho}{|x-y|^3}~.$$
Plugging it into \eqref{twopt}, the expectation value can be written in terms of linking number
$$
\label{two-pt}\langle W_{K_1}(A) W_{K_2}(A)\rangle=\exp \left(\frac{4\pi i}{k} Lk(K_1,K_2)\right)~.
$$





\subsection{Non-Abelian Chern-Simons theory}

Generalization to non-Abelian Chern-Simons theory has been done by the seminal paper of Witten \cite{Witten:1988hf}.
Let us consider Wilson loop in non-Abelian Chern-Simons theory where the connections no longer commute. Therefore, the holonomy group should be written 
$$
u_0\to u_0\cdot P \exp (i\oint_K A)
$$
where $P$ is the path-ordered integral due to non-commutativity: 
$$P \exp (i\oint_K A)=\prod _{t=0}^{1}e^{A(\gamma(t'))\,dt'}\equiv \lim _{N\rightarrow \infty }\left(e^{A(\gamma(t_{N}))\Delta t}e^{A(\gamma(t_{N-1}))\Delta t}\cdots e^{A(\gamma(t_{1}))\Delta t}e^{A(\gamma(t_{0}))\Delta t}\right)
$$
where we subdivide  $1\ge t_N\ge\cdots \ge t_0\ge0$ by $\Delta t=\frac1N$.
If the starting point is different $u_0\to u_0'=u_0\cdot g$, then holonomy group is 
$$
P \exp (i\oint_K A)\to g \cdot P \exp (i\oint_K A)\cdot g^{-1}~.
$$
Therefore, we can define the operator independent of a starting point by taking trace
$$
W_K:=\Tr P \exp (i\oint_K A)~.
$$
When $G = \SU(2)$, the expectation value of the Wilson loops 
$$
\langle W_K\rangle =\int_{\mathscr{A}/\mathscr{G}} \mathcal{D}A ~W_K(A)~ e^{iS_{CS}(A)}~.
$$
gives the Jones polynomial which are invariants of knots and links \cite{Witten:1988hf}.


Jones polynomials are knot invariants which can be computed by the following skein relation
$$q^2 J\left({\raisebox{-.2cm}{\includegraphics[width=.6cm]{overcrossing}}}\right)
- q^{-2}J\left({\raisebox{-.2cm}{\includegraphics[width=.6cm]{undercrossing}}}\right)
=
(q-q^{-1}) J\left({\raisebox{-.2cm}{\includegraphics[width=.6cm]{smoothing}}}\right)\,.
$$
where the ``quantum'' parameter $q$ is expressed as the Chern-Simons level
$$
q=\exp \left( \frac{2\pi i}{k+2}\right)~.
$$


\begin{thebibliography}{99}
\bibitem{Schwarz:1978}
Albert S. Schwarz,  \emph{The partition function of degenerate quadratic functional and Ray-Singer invariants}, Letters in Mathematical Physics 2.3 (1978): 247-252.

\bibitem{Witten:1988hf}
E.~Witten, {\it {Quantum Field Theory and the Jones Polynomial}},  {\em Commun.
  Math. Phys.} {\bf 121} (1989) 351--399.

\end{thebibliography}

\end{document}
