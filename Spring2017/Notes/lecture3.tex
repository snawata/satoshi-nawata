 \documentclass[12pt,a4paper]{article}
\usepackage{hyperref} % Use the Charter font for the document text
%\usepackage[UTF8]{ctex}
\usepackage{fullpage}
\usepackage{amsfonts,amssymb,amsmath}
\usepackage{mathtools}
\usepackage{tikz-cd}
\usepackage{tikz}

\usepackage{alltt}
\usepackage{amsfonts}
\usepackage{amsmath}
\usepackage{amssymb}
\usepackage{amsthm}
\usepackage{booktabs}
\usepackage{caption}
\usepackage{enumitem}
\usepackage{fancyhdr}
\usepackage{graphicx}
\usepackage{mathdots}
\usepackage{mathtools}
\usepackage{microtype}
\usepackage{multirow}
\usepackage{pdflscape}
\usepackage{pgfplots}
\usepackage{siunitx}
\usepackage{slashed}
\usepackage{tabularx}
\usepackage{tikz}
\usepackage{tkz-euclide}
\usepackage[normalem]{ulem}
\usepackage[all]{xy}
\usepackage{imakeidx}

\newcommand{\bA}{\ensuremath{\mathbb{A}}}
\newcommand{\bB}{\ensuremath{\mathbb{B}}}
\newcommand{\bC}{\ensuremath{\mathbb{C}}}
\newcommand{\bD}{\ensuremath{\mathbb{D}}}
\newcommand{\bE}{\ensuremath{\mathbb{E}}}
\newcommand{\bF}{\ensuremath{\mathbb{F}}}
\newcommand{\bG}{\ensuremath{\mathbb{G}}}
\newcommand{\bH}{\ensuremath{\mathbb{H}}}
\newcommand{\bI}{\ensuremath{\mathbb{I}}}
\newcommand{\bJ}{\ensuremath{\mathbb{J}}}
\newcommand{\bK}{\ensuremath{\mathbb{K}}}
\newcommand{\bL}{\ensuremath{\mathbb{L}}}
\newcommand{\bM}{\ensuremath{\mathbb{M}}}
\newcommand{\bN}{\ensuremath{\mathbb{N}}}
\newcommand{\bO}{\ensuremath{\mathbb{O}}}
\newcommand{\bP}{\ensuremath{\mathbb{P}}}
\newcommand{\bQ}{\ensuremath{\mathbb{Q}}}
\newcommand{\bR}{\ensuremath{\mathbb{R}}}
\newcommand{\bS}{\ensuremath{\mathbb{S}}}
\newcommand{\bT}{\ensuremath{\mathbb{T}}}
\newcommand{\bU}{\ensuremath{\mathbb{U}}}
\newcommand{\bV}{\ensuremath{\mathbb{V}}}
\newcommand{\bW}{\ensuremath{\mathbb{W}}}
\newcommand{\bX}{\ensuremath{\mathbb{X}}}
\newcommand{\bY}{\ensuremath{\mathbb{Y}}}
\newcommand{\bZ}{\ensuremath{\mathbb{Z}}}


%
%\parskip=1em
%\parindent=0.3in
%\setlength\oddsidemargin{0.5in} \setlength\evensidemargin{0.5in}
%\setlength\textwidth{5.5in}
%
%\hfuzz6pt % Don't bother to report over-full boxes if over-edge is < 6pt
%
%\newlength{\defbaselineskip}
%\setlength{\defbaselineskip}{\baselineskip}
%\newcommand{\setlinespacing}[1]%
%           {\setlength{\baselineskip}{#1 \defbaselineskip}}
%\newcommand{\doublespacing}{\setlength{\baselineskip}%
%                           {2.0 \defbaselineskip}}
%\newcommand{\singlespacing}{\setlength{\baselineskip}{\defbaselineskip}}
%
%\newcommand{\properpagestyle}{\pagestyle{myheadings}\markboth{}{}\markright{}}




\def\Ric{\mathop{\rm Ric}}
\def\cRic{\mathop{\stackrel{\circ}{\Ric}}}
\def\Scal{\mathop{\rm R}}
\def\scL{\mathop{\mathcal L}}
\def\Hess{\mathop{\rm Hess}}
\def\bt{\mathop{\bar\tau}}
\def\dist{\mathop{\rm dist}}
\def\Cut{\mathop{\rm Cut}}
\def\Riem{\mathop{\rm Rm}}
\def\scal{\mathop{\rm scal}}
\def\Sec{\mathop{\rm Sec}}
\def\Diam{\mathop{\rm Diam}}
\def\CS{\mathop{\rm C_S}}
\def\V{\mathop{\rm V}}
\def\Vol{\mathop{\rm Vol}}
\def\Area{\mathop{\rm Area}}
\def\VR{\mathop{\rm VR}}
\def\supp{\mathop{\rm supp}}
\def\div{\mathop{\rm div}}
\def\inj{\mathop{\rm inj}}
\def\diam{\mathop{\rm diam}}
\def\Id{\mathop{\rm Id}}
\def\RRR{\mathop{\mathcal{R}}}
\def\MMM{\mathop{\mathcal{M}}}
\def\HHH{\mathop{\mathcal{H}}}
\def\VVV{\mathop{\mathcal{V}}}
\def\FF{\mathop{\mathbb{F}}}
\def\RR{\mathop{\mathbb{R}}}
\def\QQ{\mathop{\mathbb{Q}}}
\def\CC{\mathop{\mathbb{C}}}
\def\ZZ{\mathop{\mathbb{Z}}}
\def\SS{\mathop{\mathbb{S}}}
\def\SSS{\mathop{\mathcal{S}}}
\def\PP{\mathop{\mathbb{P}}}
\def\End{\mathop{\rm End}}
\def\Aut{\mathop{\rm Aut}}
\def\Ad{\mathop{\rm Ad}}
\def\ad{\mathop{\rm ad}}
\def\hht{\mathop{\rm ht}}
\def\gl{\mathop{\mathfrak{gl}}}
\def\ssl{\mathop{\mathfrak{sl}}}
\def\TP{\mathop{\mathcal{TP}}}
\def\PPP{\mathop{\mathcal{P}}}
\def\gggg{\mathop{\mathfrak{g}}}
\def\ffff{\mathop{\mathfrak{f}}}
\def\OO{\mathop{\mathcal{O}}}
\def\oo{\mathop{\mathfrak{o}}}
\def\GG{\mathop{\mathcal{G}}}
\def\WWW{\mathop{\mathcal{W}}}
\def\Rad{\mathop{\rm Rad}}
\def\Der{\mathop{\rm Der}}
\def\Ker{\mathop{\rm Ker}}
\def\Im{\mathop{\rm Im}}

\def\be{\begin{eqnarray}}
\def\ee{\end{eqnarray}}
\def\beg{\begin{eqnarray*}}
\def\ees{\end{eqnarray*}}


%\newcommand{\qed}{\hfill$\Box$}
\theoremstyle{definition}
\newtheorem*{aim}{Aim}
\newtheorem*{axiom}{Axiom}
\newtheorem*{claim}{Claim}
\newtheorem*{cor}{Corollary}
\newtheorem*{conjecture}{Conjecture}
\newtheorem*{defi}{Definition}
\newtheorem*{eg}{Example}
\newtheorem*{ex}{Exercise}
\newtheorem*{fact}{Fact}
\newtheorem*{law}{Law}
\newtheorem*{lemma}{Lemma}
\newtheorem*{notation}{Notation}
\newtheorem*{prop}{Proposition}
\newtheorem*{question}{Question}
\newtheorem*{thm}{Theorem}





% Maths symbols
\newcommand{\abs}[1]{\left\lvert #1\right\rvert}
%\newcommand\ad{\mathrm{ad}}
\newcommand\AND{\mathsf{AND}}
\newcommand\Art{\mathrm{Art}}
\newcommand{\Bilin}{\mathrm{Bilin}}
\newcommand{\bket}[1]{\left\lvert #1\right\rangle}
\newcommand{\B}{\mathcal{B}}
\newcommand{\bolds}[1]{{\bfseries #1}}
\newcommand{\brak}[1]{\left\langle #1 \right\rvert}
\newcommand{\braket}[2]{\left\langle #1\middle\vert #2 \right\rangle}
\newcommand{\bra}{\langle}
\newcommand{\cat}[1]{\mathsf{#1}}
\newcommand{\C}{\mathbb{C}}
\newcommand{\CP}{\mathbb{CP}}
\newcommand{\cU}{\mathcal{U}}
%\newcommand{\Der}{\mathrm{Der}}
\newcommand{\D}{\mathrm{D}}
\newcommand{\dR}{\mathrm{dR}}
\newcommand{\E}{\mathbb{E}}
\newcommand{\F}{\mathbb{F}}
\newcommand{\Frob}{\mathrm{Frob}}
%\newcommand{\GG}{\mathbb{G}}
%\newcommand{\gl}{\mathfrak{gl}}
\newcommand{\GL}{\mathrm{GL}}
\newcommand{\G}{\mathcal{G}}
\newcommand{\Gr}{\mathrm{Gr}}
\newcommand{\haut}{\mathrm{ht}}
\newcommand{\Hol}{\mathrm{Hol}}
\newcommand{\hol}{\mathfrak{hol}}
%\newcommand{\Id}{\mathrm{Id}}
\newcommand{\ket}{\rangle}
\newcommand{\lie}[1]{\mathfrak{#1}}
\newcommand{\Mat}{\mathrm{Mat}}
\newcommand{\N}{\mathbb{N}}
\newcommand{\norm}[1]{\left\lVert #1\right\rVert}
\newcommand{\normalorder}[1]{\mathop{:}\nolimits\!#1\!\mathop{:}\nolimits}
\newcommand{\NOT}{\mathsf{NOT}}
\newcommand{\op}{\mathrm{op}}
\newcommand{\Oc}{\mathcal{O}}
\newcommand{\Or}{\mathrm{O}}
\newcommand\OR{\mathsf{OR}}
\newcommand{\ort}{\mathfrak{o}}
\newcommand{\PGL}{\mathrm{PGL}}
\newcommand{\ph}{\,\cdot\,}
\newcommand{\pr}{\mathrm{pr}}
\newcommand{\Prob}{\mathbb{P}}
\newcommand{\PSL}{\mathrm{PSL}}
\newcommand{\Ps}{\mathcal{P}}
\newcommand{\PSU}{\mathrm{PSU}}
\newcommand{\pt}{\mathrm{pt}}
\newcommand{\qeq}{\mathrel{``{=}"}}
\newcommand{\Q}{\mathbb{Q}}
\newcommand{\R}{\mathbb{R}}
\newcommand{\RP}{\mathbb{RP}}
\newcommand{\Rs}{\mathcal{R}}
\newcommand{\SL}{\mathrm{SL}}
\newcommand{\so}{\mathfrak{so}}
\newcommand{\SO}{\mathrm{SO}}
\newcommand{\Spin}{\mathrm{Spin}}
\newcommand{\Sp}{\mathrm{Sp}}
\newcommand{\su}{\mathfrak{su}}
\newcommand{\SU}{\mathrm{SU}}
\newcommand{\term}[1]{\textbf{#1}\index{#1}}
\newcommand{\T}{\mathbb{T}}
\newcommand{\tv}[1]{|#1|}
\newcommand{\U}{\mathrm{U}}
\newcommand{\uu}{\mathfrak{u}}
\newcommand{\Vect}{\mathrm{Vect}}
\newcommand{\wsto}{\stackrel{\mathrm{w}^*}{\to}}
\newcommand{\wt}{\mathrm{wt}}
\newcommand{\wto}{\stackrel{\mathrm{w}}{\to}}
\newcommand{\Z}{\mathbb{Z}}
\renewcommand{\d}{\mathrm{d}}
\renewcommand{\H}{\mathbb{H}}
\renewcommand{\P}{\mathbb{P}}
\renewcommand{\sl}{\mathfrak{sl}}
\renewcommand{\vec}[1]{\boldsymbol{\mathbf{#1}}}
%\renewcommand{\F}{\mathcal{F}}

\let\Im\relax
\let\Re\relax

\DeclareMathOperator{\adj}{adj}
\DeclareMathOperator{\Ann}{Ann}
\DeclareMathOperator{\area}{area}
%\DeclareMathOperator{\Aut}{Aut}
\DeclareMathOperator{\Bernoulli}{Bernoulli}
\DeclareMathOperator{\betaD}{beta}
\DeclareMathOperator{\bias}{bias}
\DeclareMathOperator{\binomial}{binomial}
\DeclareMathOperator{\card}{card}
\DeclareMathOperator{\ccl}{ccl}
\DeclareMathOperator{\Char}{char}
\DeclareMathOperator{\ch}{ch}
\DeclareMathOperator{\cl}{cl}
\DeclareMathOperator{\cls}{\overline{\mathrm{span}}}
\DeclareMathOperator{\coker}{coker}
\DeclareMathOperator{\conv}{conv}
\DeclareMathOperator{\corr}{corr}
\DeclareMathOperator{\cosec}{cosec}
\DeclareMathOperator{\cosech}{cosech}
\DeclareMathOperator{\cov}{cov}
\DeclareMathOperator{\covol}{covol}
\DeclareMathOperator{\diag}{diag}
%\DeclareMathOperator{\diam}{diam}
\DeclareMathOperator{\Diff}{Diff}
\DeclareMathOperator{\disc}{disc}
\DeclareMathOperator{\dom}{dom}
%\DeclareMathOperator{\End}{End}
\DeclareMathOperator{\energy}{energy}
\DeclareMathOperator{\erfc}{erfc}
\DeclareMathOperator{\erf}{erf}
\DeclareMathOperator*{\esssup}{ess\,sup}
\DeclareMathOperator{\ev}{ev}
\DeclareMathOperator{\Ext}{Ext}
\DeclareMathOperator{\fst}{fst}
\DeclareMathOperator{\Fit}{Fit}
\DeclareMathOperator{\fix}{fix}
\DeclareMathOperator{\Frac}{Frac}
\DeclareMathOperator{\Gal}{Gal}
\DeclareMathOperator{\gammaD}{gamma}
\DeclareMathOperator{\gr}{gr}
\DeclareMathOperator{\hcf}{hcf}
\DeclareMathOperator{\Hom}{Hom}
\DeclareMathOperator{\id}{id}
\DeclareMathOperator{\Image}{image}
\DeclareMathOperator{\Im}{Im}
\DeclareMathOperator{\Ind}{Ind}
\DeclareMathOperator{\Int}{Int}
\DeclareMathOperator{\Isom}{Isom}
\DeclareMathOperator{\lcm}{lcm}
\DeclareMathOperator{\length}{length}
\DeclareMathOperator{\Lie}{Lie}
\DeclareMathOperator{\like}{like}
\DeclareMathOperator{\Lk}{Lk}
\DeclareMathOperator{\Maps}{Maps}
\DeclareMathOperator{\mse}{mse}
\DeclareMathOperator{\multinomial}{multinomial}
\DeclareMathOperator{\orb}{orb}
\DeclareMathOperator{\ord}{ord}
\DeclareMathOperator{\otp}{otp}
\DeclareMathOperator{\Poisson}{Poisson}
\DeclareMathOperator{\poly}{poly}
\DeclareMathOperator{\rank}{rank}
\DeclareMathOperator{\rel}{rel}
%\DeclareMathOperator{\Rad}{Rad}
\DeclareMathOperator{\Re}{Re}
\DeclareMathOperator*{\res}{res}
\DeclareMathOperator{\Res}{Res}
%\DeclareMathOperator{\Ric}{Ric}
\DeclareMathOperator{\rk}{rk}
\DeclareMathOperator{\Rees}{Rees}
\DeclareMathOperator{\Root}{Root}
\DeclareMathOperator{\sech}{sech}
\DeclareMathOperator{\sgn}{sgn}
\DeclareMathOperator{\snd}{snd}
\DeclareMathOperator{\Spec}{Spec}
\DeclareMathOperator{\spn}{span}
\DeclareMathOperator{\stab}{stab}
\DeclareMathOperator{\St}{St}
%\DeclareMathOperator{\supp}{supp}
\DeclareMathOperator{\Syl}{Syl}
\DeclareMathOperator{\Sym}{Sym}
\DeclareMathOperator{\tr}{tr}
\DeclareMathOperator{\Tr}{Tr}
\DeclareMathOperator{\var}{var}
\DeclareMathOperator{\vol}{vol}
\usetikzlibrary{knots}




\pgfarrowsdeclarecombine{twolatex'}{twolatex'}{latex'}{latex'}{latex'}{latex'}
\tikzset{->/.style = {decoration={markings,
                                  mark=at position 1 with {\arrow[scale=2]{latex'}}},
                      postaction={decorate}}}
\tikzset{<-/.style = {decoration={markings,
                                  mark=at position 0 with {\arrowreversed[scale=2]{latex'}}},
                      postaction={decorate}}}
\tikzset{<->/.style = {decoration={markings,
                                   mark=at position 0 with {\arrowreversed[scale=2]{latex'}},
                                   mark=at position 1 with {\arrow[scale=2]{latex'}}},
                       postaction={decorate}}}
\tikzset{->-/.style = {decoration={markings,
                                   mark=at position #1 with {\arrow[scale=2]{latex'}}},
                       postaction={decorate}}}
\tikzset{-<-/.style = {decoration={markings,
                                   mark=at position #1 with {\arrowreversed[scale=2]{latex'}}},
                       postaction={decorate}}}
\tikzset{->>/.style = {decoration={markings,
                                  mark=at position 1 with {\arrow[scale=2]{latex'}}},
                      postaction={decorate}}}
\tikzset{<<-/.style = {decoration={markings,
                                  mark=at position 0 with {\arrowreversed[scale=2]{twolatex'}}},
                      postaction={decorate}}}
\tikzset{<<->>/.style = {decoration={markings,
                                   mark=at position 0 with {\arrowreversed[scale=2]{twolatex'}},
                                   mark=at position 1 with {\arrow[scale=2]{twolatex'}}},
                       postaction={decorate}}}
\tikzset{->>-/.style = {decoration={markings,
                                   mark=at position #1 with {\arrow[scale=2]{twolatex'}}},
                       postaction={decorate}}}
\tikzset{-<<-/.style = {decoration={markings,
                                   mark=at position #1 with {\arrowreversed[scale=2]{twolatex'}}},
                       postaction={decorate}}}


\tikzset{circ/.style = {fill, circle, inner sep = 0, minimum size = 3}}
\tikzset{scirc/.style = {fill, circle, inner sep = 0, minimum size = 1.5}}
\tikzset{mstate/.style={circle, draw, blue, text=black, minimum width=0.7cm}}

\tikzset{eqpic/.style={baseline={([yshift=-.5ex]current bounding box.center)}}}
\tikzset{commutative diagrams/.cd,cdmap/.style={/tikz/column 1/.append style={anchor=base east},/tikz/column 2/.append style={anchor=base west},row sep=tiny}}


\definecolor{mblue}{rgb}{0.2, 0.3, 0.8}
\definecolor{morange}{rgb}{1, 0.5, 0}
\definecolor{mgreen}{rgb}{0.1, 0.4, 0.2}
\definecolor{mred}{rgb}{0.5, 0, 0}


%\title{ Lecture 4}
\begin{document}\thispagestyle{empty}

\centerline{\Large \bf Lecture 3}

\centerline{\Large \bf Nakahara section 5.4, 5.5}


\section{Cotangent bundles}


Given a vector space $V$, one can take its dual space 
$$
V^*=\{f:V\to\R | f(\alpha_1 v_1+\alpha_2 v_2)=\alpha_1 f(v_1)+\alpha_2 f(v_2)\}
$$
The dual space $V^*$ is also a vector space: $\beta_1 f_1 +\beta_2 f_2 \in V^*$ for $f_1,f_2\in V^*$ and $\beta_1,\beta_2\in \R$.
The dual vector space $T_p^*M$ of the tangent space $T_pM$ is called the \term{cotangent space}. In fact, given $f\in C^\infty(M)$, we can define its differential $df_p$ at $p$
$$
df_p:T_pM\to \R;X_p \mapsto X_p(f)~.
$$
For a local coordinate $(U,\varphi=(x^1,\cdots,x_n))$, we have seen that $(\frac{\partial}{\partial x^1}\Big|_p,\cdots,\frac{\partial}{\partial x^n}\Big|_p)$ is  a basis of $T_pM$. On the other hand, we can take $(dx^1|_p,\cdots,dx^n|_p)$ as a basis of $T^*_pM$ so that $$dx^i|_p(\frac{\partial}{\partial x^j}\Big|_p)=\delta^j_i~.$$
Therefore, in this basis, we can write
$$
df_p=\sum_{i=1}^n\frac{\partial f}{\partial x^i}(p) dx^i|_p
$$
Like the tangent bundle, we can consider a collection of the cotangent spaces
$$
T^*M=\cup_{p\in M} T^*_pM
$$
which has a manifold structure. We call $T^*M$ the \term{cotangent bundle} of $M$. Moreover, the section of the cotangent bundle is called one-form, and we denote the set of one-form by $\Omega^1(M)=\Gamma(T^*M)$.
  For example, if $f$ is a smooth function on $M$, then $d f \in \Omega^1(M)$, which can take a paring with ${}^\forall X \in \mathfrak{X}(M)$
  \[
    d f(X) = X(f)~.
  \]
 


\subsection{push-forward and pull-back}

Let $f: M \to N$  be a smooth map between smooth manifolds $M$ and $N$. It induces a push-forward of tangent vectors
$$
f_*:T_pM\to T_{f(p)}N
$$
which is defined by
$$
f_*(X_p)(g)=X_p(g\circ f)
$$
for $g\in C^\infty(N)$. 
\begin{figure}[h]\centering
\includegraphics{fig_tangent_map}
\end{figure}

On the other hand, it induces pull-back of the cotangent space
$$
f^*: T^*_{f(p)}N\to T^*_pM
$$
which is defined by
$$
\langle f^*\omega_{f(p)}, X_p\rangle=\langle\omega_{f(p)},f_*X_p\rangle
$$
for $\omega_{f(p)}\in T_{f(p)}N$ where $\langle \cdot ,\cdot \rangle$ is a natural paring between the tangent and cotangent space. 






\section{Differential forms}
In general, given a vector space $V$ and its dual vector space $V^*$, one can consider $k$-forms, which are alternating $k$-linear maps $\wedge^k V^*$.
Let us start with a product of 1-forms. If $\alpha_a$ are 1-forms (i.e., $\alpha_a \in V^*$), their wedge products are defined by
$$\alpha_1\wedge\alpha_2\wedge\cdots\wedge\alpha_k(v_1,v_2,\cdots,v_k)=\det(\alpha_a(v_b))~.$$ 
More generally, if $\alpha$ is a $k$-form and $\beta$ is an $\ell$-form,
$$(\alpha\wedge \beta)(v_1, \cdots, v_{k+\ell}) = \frac{1}{(k+\ell)!} \sum_{\sigma\in S_{k+\ell}} \textrm{sign} \sigma 
~  \alpha(v_{\sigma(1)} , \cdots, v_{\sigma(k)} )\beta(v_{\sigma(k+1)} , \cdots, v_{\sigma(k+l)} )~.$$ k!l!
We can choose a basis of the space of $k$-forms, $\wedge^k V^*$, as $e_{i_1}\wedge\cdots\wedge e_{i_k}$.
 Any $k$-form $\alpha$ can be expanded as
$$\alpha= \frac{1}{ k!}\sum_{I} \alpha_{i_1,\cdots,i_k}e_{i_1}\wedge\cdots\wedge e_{i_k}.$$



Therefore, one can consider the $k$-th wedge product $\wedge^k T_p^*M$ of the cotangent space $T^*_pM$ and its bundle $\wedge^k T^*M$.   We write the set of sections as
  \[
    \Omega^k (M) = \Gamma(\Lambda^k T^*M) = \{\text{$k$-forms on $M$}\}.
  \]
  An element of $\Omega^k(M)$ is known as a \term{differential $k$-form}.
  In particular, we have
  \[
    \Omega^0(M) = C^\infty(M).
  \]
In local coordinates $x^1, \cdots, x^n$ on $U$ we can write $\omega \in \Omega^k(M)$ as
\[
  \omega = \sum_{i_1 < \ldots < i_p} \omega_{i_1, \ldots, i_p} d x^{i_1} \wedge \cdots \wedge d x^{i_p}
\]
for some smooth functions $\omega_{i_1, \ldots, i_p}$.


Moreover, there exists a unique linear map called \term{exterior derivative}
  \[
    d : \Omega^k(M) \to \Omega^{k + 1}(M)~,
  \]
  such that
  \begin{enumerate}
    \item On $\Omega^0(M)$ this is as previously defined, i.e.
      \[
        d f (X) = X(f)\text{ for all }X \in \mathfrak{X}(M).
      \]
    \item We have
      \[
        d \circ d = 0: \Omega^k(M) \to \Omega^{k + 2}(M).
      \]
    \item It satisfies the \term{Leibniz rule}
      \[
        d (\omega \wedge \sigma) = d \omega \wedge \sigma + (-1)^k \omega \wedge d \sigma.
      \]
  \end{enumerate}



In term of local coordinates $x^1, \cdots, x^n$, we can define the exterior derivative as
  \[
    d\left(\sum_{i_1 < \ldots < i_p} \omega_{i_1, \ldots, i_p}\;d x^{i_1} \wedge \cdots \wedge d x^{i_p}\right) = \sum d \omega_{i_1, \ldots, i_p}\wedge d x^{i_1} \wedge \cdots \wedge d x^{i_p}~.
  \]
  
  
  
  Let $f: M \to N$  be a smooth map between smooth manifolds $M$ and $N$. The pull-back of differential forms associated to $f$ can be defined as
  \[
    (f^*\omega|_p)(v_1, \cdots, v_k) = \omega|_{f(p)} (df|_p(v_1), \cdots, df|_p(v_k)).
  \]
  for $\omega\in \Omega^k(N)$ and  $v_1, \cdots, v_k \in T_p M$. Note that the pull-back $f^*$ has the following property
  \begin{enumerate}
    \item $f^*:\Omega^k(N) \to \Omega^k(M)$ is a linear map.
    \item $f^*(\omega \wedge \sigma) = f^*\omega \wedge f^*\omega$.
    \item If $g :N\to L$ is a smooth map between two manifolds $N$ and $L$, then $(g \circ f)^* = f^* \circ g^*$.
    \item It commutes with exterior derivative: $df^* = f^* d$.
  \end{enumerate}
  
  \section{Integrals of differential forms}
  
    Let $M$ be $n$-dimensional orientable smooth manifold and $\omega\in \Omega^n(M)$.  We would like to define the integral of $\omega$ over $M$. To this end, we define a partition of unity.

  
  \begin{defi}[Partition of unity]\index{partition of unity}
  Let $\{U_\alpha\}$ be a locally-finite open cover of a manifold $M$. A \term{partition of unity} associated to $\{U_\alpha\}$ is a collection $\chi_\alpha \in C^\infty(M, \R)$ such that
  \begin{enumerate}
    \item $0 \leq \chi_\alpha \leq 1$
    \item $\supp(\chi_\alpha) \subseteq U_\alpha$
    \item $\sum_\alpha \chi_\alpha = 1$.
  \end{enumerate}
\end{defi}

If we use local coordinate $(x^1,\cdots,x^n)$ on a chart $U_\alpha$, we can write
$$
\chi_\alpha \omega=f_\alpha(x) x^1\wedge\cdots\wedge x^n~.
$$
Therefore, we define its integral
$$
\int_M \chi_\alpha \omega=\int \cdots \int f_\alpha(x) x^1\cdots x^n~.
$$
Then, we can define
$$
\int_M\omega=\sum_\alpha\int_M \chi_\alpha \omega~.
$$
One can show that this is independent of choice of a locally-finite open covering $\{U_\alpha\}$  and a partition of unity on $\{U_\alpha\}$ .
  
  \begin{thm}[Stokes' theorem]\index{Stokes' theorem}
  Let $M$ be an oriented manifold with boundary of dimension $n$. Then if $\omega \in \Omega^{n - 1}(M)$ has compact support, then
  \[
    \int_M d \omega = \int_{\partial M}\omega.
  \]
  In particular, if $M$ has no boundary, then
  \[
    \int_M d \omega = 0~.
  \]
\end{thm}
  
\end{document}
