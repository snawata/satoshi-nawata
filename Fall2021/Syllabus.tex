\documentclass[12pt,a4paper]{article}
\usepackage{hyperref} % Use the Charter font for the document text
%\usepackage[UTF8]{ctex}
\usepackage{fullpage}
\usepackage{color}

\usepackage{graphicx}


\begin{document}\thispagestyle{empty}

\centerline{\Large \bf Syllabus}

\begin{description}
\item{\bf Course name:} PHYS130021.01 and PHYS630046: Introduction to 2d conformal field theory
\item{\bf Instructor:} Satoshi Nawata  Physics S422, Jiangwan campus,  \href{mailto:snawata@fudan.edu.cn}{snawata@fudan.edu.cn} \\ Zeng Xiangdong (TA) \href{mailto:18110190010@fudan.edu.cn}{18110190010@fudan.edu.cn}
\item{\bf Hours:} Thursday 9:55-12:30
\item{\bf Place:} HGX105
\item{\bf Office hour:}  Any time, but ask me via wechat beforehand.
\item{\bf Prerequisites:}

Quantum Mechanics, Electrodynamics, Statistical mechanics

It would be desirable (but not necessary) if you have a basic knowledge of quantum field theory, and Lie group and algebra.
\item{\bf About the course:}



Conformal field theories (CFTs) play distinctive role in quantum field theories, string theory, integrable systems, and condensed matter physics. Moreover, 2d CFTs are particularly rich because of infinite-dimensional symmetries. Thanks to infinite-dimensional symmetries, physical observables in many models become exactly solvable. Consequently, they provide an indispensable opportunity to learn tools and intuitions necessary to understand more general quantum field theories. Moreover, 2d CFTs have shed new light even in mathematics, ranging from infinite-dimensional Lie algebras to topology, algebraic geometry and number theory.

In this course, we introduce to basics of 2d CFTs.  First, we will learn Virasoro algebra, OPEs, free fields, 2d renormalization group, and correlation and partition functions. Next, we investigate some exactly solvable models like minimal (Ising) models, (1+1)-d quantum many-body systems, Wess-Zumino-Witten models. If time permits, we also learn some applications to critical phenomena in 2d systems or string theory.



Any students are very welcome to audit this course.




\item{\bf Main content:}
\begin{itemize}
\item Basics of 2d CFTs
\item Free bosons and fermions
\item Minimal models
\item Renormalization group flow and c-theorem
\item Wess-Zumino-Novikov-Witten Models
\item Advanced topics
\end{itemize}



\item{\bf Main references:} I will upload my own lecture notes, but I recommend you to refer to the following classics.

P.~H. Ginsparg.
\newblock {\em Applied Conformal Field Theory}.
\newblock  \href{http://arxiv.org/abs/hep-th/9108028}{{\tt hep-th/9108028}}



\item{\bf Supplementary references:} There are too many references on 2d CFTs. In the first lecture, I will list some of them classified by viewpoints. However, the following two books are standard references.


P.~Francesco, P.~Mathieu, and D.~S{\'e}n{\'e}chal.
\newblock {\em Conformal field theory}.
\newblock  Graduate Texts in Contemporary Physics, Springer Science \& Business Media, 1996.


R.~Blumenhagen and E.~Plauschinn.
\newblock {\em Introduction to conformal field theory}.
\newblock {Lect. Notes Phys.}, 779:1--256, 2009.

M. Henkel, {\em Conformal Invariance and Critical Phenomena}. Texts and Monographs in Physics, Springer Science \& Business Media,  1999.

\item{\bf Grading:} Grade will be determined based on homework sets (80\%) given every week and the final test (20\%).


\end{description}



\end{document}
