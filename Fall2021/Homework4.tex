\documentclass[12pt,a4paper]{article}
%\usepackage{hyperref} % Use the Charter font for the document text
%\usepackage[UTF8]{ctex}



\usepackage{macros}

\begin{document}\thispagestyle{empty}

\centerline{\Large \bf Homework 4: Due at class on Oct 14}


\section{Derivations}
\subsection{Commutation relations in free boson}
Derive the commutation relations (4.7) and (4.8) from (4.6).
\subsection{Hamiltonian in free boson}
Derive the Hamiltonian (4.10) from (4.9).
\subsection{Action of free fermion}
Derive (4.33) from (4.30).
\subsection{$TT$ OPE in free fermion}
Compute $TT$ OPE (4.45) in the free fermion.

% %
% \section{$bc$ ghost system}
%
%
% The action of the $bc$ ghost system in  the Euclidian signature is given by
% $$ S_{\rm gh} = \frac{1}{2\pi}\int d^2\sigma \sqrt{g}\ b^{ab}\nabla_a c_b~,$$
% The energy-momentum tensor for the $bc$ ghosts can be calculated by the definition
% $$
% T_{ab}=-\frac{4\pi}{\sqrt{g}}\frac{\d S}{\d g^{ab}}~,
% $$
% yielding
% $$ T(z) =- 2:b(z)\partial c(z): + :c(z)\partial b(z):\label{ST}$$
% in the conformally flat metric. (If you derive this explicit form of the energy-momentum tensor, you will obtain a bonus point.) Suppose that the $bc$ OPE is given by
% \begin{align} b(z)\,c(w) &= \frac{1}{z-w} + \ldots\nonumber\\ c(w)\,b(z) &= \frac{1}{w-z} + \ldots\nonumber\end{align}
% where $b$, $c$ fields obey the Fermi statistics so that  the second equation follows from the first equation.
%
% Compute $T(z)b(w)$, $T(z)c(w)$, and $T(z)T(w)$ OPEs, and find the conformal dimensions of $b$ and $c$ as well as the central charge of the $bc$ ghost system.

%
%
% \section{Schwarzian derivatives}
% In the lecture note (3.87), we have learned that,  under the conformal transformation
% $z \rightarrow w(z)$, the energy-momentum tensor transforms
% $$
%   T(w) = \bigg(\frac{dw}{dz}\bigg)^{-2}
%   \big[
%        T(z) -\frac{c}{12}\{w;z\}
%   \big]\, ,
% $$
% where $\{w;z\}$ is the additional term
% called the {Schwarzian derivative}:
% $$
%   \{w;z\} = \frac{w^{'''}}{w'}
%   - \frac{3}{2}
%   \bigg(
%   \frac{w^{''}}{w'}
%   \bigg)^2
%   \, .
% $$
% \subsection{}
% Derive the infinitesimal transformation (3.86) from the finite version (3.87).
%
% \subsection{Under $\SL(2,\bC)$}
% For an element of $\SL(2,\bC)$
% $$\begin{pmatrix}a&b\\c&d\end{pmatrix} \in \SL(2,\bC)$$
% show that
% $$
% \{w;z\}=0\quad \textrm{for} \ w=\frac{az+b}{cz+d}~,
% $$
% and
% $$
% \left\{\frac{aw+b}{cw+d};z\right\}=\{w;z\}~.
% $$
% Show that the energy-momentum tensor $T(z)$ is a quasi-primary but not primary.
%
% %
%
% \subsection{Free boson (Bonus problem: 2pt)}
% The energy-momentum tensor of the free boson is
% $$
%   T(z) = - \frac12 :\partial_z\varphi \partial_z\varphi:\,
% $$
% where the normal ordering can be defined as
% $$
% :\partial_z\varphi \partial_z\varphi: =\lim_{w\to z} \left( \partial_z\varphi(z) \partial_{w}\varphi(w)+\frac1{(z-w)^2}\right)~.
% $$
% Since $\partial_z\varphi$ is the primary field of conformal dimension one, it transforms as
% $$
%  \partial_z\varphi(z) \partial_{w}\varphi(w)=f'(z)f'(w)\partial_{\wt z}\varphi(\wt z) \partial_{\wt w}\varphi(\wt w)
% $$
% under the conformal transformation $z\to \wt z= f(z)$. Hence we have
% $$
% :\partial_z\varphi (z)\partial_w\varphi(w): -\frac1{(z-w)^2}=f'(z)f'(w)\left[:\partial_{\wt z}\varphi(\wt z) \partial_{\wt w}\varphi(\wt w): -\frac1{(\wt z-\wt w)^2}\right]
% $$
% Taking limit $z\to w$, show that
% $$
% \lim_{z\to w}\left[\frac{f'(z)f'(w)}{(f(z)-f(w))^2} -\frac1{(z-w)^2} \right]=\frac16\{f(w);w\}~.
% $$
% What is the central charge of the free boson?



\end{document}
