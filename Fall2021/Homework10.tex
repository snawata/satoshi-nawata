\documentclass[12pt,a4paper]{article}
%\usepackage{hyperref} % Use the Charter font for the document text
%\usepackage[UTF8]{ctex}


\usepackage{macros}

\begin{document}\thispagestyle{empty}

\centerline{\Large \bf Homework 10: Due at class on Nov 28}



\section{$\SU(2)$ current algebra with level 1}
\subsection{}
Show the OPE in (7.5) of the lecture note.


\subsection{}
 Let us define
$$
V_{\uparrow}=:e^{\frac{i\varphi ( z )}{\sqrt{2}}}:~,\qquad V_{\downarrow}=:e^{-\frac{i\varphi ( z )}{\sqrt{2}}}:~.
$$
Show the following OPEs
\bea
J^+(z)V_{\downarrow}(0)&\sim\frac{V_{\uparrow}(0)}{z}\cr
J^-(z)V_{\uparrow}(0)&\sim\frac{V_{\downarrow}(0)}{z}\cr
J(z)V_{\uparrow,\downarrow}(0)&\sim\frac{\pm \frac12 V_{\uparrow,\downarrow}(0)}{z}
\eea
Can you find the states corresponding to the operators $V_{\uparrow}$ and $V_{\downarrow}$ via the state-operator correspondence?






\subsection{}
We have seen in (4.98) of the lecture note that the partition function
\begin{equation}
Z_{R}(\tau,\bar{\tau}) = \frac{1}{|\eta(\tau)|^2}\sum_{m,n\in \bZ} q^{\frac12 (\frac{m}{R}+\frac{Rn}{2})^2} \bar{q}^{\frac12 (\frac{m}{R}-\frac{Rn}{2})^2}. \label{eq:bosonZ}
\end{equation}
of the compactified boson of radius $R$. Show that at the self-dual radius $R=\sqrt{2}$ it can be written as
$$
Z_{R=\sqrt{2}}=|\chi_0(q)|^2+|\chi_{\frac12}(q)|^2\,, \qquad  \textrm{where} \quad \chi_0=\frac 1 {\eta(\tau)} \sum_{n\in \bZ}  q^{n^2} ~,
\quad \chi_{\frac12}=\frac 1 {\eta(\tau)} \sum_{n\in \bZ} q^{(n + 1 / 2)^2}
$$
In the lecture, we have seen that the highest weight states of the $\SU(2)$ current algebra with level $k=1$ are $|j=0,\frac12\rangle$. In fact, the $\chi_j$ are the spin-$j$ characters of the $\SU(2)$ affine Lie algebra of level $k=1$.




\section{$\SU(2)$ current algebra with level $k$}
The OPE of $\SU(2)$ current with level $k$ is given by
$$
J ^ {a} ( z ) J ^ {b} ( w ) \sim \frac {\frac {k} {2} \delta_{a b}} {( z - w ) ^ {2}} + \frac {i \epsilon ^ {a b c} J ^ {c} ( w )} {z - w}~.
$$
Suppose that the energy momentum tensor is defined as
$$
T(z)=\frac{1}{k+2}\sum_{a=1}^3:J^aJ^a(z):
$$
By computing $TJ^a$ OPE (use the generalized Wick theorem in 6.B of the yellow book), find the conformal dimension of the current $J^a$ and derive the commutation relations
$[L_n,J^a_m] $.
 Compute the central charge via the $TT$ OPE.

\section{Wakimoto construction (Bonus problem)}
\subsection{}
We define the bosonic field $a(z) =i\sqrt{\frac{\kappa}{2}}\partial \varphi(z)$ so that the OPE is
$$a ( z ) a ( w )  \sim \frac { \frac \kappa2 } { ( z - w ) ^ { 2 } } $$
Let us introduce the bosonic fields $\beta(z),\gamma(z)$ where they satisfy the OPE
$$
\beta ( z ) \gamma ( w ) \sim \frac { 1 } { z - w }
$$
and the rest of OPEs is trivial. Furthermore, let us define operators
$$
\begin{aligned} E ( z ) & = \beta ( z ) \\ H ( z ) & = -  : \gamma ( z ) \beta ( z ) : + a ( z ) \\ F ( z ) & = - : \gamma ( z ) ^ { 2 } \beta ( z ) : +2 \gamma ( z ) a ( z ) + k \gamma ^ { \prime } ( z ) \end{aligned}~,
$$
where $\kappa=k+2$.
Compute all the possible OPEs of $E,F,H$ and find the relation to current algebra.


\subsection{}
If the energy-momentum tensor is defined as
$$
T ( z ) = : \frac { 1 } {  \kappa } \left\{ a ( z ) ^ { 2 } -  a ^ { \prime } ( z ) \right\} + \beta ( z ) \gamma ^ { \prime } ( z ) :~,
$$
compute the central charge. Derive the conformal dimensions of $E,F,H$. Find the conformal dimension of the vertex operator $V_{\alpha}(z)=:e^{i\alpha\sqrt{\frac{2}{\kappa}}\varphi ( z ) }:$.



\subsection{}
Defining the vertex operator
$$
V_{\alpha=-1}( z ) = e ^ { - i\sqrt{\frac{2}{\kappa}}\varphi ( z ) } \beta ( z )~,
$$
compute its OPEs with $E,F,H$ and $T$. Show that it behaves as a screening current.












\end{document}
