\documentclass[12pt,a4paper]{article}
%\usepackage{hyperref} % Use the Charter font for the document text
%\usepackage[UTF8]{ctex}


\usepackage{macros}
\begin{document}\thispagestyle{empty}

\centerline{\Large \bf Homework 6: Due at class on Oct 31}
%
% \section{$\zeta$-function regularization}
% To compute Casimir energy of free fermion, the important part is
% $$
% L_0:=\frac12\sum_{n\in \bZ+\nu}n:b_{-n}b_{n}:
% $$
% where the cost of normal ordering provides
% \bea\label{Casimir}
% -\tfrac12\sum_{n=0}^\infty n& \qquad \textrm{R-sector}\cr
% -\tfrac12\sum_{n=0}^\infty (n+\tfrac12) & \qquad \textrm{NS-sector}~.
% \eea
% Though these summations naively provide negative infinity, the proper regularization can avoid infinity. To do that, let us regularize the summations as
% $$
% \sum _ { n = 0 } ^ { \infty } ( n + \theta ) \Longrightarrow \sum _ { n = 0 } ^ { \infty } ( n + \theta ) - \int _ { 0 } ^ { \infty } k d k~.
% $$
% By introducing the convergence parameter $\e$
% \bea\nonumber
% &\sum _ { n = 0 } ^ { \infty } ( n + \theta ) e ^ { - \epsilon ( n + \theta ) } - \int _ { 0 } ^ { \infty } d k k e ^ { - \epsilon k }\cr
% =& - \frac { d } { d \epsilon } \sum _ { n = 0 } ^ { \infty } e ^ { - e ( n + \theta ) } - ( - 1 ) \frac { d } { d \epsilon } \int _ { 0 } ^ { \infty } d k e ^ { - \epsilon k }~,
% \eea
% provide the finite numbers to the summations in \eqref{Casimir} in the limit of $\e\to +0$, and compare (4.68) in the lecture note.



\section{Derivation}

\subsection{}
Show (5.26) and find the fusion rule $\phi_{1,2} \times \phi_{r, s}=\left[\phi_{r, s-1}\right]+\left[\phi_{r, s+1}\right]$.

\subsection{}
Show (5.29).
\subsection{}
Following (5.34) of the lecture note, find the explicit fusion rule of the product
$$
\phi_{1,3}\times \phi_{2,2}
$$



\section{Correlation functions of vertex operators}
A vertex operator $V_k(z)= :e^{ik\varphi}$ defined by using the normal-ordering can be understood as
$$
V_k(z):=V_k^+(z)V_k^-(z)
$$
where $V_k^+(z)$ (resp. $V_k^-(z)$) consists of creation (resp. annihilation) operators
\bea\nonumber
V_k^+(z)&:=\exp\left( k \sum_{n=0}^\infty t_n(z)\right)~, \quad  t_0=i\varphi_0 \qquad t_{n>0} =\frac{a_{-n}}{n}z^n~,\cr
V_k^-(z)&:=\exp\left( k \sum_{n=0}^\infty u_n(z)\right)~, \quad  u_0=a_0\log z \qquad u_{n>0} =-\frac{a_{n}}{n}z^{-n}~.
\eea
Show that their OPE is
$$
V_{k_1}(z)V_{k_2}(w)\sim(z-w)^{k_1k_2}:V_{k_1}(z)V_{k_2}(w):~,
$$
where
\bea\nonumber
V_{k_1}(z)V_{k_2}(w)&=V_{k_1}^+(z)V_{k_1}^-(z)V_{k_2}^+(w)V_{k_2}^-(w) \cr
:V_{k_1}(z)V_{k_2}(w):&=V_{k_1}^+(z)V_{k_2}^+(w)V_{k_1}^-(z)V_{k_2}^-(w)~.
\eea
In addition, show that the two point function of vertex operators is
$$
\langle V_{k_1}(z)V_{k_2}(w)\rangle=\frac{\delta_{k_1,-k_2}}{(z-w)^{k_1^2}}~.
$$
In a similar fashion, show that multi-point correlation function of vertex operators is given by
$$
\langle V_{k_1}(z_1)V_{k_2}(z_2)\cdots V_{k_n}(z_n)\rangle=\delta_{\sum k_i,0}\prod_{1\le i<j\le n}(z_i-z_j)^{k_ik_j}~.
$$

\section{Linear dilaton CFT}
In the lecture, we have consider the free boson. Now let us consider another important theory whose action is
$$
S=\frac1{8\pi}\int d^2z \sqrt{g} \nabla_\a \varphi \nabla^\a \varphi+\frac{Q}{4\pi} \int d^2z\sqrt{g} R\varphi
$$
where $g$ is the metric of the two-dimensional world-sheet and $R$ is the Ricci scalar of $g$. This model is called linear dilaton CFT and $Q$ is called background charge.  Using the definition
$$
T_{\a\b}=-\frac{4\pi}{\sqrt{g}}\frac{\delta S}{\delta g^{\a\b}}~,
$$
one can show that  the energy-momentum tensor in the flat metric $g_{\a\b}=\d_{\a\b}$ is
$$T ( z ) = - \frac { 1 } { 2} : \partial \varphi ( z ) \partial \varphi( z ) : +Q \partial ^ { 2 } \varphi( z )~.$$
(If you derive this, I will give you an extra point.)
\subsection{}
Compute $TT$ OPE and find the central charge of the linear dilaton CFT.
\subsection{}
Show that $\partial \varphi$ is not primary unless $Q = 0$, but is quasi-primary with conformal dimension
$h = 1$. Show that the vertex operator $V_k(z)=: e^{ik\varphi} :$ is primary and compute its conformal dimension. In addition, compute the OPE
$\partial \varphi(z) V_k(w)$.
\subsection{}
Let us define the state $|k\rangle=\lim_{z\to 0}V_k(z)|0\rangle$ that corresponds to the vertex operator $V_k(z)$. Show that
$$
a_0|k\rangle=k|k\rangle~, \qquad L_0|k\rangle=\frac{k(k+2iQ)}2|k\rangle~, \qquad L_n|k\rangle=0 \quad\textrm{for} \ n>0.
$$
\subsection{}
When ${\sqrt{2}\a_\pm}=-iQ\pm\sqrt{2-Q^2}$, find that
$$
T(z)V_{\sqrt{2}\a_\pm}(w)\sim\frac{\partial}{\partial w} \frac{V_{\sqrt{2}\a_\pm}(w)}{z-w}
$$
so that
$$
\Big[T(z),\oint \frac{dw}{2\pi i} V_{\sqrt{2}\a_\pm}(w)\Big]=0~.
$$
In fact, $\mathbf{S}_\pm=\oint \frac{dw}{2\pi i} V_{\sqrt{2}\a_\pm}(w)$ is called the screening charge.




\section{Feigin-Fuchs representation}
We shall derive the Kac determinant (5.15) of the lecture note by following Feigin-Fuchs.
\subsection{Kac determinant}
Show that the parameters in the Kac determinant (5.15)  obey
$$
\a_++\a_-=2\sqrt{-h_0}~,\qquad \a_+\a_-=-1~.
$$

\subsection{Screening charge}
 Let us consider the linear dilaton CFT in the previous problem where the background charge is set to
$$
Q=\sqrt{2h_0}=i\sqrt{\frac{1-c}{12}}~.
$$
Recalling that a vertex operator $V_k(z)=:e^{ik\varphi(z)}:$ in the linear dilaton CFT has conformal dimension $h=k(k+2iQ)/2$, derive that the conformal dimension of the vertex operator $V_{\sqrt{2 }\a_\pm}(w)$ is equal to one. Hence,  $\mathbf{S}_\pm=\oint \frac{dw}{2\pi i} V_{\sqrt{2 }\a_\pm}(w)$ are the screening charge. Show that the state
$$
\mathbf{S}_\pm^r|k\rangle \qquad \textrm{for} \ r\in \bZ_{>0}
$$
is a singular vector where $|k\rangle=\lim_{z\to 0}V_k(z)|0\rangle$ if it is not zero.



\subsection{Integral representation}\label{singular}
It admits the integral representation as
\bea\nonumber
\mathbf{S}_+^r|k\rangle &=\frac{1}{(2\pi i)^r} \int d z _ { 1 } \cdots d z _ { r } V _ { \sqrt{2}\a_+  } \left( z _ { 1 } \right) \cdots V _ { \sqrt{2}\a_+  } \left( z _ { r } \right) | k \rangle \cr
&=\frac{1}{(2\pi i)^r} \int d z _ { 1 } \cdots d z _ { r } \prod _ { 1 \leq i < j \leq r } \left( z _ { i } - z _ { j } \right) ^ {2\a_+^2  } : V _ { \sqrt{2}\a_+  } \left( z _ { 1 } \right) \cdots V _ { \sqrt{2}\a_+  } \left( z _ { r } \right) : | k \rangle~.
\eea
Let us recall that
$$
a_0|k\rangle =k|k\rangle~,\quad e^{ip\varphi_0}|k\rangle=|k+p\rangle~,\quad a_n|k\rangle=0~, \ (n>0)~.
$$
In addition, the normal-ordering can be understood as
$$
: V _ { \sqrt{2}\a_+  } \left( z _ { 1 } \right) \cdots V _ { \sqrt{2}\a_+  } \left( z _ { r } \right) : = V^+ _ { \sqrt{2}\a_+  } \left( z _ { 1 } \right) \cdots V^+ _ { \sqrt{2}\a_+  } \left( z _ { r } \right)  V^- _ { \sqrt{2}\a_+  } \left( z _ { 1 } \right) \cdots V^- _ { \sqrt{2}\a_+  } \left( z _ { r } \right) ~.
$$
Here we write the creation operator $V_k^+(z)$
\bea\nonumber
V_{\sqrt{2}\a_+  }^+(z)&=\exp\left( \sqrt{2}\a_+   \sum_{n=0}^\infty t_n(z)\right)=\left(  \sum_{n=0}^\infty p_mz^m\right)e^{i\sqrt{2}\a_+ \varphi_0}~.\cr
\eea
Then, show that the integral expression can be written as
$$
\mathbf{S}_+^r|k\rangle =\frac{1}{(2\pi i)^r} \int d z _ { 1 } \cdots d z _ { r } \prod _ { 1 \leq i < j \leq r } \left( z _ { i } - z _ { j } \right) ^ { 2\a_+^2 } \sum _ { m _ { i } \geq 0 } \prod _ { i = 1 } ^ { r } p _ { m _ { i } } z _ { i } ^ { m _ { i } + k  \sqrt{2}\a_+   } | k + r   \sqrt{2}\a_+   \rangle~.
$$

In order for this state to be non-zero, the integral should be invariant under the scaling $z_i\to \lambda z_i$, which requires
$$
r + \frac { r ( r - 1 ) } { 2 } 2\a_+^2 + \sum _ { i = 1 } ^ { r } m _ { i } + r k  \sqrt{2}\a_+ = 0~.
$$
If we assume that the level $ \sum _ { i = 1 } ^ { r } m _ { i } $ of the singular vector $\mathbf{S}_+^r|k\rangle $ is equal to $rs$, show that $k$ has to be
\be\label{momenta}k = \frac { 1 - r } { 2 } \sqrt{2 }\a_+ + \frac { 1 + s } { 2 }\sqrt{2 }\a_-~.\ee
This implies that $\mathbf{S}_+^r|k\rangle $ is a non-zero singular vector at level $rs$ if $k$ is subject to \eqref{momenta}. Compute the conformal dimension of $\mathbf{S}_+^r|k\rangle$ and show that it is equal to $h_{r,s}$ in the Kac determinant (5.15) of the lecture note.




\end{document}
