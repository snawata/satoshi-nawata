\documentclass[12pt,a4paper]{article}
%\usepackage{hyperref} % Use the Charter font for the document text
%\usepackage[UTF8]{ctex}

\usepackage{macros}

\begin{document}\thispagestyle{empty}

\centerline{\Large \bf Homework 2: Due at class on Sep 30}



% \section{ $\SL(2,\bC)$}
% \subsection{relation to Lorentz group}
% Let us define
% $$
% \sigma _{ \mu } \equiv ( \mathbf { 1 } , \vec { \sigma } )
% $$
% where $\sigma_i$ are the Pauli matrices. Compute that $\det X$ where
% $$
% X:=x ^ { \mu } \sigma _ { \mu } = \left( \begin{array} { c c } { t + z } & { x - i y } \\ { x + i y } & { t - z } \end{array} \right)~.
% $$
% Show that $\SL(2,\bC)/\bZ_2$ is isomorphic to the Lorentz group with $\Lambda^0_{~0}\ge1$, namely $\SL(2,\bC)/\bZ_2\cong \SO^+(1,3)$.

\section{}
\begin{itemize}
  \item Derive (3.13) from (3.11) in the lecture note.
  \item Write $l_0, \bar l_0$ in terms of polar coordinate as in (2.34).
\end{itemize}


\section{M\"obius transformation}

Let us consider the Riemann sphere $S^2  = \bC \cup \{\infty\}$. The action of $\SL(2, \bC)$ defined by
$$z  \mapsto w = \frac{az + b}{ cz + d}~,\qquad \begin{pmatrix}a&b\\c&d\end{pmatrix}\in \SL(2, \bC)~,$$
maps the Riemann sphere onto itself. These transformations are called fractional linear transformations.

\vspace{.3cm}
\noindent $\bullet$ Given three points $z_1,z_2,z_3$, find a fractional linear transformation which maps the points
to $0, 1, \infty$.

\vspace{.3cm}
\noindent $\bullet$ Given four points $z_1,z_2,z_3,z_4$, their \textbf{cross ratio} is defined by
$$
[z_1,z_2,z_3,z_4]= {\frac  {(z_{1}-z_{3})(z_{2}-z_{4})}{(z_{2}-z_{3})(z_{1}-z_{4})}}~.
$$
Show that the cross ratio is preserved by any fractional linear transformation
$$
[z_1,z_2,z_3,z_4]=[w_1,w_2,w_3,w_4] ~.
$$



\section{Energy-momentum tensor}
Let us consider the free boson on a $d$-dimensional flat space
$$
  S = \frac{1}{2}\int d^d x ~
     \partial_\mu \varphi
     \partial^\mu \varphi
~ .
$$
Derive the energy-momentum tensor by using the definition
$$
  T^{\mu\nu}=
  -\eta^{\mu\nu}\mathcal{L}
  +\frac{\partial\mathcal{L}}{\partial(\partial_\mu\varphi)}
  \partial^\nu \varphi \, .
$$
Furthermore, let us consider the action of the free boson on a curved space-time where $g$ is the metric on $M$
$$
  S = \frac{1}{2}\int_M d^d x \sqrt{g} ~g^{\mu\nu}
     \partial_\mu \varphi
     \partial_\nu \varphi
~ .
$$
Find the explicit form of the following tensor
$$
\wt T^{\mu\nu}=-\frac{2}{\sqrt{g}}\frac{\d S}{\d g_{\mu\nu}}~.
$$




\section{Three-point function for primary fields}
Derive the form of the three-point function for quasi-primary fields
$$
  \langle{\phi_1(z_1)\phi_2(z_2)\phi_3(z_3)}\rangle
  =\frac{C_{123}}{z_{12}^{h_1+h_2-h_3}
  z_{23}^{h_2 + h_3 - h_1} z_{13}^{h_3+ h_1- h_2}}
  \, .
$$
\end{document}
