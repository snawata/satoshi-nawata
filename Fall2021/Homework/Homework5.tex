\documentclass[12pt,a4paper]{article}
%\usepackage{hyperref} % Use the Charter font for the document text
%\usepackage[UTF8]{ctex}

\usepackage{macros}

\begin{document}\thispagestyle{empty}

\centerline{\Large \bf Homework 5: Due at class on Oct 21}

\section{Modular invariant}
Show that (4.85) is modular $S$-invariant by using the Poisson resummation formula (4.86).

\section{Modular transformation}
The modular group is defined by
$$\textrm{SL} ( 2 , \mathbb { Z } ) = \left\{ \left( \begin{array} { l l } { a } & { b } \\ { c } & { d } \end{array} \right) \bigg|\quad a , b , c , d \in \mathbb { Z } ~, \quad a d - b c = 1 \right\}~.$$
Show that
$$ \operatorname { Im } ( \gamma\cdot \tau ) = \frac { \operatorname { Im } \tau } { | c \tau + d | ^ { 2 } } ~,$$
where
$$
\gamma:\tau\mapsto \gamma\cdot \tau =\frac{a\tau +b}{c\tau +d}~.
$$
Therefore, if $ \operatorname { Im } \tau>0$, then $ \operatorname { Im } ( \gamma\cdot \tau )>0$ so that the upper half-plane
$$ \mathbb {H} =\{x+iy\mid y>0;x,y\in \mathbb {R} \}~$$
receives the action of the modular group. Express elements of modular group that maps the gray region to blue, red and green region in terms of modular $S$ and $T$ transformations
$$
T:\tau\to \tau+1~, \qquad S:\tau\to -1/\tau~.
$$

\begin{figure}[h]\centering
\includegraphics[width=12cm]{modspacetor}
\end{figure}

\section{Boson-Fermion correspondence}
We have learnt that the OPE of the free boson is
$$\varphi(z)\varphi(0) \sim - \ln z~.$$
Let us also consider two Majorana-Weyl fermions $\psi^1, \psi^2$ with OPE
$$\psi^i(z)\psi^j(0) \sim \frac{\d^{ij}}{z}~.$$
We can define the complex fermion $$\psi(z) = 2^{-1/2} (\psi^1(z) +i\psi^2(z))\,,\qquad \overline\psi(z) = 2^{-1/2} (\psi^1(z) -i\psi^2(z))~.$$
Show the equivalence of operators in boson and fermion
$$:e^{i\varphi} :\cong\psi\,,\quad :e^{-i\varphi} :\cong\overline \psi\,,\quad i\partial \varphi \cong :\psi\overline \psi:\,,\quad T_\varphi \cong T_\psi~,$$
by calculating the OPEs of operators in both theories and comparing. Note that the energy-momentum tensor of the complex fermion is
$$
T_\psi:=-\frac12: \psi\partial \overline \psi: -\frac12 :\overline \psi\partial  \psi :~.
$$



\end{document}
