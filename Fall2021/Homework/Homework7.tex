\documentclass[12pt,a4paper]{article}
%\usepackage{hyperref} % Use the Charter font for the document text
%\usepackage[UTF8]{ctex}


\usepackage{macros}

\begin{document}\thispagestyle{empty}

\centerline{\Large \bf Homework 7: Due on Nov 7}



\section{Null states at level 3}
\subsection{}
Show that the linear combination that gives rise to null-vectors at level $N = 3$ is
given by
$$|\chi_I\rangle=\left[ (h_I + 1 )(h_I+2) L _ { - 3 } -  2 (h_I + 1 ) L _ { - 1 } L _ { - 2 } + L _ { - 1 } ^ { 3 } \right]|\phi_I\rangle$$
where $I=(1,3)$ or (3,1).
\subsection{}
Determine the differential equation satisfied by the correlators of the primary
fields $\phi_{1,3}$ and $\phi_{3,1}$.

\section{Minimal models}
Consider the minimal models $\cM_{2,2n+1}$ ($n = 1, 2, \ldots$). Compute the central charge, and determine the fusion rules of these models.

%\section{Correlation functions}
%Show that the correlation function of multiple chiral fermions is expressed as
%$$
%\left\langle \psi \left( z _ { 1 } \right) \cdots \psi \left( z _ { 2 n } \right) \right\rangle = \operatorname { Pf } \left( \frac { 1 } { z _ { i } - z _ { j } } \right) _ { 1 \leq i , j \leq 2 n }~,
%$$
%where the Pfaffian is defined as
%$$
%\operatorname { Pf } ( A ) = \frac { 1 } { n ! 2 ^ { n } } \sum _ { \sigma \in S _ { 2 n } } ( - 1 ) ^ { | \sigma | } \prod _ { i = 1 } ^ { n } A _ { \sigma } ( 2 i - 1 ) , \sigma ( 2 i )~.
%$$
%Using the fact that the energy density operator $\varepsilon(z,\overline{z})=i:\psi(z)\overline\psi(\overline z):$ in the Ising model, show that
%$$
%\left\langle \varepsilon \left( z _ { 1 } , \overline { z } _ { 1 } \right) \cdots \varepsilon \left( z _ { 2 n } , \overline { z } _ { 2 n } \right) \right\rangle^2=\left|\operatorname { det } \left[ \frac { 1 } { z _ { i } - z _ { j } } \right]\right|^2~.
%$$





\section{Star-triangle relation}
\begin{figure}[h]\centering
\includegraphics{star-triangle}
\caption{Hexagonal and triangular lattices that are dual each other.}\label{fig:star-triangle}
\end{figure}
Let us consider the Ising model in the Hexagon lattice (white dotes) and the triangle lattice  (black dotes) in Figure \ref{fig:star-triangle}. In fact, as Figure \ref{fig:star-triangle} illustrates, the triangular lattice of $N$ sites is the dual of the hexagonal lattice with $2N$ sites. By defining the coupling constant $K$ and $L$ between the nearest spin in the hexagon and triangle lattice, the partition functions are written as
\bea
Z ^ { \mathrm { H } } ( \mathcal { L } ) &= \sum _ { \{ \sigma \} } \exp \left[ \mathcal { L }  \sum \sigma _ { l } \sigma _ { i } \right]~,\cr
Z ^ { \mathrm { T } } ( \mathcal { K } ) &= \sum _ { \{ \sigma \} } \exp \left[ \mathcal { K } \sum \sigma _ { l } \sigma _ { i } \right]~,
\eea
where $\cK =- K/k_BT$ and $\cL = -L/k_BT$. Let us show the equivalence of these partition functions
$$
Z ^ { \mathrm { H } } ( \mathcal { L } )=R^N Z ^ { \mathrm { T } } ( \mathcal { K } ) ~,
$$
which is called the \textbf{star-triangle identity}
\begin{figure}[h]\centering
\includegraphics[width=10cm]{bipartite}
\caption{Bipartition of the hexagonal lattice}\label{fig:bipartite}
\end{figure}



First let us split  sites of the hexagonal lattice into two classes as in Figure \ref{fig:bipartite}. Then, we introduce the Boltzmann weight
$$
W \left( \sigma _ { b } ; \sigma _ { i } , \sigma _ { j } , \sigma _ { k } \right) = \exp \left[ \mathcal { L }\sigma _ { b } (  \sigma _ { i } +\sigma _ { j } +  \sigma _ { k }) \right]
$$
where $\sigma_b$ is a spin of a black site and $(\sigma_i,\sigma_j,\sigma_k)$ are the spins at white sites next to $\sigma_b$.
Subsequently, the partition function is
$$
Z ^ { \mathrm { H } } ( \mathcal { L } ) = \sum _ { \sigma _ { a }:\mathrm{white} }~ \prod _ { i , j , k } w^{\mathrm{H}}  \left( \sigma _ { i } , \sigma _ { j } , \sigma _ { k } \right)
$$
where
$$
w^{\mathrm{H}}  \left( \sigma _ { i } , \sigma _ { j } , \sigma _ { k } \right) = \sum _ { \sigma _ { b } = \pm 1 } W \left( \sigma _ { b } ; \sigma _ { i } , \sigma _ { j } , \sigma _ { k } \right) = 2 \cosh \left( \mathcal { L } ( \sigma _ { i } +\sigma _ { j } + \sigma _ { k }) \right)~.
$$
Derive that
\be\label{hexagon}
w^{\mathrm{H}}  \left( \sigma _ { i } , \sigma _ { j } , \sigma _ { k } \right) =2\cosh^3(\cL) +2\cosh(\cL) \sinh^2(\cL) [\sigma_i\sigma_j+\sigma_j\sigma_k+\sigma_k\sigma_i]~.
\ee

On the other hand, for the triangle lattice, we can just assign the Boltzmann weight to each triangle $(\sigma_i,\sigma_j,\sigma_k)$ of spins
\bea\label{triangle}
w^{\mathrm{T}} \left( \sigma _ { i } , \sigma _ { j } , \sigma _ { k } \right) &=R\exp \left( \cK  [\sigma_i\sigma_j+\sigma_j\sigma_k+\sigma_k\sigma_i] \right)~,\cr
&=R(\cosh^3(\cK)+\sinh^3(\cK))\cr
&\ \ +R\cosh(\cK)\sinh(\cK)(\cosh(\cK)+\sinh(\cK))[\sigma_i\sigma_j+\sigma_j\sigma_k+\sigma_k\sigma_i]~.
\eea
so that the partition function is
$$
R^NZ ^ { \mathrm { T } } ( \mathcal { L } ) = \sum _ { \sigma _ { a }:\mathrm{white} }~ \prod _ { i , j , k } w^{\mathrm{T}}  \left( \sigma _ { i } , \sigma _ { j } , \sigma _ { k } \right)
$$


It is easy to see that \eqref{hexagon} and \eqref{triangle} are written in the same form. Show that $w^{\mathrm{H}}$ and $w^{\mathrm{T}}$ coincide  when
$$
R^2\sinh (2\cK) =2\sinh^2(2\cL)~,\qquad (R/2)^4=\cosh^3(\cL)\cosh(3\cL)~.
$$








\end{document}
