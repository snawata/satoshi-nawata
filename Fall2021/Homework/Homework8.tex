\documentclass[12pt,a4paper]{article}
%\usepackage{hyperref} % Use the Charter font for the document text
%\usepackage[UTF8]{ctex}


\usepackage{macros}

\begin{document}\thispagestyle{empty}

\centerline{\Large \bf Homework 8: Due on Nov 14}



\section{Derivation}
Derive eigenvalues and eigenvectors of the following $2N\times 2N$ matrix
\begin{equation}\nonumber
-iM=-i\left(\begin{array}{cccccccccccc}
0 & B& & & & & & & & -J \\
-B & 0 & & & & & && &  \\
& & \ddots & \ddots & & & & & & \\
& & \ddots & 0 & B & & & & & \\
& & & -B & 0 & J & & & & \\
& & & & -J & 0 & B & & & \\
& & & & & -B & 0 & J & & \\
& & & & & & -J & 0 & \ddots & \\
& & & & & & & \ddots &\ddots & B \\
J& & & & & & & &  -B & 0
\end{array}\right)
\end{equation}



\section{Onsager's exact solution}
Let us consider an $M\times N$ square periodic lattice. At each vertex, one can define fermionic operators
\begin{equation}\nonumber
\psi_{i}(x)=\sigma(x) \mu\left(x+a_{i}\right)~,\qquad (i=1,\ldots,4)
\end{equation}
where the position is defined as below
\begin{figure}[h]\centering
  \includegraphics{lattice}
\end{figure}

Suppose that the branch cut from the disordered operator $\mu$ runs to the left. Then, we can write
\begin{equation}\nonumber
\begin{aligned}
\psi_{1}(x) &=\sigma(x) \mu\left(x+a_{1}\right) \\
&=\sigma(x) \mu\left(x+a_{2}\right) e^{-2 K \sigma(x) \sigma\left(x+e_{2}\right)} \\
&=\sigma(x) \mu\left(x+a_{2}\right)\left(\operatorname{ch}(2 K)-\operatorname{sh}(2 K) \sigma(x) \sigma\left(x+e_{2}\right)\right) \\
&=\operatorname{ch}(2 K) \psi_{2}(x)-\operatorname{sh}(2 K) \psi_{3}\left(x+e_{2}\right)
\end{aligned}
\end{equation}
Show that
\begin{equation}\nonumber
\slashed{D}\Psi=0, \quad \slashed{D}=\left[\begin{array}{cccc}
-1 & \operatorname{ch}(2 K) & -\operatorname{sh}(2 K) D_{2} & 0\\
0& -1 & \operatorname{ch}(2 K) & -\operatorname{sh}(2 K) D_{1}^{-1} \\
\operatorname{sh}(2 K) D_{2}^{-1} &0 & -1 &\operatorname{ch}(2 K)\\
-\operatorname{ch}(2 K) & \operatorname{sh}(2 K) D_{1} & 0& -1
\end{array}\right], \quad \Psi=\left[\begin{array}{l}
\psi_{1} \\
\psi_{2} \\
\psi_{3} \\
\psi_{4}
\end{array}\right]
\end{equation}
where $D_i$ is the shift operator along $e_i$. Essentially, the partition function of fermionic model of Ising model can be understood as
\begin{equation}\nonumber
Z_{M N}=\int d\Psi e^{\Psi \slashed{D}\Psi}=(\det \slashed{D})^{1/2}
\end{equation}
Show that the free energy of the Ising model is
\begin{align}\nonumber
&\lim_{M,N\to\infty} \frac{1}{M N} \log Z_{M N}\cr =&\frac12\log[ 2\operatorname{ch}^{2}(2 K)]+\frac12\int_{0}^{2 \pi} \int_{0}^{2 \pi} \log \left(\operatorname{ch}^{2}(2 K)-\operatorname{sh}(2 K)\left(\cos \theta_{1}+\cos \theta_{2}\right)\right) \frac{d \theta_{1}}{2 \pi} \frac{d \theta_{2}}{2 \pi}
\end{align}
which is the Onsager's exact solution.


\section{XXZ spin chain}
Consider the quantum spin chain of the XXZ model with Hamiltonian
$$
H=\frac{J_{1}}4 \sum_{k=1}^{N}\left(\sigma_{k}^{x} \sigma_{k+1}^{x}+\sigma_{k}^{y} \sigma_{k+1}^{y}\right)+\frac{J_{2}}4 \sum_{k} \sigma_{k}^{z} \sigma_{k+1}^{z}
$$
with periodic boundary conditions. Defining the operators
\begin{equation}\nonumber
c_{i}=\left(\prod_{j<i} {\sigma}_{j}^{z}\right) {\sigma_{i}}^{+} ~, \qquad c_{i}^{\dagger}=\left(\prod_{j<i} {\sigma}_{j}^{z}\right) {\sigma}_{i}^{-}
\end{equation}
where $\sigma_{i}^{\pm}=\frac12(\sigma_{i}^{x} \pm i \sigma_{i}^{y})$. Show that the hamiltonian can be written as
$$
\begin{aligned}
  H= \frac{J_{1}}{2} \sum_{a=1}^{N}\left[c^{\dagger}_{a} c_{a+1}+c^{\dagger}_{a+1} c_{a}\right]+J_{2} \sum_{a=1}^{N}\left(c^{\dagger}_{a} c_{a}-\frac{1}{2}\right)\left(c^{\dagger}_{a+1} c_{a+1}-\frac{1}{2}\right)
\end{aligned}
$$
Setting $J_{2}=0$, diagonalize the hamiltonian.









\end{document}
