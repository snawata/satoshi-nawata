\documentclass[12pt,a4paper]{article}
%\usepackage{hyperref} % Use the Charter font for the document text
%\usepackage[UTF8]{ctex}


\usepackage{macros}

\usepackage{graphicx}

\begin{document}\thispagestyle{empty}

\centerline{\Large \bf Homework 11: Due at class on Dec 12}

\section{Derivation}
Derive (7.17) for the variation of the action (7.15) under $g\to g+\d g$.


\section{Solutions to KZ equations}
\Yboxdim7pt



\subsection{}
The spin-$j$ representation $V_{j}$ of $\mathfrak{su}(2)$ can be labelled by the Young diagrams $\underbrace{\yng(3)\cdots \yng(3)}_{2j}$ with one row. For instance, the spin-$\frac12$ representation is two-dimensional and the basis are spanned by $|+\rangle$ and $|-\rangle$. It corresponds to a single box $\yng(1)$. The tensor product of the two spin-$\frac12$ representations is decomposed into
$$
(\yng(1))^{\otimes 2}=\yng(1)\otimes \yng(1)=\yng(2)\oplus \mathbf{1}
$$
where $\mathbf{1}$ is the trivial one-dimensional representation. This can be understood as the fusion rule.
Let us write the basis $|j_1\rangle\otimes |j_2\rangle:=|j_1,j_2\rangle$ of the tensor product $(\yng(1))^{\otimes 2}$. Write the basis of the representation labelled by $\yng(2)$ as well as $\mathbf{1}$.

In addition, find the fusion rules of the following tensor products
$$
(\yng(1))^{\otimes 3}~, \qquad (\yng(1))^{\otimes 4}~,\quad \textrm{and} \quad (\yng(1))^{\otimes 5}~.
$$




\subsection{}
In the lecture, we have learned the correlation functions of $N$ WZW primary fields obey the Knizhnik-Zamolodchikov (KZ) equation
$$
\bigg[
\partial_{z_i}+ \frac{2}{k+h^\vee}
 \sum _ { j ( \neq i ) = 1 } ^ { n }
\frac{\sum_a t^a_{\lambda_i} \otimes t^a_{\lambda_j}}{z_i-z_j}
\bigg]
\langle\phi_{\lambda_1}(z_1)\cdots\phi_{\lambda_n}(z_n)\rangle
=0\, .
$$
Let us consider the situation that $\frakg=\mathfrak{su}(2)$ ($h^\vee=2$) and  all the primary fields are labelled by the spin-$\frac12$ representation, \textit{i.e.} $\phi_{\lambda_i}(z_i)=\phi_{{\Yboxdim4pt\yng(1)}}(z_i)$. In this situation, we write
$$
\Omega_{ij}=\sum_a (t^a_{\Yboxdim4pt\yng(1)})_i \otimes (t^a_{\Yboxdim4pt\yng(1)})_j
$$
where $t^a_{\Yboxdim4pt\yng(1)}=\frac12\sigma^a$ with the Pauli matrices $\sigma^a$. By studying the action  $\Omega _ { 12 }$  on $|\pm,\pm\rangle$ explicitly, show
$$
\Omega _ { 12} = \frac12\Big(s _ { 12 } - \frac { 1 } { 2 }\Big)
$$
where $s _ { 12 }$ is the exchange of the first and second spin.

\subsection{}
As in Problem 2.1, the fusion rule of $n$ primary fields labelled by $\yng(1)$ are
\bea\nonumber
(\yng(1))^{\otimes n}&=V_{\frac n2}\oplus (n-1)V_{\frac{n-2}{2}}\oplus \cdots\cr
&=\underbrace{\yng(3)\cdots \yng(3)}_{n}\ \oplus \ (n-1) \underbrace{\yng(3)\cdots \yng(3)}_{n-2}\ \oplus \ \cdots
\eea
Let us find the solutions of the KZ equations corresponding to $V_{\frac n2}$ and   $V_{\frac {n-2}2}$.
The highest weight state of the representation labelled by $V_{\frac n2}$ is $|+\cdots+\rangle$. Writing down the correlation function corresponding to  this state by $$\Phi_{\frac n2}(z_1,\ldots,z_n)=\psi_0(z_1,\ldots,z_n)|+\cdots+\rangle~,$$
find the solution of the KZ equation
$$
\bigg[
\partial_{z_i}+ \frac{2}{k+2}
 \sum _ { j ( \neq i ) = 1 } ^ { n }
\frac{\Omega_{ij}}{z_i-z_j}
\bigg]\Phi_{\frac n2}(z_1,\ldots,z_n)=0~.
$$


The correlation function corresponding to the highest weight state of $V_{\frac {n-2}2}$ can be written as
$$
\Phi_{\frac {n-2}2}(z_1,\ldots,z_n)=\psi_0(z_1,\ldots,z_n)\sum_{i=1}^n\psi_i(z_1,\ldots,z_n)|v_i\rangle
$$
where
$$
| v _ { 1 } \rangle = | -+ \cdots + \rangle ~, \quad | v _ { 2 } \rangle = | +-+ \cdots + \rangle ~,\quad \cdots ,\quad | v _ { n} \rangle = | + \cdots +- \rangle~,
$$
and
$$
\sum_{i=1}^n\psi_i(z_1,\ldots,z_n)=0~.
$$
Show that the KZ equations
$$
\bigg[
\partial_{z_1}+ \frac{2}{k+2}
 \sum _ { j =2 } ^ { n }
\frac{\Omega_{1j}}{z_1-z_j}
\bigg]\Phi_{\frac {n-2}2}(z_1,\ldots,z_n)=0~
$$
reduce to
$$
(k+2) \frac { \partial } { \partial z _ { 1 } } \psi _ { 1 } ( z ) + \frac { \psi _ { 2 } - \psi _ { 1 } } { z _ { 1 } - z _ { 2 } } + \frac { \psi _ { 3 } - \psi _ { 1 } } { z _ { 1 } - z _ { 3 } } + \dots + \frac { \psi _ { N } - \psi _ { 1 } } { z _ { 1 } - z _ { N } }=0~.
$$
and
$$
(k+2)  \frac { \partial } { \partial z _ { 1 } } \psi _ { 2 } ( z ) + \frac { \psi _ { 1 } - \psi _ { 2 } } { z _ { 1 } - z _ { 2 } }=0~.
$$
Show that
$$
\psi _ { i } ( z ) = \int _ { C } d t \prod _ { a = 1 } ^ { n} \left( z _ { a } - t \right) ^ { \frac { 1 } { k+2 } } \frac { 1 } { z _ { i } - t }
$$
becomes the solution of the KZ equations.
In fact, there are $n-1$ solutions by taking the different contours $C$.




\section{Verlinde algebra of Ising model}

There are finitely many primary fields in a rational conformal field theory like a minimal model and WZW model. The fusion rule of primary fields are closed under themselves
$$
\left[ \phi_{ i } \right] \times \left[ \phi_{ j } \right] = \sum _ { k } \mathcal { N } _ { i j } ^ { k } \left[ \phi_{ k } \right]~,
$$
where ${\cN } _ { i j } ^ { k } $ are called the fusion coefficients. As we have seen in  the minimal models and WZW models, one can construct a highest weight representation associated to a primary field $\phi_i$, and we write the corresponding character by $\chi_i(\tau)$. The modular transformations are
$$
\chi_i(-1/\tau)=\sum_{j}S_{ij}\,\chi_j(\tau)~, \qquad \chi_i(\tau+1)=\sum_{j}T_{ij}\,\chi_j(\tau)~.
$$
E. Verlinde has found the remarkable relationship between fusion rule and the modular $S$-matrices \cite{verlinde1988fusion}
\be\label{Verlinde}
\mathcal { N } _ { i j } ^ { k } = \sum _ { l } \frac { S _ { j l } S _ { i l } \left( S ^ { - 1 } \right) _ { l k } } { S _ { 0 l } }~.
\ee


\subsection{Ising model revisited}
Let us recall that the Ising model is the unitary minimal model $\cM_{p=3}$ where the primary fields are associated to
$$
\begin{array} { r l } { 1 } & { \Leftrightarrow \phi_{ 1 } : = \phi_{ 1,1 } } \\ { \epsilon } & { \Leftrightarrow \phi_{ 2 } : = \phi_{ 2,1 } } \\ { \sigma } & { \Leftrightarrow \phi_{ 3 } : = \phi_{ 2,2 } } \end{array}
$$
where $\e$ is the energy density filed and $\sigma$ is the spin field. The corresponding characters are given in (5.89) of the lecture notes
$$
\begin{aligned} \chi _ {1} ( \tau ):= \chi _ { 0 } ( \tau ) & = \frac { 1 } { 2 } \left( \sqrt { \frac { \vartheta_{ 3 } ( \tau ) } { \eta ( \tau ) } } + \sqrt { \frac { \vartheta_{ 4 } ( \tau ) } { \eta ( \tau ) } } \right) \\ \chi _ {2} ( \tau ):= \chi _ { \frac { 1 } { 2 } } ( \tau ) & = \frac { 1 } { 2 } \left( \sqrt { \frac { \vartheta_{ 3 } ( \tau ) } { \eta ( \tau ) } } - \sqrt { \frac { \vartheta_{ 4 } ( \tau ) } { \eta ( \tau ) } } \right) \\ \chi _ {3} ( \tau ):= \chi _ { \frac { 1 } { 16 } } ( \tau ) & = \frac { 1 } { \sqrt { 2 } } \sqrt { \frac { \vartheta_{ 2 } ( \tau ) } { \eta ( \tau ) } } \end{aligned}
$$
Using the properties of the $\vartheta$ and $\eta$-functions, find the $3\times 3$ $S$-matrix of the Ising model.  Then,  compute the three $3\times 3$ fusion matrices $\mathcal { N } _ { i j } ^ { k }$ ($i=1,2,3$) by using the Verlinde formula \eqref{Verlinde}. Check they reproduce the fusion rule in (5.87) of \S5.4 of the lecture note.





\bibliography{conformal-ref}
\bibliographystyle{hyperamsalpha}



\end{document}
