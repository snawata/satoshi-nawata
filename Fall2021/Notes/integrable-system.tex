%%Line 1810 uncomment

\documentclass[2dCFT-lecture.tex]{subfiles}

\allowdisplaybreaks
\begin{document}
\section{Integrable system}

\subsection{Quantum many-body system with delta-function potential}
free to move along a line. The Hamiltonian operator describing this model is given by
$$
H=-\sum_{i=1}^{N} \frac{\partial^{2}}{\partial x_{i}^{2}}+2 c \sum_{i<j} \delta\left(x_{i}-x_{j}\right)~, \quad c>0
$$
and where $\hbar=2 m=1$ is set for simplicity. The delta-function $\delta$ in (1.5) means that two particles affect each other only when they are at the same position. The real constant $c$ is a measure of strength of the interaction and is called coupling parameter. The case $c>0$ corresponds to repulsive interaction, $c<0$ to attractive interaction, and $c=0$ to free particles. The particles can pass through each other, except for the impenetrable limit $c \rightarrow \infty$.


\subsection*{Two bosons}
In the case of two bosons, which exhibits fully symmetric exchange statistics, the wave function takes the form
$$
\psi\left(x_{1}, x_{2}\right)=\left\{\begin{array}{ll}
A(12) e^{k_{1} x_{1}+k_{2} x_{2}}+A(21) e^{k_{2} x_{1}+k_{1} x_{2}} & x_{1}<x_{2} \cr
A(12) e^{k_{1} x_{2}+k_{2} x_{1}}+A(21) e^{k_{2} x_{2}+k_{1} x_{1}} & x_{1}>x_{2}
\end{array} .\right.
$$

The derivatives have a discontinuity at $x_{1}=x_{2}$. By changing variables to the center of mass coordinate $y=x_{1}+x_{2}$ and the relative coordinate $x=x_{1}-x_{2}$, the second boundary condition can be shown to be
\begin{equation}
\left.\left(\frac{\partial \psi}{\partial x_{1}}-\frac{\partial \psi}{\partial x_{2}}\right)\right|_{x_{1}=x_{2}+0}-\left.\left(\frac{\partial \psi}{\partial x_{1}}-\frac{\partial \psi}{\partial x_{2}}\right)\right|_{x_{1}=x_{2}-0}=\left.2 c \psi\right|_{x_{1}=x_{2}}
\end{equation}
It is easily seen that the continuity conditions are satisfied. Inserting the wave function into the discontinuity conditions (2.11) and (2.12) respectively, yield after some restructuring
$$
A(12)=-\frac{c-i\left(k_{1}-k_{2}\right)}{c+i\left(k_{1}-k_{2}\right)} A(21)=-e^{i \theta_{12}} A(21)
$$
$$
A(21)=-\frac{c-i\left(k_{2}-k_{1}\right)}{c+i\left(k_{2}-k_{1}\right)} A(12)=-e^{i \theta_{21}} A(12) .
$$
Here we have introduced the phase factor $\theta_{j s}=\theta\left(k_{j}-k_{s}\right)$, defined as
$$
\theta(r)=-2 \tan ^{-1}\left(\frac{r}{c}\right), \quad-\pi \leq \theta \leq \pi
$$
by using the formula
$$
\frac{1-i x}{1+i x}=\exp \left(-2 i \cdot \tan ^{-1}(x)\right)
$$
We see that the particles pick up a phase between them when they pass through each other. In view of of equations $(4.2)$ and $(4.3)$, we get the scattering $S$ matrices to
$$
\begin{aligned}
&S\left(k_{1}+k_{2}\right)=S\left(k_{1}-k_{2}\right)=-e^{i \theta_{12}}=-\frac{c-i\left(k_{1}-k_{2}\right)}{c+i\left(k_{1}-k_{2}\right)}+ \cr
&S\left(k_{2}, k_{1}\right)=S\left(k_{2}-k_{1}\right)=-e^{i \theta_{21}}=-\frac{c-i\left(k_{2}-k_{1}\right)}{c+i\left(k_{2}-k_{1}\right)}
\end{aligned}
$$
By applying the PBC's to this system, we get the relations
$$
A(12)=A(21) e^{i k_{1} L}, \quad A(21)=A(12) e^{i k_{2} L} .
$$
If we then combine (4.2) and (4.3) with (4.5), we see that the momenta $k_{1}$ and $k_{2}$ are quantized as follows
$$
\left.\begin{array}{l}
k_{1} L=2 \pi I_{1}+\theta_{12} \cr
k_{2} L=2 \pi I_{2}+\theta_{21}
\end{array}\right\} \quad I_{1}, I_{2}=\pm \frac{1}{2}, \pm \frac{3}{2}, \pm \frac{5}{2}, \ldots .
$$
If $k_{1}=k_{2}$ it follows from (4.2) that $A(12)=-A(21)$ and the wave function vanishes everywhere. Thus, the momenta must be unequal to avoid the wave function to vanish, which also results in $I_{1} \neq I_{2}$.

\subsection*{Three bosons}
The wave function (2.18) for three bosons reads
$$
\begin{aligned}
\psi_{1}\left(x_{1}, x_{2} x_{3}\right) &=A(123) e^{k_{1} x_{1}+k_{2} x_{2}+k_{3} x_{3}}+A(213) e^{k_{2} x_{1}+k_{1} x_{2}+k_{3} x_{3}} \cr
&+A(231) e^{k_{2} x_{1}+k_{3} x_{2}+k_{1} x_{3}}+A(321) e^{k_{3} x_{1}+k_{2} x_{2}+k_{1} x_{3}} \cr
&+A(312) e^{k_{3} x_{1}+k_{1} x_{2}+k_{2} x_{3}}+A(132) e^{k_{1} x_{1}+k_{3} x_{2}+k_{2} x_{3}}
\end{aligned}
$$
where the wave function is symmetric with respect to any exchange of two bosons. The boundary condition $(2.17)$ at $x_{1}=x_{2}$ implies the following relations among the amplitudes
$$
\frac{A(213)}{A(123)}=-e^{i \theta_{21}}, \frac{A(312)}{A(132)}=-e^{i \theta_{31}}, \frac{A(321)}{A(231)}=-e^{i \theta_{32}},
$$
where $\theta_{i j}$ is the phase factor $(4.4)$. The boundary condition $(2.17)$ at $x_{2}=x_{3}$ implies the relations
$$
\frac{A(132)}{A(123)}=-e^{i \theta_{32}}, \quad \frac{A(231)}{A(213)}=-e^{i \theta_{31}}, \quad \frac{A(321)}{A(312)}=-e^{i \theta_{21}} .
$$
We can see that, according to equations (4.8) and (4.9), the interchange of two particles is independent of the third particle. This means that an interaction involving more than two particles can be factorized into a product of two-body interactions. For example, if we start from the incoming amplitude $A(123)$, we can reach the outgoing amplitude $A(321)$ along two different paths:
$$
A(123) \rightarrow\left\{\begin{array}{ll}
A(213) & \rightarrow A(231) \cr
A(132) & \rightarrow A(312)
\end{array}\right\} \rightarrow A(321) .
$$
By using (4.8) and (4.9), we can see that both paths give the same result
$$
\frac{A(321)}{A(123)}=-e^{\theta_{21}+\theta_{31}+\theta_{32}}=S\left(k_{2}, k_{1}\right) S\left(k_{3}, k_{1}\right) S\left(k_{3}, k_{2}\right) .
$$
This equivalence of two different paths is the precursor of the famous YangBaxter equations for the scattering matrices of particles with non-zero spin, which is discussed in Section 5.3.
The PBC for the $x_{1}$-particle gives the relations
$$
A(123)=A(231) e^{i k_{1} L}, A(213)=A(132) e^{i k_{2} L}, A(312)=A(123) e^{i k_{3} L} .
$$
By applying equations (4.8) and (4.9) to equation (4.11) we see that the momenta $k_{1}, k_{2}$ and $k_{3}$ are quantized as follows
$$
\left.\begin{array}{l}
k_{1} L=2 \pi I_{1}+\theta_{12}+\theta_{13} \cr
k_{2} L=2 \pi I_{2}+\theta_{21}+\theta_{23} \cr
k_{3} L=2 \pi I_{3}+\theta_{31}+\theta_{32}
\end{array}\right\} \quad I_{1}, I_{2}, I_{3}=\text { integer. }
$$


\subsection*{$N$ bosons}
\begin{equation}
\exp \left(i k_{j} L\right)=\prod_{i \neq j} S_{i j}=\prod_{i \neq j} \frac{k_{j}-k_{i}+i c}{k_{j}-k_{i}-i c}
\end{equation}

\begin{equation}
\begin{aligned}
k_{j} L &=2 \pi I_{j}-2 \sum_{s=1}^{N} \tan ^{-1}\left(\frac{k_{j}-k_{s}}{c}\right) \cr
&=2 \pi I_{j}+\sum_{s=1}^{N} \theta_{j s}
\end{aligned}
\end{equation}

\subsection{Heisenberg XXX Spin Chain}
\begin{equation}
H=J \sum_{(i j)} S_{i} \cdot S_{j}=J \sum_{(i j)}\left[S_{i}^{z} S_{j}^{z}+\frac{1}{2}\left(S_{i}^{+} S_{j}^{-}+S_{i}^{-} S_{j}^{+}\right)\right]
\end{equation}


For $M=0$, we consider the invariant subspace $\mathcal{H}_{0}$, which is one dimensional and spanned by the vector $|0\rangle=|\uparrow \uparrow \ldots \uparrow\rangle$. Acting with the Hamiltonian on this state, we see that the first two terms inside the summation yield 0, because $S_{n}^{-}$ kills spin-up states. The last term inside the summation yields a factor $\frac{1}{2}$ for each $n \in \mathbb{Z} / L Z \mathbb{Z} .$ Therefore $\hat{H}|0\rangle=E_{0}|0\rangle$, where the eigenvalue is given by
$$
E_{0}=-\frac{J L}{4}
$$
which concludes the spectrum for $M=0 .$
Note that while $|0\rangle$ is indeed the ground state of the ferromagnetic state, for the case $J<0$ the notation $|0\rangle$ is deceiving! Indeed, for the anti-ferromagnetic state $(J<0)$ the ground state is a state in which nearest neighbours are misaligned. We can however safely ignore this subtlety, as we will only consider $J>0$.


\subsection*{One magnon}

The coordinate Bethe Ansatz
\begin{equation}
\left|\psi_{k}\right\rangle=\frac{1}{\sqrt{L}} \sum_{n \in Z / L Z} e^{i k n}|n\rangle
\end{equation}
To find the corresponding eigenvalues, we act with the Hamiltonian on (1.11). Let us first write down the action of the terms in $\hat{H}$ on the basis states. When the operator $\sum_{n} S_{n}^{+} S_{n+1}^{-}$acts on a basis state $|m\rangle$, the $S_{n}^{+}$will yield 0, except when $n=m$, as follows from $(1.5)$. For $n=m$, the down spin is flipped to up, after which the up spin one site to the right is flipped down by $S_{n+1}^{-} .$Hence we have that
$$
\sum_{n \in \mathbb{Z} / L Z} S_{n}^{+} S_{n+1}^{-}|m\rangle=|m+1\rangle
$$
We see that this operator just acts like a shift operator to the right on the down spin. Similarly, the operator $\sum_{n} S_{n}^{-} S_{n+1}^{+}$acts like a shift operator to the left:
$$
\sum_{n \in \mathbb{Z} / L Z} S_{n}^{+} S_{n+1}^{-}|m\rangle=|m-1\rangle
$$
Finally, the the operator $\sum_{n} S_{n}^{z} S_{n+1}^{z}$ gives a contribution of $-\frac{1}{4}$ for each pair of misaligned adjacent spins and a contribution of $\frac{1}{4}$ for all other adjacent pairs. Since each basis state has two misaligned adjacent pairs, we get
$$
\sum_{n \in Z / L Z} S_{n}^{z} S_{n+1}^{z}|m\rangle=\frac{L-4}{4}|m\rangle .
$$

With these last three equations, it is easy to write down the action of the Hamiltonian on the vector (1.11). The result is
$$
\begin{aligned}
-\frac{2}{J} \hat{H}\left|\psi_{k}\right\rangle &=\sum_{m \in Z / L Z}\left(S_{m}^{+} S_{m+1}^{-}+S_{m}^{-} S_{m+1}^{+}+2 S_{m}^{z} S_{m+1}^{z}\right) \frac{1}{\sqrt{L}} \sum_{n \in Z / L Z} e^{i k n}|n\rangle \cr
&=\frac{1}{\sqrt{L}} \sum_{n \in Z / L Z} e^{i k n}\left(|n+1\rangle+|n-1\rangle+\frac{L-4}{2}|n\rangle\right) \cr
&=\frac{1}{\sqrt{L}} \sum_{n \in Z / L Z}\left(e^{i k(n-1)}+e^{i k(n+1)}+\frac{L}{2}-2\right)|n\rangle \cr
&=2\left(\cos k-1+\frac{L}{4}\right)\left|\psi_{k}\right\rangle=:-\frac{2}{J} E_{k}\left|\psi_{k}\right\rangle
\end{aligned}
$$
from which we find the following eigenvalues $E_{k}$ :
$$
E_{k}-E_{0}=J(1-\cos k):=E_{1}(k)
$$
where we used $(1.10)$. Note we indeed have that $E_{k=0}=E_{0}$. This degeneracy (i.e. an eigenvalue with an eigenspace of dimension $>1$ ) is a consequence of the full $S U(2)$ symmetry: note that we have only exploited a subgroup $U(1)_{z} \subset S U(2)$. We have found $L$ eigenvectors and the corresponding eigenvalues, so we are done with the case $M=1$.

\subsection*{Two magnon}

For $M=2$, we consider the subspace $\mathcal{H}_{2}$ which has dimension $\left(\begin{array}{l}L \cr 2\end{array}\right)=\frac{1}{2} L(L-1)$ We will denote the basis states as $\left|n_{1}, n_{2}\right\rangle:=S_{n_{1}}^{-} S_{n_{2}}^{-}|0\rangle .$ Unlike the case $M=1$, the translational invariance is not sufficient to give an Ansatz for the eigenstates. Let us first write down the expression for a general state:
$$
|\psi\rangle=\sum_{n_{2}>n_{1}} f\left(n_{1}, n_{2}\right)\left|n_{1}, n_{2}\right\rangle
$$

$\sum_{n \in Z / L z} S_{n}^{+} S_{n+1}^{-}\left|m_{1}, m_{2}\right\rangle= \begin{cases}\left|m_{1}, m_{1}+2\right\rangle & \text { if } m_{2}=m_{1} \pm 1 \bmod L \cr \left|m_{1}+1, m_{2}\right\rangle+\left|m_{1}, m_{2}+1\right\rangle & \text { otherwise. }\end{cases}$
(1.19)
The operator $\sum_{n} S_{n}^{-} S_{n+1}^{+}$works similarly as a shift operator to the left. If the spins are adjacent, only the spin at site $m_{1}$ can be shifted. We get:
$$\sum_{n \in Z / L z} S_{n}^{-} S_{n+1}^{+}\left|m_{1}, m_{2}\right\rangle= \begin{cases}\left|m_{1}-1, m_{1}+1\right\rangle & \text { if } m_{2}=m_{1} \pm 1 \bmod L \cr \left|m_{1}-1, m_{2}\right\rangle+\left|m_{1}, m_{2}-1\right\rangle & \text { otherwise. }\end{cases}$$
(1.20)
As in the case $M=1$, the operator $\sum_{n} S_{n}^{z} S_{n+1}^{z}$ gives a contribution $-\frac{1}{4}$ for each pair of misaligned adjacent spins and a contribution $\frac{1}{4}$ for all other adjacent pairs spins. If the two down spins are not adjacent, there are four pairs of misaligned spins; if the down spins are adjacent there are two such pairs. Therefore we get:
$$\sum_{n \in Z / L Z} S_{n}^{z} S_{n+1}^{z}\left|m_{1}, m_{2}\right\rangle= \begin{cases}\left(\frac{L-2}{2}-\frac{1}{2}\right)\left|m_{1}, m_{1}+1\right\rangle & \text { if } m_{2}=m_{1} \pm 1 \bmod L \cr \left(\frac{L-4}{4}-1\right)\left|m_{1}, m_{2}\right\rangle & \text { otherwise. }\end{cases}$$

We start with splitting the sum in the case where $m_{1}$ and $m_{2}$ are adjacent and the case in which they are not. This yields:
\begin{align}
-\frac{2}{J} \hat{H}|\psi\rangle=& \sum_{n_{2}>n_{1}} f\left(n_{1}, n_{2}\right) \sum_{m \in \mathbb{/} L Z}\left(S_{m}^{+} S_{m+1}^{-}+S_{m}^{-} S_{m+1}^{+}+2 S_{m}^{z} S_{m+1}^{z}\right)\left|n_{1}, n_{2}\right\rangle \cr
=& \sum_{n_{2}>n_{1}+1} f\left(n_{1}, n_{2}\right)\left(\left|n_{1}+1, n_{2}\right\rangle+\left|n_{1}, n_{2}+1\right\rangle\right.\cr
&\left.+\left|n_{1}-1, n_{2}\right\rangle+\left|n_{1}, n_{2}-1\right\rangle+\frac{L-8}{2}\left|n_{1}, n_{2}\right\rangle\right) \cr
&+\sum_{n \in Z / L z} f(n, n+1)\left(|n, n+2\rangle+|n-1, n+1\rangle+\frac{L-4}{2}|n, n+1\rangle\right) \cr
=& \sum_{n_{2}>n_{1}} f\left(n_{1}-1, n_{2}\right)\left|n_{1}, n_{2}\right\rangle+\sum_{n_{2}>n_{1}+2} f\left(n_{1}, n_{2}-1\right)\left|n_{1}, n_{2}\right\rangle \cr
&+\sum_{n_{2}>n_{1}+2} f\left(n_{1}+1, n_{2}\right)\left|n_{1}, n_{2}\right\rangle+\sum_{n_{2}>n_{1}} f\left(n_{1}, n_{2}+1\right)\left|n_{1}, n_{2}\right\rangle \cr
&+\sum_{n_{2}>n_{1}+1} \frac{L-8}{2} f\left(n_{1}, n_{2}\right)\left|n_{1}, n_{2}\right\rangle \cr
&+\sum_{n \in Z / L Z} f(n, n+1)\left(|n, n+2\rangle+|n-1, n+1\rangle+\frac{L-4}{2}|n, n+1\rangle\right)\cr
=& \sum_{n_{2}>n_{1}+1}\left(f\left(n_{1}-1, n_{2}\right)+f\left(n_{1}, n_{2}-1\right)+f\left(n_{1}+1, n_{2}\right)\right.\cr &\left.+f\left(n_{1}, n_{2}+1\right)+\frac{L-8}{2} f\left(n_{1}, n_{2}\right)\right)\left|n_{1}, n_{2}\right\rangle \cr
&+\sum_{n \in \mathbb{Z} / L Z}\left(f(n-1, n+1)+f(n, n+2)-\frac{L-4}{2} f(n, n+1)\right)|n, n+1\rangle
\end{align}
Subsequently we read off the functions $\alpha$ and $\beta .$ Equation (1.23) then becomes
$$
\begin{gathered}
\left(E-E_{0}\right) f\left(n_{1}, n_{2}\right)=\frac{J}{2}\left(4 f\left(n_{1}, n_{2}\right)-f\left(n_{1}-1, n_{2}\right)-f\left(n_{1}, n_{2}-1\right)\right. \cr
\left.-f\left(n_{1}+1, n_{2}\right)-f\left(n_{1}, n_{2}+1\right)\right) \quad \text { for } n_{2}>n_{1}+1(1.26) \cr
\left(E-E_{0}\right) f(n, n+1)=\frac{J}{2}(2 f(n, n+1)-f(n-1, n+1)-f(n, n+2))(1.27)
\end{gathered}
$$
where we used (1.10).


include the effect of this interaction, we use the celebrated Ansatz made by Hans Bethe in his 1931 paper:
$$
f\left(n_{1}, n_{2}\right)=A e^{i\left(k_{1} n_{1}+k_{2} n_{2}\right)}+B e^{i\left(k_{2} n_{1}+k_{1} n_{2}\right)} .
$$
This Ansatz is known as the Coordinate Bethe Ansatz. Physically, we can see from this Ansatz that the magnons interact and in doing so they exchange momenta. Bethe's brilliant insight was that in the regions where the down spins are separated, the solution should look like a free wave, because we only consider nearest-neighbour interactions. One way to see these free waves is by rewriting the general state as
$$
|\psi\rangle=A \sum_{n_{2}>n_{1}} e^{i\left(k_{1} n_{1}+k_{2} n_{2}\right)}\left|n_{1}, n_{2}\right\rangle+B \sum_{n_{1}>n_{2}} e^{i\left(k_{1} n_{1}+k_{2} n_{2}\right)}\left|n_{1}, n_{2}\right\rangle,
$$
where in the second sum we switched the dummy indices $n_{1}$ and $n_{2}$. We see that this yields a product of free waves in the regions $n_{2}>n_{1}$ and $n_{1}>n_{2}$, but when passing between these regions the amplitude changes due to the interaction. ${ }^{1}$
We can now determine the spectrum by inserting the Coordinate Bethe Ansatz $(1.29)$ in equation (1.26). This yields
$$
\begin{aligned}
\left(E-E_{0}\right) f\left(n_{1}, n_{2}\right)=& \frac{J}{2}\left(4\left(A e^{i\left(k_{1} n_{1}+k_{2} n_{2}\right)}+B e^{i\left(k_{2} n_{1}+k_{1} n_{2}\right)}\right)\right.\cr
&-\left(A e^{i\left(k_{1} n_{1}+k_{2} n_{2}\right)} e^{-i k_{1}}+B e^{i\left(k_{2} n_{1}+k_{1} n_{2}\right)} e^{-i k_{2}}\right) \cr
&-\left(A e^{i\left(k_{1} n_{1}+k_{2} n_{2}\right)} e^{i k_{1}}+B e^{i\left(k_{2} n_{1}+k_{1} n_{2}\right)} e^{i k_{2}}\right) \cr
&-\left(A e^{i\left(k_{1} n_{1}+k_{2} n_{2}\right)} e^{-i k_{2}}+B e^{i\left(k_{2} n_{1}+k_{1} n_{2}\right)} e^{-i k_{1}}\right) \cr
&-\left(A e^{i\left(k_{1} n_{1}+k_{2} n_{2}\right)} e^{i k_{2}}+B e^{i\left(k_{2} n_{1}+k_{1} n_{2}\right)} e^{i k_{1}}\right) \cr
&=\frac{J}{2} f\left(n_{1}, n_{2}\right)\left(4-e^{i k_{1}}-e^{i k_{2}}-e^{-i k_{1}}-e^{-i k_{2}}\right) \cr
\Leftrightarrow E-E_{0}=& J\left(2-\cos k_{1}-\cos k_{2}\right)=\sum_{i=1}^{2} E_{1}\left(k_{i}\right)
\end{aligned}
$$

Having found the eigenvalues, we note that inserting $(1.31)$ back in $(1.26)$ and putting $n_{2}=n_{1}+1$, the equation still holds if we define $f\left(n_{1}, n_{2}\right)$ by the Coordinate Bethe Ansatz $(1.29)$ for $n_{1}=n_{2}$ as well. Subsequently we subtract $(1.26)$ for $n_{2}=n_{1}+1$ from $(1.27)$ to get
$$
f(n, n)+f(n+1, n+1)-2 f(n, n+1)=0 .
$$
Note that the extension of $f\left(n_{1}, n_{2}\right)$ to $n_{1}=n_{2}$ has no physical meaning. It is merely defined by $(1.29)$. Equation (1.32) therefore depends on the form of the Coordinate Bethe Ansatz.

If we insert the Bethe Ansatz in equation $(1.32)$, we obtain a restriction on the amplitudes $A$ and $B$ :
$$
\begin{aligned}
&(A+B) e^{i\left(k_{1}+k_{2}\right) n}+(A+B) e^{i\left(k_{1}+k_{2}\right)(n+1)}-2\left(A e^{i\left(k_{1} n+k_{2}(n+1)\right.}+B e^{i\left(k_{1}(n+1)+k_{2} n\right)}\right)= \cr
&\Leftrightarrow(A+B)\left(1+e^{i\left(k_{1}+k_{2}\right)}\right)-2 A e^{i k_{2}}-2 B e^{i k_{1}}=0 \cr
&\Leftrightarrow \frac{A}{B}=-\left(\frac{e^{i\left(k_{1}+k_{2}\right)}+1-2 e^{i k_{1}}}{e^{i\left(k_{1}+k_{2}\right)}+1-2 e^{i k_{2}}}\right)
\end{aligned}
$$
If we suppose that $k_{1}, k_{2} \in \mathbb{R}$, we note that
$$
\begin{aligned}
\left|\frac{A}{B}\right|^{2} &=\left(\frac{e^{i\left(k_{1}+k_{2}\right)}+1-2 e^{i k_{1}}}{e^{i\left(k_{1}+k_{2}\right)}+1-2 e^{i k_{2}}}\right)\left(\frac{e^{-i\left(k_{1}+k_{2}\right)}+1-2 e^{-i k_{1}}}{e^{-i\left(k_{1}+k_{2}\right)}+1-2 e^{-i k_{2}}}\right) \cr
&=\left(\frac{e^{i\left(k_{1}+k_{2}\right)}+1-2 e^{i k_{1}}}{e^{i\left(k_{1}+k_{2}\right)}+1-2 e^{i k_{2}}}\right)\left(\frac{1+e^{i\left(k_{1}+k_{2}\right)}-2 e^{i k_{2}}}{1+e^{i\left(k_{1}+k_{2}\right)}-2 e^{i k_{1}}}\right)=1
\end{aligned}
$$
hence $\frac{A}{B}$ is a phase and we can write
$$
e^{i \theta}:=\frac{A}{B}=-\left(\frac{e^{i\left(k_{1}+k_{2}\right)}+1-2 e^{i k_{1}}}{e^{i\left(k_{1}+k_{2}\right)}+1-2 e^{i k_{2}}}\right)
$$

\begin{equation}
\begin{aligned}
&S_{i j}=\frac{\lambda_{i}-\lambda_{j}-2 i}{\lambda_{i}-\lambda_{j}+2 i}\cr
&\lambda_{j}=\cot \left(k_{j} / 2\right), \quad 0 \leqq k_{j} \leqq \pi
\end{aligned}
\end{equation}



$$
e^{i k_{j} L}=\left[\frac{\lambda_{j}+i}{\lambda_{j}-i}\right]^{L}=\prod_{\ell \neq j} \frac{\lambda_{j}-\lambda_{\ell}+2 i}{\lambda_{j}-\lambda_{\ell}-2 i}
$$
Taking log, we obtain
$$
2 L \tan ^{-1}\left(\lambda_{j}\right)=2 \pi I_{j}+2 \sum_{i \neq j} \tan ^{-1}\left[\left(\lambda_{j}-\lambda_{i}\right) / 2\right]
$$
The total energy is
$$
E=-J\left[\sum_{j=1}^{M}\left(1-\cos k_{j}\right)-\frac{L}{4}\right]=-2 J \sum_{j=1}^{M} \frac{1}{\lambda_{j}^{2}+1}+\frac{L J}{4}
$$
The total momentum
$$
P=\sum_{j=1}^{M} k_{j}=2 \sum_{j=1}^{M}\left(-\tan ^{-1} \lambda_{j}+\pi / 2\right)=-\sum_{i=1}^{M} 2 \pi I_{j} / L+M \pi
$$




\subsection*{Ground state energy}
For large $L, L d f$ is the number of quantum numbers in the interval $(f, f+d f)$. The $k$ 's are ascending numbers so that $k_{j+1}-k_{j}>0$, and where the difference is of the order $\frac{1}{L}$. We then define
$$
k_{j+1}-k_{j}=\frac{1}{L \rho\left(\lambda_{j}\right)}, \quad \rho(\lambda) \geq 0
$$
The meaning of $\rho(\lambda)$ is that for large $L, L \rho(\lambda) d \lambda$ is the number of $k$ 's in the interval $(\lambda, k+d \lambda)$ and $\rho(\lambda)$ can be interpreted as the ground-state particle density in $k$-space. By combining $(4.19)$ and $(4.20)$, we get
$$
f_{j+1}-f_{j}=\frac{I_{j+1}-I_{j}}{L}=\frac{1}{L}=\rho\left(\lambda_{j}\right)\left(\lambda_{j+1}-k_{j}\right) .
$$
Thus, we have that $d f=\rho(\lambda) d \lambda$, which can be expressed as
$$
\frac{d}{d \lambda} f(\lambda)=\rho(\lambda), \quad f(\lambda)=\int_{0}^{k} \rho\left(\lambda^{\prime}\right) d \lambda^{\prime}
$$
Any summation over $k$ can therefore be replaced by an integral over $k$ through the prescription
$$
\sum_{k}(\cdots)=L \int_{-Q}^{Q}(\cdots) \rho(\lambda) d \lambda
$$
where $Q$ is some yet unknown limit which the density $\rho(\lambda)=\rho(-k)$ are symmetrically distributed over. Since $\sum_{k=1}^{N} 1=N$, it follows that
$$
n=\frac{N}{L}=\int_{-Q}^{Q} \rho(\lambda) d \lambda .
$$
The ground-state energy per unit length reads
$$
e_{0}=\frac{E_{0}}{L}=\int_{-Q}^{Q} k^{2} \rho(\lambda) d \lambda .
$$
Thus, we have to obtain an equation for the density $\rho(\lambda)$ in the ground-state. In taking the continuum limit of the Bethe equations (4.18), using (4.19) and $(4.21)$, we obtain
$$
k=2 \pi f(\lambda)+\int_{-Q}^{Q} \theta\left(\lambda-\lambda^{\prime}\right) \rho\left(\lambda^{\prime}\right) d \lambda^{\prime}
$$
Differentiating this equation with respect to $k$ leads to
$$
1=2 \pi \rho(\lambda)+\int_{-Q}^{Q} \theta^{\prime}\left(\lambda-\lambda^{\prime}\right) \rho\left(\lambda^{\prime}\right) d \lambda^{\prime},
$$
where
$$
\theta^{\prime}(\lambda)=\frac{\partial}{\partial k}\left(-2 \tan ^{-1}(\lambda / c)\right)=-\frac{2 c}{c^{2}+k^{2}}
$$
Equation (4.25) can then be expressed as
$$
\rho(\lambda)=\frac{1}{2 \pi}+\int_{-Q}^{Q} \frac{c / \pi}{c^{2}+\left(\lambda-\lambda^{\prime}\right)^{2}} \rho\left(\lambda^{\prime}\right) d \lambda^{\prime} .
$$
This integral equation is an inhomogeneous Fredholm equation of the second kind. The ground-state energy per unit length at a given density $N / L$ can be obtained by solving the set of coupled integral equations $(4.22),(4.23)$ and $(4.26)$. This can be done with numerical methods or in some cases analytically.


The Bethe Equations can be written as
$$
\arctan \lambda_{j}=\frac{\pi}{N} j+\frac{1}{N} \sum_{k} \arctan \left(\frac{\lambda_{j}-\lambda_{k}}{2}\right)
$$
In the $N \rightarrow \infty$ limit, the variable $x=\frac{1}{N}$ becomes continuous and limited in the range $-1 / 4 \leq$ $x \leq 1 / 4$. The set of roots $\lambda_{j}$ turn into a function $\lambda(x)$ and $(3.76)$ becomes
$$
\arctan \lambda(x)=\pi x+\int_{-1 / 4}^{1 / 4} \arctan \left(\frac{\lambda(x)-\lambda(y)}{2}\right) \mathrm{d} y .
$$
As observables depend on (are best expressed in terms of) the rapidities $\lambda_{j}$ and not on the integers $I_{0, j}$, we like to perform a change of variables and integrate over $\lambda$ rather than $x$ :
$$
\sum_{j} f\left(\lambda_{j}\right)=N \int_{-1 / 4}^{1 / 4} f(\lambda(x)) \mathrm{d} x=N \int_{-\infty}^{\infty} f(\lambda) \rho_{0}(\lambda) \mathrm{d} \lambda_{+}
$$
where the change of variables $x \rightarrow \lambda(x)$ maps interval $-1 / 4 \leq x \leq 1 / 4$ into whole real line $-\infty<\lambda<\infty$ due to the monotonicity of $\lambda(x)$. More explicitly, the density $\rho(\lambda)$ is
$$
\rho_{0}(\lambda)=\frac{\mathrm{d} x}{\mathrm{~d} \lambda}=\left.\frac{1}{\lambda^{\prime}(x)}\right|_{x=\lambda^{-1}(\lambda)}
$$
Finally, differentiating $(3.77)$ with respect to $\lambda$ we obtain an linear integral equation for the density of real roots $\rho_{0}(\lambda)$ :
$$
\rho_{0}(\lambda)=\frac{1}{\pi} \frac{1}{1+\lambda^{2}}-\frac{1}{\pi} \int_{-\infty}^{\infty} \frac{2}{(\lambda-\mu)^{2}+4} \rho_{0}(\mu) \mathrm{d} \mu .
$$
Notice that this integral equation is of the same time as the one we found for the Lieb-Liniger model and can be casted in the same form by remembering the definition of the scattering phase (4.24)
$$
\rho_{0}(\lambda)=\frac{1}{2 \pi} \theta_{1}^{\prime}(\lambda)-\frac{1}{2 \pi} \int_{-\infty}^{\infty} \mathcal{K}(\lambda-\nu) \rho_{0}(\mu) \mathrm{d} \mu
$$
where we introduced the kernel
$$
\mathcal{K}(\lambda) \equiv \frac{\mathrm{d}}{\mathrm{d} \lambda} \theta_{2}(\lambda)=\frac{2}{\lambda^{2}+4}
$$
However, since in this case the support of the density is on the whole real axis, this integral equation can be solved by Fourier transform:
$$
\tilde{\rho}_{0}(\omega)=\int_{-\infty}^{\infty} \mathrm{e}^{-\mathrm{i} \omega \lambda} \rho_{0}(\lambda) \mathrm{d} \lambda .
$$
Using
$$
\frac{1}{\pi} \int \frac{n}{\lambda^{2}+n^{2}} \mathrm{e}^{-\mathrm{i} \lambda \omega} \mathrm{d} \lambda=\mathrm{e}^{-n|\omega|}
$$
we can turn the integral equation (3.80) into
$$
\tilde{\rho}_{0}(\omega)\left(1+\mathrm{e}^{-2|\omega|}\right)=\mathrm{e}^{-|\omega|},
$$
which yields
$$
\rho_{0}(\lambda)=\frac{1}{2 \pi} \int_{-\infty}^{\infty} \mathrm{e}^{i \omega \lambda} \tilde{\rho}_{0}(\omega) \mathrm{d} \omega=\frac{1}{4 \cosh \left(\frac{\pi \lambda}{2}\right)}
$$
The momentum and energy of the ground state are then given by
$$
\begin{aligned}
&K=N \int p_{0}(\lambda) \rho_{0}(\lambda) \mathrm{d} \lambda=\frac{\pi}{2} N \bmod 2 \pi \equiv K_{\mathrm{AFM}} \\
&E=E_{0}+N \int \epsilon_{0}(\lambda) \rho_{0}(\lambda) \mathrm{d} \lambda=N\left(\frac{1}{4}-\ln 2\right) \equiv E_{\mathrm{AFM}}
\end{aligned}
$$
where $\rho_{0}(\lambda)$ and $\epsilon_{0}(\lambda)$ where defined in $(3.52,3.53)$.






\subsection{Yang-Baxter equations}


The Bethe ansatz for this problem, without considering any exchange symmetry, is
$$
\psi=\sum_{P \in S_{N}}[Q, P] \exp \left(i k_{P(1)} x_{Q(1)}+\cdots i k_{P(N)} x_{Q(N)}\right),
$$
where $[Q, P]$ is an $N ! \times N !$ coefficient-matrix. The rows correspond to permutation of the ordering $Q$ and the columns correspond to permutation of momenta $P$. The coefficient in the matrix corresponding to ordering $Q$ and permutation of momenta $P$ are denoted as $A(Q|P)$.

Two adjacent regions are separated by a hyperplane $x_{j}=x_{k}$, such that $x_{j}=x_{Q(i)}$ and $x_{k}=x_{Q(i+1)}$ for some $i=1,2 \ldots N-1$ or vice versa. The wave function shall at all hyperplanes satisfy the two boundary conditions
$$
\begin{gathered}
\left.\psi\right|_{x_{j}=x_{k}+0}=\left.\psi\right|_{x_{j}=x_{k}-0} \\
\left.\left(\frac{\partial \psi}{\partial x_{j}}-\frac{\partial \psi}{\partial x_{k}}\right)\right|_{x_{j}=x_{k}+0}-\left.\left(\frac{\partial \psi}{\partial x_{j}}-\frac{\partial \psi}{\partial x_{k}}\right)\right|_{x_{j}=x_{k}-0}=\left.2 c \psi\right|_{x_{j}=x_{k}}
\end{gathered}
$$
The wave function shall, in addition to these conditions, satisfy the PBC's
$$
\psi\left(x_{1}, \ldots x_{j}, \ldots, x_{N}\right)=\psi\left(x_{1}+\ldots, x_{j}+L, \ldots, x_{N}\right),
$$
for all $j=1,2, \ldots, N$.
$5.2$ Continuity of $\psi$ and discontinuity of its derivative
As was explained in Section $2.4$, the terms in the wave function come in pairs at the hyperplanes, and each such pair must individually satisfy the boundary conditions. Let $P$ be an arbitrary permutation of the momenta and let $P^{\prime}$ be the permutation obtained from $P$ by interchanging $P(j)$ and $P(j+1)$. Similarly, let $Q$ be an arbitrary permutation of the ordering and let $Q^{\prime}$ be the permutation obtained from $Q$ by interchanging $Q(j)$ and $Q(j+1)$. A direct substitution of the wave function (5.1) into the first and second boundary conditions (5.2) and (5.3), respectively, gives
$$
A(Q|P)+A\left(Q|P^{\prime}\right)=A\left(Q^{\prime}|P\right)+A\left(Q^{\prime}|P^{\prime}\right)
$$


$$
\begin{aligned}
i\left(k_{P(j)}-k_{P(j+1)}\right)\left(A\left(Q^{\prime}|P^{\prime}\right)-A\left(Q^{\prime}|P\right)\right.&\left.+A\left(Q|P^{\prime}\right)-A(Q|P)\right)=\\
&=2 c(A(Q|P)+A(Q|P))
\end{aligned}
$$
for all $j=1,2, \ldots, N-1$. This second equation gives a relation involving four different coefficients. By applying (5.5) to (5.6), one coefficient can be eliminated, giving the formula
$$
A\left(Q|P^{\prime}\right)=\frac{c}{i\left(k_{P(j)}-k_{P(j+1)}\right)-c} A(Q|P)+\frac{i\left(k_{P(j)}-k_{P(j+1)}\right)}{i\left(k_{P(j)}-k_{P(j+1)}\right)-c} A\left(Q^{\prime}|P\right) .
$$
Note that these equations involve the same expression for $R(u)$ and $T(u)$ as in the one particle problem with delta-function potential, where $u=k_{P(j)}-$ $k_{P(j+1)}$. This is not surprising, since the relative motion we used for any two particles is the same model as for the one particle case. Equation (5.7) may be written as
$$
\xi_{P^{\prime}}=Y_{P_{0 h}, P_{(j+1)}}^{j, j+1} \xi_{P,}
$$
where $\xi_{P}$ and $\xi_{P}$ are the column vectors of the matrix $[Q, P]$ corresponding to the permutation of momenta $P$ and $P^{\prime}$, respectively. The $Y$-operators are defined as
$$
Y_{i j}^{a b}=\frac{-x_{i j}}{1+x_{i j}}+\frac{1}{1+x_{i j}} \hat{P}_{a b}
$$
where
$$
x_{i j}=\frac{i c}{k_{i}-k_{j}}
$$
and where $\hat{P}_{a b}$ is the permutation operator on the coefficients so that it interchanges $Q(a)$ and $Q(b)$. Note that the $Y$-operators represent the scattering of two distinguishable particles, analogous to the $S$-matrices for identical particles.
In the case of two particles, equation (5.8) is written explicitly as
$$
\left(\begin{array}{c}
A(12|21) \\
A(21|21)
\end{array}\right)=\frac{-x_{12}}{1+x_{12}}\left(\begin{array}{c}
A(12|12) \\
A(21|12)
\end{array}\right)+\frac{1}{1+x_{12}}\left(\begin{array}{c}
A(21|12) \\
A(12|12)
\end{array}\right) .
$$

\end{document}
