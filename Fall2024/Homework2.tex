\documentclass[12pt,a4paper]{article}
%\usepackage{hyperref} % Use the Charter font for the document text
% %\usepackage[UTF8]{ctex}
% \usepackage{jheppub}
\usepackage{macros}
%%% Yokoyama def %%%
\providecommand{\vcentcolon}{\mathrel{\mathop{:}}}
%%%%%%%%%%%%%%%%%%%%
\usepackage{graphicx}


\begin{document}\thispagestyle{empty}

\centerline{\Large \bf Homework 2: Due on Oct 20}
\bigskip
\textbf{Caution:} 
When solving homework problems, it is important to show your derivation at each step. Nowadays, many online tools make it easy to find answers, but the primary goal of these assignments is to deepen your understanding through hands-on problem-solving. By working through the calculations yourself, you engage more deeply with the material, making the learning process more meaningful, rather than simply copying answers from external sources.

\section{Null geodesic motion}

\begin{enumerate}
    \item Obtain the geodesic equation of massless particles, with affine parameter $\lambda$, in the equatorial plane of a Schwarzschild black hole. Where is the unstable circular orbit located?
    
    \item For massless particles in the equatorial plane of a Kerr black hole, show that the geodesic equation for the radial motion is given by:
    \be 
    \left(\frac{d r}{d \lambda}\right)^2=\frac{\left(r^2+a^2\right)^2-a^2 \Delta}{r^4}\left(\e-V_{+}(r)\right)\left(\e-V_{-}(r)\right)
    \ee 
    and derive the explicit form of $V_\pm$.
    Where is the unstable circular orbit located?
\end{enumerate}

 


\section{Kerr-Newman solution}

\begin{enumerate}
    \item The Kerr-Newman metric describes a rotating, charged black hole.  It can be written by modifying the Kerr metric with the substitution:
    \[
    \Delta = r^2 - 2 M r + a^2 + e^2
    \]
    where $e = \sqrt{Q^2 + P^2}$, and $Q$ and $P$ represent the electric and magnetic (monopole) charges, respectively. Show that in the limit where $a = 0$ and $J = 0$, the Kerr-Newman solution reduces to the Reissner-Nordström solution.

    
    \item The Maxwell 1-form for the Kerr-Newman solution is given by:
    \[
    A = \frac{Q r \left( dt - a \sin^2 \theta \, d\phi \right) - P \cos \theta \left[ a dt - \left( r^2 + a^2 \right) d\phi \right]}{\Sigma}
    \]
    Use Mathematica (or a similar tool) to verify that this solution satisfies Einstein's equations. Compute the Kretschmann scalar, $R_{\mu\nu\rho\sigma} R^{\mu\nu\rho\sigma}$, to locate the genuine singularity and analyze its topology. Discuss how the positions of the event horizons change when you introduce a non-zero charge parameter $e$, compared to the case of the Kerr black hole ($e=0$)?
\end{enumerate}



\end{document}
