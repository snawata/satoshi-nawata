\documentclass[12pt,a4paper]{article}
%\usepackage{hyperref} % Use the Charter font for the document text
% %\usepackage[UTF8]{ctex}
% \usepackage{jheppub}
\usepackage{macros}
%%% Yokoyama def %%%
\providecommand{\vcentcolon}{\mathrel{\mathop{:}}}
%%%%%%%%%%%%%%%%%%%%
\usepackage{graphicx}


\begin{document}\thispagestyle{empty}

\centerline{\Large \bf Homework 4: Due on Nov 24}
\bigskip
\textbf{Caution:} 
When solving homework problems, it is important to show your derivation at each step. Nowadays, many online tools make it easy to find answers, but the primary goal of these assignments is to deepen your understanding through hands-on problem-solving. By working through the calculations yourself, you engage more deeply with the material, making the learning process more meaningful, rather than simply copying answers from external sources.


\section{Surface Gravity of Kerr Black Holes}
Find Killing horizon and its surface gravity in Kerr black hole.

\section{Isotropic Coordinate in the Schwarzschild Metric}

To gain insight into the geometry of $t = \text{constant}$ hypersurfaces in the Schwarzschild spacetime, we can rewrite the Schwarzschild metric in isotropic coordinates $(t, \rho, \theta, \phi)$, where $\rho$ is the new radial coordinate given by
\[
r = \left(1 + \frac{M}{2 \rho}\right)^2 \rho
\]
Using this coordinate transformation, show that the Schwarzschild metric becomes
\[
d s^2 = -\left(\frac{1 - \frac{M}{2 \rho}}{1 + \frac{M}{2 \rho}}\right)^2 d t^2 + \left(1 + \frac{M}{2 \rho}\right)^4 \left[d \rho^2 + \rho^2 d \Omega^2\right].
\]
Determine the value of $\rho$ corresponding to the Schwarzschild horizon in this coordinate. 




\section{Raychaudhuri Equation in FLRW Spacetime}

We consider the congruence of comoving world lines in a FLRW spacetime. The metric for this spacetime is given by
\begin{equation}
\mathrm{d}s^2 = -\mathrm{d}t^2 + a^2(t) \left( \frac{\mathrm{d}r^2}{1 - k r^2} + r^2 \, \mathrm{d}\Omega^2 \right),
\end{equation}
where \( a(t) \) is the scale factor and \( k \) is a constant normalized to \( \pm 1 \) or \( 0 \). 

Let \( u^\alpha = \frac{\partial x^\alpha}{\partial t} \) be the vector tangent to the congruence of comoving world lines.

\begin{enumerate}
    \item Show that this congruence is geodesic, i.e., that \( u^\alpha \) satisfies the geodesic equation.
    \item Calculate the expansion, shear, and rotation associated with this congruence.
    \item Using the Raychaudhuri equation, deduce the following expression:
    \begin{equation}
    \frac{\ddot{a}}{a} = -\frac{4 \pi}{3} (\rho + 3 p),
    \end{equation}
    where \( \rho \) is the energy density and \( p \) is the pressure of a perfect fluid with four-velocity \( u^\alpha \).
\end{enumerate}


\end{document}