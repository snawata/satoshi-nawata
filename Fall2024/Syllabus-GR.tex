\documentclass[12pt,a4paper]{article}
\usepackage{hyperref} % Use the Charter font for the document text
%\usepackage[UTF8]{ctex}
\usepackage{fullpage}
\usepackage{color}

\usepackage{graphicx}


\begin{document}\thispagestyle{empty}

\centerline{\Large \bf Syllabus}

\begin{description}
\item{\bf Course name:} Advanced Course on Gravity (PHYS)
\item{\bf Lecturer:} Satoshi Nawata, Physics S422, Jiangwan, \href{mailto:snawata@fudan.edu.cn}{snawata@fudan.edu.cn}
\item{\bf Teaching Assistant:} Zixiao Huang
\item{\bf Hours:} 9:55 -- 12:30, Friday
\item{\bf Place:}  JB202
\item{\bf Office hour:} Whenever
\item{\bf Prerequisites:}

Quantum Mechanics, Electrodynamics, \\
Taking General Relativity is desirable but not necessary

\item{\bf About the course:}

My courses are well-known for being notoriously difficult, challenging, and mathematically intensive. However, I know that Chinese students prefer step-by-step derivations and a more down-to-earth approach. Therefore, this time, I aim to make the course enjoyable for students by emphasizing the physical aspects of spacetime structures, focusing on black hole physics. Homework sets will be given every other week and will be less difficult.

This course offers an in-depth exploration of black hole physics, extending beyond the classical Schwarzschild solutions to include Kerr, Reissner-Nordström, and Kerr-Newman black holes. Students will study intricate structures of spacetime, such as Penrose diagrams, event horizons, and singularities. The course also covers the ADM energy and the positive energy theorem. A significant portion of the course is dedicated to black hole thermodynamics, including the laws of entropy, temperature, and radiation. Students will be introduced to the basics of quantum field theory in curved spacetime, the Unruh effect, and particle creation. The course concludes with an examination of Hawking radiation, black hole evaporation, and the information paradox.

All lectures will be recorded and made available online. Students are welcome to audit the course.




\item{\bf Main references:} 
\begin{itemize}
    \item \textit{General Relativity} by Robert M. Wald. University of Chicago Press, 1984. ISBN: 978-0-226-87033-5.
    
        \item \textit{Black Holes: Lecture Notes} by Townsend, 1997. Available at: \href{https://arxiv.org/abs/gr-qc/9707012}{gr-qc/9707012}.

    \item \textit{Quantum Field Theory in Curved Spacetime} by Leonard Parker. Cambridge University Press, 2009. ISBN: 9780511813924.
        
    \item \textit{Quantum Fields in Curved Space} by N.D. Birrell and P.C.W. Davies. Cambridge University Press, 1982. ISBN: 97805116226.
    
    \item \textit{Quantum field theory in curved spacetime and black hole thermodynamics} by Robert M. Wald,  University of Chicago Press (1994)
    
        \item \textit{Spacetime and Geometry: An Introduction to General Relativity} by Sean Carroll. Benjamin Cummings, 2004.
\end{itemize}



\item{\bf Grading scheme:}  Homework sets will be assigned every other week, with a maximum of six sets throughout the course.

\end{description}



\end{document}
