\documentclass[12pt,a4paper]{article}
%\usepackage{hyperref} % Use the Charter font for the document text
%\usepackage[UTF8]{ctex}


\usepackage{amsfonts,amssymb,amsmath}
\usepackage{mathtools}
\usepackage{tikz-cd}
\usepackage{fullpage}
\usepackage{tikz}
\usepackage{alltt}
\usepackage{amsfonts}
\usepackage{amsmath}
\usepackage{amssymb}
\usepackage{amsthm}
\usepackage{booktabs}
\usepackage{caption}
\usepackage{enumitem}
\usepackage{fancyhdr}
\usepackage{graphicx}
% \usepackage{mathdots}
\usepackage{mathtools}
\usepackage{microtype}
\usepackage{multirow}
\usepackage{pdflscape}
\usepackage{pgfplots}
\usepackage{siunitx}
\usepackage{slashed}
\usepackage{tabularx}
\usepackage{tikz}
\usepackage{tkz-euclide}
% \usepackage[normalem]{ulem}
\usepackage[all]{xy}
\usepackage{imakeidx}
\usepackage{mathtools}
\usepackage{wrapfig}
\usepackage{youngtab}



%%%%%%%  Greek letters %%%%%%%%%%%%%%%%%%
\def\a{\alpha}
\def\b{\beta}
\def\c{\gamma} \def\g{\gamma}
\def\d{\delta}
\def\e{\epsilon}
\def\f{\phi}
\def\vf{\varphi}  \def\tvf{\tilde{\varphi}}
\def\vp{\varphi}
\def\h{\eta}
\def\i{\iota}
\def\j{\psi}
\def\k{\kappa}
\def\m{\mu}
\def\n{\nu}
\def\o{\omega}  \def\w{\omega}
\def\q{\theta}  \def\th{\theta}
\def\r{\rho}
\def\s{\sigma}
\def\t{\tau}
\def\u{\upsilon}
\def\x{\xi}
\def\z{\zeta}

\def\A{\Alpha}
\def\B{\Beta}
\def\G{\Gamma}
\def\D{\Delta}
\def\E{\Epsilon}
\def\F{Phi}
\def\h{\eta}
\def\I{\Iota}
\def\J{Psi}
\def\K{\Kappa}
\def\L{\Lambda}
\def\M{\Mu}
\def\N{\Nu}
\def\O{\Omega}  \def\w{\omega}
\def\Q{\Theta}  \def\Th{\Theta}
\def\R{\Rho}
\def\Si{\Sigma}
\def\T{\Tau}
\def\Up{\Upsilon}
\def\X{\Xi}
\def\Z{\Zeta}








%%%%%%%%%%%% math fonts %%%%%%%%%%%%%%%%%%%%%%%%%%%%%%%%%%%%%
%
%---------- mathbb font --------------------------------
%

\newcommand{\bA}{\ensuremath{\mathbb{A}}}
\newcommand{\bB}{\ensuremath{\mathbb{B}}}
\newcommand{\bC}{\ensuremath{\mathbb{C}}}
\newcommand{\bD}{\ensuremath{\mathbb{D}}}
\newcommand{\bE}{\ensuremath{\mathbb{E}}}
\newcommand{\bF}{\ensuremath{\mathbb{F}}}
\newcommand{\bG}{\ensuremath{\mathbb{G}}}
\newcommand{\bH}{\ensuremath{\mathbb{H}}}
\newcommand{\bI}{\ensuremath{\mathbb{I}}}
\newcommand{\bJ}{\ensuremath{\mathbb{J}}}
\newcommand{\bK}{\ensuremath{\mathbb{K}}}
\newcommand{\bL}{\ensuremath{\mathbb{L}}}
\newcommand{\bM}{\ensuremath{\mathbb{M}}}
\newcommand{\bN}{\ensuremath{\mathbb{N}}}
\newcommand{\bO}{\ensuremath{\mathbb{O}}}
\newcommand{\bP}{\ensuremath{\mathbb{P}}}
\newcommand{\bQ}{\ensuremath{\mathbb{Q}}}
\newcommand{\bR}{\ensuremath{\mathbb{R}}}
\newcommand{\bS}{\ensuremath{\mathbb{S}}}
\newcommand{\bT}{\ensuremath{\mathbb{T}}}
\newcommand{\bU}{\ensuremath{\mathbb{U}}}
\newcommand{\bV}{\ensuremath{\mathbb{V}}}
\newcommand{\bW}{\ensuremath{\mathbb{W}}}
\newcommand{\bX}{\ensuremath{\mathbb{X}}}
\newcommand{\bY}{\ensuremath{\mathbb{Y}}}
\newcommand{\bZ}{\ensuremath{\mathbb{Z}}}


%
%---------- mathbf font --------------------------------
%


\newcommand{\bfA}{\ensuremath{\mathbf{A}}}
\newcommand{\bfB}{\ensuremath{\mathbf{B}}}
\newcommand{\bfC}{\ensuremath{\mathbf{C}}}
\newcommand{\bfD}{\ensuremath{\mathbf{D}}}
\newcommand{\bfE}{\ensuremath{\mathbf{E}}}
\newcommand{\bfF}{\ensuremath{\mathbf{F}}}
\newcommand{\bfG}{\ensuremath{\mathbf{G}}}
\newcommand{\bfH}{\ensuremath{\mathbf{H}}}
\newcommand{\bfI}{\ensuremath{\mathbf{I}}}
\newcommand{\bfJ}{\ensuremath{\mathbf{J}}}
\newcommand{\bfK}{\ensuremath{\mathbf{K}}}
\newcommand{\bfL}{\ensuremath{\mathbf{L}}}
\newcommand{\bfM}{\ensuremath{\mathbf{M}}}
\newcommand{\bfN}{\ensuremath{\mathbf{N}}}
\newcommand{\bfO}{\ensuremath{\mathbf{O}}}
\newcommand{\bfP}{\ensuremath{\mathbf{P}}}
\newcommand{\bfQ}{\ensuremath{\mathbf{Q}}}
\newcommand{\bfR}{\ensuremath{\mathbf{R}}}
\newcommand{\bfS}{\ensuremath{\mathbf{S}}}
\newcommand{\bfT}{\ensuremath{\mathbf{T}}}
\newcommand{\bfU}{\ensuremath{\mathbf{U}}}
\newcommand{\bfV}{\ensuremath{\mathbf{V}}}
\newcommand{\bfW}{\ensuremath{\mathbf{W}}}
\newcommand{\bfX}{\ensuremath{\mathbf{X}}}
\newcommand{\bfY}{\ensuremath{\mathbf{Y}}}
\newcommand{\bfZ}{\ensuremath{\mathbf{Z}}}
\newcommand{\bfa}{\ensuremath{\mathbf{a}}}
\newcommand{\bfb}{\ensuremath{\mathbf{b}}}
\newcommand{\bfc}{\ensuremath{\mathbf{c}}}
\newcommand{\bfd}{\ensuremath{\mathbf{d}}}
\newcommand{\bfe}{\ensuremath{\mathbf{e}}}
\newcommand{\bff}{\ensuremath{\mathbf{f}}}
\newcommand{\bfg}{\ensuremath{\mathbf{g}}}
\newcommand{\bfh}{\ensuremath{\mathbf{h}}}
\newcommand{\bfi}{\ensuremath{\mathbf{i}}}
\newcommand{\bfj}{\ensuremath{\mathbf{j}}}
\newcommand{\bfk}{\ensuremath{\mathbf{k}}}
\newcommand{\bfl}{\ensuremath{\mathbf{l}}}
\newcommand{\bfm}{\ensuremath{\mathbf{m}}}
\newcommand{\bfn}{\ensuremath{\mathbf{n}}}
\newcommand{\bfo}{\ensuremath{\mathbf{o}}}
\newcommand{\bfp}{\ensuremath{\mathbf{p}}}
\newcommand{\bfq}{\ensuremath{\mathbf{q}}}
\newcommand{\bfr}{\ensuremath{\mathbf{r}}}
\newcommand{\bfs}{\ensuremath{\mathbf{s}}}
\newcommand{\bft}{\ensuremath{\mathbf{t}}}
\newcommand{\bfu}{\ensuremath{\mathbf{u}}}
\newcommand{\bfv}{\ensuremath{\mathbf{v}}}
\newcommand{\bfw}{\ensuremath{\mathbf{w}}}
\newcommand{\bfx}{\ensuremath{\mathbf{x}}}
\newcommand{\bfy}{\ensuremath{\mathbf{y}}}
\newcommand{\bfz}{\ensuremath{\mathbf{z}}}



%
%\parskip=1em
%\parindent=0.3in
%\setlength\oddsidemargin{0.5in} \setlength\evensidemargin{0.5in}
%\setlength\textwidth{5.5in}
%
%\hfuzz6pt % Don't bother to report over-full boxes if over-edge is < 6pt
%
%\newlength{\defbaselineskip}
%\setlength{\defbaselineskip}{\baselineskip}
%\newcommand{\setlinespacing}[1]%
%           {\setlength{\baselineskip}{#1 \defbaselineskip}}
%\newcommand{\doublespacing}{\setlength{\baselineskip}%
%                           {2.0 \defbaselineskip}}
%\newcommand{\singlespacing}{\setlength{\baselineskip}{\defbaselineskip}}
%
%\newcommand{\properpagestyle}{\pagestyle{myheadings}\markboth{}{}\markright{}}


%---------- mathscript font -----------------------------
%

\newcommand{\scA}{\ensuremath{\mathscr{A}}}
\newcommand{\scB}{\ensuremath{\mathscr{B}}}
\newcommand{\scC}{\ensuremath{\mathscr{C}}}
\newcommand{\scD}{\ensuremath{\mathscr{D}}}
\newcommand{\scE}{\ensuremath{\mathscr{E}}}
\newcommand{\scF}{\ensuremath{\mathscr{F}}}
\newcommand{\scG}{\ensuremath{\mathscr{G}}}
\newcommand{\scH}{\ensuremath{\mathscr{H}}}
\newcommand{\scI}{\ensuremath{\mathscr{I}}}
\newcommand{\scJ}{\ensuremath{\mathscr{J}}}
\newcommand{\scK}{\ensuremath{\mathscr{K}}}
\newcommand{\scL}{\ensuremath{\mathscr{L}}}
\newcommand{\scM}{\ensuremath{\mathscr{M}}}
\newcommand{\scN}{\ensuremath{\mathscr{N}}}
\newcommand{\scO}{\ensuremath{\mathscr{O}}}
\newcommand{\scP}{\ensuremath{\mathscr{P}}}
\newcommand{\scQ}{\ensuremath{\mathscr{Q}}}
\newcommand{\scR}{\ensuremath{\mathscr{R}}}
\newcommand{\scS}{\ensuremath{\mathscr{S}}}
\newcommand{\scT}{\ensuremath{\mathscr{T}}}
\newcommand{\scU}{\ensuremath{\mathscr{U}}}
\newcommand{\scV}{\ensuremath{\mathscr{V}}}
\newcommand{\scW}{\ensuremath{\mathscr{W}}}
\newcommand{\scX}{\ensuremath{\mathscr{X}}}
\newcommand{\scY}{\ensuremath{\mathscr{Y}}}
\newcommand{\scZ}{\ensuremath{\mathscr{Z}}}
\newcommand{\scAH}{\ensuremath{\mathscr{A}\!\!\scH}}

%
%---------- mathfrak font -----------------------------
%

\newcommand{\frakA}{\ensuremath{\mathfrak{A}}}
\newcommand{\frakB}{\ensuremath{\mathfrak{B}}}
\newcommand{\frakC}{\ensuremath{\mathfrak{C}}}
\newcommand{\frakD}{\ensuremath{\mathfrak{D}}}
\newcommand{\frakE}{\ensuremath{\mathfrak{E}}}
\newcommand{\frakF}{\ensuremath{\mathfrak{F}}}
\newcommand{\frakG}{\ensuremath{\mathfrak{G}}}
\newcommand{\frakH}{\ensuremath{\mathfrak{H}}}
\newcommand{\frakI}{\ensuremath{\mathfrak{I}}}
\newcommand{\frakJ}{\ensuremath{\mathfrak{J}}}
\newcommand{\frakK}{\ensuremath{\mathfrak{K}}}
\newcommand{\frakL}{\ensuremath{\mathfrak{L}}}
\newcommand{\frakM}{\ensuremath{\mathfrak{M}}}
\newcommand{\frakN}{\ensuremath{\mathfrak{N}}}
\newcommand{\frakO}{\ensuremath{\mathfrak{O}}}
\newcommand{\frakP}{\ensuremath{\mathfrak{P}}}
\newcommand{\frakQ}{\ensuremath{\mathfrak{Q}}}
\newcommand{\frakR}{\ensuremath{\mathfrak{R}}}
\newcommand{\frakS}{\ensuremath{\mathfrak{S}}}
\newcommand{\frakT}{\ensuremath{\mathfrak{T}}}
\newcommand{\frakU}{\ensuremath{\mathfrak{U}}}
\newcommand{\frakV}{\ensuremath{\mathfrak{V}}}
\newcommand{\frakW}{\ensuremath{\mathfrak{W}}}
\newcommand{\frakX}{\ensuremath{\mathfrak{X}}}
\newcommand{\frakY}{\ensuremath{\mathfrak{Y}}}
\newcommand{\frakZ}{\ensuremath{\mathfrak{Z}}}
\newcommand{\fraka}{\ensuremath{\mathfrak{a}}}
\newcommand{\frakb}{\ensuremath{\mathfrak{b}}}
\newcommand{\frakc}{\ensuremath{\mathfrak{c}}}
\newcommand{\frakd}{\ensuremath{\mathfrak{d}}}
\newcommand{\frake}{\ensuremath{\mathfrak{e}}}
\newcommand{\frakf}{\ensuremath{\mathfrak{f}}}
\newcommand{\frakg}{\ensuremath{\mathfrak{g}}}
\newcommand{\frakh}{\ensuremath{\mathfrak{h}}}
\newcommand{\fraki}{\ensuremath{\mathfrak{i}}}
\newcommand{\frakj}{\ensuremath{\mathfrak{j}}}
\newcommand{\frakk}{\ensuremath{\mathfrak{k}}}
\newcommand{\frakl}{\ensuremath{\mathfrak{l}}}
\newcommand{\frakm}{\ensuremath{\mathfrak{m}}}
\newcommand{\frakn}{\ensuremath{\mathfrak{n}}}
\newcommand{\frako}{\ensuremath{\mathfrak{o}}}
\newcommand{\frakp}{\ensuremath{\mathfrak{p}}}
\newcommand{\frakq}{\ensuremath{\mathfrak{q}}}
\newcommand{\frakr}{\ensuremath{\mathfrak{r}}}
\newcommand{\fraks}{\ensuremath{\mathfrak{s}}}
\newcommand{\frakt}{\ensuremath{\mathfrak{t}}}
\newcommand{\fraku}{\ensuremath{\mathfrak{u}}}
\newcommand{\frakv}{\ensuremath{\mathfrak{v}}}
\newcommand{\frakw}{\ensuremath{\mathfrak{w}}}
\newcommand{\frakx}{\ensuremath{\mathfrak{x}}}
\newcommand{\fraky}{\ensuremath{\mathfrak{y}}}
\newcommand{\frakz}{\ensuremath{\mathfrak{z}}}
\newcommand{\fraksl}{\ensuremath{\mathfrak{sl}}}
\newcommand{\frakso}{\ensuremath{\mathfrak{so}}}
\newcommand{\fraksp}{\ensuremath{\mathfrak{sp}}}

%%%%%%%%%%%%  Calligraphic, Roman and Maths integers %%%%%%%%%%%%%%%%%%

\newcommand{\cA}{\mathcal{A}}
\newcommand{\cB}{\mathcal{B}}
\newcommand{\cC}{\mathcal{C}}
\newcommand{\cD}{\mathcal{D}}
\newcommand{\cE}{\mathcal{E}}
\newcommand{\cF}{\mathcal{F}}
\newcommand{\cG}{\mathcal{G}}
\newcommand{\cH}{\mathcal{H}}
\newcommand{\cI}{\mathcal{I}}
\newcommand{\cJ}{\mathcal{J}}
\newcommand{\cK}{\mathcal{K}}
\newcommand{\cL}{\mathcal{L}}
\newcommand{\cM}{\mathcal{M}}
\newcommand{\cN}{\mathcal{N}}
\newcommand{\cO}{\mathcal{O}}
\newcommand{\cQ}{\mathcal{Q}}
\newcommand{\cS}{\mathcal{S}}
\newcommand{\cX}{\mathcal{X}}
\newcommand{\cY}{\mathcal{Y}}
\newcommand{\cW}{\mathcal{W}}
\newcommand{\cR}{\mathcal{R}}
\newcommand{\cT}{\mathcal{T}}
\newcommand{\cZ}{\mathcal{Z}}

%%%%%%%%%%%%%%%%%%%%%%%%%%%%%%%%%%%%%%%%%%%%%%%%%%%%%%%%%%%%%%%%
\newcommand{\SU}{\mathrm{SU}}
\newcommand{\SO}{\mathrm{SO}}
\newcommand{\SL}{\mathrm{SL}}
\newcommand{\Sp}{\mathrm{Sp}}
\newcommand{\su}{\mathrm{su}}
\newcommand{\so}{\mathrm{so}}
\newcommand{\spl}{\mathrm{sp}}
\newcommand{\gl}{\mathrm{gl}}
\newcommand{\sll}{\mathrm{sl}}
\newcommand{\U}{\mathrm{U}}
\newcommand{\ul}{\mathrm{u}}
\newcommand{\Spin}{\mathrm{Spin}}
\newcommand{\Pin}{\mathrm{Pin}}
%%%%%%%%%%%%%%%%%%%%%%%%%%%%%%%%%%%%%%%%%%%%%%%%%%%%%%%%%%%%%%%%
\renewcommand{\Im}{{\rm Im}}
\renewcommand{\Re}{{\rm Re}}
\newcommand{\Tr}{\mbox{Tr}}
\newcommand{\Pf}{\mbox{Pf}}
\newcommand{\sgn}{\mbox{sgn}}
\newcommand{\Vir}{{\rm Vir}}
\newcommand{\Li}{{\rm Li}}

\def\tl{\tilde}
\def\wt{\widetilde}
\def\wh{\widehat}
\def\bar{\overline}


\def\ap{{\alpha^\prime}}
\def\bz{\bar{z}}


\def \bea {\begin{equation}\begin{aligned}}
\def \eea {\end{aligned}\end{equation}}
\def\be{\begin{equation}}
\def\ee{\end{equation}}
\def\ba{\begin{align}}
\def\ea{\end{align}}


\usepackage{graphicx}

\begin{document}\thispagestyle{empty}
\Yboxdim7pt


\centerline{\Large \bf Homework 11: Due at class on Dec 7}


\section{Modular transformations of $\vartheta$ and $\eta$-functions}

Although we have been using modular transformations of the theta and eta functions,
we have not proven those properties yet. Here, let us achieve this from ``boson-fermion correspondence'', which is confirmed in homework 4.


\begin{enumerate}

 \item 
As we saw in the homework 4, a vertex operator, $V_\alpha =\; :\!e^{i\alpha\varphi}\!:$, plays an important role.
In order for the vertex operator to be self-consistent the scalar $\varphi$ should be compactified, and
let us assume that the period of $\varphi$ is $2\pi R$.
Derive possible values of $\alpha$.
Using T-duality (i.e. $R \leftrightarrow \frac{2}{R}$) confirm that $R = \sqrt{2k}$ with $k=2$ is the correct value for the correspondence.

 \item
Free complex fermion system has following non-trivial commutation relations:
\begin{align}
 &\{\psi_r,\bar\psi_s\} = \delta_{r+s,0}, \qquad [J_m,J_n] = m\delta_{m+n,0} , \qquad [L_m,J_n] = -nJ_{m+n} , \nonumber\\
 &[J_m,\psi_s] = +\psi_{m+s}, \qquad [J_m,\bar\psi_s] = -\bar\psi_{m+s}, \qquad [L_m, \psi_r] = \left(-\frac{m}{2} -r\right)\psi_{m+r} ,
\end{align}
where $\psi_r,\ \bar\psi_s$ are modes of complex fermion $\psi,\ \bar\psi$, $J_m$ are modes of $U(1)$ current $J=\psi\bar\psi$, 
we consider NS sector and $r,s$ are half-integers.
Since $L_0$ and $J_0$ commute each other they can simultaneously have eigenvalues.
Derive those eigenvalues for the eigenstates of $\psi$: 
$| n_{1/2},n_{3/2},\ldots\rangle = (\psi_{-\frac{1}{2}})^{n_{1/2}} (\psi_{-\frac{3}{2}})^{n_{3/2}} \cdots |0\rangle$.
Calculate the holomorphic part of partition function (character):
\begin{align}
 \chi(\tau,z) = \Tr \left( q^{L_0 -\frac{1}{24}} y^{J_0} \right) , \qquad q = e^{2\pi i \tau}, \quad y= e^{2\pi i z}.
\end{align}
The result is
\begin{align}
 \chi(\tau,z) = q^{-\frac{1}{24}} \prod_{n=1}^\infty (1+q^{n-\frac{1}{2}}y) (1+q^{n-\frac{1}{2}}y^{-1}).
\end{align} 

 \item
Now, we compute the same character in terms of  boson.
Eigenstates are given by
\begin{align}
 | \alpha, n_1, n_2, \ldots \rangle = \lim_{z,\bar z \to 0} J_{-1}^{n_1} aJ_{-2}^{n_2} \cdots V_\alpha (z,\bar z) |0\rangle 
\end{align}
where $J_n=a_n$ are the modes of the free boson in this case. 
Derive eigenvalues of $L_0$ and $J_0$ for the states, where the eigenvalues of $V_\alpha$ are given by $(L_0,J_0) = (\frac{\alpha^2}{2},\alpha)$ 
(see homework 3). Calculate the character. The result is
\begin{align}
 \chi(\tau,z) = \frac{1}{\eta (\tau)} \sum_{n \in \mathbb Z} q^{\frac{n^2}{2}}y^n.
\end{align}

 \item
Using the last two results we have
\begin{align}
  q^{-\frac{1}{24}} \prod_{n=1}^\infty (1+q^{n-\frac{1}{2}}y) (1+q^{n-\frac{1}{2}}y^{-1}) = \frac{1}{\eta (\tau)} \sum_{n \in \mathbb Z} q^{\frac{n^2}{2}}y^n.
\end{align}
This is called the Jacobi triple product identity (triple-ness can be seen if you move the $\eta(\tau)$ to the LHS ).
Using this identity express theta functions ($\vartheta_2(\tau),\ \vartheta_3(\tau),\ \vartheta_4(\tau),\ $)
in terms of sum (not product), where the definitions of the theta functions are given in Sec.~4.3 of lecture note.
Derive the S- and T-transformations of the theta functions (you may use Poisson resummation formula).

 \item
Show that
\begin{align}
 \sqrt{\frac{\vartheta_2(\tau) \vartheta_3(\tau) \vartheta_4(\tau)}{2\eta(\tau)^3}} = 1 ,
\end{align}
and derive the S- and T-transformations of $\eta(\tau)$.

 \item
Finally, confirm that the equality of a free boson on a circle with $R=1 \text{ or } 2$ is equivalent to free complex fermion by
comparing those torus partition functions, namely, show that
\begin{align}
 \sum_{-1 \le m \le 2} \left| \frac{\Theta_{m,2} (\tau)}{\eta(\tau)} \right|^2
 = \frac{1}{2} \left\{ \left|\frac{\vartheta_2(\tau)}{\eta(\tau)}\right|^2 +\left|\frac{\vartheta_3(\tau)}{\eta(\tau)}\right|^2
 +\left|\frac{\vartheta_4(\tau)}{\eta(\tau)}\right|^2  \right\} ,
\end{align}
where
\begin{align}
 \Theta_{m,k} (\tau) = \sum_{n \in \mathbb Z} q^{k\left(n+\frac{m}{2k}\right)^2} .
\end{align}


\end{enumerate}

\section{Verlinde algebra of Ising model}

There are finitely many primary fields in a rational conformal field theory like a minimal model and WZW model. The fusion rule of primary fields are closed under themselves
$$
\left[ \phi_{ i } \right] \times \left[ \phi_{ j } \right] = \sum _ { k } \mathcal { N } _ { i j } ^ { k } \left[ \phi_{ k } \right]~,
$$
where ${\cN } _ { i j } ^ { k } $ are called the fusion coefficients. As we have seen in  the minimal models and WZW models, one can construct a highest weight representation associated to a primary field $\phi_i$, and we write the corresponding character by $\chi_i(\tau)$. The modular transformations are 
$$
\chi_i(-1/\tau)=\sum_{j}S_{ij}\,\chi_j(\tau)~, \qquad \chi_i(\tau+1)=\sum_{j}T_{ij}\,\chi_j(\tau)~.
$$
E. Verlinde has found the remarkable relationship between fusion rule and the modular $S$-matrices \cite{verlinde1988fusion}
\be\label{Verlinde}
\mathcal { N } _ { i j } ^ { k } = \sum _ { l } \frac { S _ { j l } S _ { i l } \left( S ^ { - 1 } \right) _ { l k } } { S _ { 0 l } }~.
\ee


\subsection{Ising model revisited}
Let us recall that the Ising model is the unitary minimal model $\cM_{p=3}$ where the primary fields are associated to 
$$
\begin{array} { r l } { 1 } & { \Leftrightarrow \phi_{ 1 } : = \phi_{ 1,1 } } \\ { \epsilon } & { \Leftrightarrow \phi_{ 2 } : = \phi_{ 2,1 } } \\ { \sigma } & { \Leftrightarrow \phi_{ 3 } : = \phi_{ 2,2 } } \end{array}
$$
where $\e$ is the energy density filed and $\sigma$ is the spin field. The corresponding characters are given in (4.111) of the lecture note
$$
\begin{aligned} \chi _ {1} ( \tau ):= \chi _ { 0 } ( \tau ) & = \frac { 1 } { 2 } \left( \sqrt { \frac { \vartheta_{ 3 } ( \tau ) } { \eta ( \tau ) } } + \sqrt { \frac { \vartheta_{ 4 } ( \tau ) } { \eta ( \tau ) } } \right) \\ \chi _ {2} ( \tau ):= \chi _ { \frac { 1 } { 2 } } ( \tau ) & = \frac { 1 } { 2 } \left( \sqrt { \frac { \vartheta_{ 3 } ( \tau ) } { \eta ( \tau ) } } - \sqrt { \frac { \vartheta_{ 4 } ( \tau ) } { \eta ( \tau ) } } \right) \\ \chi _ {3} ( \tau ):= \chi _ { \frac { 1 } { 16 } } ( \tau ) & = \frac { 1 } { \sqrt { 2 } } \sqrt { \frac { \vartheta_{ 2 } ( \tau ) } { \eta ( \tau ) } } \end{aligned}
$$
Using the properties of the $\vartheta$ and $\eta$-functions, find the $3\times 3$ $S$-matrix of the Ising model.  Then,  compute the three $3\times 3$ fusion matrices $\mathcal { N } _ { i j } ^ { k }$ ($i=1,2,3$) by using the Verlinde formula \eqref{Verlinde}. Check they reproduce the fusion rule at the end of  \S5.5 of the lecture note.






\bibliography{conformal-ref}
\bibliographystyle{hyperamsalpha}




\end{document}
