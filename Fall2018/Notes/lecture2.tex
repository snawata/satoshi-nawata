%%Line 1810 uncomment

\documentclass[2dCFT-lecture.tex]{subfiles}
%\usepackage{subfiles}
%\usepackage{epsfig}
%\usepackage{amsmath}
%\usepackage{amssymb}
%\usepackage{amsthm}
%\usepackage{indentfirst}
%\usepackage{xspace}
%\usepackage{multirow}
%\usepackage{hyperref}
%\usepackage{xcolor}
%\usepackage{verbatim}
%%\hypersetup{colorlinks=true,urlcolor=darkred,linkcolor=darkred,citecolor=darkred}
%%\usepackage{verbatim}
%\usepackage[letterpaper,margin=0.9in,headheight=15pt]{geometry}
%\usepackage{mathpazo}
%\usepackage{authblk}
%\usepackage{empheq}
%\usepackage{feynmp}
%\usepackage{graphicx}
%\usepackage[matrix,arrow]{xy}
%\usepackage{young}
%\usepackage[vcentermath]{youngtab}
%\usepackage{slashed}
%%\usepackage{fontds}
%%
%\usepackage{bbm}
%\usepackage{youngtab}
%\usepackage{rotfloat}
%\usepackage{stmaryrd}
%\usepackage{amsfonts,amssymb,amsmath}
%\usepackage{tikz-cd}
%\usepackage{thmtools}
%\usepackage{dashrule}
%\usepackage[missing=]{gitinfo2}
%\usepackage{fancyhdr}
%\usepackage{mdframed}
%
%\usepackage{subfiles}



%
%
%\definecolor{darkblue}{rgb}{0.1,0.1,0.7}
%\definecolor{darkred}{rgb}{0.5,0.1,0.1}
%\definecolor{darkgreen}{rgb}{0.0,0.42,0.06}
%\hypersetup{colorlinks=true,urlcolor=darkred,linkcolor=darkblue,citecolor=darkred}
%\definecolor{shadecolor}{rgb}{0.85,0.85,0.85}
%
%
%
%% theorem environments
%\declaretheoremstyle[spaceabove=0.25cm,spacebelow=0.25cm,notefont=\normalfont\bfseries, notebraces={(}{)}]{theorem}
%\declaretheoremstyle[spaceabove=0.25cm,spacebelow=0.25cm,bodyfont=\normalfont,notefont=\normalfont\bfseries, notebraces={(}{)}]{noital}
%\declaretheoremstyle[spaceabove=0.25cm,spacebelow=0.25cm,bodyfont=\normalfont\color{darkgreen},notefont=\normalfont\bfseries, notebraces={(}{)}]{green}
%\declaretheoremstyle[spaceabove=0.25cm,spacebelow=0.25cm,bodyfont=\normalfont,notefont=\normalfont\bfseries,qed=$\qedsymbol$,notebraces={(}{)}]{proofstyle}
%
%\declaretheorem[name=Theorem,numberwithin=section,style=theorem]{thm}
%\declaretheorem[name=Proposition,sibling=thm,style=theorem]{prop}
%\declaretheorem[name=Corollary,sibling=thm,style=theorem]{cor}
%\declaretheorem[name=Lemma,sibling=thm,style=theorem]{lem}
%\declaretheorem[name=Remark,style=theorem,numbered=no]{rem}
%\declaretheorem[name=Definition,sibling=thm,style=noital]{defn}
%\declaretheorem[name=Example,sibling=thm,style=noital]{example}
%\declaretheorem[name=Exercise,numberwithin=section,style=green]{exercise}
%\declaretheorem[name=Proof,style=proofstyle,numbered=no]{pf}
%
%\numberwithin{equation}{section}
%
%
%\usepackage[most,listings]{tcolorbox}
%\tcbuselibrary{breakable}
%
%\tcbset{colframe=white}
%
%\def \bthm {\begin{tcolorbox}[breakable,boxsep=5pt,left=0pt,right=0pt,top=0pt,bottom=0pt]\begin{thm}}
%\def \ethm {\end{thm}\end{tcolorbox}}
%\def \bprop {\begin{tcolorbox}[breakable,boxsep=5pt,left=0pt,right=0pt,top=0pt,bottom=0pt]\begin{prop}}
%\def \eprop {\end{prop}\end{tcolorbox}}
%\def \blem {\begin{tcolorbox}[breakable,boxsep=5pt,left=0pt,right=0pt,top=0pt,bottom=0pt]\begin{lem}}
%\def \elem {\end{lem}\end{tcolorbox}}
%\def \bcor {\begin{tcolorbox}[breakable,boxsep=5pt,left=0pt,right=0pt,top=0pt,bottom=0pt]\begin{cor}}
%\def \ecor {\end{cor}\end{tcolorbox}}
%\def \bdefn {\begin{tcolorbox}[breakable,boxsep=5pt,left=0pt,right=0pt,top=0pt,bottom=0pt]\begin{defn}}
%\def \edefn {\end{defn}\end{tcolorbox}}
%\def \bexample {\begin{tcolorbox}[colback=blue!2,breakable,boxsep=5pt,left=0pt,right=0pt,top=0pt,bottom=0pt]\begin{example}}
%\def \eexample {\end{example}\end{tcolorbox}}
%\def \brem {\begin{tcolorbox}[colback=blue!2,breakable,boxsep=5pt,left=0pt,right=0pt,top=0pt,bottom=0pt]\begin{rem}}
%\def \erem {\end{rem}\end{tcolorbox}}
%
%
%%%%%%%%  Greek letters %%%%%%%%%%%%%%%%%%
%\def\a{\alpha}
%\def\b{\beta}
%\def\c{\gamma} \def\g{\gamma}
%\def\d{\delta}
%\def\e{\epsilon}          
%\def\f{\phi}              
%\def\vf{\varphi}  \def\tvf{\tilde{\varphi}}
%\def\vp{\varphi}
%\def\h{\eta}
%\def\i{\iota}
%\def\j{\psi}
%\def\k{\kappa}    
%\def\m{\mu}
%\def\n{\nu}
%\def\o{\omega}  \def\w{\omega}
%\def\q{\theta}  \def\th{\theta}                  
%\def\r{\rho}                                     
%\def\s{\sigma}                                  
%\def\t{\tau}
%\def\u{\upsilon}
%\def\x{\xi}
%\def\z{\zeta} 
%
%\def\A{\Alpha}
%\def\B{\Beta}
%\def\G{\Gamma}
%\def\D{\Delta}
%\def\E{\Epsilon}         
%\def\F{Phi}          
%\def\h{\eta}
%\def\I{\Iota}
%\def\J{Psi}
%%\def\K{\Kappa}                    
%\def\L{\Lambda}
%\def\M{\Mu}
%\def\N{\Nu}
%\def\O{\Omega}  \def\w{\omega}
%\def\Q{\Theta}  \def\Th{\Theta}                  
%\def\R{\Rho}                                    
%\def\Si{\Sigma}                                   
%\def\T{\Tau}
%\def\Up{\Upsilon}
%\def\X{\Xi}
%\def\Z{\Zeta}
%
%
%
%
%%%%%%%%%%%%% math fonts %%%%%%%%%%%%%%%%%%%%%%%%%%%%%%%%%%%%%
%%
%%---------- mathbb font --------------------------------
%%
%
%\newcommand{\bA}{\ensuremath{\mathbb{A}}}
%\newcommand{\bB}{\ensuremath{\mathbb{B}}}
%\newcommand{\bC}{\ensuremath{\mathbb{C}}}
%\newcommand{\bD}{\ensuremath{\mathbb{D}}}
%\newcommand{\bE}{\ensuremath{\mathbb{E}}}
%\newcommand{\bF}{\ensuremath{\mathbb{F}}}
%\newcommand{\bG}{\ensuremath{\mathbb{G}}}
%\newcommand{\bH}{\ensuremath{\mathbb{H}}}
%\newcommand{\bI}{\ensuremath{\mathbb{I}}}
%\newcommand{\bJ}{\ensuremath{\mathbb{J}}}
%\newcommand{\bK}{\ensuremath{\mathbb{K}}}
%\newcommand{\bL}{\ensuremath{\mathbb{L}}}
%\newcommand{\bM}{\ensuremath{\mathbb{M}}}
%\newcommand{\bN}{\ensuremath{\mathbb{N}}}
%\newcommand{\bO}{\ensuremath{\mathbb{O}}}
%\newcommand{\bP}{\ensuremath{\mathbb{P}}}
%\newcommand{\bQ}{\ensuremath{\mathbb{Q}}}
%\newcommand{\bR}{\ensuremath{\mathbb{R}}}
%\newcommand{\bS}{\ensuremath{\mathbb{S}}}
%\newcommand{\bT}{\ensuremath{\mathbb{T}}}
%\newcommand{\bU}{\ensuremath{\mathbb{U}}}
%\newcommand{\bV}{\ensuremath{\mathbb{V}}}
%\newcommand{\bW}{\ensuremath{\mathbb{W}}}
%\newcommand{\bX}{\ensuremath{\mathbb{X}}}
%\newcommand{\bY}{\ensuremath{\mathbb{Y}}}
%\newcommand{\bZ}{\ensuremath{\mathbb{Z}}}
%
%
%
%%
%%---------- mathbf font --------------------------------
%%
%
%
%\newcommand{\bfA}{\ensuremath{\mathbf{A}}}
%\newcommand{\bfB}{\ensuremath{\mathbf{B}}}
%\newcommand{\bfC}{\ensuremath{\mathbf{C}}}
%\newcommand{\bfD}{\ensuremath{\mathbf{D}}}
%\newcommand{\bfE}{\ensuremath{\mathbf{E}}}
%\newcommand{\bfF}{\ensuremath{\mathbf{F}}}
%\newcommand{\bfG}{\ensuremath{\mathbf{G}}}
%\newcommand{\bfH}{\ensuremath{\mathbf{H}}}
%\newcommand{\bfI}{\ensuremath{\mathbf{I}}}
%\newcommand{\bfJ}{\ensuremath{\mathbf{J}}}
%\newcommand{\bfK}{\ensuremath{\mathbf{K}}}
%\newcommand{\bfL}{\ensuremath{\mathbf{L}}}
%\newcommand{\bfM}{\ensuremath{\mathbf{M}}}
%\newcommand{\bfN}{\ensuremath{\mathbf{N}}}
%\newcommand{\bfO}{\ensuremath{\mathbf{O}}}
%\newcommand{\bfP}{\ensuremath{\mathbf{P}}}
%\newcommand{\bfQ}{\ensuremath{\mathbf{Q}}}
%\newcommand{\bfR}{\ensuremath{\mathbf{R}}}
%\newcommand{\bfS}{\ensuremath{\mathbf{S}}}
%\newcommand{\bfT}{\ensuremath{\mathbf{T}}}
%\newcommand{\bfU}{\ensuremath{\mathbf{U}}}
%\newcommand{\bfV}{\ensuremath{\mathbf{V}}}
%\newcommand{\bfW}{\ensuremath{\mathbf{W}}}
%\newcommand{\bfX}{\ensuremath{\mathbf{X}}}
%\newcommand{\bfY}{\ensuremath{\mathbf{Y}}}
%\newcommand{\bfZ}{\ensuremath{\mathbf{Z}}}
%
%
%
%
%%
%%---------- mathcal font -----------------------------
%%
%
%\newcommand{\scA}{\ensuremath{\mathscr{A}}}
%\newcommand{\scB}{\ensuremath{\mathscr{B}}}
%\newcommand{\scC}{\ensuremath{\mathscr{C}}}
%\newcommand{\scD}{\ensuremath{\mathscr{D}}}
%\newcommand{\scE}{\ensuremath{\mathscr{E}}}
%\newcommand{\scF}{\ensuremath{\mathscr{F}}}
%\newcommand{\scG}{\ensuremath{\mathscr{G}}}
%\newcommand{\scH}{\ensuremath{\mathscr{H}}}
%\newcommand{\scI}{\ensuremath{\mathscr{I}}}
%\newcommand{\scJ}{\ensuremath{\mathscr{J}}}
%\newcommand{\scK}{\ensuremath{\mathscr{K}}}
%\newcommand{\scL}{\ensuremath{\mathscr{L}}}
%\newcommand{\scM}{\ensuremath{\mathscr{M}}}
%\newcommand{\scN}{\ensuremath{\mathscr{N}}}
%\newcommand{\scO}{\ensuremath{\mathscr{O}}}
%\newcommand{\scP}{\ensuremath{\mathscr{P}}}
%\newcommand{\scQ}{\ensuremath{\mathscr{Q}}}
%\newcommand{\scR}{\ensuremath{\mathscr{R}}}
%\newcommand{\scS}{\ensuremath{\mathscr{S}}}
%\newcommand{\scT}{\ensuremath{\mathscr{T}}}
%\newcommand{\scU}{\ensuremath{\mathscr{U}}}
%\newcommand{\scV}{\ensuremath{\mathscr{V}}}
%\newcommand{\scW}{\ensuremath{\mathscr{W}}}
%\newcommand{\scX}{\ensuremath{\mathscr{X}}}
%\newcommand{\scY}{\ensuremath{\mathscr{Y}}}
%\newcommand{\scZ}{\ensuremath{\mathscr{Z}}}
%
%%
%%---------- mathfrak font -----------------------------
%%
%
%\newcommand{\frakA}{\ensuremath{\mathfrak{A}}}
%\newcommand{\frakB}{\ensuremath{\mathfrak{B}}}
%\newcommand{\frakC}{\ensuremath{\mathfrak{C}}}
%\newcommand{\frakD}{\ensuremath{\mathfrak{D}}}
%\newcommand{\frakE}{\ensuremath{\mathfrak{E}}}
%\newcommand{\frakF}{\ensuremath{\mathfrak{F}}}
%\newcommand{\frakG}{\ensuremath{\mathfrak{G}}}
%\newcommand{\frakH}{\ensuremath{\mathfrak{H}}}
%\newcommand{\frakI}{\ensuremath{\mathfrak{I}}}
%\newcommand{\frakJ}{\ensuremath{\mathfrak{J}}}
%\newcommand{\frakK}{\ensuremath{\mathfrak{K}}}
%\newcommand{\frakL}{\ensuremath{\mathfrak{L}}}
%\newcommand{\frakM}{\ensuremath{\mathfrak{M}}}
%\newcommand{\frakN}{\ensuremath{\mathfrak{N}}}
%\newcommand{\frakO}{\ensuremath{\mathfrak{O}}}
%\newcommand{\frakP}{\ensuremath{\mathfrak{P}}}
%\newcommand{\frakQ}{\ensuremath{\mathfrak{Q}}}
%\newcommand{\frakR}{\ensuremath{\mathfrak{R}}}
%\newcommand{\frakS}{\ensuremath{\mathfrak{S}}}
%\newcommand{\frakT}{\ensuremath{\mathfrak{T}}}
%\newcommand{\frakU}{\ensuremath{\mathfrak{U}}}
%\newcommand{\frakV}{\ensuremath{\mathfrak{V}}}
%\newcommand{\frakW}{\ensuremath{\mathfrak{W}}}
%\newcommand{\frakX}{\ensuremath{\mathfrak{X}}}
%\newcommand{\frakY}{\ensuremath{\mathfrak{Y}}}
%\newcommand{\frakZ}{\ensuremath{\mathfrak{Z}}}
%\newcommand{\fraka}{\ensuremath{\mathfrak{a}}}
%\newcommand{\frakb}{\ensuremath{\mathfrak{b}}}
%\newcommand{\frakc}{\ensuremath{\mathfrak{c}}}
%\newcommand{\frakd}{\ensuremath{\mathfrak{d}}}
%\newcommand{\frake}{\ensuremath{\mathfrak{e}}}
%\newcommand{\frakf}{\ensuremath{\mathfrak{f}}}
%\newcommand{\frakg}{\ensuremath{\mathfrak{g}}}
%\newcommand{\frakh}{\ensuremath{\mathfrak{h}}}
%\newcommand{\fraki}{\ensuremath{\mathfrak{i}}}
%\newcommand{\frakj}{\ensuremath{\mathfrak{j}}}
%\newcommand{\frakk}{\ensuremath{\mathfrak{k}}}
%\newcommand{\frakl}{\ensuremath{\mathfrak{l}}}
%\newcommand{\frakm}{\ensuremath{\mathfrak{m}}}
%\newcommand{\frakn}{\ensuremath{\mathfrak{n}}}
%\newcommand{\frako}{\ensuremath{\mathfrak{o}}}
%\newcommand{\frakp}{\ensuremath{\mathfrak{p}}}
%\newcommand{\frakq}{\ensuremath{\mathfrak{q}}}
%\newcommand{\frakr}{\ensuremath{\mathfrak{r}}}
%\newcommand{\fraks}{\ensuremath{\mathfrak{s}}}
%\newcommand{\frakt}{\ensuremath{\mathfrak{t}}}
%\newcommand{\fraku}{\ensuremath{\mathfrak{u}}}
%\newcommand{\frakv}{\ensuremath{\mathfrak{v}}}
%\newcommand{\frakw}{\ensuremath{\mathfrak{w}}}
%\newcommand{\frakx}{\ensuremath{\mathfrak{x}}}
%\newcommand{\fraky}{\ensuremath{\mathfrak{y}}}
%\newcommand{\frakz}{\ensuremath{\mathfrak{z}}}
%\newcommand{\fraksl}{\ensuremath{\mathfrak{sl}}}
%\newcommand{\frakso}{\ensuremath{\mathfrak{so}}}
%\newcommand{\fraksp}{\ensuremath{\mathfrak{sp}}}
%
%%%%%%%%%%%%%  Calligraphic, Roman and Maths integers %%%%%%%%%%%%%%%%%%
%
%\newcommand{\cA}{\mathcal{A}}
%\newcommand{\cB}{\mathcal{B}}
%\newcommand{\cC}{\mathcal{C}}
%\newcommand{\cD}{\mathcal{D}}
%\newcommand{\cE}{\mathcal{E}}
%\newcommand{\cF}{\mathcal{F}}
%\newcommand{\cG}{\mathcal{G}}
%\newcommand{\cH}{\mathcal{H}}
%\newcommand{\cJ}{\mathcal{J}}
%\newcommand{\cK}{\mathcal{K}}
%\newcommand{\cL}{\mathcal{L}}
%\newcommand{\cM}{\mathcal{M}}
%\newcommand{\cN}{\mathcal{N}}
%\newcommand{\cO}{\mathcal{O}}
%\newcommand{\cP}{\mathcal{P}}
%\newcommand{\cQ}{\mathcal{Q}}
%\newcommand{\cS}{\mathcal{S}}
%\newcommand{\cU}{\mathcal{U}}
%\newcommand{\cX}{\mathcal{X}}
%\newcommand{\cY}{\mathcal{Y}}
%\newcommand{\cW}{\mathcal{W}}
%\newcommand{\cR}{\mathcal{R}}
%\newcommand{\cT}{\mathcal{T}}
%\newcommand{\cZ}{\mathcal{Z}}
%
%
%%%%%%%%%%%%% mathsf%%%%%%%%%%%%%%%%%%
%
%
%\newcommand{\sfA}{\ensuremath{\mathsf{A}}}
%\newcommand{\sfB}{\ensuremath{\mathsf{B}}}
%\newcommand{\sfC}{\ensuremath{\mathsf{C}}}
%\newcommand{\sfD}{\ensuremath{\mathsf{D}}}
%\newcommand{\sfE}{\ensuremath{\mathsf{E}}}
%\newcommand{\sfF}{\ensuremath{\mathsf{F}}}
%\newcommand{\sfG}{\ensuremath{\mathsf{G}}}
%\newcommand{\sfH}{\ensuremath{\mathsf{H}}}
%\newcommand{\sfJ}{\ensuremath{\mathsf{J}}}
%\newcommand{\sfK}{\ensuremath{\mathsf{K}}}
%\newcommand{\sfL}{\ensuremath{\mathsf{L}}}
%\newcommand{\sfM}{\ensuremath{\mathsf{M}}}
%\newcommand{\sfN}{\ensuremath{\mathsf{N}}}
%\newcommand{\sfO}{\ensuremath{\mathsf{O}}}
%\newcommand{\sfP}{\ensuremath{\mathsf{P}}}
%\newcommand{\sfQ}{\ensuremath{\mathsf{Q}}}
%\newcommand{\sfS}{\ensuremath{\mathsf{S}}}
%\newcommand{\sfU}{\ensuremath{\mathsf{U}}}
%\newcommand{\sfX}{\ensuremath{\mathsf{X}}}
%\newcommand{\sfY}{\ensuremath{\mathsf{Y}}}
%\newcommand{\sfW}{\ensuremath{\mathsf{W}}}
%\newcommand{\sfR}{\ensuremath{\mathsf{R}}}
%\newcommand{\sfT}{\ensuremath{\mathsf{T}}}
%\newcommand{\sfZ}{\ensuremath{\mathsf{Z}}}
%
%%%%%%%%%%%%%  Special letters for Lie groups %%%%%%%%%%%%%%%%%%
%
%\newcommand{\biA}{{\mathbi{A}}}
%\newcommand{\biB}{{\mathbi{B}}}
%\newcommand{\biC}{{\mathbi{C}}}
%\newcommand{\biD}{{\mathbi{D}}}
%\newcommand{\biE}{{\mathbi{E}}}
%\newcommand{\biF}{{\mathbi{F}}}
%\newcommand{\biG}{{\mathbi{G}}}
%\newcommand{\biH}{{\mathbi{H}}}
%\newcommand{\biJ}{{\mathbi{J}}}
%\newcommand{\biK}{{\mathbi{K}}}
%\newcommand{\biL}{{\mathbi{L}}}
%\newcommand{\biM}{{\mathbi{M}}}
%\newcommand{\biN}{{\mathbi{N}}}
%\newcommand{\biO}{{\mathbi{O}}}
%\newcommand{\biP}{{\mathbi{P}}}
%\newcommand{\biQ}{{\mathbi{Q}}}
%\newcommand{\biS}{{\mathbi{S}}}
%\newcommand{\biU}{{\mathbi{U}}}
%\newcommand{\biX}{{\mathbi{X}}}
%\newcommand{\biY}{{\mathbi{Y}}}
%\newcommand{\biV}{{\mathbi{V}}}
%\newcommand{\biW}{{\mathbi{W}}}
%\newcommand{\biR}{{\mathbi{R}}}
%\newcommand{\biT}{{\mathbi{T}}}
%\newcommand{\biZ}{{\mathbi{Z}}}
%
%
%
%
%%%%%%%%%%%%%%%%%%%%%%%%%%%%%%%%%%%%%%%%%%%%%%%%%%%%%%%%%%%%%%%%%
%\newcommand{\SU}{\mathrm{SU}}
%\newcommand{\SO}{\mathrm{SO}}
%\newcommand{\SL}{\mathrm{SL}}
%\newcommand{\GL}{\mathrm{GL}}
%\newcommand{\Sp}{\mathrm{Sp}}
%\newcommand{\U}{\mathrm{U}}
%\newcommand{\ul}{\mathrm{u}}
%\newcommand{\Spin}{\mathrm{Spin}}
%\newcommand{\Pin}{\mathrm{Pin}}
%%%%%%%%%%%%%%%%%%%%%%%%%%%%%%%%%%%%%%%%%%%%%%%%%%%%%%%%%%%%%%%%%
%
%
%
%
%
%\renewcommand{\Im}{{\rm Im}}
%\renewcommand{\Re}{{\rm Re}}
%\newcommand{\Tr}{\mbox{Tr}}
%\newcommand{\Pf}{\mbox{Pf}}
%\newcommand{\sgn}{\mbox{sgn}}
%\newcommand{\Vir}{{\rm Vir}}
%\newcommand{\Li}{{\rm Li}}
%
%
%
%%\def\cD{{\cal D}}
%%\def\cA{{\cal A}}
%%\def\cN{{\cal N}}
%%\def\cS{{\cal S}}
%%\def\cD{{\cal D}}
%%\def\cA{{\cal A}}
%%%%%%%%%%%%%%%%%%%%%%%%%%%%%%%%%%%%%%%%%%%%%%%%%%%%%%%%%%%%%%%%%
%
%
%\newcommand{\valp}{{\vec\alpha}}
%\newcommand{\vbet}{{\vec\beta}}
%\newcommand{\vgam}{{\vec\gamma}}
%\newcommand{\vrho}{{\vec\rho}}
%\newcommand{\vome}{{\vec\omega}}
%\newcommand{\vlam}{{\vec\lambda}}
%\newcommand{\vphi}{{\vec\varphi}}
%\newcommand{\va}{{\vec a}}
%\newcommand{\vb}{{\vec b}}
%\newcommand{\ve}{{\vec e}}
%\newcommand{\vx}{{\vec x}}
%\newcommand{\vW}{{\vec W}}
%\newcommand{\vY}{{\vec Y}}
%\newcommand{\vha}{{\vec{\hat a}}}
%\newcommand{\vhb}{{\vec{\hat b}}}
%\newcommand{\vhc}{{\vec{\hat c}}}
%
%%%%%%%%%%%%%%%%%%%%%%%%%%%%%%%%%%%%%%%%%%%%%%%%%%%%%%%%%%%%%%%%%
%
%
%
%\newcommand{\dalpha}{{\dot \alpha}}
%\newcommand{\dbeta}{{\dot \beta}}
%\newcommand{\dgamma}{{\dot \gamma}}
%\newcommand{\dmu}{{\dot \mu}}
%\newcommand{\dnu}{{\dot \nu}}
%\newcommand{\drho}{{\dot \rho}}
%\newcommand{\dsigma}{{\dot \sigma}}
%\newcommand{\dlambda}{{\dot \lambda}}
%\newcommand{\dtau}{{\dot \tau}}
%\newcommand{\ha}{{\hat a}}
%\newcommand{\hb}{{\hat b}}
%\newcommand{\hc}{{\hat c}}
%\newcommand{\hj}{j^\circ}
%\newcommand{\vh}{{\vec h}}
%\newcommand{\vm}{{\vec m}}
%\newcommand{\vn}{{\vec n}}
%\newcommand{\vl}{{\vec l}}
%
%\newcommand{\HK}{{hyper-K\"ahler }}
%\newcommand{\K}{{K\"ahler }}
%\newcommand{\pl}{\textrm{pl}}
%\newcommand{\Ind}{\textrm{Ind}}
%
%
%
%
%\def \be  {\begin{equation}}
%\def \ee  {\end{equation}}
%\def \bea {\begin{equation}\begin{aligned}}
%\def \eea {\end{aligned}\end{equation}}
%\def \ba  {\begin{eqnarray}}
%\def \ea  {\end{eqnarray}}
%
%
%
% 


\def\wt{\widetilde}
%\def\bZ{{\bar{z}}}






%
%
%
%\usepackage{accents}
%\newcommand*{\dt}[1]{%
%  \accentset{\mbox{\large\bfseries .}}{#1}}
%\newcommand*{\ddt}[1]{%
%  \accentset{\mbox{\large\bfseries .\hspace{-0.25ex}.}}{#1}}
%
%\newcommand{\Spec}{\operatorname{Spec}\nolimits}
%\newcommand{\Gr}{\mathrm{Gr}}
%
%\newcommand{\Hom}{\textrm{Hom}}
%\newcommand{\End}{\textrm{End}}
%\def\BunG{{\text{Bun}_G}}
%%\def\bfN{{\text{Nilp}}}
%\def\Bcc{{\mathfrak{B}_{cc}}}
%
%
%
%
%\def\MF{{\cM_\text{flat}}}
%\def\MH{{{\cM}_H}}
%\def\SH{\mathbi{S}\!\ddt{\mathbi{H}}}
%\def\HH{\ddt{\mathbi{H}}}
%\def\WW{ \ddt{\mathbi{W}}}
%\def\BB{\ddt{\mathbi{B}}\mathbi{r}}
%\def\wh{\widehat}
%\def\id{\mathrm{id}}
%\def\Td{\mathrm{Td}}
%
%
%
%\newcommand{\MyRed}{\color [rgb]{0.9,0,0}}
%\newcommand{\MyGreen}{\color [rgb]{0,0.5,0}}
%\newcommand{\MyBlue}{\color [rgb]{0,0,0.8}}
%\newcommand{\MyBrown}{\color [rgb]{0.8,0.3,0.2}}
%\newcommand{\MyPurple}{\color [rgb]{0.6,0.0,0.7}}
%
%
%\def\SN#1{{\MyRed [SN: #1]}}
%


%\title{Introduction to 2d conformal field theories}
%\date{}
%\author{Satoshi Nawata\cr email \href{mailto:snawata@gmail.com}{snawata@gmail.com} }
%%\author[2]{Anindya Dey}
%%\author[3]{Gregory W. Moore}
%\affil{Department of Physics and Center for Field Theory and Particle Physics, Fudan University, 220
%Handan Road, 200433 Shanghai, China}




\begin{document}
%\maketitle
%\abstract{These are lecture notes on 2d conformal field theories in Fall 2018. If you find typos, please email me. }
%


\setcounter{tocdepth}{2}
%\maketitle


\subsection{2d Conformal Transformation}
In this section, we will focus on the 
special case of two dimensions with 
Euclidean metric.

From \eqref{inf-cft-e-condition}, the infinitesimal condition for 
conformal transformation 
in two dimensions 
reads as follows:
\begin{equation}
\label{C-R equation}
  \partial_1\e_1
  =
  \partial_2\e_2
  \, ,
  \qquad
  \partial_1\e_2
  =
  -\partial_2\e_1
  \, .
\end{equation}  
\eqref{C-R equation} 
tells us that 
if we introduce complex variables :
\begin{align}
\label{coordinate transformation}
  z= x^1 + i x^2
  \,  ,\qquad
  \e = \e^1 + i \e^2
  \,  ,\notag \\
  \bar{z}= x^1 - ix^2
  \,  ,\qquad
  \bar{\e}= \e^1-i \e^2 
  \, ,
\end{align}
according to
Cauchy-Riemann equations
in complex analysis, 
$\e(z)$ is a holomorphic function
(in some open set).
And, equivalently any infinitesimal
holomorphic transformation 
$z'=z+ \e(z)$ 
gives rise to an 
two-dimensional conformal transformation.

In fact, any analytic mapping of 
the complex plane onto itself
is conformal
(i.e, preserves angles).
To prove this, 
we should rewrite
the definition of 
a conformal transformation
\eqref{def-flat-ct}
with complex variables.
Let us first, rewrite the line element:
\begin{equation}
  ds^2=(\dd x^1)^2+(\dd x^2)^2
      =\dd z \dd \bar{z}
      =g_{\alpha\beta}z^\alpha z^\beta
      \, ,
\end{equation}
with $z^1=z$, $z^2=\bar{z}$. 
The flat space metric 
in complex coordinates
can be obtained:
\begin{equation}
\label{metric-complex-space}
  g_{\alpha\beta}
  =
  \mqty(
       0 & \frac{1}{2}\\
       \frac{1}{2} & 0
       )
  \, ,
  \qquad
  g^{\alpha\beta}
  =
  \mqty(
       0 & 2 \\
       2 & 0
       )
  \, .
\end{equation}
The condition 
\eqref{def-flat-ct}
now becomes:
\begin{equation}
  \label{def-flat-ct-complex}
  g_{\alpha\beta} 
  \pdv{z'^\alpha}{z^\gamma}
  \pdv{z'^\beta}{z^\delta}
  =
  \Lambda(z,\bar{z})g_{\gamma\delta}
  \, ,  
\end{equation}
which is satisfied, 
if $z'=f(z)$ 
is a holomorphic function.
And, the scale factor is then
$\Lambda(z,\bar{z}) 
= 
\abs{\pdv{f(z)}{z}}^2$.
Fig. \ref{fig:2dcft-ill}, 
shows some examples on 
coordinate transformations.
We see when 
$z'=f(z)$ is holomorphic,
all included angles 
remain right angles,
which implies a conformal mapping.   
\begin{figure}[h]
	\centering
	\caption{Coordinate transformation: 
		The transformation from 
		square lattice (a), 
		onto the lattice in (b) and (c) 
		is conformal, 
		while the transformation 
		onto the lattice in (d) is not.}
	\label{fig:2dcft-ill} 
	\subfigure[$z$]{
	\includegraphics[width=0.3\textwidth]{picture/2dcft-1}}
    \subfigure[$z'=z^2$]{
	\includegraphics[width=0.3\textwidth]{picture/2dcft-2}}
    \subfigure[$z'=1/z$]{
	\includegraphics[width=0.3\textwidth]{picture/2dcft-3}}
    \subfigure[$z'=z\abs{z}$]{
    \includegraphics[width=0.3\textwidth]{picture/2dcft-4}}
\end{figure}

\subsubsection*{Witt algebra}

As we have seen above,
for an infinitesimal conformal
transformation in two dimensions,
$\e(z)$ has to be holomorphic in 
some open set. 
We can locally around say
$z = 0$
perform a Laurent expansion of
$\e(z)$.
A general infinitesimal conformal 
transformation thus can be written as
\begin{align}
\label{2d-inf-Laurant}
  z'&=z+\e(z)
     =z 
     + \sum_{n\in\mathbb{Z}}
     \e_n(-z^{n+1})
     \notag \, , \\
  \bar{z}' &=z+ \bar{\e}(\bar{z})
     =\bar{z}
     +  \sum_{n\in\mathbb{Z}}
     \bar{\e}_n
     (-\bar{z}^{n+1})
     \, ,      
\end{align} 
where $\e_n$ and $\bar{\e}_n$ are 
infinitesimal parameters.
Supposing a spinless and dimensionless
field $\phi(z,\bar{z})$
living on the plane,
the effect of such a mapping would be:
\begin{equation}
  \label{tran-scalar-field}
  \phi(z,\bar{z})
  \rightarrow
  \phi'(z',\bar{z}')
  =
  \phi(z,\bar{z})
  \, ,
\end{equation} 
or
\begin{align}
  \delta{\phi}
  &\equiv
  \phi'(z,\bar{z})
  -\phi(z,\bar{z})
  =\phi'(z'-\e(z),\bar{z}'-\bar\e(\bar{z}))  
  -\phi(z,\bar{z})
  \notag\\
  &=
  -\e(z)\partial\phi 
  -\bar\e(\bar{z})\bar{\partial}\phi
  +o(\e)
  =
  -\sum_{n\in\mathbb{Z}}
  (\e_n l_n \phi(z,\bar{z})
   +
   \bar\e_n \bar{l}_n \phi(z,\bar{z})
  )
  \, ,
\end{align}
where we have introduced the generators
\begin{equation}
l_n = - z^{n+1}\partial\,,
\qquad
\bar{l}_n = -\bar{z}^{n+1} \bar\partial
\, .
\end{equation} 
We find the number of 
independent infinitesimal conformal transformations is infinite.
The commutators for these
local generators can be calculated straightforward then:
\begin{align}
\label{Witt-algebra}
  \comm{l_n}{l_m} 
  &=
  (m-n) l_{n+m}
  \, ,\notag \\
  \comm{\bar{l}_m}{\bar{l}_n}
  &=
  (m-n)\bar{l}_{m+n}
  \, , \\
  \comm{l_m}{\bar{l}_n}
  &=0
  \, .\notag
\end{align}
The first commutation
relations define one copy of the
so-called Witt algebra.
We can observe that the local 
conformal algebra 
\eqref{Witt-algebra} 
is the direct sum of two
Witt algebras.

\subsubsection*{M\"obius transformation}
A global conformal transformation 
in two dimensions
is defined to be invertible
and well-defined on the 
Riemann sphere
$S^2 \simeq \mathbb{C}\cup
\{\infty\}$.
Thus, it should be generated
by the global-defined
infinitesimal generators.
At $z=0$, we find that
$l_n = - z^{n+1}\partial$,
is non-singular only for
$n\geq-1$.
At $z=\infty$,
let us perform the change
of variable 
$z=-\frac{1}{w}$,
and study $w \rightarrow 0$.
We then observe that
$l_n=-(-\frac{1}{w})^{n-1}\partial_w$,
is nonsingular at 
$w=0$ only for
$n\leq 1$.
Therefore, globally defined
conformal transformations 
on the Riemann sphere
$S^2=\mathbb{C}\cup\{\infty\}$
are generated by 
$\{l_{-1},l_0,l_{+1} \}$.
Through Wit algebra 
\eqref{Witt-algebra},
we immediately see that these three 
generators form a close subalgebra 
of Wit algebra, 
which implies 
the corresponding conformal transformations
form a globally defined group,
called the conformal group
in two dimensions.
The discussion of
antiholomorphic counterpart
is similar.

Now, we are going to study
the conformal group in two dimensions
deeply. 
From the Laurent expansion 
\eqref{2d-inf-Laurant}, 
the operator 
$l_{-1}= -\partial_z$, generates
the infinitesimal transformation
$z'=z-\e_{-1}$.
Thus, $l_{-1}$ is the 
generator of translations
$z\mapsto z+b$ with $b$
an arbitrary complex constant.

For $n = 0$, the operator 
$l_0 = -z\partial_z$, 
generates the linear term of
\eqref{2d-inf-Laurant}. 
It is not difficult to verify that
the corresponding finite transformation is
$z'=a z$, with $a$ 
an arbitrary complex constant. 
The modulus of $a$ 
turns out to lead a dilatation,
while the phase of $a$ gives rise to
a rotation. 
In order to get a geometric intuition of
such transformations, we can perform the
change of variables $z=r e^{i\phi}$ to find
\begin{equation}
  l_0 = -\frac{1}{2} r \partial_r
        + \frac{i}{2}\partial_\phi\, ,
  \qquad
  \bar{l}_0 = - \frac{1}{2}r \partial_{r}
        - \partial_\phi \, .
\end{equation}
We can perform the linear combinations
\begin{equation}
  \label{l0combinations}
  l_0 + \bar{l}_0 = - r \partial_r\, ,
  \qquad 
  i(l_0-\bar{l}_0) = -\partial_\phi\, .
\end{equation}
Therefore, we see that $l_0 + \bar{l}_0$ is the
generator for two-dimensional dilations and
that $i(l_0 - \bar{l}_0)$ is the generator of 
rotations.

Finally, the operator $l_{+1}$
generates the infinitesimal quadratic term.
By the variable changing 
$w = -\frac{1}{z}$, 
the operator turns out to be
$l_{+1} = -\partial_w$, 
which corresponds to a translation of $w$: 
$w\mapsto w-c$. 
Therefore, the operator $l_{+1}$ generates
Special Conformal Transformations:
$z\mapsto \frac{z}{cz+1}$.

The complete set of mappings discussed 
above is of the form
\begin{equation}
  z\mapsto \frac{a z+b}{cz + d}
  \, ,
\end{equation}
with $a,b,c,d \in \mathbb{C} $. 
For this transformation
to be invertible, we have to require 
that $ad-bc \neq 0$. 
If this is the case, we can
scale the constants $a,b,c,d$, 
such that $ad-bc =1$.
Furthermore, the expression is 
unaffected by taking all of 
$a,b,c,d$ to minus themselves.
These mappings are called projective
transformations.
We see each global conformal transformation
in 2 dimensions can be associated
with a matrix
$\mqty(a & b \\ c & d)$. 
For instance:
\begin{align}
	&\text{translation}:\qquad
	z\mapsto z+b 
	\quad
	\Leftrightarrow
	\quad 
	\mqty(1 & b \\ 0 & 1)
	\, ,\notag\\
	&\text{dilatation}:\quad
	z\mapsto a z\, , a\in \mathbb{R}
	\, 
	\Leftrightarrow
	\,
	\mqty(\sqrt{a} & 0\\
	      \sqrt{0} & \sqrt{1/a}	
	     )	
	\, ,\notag \\
	&\text{rotation}:
	\qquad
	z\mapsto e^{i\theta}\, , 
	\theta\in \mathbb{R}
	\,
	\Leftrightarrow
	\,
	\mqty(e^{i\theta/2} & 0\\
	      0 & e^{-i\theta/2}
	     )
	\, ,\notag \\
	&SCT:
	\qquad\qquad
	z\mapsto \frac{z}{c z +1}
	\quad
	\Leftrightarrow
	\quad
	\mqty(1 & 0 \\
	      c & 1)
	 \, .\notag
\end{align}
We can infer that the conformal 
group of the Riemann sphere
$S^2 = \mathbb{C} \cup \{\infty \}$
is the  M\"obius group 
$SL(2,\mathbb{C})/\mathbb{Z}_2$.









	
\section{Basics of Conformal Field Theory}

\subsection{Noether theorem and Ward-Takahashi identity}
In this section, we are going to review
some basic concepts and theorem
in classical field theory
as well as in quantum field theory. 

\subsubsection*{Continuous Symmetry Transformation}
In the foregoing sections,
we focus on coordinate transformation,
while in field theory, 
we still need to consider 
the transformation of fields.
In the previous sections, we have used transformations of spinless field
\eqref{tran-scalar-field}, 
to obtain the
corresponding generators. 
However, sometimes field 
might have its own 
non-trivial transformation rules.
Fig.\ref{fig:symmetry} shows the rotation 
transformation of a vector field.
We see field $\Phi$ rotates under the
coordinate transformation. 
\begin{equation}
  \label{trans-field-general}
  x\rightarrow x'\, , 
  \qquad
  \Phi(x)\rightarrow\Phi'(x')
  =\mathcal{F}(\Phi(x)) \, .
\end{equation}

\begin{figure}
	\centering
	\caption{Rotation transformation of a vector field.}
	\includegraphics[width=0.5\linewidth]{picture/Symmetry}
	\label{fig:symmetry}
\end{figure}

Functional change $\mathcal{F}$
can be obtained by studying the 
representation theory of coordinate
transformation group. We will not 
illustrate more on this, 
but give some examples
for later use.

Let's starting with a rather trivial one:
a translation:
\begin{equation}
  \label{translation-field}
  x' = x + a\, , 
  \qquad
  \Phi'(x+a) = \Phi(x)\, ,
\end{equation}
which implies that functional change 
$\mathcal{F}$ is trivial here.

Next, we consider a Lorentz transformation
(rigid rotation).
In general it takes the following form:
\begin{equation}
  \label{lorentz-field}
  x'^\mu = {\Lambda^\mu}_\nu x^\nu\, ,
  \qquad
  \Phi'(\Lambda x) =U(\Lambda)\Phi(x)\, ,
\end{equation}
\footnote{
	In quantum field theory, 
	$\Phi$ is more likely to treat as an operator with transformation rule:
	\begin{equation}
	\Phi'(\Lambda x)
	=U^{-1}(\Lambda)\Phi(x)U(\Lambda)\, .
	\notag
	\end{equation}
}
where $U(\Lambda)$ is a transformation matrix acting on $\Phi$. 
The infinitesimal form of 
$U(\Lambda)$ is 
\begin{equation}
 \label{U-Lambda-inf}
 U(\Lambda)=1-\frac{1}{2}i
 \omega_{\rho\nu}
 S^{\rho\nu}\, ,
\end{equation}
where $\omega_{\rho\nu}$ is infinitesimal 
parameter for Lorentz transformation
and $S^{\rho\nu}$ is some matrix
obeying the Lorentz algebra.

Finally, under scale transformations, 
we have:
\begin{equation}
  \label{scale-field}
  x'=\lambda x \, ,
  \qquad
  \Phi'(\lambda x) = \lambda^{-\Delta}
  \Phi(x) \, ,
\end{equation}
where $\Delta$ is the scaling dimension 
of the field.

\subsubsection*{Generator}
In a field theory, the generator 
of a symmetry depends on 
transformation of field 
as well as of coordinate.
The infinitesimal form of 
\ref{trans-field-general}
can be written as:
\begin{align}
\label{trans-field-general-inf}
  x'^{\mu} &= x^\mu + \omega_a 
  \frac{\delta x^\mu}{\delta \omega_a}
  \, ,
  \notag\\
  \Phi'(x') &= \Phi(x)
  +
  \omega_a \frac{\delta\mathcal{F}}
  {\delta \omega_a}(x)
  \, .
\end{align}
Here $\{\omega_a\}$ is a set of
infinitesimal parameters.
Now it is customary to define  
the generator $G_a$ of a
symmetry transformation by 
the following expression:
\begin{equation}
\label{def-generater-general}
  \delta_\omega
  \Phi(x) 
  \equiv \Phi'(x)-\Phi(x)
  \equiv 
  -i\omega_a G_a \Phi(x)\, ,
\end{equation}
which is just
the infinitesimal transformation
of field at a same point. 
Together with 
\eqref{trans-field-general-inf},
to first order, we obtain the 
explicit expression:
\begin{equation}
  \label{def-generator-general-explicit}
  iG_a \Phi 
  =
  \frac{\delta x^\mu}{\delta \omega_a}
  \partial_\mu \Phi
  -
  \frac{\delta\mathcal{F}}{\delta \omega_a}
  \, .
\end{equation}
For a field satisfying $\Phi'(x')=\Phi(x)$,
the expression reduces to 
\eqref{def-generator}.
Using the definition 
\eqref{def-generator-general-explicit},
and combining with
the infinitesimal field of 
\eqref{translation-field},
\eqref{lorentz-field} and
\eqref{scale-field},
the generators in field space listed
below can be derived:
\begin{align}
  \label{generator-three-field}
  &\text{translation}:
  \qquad  P_\nu = -i\partial_\nu \, ,
  \notag \\
  &\text{rotation} :
  \qquad L^{\rho\nu}
  =i(x^\rho\partial^\nu
  -x^\nu\partial^\rho)
  +S^{\rho\nu}
  \, ,\\
  &\text{dilatation}:
  \qquad
  D = -i x^\nu \partial_\nu 
  -i\Delta
  \, . \notag 
\end{align}
We see that when field transformation
$\mathcal{F}(\Phi)$ is trivial, 
three expressions in
\eqref{generator-three-field} reduce 
to the first three generators in
Table. \ref{tab:ModelASymbol}.

\subsubsection*{Noether's Theorem}

Consider theory in d dimensions
with action depending 
in general on $\Phi$ 
and its first derivatives.
\begin{equation}
 S = \int \dd^d x
 \mathcal{L}(\Phi,\partial_\mu \Phi)
\end{equation}
A theory with a symmetry means that
its action
(or effective action 
more accurately in quantum field theory)
keeps invariant under such a symmetric 
transformation.

We now demonstrate Noether's theorem, 
which states that to every continuous 
symmetry of the action one may associate 
a current that is classically conserved.
An elegant way to derive Noether's theorem 
is to suppose that $\omega$
the parameters
of infinitesimal transformation depend
on the position. First, under continuous 
transformation 
\eqref{trans-field-general}, 
the action becomes
\begin{align}
  S' &= \int \dd^d x'
  \mathcal{L}(
  \Phi'(x'),\partial_\mu'\Phi'(x'))
  \notag\\
  & = \int \dd^d x \abs{\pdv{x'}{x}}
  \mathcal{L}
  (\mathcal{F}(\Phi(x)),(\pdv{x^\nu}{x'^\mu})
  \partial_\nu \mathcal{F}(\Phi(x)))\, .
\end{align}
Using the infinitesimal form for transformation
\eqref{trans-field-general-inf}, 
we get
\begin{align}
S' = &\int \dd^d x (1+\partial_\mu
     (\omega_a \frac{\delta x^\mu}
     {\delta \omega_a}))
     \notag\\
     &\times
     \mathcal{L}
     (
     \Phi + \omega_a 
     \frac{\delta\mathcal{F}}
     {\delta\omega_a},
     (
     \delta^\nu_\mu
     -
     \partial_\mu
     (
     \omega_a\frac{\delta x^\nu}
     {\delta \omega_a}
     )
     )
     (
     \partial_\nu \Phi+
     \partial_\nu
     (
     \omega_a\frac{\delta\mathcal{F}}
     {\delta\omega_a}
     )
     )
     )\, .
\end{align}
Now using the symmetric condition that 
the variation $\delta S = S'-S$ of the
action vanishes if $\omega_a$ is a constant.
Thus sum up of all terms with no derivatives
of $\omega_a$ in the variation must be zero.
Therefore, to first order, the variation
involves only the first derivatives of
$\omega_a$, obtained by expanding the 
Lagrangian. 
We write
\begin{equation}
   \delta S = - \int \dd^d x
   j^\mu_a \partial_\mu \omega_a\, ,
\end{equation}
where $j^\mu_a$ is called the current
associated with the infinitesimal transformation:
\begin{equation}
  \label{cano-current}
  j^\mu_a =(
  \pdv{\mathcal{L}}{
   (\partial_\mu\Phi)}
   \partial_\nu \Phi
   -
   \delta^\mu_\nu
   \mathcal{L}
  )
  \frac{\delta x^\nu}{\delta\omega_a}
  -
  \pdv{\mathcal{L}}{(\partial_\mu \Phi)}
  \frac{\mathcal{F}}{\delta\omega_a}
  \, .
\end{equation}
Integrating by parts, we have
\begin{equation}
  \label{variation-action-locally}
  \delta S = \int \dd^d x \, \partial_\mu
  j^\mu_a \omega_a
  \, .
\end{equation} 
If the field configuration satisfy the
classical equation of motion, 
the action is stationary under any 
variation of the fields. Thus 
$\delta S$ should vanish for any parameters
$\omega_a(x)$. This implies the 
conservation law:
\begin{equation}
\partial_\mu j^\mu_a = 0 \, .
\end{equation} 
The associated conserved charge is
\begin{equation}
\label{def-conserved-charge}
  Q_a = \int \dd^{d-1} x \, j^0_a \, .
\end{equation}
Its time derivative vanishes:
\begin{equation}
  \dot{Q}_a = \int \dd^{d-1} x \,
  \partial_0 j_a^0
  =-\int \dd^{d-1}x \, \partial_i j^i
  =0\, .
\end{equation}
One remark  to be mentioned here
is that we may freely add to the expression
\eqref{cano-current} 
the divergence of
an antisymmetric tensor without 
affecting the conservation law:
\begin{equation}
  \label{current-freedom}
  j^\mu_a \rightarrow 
  j^\mu_a +\partial_\nu B^{\nu\mu}_a
  \, ,
  \qquad
  B^{\nu\mu}_a = - B^{\mu\nu}_a
  \, .
\end{equation}

Now we are going to focus on some useful 
currents as examples. First of all, 
if the action is invariant under translation
$x'^\mu\rightarrow x^\mu +\e^\mu$, 
we can associate it with a current by using
\eqref{cano-current}:
\begin{equation}
  \label{cano-ener-mon-tensor}
  T^{\mu\nu}_{\text{c}}=
  -\eta^{\mu\nu}\mathcal{L}
  +\pdv{\mathcal{L}}{(\partial_\mu\Phi)}
  \partial^\nu \Phi \, .
\end{equation}
The conserved current associated with
translation invariance is the energy-momentum
tensor. 
In general, the canonical energy-momentum
tensor $T^{\mu\nu}_\text{c}$ is not
symmetric. However, we have freedom to 
modify this tensor as is illustrated
in \eqref{current-freedom}, 
to make the energy momentum tensor
symmetric.
\begin{equation}
  \label{freedom-energy-momentum-tensor}
  T^{\mu\nu}_{\text{B}}=
  	T^{\mu\nu}_\text{c}
  	+\partial_\rho B^{\rho\mu\nu}\,
  	,\quad
  	B^{\rho\mu\nu} = -B^{\mu\rho\nu}\, .
\end{equation}

Now consider the conserved current
associated with Lorentz transformation.
By using the infinitesimal form 
\eqref{U-Lambda-inf}, the conserved 
current becomes
\begin{equation}
  j^{\mu\nu\rho}=
  T^{\mu\nu}_{\text{c}}x^\rho-
  T^{\mu\rho}_{\text{c}}x^\nu+
  \frac{i}{2}
  \pdv{\mathcal{L}}{(\partial_\mu\Phi)}
  S^{\nu\rho}\Phi \, .
\end{equation}
We can look for $B^{\rho\mu\nu}$ such
that this current may be expressed as
\begin{equation}
\label{current-lorentz-B}
  j^{\mu\nu\rho}
  = T^{\mu\nu}_\text{B} x^\rho
  -T^{\mu\rho}_\text{B} x^\nu\, .
\end{equation}
This relation ensures that
$T^{\mu\nu}_\text{B}=T^{\nu\mu}_\text{B}$
classically,
as is easily seen by applying the 
conservation laws. The energy-momentum
tensors with the above identity are 
called Belinfante tensor.
An explicit expression for
$B^{\rho\mu\nu}$ can be found:
\begin{equation}
  B^{\mu\rho\nu} = \frac{1}{4}i\big[
  \pdv{\mathcal{L}}{(\partial_\mu \Phi)}
  S^{\nu\rho} \Phi 
  +
  \pdv{\mathcal{L}}{(\partial_\rho \Phi)}
  S^{\mu\nu} \Phi
  +
  \pdv{\mathcal{L}}{(\partial_\nu \Phi)}
  S^{\mu\rho} \Phi
  \big]\, .
\end{equation}
Actually such kind of $B^{\mu\rho\nu}$
is not unique.

Finally, if the action of theory is also 
invariant under dilatation, the current 
associated with scale invariance can be
obtained by using the infinitesimal form of
\eqref{scale-field} and the definition
\eqref{cano-current}:
\begin{equation}
  j^{\mu}_\text{D} =
  {T_c^\mu}_\nu x^\nu+\Delta
  \pdv{\mathcal{L}}{(\partial_\mu\Phi)}
   \Phi \, .
\end{equation}
We can also modify the energy momentum 
to become
\begin{equation}
  \label{dilatation-current}
  j^\mu_\text{D}= {T^\mu}_\nu x^\nu\, ,
\end{equation}
without spoiling the symmetric condition for a Belinfante tensor.
Further more,
by using conservation laws, we may find
such kind of modified energy momentum
tensor is traceless classically, i.e.
\begin{equation}
 {T^\mu}_\mu = 0\, .
\end{equation}
In later discussion, we will always assume 
the energy momentum tensor we have constructed
in a conformal invariance theory
is classically symmetric and traceless.

\subsubsection*{Correlation Functions}
In previous sections, we focus on the effect
of a continuous symmetry classically. 
At the quantum level, the main object of 
study is correlation function. 
A general correlation function is defined
as following
\begin{equation}
\label{def-correlation-function}
  \expval{\Phi(x_1)\cdots \Phi(x_n)}=
  \frac{1}{Z}\int [\dd\Phi]
  \Phi(x_1)\cdots \Phi(x_n)
  \exp{-S[\Phi]}\, ,
\end{equation}
where $Z$ is the vacuum functional:
\begin{equation}
  Z = \int [\dd \Phi] \exp(-S[\Phi])
  \, .
\end{equation}
The path integral denotation $[\dd \Phi]$ means to integrate out all the field configurations of $\Phi$. 
The operators in the correlation function
$\Phi(x_1)\cdots \Phi(x_n)$ are automatically
time-ordered by using path integral method,
i.e.
\begin{equation}
\expval{\mathcal{T}(\Phi(x_1)
	\cdots \Phi(x_n))}
= \expval{\Phi(x_1)\cdots\Phi(x_n)}\, ,
\end{equation}
where $\mathcal{T}$ is the time ordering operator
which makes the field operator in time order:
\begin{equation}
  \mathcal{T}(\Phi(x_1)
  	\cdots \Phi(x_n))
  = \Phi(x_1)\cdots\Phi(x_n)\, ,
  \qquad
  \text{if}
  \quad
  t_1 > t_2 > \cdots > t_n \, .
\end{equation}
However, all we have discussed are in
Euclidean space. So time direction can
be arbitrary by now. We will specify time
direction in two dimension case later.

A continuous symmetry leads to constraints
to the correlation function:
\begin{equation}
\label{tran-correlation-function}
 \expval{\Phi(x_1')\cdots\Phi(x_2')}
 =
 \expval{\mathcal{F}(\Phi(x_1))\cdots 
         \mathcal{F}(\Phi(x_n))   
    }\, .
\end{equation}
The above identity is the consequence of 
the invariance of the action 
and of the path 
integral measure under the transformation:
\begin{align}
  \expval{\Phi(x_1')\cdots \Phi(x_n')}
  &=
  \frac{1}{Z}
  \int [\dd \Phi] \Phi(x_1')\cdots \Phi(x_n')
  \exp(-S[\Phi])
  \notag\\
  &=
  \frac{1}{Z}
  \int [\dd \Phi']
  \Phi'(x'_1)\cdots \Phi'(x'_n)
  \exp(-S[\Phi'])
  \notag \\
  &=
  \frac{1}{Z}
  \int [\dd \Phi]
  \mathcal{F}(\Phi(x_1))\cdots
  \mathcal{F}(\Phi(x_n))
  \exp(-S[\Phi])
  \notag\\
  &=
  \expval{\mathcal{F}(\Phi(x_1))\cdots 
  	\mathcal{F}(\Phi(x_n))} 
  	\notag\, .
\end{align}
On the second line, we just rename the
integration variable $\Phi \rightarrow \Phi'$.
On the third line, the symmetry assumptions
for action and functional integral measure 
were used. 
\eqref{tran-correlation-function} immediately
implies that correlation function
is invariant under translation.

\subsubsection*{Ward-Takahashi Identity}
The consequence of a symmetry of the action
and the measure on correlation functions may
also be expressed via the Ward-Takahashi Identity, which we shall demonstrate now.
We first change the functional field 
variable in 
\eqref{def-correlation-function}
by an infinitesimal transformation
given by
\eqref{def-generater-general}
\begin{equation}
  \Phi'(x) = \Phi(x)- i \omega_a G_a \Phi(x)
  \, ,
\end{equation}
Denoting the collection
$\Phi(x_1)\cdots \Phi(x_n)$ of fields
in \eqref{def-correlation-function}
by $X$. And since now $\omega_a$ is the 
function of position $x$,
the action transforms
according to 
\eqref{variation-action-locally}
$S[\Phi] \rightarrow 
S[\Phi]+\int \dd x \partial_\mu j^\mu_a 
\omega_a(x) $.
Under this local transformation,
we denote the transformation of $X$ to be
$X\rightarrow X+\delta X$. The correlation
function is invariant through the functional
variable exchange, thus 
we can write:
\begin{equation}
 \expval{X}=\frac{1}{Z}
 \int [\dd  \Phi']
 (X+\delta X)
 \exp{-S[\Phi]-\int \dd x 
 \partial_\mu j^\mu_a \omega_a (x)}
 \, .
\end{equation}
We again assume that the functional
integration is invariant under the
local transformation(i.e.,
$[\dd \Phi'] = [\dd \Phi])$.
When expanded to first order of
$\omega_a(x)$, the above yields
\begin{equation}
  \label{Ward-identity-ver2}
  \expval{\delta X}=\int \dd^d x
  \partial_\mu
  \expval{j^\mu_a (x)X} \omega_a(x)\, .
\end{equation}

The variation $\delta X$ is explicitly 
given by 
\begin{align}
  \delta X 
  &= -i \sum_{i=1}^n
  (
  \Phi(x_1)\cdots G_a\Phi(x_i)
  \cdots \Phi(x_n)
  )
  \omega_a(x_i)
  \notag \\
  &=-i \int \dd^d x \, \omega_a(x)
  \sum_{i=1}^{n}
  (
  \Phi(x_1)\cdots G_a \Phi(x_i)
  \cdots \Phi(x_n)
  )
  \delta(x-x_i)
\end{align}
Since the above two expression hold
for any infinitesimal function 
$\omega_a(x)$, we may write the 
following local relation:
\begin{align}
  \label{Ward-Takahashi-Identity}
  \pdv{x^\mu}
  &\expval{
  	j^\mu_a(x)\Phi(x_1)\cdots
  	\Phi(x_n)
  }\notag\\
  &=-i\sum_{i=1}^{n}
  \delta(x-x_i)
  \expval{
  \Phi(x_1)\cdots G_a\Phi(x_i)
  \cdots 
  \Phi(x_n)
  }\, .
\end{align}
This is the Ward-Takahashi identity for the 
current $j^\mu_a$. Note that the modification
\eqref{current-freedom}
doesn't affect the identity.

The Ward-Takahashi identity allows us to 
identify the conserved charge $Q_a$ defined
in \eqref{def-conserved-charge}
as the generator of the symmetry transformation
in the Hilbert space of quantum states.
Let $Y= \Phi(x_2)\cdots \Phi(x_n)$ and 
suppose that the time $t=x^0_1$ is larger
than all the times in $Y$. We integrate the
Ward-Takahashi identity \eqref{Ward-Takahashi-Identity}
in a very thin box bounded by 
$t_-<t$, by $t_+>t$, and by spatial
infinity, which excludes all the 
other points $x_2,\cdots,x_n$.
\begin{equation}
  \expval{Q_a(t_+)\Phi(x_1)Y}
  -
  \expval{Q_a(t_-)\Phi(x_1)Y}
  = -i \expval{G_a \Phi(x_1)Y}\, .
\end{equation}
In the limit $t_-\rightarrow t_+$, for 
arbitrary set of fields $Y$, we obtain
\begin{equation}
  \label{charge-com-generator}
  \comm{Q_a}{\Phi} = -i G_a \Phi\, .
\end{equation}
We see that, the conserved charge $Q_a$ is 
the generator in the operator formalism.


\subsection{Primary fields}
In the previous section, we have
reviewed some concepts of field 
theory in d dimensions,  and derived 
the current expression for three important
currents in conformal symmetry. 
Now, we should fix our concentration in 
two dimensions. First, we need to define
the field transformation under conformal transformation. 
From \eqref{Witt-algebra}, we find 
$l_0$ and $\bar{l}_0$ are the Cartan 
subalgebra of conformal algebra. 
Thus we can assume physical states 
to be eigenstates of $l_0$ and $\bar{l}_0$
with eigenvalues $\{ h, \bar{h}\}$.
$h$ are called the holomorphic conformal
dimension and $\bar{h}$ the antiholomorphic 
one. 
There is an equivalent definition of 
conformal dimension
for an field operator which is more
explicit. 
Under conformal transformation
$z\rightarrow \lambda z,
 \bar{z} \rightarrow \bar\lambda \bar{z}$, 
the field operator transforms as
\begin{equation}
\label{def-conformal-dimension}
  \phi'(\lambda z,\overline\lambda \overline{z})
  =
  (\lambda)^{-h}
  (\overline\lambda)^{-\overline{h}}
  \phi(z,\overline{z})\, .
\end{equation}
Under an arbitrary global conformal map 
$z\rightarrow w(z), 
\overline{z} \rightarrow \overline{w}(\overline{z})$,
if a field transforms as
\begin{equation}
\label{def-quasi-primary-field}
  \phi'(w,\overline{w})
  =\bigg(\frac{\dd w}{\dd z}\bigg)^{-h}
  \bigg(\frac{\dd \overline{w}}{\dd \overline{z}}\bigg)^{-\overline{h}}
  \phi(z,\overline{z})\, ,
\end{equation}
then we call such field a quasi-primary field.
Under infinitesimal transformation 
$z \rightarrow z + \e(z)$, we obtain the
infinitesimal transformation of quasi-primary
field:
\begin{equation}
\label{trans-inf-quasi-primary}
  \delta_{\e,\overline\e}\phi
  \equiv \phi'(z,\overline{z})
  - \phi(z, \overline{z})
  =-(h\phi\partial_z \e + \e \partial_z \phi)
  -
  (\overline{h}\phi\partial_{\bar{z}}
  \overline{\e}
  +
  \overline{\e}\partial_{\bar{z}}\phi
  )\, .
\end{equation}
A field whose transformation under
any local conformal transformation 
given by 
\eqref{def-quasi-primary-field}
or equivalently
\eqref{trans-inf-quasi-primary}
is called primary field. Therefore
all primary fields are also quasi 
primary, but the reverse is not true.
A field which is not primary is 
generally called secondary.

The definition \eqref{def-quasi-primary-field}
put strong constraints to their correlation 
function by using \eqref{tran-correlation-function}.
For instance for the two-point function of 
chiral quasi-primary fields, the translation 
invariance implies that the two-point function
is of the form
\begin{equation}
  \expval{\phi_1(z)\phi_2(w)}
  = g(z-w) \, .
\end{equation}
Then under conformal transformation  
$z \rightarrow \lambda z $, we have
\begin{equation}
  \lambda^{h_1+h_2} g(z-w) = g(\lambda(z-w))\, ,
\end{equation}
where $h_1$,$h_2$ is the conformal dimensions of
$\phi_1$  and $\phi_2$ correspondingly. This leads 
a constraint for the two point function.
since the relation above satisfies for arbitrary 
$\lambda$ we obtain that $g(z-w)$ is of the form
\begin{equation}
  g(z-w) = \frac{C_{12}}{(z-w)^{h_1+h_2}}\, ,
\end{equation}
with $C_{12}$ a constant. Finally, when apply the
conformal transformation $z \rightarrow 1/z$, for
the two-point function to be non-trivial, the 
conformal dimensions of the two fields must be 
the same, i.e.
\begin{equation}
  \label{two-point-func-form}
  \expval{\phi_1(z)\phi_2(w)} 
  = \frac{C_{12}}{(z_1-z_2)^{2h}}\, ,
  \quad
  \text{if}
  \quad
  h_1 = h_2 = h\, . 
\end{equation}
The same treatment can be used to obtain the
form of three-point function for the 2 dimensional
primary operators. 
For chiral part, we have
\begin{equation}
  \expval{\phi_1(z_1)\phi_2(z_2)\phi_3(z_3)}
  =\frac{C_{123}}{z_{12}^{h_1+h_2-h_3}
  z_{23}^{h_2 + h_3 - h_1} z_{13}^{h_3+ h_1- h_2}}
  \, ,
\end{equation}
where $z_{ij} = z_i - z_j $.  
the four points function and beyond are
more complex, because of
the existence of anharmonic ratios.

Now, we want to apply Ward-Takahashi identity
with respect to primary fields. First, we need 
to write the generators 
\eqref{generator-three-field}
in terms of conformal dimensions.           
Plugging the infinitesimal transformation
of translation, rotation and dilatation into 
\eqref{def-quasi-primary-field}
and rewrite the expression in real coordinate
, it turns out that 
\eqref{generator-three-field}
now becomes
\begin{align}
\label{generator-three-field-2d}
  &\text{translation}:
   \qquad  P_\nu = -i\partial_\nu \, ,
   \notag \\
  &\text{rotation} :
  \qquad L^{\rho\nu}
  =i(x^\rho\partial^\nu
  -x^\nu\partial^\rho)
  +s \e^{\rho\nu}
   \, ,\\
  &\text{dilatation}:
   \qquad
   D = -i x^\nu \partial_\nu 
  -i\Delta
   \, , \notag 
\end{align}
with the spin $s$ and scale factor
$\Delta$ equal to
\begin{equation}
  \label{rel-spin-scale-conformal dimension}
  s= h-\overline{h}\, ,\qquad
  \Delta = h + \overline{h}\, ,
\end{equation}
which accord with the operator relations
\eqref{l0combinations}.
Now, we can apply Ward-Takahashi identity
\eqref{Ward-Takahashi-Identity}
to three transformations above. For 
translation invariance, 
the associated current is just 
energy momentum tensor. Plugging
the translation generator in 
\eqref{generator-three-field-2d} 
to Ward-Takahashi identity, we obtain
\begin{equation}
\label{Ward-identity-translation}
 \partial_\mu\expval{
 	{T^\mu}_\nu X
 }=
\sum_i \delta(x-x_i)
\pdv{x_i^\mu}\expval{X}\, ,
\end{equation} 
where $X$ stands for a product of 
$n$ local quasi-primary fields $\cO_i(x_i)$, 
$$ X(x_1,\ldots,x_n):=\cO_1(x_1)\cdots \cO_n(x_n)~.$$
 This identity holds
after a modification in \eqref{freedom-energy-momentum-tensor}.

If a theory still has Lorentz symmetry, 
we can assume the associated current to
take the form of 
\eqref{current-lorentz-B}.
With the generator
given on the second line of 
\eqref{generator-three-field-2d}
the Ward-Takahashi identity is
\begin{equation}
  \partial_\mu\expval{
  (T^{\mu\nu}x^\rho-
  T^{\mu\rho}x^\nu
  )X}=
  \sum_i \delta(x-x_i)[
   (x^\nu_i\partial_i^\rho
    -
    x_i^\rho\partial_i^\nu )
    \expval{X}
    -
    i \e^{\nu\rho} s_i
    \expval{X}
  ]\, ,
\end{equation}
where $s_i$ is the spin of the 
$i$-th field of the set $X$. 
Using \eqref{Ward-identity-translation},
we can reduce the expression to 
\begin{equation}
  \expval{(T^{\rho\nu}
  	-T^{\nu\rho})X}
  =-i \sum_i
  \delta(x-x_i)s_i \e^{\nu\rho}
  \expval{X}\, .
\end{equation}
We see that the energy momentum tensor 
is symmetric in correlation except 
at the position of other fields. We always
ignore such kinds of differences, since these
exceptional points do not offer contributions
when putting in the contour integral as we 
will see.
Taking the contraction of $\e_{\nu\rho}$
from both sides of above, we reach the 
final expression of Ward-Takahashi identity
of Lorentz transformation for quasi-primary field:
\begin{equation}
\label{Ward-identity-lorentz}
\e_{\mu\nu}\expval{T^{\mu\nu}X}
= -i \sum_{i=1}^{n}s_i \delta(x-x_i)
\expval{X}\, .
\end{equation}

Finally, if a theory has scale invariance, 
using current expression
associated with scale invariance
\eqref{dilatation-current}
and the dilatation generator in 
\eqref{generator-three-field-2d},
the Ward-Takahashi identity becomes
\begin{equation}
  \partial_\mu \expval{
  {T^\mu}_\nu x^\nu X
  }
  =-
  \sum_i \delta(x-x_i)
  \big[
  	x^\nu_i 
  	\pdv{x^\nu_i}\expval{X}
  	+\Delta_i
  	\expval{X}
  \big]\, ,
\end{equation}
with $\Delta_i$ the scaling dimension
of the $i$th local operator. Using \eqref{Ward-identity-translation},
the identity reduces to
\begin{equation}
\label{Ward-identity-scale}
  \expval{{T^\mu}_\mu X}=
  -\sum_i \delta(x-x_i)
  \Delta_i
  \expval{X}\, .
\end{equation}
We see that the energy momentum tensor is
traceless in correlation function except 
some positions of other fields. 
Now, we have three Ward-Takahashi identity
associated with three important global 
invariances in conformal transformations.
We want to rewrite these identities in
terms of complex coordinates by using
coordinate transformation
\eqref{coordinate transformation}.
Metric after transformation has been obtained
in \eqref{metric-complex-space}, and antisymmetric tensor can be gained by 
using transformation rule of tensor:
\begin{equation}
  \e_{\alpha\beta}
  =
  \mqty(
  0 & \frac{1}{2}i \\
  -\frac{1}{2}i & 0 
  )
  \, , \qquad
  \e^{\alpha\beta}
  =
  \mqty(
  0 & -2i\\
  2i & 0
  )\, .
\end{equation}
The energy momentum tensor 
$T^{\mu\nu}\leftarrow T^{\alpha\beta}$ 
under coordinate transformation. Writing 
each component explicitly, we have
\begin{align}
  T_{zz} =&  \frac{1}{4}(T_{00}-2iT_{10}- T_{11})
  \, , \\ 
  T_{\bar{z}\bar{z}} =&
  \frac{1}{4}(T_{00}+2iT_{10}-T_{11})
  \, , \\
  T_{z\bar{z}}
  =&
  T_{\bar{z}z}
  =\frac{1}{4}
  (T_{00}+T_{11})
\end{align}
We find the classically traceless of the
energy momentum tensor implies that
$T_{z\bar{z}}=T_{\bar{z}z}=0$ classically.
From the conservation law 
$\partial_\mu T^{\mu \nu} =0$, we see that
$T_{zz}$ and $T_{\bar{z}\bar{z}}$ 
are holomorphic and antiholomorphic respectively.

For the delta functions, we can use the
identity
\begin{equation}
  \delta(x)=\frac{1}{\pi}
  \partial_{\bar{z}} \frac{1}{z}
  =\frac{1}{\pi}\partial_z
  \frac{1}{\overline{z}}\, .
\end{equation}
One may verify the
first one satisfy 
the property of delta function  
if the integrand is holomorphic
by using the Gauss's theorem.
However we will find since
the integrands in the three
Ward-Takahashi identities
we have derived
are independent of $z$ or $\overline{z}$,
we can just choose appropriate
one of the two representations to
obtain the result we want.
The Ward-Takahashi identity associated
with translation symmetry 
\eqref{Ward-identity-lorentz}  
now split into two expression:
\begin{align}
  \label{temp-Ward-translation-1}
  2\pi \partial_z \expval{
  T_{\bar{z}z} X}
  +
  2\pi \partial_{\bar{z}}
  \expval{T_{zz} X}
  &=
  -\sum_{i} \partial_{\bar{z}}
  \frac{1}{z-w_i}\partial_{w_i}
  \expval{X}\, ,\\
  \label{temp-Ward-translation-2}
  2\pi \partial_z \expval{
  	T_{\bar{z}\bar{z}} X}
  +
  2\pi \partial_{\bar{z}} \expval{
  	T_{z\bar{z}} X}
  &=
  -\sum_{i} \partial_z
  \frac{1}{\bar{z}-\bar{w_i}}
  \partial_{\bar{w_i}}
  \expval{X}\, ,
\end{align}
where $T_{zz}\cdots$ are just new components
of energy momentum tensor after coordinate
transformation $x\rightarrow z$ and local coordinates $x_i\rightarrow w_i$. 
Ward-Takahashi identity of rotation and
dilatation can be obtained similarly.
\begin{align}
\label{temp-Ward-scale}
  2\expval{T_{z\bar{z}}X}
  +
  2\expval{T_{\bar{z}z}X}
  &=
  -\sum_i \delta(x-x_i)\Delta_i\expval{X}
  \, , \\
  \label{temp-Ward-rotation}
  -2\expval{T_{z\bar{z}}X}
  +2\expval{T_{\bar{z}z}X}
  &=
  -\sum_i \delta(x-x_i)s_i
  \expval{X}\, .
\end{align}
If we add and subtract the last two
equations of 
\eqref{Ward-identity-lorentz}
and
\eqref{Ward-identity-scale},
by using 
\eqref{rel-spin-scale-conformal dimension}
for quasi-primary field
and replace appropriate delta function,
we find
\begin{align}
  \label{traceless-condition-correlation1}
  2\pi\expval{T_{\bar{z}z}X} &=
  -\sum_i \partial_{\bar{z}}
  \frac{1}{z-w_i}h_i
  \expval{X}\, ,\\
  \label{traceless-condition-correlation2}
  2\pi\expval{T_{z\bar{z}}X} &=
  -\sum_i \partial_{z}
  \frac{1}{\overline{z}-\overline{w}_i}
  \overline{h}_i
  \expval{X}\, .
\end{align}
From above relation, we find the traceless
condition for energy momentum tensor is still
correct in the correlation function, except 
the positions of other field points.
Inserting these relations back to
\eqref{temp-Ward-translation-1}
and
\eqref{temp-Ward-translation-2},
we find
\begin{align}
  \partial_{\bar{z}}\bigg\{
  \expval{T(z,\bar{z})X}-
  \sum_i
  \big[
  \frac{1}{z-w_i}\partial_{w_i}
  \expval{X}
  +
  \frac{h_i}{(z-w_i)^2}
  \expval{X}
  \big]
  \bigg\}
  &=0 \, ,\\
  \partial_{z}\bigg\{
  \expval{\overline{T}(z,\bar{z})X}-
  \sum_i
  \big[
  \frac{1}{\overline{z}-\overline{w}_i}
  \partial_{\overline{w}_i}
  \expval{X}
  +
  \frac{\overline{h}_i}{(\overline{z}-\overline{w}_i)^2}
  \expval{X}
  \big]
  \bigg\}
  &=0 \, ,
\end{align}
where we have introduced
\begin{equation}
  T=-2\pi T_{zz}\, , \quad
  \overline{T} = - 2\pi T_{\bar{z}\bar{z}}
  \, .
\end{equation}
From the expression above , we see 
$T$ and $\overline{T}$
are holomorphic and antiholomorphic 
respectively in the correlation function 
, which is consistent with the classical
conclusion. 
In fact, we can always write an operator
in holomorphic form in correlation function
if it proves to be holomorphic classically. 

Furthermore, we can write 
an important expression for quasi-primary
field.
\begin{equation}
 \expval{T(z)X} = \sum_i
 \bigg\{
   \frac{1}{z-w_i}\partial_{w_i}
   \expval{X}
   +
   \frac{h_i}{(z-w_i)^2}\expval{X}
 \bigg\}
 +\text{reg.}\, ,
\end{equation}
where "reg." stands for a holomorpohic function
of z which is regular at $z= w_i$. The
antiholomorphic counterpart is similar.
This expression gives the singular
behavior of the correlator of fields
$T(z)$ with a list of primary field
$X$ as z approaches the point $w_i$.
This implies the OPE (operator product expansion) 
of the energy-momentum
tensor with primary fields by removing 
the bracket $\expval{\cdots}$
\begin{equation}
 T(z)X \sim \sum_i
\bigg\{
\frac{1}{z-w_i}\partial_{w_i}
 X
+
\frac{h_i}{(z-w_i)^2} X
\bigg\}
\, .
\end{equation}
The symbol $\sim$ means the regular terms 
are ignored. OPE also contains an infinite
numbers of regular terms which, for 
energy-momentum tensor, can not be obtained
from the conformal Ward identity. 
The operator product expansion, or OPE, 
is the representation of a product of operator,
describes the infinite fluctuations when each
point tend toward each other.
In general,
we would write the OPE of two fields $A(z)$, 
$B(w)$ as
\begin{equation}
  A(z)B(w) = \sum_{n=-\infty}^{\infty}
  \frac{\{AB\}_n(w)}{(z-w)^n} \, ,
\end{equation}
where the composite fields $\{AB\}_n(w)$ are 
nonsingular at $w = z$.
In particular for a single primary field $\phi$
with conformal dimensions $h$ and $\overline{h}$
, we have:
\begin{align}
\label{OPE-T-primaryfield-z}
  T(z)\phi(w,\overline{w})&\sim 
  \frac{h}{(z-w)^2}\phi(w,\overline{w})
  +
  \frac{1}{z-w}\partial_w \phi(w,\overline{w})
  \, ,\\
\label{OPE-T-primaryfield-barz}
  \overline{T}(\overline{z})
  \phi(w,\overline{w})&\sim 
  \frac{\overline{h}}
  {(\overline{z}-\overline{w})^2}
  \phi(w,\overline{w})
  +
  \frac{1}{\overline{z}-\overline{w}}
  \partial_{\overline{w}} \phi(w,\overline{w})
  \, .
\end{align}
The expression above can be treated as an
alternative definition for primary field.
For a secondary field, but with conformal dimension
$h$ and $\overline{h}$, 
its OPE with energy-momentum tensor has more
singular terms, i.e.

\begin{align}
\label{OPE-T-seconfield-z}
T(z)\phi(w,\overline{w})&\sim \cdots+
\frac{h}{(z-w)^2}\phi(w,\overline{w})
+
\frac{1}{z-w}\partial_w \phi(w,\overline{w})
\, ,\\
\label{OPE-T-seconfield-barz}
\overline{T}(\overline{z})
\phi(w,\overline{w})&\sim \cdots +
\frac{\overline{h}}
{(\overline{z}-\overline{w})^2}
\phi(w,\overline{w})
+
\frac{1}{\overline{z}-\overline{w}}
\partial_{\overline{w}} \phi(w,\overline{w})
\, .
\end{align}

\subsection{Conformal Ward-Takahashi identity}

In this section, we are going to write the 
Ward-Takahashi identities for general conformal invariance- conformal Ward-Takahashi identity.
By observing from 
\eqref{current-lorentz-B} and
\eqref{dilatation-current},
we find for a conformal transformation
with infinitesimal variation
$\e^\mu$, its current can be written as
\begin{equation}
  j^\mu_a(x)\omega_a = T^{\mu\nu}\e_\nu\, .
\end{equation}
In fact, the above relation is satisfied 
for not only translation, dilatation and
rotation, but for arbitrary conformal variation.
Plugging into 
\eqref{Ward-identity-ver2},
we obtain the conformal Ward-Takahashi identity
\begin{equation}
  \delta_\e\expval{X}
  =\int_M \dd^d x \partial_\mu
  \expval{T^{\mu\nu}(x)\e_\nu(x)X}\, .
\end{equation}
Here the integral is taken over the domain 
M containing the positions of all the fields 
in the $X$. Using Gauss's theorem and in
2 dimensions, one find
\begin{equation}
  \delta_{\e,\overline\e}\expval{X}
  = \frac{1}{2} i \int_c
  \{
   -dz\expval{T^{\bar{z}\bar{z}}\e_{\bar{z}}X}
   +d\bar{z}\expval{T^{zz}\e_z X}
  \}
  \, ,
\end{equation}
where we have defined 
$\e(z)= \e^z$ and 
$\overline{\e}(\overline{z}) = \e^{\overline{z}}$,
respectively. Note that 
$\expval{T_{\bar{z}z}X}$ and
$\expval{T_{z\bar{z}}}$ do not contribute
to the contour integrals from
\eqref{traceless-condition-correlation1}
and
\eqref{traceless-condition-correlation2}
since the contours do not exactly go through
the positions contained in $X$. Finally, we get
\begin{equation}
\label{Conformal-Ward-Identity-2d}
  \delta_{\e,\overline\e}\expval{X}
  =- \frac{1}{2\pi i}
  \oint_C \dd z \,\e(z)\expval{T(z)X}
  +
  \frac{1}{2\pi i}
  \oint_C \dd \overline{z}\, 
  \overline{\e}(\overline{z})
  \expval{\overline{T}(\overline{z})X}\, ,
\end{equation}
where the contour $C$ should contain all the
positions of fields in $X$. Notice that the 
validity of \eqref{Conformal-Ward-Identity-2d}
extends beyond primary fields, thus it may be
taken as a definition of the effect of conformal
transformations on an arbitrary local field 
within a correlation function.

If $X$ is primary field :
 $\phi(z,\overline{z})$, 
plugging the OPE relation 
\eqref{OPE-T-primaryfield-z}
and 
\eqref{OPE-T-primaryfield-barz}
back to conformal Ward-Takahashi identity
\eqref{Conformal-Ward-Identity-2d},
we can verify the result should be the
same with the variation of a primary field 
\eqref{trans-inf-quasi-primary}.
\subsection{Example: Free boson}

In this section, we will study free boson 
field as an example. We first review some basic
concepts in field theory.
\subsubsection*{Calculation of Two Point Function}

Two point function is a correlation function 
of two fields. Therefore, from the definition of 
correlation function 
\eqref{def-correlation-function}, we
can easily write the definition of two point 
function for Bosonic field as
\begin{equation}
\label{def-two-point-function}
  \expval{\mathcal{T}(\phi(x_1)\phi(x_2))}
  \frac{1}{Z}\int [\dd \phi]
  \phi(x_1)\phi(x_2) \exp{-S[\phi]}\, .
\end{equation}
The time ordering operation $\mathcal{T}$ is automatically
satisfied in a path integral.
Then, we can define a generating functional
for the theory:
\begin{equation}
\label{def-generating-functional}
  Z[J] = \int [\dd\phi] \exp{-(S-\int \dd ^d x \phi(x) J(x)}\, ,
\end{equation}
where $J(x)$ is an auxiliary "current". Then the
two point function can be generated bu functional
derivative of $J$ :
\begin{equation}
  \label{two-point-correlation-functional-derivative}
  \expval{\phi(x_1)\phi(x_2)}=
  Z[0]^{-1} \frac{\delta}{\delta J(x_1)}
  \frac{\delta}{\delta J(x_2)} Z[J]|_{J=0}\, .
\end{equation} 
For calculating the two point function more 
generally, we first assume the action $S$ take 
the form of following:
\begin{equation}
  \label{general-form-action}
  S = \frac{1}{2}
  \int \dd^d x \dd^d y\, 
  \phi(x)A(x,y)\phi(y)\, .
\end{equation}
Let's introduce a variable transformation
$\phi\rightarrow \phi + \phi_0$
without changing the path integral measure and
result as we have illustrated before. The term 
in the exponential of generating functional 
becomes:
\begin{align}
  -S+\int \dd^d x \phi(x)J(x)
  \rightarrow& 
  -\frac{1}{2}\int \dd^d x \dd^d y
  \phi(x)A(x,y)\phi(y) 
  -\int \dd^d x \dd^d y
  \phi(x)A(x,y) \phi_0(y)-
  \notag \\
  &- \frac{1}{2}\int \dd^d x \dd^d y
  \phi_0(x)A(x,y)\phi_0(y)
  +\int \dd^d x\phi_0(x)J(x)
  +\int \dd^d x \phi(x)J(x)\, ,
  \notag
\end{align}
Now we choose $\phi_0$ to satisfy the equation:
\begin{equation}
 \int \dd^d y A(x,y) \phi_0(y) = J(x)\, 
  \rightarrow
 \phi_0(x) = \int \dd^d y A^{-1}(x,y)J(y)\, .
\end{equation}
Then, then the generating functional can be
written as 
\begin{equation}
  Z[J] = N \exp{\frac{1}{2}
  	\int \dd^d x \dd^d y \,
  	J(x) A^{-1}(x,y) J(y)
  }\, ,
\end{equation}
with $N$ independent of $J$
\begin{equation}
  N = \int [\dd \phi] 
  \exp{-\frac{1}{2} \int \dd^d x \dd^d y \,
  \phi(x) A(x,y) \phi(y)}\, .
\end{equation}
Therefore, by using 
\eqref{two-point-correlation-functional-derivative},
we have
\begin{equation}
\label{two-point-function-calculation}
  \expval{\mathcal{T}(\phi(x_1)\phi(x_2))}
  =A^{-1}(x,y)
\end{equation} 
We will apply the formula above to calculate the
two-point function of some
specific theories.

\subsubsection*{Two-Point function in Free Boson}
In two dimensions, the free boson has the following 
Euclidian action:
\begin{equation}
\label{def-Free-boson action}
  S = \frac{1}{2}g\int \dd^2 x 
  \{
     \partial_\mu \varphi 
     \partial^\mu \varphi
     +m^2 \varphi^2
  \}\, ,
\end{equation} 
With $g$ a normalization parameter. Now 
$\varphi$ is a scalar field, i.e. remains invariant
under rotation and translation.
We focus on massless case $m=0$, in which
the theory becomes conformal.
Comparing with \eqref{general-form-action}, we have
\begin{equation}
  A(x,y) = -g \delta^{(2)}(x-y)\square \, ,
\end{equation} 
From \eqref{two-point-function-calculation}, we
can calculate the two-point function 
$K(x,y) \equiv \expval{\varphi(x_1)\varphi(x_2)}$
by solving the following equation:
\begin{equation}
  -g \square K(x,y) = \delta^{(2)}(x-y)\, ,
\end{equation}
From \eqref{tran-correlation-function} and
the trivial nature of $\mathcal{F}$ for a scalar field, 
$K(x,y)$ is invariant under translation and rotation.
Thus, we can write $K(x,y)\equiv K(\rho)$
with $\rho = \abs{x-y}$,
and integrate over $x$ within a Disk of radius $r$
around y. We find
\begin{align}
  1 &= 2\pi g \int^{r}_0 \dd \rho \rho
  \bigg(
    -\frac{1}{\rho}\pdv{\rho}(\rho K'(\rho))
  \bigg)
  \notag \\
    &= 2 \pi g(-rK'(r))\, .
\end{align}   
The solution of two-point function for massless free boson
can be obtained up to an additive constant,
\begin{equation}
  \label{two-point-function-free-Boson}
  \expval{\varphi(x)\varphi(y)}=-\frac{1}{4\pi g}
  \ln{\abs{x-y}}^2\, .
\end{equation} 

\subsubsection*{Wick's Theorem}
In the previous part, we have introduced two-point function
for a theory. In this part, we are going to briefly review
the context of Wick's Theorem.
We will see two-point function 
is a basic quantity of a theory
due to Wick's Theorem.

In quantum field theory, a free(elementary)
field operator can be split
into two part:
\begin{equation}
  \phi(x) = \phi^{+}(x)+\phi^{-}(x)\, ,
\end{equation}
with $\phi^+(x)$ which contains only annihilation operators
and $\phi^-(x)$ contains only creation operators.
A normal-ordered operator $::$ is defined to put all the 
annihilation operators to the right side of creation
operators.  For instance,
\begin{equation}
  :\phi(x_1)\phi(x_2): = \phi(x_1)^+\phi(x_2)^+ 
  + \phi(x_1)^-\phi(x_2)^- + \phi(x_1)^-\phi(x_2)^+
  + \phi(x_2)^-\phi(x_1)^+ \, .\notag
\end{equation}
And the time ordering product for $x_1^0 > x_2^0$ gives
\begin{equation}
  \mathcal{T}(\phi(x_1)\phi(x_2))
  =\phi(x_1)^+\phi(x_2)^+ 
  + \phi(x_1)^-\phi(x_2)^- + \phi(x_1)^-\phi(x_2)^+
  + \phi(x_1)^+\phi(x_1)^- \, .\notag
\end{equation}
Thus, we have the relation for time ordering product and
normal ordering product for  $x_1^0 > x_2^0$ as
\begin{equation}
  \mathcal{T}(\phi(x_1)\phi(x_2))
  =:\phi(x_1)\phi(x_2):
  + \comm{\phi(x_1)^+}{\phi(x_2)^-}\, .
\end{equation}
We know from some specific field theory that
the commutator from right side of above is
a constant. And, by taking the expectation
value of both sides, we find the constant
is just a two-point function, i.e.
\begin{equation}
  \mathcal{T}(\phi(x_1)\phi(x_2)) = 
  :\phi(x_1)\phi(x_2): +
  \expval{\phi(x_1)\phi(x_2)}\, .
\end{equation} 
For $x^0_1 < x^0_2$, we get the same expression. The above relation can be generalized by defining 
the contraction of two operators in different space time,
\begin{equation}
  \contraction{}{\phi_1}{}{\phi_2}
  \phi_1\phi_2
  \equiv 
  \expval{\phi_1\phi_2}\, .
\end{equation}
The Wick's theorem states that
the time-ordered product of operators in different
space time is equal to the normal-ordered
product, plus all possible ways of contracting pairs of fields
within it. For instance,
\begin{align}
  \mathcal{T}(\phi_1\phi_2\phi_3\phi_4)
  =&
  :\phi_1\phi_2\phi_3\phi_4:
  +:
  \contraction{}{\phi_1}{}{\phi_2}
  \phi_1\phi_2\phi_3\phi_4
  :
  +:
  \contraction{}{\phi_1}{\phi_2}{\phi_3}
  \phi_1\phi_2\phi_3\phi_4
  :
  +:
  \contraction{}{\phi_1}{\phi_2\phi_3}{\phi_4}
  \phi_1\phi_2\phi_3\phi_4
  :\notag \\
  &+:
  \contraction{\phi_1}{\phi_2}{}{\phi_3}
  \phi_1\phi_2\phi_3\phi_4
  :
  +:
  \contraction{\phi_1}{\phi_2}{\phi_3}{\phi_4}
  \phi_1\phi_2\phi_3\phi_4
  :
  +:
  \contraction{\phi_1\phi_2}{\phi_3}{}{\phi_4}
  \phi_1\phi_2\phi_3\phi_4
  :
  +:
  \contraction{}{\phi_1}{}{\phi_2}
  \contraction{\phi_1\phi_2}{\phi_3}{}{\phi_4}
  \phi_1\phi_2\phi_3\phi_4
  :
  \notag\\
  &+:
  \contraction{}{\phi_1}{\phi_2}{\phi_3}
  \contraction[2ex]{\phi_1}{\phi_2}{\phi_3}{\phi_4}
  \phi_1\phi_2\phi_3\phi_4
  :
  +:
  \contraction{}{\phi_1}{\phi_2\phi_3}{\phi_4}
  \contraction[2ex]{\phi_1}{\phi_2}{}{\phi_3}
  \phi_1\phi_2\phi_3\phi_4
  :
  \notag \, .
\end{align}
When applying Wick's theorem, there might be a case
that some normal-ordered products may appear in left side 
of the time-ordered product, within which all the operators
are in the same space time. In this case, we can ignore the
normal ordering left side, but do not contract the pair operators
with same space time. For example,
\begin{align}
  &\mathcal{T}(
  :\phi_1(x)\phi_2(x)::\phi_3(y)\phi_4(y):)
  \notag\\
  =&
  :\phi_1(x)\phi_2(x)\phi_3(y)\phi_4(y):
  +:
  \contraction{}{\phi_1}{(x)\phi_2(x)}{\phi_3}
  \phi_1(x)\phi_2(x)\phi_3(y)\phi_4(y)
  :
  +:
  \contraction{}{\phi_1}{(x)\phi_2(x)\phi_3(y)}{\phi_4}
  \phi_1(x)\phi_2(x)\phi_3(y)\phi_4(y)
  :
  \notag\\
  &+:
  \contraction{\phi_1(x)}{\phi_2}{(x)}{\phi_3}
  \phi_1(x)\phi_2(x)\phi_3(y)\phi_4(y)
  :
  +:
  \contraction{\phi_1(x)}{\phi_2}{(x)\phi_3(y)}{\phi_4}
  \phi_1(x)\phi_2(x)\phi_3(y)\phi_4(y)
  :
  +:
  \contraction{}{\phi_1}{(x)\phi_2(x)}{\phi_3}
  \contraction[2ex]{\phi_1(x)}{\phi_2}{(x)\phi_3(y)}{\phi_4}
  \phi_1(x)\phi_2(x)\phi_3(y)\phi_4(y)
  :
  \notag\\
  &+:
  \contraction{}{\phi_1}{(x)\phi_2(x)\phi_3(y)}{\phi_4}
  \contraction[2ex]{\phi_1(x)}{\phi_2}{(x)}{\phi_3}
  \phi_1(x)\phi_2(x)\phi_3(y)\phi_4(y)
  :\notag\, .
\end{align}
The Wick's theorem for fermion is almost the same except that
the exchange of two fermions contributes a minus sign. 






\end{document}
