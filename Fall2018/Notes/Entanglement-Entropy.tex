
\documentclass[2dCFT-lecture.tex]{subfiles}



\begin{document}
%\maketitle
%\abstract{These are lecture notes on 2d conformal field theories in Fall 2018. If you find typos, please email me.}
%


\setcounter{tocdepth}{2}
%\maketitle
\section{Entanglement Entropy}

In this section, we will learn how 2d CFTs are related to entanglement entropy \cite{Calabrese:2004eu} and holography.  There are nice reviews \cite{Calabrese:2009qy,Nishioka:2009un,Takayanagi:2012kg,Solodukhin:2011gn,Harlow:2014yka,Rangamani:2016dms}. At the end of this section, we will introduce the celebrated Ryu-Takayanagi entanglement entropy formula \cite{Ryu:2006bv,Ryu:2006ef}.


\subsection{Introduction to Entanglement Entropy}\label{sec:EE}
In this subsection, we briefly introduce some basic concepts
and properties in entanglement entropy.



In quantum mechanics, the state of a quantum system is represented by a state vector, denoted $| \psi \rangle$, whose time evolution is governed by the Schr\"odinger equation$$
i \frac{\partial}{\partial t} | \Psi \rangle = H | \Psi \rangle
$$
A quantum system with a state vector $ | \psi \rangle$  is called a \textbf{pure state}.

However, it is also possible for a system to be in a statistical ensemble of different state vectors: For example, there may be a 50\% probability that the state vector is $| \psi_1 \rangle$  and a 50\% chance that the state vector is $| \psi_2 \rangle$. This system would be in a \textbf{mixed state}. The density matrix $\rho_{\text{tot}}$ is especially useful for mixed states, because any state, pure or mixed, can be characterized by a single density matrix.




%
%A mixed state is different from a quantum superposition. The probabilities in a mixed state are classical probabilities (as in the probabilities one learns in classic probability theory / statistics), unlike the quantum probabilities in a quantum superposition. In fact, a quantum superposition of pure states is another pure state, for example $| \psi \rangle = (| \psi_1 \rangle + | \psi_2 \rangle)/\sqrt{2} $. In this case, the coefficients  $1/{\sqrt {2}}$ are not probabilities, but rather probability amplitudes.

A density matrix  $\rho_{\text{tot}}$ is a matrix that describes the statistical state of a system in quantum mechanics. For instance, the density matrix in the canonical ensemble of a system with Hamiltonian $H$ and temperature $T$ is
$$
\rho_{\text{tot}}= \frac{e^{- \beta H}}{\cZ}~.
$$
Note that we always normalize $\Tr_{\cH_{\text{tot}}} \rho_{\text{tot}}$. Then, the entropy can be expressed in terms of the density matrix
\be\label{VN-entropy}
S=\frac{E-F}{T}=\frac{- 1}{\cZ} \left( \operatorname {Tr}_{\mathcal {H}_{\textrm{tot}}} \left[ \beta H e^{- \beta H} \right] + Z \log Z \right)=-\rho_{\text{tot}} \log \rho_{\text{tot}}
\ee
which is called \textbf{von Neumann entropy}.



If we express a pure state by the  density operator
\begin{equation}
  \rho=\ket{\phi}\bra{\phi} \, .
\end{equation}
The expectation value of operator $\cO$ can be expressed by
density operator
\begin{equation}
 \expval{\cO}_\rho := \Tr(\rho \cO)\, .
\end{equation}

\begin{figure}[ht]\centering
\includegraphics[width=8cm]{picture/EE-illustration}
	\caption{}
	\label{fig:ee-illustration}
\end{figure}


Let us consider the situation in which a system is divided into two subsystems $A$ and $B$ where  the Hilbert space is simply a direct product
$$
\mathcal {H}_ {\textrm{tot}} = \mathcal {H}_{A} \otimes \mathcal {H}_{B}
$$
Then, we can consider the density operator $\rho_A$  restricted to the subsystem $A$ by taking the trace over $\cH_B$
\begin{equation}
 \rho_A := \Tr_{\cH_B}(\rho_{\text{tot}})\, .
\end{equation}
Then, the \textbf{entanglement entropy} in the subsystem $A$  is defined as
\begin{equation}
  S_A := -\Tr_{\cH_A}(\rho_A \log \rho_A)\, ,
\end{equation}
like the von Neumann entropy \eqref{VN-entropy}.


For example, let $\cH_A$
and $\cH_B$ be two Hilbert spaces spanned by $\{\ket{0}_A,\ket{1}_A\}$ and $\{\ket{0}_B,\ket{1}_B\}$, respectively.
Considering a state
\begin{equation}
\label{entanglement-state}
  \ket{\phi}=\frac{1}{\sqrt{2}}
  (\ket{1}_A \otimes \ket{0}_B+\ket{0}_A\otimes\ket{1}_B)
\end{equation}
However long the subsystems are apart, if measurement in $A$ is $\ket{1}_A$, then
then the measurement in
$B$ is bound to measure $\ket{0}_B$. Hence the subsystem $A$ is
is  \textbf{entangled} with the subsystem $B$. This is called \textbf{quantum entanglement}.



In fact, the density matrix for the state \eqref{entanglement-state},
\bea
\rho_{\text{tot}} =& |\phi\rangle \langle \phi|\cr
=& \frac{1}{2}
(\ket{0}_A\bra{0}_A\otimes\ket{1}_B\bra{1}_B
+\ket{0}_A\bra{1}_A\otimes\ket{1}_B\bra{0}_B\notag\\
&+\ket{1}_A\bra{0}_A\otimes\ket{0}_B\bra{1}_B
+\ket{1}_A\bra{1}_A\otimes\ket{0}_B\bra{0}_B)\,,
\eea
and the restricted one is
\begin{equation}
 \rho_A:=\Tr_{\cH_B}(\rho_{\text{tot}})=\frac{1}{2}
 (\ket{0}_A\bra{0}_A+ \ket{1}_A\bra{1}_A)\,~.
\end{equation}
Thus the entanglement entropy can be evaluated as
\begin{equation}
  S_A=\log 2\,.
\end{equation}
This is indeed related to the fact that both the Hilbert spaces $\cH_A$ and $\cH_B$ are two-dimensional. In fact, the maximal entanglement entropy is given by $\min(\log \dim \cH_A,\log \dim \cH_B )$. Therefore, the state \eqref{entanglement-state} is maximally entangled.


On the other hand,  a state $\ket{\psi}$ is called \textbf{separate} if it can be written as
the direct product of two pure states, i.e
\begin{equation}
 \ket{\psi}=\ket{\phi}_A \otimes \ket{\phi}_B \, ,
 \qquad
 \ket{\phi}_A \in \cH_A\, \&
 \ket{\phi}_B \in \cH_B\, .
\end{equation}
In this case, it is easy to check that
the entanglement entropy  vanishes, which
means there is no entanglement between both systems.


In conclusion, the information lost by the trace over the Hilbert space $\cH_B$ is somehow restored  in $S_A$ by means of quantum entanglement. Hence,
The entanglement entropy plays a very important role when we cannot make observation of a subsystem $B$.



There are some equivalent formula that will be useful in
later calculation:
\bea
S_A=\lim_{n\to 1} \frac{\Tr_{\cH_A} \rho^n_A -1}{1-n}
=-\pdv{}{n}\Tr_{\cH_A} \rho^n_A|_{n=1}
=- \pdv{}{n}\log \Tr_{\cH_A} \rho_A^n|_{n=1}\,.
\eea



\subsubsection*{Properties of Entanglement Entropy}
There are several useful properties which the entanglement entropy
enjoys generally. We summarize some of them as follows.
\begin{itemize}
\item
If the density matrix $\rho_{\text{tot}}$ is constructed from a pure state such as in the
zero temperature system, then we find the following relation
assuming $B$ is the complement of $A$
\begin{equation}
\label{complement-prop}
 S_A=S_B \, .
\end{equation}
This shows that the entanglement entropy is not an extensive
quantity. For a mixed state, this relation does not hold.

\item
For any three subsystems $A$, $B$ and $C$ that do not intersect each
other, the following inequalities hold:
\bea
\label{strong-subadditivity-property}
S_{A+B+C}+S_B &\leq S_{A+B}+S_{B+C}\,,\cr
S_A+S_C &\leq S_{A+B}+S_{B+C}\, .
\eea
These inequalities are called the \textbf{strong subadditivity}.

\item
By setting $B$ empty in \eqref{strong-subadditivity-property}, we can find
the subadditivity relation
\begin{equation}
\label{subadditivity relation}
S_{A+B} \leq S_A+S_B \, .
\end{equation}
\item
The subadditivity allows us to define an quantity called \textbf{mutual
information} $I(A,B)$ by
\begin{equation}
  I(A,B)=S_A+S_B-S_{A+B}\geq 0 \, .
\end{equation}
\end{itemize}

\subsection{Conformal Field Theory on $\bZ_n$ Orbifold}
Before studying entanglement entropy in conformal field theory, we shall generalize the twist operator discussed in \S\ref{sec:Ising} to the $\bZ_n$ orbifold. In \S\ref{sec:Ising}, the continuum limit of the Ising model is described by the free fermion $\psi$, and the presence of a spin field $\sigma$ introduce the $\bZ_2$ twist condition to the free fermion in the complex plane:
\begin{equation}
  \psi(e^{2\pi i}z)=- \psi(z) \, ,
\end{equation}
which means the field operator produces a minus sign once being taken around the origin. This is a global effect and the field operator is double-valued in the space time. The $\bZ_2$ twist operator $\sigma$ can be generalized to any finite group. In particular, we
are interested in $\bZ_n$ twists, which is defined as following
\begin{equation}
\label{ZN-orbifold-definition}
 X(e^{2\pi i}z, e^{-2\pi i}\overline{z}) =
 e^{2\pi k i /N} X(z,\overline{z})\, ,
\end{equation}
where $k$ is the integer called monodromy of the field $X$.
For the definition of \eqref{ZN-orbifold-definition} makes
sense, $X$ must be complex. Therefore, We combine two copies of free boson $\varphi_1$, $\varphi_2$ to introduce the complex free boson:
\begin{equation}
X=\varphi_1(z,\overline{z})+i \varphi_2(z,\overline{z})\, ,
\qquad
\overline{X}=\varphi_1(z,\overline{z})-i \varphi_2(z,\overline{z})\, .
\end{equation}
$\varphi_1$ $\varphi_2$ are real scalar field. We have study
these field in Chapter 4. Thus, the properties for complex free
boson are easy to derive. Let us list some of them in brief.
First the holomorphic fields for complex free boson are as following:
\def\pd{\partial}
\begin{equation}
  \pd X(z)=\pd \varphi_1(z)+i \pd \varphi_2(z) \, .
\end{equation}
The holomorphic energy-momentum tensor is
\begin{equation}
 T(z)=\frac{1}{2} \big(
 :\pd\varphi_1 \pd\varphi_1:
 +
 :\pd\varphi_2\pd\varphi_2:
 \big)
=-\frac{1}{2}
 :\pd X \pd \overline{X} : \, ,
\end{equation}
which is just the sum of the energy-momentum tensor
of two real scalar field. The central charge of this theory
becomes $2$. The OPE of $\pd X$ and $\pd\overline{X}$ is
\begin{equation}
  \pd X(z) \pd X(w) \sim
  -\frac{2}{(z-w)^2} \, .
\end{equation}
In a similar fashion to \eqref{def-EM-tensor 2}, the vacuum expectation value of energy-momentum is
\begin{equation}
 \expval{T(z)} := \bigg[
 \expval{
 -\frac{1}{2} \pd X(z) \pd \overline{X}(w)
}
-\frac{1}{(z-w)^2}
 \bigg]_{z=w}\,.
\end{equation}
To generalize the definition, we need to replace the original vacuum state with the twist
ground state. The twist ground state
for $\bZ_n$ orbifold is defined by acting a twist operator on the vacuum state at the zero point.
\begin{equation}
  \sigma_{k/N}(0)\ket{0}=\ket{\sigma_{k/N}}\, .
\end{equation}
$\sigma_{k/N}$ is defined to twist the field $X$ by
$e^{2\pi i k/N}$ when $X$ turn around it counterclockwise.
The conjugate state then is given by inserting anti-twist
operator $\sigma_{-k/N}$  at infinity which is denoted by
$\bra{\sigma_{-k/N}}$. By introducing twist operators, the global ground effect can be illustrated by inserting local operator \cite[ISZ88-No.47]{Dixon:1986qv}.

\def\sm{\bra{\sigma_{-k/N}}}
\def\sp{\ket{\sigma_{k/N}}}

The expectation value of energy-momentum in the $\bZ_n$ twist
ground state now can be evaluated as
\begin{equation}
\label{def-vacuum-expectation-e-m-tensor-complex-boson}
  \frac{\bra{\sigma_{-k/N}} T(z) \ket{\sigma_{k/N}}}
 {\bra{\sigma_{-k/N}}\ket{\sigma_{k/N}}} :=
  \bigg[
  \frac{\sm -\frac{1}{2}\pd X(z)\pd \overline{X}(w) \sp}
 {\bra{\sigma_{-k/N}}\ket{\sigma_{k/N}}}
 -\frac{1}{(z-w)^2}
  \bigg]_{z=w}\, ,
\end{equation}
where the normalized factor in the denominator can be adjusted
to $1$.
We have calculated this quantity in the R sector of free fermion. To perform the calculation we need to study mode expansion of $X$ first.

From \eqref{ZN-orbifold-definition}, we immediately obtain that
\bea
 \pd X(e^{2\pi i} z , e^{-2\pi i}\overline{z})
 &= e^{2\pi i(k/N-1)} \pd X(z,\overline{z})\, ,\\
 \pd \overline{X} (e^{-2\pi i}\overline{z},e^{2\pi i} z)
 &= e^{-2\pi i(k/N+1)} \pd \overline{X}(\overline{z},z)\, .
\eea
Therefore, the Laurent expansions of $\pd X$ and $\pd \overline{X}$
must have the form
\bea
\label{ZN-twist-mode}
 i\pd_z X &= \sum_{m\in\mathbb{Z}}
 \alpha_{m-k/N} z^{-m-1+k/N} \, , \notag\\
 i\pd_z \overline{X} &=\sum_{m\in\mathbb{Z}}
 \overline{\alpha}_{m+k/N} z^{-m-1-k/N}\, .
\eea
The mode operators have the following canonical commutation
relations
\begin{equation}
 \comm{\overline{\alpha}_{m+k/N}}{\alpha_{n-k/N}}
 =2(m+k/N)\delta_{m,-n}\, ,
\end{equation}
where $2$ comes from the fact that complex boson field
consists of two real scalar fields. And the twist ground
state $\sp$ is annihilated by all the positive frequency
mode operators
\bea
  \alpha_{m-k/N} \sp &= 0\, , \qquad m>0 \,,\\
  \overline{\alpha}_{m+k/N}\sp &=0\, , \qquad m\geq 0\, .
\eea
Using the mode expansion \eqref{ZN-twist-mode} and their
properties we mentioned above, we can calculate the expectation
value in the twist ground state
\begin{equation}
 -\frac{1}{2}\sm\pd X(z)\overline{\pd}X(w)\sp
=z^{-(1-k/N)}w^{-k/N}
 \bigg[
 \frac{(1-k/N)z+k w/N}{(z-w)^2}
 \bigg] \, .
\end{equation}
From \eqref{def-vacuum-expectation-e-m-tensor-complex-boson},
we can compute
\begin{equation}
\label{T-expval-twist-ground}
 \sm T(z) \sp =\frac{1}{z^2}\cdot \frac{1}{2}\frac{k}{N}\bigg(1-\frac{k}{N}\bigg)\, .
\end{equation}
From previous study, we know that as a local operator, $T \sigma_{k/N}$ OPE should have the following form
\begin{equation}
 T(z)\sigma_{k/N}(0) =\cdots \frac{h \sigma_{k/N}(0)}{z^2}+\cdots\, ,
\end{equation}
where $h$ is the conformal dimension for twist field $\sigma_{k_N}$. Plugging into \eqref{T-expval-twist-ground},
we immediately obtain the conformal dimension for $\sigma_{k/N}$ is $ \frac{1}{2}\frac{k}{N}\big(1-\frac{k}{N}\big)$. The same analysis can be applied for  for $\sigma_{-k/N}$ and the anti-holomorphic sector.

\subsection{Entanglement Entropy in 2d CFTs}
Now, we are going to study the Entanglement Entropy in
2d CFTS. We define a subsystem $A$ at fixed time $t=t_0$, and
call its complement $B$. The boundary of $A$ is denoted by $\pd A$. Then we can define the entanglement entropy $S_A$ by the
following formula recall that $\Tr_{\cH_A}\rho_A =1$.
\bea
\label{SA-cal-formula}
 S_A &= \lim_{n\to 1} \frac{\Tr_{\cH_A} \rho^n_A -1}{1-n}\notag\\
 &=-\pdv{}{n}\Tr_{\cH_A} \rho^n_A|_{n=1}
 =- \pdv{}{n}\log \Tr_{\cH_A} \rho_A^n|_{n=1}\,.
\eea
This is called the \textbf{replica trick}. Therefore, what we need to
do is to evaluate $\Tr_{\cH_A}\rho^n_A$ in our 2D system.

This can be done in the path integral formalism as follows.
We first assume that $A$ is the single interval $x\in[u,v]$
at $t_E=0$ in the flat Euclidean coordinates $(t_E,x)\in \mathbb{R}^2$. The ground state wave function can be
found by path integrating from $t_E=-\infty$ to $t_E =0$
\begin{equation}
\Psi(\phi_0(x))=\int_{t_E=-\infty}^{\phi(t_E=0,x)=\phi_0(x)}
D\phi e^{-\cS(\phi)}\,,
\end{equation}
where $\phi(t_E,x)$ denotes the field which defines the 2D
CFT. $\phi_0$ is the value of the field which depends on the
spacial position $x$. The total density matrix $\rho$ is given
by two copies of the wave function
$[\rho]_{\phi_0\phi_0'}=\Psi(\phi_0)\overline{\Psi}(\phi'_0)$.
The complex conjugate one $\overline{\Psi}$ can be obtained by
path-integrating from $t_E=\infty$ to $t_E=0$. To obtain
the reduced density matrix $\rho_A$, we need to integrate
$\phi_0$ on $B$ assuming $\phi_0(x)=\phi'_0(x)$ when $x \in B$.

\begin{equation}
\label{reduce-density-matrice}
 [\rho_A]_{\phi_+\phi_-}
 =(\cZ_1)^{-1} \int^{t_E=\infty}_{t_E=-\infty}
 D\phi e^{-\cS(\phi)}\prod_{x\in A}
 \delta(\phi(+0,x)-\phi_+(x))\dot
 \delta(\phi(-0,x)-\phi_-(x))\,,
\end{equation}
where $\phi_+$ and $\phi_-$ are both boundary function in $A$, and $\cZ_1$ is the vacuum partition function on $\mathbb{R}^2$ in order to make $\Tr\rho_A =1$. The computation is sketched in
Figure \ref{fig:9-1}.



\begin{figure}[ht]
	\centering
	\includegraphics[width=0.3\linewidth]{picture/9-1}
\qquad
		\includegraphics[width=0.3\linewidth]{picture/9-2}
	\caption{Left: the path integral representation of the
	reduced matrix $[\rho_A]_{\phi_+\phi_-}$ . Right: the $n$- sheeted Riemann surface $\mathcal{R}_n$}
	\label{fig:9-1}
\end{figure}


To  find $\Tr_{\cH_A} \rho^n_A$, we can prepare $n$ copies of
\eqref{reduce-density-matrice}
\begin{equation}
 [\rho_A]_{\phi_{1+}\phi_{1-}}
  [\rho_A]_{\phi_{2+}\phi_{2-}}
  \cdots
   [\rho_A]_{\phi_{n+}\phi_{n-}}\,,
\end{equation}
and take the trace successively. In the path-integral formalism
this is realized by gluing $\{\phi_{i\pm}(x)\}$ as
$\phi_{i-}(x)=\phi_{(i+1)+}(x) (i=1,2,\cdots, n)$ and integrating $\phi_{i+}(x)$. In this way, $\Tr_{\cH_A} \rho^n_A$
is given in terms of the path-integral on an $n$-sheeted Riemann
surface $\mathcal{R}_n$ (Figure \ref{fig:9-1})

%\begin{figure}
%	\centering
%	\includegraphics[width=0.3\linewidth]{picture/9-2}
%	\caption{The $n$- sheeted Riemann surface $\mathcal{R}_n$}
%	\label{fig:9-1}
%\end{figure}

\begin{equation}
 \Tr_{\cH_A} \rho^n_A=(\cZ_1)^{-n} \int_{(t_E,x)\in\mathcal{R}_n}
 D\phi e^{-\cS(\phi)} \equiv \frac{\cZ_n}{(\cZ_1)^n}\,.
\end{equation}
To evaluate the path-integral on $\mathcal{R}_n$, it is useful
to introduce replica fields. Let us first take $n$ disconnected
sheets. The field on each sheet is denoted by $\phi_k(t_E,x) (k=1, 2,\cdots n)$\, . In order to obtain a CFT on the flat complex plane $\mathbb{C}$ which is equivalent to the present
one on $\mathrm{R}_n$, we impose the twisted boundary conditions
\begin{equation}
\phi_k (e^{2\pi i}(w-u))=\phi_{k+1}(w-u)\,, \qquad
\phi_k(e^{2\pi i}(w-u))=\phi_{k-1}(w-v) \, .
\end{equation}
Assuming that $\phi$ is a complex scalar field with central charge $c=2$, then we
can introduce $n$ new fields
$\wt{\phi}_k=\frac{1}{n}\sum_{l=1}^n e^{2\pi ilk/n} \phi_l$. They obey the boundary condition
\begin{equation}
 \wt{\phi}_k(e^{2\pi i}(w-u))=e^{2\pi ik/n}
 \wt{\phi}_k(w-u)\,,
 \qquad
 \wt{\phi}_k(e^{2\pi i}(w-v))=e^{-2\pi i k/n}
 \wt{\phi}_k(w-v)\, .
\end{equation}
As we have learned in the last section the system is equivalent to $n$- disconnected sheets with two twist operators $\sigma_{k/n}$ and $\sigma_{-k/n}$ inserted in the $k$-th sheet
for each values of $k$. In the end we find
\begin{equation}
\label{trrhonA}
 \Tr_{\cH_A} \rho^n_A=\prod_{k=0}^{n-1}
 \expval{\sigma_{k/n}(u)\sigma_{-k/n}(v)}\sim
 (u-v)^{-4\sum_{k=0}^{n-1}\Delta_{k/n}}
 =(u-v)^{-\frac{1}{3}(n-1/n)}\,,
\end{equation}
where $\Delta_{k/n}= \frac{1}{2}(\frac{k}{n})^2+\frac{1}{2}\frac{k}{n}$ is the
conformal dimension of $\sigma_{k/n}$. When we have $m$ such
complex scalar fields we simply obtain
\begin{equation}
\label{trace-rho-A-n}
 \Tr_{\cH_A} \rho^n_A \sim (u-v)^{-\frac{c}{6}(n-1/n)}\, ,
\end{equation}
setting the central charge $c= 2m$.
Applying the formula \eqref{SA-cal-formula} to \eqref{trace-rho-A-n}, we find that
\begin{equation}
 S_A \sim \frac{c}{3}\log \ell\,
\end{equation}
where we set $\ell\equiv u-v$. Refine the UV cut-off $\e$ into the expression above we get
\begin{equation}
\label{EE-infinite}
 S_A=\frac{c}{3}\log \frac{\ell}{\e}\, .
\end{equation}


\subsubsection*{Entanglement Entropy of Finite Size}
In previous section, we assume the space direction $x$ is
infinite, while in this section we consider the subsystem
$A$ is defined in a finite space region.
\begin{equation}
 A=\{
 x| x\in [r, s]
 \}\, ,
\end{equation}
where we assume $-\frac{L}{2} < r < s \leq \frac{L}{2}$.
The system is related to the previous system via the
conformal map
\begin{equation}
 w=\tan\bigg(
 \frac{\pi \w '}{L}
 \bigg)\, .
\end{equation}
And $u=\tan \big(
\frac{\pi r}{L}\big)$ and
$v=\tan\big(\frac{\pi s}{L}\big)$.
From \eqref{trrhonA}, we can calculate $\Tr_{\cH_A} \rho^n_{A\omega'} $
by applying conformal transformation of local twist field.
\bea
\Tr_{\cH_A} \rho^n_{A\omega'}
&= \prod_{k=0}^{n-1}
\bigg(
\dv{w}{w'}
\bigg)^{2\Delta_{k/n}}_r
\bigg(
\dv{w}{w'}
\bigg)^{2\Delta_{k/n}}_s
\expval{\sigma_{k/n}(u) \sigma_{-k/n}(v)}\notag\\
& \sim \bigg[
\frac{L}{\pi}\cos\bigg(\frac{\pi}{L}r\bigg)
\cos\bigg(\frac{\pi}{L}s\bigg) (u-v)
\bigg]^{-\frac{c}{6}(n-1/n)}\notag\\
& \sim \bigg[
\frac{L}{\pi} \sin\frac{\pi}{L}(r-s)
\bigg]^{-\frac{c}{6}(n-1/n)}\,.
\eea
Therefore, following the same calculation step, we
get the entanglement entropy in a finite space region
\begin{equation}\label{EE-circle}
S_A=\frac{c}{3} \cdot \log\bigg(
\frac{L}{\pi \e} \sin\bigg(
\frac{\pi \ell}{L}
\bigg)
\bigg)\, ,
\end{equation}
where $\ell= r-s$. It is invariant under the exchange $\ell\to L-\ell$,
and thus satisfy the property \eqref{complement-prop}.

\subsubsection*{Entanglement Entropy at Finite Temperature}
Considering in the Euclidean 2-dimension theory, the Euclidean
time is equivalent to the inverse of temperature. At finite temperature $T=\beta^{-1}$, we can compactify the Euclidean
time as $t_E \sim t_E+\beta$. We can map this system
to the infinite system via the conformal map
\begin{equation}
 w=e^{\frac{2\pi}{\beta} w'}\, .
\end{equation}
And $u=e^{\frac{2 \pi r}{\beta}}$ ,
$v=e^{\frac{2\pi s}{\beta}}$. $t_E$ is the imaginary part of
$w'$.This conformal map will
lead to the extra factor
\begin{equation}
\bigg[
  \frac{\beta}{2 \pi} e^{-\frac{\pi}{\beta}}
\bigg]^{-\frac{c}{6}(n-1/n)}
\end{equation}
Follow the similar calculation steps, we obtain
\begin{equation}\label{entropytemp}
S_A=\frac{c}{3} \log
\bigg(
\frac{\beta}{\pi \e}
\sinh\bigg(
\frac{\pi \ell}{\beta}
\bigg)
\bigg)\, ,
\end{equation}
where $\ell= r-s$. In the zero temperature limit $T\to 0$,
this reduces to the previous result \eqref{EE-infinite}.
In the high temperature limit  $T-> \infty$, it
approaches
\begin{equation}
S_A \simeq \frac{c}{3} \log \frac{\beta}{2 \pi \epsilon} +
\frac{\pi c}{3}\ell T\,.
\end{equation}
This is the same as the thermal entropy for the subsystem $A$
as expected.

\subsection{Zamolodchikov $c$-theorem revisited}

In \S\ref{sec:Zc}, we have learned Zamolodchikov $c$-theorem stating that in a unitary, rotational (Lorentz) invariant 2d theory, there exists a monotonically decreasing function $C$ with respect to the length scale $R$
$$
\frac{d C (R )}{d R} \leq 0
$$
where   $C(R=0)$ is the central charge at ultra-violet and $C(R=\infty)$ is the one at infra-red. This theorem can be shown easily by using the strong subadditivity \eqref{strong-subadditivity-property} of entanglement entropy.



\begin{figure}[ht]\centering
\includegraphics[width=9cm]{picture/13-1}
\caption{The black lines represent subsystems and the blue lines express the light cone}\label{fig:subsystem}
\end{figure}

We consider the subsystems $A$ and $B$ as in Figure \ref{fig:subsystem}. Although $A$ and $B$ do not lie at the constant time slice in Figure \ref{fig:subsystem}, an appropriate Lorentz boost will bring them at the constant time slice so that we can define the Hilbert space $\mathcal{H}_{A} $ and $ \mathcal{H}_{B}$.  The Hilbert spaces $\mathcal{H}_{A} \cup \mathcal{H}_{B}$ and  $\mathcal{H}_{A} \cap \mathcal{H}_{B}$ are defined on $A\cup B$ and $A\cap B$.
Let us write the Lorentz-invariant length of the subsystem $A$ by $\ell(A)$. Then, a simple computation yields
\be
\ell(A ) \cdot \ell(B )=\ell(A \cup B ) \cdot \ell(A \cap B )~.
\ee
In particular, when
$$
\ell(A )=\ell(B )=e^{\frac{a+b}{2}} , \quad \ell(A \cup B )=e^{a} , \quad \ell(A \cap B )=e^{b}~,
$$
the strong subadditivity \eqref{strong-subadditivity-property} implies that
$$
2 S \left(\frac{a+b}{2} \right) \geq S (a )+S (b )~.
$$
Thus, $S (R)$ is concave and
$$
\frac{d^{2}}{d R^{2}} S (R) \leq 0
$$
where $\ell =e^R$.
If we define the function
$$
C(R )=3 \frac{d S (R)}{d R}~,
$$
then it is monotonically decreasing
$$
\frac{dC}{d R}  \leq 0~.
$$
Comparing with \eqref{EE-infinite}, it is equal to the central charge at the fixed point.  This corresponds to Zamolodchikov $C$-function \eqref{c-fn}.

















\subsection{Black hole thermodynamics and Bekenstein-Hawking entropy}

The aforementioned Ryu-Takayanagi formula evaluates entangle entropy of a CFT from the holographic viewpoint. First, we shall  provide a brief introduction to black hole thermodynamics and AdS/CFT correspondence based on which Ryu-Takayanagi formula is constructed. However, we glimpse only a tip of iceberg  and the subject would actually deserve the entire one semester. If you are interested in this fertile subject, we refer to the standard references \cite{wald2010general,Townsend:1997ku,Aharony:1999ti}.



Let us recall the Schwarzschild black hole of mass $M$ with metric
$$
d s ^{2}=-\left( 1-\frac{r_{H}}{r}\right) c^2d t ^{2}+ \frac{d r ^{2}}{1-\frac{r_{H}}{r}}+r ^{2}d \Omega ^{2}
$$
where the radius of the horizon is given by
$$
r_{H}= \frac{2 GM}{c^2}~.
$$
To see the thermodynamical property of the Schwarzschild black hole, we use the naive trick.
It turns out that much of the interesting physics having to
do with the quantum properties of black holes comes from the
region near the event horizon.
To examine the region \emph{near the horizon $r_H$}, we analytically continued to the Euclidean metric $t= -it_E$, and we set
$$
r-\frac{2GM}{c^2}=\frac{x^2c^2}{8GM}~.
$$
Then, the metric near the event horizon $x\ll1$
$$
ds^2_{\textrm{E}} \approx  (\kappa c x)^2dt_E^2+dx^2
+\frac{1}{4\kappa^2}d\Omega^2~,
$$
where $\kappa=\frac{c^2}{4GM}$ is called the \textbf{surface gravity} because it is indeed the acceleration of a static
particle near the horizon as measured at spatial infinity.
The first part of the metric is just $\bR^2$ with polar coordinates if we make the
{periodic identification}
$$
t_E \sim t_E +\frac{2\pi}{c\kappa}~.
$$
Using the relation between Euclidean periodicity and temperature,
we can deduce \textbf{Hawking temperature} of the Schwarzschild black hole
\begin{equation}\label{hawktemp}
T_H = {\hbar c\kappa \over 2 \pi}={\hbar  c^3\over 8\pi GM}~.
\end{equation}
This is a very heuristic way to introduce the Hawking temperature which is not  found originally in this way.


If the black hole has temperature, then it obeys the thermodynamics law. In general, a stationary black hole can have electric charge $Q$ and angular momentum $J$.  Bardeen, Carter, and Hawking \cite{Bardeen:1973gs} noticed that the laws of black hole mechanics with mass $M$,  angular momentum $J$,  and charge $Q$ bears  a striking resemblance with the
three laws of thermodynamics. This is quite
surprising because \emph{a priori} there is no reason to expect
that the spacetime geometry of black holes has anything to do with
thermal physics.

\begin{enumerate}
\item[{(0)}] Zeroth Law: In thermal physics, the zeroth law states
that the temperature $T$ of a body at thermal equilibrium is
constant throughout the body. Correspondingly for stationary black holes, one can
show that surface gravity $\kappa$ is constant on the event
horizon.

\item[{(1)}] First Law: Energy is conserved, $dE=TdS+\mu dQ +
\Omega dJ$,  where $E$ is the energy, $Q$ is the charge with chemical
potential $\mu$ and  $J$ is the angular momentum with chemical potential
$\Omega$. Correspondingly for black holes, one has $dM ={\kappa
\over 8\pi G} dA+\mu dQ+\Omega dJ$. Here $A$ is the area of the horizon,  and $\kappa$ is the surface gravity, $\mu$ is the chemical potential conjugate to $Q$, and $\Omega$ is the angular velocity conjugate to $J$.

\item[{(2)}] Second Law: In a physical process the total entropy
$S$ never decreases, $\Delta S \geq 0$. Correspondingly for black
holes one can prove the area theorem that the net area in any process never
decreases,  $\Delta A \geq 0$. For example,  two Schwarzschild
black holes with masses $M_1$ and $M_2$  can coalesce to form a
bigger black hole of mass $M$. This is consistent with the area
theorem, since the area is proportional to the square of the mass,
and $(M_1+M_2)^2 \geq M^2_1+M^2_2$. The opposite process where
a bigger black hole fragments is however disallowed by this  law.
\end{enumerate}



\begin{table}[ht]
\centering
\begin{tabular}{|c|c|}
  % after \\: \hline or \cline{col1-col2} \cline{col3-col4}...
\hline
\textbf{Laws of Thermodynamics} & \textbf{Laws of Black Hole Mechanics}\\
 \hline Temperature is constant
 & Surface gravity is constant
 \\
 throughout a body at equilibrium. &  on the event horizon.\\
 $T$=
constant. & $\kappa$ =constant.\\
 \hline
Energy is conserved.
& Energy is conserved. \\
$dE=T dS+\mu dQ+\Omega dJ. $& $dM=\frac{\kappa}{8\pi} dA+\mu dQ+\Omega dJ . $\\
 \hline
 Entropy never decrease.  & Area never decreases.\\
 $\Delta S \geq 0$. & $ \Delta A \geq 0 $. \\
 \hline
\end{tabular}
\caption{\small{Laws of Black Hole Thermodynamics}}
\label{blackholelaws}
\end{table}






This result can be understood as one of the highlights of general relativity.
Moreover, Hawking has shown this is indeed more than an analogy \cite{Hawking:1974sw}. There is a deep connection
between black hole geometry, thermodynamics and quantum mechanics.
Quantum mechanically, a black hole is not quite black. Hawking has applied techniques of quantum field theories on a curved background to the near horizon region of a black hole and
showed that a black hole indeed radiates \cite{Hawking:1974sw}. In a
quantum theory, particle-antiparticle are constantly being created
and annihilated even in vacuum. Near the horizon, an antiparticle
can fall  in once in a while and the particle can escapes to
infinity. Although this lecture does not deal with Hawking's calculation unfortunately (see \cite{Townsend:1997ku}), it actually revealed that the spectrum
emitted by the black hole is  precisely thermal with temperature
\eqref{hawktemp}.

In fact, because the energy of the black hole is equal to its mass, the first law of
thermodynamics can be written as
$$
T_{H}d S=c^2d M
$$
Using the formula
for the Hawking temperature \eqref{hawktemp}, the Schwarzschild black hole has the entropy
$$
S=4 \pi \frac{GM ^{2}}{c}~.
$$
The entropy is proportional to the area $A$ of the horizon
\be\label{BH-entropy}
S=\frac{c^3 \operatorname{Area} }{4G\hbar} ~.
\ee
This is a universal result for any black hole, and this remarkable relation between the thermodynamic properties
of a black hole  and its geometric properties is called the celebrated \textbf{Bekenstein-Hawking entropy formula} \cite{bekenstein1972black,Bekenstein:1973ur,Hawking:1974sw}.
This formula involves all three fundamental constants of
nature, and this is the first place where the Newton constant $G$ meets with the Planck constant $\hbar$.



In thermodynamics, the energy and entropy are examples of \textbf{extensive properties} \textit{i.e.} they are proportional to the size of the system. However, the Bekenstein-Hawking entropy is proportional to the area. Moreover, in general relativity, the maximal entropy in a certain subsystem $R$ is proportional to the area of $\partial R$
$$
S\le \frac{\operatorname{Area}(\partial R)}{4G}~,
$$
which is called \textbf{entropy bound}. Subsequently, people come up with the idea that a gravity theory in $(d+1)$-dimension is equivalent to a quantum field theory without gravity  in $d$-dimension, which is called the \textbf{holographic principle}.




%
%
%Bekenstein asked a simple-minded but incisive question. If we throw a bucket of hot water
%into a black hole, then the net entropy of the world outside would
%seem to decrease, treating the black hole as a geometric object.  Do we have to give up  the second law of
%thermodynamics in the presence of black holes?
%
%Note that since the black hole carries mass, the total energy is conserved during the process so that it does not violate the first law of thermodynamics.
% This suggests that one can save
%the second law of thermodynamics if somehow the black hole also
%has entropy. Following this reasoning with the
%analogy between the area of the black hole and entropy, Bekenstein proposed that a black hole
%must have entropy proportional to its area \cite{Bekenstein:1973ur}.
%
%If a black hole
%has energy $E$ and entropy $S$, then it must also have temperature
%$T$ given by
%\[
%{1 \over T} ={\partial S \over \partial E}.
%\]
%For example, for  a Schwarzschild black hole, the area and the
%entropy scales as $ S \sim M^2$. Therefore, one would expect
%inverse temperature that scales as $M$
%$$
%{1 \over T} ={\partial S \over \partial M} \sim{\partial M^2
%\over
%\partial M} \sim M.
%$$
%Moreover, if the black hole has temperature like any hot body, it
%must thermally radiate. The understanding of thermal properties of black holes requires the treatment beyond classical general relativity.



%
%For ordinary objects, Boltzmann has given statistical interpretation of the thermodynamic entropy of a system.
%We fix the macroscopic parameters (e.g. total electric
%charge, energy etc.) and count the number $\Omega$ of quantum
%states-known as microstates-each of which has the
%same values for the macroscopic parameters, and the entropy is expressed as
%\[
%S=k \log \Omega,
%\]
%where $k$ is Boltzmann constant. Since
%the Bekenstein-Hawking entropy behaves in every other respect like
%the ordinary thermodynamic entropy, it is therefore natural to ask
%whether the entropy of a black hole has a similar
%statistical interpretation.
%
%
%
%Furthermore, one of the most dramatic results of Hawking's work was the implication that black holes are associated with information loss. Physically speaking, we can associate information with pure states in quantum mechanics.
%If we throw in a pure quantum state in the s-wave to form a
%black hole, and after the black hole evaporates completely,
%it comes out as a thermal (mixed) state.  Thus the net result of this
%process is the evolution of a pure quantum state into a
%mixed state, which violates the law (unitarity) of quantum mechanics. This is called \textbf{information paradox}  \cite{Hawking:1976ra}.
%In fact, the information paradox stems from the absence of such a microscopic description in case of thermal radiation from a black hole.
%
%
%In order to investigate the microscopic description of black hole
%entropy we need a quantum theory of gravity. This is precisely what string theorists have attempted to do and have been partially successful.
%
%
%








































\subsection{AdS/CFT correspondence}

The AdS/CFT correspondence \cite{Maldacena:1997re} is a special case of the holographic principle, which states that a quantum gravity in an $\AdS_d$ space is equivalent to a CFT in $d$-dimensions. Even though the correspondence was originally proposed in the context of string theory, it is now generalized in broader contexts. This subsection briefly introduce to a basics of the  AdS/CFT correspondence in order to pave the way to the Ryu-Takayanagi formula.




\subsubsection*{AdS space}
To begin with, let us investigate geometry of AdS spaces.
An anti-de Sitter (AdS) space is a maximally symmetric manifold with constant negative scalar curvature.
It is a solution of Einstein's equations for an empty universe with negative cosmological constant.
The easiest way to understand it is as follows.


A Lorentzian AdS$_{d+1}$ space can be illustrated by the hyperboloid in $(2,d)$ Minkowski space:
\be
 X_0^2 +X_{d+1}^2 -\sum_{i=1}^{d} X_i^2 = R^2 \ .
 \label{embeding}
\ee
The metric can be naturally induced from the Minkowski space
\be
 ds^2 = -dX_0^2 -dX_{d+1}^2 +\sum_{i=1}^{d} dX_i^2 \ .
\ee
By construction it has $\SO(2,d)$ isometry, which is a first connection to
the conformal group in $d$-dimensions studied in \S\ref{sec:conf-gen}.


\subsubsection*{Global coordinate}




Simple solution to (\ref{embeding}) is given as follows.
\bea\nonumber
 &X_0^2 +X_{d+1}^2=R^2 \cosh^2 \rho \ , \cr
 &\sum_{i=1}^{d} X_i^2=R^2 \sinh^2 \rho \ .
\eea
Or, more concretely,
\bea\nonumber
 X_{0} &= R \cosh \rho\ \cos \tau \ , \qquad
 X_{d+1}=R \cosh \rho\ \sin \tau \ ,  \nonumber \cr
 X_i &= R \sinh \rho\ \Omega_{i} \quad (i=1,\cdots,d,
 \text{and} \sum_i \Omega_i^2=1).
\eea
These are $S^{1}$ and $S^{d-1}$ with radii $R\cosh\rho$ and $R\sinh\rho$, respectively.
The metric is
\bea\nonumber
 ds^2 &= R^2 \left(-\cosh^2 \rho \ d\tau^2 +d\rho^2 +\sinh^2 \rho \ d\Omega_{(d-1)}^2 \right) \ .
\eea
Note that $\tau$ is a periodic variable and if we take $0 \le \tau <2\pi$
the coordinate wrap the hyperboloid precisely once.
This is why this coordinate is called \textbf{global coordinate}.
The manifest sub-isometries are $\SO(2)$ and $\SO(d)$ of $\SO(2,d)$.
To obtain a causal spacetime, we simply unwrap the circle $S^1$,
namely, we take the region $-\infty < \tau < \infty$ with no identification,
which is called \textbf{universal cover} of the hyperboloid.


In literatures another global coordinate is also used,
which can be derived by redefinitions
$r \equiv R \sinh \rho$ and $dt \equiv R d\tau$:
\bea\nonumber
 ds^2 &=-f(r) dt^2 +\frac{1}{f(r)} dr^2+r^2 d\Omega_{(d-1)}^2 \ , \qquad
 f(r)=1+\frac{r^2}{R^2} \ .
\eea




\subsubsection*{Poincar\'e coordinates}

There is yet another coordinates, called \textbf{Poincar\'e coordinates}.
As opposed to the global coordinate, this coordinate covers only the half of the hyperboloid.
It is most easily (but naively) seen in $d=1$ case:
\bea\nonumber
 x^2 -y^2=R^2 \ ,
\eea
which is the hyperbolic curve.
The curve consists of two isolated parts in regions $x>R$ and $x<-R$.
We simply use one of them to construct the coordinate.



Let us get back to general $d$-dim.
We define the coordinate as follows.
\bea
 X_{0} &= \frac{1}{2u} \left(1+u^2 \left(R^2 +x_i^2-t^2 \right) \right) \ , \cr
 X_{i} &= R u x_i \qquad (i=1,\cdots,d-1) \ , \cr
 X_{d} &= \frac{1}{2u} \left(1-u^2 \left(R^2 -x_i^2+t^2 \right) \right) \ , \cr
 X_{d+1} &= R u t \ ,
\eea
where $u > 0$.
As it is stated
the coordinate covers the half of the hyperboloid; in the region, $X_0 > X_{d}$.
The metric is
\bea\label{ads-poincare}
 ds^2 &= R^2 \left(\frac{du^2}{u^2} +u^2 (-dt^2 +dx_i^2) \right)
=R^2 \left(\frac{du^2}{u^2} +u^2 dx_\mu^2 \right) \ .
\eea
The coordinates $(u,t,x_i)$ is called \textbf{Poincar\'e coordinates}.
This metric has manifest $ISO(1,d-1)$ and $\SO(1,1)$ sub-isometries of $\SO(2,d)$;
the former is the Poincar\'e transformation and the latter corresponds to the dilatation
\bea\label{scaling}
 (u,t,x_i) \to (\lambda^{-1}u,\lambda t,\lambda x_i) \ .
\eea


If we further define $z=1/u$ ($z>0$), then,
\bea\label{ads-poincare2}
 ds^2 &= \frac{R^2}{z^2} \left(dz^2 +dx_\mu^2 \right) \ .
\eea
This is called \textbf{the upper (Poincar\'e) half-plane model}.
The hypersurface given by $z=0$ is called \textbf{(asymptotic) boundary} of the AdS space,
which corresponds to $u \sim r \sim \rho=\infty$.


\subsubsection*{AdS/CFT correspondence}
The AdS/CFT correspondence \cite{Gubser:1998bc,Witten:1998qj} is the an exact duality between a quantum gravity in an asymptotically $\AdS_{d+1}$ spacetime and a $\CFT_d$ without
gravity.  It is \textbf{holographic} since the gravitational
theory lives in one extra dimension. It is often useful to think that the CFT lives at the \textbf{boundary} of the  \textbf{the bulk} AdS space.
Indeed, the CFT lives in a spacetime parameterized by $x=(z, \vec{x})$, whereas gravity
fields are functions of $\vec{x}$ and the radial coordinate $z$.
For instance,  the scaling transformation of the CFT can be translated into the transformation of the AdS coordinate \eqref{ads-poincare2}
$$
x_{\mu} \rightarrow \lambda x_{\mu} , \quad z \rightarrow \lambda z~.
$$
Hence, the coordinate $z$ parametrizes the energy scale, and the ultra-violet limit corresponds to the boundary at $z=0$. As $z$ increases, the energy scales decreases.

\begin{figure}[ht]\centering
\includegraphics[width=6cm]{picture/AdS-CFT}\qquad
\includegraphics[width=7.5cm]{picture/scaling-transf}
\caption{}\label{fig:AdS-CFT}
\end{figure}
%
%However, it is not quite accurate to say that the CFT
%lives on the boundary, for two reasons. First, we should not think about having both
%theories at once; we either consider a CFT or we have a gravity on the AdS spacetime, never both at the
%same time. Second, the CFT is dual to the entire gravity theory so that in a sense it lives
%everywhere.

The theories are believed to be entirely
equivalent: any physical (gauge-invariant) quantity that can be computed in one theory
can also be computed in the dual. As typical feature of a duality, computation of physical quantity becomes much easier on one side than on the other side.
More precisely, the gravitational partition function on asymptotically AdS space is equal to the generating function of correlation functions of the corresponding CFT:
\bea\label{GKPW}
Z_{\textrm{grav}}[\phi\to \phi_0]
=
\Bigg\langle \exp \biggl (\int _{\partial AdS} \overline \phi_0 \cO \biggr)
\Bigg\rangle_{\textrm{CFT}}
\eea
that is called \textbf{GKPW} relation.
For any bulk field $\phi$ in gravity theory on AdS, there exists the corresponding an operator
$\cO$ in the CFT. The gravitational partition function can be schematically written as
$$
Z_{\textrm{grav}}[\phi\to \phi_0]=\int_{\phi\to \phi_0}\cD\phi ~e^{-S_{\textrm{string}}[\phi]}~.
$$
%For instance, in the regime $\lambda\gg1$, we can use supergravity description
%$$
%Z_{\textrm{grav}}[\phi\to \phi_0]=\sum_{\textrm{saddle point}}~e^{-S_{\textrm{SUGRA}}[\phi\to \phi_0]}~.
%$$


\subsubsection*{Bulk field/boundary operator}
Each field propagating on AdS space
is in a one to one correspondence with
an operator in the field theory. The spin of the
bulk field is equal to the spin of the CFT operator; the mass of the bulk field fixes the
scaling dimension of the CFT operator. Here are some examples:

\begin{itemize}
\item  Every theory of gravity has a massless spin-2 particle, the graviton $g_{\m\n}$. This
is dual the stress tensor $T_{\m\n}$ in CFT. This makes sense since every CFT has a stress
tensor. The fact that the graviton is massless corresponds to the fact that the CFT
stress tensor is conserved.
\item  If our theory of gravity has a spin-1 vector field $A_\m$, then the dual CFT has a
spin-1 operator $J_\m$. If $A_\m$ is gauge field (massless), then $J_\m$
 is a conserved current so that gauge symmetries in the
bulk correspond to global symmetries in the CFT.
\item A bulk scalar field  is dual to a scalar operator in CFT. The boundary
value of
acts as a source in CFT.
\end{itemize}
For more details, we refer the reader to \cite{Aharony:1999ti}.























\subsection{Ryu-Takayanagi formula}
The Bekenstein-Hawking entropy encodes the information restored in the interior of the horizon. As explained in \S\ref{sec:EE}, the entanglement entropy can be used when the observer cannot access to a certain subsystem like the interior of the horizon. Hence, it is natural to seek the relation between the Bekenstein-Hawking entropy and the  entanglement entropy. As the first step to understand the relation, one can also ask how the $\AdS_{d+1}$ space time encodes the entanglement entropy in the subsystem $A$ of a $\CFT_d$. The answer to this question is the Ryu-Takayanagi formula \cite{Ryu:2006bv,Ryu:2006ef}:
\begin{equation}
  \label{R-T formula}
S_{A}=\frac{\operatorname{Area} \left(\gamma_{A} \right)}{4 G^{(d+1 )}}.
\end{equation}
The manifold $\gamma_A$ is the $d$-dimensional static submanifold with minimal area in $\mathrm{AdS}_{d+1}$ whose
boundary is given by $\pd A$ as shown in Figure \ref{fig:holographicEE}. Its area is denoted by $\operatorname{Area}(\gamma_A)$. $G^{(d+1)}_N$ is the $d+1$ dimensional Newton constant. Since $\sqrt{g}$ diverges at the boundary of AdS space $z=0$,
the area $\gamma_A$ is divergent there. Thus, the result still has divergent of order $\e^{-(d-2)}$
$$\begin{aligned} S_{A}& \sim \frac{R^{d-1}}{G } \operatorname{Area} \left(\partial{A} \right) \int_{\epsilon} \frac{d z}{z^{d-1}}\\ & \sim \frac{R^{d-1}}{G}\frac{\operatorname{Area} \left(\partial{A} \right)}{\e^{d-2}} \end{aligned}~.$$

\begin{figure}[ht]
	\centering
	\includegraphics[width=8cm]{picture/holographicEE}
	\caption{Holographic entanglement entropy}
	\label{fig:holographicEE}
\end{figure}



As we will see at the end, when the region $A$ is large enough, the minimal surface (submanifold) wraps the horizon of AdS black hole, and  the  Ryu-Takayanagi formula can explain the Bekenstein-Hawking entropy as a special case. However, minimal surfaces (submanifolds) are more general so that   Ryu-Takayanagi formula can be interpreted as a generalization of the Bekenstein-Hawking entropy formula. Moreover, the entanglement entropy depends on a CFT and it also encodes the quantum entanglement. Therefore, the entanglement entropy in a CFT can be also understood as the quantum correction to the Bekenstein-Hawking entropy in the context of the AdS/CFT correspondence.




\subsubsection*{Holographic derivation of strong subaddivity}
One of the most important properties of the entanglement entropy is the
strong subadditivity  \eqref{strong-subadditivity-property}. Using the Ryu-Takayanagi formula, one can easily derive
strong subadditivity.

Let us start with three regions $A$, $B$ and $C$ on a spatial slice
that do not intersect with each other as shown
in Figure \ref{fig: holographic proof}. We extend this boundary setup
toward the bulk AdS. First consider the entanglement entropy $S_{A+B}$
and $S_{B+C}$. From the  Ryu-Takayanagi formula, they are given by the minimal area
surfaces $\gamma_{A+B}$ and $\gamma_{B+C}$ which satisfy
$\pd{\gamma_{A+B}} = \pd(A+B)$ and $\pd\gamma_{B+C} = \pd(B+C)$ as before.
Then we divide these two minimal surfaces into four pieces and
recombine into (i) two surface $\gamma'_B$ and $\gamma'_{A+B+C}$ or
(ii) two surfaces $\gamma'_A$ and $\gamma'_C$, corresponding to
two different ways of the combination. But now $\gamma'_X$s are
not the minimal area surface, we have $\text{Area}(\gamma'_X) \geq \text{Area}(\gamma_X)$,
therefore, we have
\begin{align*}
 \text{Area}(\gamma_{A+B}) + \text{Area}(\gamma_{B+C})
 &= \text{Area}(\gamma'_{B}) + \text{Area}(\gamma'_{A+B+C})
 \geq \text{Area}(\gamma_B) + \text{Area}(\gamma_{A+B+C})\, ,\\
 \text{Area}(\gamma_{A+B}) + \text{Area}(\gamma_{B+C})
 &= \text{Area}(\gamma'_{A}) + \text{Area}(\gamma'_{C})
 \geq \text{Area}(\gamma_A) + \text{Area}(\gamma_{C})\, .
\end{align*}

\begin{figure}[ht]
	\centering
	\includegraphics[width=8cm]{picture/SSA}
	\caption{Holographic proof of strong subadditivity}
	\label{fig: holographic proof}
\end{figure}


\subsubsection*{Holographic entanglement entropy}
We focus our discussion in $\mathrm{AdS}_3/\mathrm{CFT}_2$ case, and the $\mathrm{AdS}_3$ space is
defined in the Poincare coordinates.
$$
d s^{2}=R^{2} \left(\frac{d z^{2}+d x^{2}-d t^{2}}{z^{2}} \right).
$$
At the fixed time $t_0$, the whole spatial region of $\mathrm{CFT}_2$ is an infinite line in the $x$ direction.
We pick up subsystem $A$ along $x$ direction: $-l/2 \leq x \leq l/2$ with its boundary coordinates given by
$$
\begin{aligned} P : (t , x , z ) &=\left(t_{0} ,-\frac{\ell}{2} , \epsilon \right), \\
 Q : (t , x , z ) &=\left(t_{0} , \frac{\ell}{2} , \epsilon \right). \end{aligned}
$$

Now $\operatorname{Area}(\gamma_A)$ in \eqref{R-T formula} is the length of geodesic line in the AdS space,
with two fixed points $P$ and $Q$ . Therefore, we can apply the Ryu-Takayanagi formula to compute the entanglement entropy
for subsystem $A$
\begin{equation}
S_{A}=\frac{2 R}{4 G} \int_{\epsilon}^{\ell/ 2} \frac{d z}{z \sqrt{1-4 z^{2} /\ell^{2}}}
=\frac{R}{2 G} \log \left(\frac{\ell}{\epsilon} \right),
\end{equation}
where we have introduce $\epsilon$ as the UV cutoff.
Since the AdS${}_3$/CFT${}_2$ correspondence identifies the central charge with the Newton constant
$$
c=\frac{3 R}{2 G}~,
$$
it is equal to \eqref{EE-infinite}.

\begin{figure}
	\centering
	\includegraphics[width=0.6\linewidth]{picture/Poincare-half-plane}
	\caption{Holographic entanglement entropy in the Poincare coordinates}
	\label{fig:poincare-disk}
\end{figure}



\subsubsection*{Holographic entanglement entropy on a circle}
In the global coordinate of $\mathrm{AdS}_3$, the metric of spacetime is written as
\be\label{ads3-metric}
d s^{2}=R^{2} \left(- \cosh^{2} \rho d t^{2}+d \rho^{2}+\sinh^{2} \rho d \theta^{2} \right)
\ee
The $\mathrm{CFT}_2$ is identified with the cylinder $(t, \theta)$ at the (regularized)
boundary $\rho = \rho_\infty$. The whole spatial region at fixed time is a circle.
The subsystem $A$ corresponds to $0\leq \theta \leq 2\pi \ell/L$.
In this case, the minimal surface $\gamma_A$ is the geodesic line which connects two
boundary points at $\theta = 0$ and $\theta = 2\pi l /L$ with $t$ fixed
The explicit form of the geodesic $\mathrm{AdS}_3$, expressed in the ambient $\vec{X}\in \mathbb{R}^{2,2}$ space, is
$$
\vec{X}=\frac{R}{\sqrt{\alpha^{2}-1}} \sinh (\lambda / R ) \cdot \vec{x}+R \left[ \cosh (\lambda / R )-\frac{\alpha}{\sqrt{\alpha^{2}-1}} \sinh (\lambda / R ) \right] \cdot \vec{y},
$$
where  $\alpha = 1+ 2 \sinh^2 \rho_0 \sin^2(\pi \ell/L)$ and $x$ and $y$ are defined by

\begin{align*} \vec{x} &=\left(\cosh \rho_{\infty} \cos t , \cosh \rho_{\infty} \sin t , \sinh \rho_{\infty} , 0 \right) \\ \vec{y} &=\left(\cosh \rho_{\infty} \cos t , \cosh \rho_{\infty} \sin t , \sinh \rho_{\infty} \cos (2 \pi\ell/L) , \sinh \rho_{\infty} \sin (2 \pi \ell/L) \right), \end{align*}

The length of geodesic can be found as
$$
 \cosh\left(\frac{\mbox{Length}(\gamma_A)}{R}\right)=1+2 \sinh^{2} \rho_{\infty} \sin^{2} \frac{\pi\ell}{L}$$
Assuming that the UV cutoff energy is large $e^{\rho_\infty} \geqq 1$,  we can obtain the entropy as follows
$$
S_{A} \simeq \frac{R}{4 G} \log \left(e^{2 \rho_{\infty}} \sin^{2} \frac{\pi\ell}{L} \right)=\frac{c}{3} \log \left(\frac{L}{\epsilon} \sin \frac{\pi\ell}{L} \right).
$$
This coincides with entanglement entropy in a finite size $L$ \eqref{EE-circle}.

\begin{figure}
	\centering
	\includegraphics[width=0.6\linewidth]{picture/AdS3-CFT2-holographic}
	\caption{Holographic entanglement entropy on a circle}
	\label{fig:ads3-cft2-holographic}
\end{figure}



\subsubsection*{BTZ black hole}
We have seen that the AdS space has the scaling symmetry \eqref{scaling}. However, one can also consider an asymptotically AdS space whose dual is a QFT which flows to a CFT at IR $z\to 0$. An typical example of asymptotically AdS spaces is an AdS black hole whose metric can be written as
\bea\nonumber
d s^{2} = R^{2}\left(- \frac{f ( z )}{z^{2}} d t^{2} +  \frac{d z^{2}}{z^{2} f ( z )} + \frac{d x^{2}}{z^{2}}  \right) ~,\qquad f ( z ) = 1 - \left( \frac{z}{z_0} \right)^{2}~,\eea
where the horizon is located at $z=z_0$. It is easy to see that the metric asymptotes to the $\AdS_3$ space as $z\to 0$. By the change $r=R/z$ of the coordinates, the metric is written as
$$
d s^{2} = -  (r^{2} - r_{0}^{2})  d t^{2} + \frac{R^{2}}{r^{2} - r_{0}^{2}} d r^{2} + r^{2} d x^{2}~.
$$
We can identify the spatial coordinate $x\sim x+2\pi$, which is called the \textbf{BTZ black hole} and it has temperature
$$
T=1/\beta=\frac{r_0}{2\pi R}~.
$$
Hence, if we use the Euclidean time $\tau=-it$, the time direction is compactified as $\tau\sim \tau+\beta$.
Therefore, the dual $\CFT_2$ is living on a torus boundary parametrized by $(\tau,x)$ with periodicities $\tau\sim\tau+\beta$ and $x\sim x+2\pi$.



Let us consider the Ryu-Takayanagi entropy  for the subsystem $A$
given by $0\leq x\leq 2\pi \ell/R$ at the boundary. To evaluate the length of the geodesic from $x=0$ to $x=2\pi \ell/R$, one can use the fact that the BTZ metric can be changed to the $\AdS_3$ metric \eqref{ads3-metric} by
$$
r\to r_0\cosh\rho~, \quad x\to i R  t~,\qquad \tau \to R \theta~.
$$
Hence, the same analysis above can be applied to compute the length by exchanging $L\leftrightarrow \beta$
$$\cosh\left(\frac{\mbox{Length}(\gamma_A)}{R}\right)
=
1 + 2 \cosh^{2} \rho_{\infty} \sinh^{2} \left( \frac{\pi \ell}{\beta} \right)
$$
The relation between the cut off $\e$ in CFT and
the one $ \rho_{\infty}$ of AdS is given by $e^{\rho_{\infty}} =\frac{\beta}{
\e}$ so that it reproduces the  CFT result
\eqref{entropytemp}.

It is also useful to understand these calculations geometrically.
The geodesic line in the BTZ black hole takes the form shown in
Fig.\ \ref{fig: ads_blackhole.eps}(a). When the size of $A$ is
small, it is almost the same as the one in the ordinary AdS$_3$. As
the size becomes large, the turning point approaches the horizon and
eventually, the geodesic line covers a part of the horizon. This is
the reason why we find a thermal extensive behavior of the entropy when
$\ell/\beta\gg 1$ in (\ref{entropytemp}). The thermal entropy in a
conformal field theory is dual to the black hole entropy in its
gravity description via the AdS/CFT correspondence. In the presence
of a horizon, it is clear that $S_A\neq S_B$ since the corresponding geodesic lines
wrap different parts of the horizon (see Fig.\ \ref{fig:
ads_blackhole.eps}(b)). As explained below \label{complement-property}, this is expected for a mixed state like a black hole.


We further expect that when $A$ becomes very large before it coincides
with the total system, $\gamma_A$ becomes separated into the horizon circle and
a small half circle localized on the boundary  (see Fig.\ \ref{fig:
ads_blackhole.eps}(c)) $$S_{B} - S_{A} = S_{\textrm{BH}}~.$$
Consequently, when $A$ is the entire boundary, the Ryu-Takayanagi formula \eqref{R-T formula} reproduces the Bekenstein-Hawking entropy \eqref{BH-entropy}.

\begin{figure}[ht]
\begin{center}
\includegraphics[width=\textwidth]{picture/ads_blackhole}
\end{center}
\caption{(a) Minimal surfaces
$\gamma_A$ in the BTZ black hole for various sizes of $A$. (b)
$\gamma_A$ and $\gamma_B$ wrap the different parts of the horizon.
(c) When $\partial A$ gets larger, $\gamma_A$ is separated into
two parts: one is wrapped on the horizon and the other localized near the boundary.}
\label{fig: ads_blackhole.eps}
\end{figure}




\end{document}
