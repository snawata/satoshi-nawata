\documentclass[2dCFT-lecture.tex]{subfiles}


\begin{document}

\setcounter{tocdepth}{2}
%\maketitle
\section{Wess-Zumino-Novikov-Witten Models}
This section is devoted to study a very important class of exactly solvable 2d conformal field theories called \textbf{Wess-Zumino-Novikov-Witten (WZNW) models} \cite{Wess:1971yu,novikov1982hamiltonian,Witten:1983ar}. Th WZNW models have many applications and they are related to condensed matter physics, AdS/CFT correspondence, knot theory, and mathematical physics.

\subsection{U(1) symmetry of free boson}
Let us recall the free boson theory in \S\ref{sec:free-boson} whose action is given by
\be\label{free-boson-action}
\cS = \frac{1}{8 \pi} \int d^{2} x \,\partial_{\mu} \varphi \partial^{\mu} \varphi= \frac{1}{8 \pi} \int d^{2} z\, \partial_{z} \varphi \partial_{\overline{z}} \varphi
\ee
This actin has a symmetry \be\label{U1-symmetry}\varphi\to \varphi+a\ee whose Noether current is $j^{\mu}=- \partial^{\mu} \varphi / 4 \pi$.
The current conservation follows from the equation of motion \eqref{eom-boson}
$$
\partial_\mu j^\mu=0~.
$$
As in in \S\ref{sec:free-boson}, one can decompose the current into holomorphic and anti-holomorphic part
\be\label{U1-current}
J (z) \equiv i \partial_z \varphi~,\qquad  \overline J(\overline z)\equiv -i \partial_{\overline{z}} \varphi
\ee
so that the (holomorphic part of) energy-momentum tensor can be written as
$$
T (z)=- \frac{1}{2} : \partial \varphi \partial \varphi : =\frac12:J(z)J(z):~.
$$
As we will see below, this is called \textbf{Sugawara construction} of the energy-momentum tensor. In addition, the action \eqref{free-boson-action}
can be written as
\begin{equation}
 \cS_0=\frac{1}{8\pi} \int_\Sigma \dd^2 x \,
\pd^\mu g^{-1} \pd_\mu g   =\frac{1}{8\pi} \int_\Sigma \dd^2 x \,
\pd_{z} g^{-1} \pd_{\overline z} g\, ,
\end{equation}
where $g(z,\overline z)$ is the map from $$g:\mathbb{C}\to \U(1);\, z\mapsto g(z,\overline{z})=\exp(i\varphi(z,\overline z))~.$$ This action can be regarded as the simplest version of WZNW models.
In fact, the symmetry can be understood as the $\U(1)$ rotation
$$
e^{i\varphi(z,\overline z)} \mapsto e^{i(\varphi(z,\overline z)+a)}~,
$$
so that \eqref{U1-current} is called the $\U(1)$ current.
The $\U(1)$ group is an abelian group and it is natural to ask if we can generalize this simple construction to non-abelian Lie groups. This section answers to this question.

\subsection{SU(2) current algebra}\label{sec:SU(2)k}
Before moving to the WZNW models, we shall provide the easiest introduction to the non-abelian generalization. In \eqref{sec:compactified-boson}, we have seen that the compactified boson enjoys the $T$-duality $R\leftrightarrow 2/R$ \eqref{T-duality} and $R=\sqrt{2}$ is the self-dual radius. At the self-radius, one can define the vertex operators
$$
J (z)=\frac{i\partial \varphi(z)}{\sqrt{2}}~,\qquad J^{\pm}(z)=: e^{\pm i \sqrt{2} \varphi(z)} :
$$
where they satisfy the OPE
\be
\begin{aligned} J (z) J^{\pm} (w) & \sim \frac{\pm J^{\pm} (w)}{z-w}  \\ J^{+} (z) J^{-} (w) & \sim \frac{1}{(z-w)^{2}}+\frac{2 J (w)}{z-w}  . \\ J (z) J (w) & \sim \frac{1 / 2}{(z-w)^{2}}  \end{aligned}
\ee
Furthermore, if we write
\be\label{redefinition}
J^{\pm} \equiv J^{1} \pm i J^{2} , \quad J   \equiv J^{3}~,
\ee
then the OPE amounts to
$$
J^{a} (z) J^{b} (w) \sim \frac{\frac{1}{2} \delta_{a b}}{(z-w)^{2}}+\frac{i \epsilon^{a b c} J^{c} (w)}{z-w}
$$
where $\epsilon^{abc}$ is the anti-symmetric tensor. It is easy to see the connection to the theory of the angular momentum which is associated to the $\fraksu(2)$ Lie algebra
\be\label{su2-algebra}[t^a,t^b]=i \epsilon^{a b c} t^{c}\ee
where $t^a=\sigma^a/2$ with the Pauli matrices $\sigma^a$ ($a=1,2,3$). In fact,
as we will learn in the homework, the OPE can be generalized to
$$
J^{a} (z) J^{b} (w) \sim \frac{\frac{k}{2} \delta_{a b}}{(z-w)^{2}}+\frac{i \epsilon^{a b c} J^{c} (w)}{z-w}~.
$$
We note that
$$
J(z)=\sum_{a=1}^3J^a(z)t^a
$$
is called the $\SU(2)$ current and $k$ is its \textbf{level}. As usual, the mode expansion
$$
J^{a} (z)=\sum_{n \in \mathbb{Z}} z^{- n-1} J_{n}^{a}
$$
leads to the following commutation relations, which is called  is called the $\SU(2)$ \textbf{current algebra},
$$
\left[ J_{m}^{a} , J_{n}^{b} \right]=i \e^{ab c} J_{m+n}^{c}+\frac{k}{2} m \delta^{ab} \delta_{m+n , 0}
$$
where the zero modes satisfy the $\fraksu(2)$ Lie algebra \eqref{su2-algebra}. In mathematics, it is called the \textbf{$\widehat{\fraksu(2)}_k$ affine Lie algebra} with level $k$. This can be further generalized to so-called \textbf{Kac-Moody algebra}.

Let us briefly study the highest weight representation of  the $\SU(2)$ current algebra. From \eqref{redefinition}, we introduce
$$
J_{n}^{\pm} \equiv J_{n}^{1} \pm i J_{n}^{2}
$$
where the Hermitian conjugates are defined as
$$
\left(J_{n}^{\pm} \right)^{\dagger}=J_{- n}^{\mp} , \quad \left(J_{n}^{3} \right)^{\dagger}=J_{- n}^{3}~.
$$
Then, the highest weight state $|h,j\rangle$ is defined as
\be\label{HWS-su2}
 J_{0}^3  |h,j \rangle =j  |h,j \rangle~,\quad  J_{0}^{+}  |h,j \rangle=0 ,\qquad  J_{n}^{a}  |h,j \rangle=0 , \quad n > 0 , \quad a=0 , \pm ~.
\ee
Note that $h$ is the conformal dimension
$$
L_0 |h,j \rangle=
\frac{j (j+1)}{k+2 }|h,j \rangle
$$
which is left in homework.
Since the zero modes satisfy the $\fraksu(2)$ Lie algebra, $j$ takes an half-integer value $j=\frac12,1,\frac32,2,\cdots$ associated to the spin-$j$ representation.
In addition, it is straightforward to check that the following generators
$$
J_{(1)}^{+}=J_{- 1}^{+} , \quad J_{(1)}^{-}=J_{+ 1}^{-} , \quad J_{(1)}^3=J_{0}^3-\frac{k}{2}
$$
also form the $\fraksu(2)$ Lie algebra, and we call it $\fraksu(2)_{(1)}$.  Since the eigenvalue of $2 J_{(1)}^3=2 J_{0}^3-k$ is an integer, $k$ has to be an integer. Moreover, the unitarity imposes
$$
\begin{aligned} 0 \leqq \left| J_{(1)}^{+}  |h,j \rangle \right|^{2} &=\langle h,j \left| J_{+ 1}^{-} J_{- 1}^{+} \right |h,j \rangle \\ &=\left\langle h,j \left| \left[ J_{+ 1}^{-} , J_{- 1}^{+} \right] \right| h,j \right\rangle \\ &=- 2 \langle j | (J_{0}^3-k / 2)  |h,j \rangle \\ &=- 2 j+k \end{aligned}
$$
Therefore, for a given level $k\in \bZ_{>0}$, the spin $j$ is allowed to take the values
$$
j=0 , \frac{1}{2} , 1 , \frac{3}{2} , \cdots , \frac{k}{2}~.
$$


\subsection{A crash course on Lie groups and Lie algebras}


It is well-known that the theory of angular momentum in quantum mechanics is described by the representation theory of the $\fraksu(2)$ Lie algebra \eqref{su2-algebra}. We want to generalize the current algebra to arbitrary Lie algebras. Thus, we shall briefly review Lie groups and Lie algebras. For more details, we refer the reader to \cite{kirillov2008introduction}.

\begin{defn}[Lie group]\index{Lie group}
  A \textbf{Lie group} is a manifold $G$ with a group structure such that multiplication $m: G \times G \to G$ and inverse $i: G \to G$ are smooth maps. The \textbf{dimension} of a Lie group $G$ is the dimension of the underlying manifold.
\end{defn}




Some of Lie groups will be given by subsets of the space $M_n(\bF)$ of  $n\times n$ matrices where $\bF=\bR$ or $\bC$ specified by certain algebraic equations. For example,
\begin{itemize}
\item  General linear group: $\GL(n, \bF)=\{A \in \mathrm{M}_n(\bF)| \det A \neq 0\}$
\item  Special linear group: $\SL(n, \bF)=\{A \in \GL(n, \bF)| \det A=1\}$
\item Symplectic group $\mathrm { Sp } ( n , \bF )=\{ A \in \GL(2n, \bF)| A^ { \mathrm { T } } J A = J \ \textrm{where} \  J=\begin{pmatrix} 0&I_n\\-I_n&0\end{pmatrix}\}$
\item Unitary group $\U(n)=\{A \in \GL(n, \bC)| AA^\dagger=I\}$
\item Special unitary group $\SU(n) =\{A \in \U(n)| \det A=1\}$
\item  Orthogonal group $\mathrm{O}(n)= \{A \in \GL(n, \bR)| AA^T=I\}$
\item Special orthogonal group $\SO(n)=\{A \in \mathrm{O}(n)| \det A=1\}$
\end{itemize}
In fact, the simple Lie groups are classified by Killing an Cartan, and $\SU(n+1)$, $\SO(2n + 1)$, $\Sp(n,\bR)$ and $\SO(2n)$ are assigned to type $A_n$, $B_n$, $C_n$ and $D_n$, respectively. In addition to these four types, there are five exceptional Lie groups called $E_6$, $E_7$, $E_8$, $F_4$, and $G_2$ type. The Dynkin diagrams of type $A$, $D$ and $E$ are all simply laced, but the other types are not.


Roughly speaking, an element $X\in \frakg$ of a Lie algebra $\frakg$ describes an element $h\in G$ near the identity element $1\in G$ via
$$
h=1+\e X+\cO(\e^2)~.
$$
More precisely, the tangent space ${T}_1(G)$ of a Lie group $G$ at the identity element $1$ naturally admits a Lie bracket
  \[
    [\cdot, \cdot]: T_1 G \times T_1 G \to T_1 G; (X,Y) \mapsto [X,Y]=XY-YX
  \]
  such that the Lie algebra of
  \[
    \mathfrak{g}=(T_1(G), [\cdot, \cdot])
  \]
  is a Lie algebra.

\begin{defn}[Lie algebra of a Lie group]\index{Lie algebra of a Lie group}
  Let $G$ be a Lie group. The \textbf{Lie algebra} of $G$, written $\mathfrak{g}$, is the tangent space $T_1 G$ under the natural Lie bracket.
\end{defn}







Given a vector $X\in \mathfrak{g} $ in the tangent space of the identity element $1$, the exponential map defines a map $\exp:\mathfrak{g}  \to G$ such that
  \[
    \exp(tX)=\sum_{\ell=0}^\infty \frac{1}{\ell!} (tX)^\ell~.
  \]
Without relying on a Lie groups, one can defines a Lie algebra as follows.

\begin{defn}[Lie algebra]\index{Lie algebra}
  A \textbf{Lie algebra} $\mathfrak{g}$ is a vector space (over $\bF=\bR$ or $\bC$) with a \textbf{bracket}
  \[
    [\cdot,\cdot] : \mathfrak{g} \times \mathfrak{g} \to \mathfrak{g}
  \]
  satisfying
  \begin{enumerate}
      \item $[\alpha X+\beta Y, Z]=\alpha [X, Z]+\beta [Y, Z]$ for all $X, Y, Z \in \mathfrak{g}$ and $\alpha, \beta \in \bF$ \hfill(bilinearity)
    \item $[X, Y]=-[Y, X]$ for all $X, Y \in \mathfrak{g}$ \hfill(antisymmetry)
    \item $[X, [Y, Z]]+[Y, [Z, X]]+[Z, [X, Y]]=0$ for all $X, Y, Z \in \mathfrak{g}$.\hfill(Jacobi identity\index{Jacobi identity})
  \end{enumerate}
  Note that linearity in the second argument follows from linearity in the first argument and antisymmetry.
\end{defn}

For instance, we have the corresponding Lie algebras
\begin{itemize}
\item $\mathfrak{gl}(n, \bF)=\{A \in \mathrm{M}_n(\bF)\}$
\item $\mathfrak{sl}(n, \bF)=\{A \in \mathrm{M}_n(\bF) |  \Tr A=0\}$
\item  $\mathfrak{ sp } ( n , \mathbb { F } ) = \{ X \in \mathfrak{gl}(2n, \bF)) | X ^ {T} J + J X = 0  \ \textrm{where} \  J=\begin{pmatrix} 0&I_n\\-I_n&0\end{pmatrix} \}$
\item $\mathfrak{su}(n)=\{A \in \mathrm{M}_n(\mathbb{C}) | \Tr A=0, A^\dagger=-A\}$
\item $\mathfrak{so}(n)=\{A \in \mathrm{M}_n(\mathbb{R}) | A^T=-A\}$
\end{itemize}



Given a finite-dimensional Lie algebra $\frakg$, we can pick a basis for $\mathfrak{g}$:
\be\label{Lie-basis}
t^a: a=1, \cdots, \dim \mathfrak{g}~
\ee
with
$$
\Tr(t^at^b)=\frac12 \delta^{ab}~,
$$
which is called the Cartan-Killing form.
Then any $X \in \mathfrak{g}$ can be written as
\[
  X=X^a t^a=i \sum_{a=1}^n X^a t^a,
\]
where $X^a \in \bF$.

By linearity, the bracket of elements $X, Y \in \mathfrak{g}$ can be computed via
\[
  [X, Y] =- X^a Y^b [t^a, t^b].
\]
In other words, the whole structure of the Lie algebra can be given by the bracket of basis vectors. We know that $[t^a, t^b]$ is again an element of $\mathfrak{g}$. So we can write
\[
  [t^a, t^b] =i f_{ab}{}^c t^c,
\]
where $f_{ab}{}^c\in \bF$ are called the \textbf{structure constants}.
By the antisymmetry of the bracket, we know
  \[
    f_{ba}{}^c=-f_{ab}{}^c.
  \]
The Jacobi identity amounts to
  \[
    f_{ab}{}^c f_{cd}{}^e+f_{da}{}^c f_{cb}{}^e+f_{bd}{}^c f_{ca}{}^e=0.
  \]



\subsection*{affinization}
Hence, one can straightforwardly generalize the current associated to a Lie algebra $\frakg$ as follows. First the current taking its value on a Lie algebra $\frakg$ can be write
$$
J(z)=\sum_{a}J^a(z)t^a~.
$$
The OPE of the current takes the form
$$
J^{a}(z) J^{b}(w) \sim \frac{\frac k2 \d^{ab}}{(z-w)^{2}}+\frac{i f^{a b }{}_c J^{c}(w)}{z-w }
$$
so that the modes are subject to
$$
\left[ J _{n }^{a}, J _{m }^{b}\right]=i f^{a b}J _{n+m }^{c}+ \frac k2 \d^{a b}n \delta _{n+m , 0 }~.
$$
This is called the $G$ current algebra with level $k$ or the $\wh \frakg_k$ affine Lie algebra with level $k$.

\subsection{WZNW models}
\subsubsection*{Nonlinear Sigma Models}
In searching for an explicit conformal field theory with the $G$ current algebra,
it is natural to first consider the \textbf{nonlinear sigma model}
\begin{equation}
\label{nonlinear-sigma-models}
 \cS_0=\frac{1}{2 a^2} \int_\Sigma \dd^2 x
 \Tr(\pd^\mu g^{-1} \pd_\mu g)\, ,
\end{equation}
where $a^2$ is a positive, dimensionless coupling constant.
The bosonic field $g(x)$ takes its value on the Lie group $G$, namely
$$
g:\Sigma\to G
$$
where $\Sigma$ is generally a two-dimensional manifold, called a Riemann surface.
%For the action to be positive real, $g(x)$ must be valued in a unitary
%representation so that
%\begin{equation}
%  \Tr(\pd^\mu g^{-1}\pd_\mu g)
% =\Tr(\pd^\mu g^\dagger \pd_\mu g) \geq 0 \, .
%\end{equation}
%However, this is not the kind of theory we are looking for, since
%the conserved currents can not factorize in a holomorphic and
%an antiholomorphic part. Let us briefly discuss the reason.
Under the variation of the action
\begin{equation}
\delta \cS_0=\frac{1}{a^2}
\int_\Sigma \dd^2 x  \Tr(g^{-1}\delta g \pd^\mu (g^{-1}\pd_\mu g))\, ,
\end{equation}
which results in the following equation of motion
\begin{equation}
\pd^\mu(g^{-1}\pd_\mu g) =0 \, .
\end{equation}
This implies the conservation of the currents
\begin{equation}
  J_\mu=g^{-1}\pd_\mu g \, .
\end{equation}
In the complex coordinate $z=x^0+i x^1$ and
$\overline{z}= x^0 -i x^1$. If we write $J_z= g^{-1}\pd_z g$
and $\overline{J}_{\overline{z}}=g^{-1}\pd_{\overline{z}}g$, we find
\begin{equation}
\pd_z \overline{J}_{\overline{z}}+\pd_{\overline{z}} J_z=0\, .
\end{equation}
So that the holomorphic and anti-holomorphic currents can not
be separately conserved.
\def\tg{\widetilde{g}}






\subsubsection*{WZNW Models}
Hence, a more complicated action must be considered to
enhance the symmetry. We should add a Wess-Zumino term \cite{Wess:1971yu}
\begin{equation}
\label{Wess-Zumino-term}
 \Gamma=\frac{-i}{12\pi} \int_B
 \dd^3 y \epsilon_{\alpha\beta\gamma}
 \Tr(\tg^{-1}\pd^\alpha \tg \tg^{-1} \pd^\beta \tg \tg^{-1}
 \pd^\gamma \tg) \,.
\end{equation}
This is defined on a three-dimensional manifold $B$, whose boundary is the Riemann surface $\partial \Sigma =B$. Here $\tg:B\to G$ is an extension of the original
field $g$. Although the extension is not unique, this is actually well-defined.
Given another choice $(B',\tg')$, we can glue together the three-manifolds $B$ and $B'$ along $\Sigma$ to get a closed three-manifold $B\cup_\Sigma -B'$. It is easy to check that $\omega\equiv \Tr(\tg^{-1}\pd^\alpha \tg \tg^{-1} \pd^\beta \tg \tg^{-1}
 \pd^\gamma \tg)$ is a closed three-form, and moreover it is pull-back of an element $\Tr(\theta\wedge\theta \wedge\theta)$ of the third homology group $H^3(G;\bZ)$ with integer coefficients \cite[Prop 4.4.5]{pressley1986loop} via $\wt g:M\to G$ where $\theta$ is the Maurer-Cartan form of $G$.
  Hence, the integral of $\omega$ over the closed three-manifold $B\cup_\Sigma -B'$ is an integer.
 Therefore, the Euclidian functional integral,
with weight $\exp(-k\Gamma)$ is perfectly well defined if the level $k$ is
an integer.


\begin{figure}[ht]\centering
\includegraphics[width=10cm]{picture/gluing}
\end{figure}

We then consider the action
\begin{equation}
\label{wess-zumino-action}
\cS=\cS_0 +k \Gamma\,,
\end{equation}
where $k$ is an integer. Although the Wess-Zumino term is
expressed as a three-dimensional integral, its variation under
$g\to g+ \delta g$ is a two-dimensional functional, because the
variation of its density can be written as a total derivative.
The final result is
\begin{equation}
\delta \Gamma=
\frac{i}{4\pi}
\int_\Sigma \dd^2 x \epsilon_{\mu\nu}
\Tr(g^{-1}\delta g \partial^\mu(g^{-1}\pd^\nu g))\, .
\end{equation}
The equation of motion of the action \eqref{wess-zumino-action}
is then
\begin{equation}
\pd^\mu(g^{-1}\pd_\mu g)+\frac{a^2 i k}{4\pi}
\epsilon_{\mu\nu}\pd^\mu (g^{-1}\pd^\nu g) =0 \, .
\end{equation}
In terms of the complex variable $z$, $\overline{z}$, the equation
of motion becomes
\begin{equation}
\bigg(
1+\frac{a^2 k}{4 \pi}
\bigg)
\pd_z(g^{-1}\pd_{\overline{z}}g)
+
\bigg(
1-\frac{a^2 k}{4 \pi}
\bigg)
\pd_{\overline{z}}(g^{-1}\pd_z g) =0\, .
\end{equation}
Thus, for
\begin{equation}
a^2=4\pi/k\, ,
\end{equation}
we find the desired conservation law
\begin{equation} \label{current-conservation-WZNW}
\pd_z(g^{-1}\pd_{\overline{z}}g) =0 \, , \qquad \pd_{\overline{z}}((\pd_{z}g)g^{-1}) =0 \, .
\end{equation}
The action now becomes
\begin{equation}
 \cS =
 \frac{k}{8\pi} \int_\Sigma \dd^2 x
 \Tr(\pd^\mu g^{-1}\pd_\mu g)+k\Gamma \, ,
\end{equation}
which is called the $\widehat{\frakg}_k$  \textbf{WZNW (WZNW) model}.
Since $a^2$ is positive, $k$ must be a positive integer.
The other solution $a^2=- 4\pi /k$ , which requires
$k<0$, implies the conservation of the dual currents.

Due to the Noether theorem, the conservation of the currents
$J_z$ and $\overline J_{\overline{z}}$ \eqref{current-conservation-WZNW} should implies the invariance of the
action. In fact, \eqref{current-conservation-WZNW} tells us that the field $g$ takes the following factorized form
$$
g(z,\overline{z})=g_L(z)g_R(\overline{z})~.
$$
We will see, under the transformation
\begin{equation}
g(z,\overline{z}) \rightarrow
\Omega(z) g(z,\overline{z}) \overline{\Omega}^{-1}(\overline{z})\, ,
\end{equation}
where $\Omega$ and $\overline{\Omega}$ are two arbitrary matrices
valued in $G$, the action is invariant. Indeed, under the infinitesimal transformation
\begin{equation}
\Omega(z)=1+\omega(z) \, ,
\qquad
\overline{\Omega}(\overline{z})
= 1+\overline{\omega}(\overline{z})\, .
\end{equation}
$g$ transforms as follows
\begin{equation}
\label{transformation-g}
 \delta_\omega g=\omega g \, , \qquad
 \delta_{\overline{\omega}} g=- g\overline{\omega} \, .
\end{equation}
With $a^2=4\pi /k$, the variation of the action for
$g\rightarrow g+\delta_\omega g+\delta_{\overline{\omega}}g$ is
\bea
\label{variation-action}
 \delta \cS &= \frac{k}{\pi}
 \int_\Sigma \dd^2 x \Tr (g^{-1}\delta g [\pd_z(g^{-1}\pd_{\overline{z}}g)])
 \notag \\
 &= \frac{k}{\pi} \int_\Sigma \dd^2 x \Tr
 [\omega(z)\pd_{\overline{z}}(\pd_z g g^{-1})-
 \overline{\omega}(\overline{z})\pd_z(g^{-1}\pd_{\overline{z}}g)]\,,
\eea
 which vanishes after an integral by parts. Thus the global
 $G \times G$ invariance of the sigma model has thus been
 extended to a local $G(z)\times G(\overline{z})$ invariance.

 By rescaling the conserved currents as
 \bea
 \label{rescaling-current}
  J(z) &\equiv -\frac k2J_z(z)=- \frac k2 \pd_z g g^{-1}\, ,\\
  \overline{J}(\overline{z}) &\equiv \frac k2 \overline J_{\overline{z}}(\overline{z})
 =\frac k2 g^{-1}\partial_{\overline{z}} g \, .
  \eea

  \def\pdz{\pd_{z}}
  \def\pdbz{\pd_{\overline{z}}}
  \def\obz{\overline{\omega}({\overline{z}})}
  \def\jbz{\overline{J}({\overline{z}})}
 We can rewrite \eqref{variation-action} in the form
 \begin{equation}
  \delta \cS=-\frac{2}{\pi}
  \int_\Sigma \dd^2 x \left\{
  \pdbz (\Tr[\omega(z)J(z)])
  +\pdz (\Tr[\obz\jbz])
  \right\}\,.
 \end{equation}
 Replace $d^2 x$ by $(-i/2)\dd z \dd \overline{z}$ and take the
 holomorphic contour to be counterclockwise and the antiholomorphic contour to be clockwise. This lead to
 \begin{equation}
 \delta_{\omega,\overline{\omega}} =
 \frac{i}{\pi}
 \oint \dd z \Tr[\omega(z)J(z)]
 -\frac{i}{\pi}
 \oint \dd \overline{z} \Tr
 [\obz\jbz]\,.
 \end{equation}
 Expanding $J$ and $\omega$ as well as their dual part in terms
 of the basis $t^{\alpha}$ of $G$
 \begin{equation}
  J=\sum_a J^a t^a\, , \qquad
  \omega=\sum_a \omega^a t^a \, ,
 \end{equation}
 The variation of the action now becomes
 \begin{equation}
  \delta_{\omega,\overline{\omega}} S =
  -\frac{1}{2 \pi i}
  \oint \dd z
  \sum_a \omega^a J^a
 +
  \frac{1}{2\pi i}
  \oint \dd \overline{z}
  \sum_a \overline{\omega}^a \overline{J}^a \,.
 \end{equation}
 Let $X$ stand for a list of fields as usual, from \eqref{variation-action-locally} and \eqref{Ward-identity-ver2}, we have
 \begin{equation}
 \delta\expval{X}=\expval{(\delta S) X}\, .
 \end{equation}
 Hence, we immediately obtain the
 corresponding Ward-identity
 \begin{equation}
 \label{WZNW-ward-identity}
 \delta_{\omega,\overline{\omega}}\expval{X}
=-\frac{1}{2\pi i}
 \oint \dd z \sum_a \omega^a \expval{J^a X}
+\frac{1}{2\pi i}
 \oint \dd \overline{z}
 \sum_a \overline{\omega}^a \expval{\overline{J}^a X}\, .
 \end{equation}
 From \eqref{rescaling-current} and \eqref{transformation-g},
 the transformation law for the current is
 \begin{equation}
 \delta_{\omega} J=
 \comm{\omega}{J}-\frac k2\partial_z \omega\, .
 \end{equation}
 Writing in terms of $J^a$ and $\omega^a$, we have
 \begin{equation}
  \delta_\omega J^a=
  \sum_{b,c} i f_{abc} \omega^b J^c- \frac k2 \partial_z \omega^a \, .
 \end{equation}
 The substitution of this transformation into \eqref{WZNW-ward-identity}, leads to so called
 current algebra.
 \begin{equation}
 J^a (z) J^b(w) \sim
 \frac{\frac k2\delta_{ab}}{(z-w)^2}+ \sum_c i f_{abc}
 \frac{J^c(w)}{(z-w)} \, .
 \end{equation}

 Introducing the modes $J^a_n$ from the Laurent expansion
 \begin{equation}
 \label{mode-expansion-Ja}
   J^a (z)=\sum_{n\in \mathbb{Z}}
   z^{-n-1}J^a_n\,.
 \end{equation}
 We can easily check (as we have done in the Virasoro case),
 the commutation relations of $J^a_m$ become
 \begin{equation}
  \comm{J^a_n}{J^b_m}=\sum_c i f_{abc}J^c_{n+m}
 +\frac k2n\delta_{ab} \delta_{n+m,0}\,,
 \end{equation}
 which is the $\widehat{\frakg}$ affine Lie algebra at level $k$. The dual part $\overline{J}$ is similar.

\subsection{Sugawara construction}
The energy-momentum tensor of the theory is given by Sugawara
construction
\begin{equation}
\label{Sugawara-tensor}
 T(z)=\frac{1}{(k+h^\vee)} \sum_a (J^a J^a)(z)\, .
\end{equation}
The prefactor
$\frac{1}{(k+h^\vee)}$ is determined by the usual form of $TT$ OPE, and $h^\vee$ is called the \textbf{dual Coxeter number} of $\frakg$.

By using the generalized wick theorem which we have encountered
in the homework, we can derive the $TJ$ OPE
\begin{equation}
\label{TJ-OPE}
 T(z)J^a(w)=
 \frac{J^a(w)}{(z-w)^2}+\frac{\pd J^a (w)}{(z-w)}
 \, .
\end{equation}
Therefore, $J^a$ is a Virasoro primary field with conformal dimension $1$.

$TT$ OPE can be calculated straightforwardly
\begin{equation}
 T(z)T(w)=\frac{c/2}{(z-w)^4}
+\frac{2T(w)}{(z-w)^2}
+\frac{\pd T(w)}{(z-w)} \, ,
\end{equation}
with the central charge $c$
\begin{equation}
 c=\frac{k \,\text{dim} (\frakg)}{k+h^\vee}\, .
\end{equation}

The Sugawara energy-momentum tensor can be written in terms
of mode expansion the same as before
\begin{equation}
 T(z) =\sum_{n\in\mathbb{Z}} z^{-n-2} L_n \, .
\end{equation}
With the OPEs listed above, we can compute the complete
affine Lie and Virasoro algebra
\bea
  \comm{L_n}{L_m} &= (n-m) L_{n+m}+\frac{c}{12}
  (n^3-n) \delta_{n+m,0}\, ,\\
  \comm{L_n}{J^a_m} &= -m J^a_{n+m}\, ,\\
  \comm{J^a_n}{J^b_m} &= \sum_c if_{abc}J^c_{n+m}
 +\frac k2n \delta_{ab}\delta_{n+m,0}\, .
\eea

Finally, for future discussion, we need to write the relation
between $L_n$ and $J^a_n$. From \eqref{Sugawara-tensor} and the mode expansion of
$J^a$ \eqref{mode-expansion-Ja}, the relation can be obtained:
\begin{equation}
\label{rel-Ln-Jan}
L_n=\frac{1}{(k+h^\vee)}
\sum_a \sum_m :J^a_m J^a_{n-m}:\, .
\end{equation}

\subsection{Knizhinik-Zamolodchikov equations}
\subsubsection*{WZNW primary field}
A WZNW primary field is defined as a field that transforms
covariantly with respect to a $G(z)\times G(\overline{z})$ transformation. From \eqref{transformation-g} and \eqref{WZNW-ward-identity}, we reformulate $Jg$ OPE as
following
\bea
J^a(z) g(w, \overline{w}) &\sim \frac{- t^a g(w,\overline{w})}{z-w}\,, \\
\overline{J}^a(z) g(w,\overline{w}) &\sim
\frac{g(w,\overline{w})t^a}{z-w} \, .
\eea
This can be generalized to define a WZNW primary field. Any
WZNW primary field $\phi_{\lambda,\mu}$ with $\lambda$ and $\mu$
specified the representation in the holomorphic and antiholomorphic sector respectively should satisfy the
following OPE
\begin{equation}
\label{WZNW-primary field}
 J^a(z) \phi_{\lambda,\mu}(w,\overline{w}) \sim
 \frac{- t^a_\lambda \phi_{\lambda,\mu}(w,\overline{w})}{z-w}\, ,
\end{equation}
\begin{equation}
 \overline{J}^a(\overline{z}) \phi_{\lambda,\mu}(w,\overline{w})
 \sim
 \frac{\phi_{\lambda,\mu}(w,\overline{w})t^a_\mu}{\overline{z}-\overline{w}}\, ,
\end{equation}
where $t^a_\lambda$ is the matrix $t^a$ in the $\lambda$
representation. By analogy with \eqref{Virasoro-primary}, the above OPE can be extended
\begin{equation}
\label{WZNW-primary-field-extension}
  J^a(z)\phi_1(z_1)\phi_2(z_2)\cdots \phi_n(z_n)
  \sim -\sum_i\frac{t^a_i}{z-z_i} \phi_1(z_1)\phi_2(z_2)
  \cdots \phi_n(z_n)\, ,
\end{equation}
where we have considered only holomorphic sector, with
$t_i^a$ a matrix acting on $\phi_i$.

By expanding the currents in terms of the modes evaluated at
$w$,
\begin{equation}
 J^a(z)=\sum_n (z-w)^{-n-1} J^a_n(w)\, .
\end{equation}
We can write their OPE with an arbitrary field $A$ as we have done in the Virasoro case
\begin{equation}
J^a(z)A(w)=\sum_n (z-w)^{-n-1} (J^a_n A)(w)\, .
\end{equation}
Thus, from \eqref{WZNW-primary field}, for the WZNW holomorphic primary field
$\phi_\lambda$, we have
\bea
 J^a_0 \phi_\lambda &= -t^a_\lambda \phi_\lambda \, ,\\
J^a_n \phi_\lambda &= 0 \qquad \text{for} \qquad n>0 \, .
\eea
Associating the state $\ket{\phi_\lambda}$ to the field
$\phi_\lambda$ as usual
\begin{equation}
\phi_\lambda(0) \ket{0}=\ket{\phi_\lambda}\,.
\end{equation}
Then, the condition for a WZNW primary field translate into
\bea
  J^a_0\ket{\phi_\lambda} &=-t^a_\lambda \ket{\phi_\lambda}\, ,\\
  J^a_n \ket{\phi_\lambda} &= 0 \qquad \text{for} \qquad n>0\, .
\eea
From \eqref{rel-Ln-Jan}, we see that in the expression for $L_n$,
with $n>0$, the rightmost factor $J^a_m$ has $m>0$ which implies that
\begin{equation}
 L_n\ket{\phi_\lambda}=0 \qquad \text{for} \qquad n>0 \, .
\end{equation}
Therefore, the WZNW primary fields are also Virasora primary fields with the conformal dimension given by
\begin{equation}
 L_0 \ket{\phi_\lambda}=\frac{1}{(k+h^\vee)}
 \sum_a J^a_0 J^a_0 \ket{\phi_\lambda}
=\frac{\sum_a t^a_\lambda t^a_\lambda}{(k+h^\vee)}\ket{\phi_\lambda}
 \, .
\end{equation}
The inversed statement is not true, for example, $J^a(z)$ is a
Virasoro primary field but not WZNW primary field which can be seen directly from $TJ$ OPE and $JJ$ OPE we have shown before.

Finally, a WZNW descendant state is defined of the form
\begin{equation}
 J^a_{-n_a} J^b_{-n_2}\cdots \ket{\phi_\lambda}\, ,
\end{equation}
with $n_1,n_2 \cdots$ all positive integers.





\subsubsection*{Knizhnik-Zamolodchikov Equations}
From the previous learning, we know that there are many
constraints to the correlation function, such as the
null fields. In this section, we briefly discuss the affine
singular vector in level one and the constraint follows form it.

For $n=-1$, from \eqref{rel-Ln-Jan}, we have
\begin{equation}
  L_{-1}\ket{\phi_i}=\frac{2}{k+h^\vee}
  \sum_a (J^a_{-1}J^a_0) \ket{\phi_i}
 =\frac{-2}{k+h^\vee} \sum_a (J^a_{-1}t_i^a) \ket{\phi_i}\, .
\end{equation}
We consider the insertion of the zero vector
\begin{equation}
 \ket{\chi}=\bigg[
 L_{-1}+ \frac{2}{k+h^\vee}\sum_a(J^a_{-1}t_i^a)
 \bigg]
 \ket{\phi_i}=0\, .
\end{equation}
First, we need to calculate the contribution of the second term.
By using \eqref{mode-expansion-Ja}
, the insertion of the operator $J^a_{-1}$ in the correlator
can be expressed as
\begin{equation}
  \expval{\phi_1(z_1)\cdots (J^a_{-1}\phi_i)(z_i)\cdots\phi_n(z_n)}
 =\frac{1}{2\pi i}
  \oint_{z_i}
  \frac{\dd z}{z-z_i}
  \expval{J^a(z)\phi_1(z_1)\cdots\phi_n(z_n)}\,.
\end{equation}
Now using \eqref{WZNW-primary-field-extension} and
perform the residue calculation, we get
\begin{equation}
\expval{\phi_1(z_1)\cdots (J^a_{-1}\phi_i)(z_i)\cdots\phi_n(z_n)}
= \sum_{j\neq i}
\frac{t^a_j}{z_i-z_j}
\expval{\phi_1(z_1)\cdots \phi_n(z_n)}\,.
\end{equation}
Thus
\begin{equation}
\expval{\phi_1(z_1)\cdots \chi(z_i)\cdots \phi_n(z_n)}=
\bigg[
\partial_{z_i}+ \frac{2}{k+h^\vee}
\sum_{j\neq i}
\frac{\sum_a t^a_i \otimes t^a_j}{z_i-z_j}
\bigg]
\expval{\phi_1(z_1)\cdots\phi_n(z_n)}
\, ,
\end{equation}
where $t^a_j$ and $t^a_i$ act on $\phi_i$ and $\phi_j$
respectively. By construction, it must  vanish so that
we have
\begin{equation}
\bigg[
\partial_{z_i}+ \frac{2}{k+h^\vee}
\sum_{j\neq i}
\frac{\sum_a t^a_i \otimes t^a_j}{z_i-z_j}
\bigg]
\expval{\phi_1(z_1)\cdots\phi_n(z_n)}
=0\, .
\end{equation}
This is called the \textbf{Knizhnik-Zamolodchikov equation} \cite[ISZ88-No.4]{knizhnik1984current}. The
solutions to this equation are the correlation functions
of WZNW primary fields.


\subsection{Coset models}\label{sec:coset}

Using current algebras, one can construct an important class of 2d CFTs called \textbf{coset construction} \cite[ISZ88-No.11]{goddard1986unitary}. This construction has many applications in physics like (supersymmetric) unitary minimal models (see the end of \S\ref{sec:su2k-character}) and Kazama-Suzuki models \cite{Kazama:1988qp}.


We start with the current algebra associated to a Lie algebra $\frakg$ which contains  $\frakh$ as subgroups. The Sugawara energy-momentum tensors of affine Lie algebra $\widehat \frakg_{k_{\frakg}}$ and $\widehat \frakh_{k_{\frakh}}$ are
\bea
T _{\mathfrak{g}} (z) &= \frac{1}{k_{\mathfrak{g}}+h^\vee _{\mathfrak{g}} }\sum _{a=1 }^{\operatorname{dim}\mathfrak{g}} : J _{\mathfrak{g}}^{a}J _{\mathfrak{g}}^{a}: (z) \cr T _{\mathfrak{h}} (z) &= \frac{1}{k_{\mathfrak{h}}+h^\vee _{\mathfrak{h}} }\sum _{b=1 }^{\operatorname{dim}\mathfrak{h}} : J _{\mathfrak{h}}^{b}J _{\mathfrak{h}}^{b}: (z) \eea
We define the energy-momentum tensor of the coset model $\widehat \frakg_{k_{\frakg}}/ \widehat \frakh_{k_{\frakh}}$ as
$$
T _{\mathfrak{g}/ \mathfrak{h}} :=T _{\mathfrak{g}}-T _{\mathfrak{h}}~.
$$
Then, we have
\bea
T _{\mathfrak{g}} (z) J _{\mathfrak{h}}^{b}(w) & =\frac{J _{\mathfrak{h}}^{b}(w)}{(z-w)^{2}}+\frac{\partial _{w}\dot{j}_{\mathfrak{h}}^{b}(w)}{z-w}+ \cdots \cr T _{\mathfrak{h}} (z)  J _{\mathfrak{h}}^{b}(w)&=\frac{J _{\mathfrak{h}}^{b}(w)}{(z-w)^{2}}+\frac{\partial _{w}J _{\mathfrak{h}}^{b}(w)}{z-w}+ \cdots \eea
which yield
\bea
\left(T _{\mathfrak{g}}-T _{\mathfrak{h}} \right) (z) J _{\mathfrak{h}}^{b}(w)&\sim0\cr
 \left(T _{\mathfrak{g}}-T _{\mathfrak{h}} \right) (z) T _{\mathfrak{h}} (w) &\sim0~.
 \eea
Therefore, one can compute the $T _{\mathfrak{g}/ \mathfrak{h}} T _{\mathfrak{g}/ \mathfrak{h}} $ OPE so that the central charge of the coset model is
\be\label{coset-central-charge}
c _{\mathrm{g}/ \mathrm{h}}=c_{\mathfrak{g}}-c _{\mathrm{h}}=\frac{k _{\mathfrak{g}} \operatorname{dim}\mathfrak{g}}{k _{\mathfrak{g}}+h^\vee _{\mathfrak{g}}}- \frac{k _{\mathfrak{h}} \operatorname{dim}\mathfrak{h}}{k _{\mathfrak{h}}+h^\vee _{\mathfrak{h}} }~.
\ee


\subsubsection*{Parafermion}
As the simplest model of the coset model, we consider $\widehat{\mathfrak{su}(2)}_k/\widehat{\mathfrak{u}(1)}_k$. The energy-momentum tensor $T_{\frakg/\frakh}=T _{\frakg }-T _{\frakh }$ where
\bea
T _{\frakg}(z) &= \frac{1}{k+2}\sum _{a=1 }^{3}: J^{a}(z) J^{a}(z) : \cr
T _{\frakh}(z) &= \frac{1}{k}: J^{3}(z) J^{3}(z) :
\eea
If we define $J^{3}$ as
$$
J^{3}(z)=i \sqrt{\frac{k}{2}} \partial \varphi (z)
$$
then $T_H$ is the energy-momentum tensor of the free boson. It is easy to read off central charge
$$
c _{\mathrm{g}/ \mathrm{h}}=\frac{3 k}{k+2}- 1=\frac{2 (k-1)}{k+2 }~,
$$
which is equal to \eqref{Zk-cc}.  Indeed, this coset model describes the 2d CFT with $\bZ_k$ symmetry \cite[ISZ88-No.14]{fateev1985parafermionic}. This model contains a current with fractional spin (conformal dimension), which does obey neither bose nor fermi statistics, and is rather subject to para-statistics. Therefore, it is called  $\bZ_k$ \textbf{parafermion}.
In terms of the free boson $\varphi$ and the $\SU(2)$ current $J^\pm$, the parafermions are defined by
$$
\left(J^{+}\right)^{\ell}= \psi _{\ell}e^{i \ell \sqrt{\frac{2}{k}} X}, \quad \left(J^{-}\right)^{\ell}= \psi _{\ell }^{\dagger}e^{- i \ell \sqrt{\frac{2}{k}} X}, \quad (\ell=0 , \cdots , k)
$$
where $ \psi _{\ell}$ and $\psi _{\ell}^\dagger$ have conformal dimension $h=\ell(k-\ell)/k$.




When $k=1$, the central charge is $c=0$ so that the theory is trivial.
When $k=2$, the central charge is $c=\frac12$ so that the theory is the free fermion.
When $k=3$, the central charge is $c=\frac43$ so that the theory is the 3-state Potts model, which corresponds to the unitary minimal model $\cM_5$. In this case, the primary field with conformal dimension $h=2/3$ corresponds to the parafermion current $\psi_1$. The partition function of the parafermion is evaluated in \cite[ISZ88-No.28]{Gepner:1986hr}.

In fact, the supersymmetric version of the parafermion describes $\cN=2$ unitary minimal models, which plays a crucial role to describe Calabi-Yau sigma models \cite{Gepner:1987vz,Gepner:1987qi}.




\subsubsection*{Rational conformal field theories}
The minimal models, the WZNW models and the coset models belong to a class of rational conformal field theories (RCFTs). A theory is endowed with a chiral algebra $\cA\otimes \overline \cA$ that contains the Virasoro algebra as a subalgebra. The definition of a RCFT is (roughly) given as follows.
\begin{itemize}
\item It is unitary with a unique vacuum.
\item There are finitely  many primary fields $\phi_i$ of the chiral algebra $\cA$. The same statement holds for anti-holomorphic part.
\item The Hilbert space is spanned by finitely  many  irreducible representations of $\cA\otimes \overline \cA$
$$
\cH=\bigoplus_{i,j} \cN_{i,j} \mathcal{V}_i\otimes \overline{\mathcal{V}}_j
$$
where the irreducible representation $\mathcal{V}_i$ arises from primary fields $\phi_i$.
\item   Their OPEs of primary fields are closed among themselves
$$
\phi_i(z)\phi_j(w)\sim \sum_{k}\frac{C_{ijk}}{(z-w)^{h_i+h_j-h_k}}\phi_j(w)+\cdots
$$
\end{itemize}
It is  called a RCFT because conformal dimensions of primary fields of the chiral algebra $\cA$ as well as the central charge are rational numbers $\in \bQ$ \cite{Anderson:1987ge,Vafa:1988ag}. In a RCFT, conformal blocks can be, in principle, determined from  structure constants $C_{ijk}$ by bootstrap equations \eqref{bootstrap}, and one can evaluate a correlation function on an arbitrary Riemann surface. For more details, we refer to \cite{Moore:1988qv,Moore:1989vd}.


\end{document}
