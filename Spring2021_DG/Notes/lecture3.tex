\documentclass[geometry-lectures-21.tex]{subfiles}

%\title{ Lecture 4}
\begin{document}

\section{Differential forms}



\subsection{Cotangent bundles}


Given a vector space $V$ on $\bR$, one can take its dual space
$$
V^*=\{\omega:V\to\R | \omega(\alpha_1 X_1+\alpha_2 X_2)=\alpha_1 \omega(X_1)+\alpha_2 \omega(X_2) \textrm{ for } X_i\in V  \textrm{ and } \alpha_i\in \bR\}
$$
The dual space $V^*$ is also a vector space: $\beta_1 \omega_1 +\beta_2  \omega_2 \in V^*$ for $\omega_1,\omega_2\in V^*$ and $\beta_1,\beta_2\in \R$.
The dual vector space $T_p^*M$ of the tangent space $T_pM$ is called the \term{cotangent space}. In fact, given $f\in C^\infty(M)$, we can define its differential $df_p$ at $p$
\be\label{df}
df_p:T_pM\to \R;X_p \mapsto X_p(f)~.
\ee
For a local coordinate $(U,\varphi=(x^1,\cdots,x^n))$, we have seen that $(\frac{\partial}{\partial x^1}\Big|_p,\cdots,\frac{\partial}{\partial x^n}\Big|_p)$ is  a basis of $T_pM$. On the other hand, we can take $(dx^1|_p,\cdots,dx^n|_p)$ as a basis of $T^*_pM$ so that $$dx^i|_p\Bigl(\frac{\partial}{\partial x^j}\Big|_p\Bigr)=\delta^j_i~.$$
Therefore, in this basis, we can write
$$
df_p=\sum_{i=1}^n\frac{\partial f}{\partial x^i}(p) dx^i|_p
$$
Like the tangent bundle, we can consider a collection of the cotangent spaces
\be\label{cotangent-bundle}
T^*M=\cup_{p\in M} T^*_pM
\ee
which has a manifold structure. We call $T^*M$ the \term{cotangent bundle} of $M$. Moreover, the section of the cotangent bundle is called one-form, and we denote the set of one-form by $\Omega^1(M)=\Gamma(T^*M)$.
  For example, if $f$ is a smooth function on $M$, then $d f \in \Omega^1(M)$, which can take a pairing with ${}^\forall X \in \mathfrak{X}(M)$
  \[
    d f(X) = X(f)~.
  \]



\subsubsection*{Push-forward and pull-back}

Let $f: M \to N$  be a smooth map between smooth manifolds $M$ and $N$. It induces a \textbf{push-forward} of tangent vectors
$$
f_*:T_pM\to T_{f(p)}N
$$
which is defined by
$$
f_*(X_p)(g)=X_p(g\circ f)
$$
for $g\in C^\infty(N)$. As in \ref{df}, $f_*$ is often denoted by $df_p$ in the literature.
\begin{figure}[ht]\centering
\includegraphics{pictures/fig_tangent_map}
\end{figure}

On the other hand, it induces a \textbf{pull-back} of the cotangent space
$$
f^*: T^*_{f(p)}N\to T^*_pM
$$
which is defined by
$$
\langle f^*(\omega_{f(p)}), X_p\rangle=\langle\omega_{f(p)},f_*(X_p)\rangle
$$
for $\omega_{f(p)}\in T_{f(p)}N$ where $\langle \cdot ,\cdot \rangle$ is a natural pairing between the tangent and cotangent space.


If $f_*:T_pM\to T_{f(p)}N$ is surjection, \textit{i.e.} $\textrm{rank}(f_*)=\dim N$, then $f$ is \textbf{regular} at the point $p\in M$. Otherwise, $p \in M$ is called a \textbf{critical point} of $f$ and $f(p)\in N$ is called the \textbf{critical value} of $f$.


\bexample
Let $f:S^n\to \bR$ be a map defined by $f(x^0,\ldots,x^n)=x^n$ where $S^n=\{(x^0,\ldots,x^n)\in \bR^{n+1}| ~|x|=1\}$. Then, the north and south pole $x^n=\pm1$ are critical points of $f$ and a generic point is regular.
\eexample


\bthm[Sard]
The set of critical values of a smooth $f:M\to N$ has Lebesgue measure zero.
\ethm

A proof is given in \cite{milnor1965topology}.

\bdefn
Let $M,N$ be smooth manifolds and $f:M\to N$ be a smooth map.  If $f_*:T_pM\to T_{f(p)}N$ is injective for ${}^\forall p\in M$, $f$ is called an \textbf{immersion}. If an immersion $f:M\to N$ is a homeomorphism onto $f(M)\subset M$, then it is called an \textbf{embedding}.
\edefn

\bexample
Let us define a map $\gamma:\bR\to S^1\times S^1$ by $ t\mapsto (e^{2\pi i t},e^{2\pi i\alpha t})$. Since its derivative is given by $\frac{d\gamma}{dt}=(2\pi i e^{2\pi i t},2\pi i\alpha e^{2\pi i\alpha t})\neq 0$, $\gamma$ is an immersion. When $\alpha$ is irrational, $\gamma$ is one-to-one, but not embedding. Although $\bZ$ are isolated in $\bR$, $\gamma(\bZ)$ is not isolated in $\gamma(\bR)$.
\eexample

\bdefn
If $f_*$ is regular for ${}^\forall p\in M$, $f$ is called a \textbf{submersion}. For $q\in f(M)$, $f^{-1}(q)$ is a submanifold of codimension $\dim N$ in $M$.
\edefn


\bexample
The projection $f:M\times N \to M$ is a submersion.
\eexample


\subsection{Differential forms}

It is natural to consider an algebra generated by the basis of $T^*_pM$ over $\bR$ with unit 1 that satisfies
$$dx^{i} \wedge dx^{j}=-dx^{j} \wedge dx^{i}$$
Here \(\wedge\), called the \textbf{wedge product}, can be understood as a multiplication of this algebra, and we call \(\wedge\) the \textbf{exterior algebra}. For an $n$-dimensional manifold, we have a direct sum decomposition
$$
\wedge^{\bullet} T^*_pM=\bigoplus_{k=0}^{n} \wedge^{k} T^*_pM~.
$$
An element $\omega\in  \wedge^{k} T^*_pM$ defines an alternating $k$-linear map
\be\label{multilinear}
T_pM\times \cdots \times T_pM\to \bR
\ee
with
$$
\omega(X_{\sigma(1)} \cdots X_{\sigma(k)})=\textrm{sign} (\sigma)~ \omega\left(X_{1}, \cdots, X_{k}\right) \quad\left(X_{i} \in V\right)
$$
for $\sigma\in S_k$.
Moreover, for an element $\omega = \omega_1\wedge\omega_2\wedge\cdots\wedge\omega_k$, we define
$$\omega_1\wedge\omega_2\wedge\cdots\wedge\omega_k(X_1,X_2,\cdots,X_k)=\frac{1}{k!}\det(\omega_a(X_b))~.$$
More generally, if $\alpha$ is a $k$-form and $\beta$ is an $\ell$-form,
$$(\omega\wedge \eta)(X_1, \cdots, X_{k+\ell}) = \frac{1}{(k+\ell)!} \sum_{\sigma\in S_{k+\ell}} \textrm{sign} (\sigma)
~  \omega(X_{\sigma(1)} , \cdots, X_{\sigma(k)} )\eta(X_{\sigma(k+1)} , \cdots, X_{\sigma(k+l)} )~.$$


Like \eqref{cotangent-bundle}, we can consider a family of the vector spaces over the manifold $M$
$$
\wedge^{k} T^{*} M=\bigcup_{p} \wedge^{k} T_{p}^{*} M~,
$$
which satisfies Definition \ref{def:manifold} of a manifold. Given two local coordinates $(U;x^1,\ldots,x^n)$ and $(V;y^1,\ldots,y^n)$, the transformation is given by
$$
dx^{i_{1}} \wedge \cdots \wedge dx^{i_{k}}=\sum_{j,<\cdots<j_{k}} \frac{D\left(x^{i_{1}}, \cdots, x^{i_{k}}\right)}{D\left(y^{j_{1}}, \cdots, y^{j_{k}}\right)} d y^{j_{1}} \wedge \cdots \wedge d y^{j_{k}}
$$
where  \(\frac{D\left(x^{i_{1}}, \cdots, x^{i_{k}}\right)}{D\left(y^{j_{1}}, \cdots, y^{j_{k}}\right)}\) denotes the Jacobian.
Moreover, we write the set of all sections as
  \[
    \Omega^k (M) = \Gamma(\Lambda^k T^*M) ~.
  \]
  In particular, we have $\Omega^0(M) = C^\infty(M)$.
An element of $\Omega^k(M)$ is known as a \term{differential $k$-form}. As a result, a $k$-form is expressed as
\bea\omega&=\sum_{i_{1}<\cdots<i_{k}} f_{i_{1} \cdots i_{k}}\left(x^{1}, \cdots, x^{n}\right) dx^{i_{2}} \wedge \cdots \wedge dx^{i_{k}}\cr
&=\sum_{i_{1}<\cdots<i_{k}} g_{i_{1} \cdots i_{k}}\left(y^{1}, \cdots, y^{n}\right) d y^{i_{2}} \wedge \cdots \wedge d y^{i_{k}}
\eea
on the intersection of two local charts  $(U;x^1,\ldots,x^n)$ and $(V;y^1,\ldots,y^n)$, and $f$ and $g$ are related by the Jacobian. Putting all $p$ together in \eqref{multilinear}, an element $\omega\in \Omega^k(M)$ defines a alternating $k$-linear map
$$\omega : \mathfrak{X}(M) \times \cdots \times \mathfrak{X}(M) \longrightarrow C^{\infty}(M)~.$$


Moreover, there exists a unique linear map called \term{exterior derivative}
  \[
    d : \Omega^k(M) \to \Omega^{k + 1}(M)~,
  \]
  such that
  \begin{enumerate}
    \item On $\Omega^0(M)$ this is as previously defined, i.e.
      \[
        d f (X) = X(f)\text{ for all }X \in \mathfrak{X}(M).
      \]
    \item We have
      \[
        d \circ d = 0: \Omega^k(M) \to \Omega^{k + 2}(M).
      \]
    \item It satisfies the \term{Leibniz rule}
      \[
        d (\omega \wedge \eta) = d \omega \wedge \eta + (-1)^k \omega \wedge d \eta.
      \]
  \end{enumerate}
  For $\omega \in \Omega^{k}(M)$, the exterior derivative can be defined as
 \begin{align}\label{d}
     (d \omega)\left(X_{1}, \cdots, X_{k+1}\right)&=\frac{1}{k+1}\Big\{\sum_{i=1}^{k+1}(-1)^{i+1} X_{i}\left(\omega\left(X_{1}, \cdots, \widehat{X}_{i}, \cdots, X_{k+1}\right)\right) \cr  &+\sum_{i<j}(-1)^{i+j} \omega\left(\left[X_{i}, X_{j}\right], X_{1}, \cdots, \widehat{X}_{i}, \cdots, \widehat{X}_{j}, \cdots, X_{k+1}\right) \Big\}~,
 \end{align}
 where ${}^\forall X_1,\ldots,X_{k+1}\in \mathfrak{X}(M)$ and $\ \widehat{}\ $ means omitting.

In term of local coordinates $x^1, \cdots, x^n$, we can define the exterior derivative as
  \[
    d\left(\sum_{i_1 < \ldots < i_k} \omega_{i_1, \ldots, i_k}\;d x^{i_1} \wedge \cdots \wedge d x^{i_k}\right) = \sum d \omega_{i_1, \ldots, i_k}\wedge d x^{i_1} \wedge \cdots \wedge d x^{i_k}~.
  \]


  Let $f: M \to N$  be a smooth map between smooth manifolds $M$ and $N$. This induces
an algebra homomorphism
  $$f^*:\Omega^\bullet(N)\to \Omega^\bullet(M)~.$$
 The pull-back of differential forms associated to $f$ can be defined as
  \[
    (f^*\omega)(X_1, \cdots, X_k) = \omega (f_*(X_1), \cdots, f_*(X_k)).
  \]
  for $\omega\in \Omega^k(N)$ and  $X_1, \cdots, X_k \in \frakX(M)$. Note that the pull-back $f^*$ has the following property
  \begin{enumerate}
    \item $f^*:\Omega^k(N) \to \Omega^k(M)$ is a linear map.
    \item $f^*(\omega \wedge \eta) = f^*\omega \wedge f^*\eta$.
    \item If $g :N\to L$ is a smooth map between two manifolds $N$ and $L$, then $(g \circ f)^* = f^* \circ g^*$.
    \item It commutes with exterior derivative: $df^* = f^* d$.
  \end{enumerate}



  Let us introduce some operations on differential forms. For $X\in\mathfrak{X}(M)$, a linear map
$$
i(X): \Omega^{k}(M) \rightarrow \Omega^{k-1}(M)
$$
is defined by
\be\label{interior}
(i(X) \omega)\left(X_{1}, \cdots, X_{k-1}\right)=k \omega\left(X, X_{1}, \cdots, X_{k-1}\right)
\ee
for $\omega \in \Omega^{k}(M), X_{1}, \cdots, X_{k-1} \in \mathfrak{X}(M)$. Note that if $k=0,$ we define
$i(X)=0$. We call $i(X) \omega$ the \textbf{interior product} of $\omega$ by $X$. By definition, $i(X)$ is obviously linear with respect to functions.
Next, we shall define a linear operator
$$
L_{X}: \Omega^{k}(M) \longrightarrow \Omega^{k}(M)
$$
called the \textbf{Lie derivative}, also involving the vector field $X \in \mathfrak{X}(M)$. This is defined by
\be\label{Lie}
\left(L_{X} \omega\right)\left(X_{1}, \cdots, X_{k}\right) \\
\quad=X \omega\left(X_{1}, \cdots, X_{k}\right)-\sum_{i=1}^{k} \omega\left(X_{1}, \cdots,\left[X, X_{i}\right], \cdots, X_{k}\right) .
\ee
The interior products and Lie derivatives provide many identities. We refer to \cite[\S2.2]{morita2001geometry} for them. 






  \subsection{Integrals of differential forms}

    Let $M$ be $n$-dimensional orientable smooth manifold and $\omega\in \Omega^n(M)$.  We shall define the integral of $\omega$ over $M$. To this end, we define a partition of unity.
\bdefn
A cover $\{U_\alpha\}_{\alpha \in A}$ of $M$ is called \textbf{locally finite} if for all points $p\in M$, there exists a neighborhood $U_p\ni p$ such that it intersects only finitely many elements of the cover, namely $U_p\bigcap U_\alpha\neq \emptyset$ for only a finite number of $\alpha \in A$. Another cover $\{V_\beta\}_{\beta \in B}$ is called a \textbf{refinement} of \(\left\{U_\alpha\right\}_{\alpha \in A}\) when for ${}^\forall \beta\in B$, there exists $\alpha\in A$ such that $V_{\beta} \subseteq U_{\alpha}$.
\edefn

\bexample
The collection of all subsets of $\bR$ of the form $(n, n + 2)$ with integer $n$ is a locally finite  open covering of $\bR$. The collection of all subsets  the form $(n+\frac13, n + \frac53)$ with integer $n$ is its refinement.
\eexample

Let us accept the following theorem:
\bthm
Let $\{U_\alpha\}_{\alpha \in A}$ be an open covering of $M$. There exists a refinement $\{V_\alpha\}$ of $\{U_\alpha\}_{\alpha \in A}$, which is locally finite.
\ethm

  \bdefn[Partition of unity]\index{partition of unity}
  Let $\{U_\alpha\}$ be a locally-finite open cover of a manifold $M$. A \term{partition of unity} associated to $\{U_\alpha\}$ is a collection $\chi_\alpha \in C^\infty(M, \R)$ such that
  \begin{enumerate}
    \item $0 \leq \chi_\alpha \leq 1$
    \item $\supp(\chi_\alpha) \subseteq U_\alpha$
    \item $\sum_\alpha \chi_\alpha = 1$.
  \end{enumerate}
\edefn

If we use local coordinate $(x^1,\cdots,x^n)$ on a chart $U_\alpha$, we can write
$$
\chi_\alpha \omega=f_\alpha(x) x^1\wedge\cdots\wedge x^n~.
$$
Therefore, we define its integral
$$
\int_M \chi_\alpha \omega=\int \cdots \int f_\alpha(x) x^1\cdots x^n~.
$$
Then, we can define
$$
\int_M\omega=\sum_\alpha\int_M \chi_\alpha \omega~.
$$
One can show that this is independent of the choice of a locally-finite open covering $\{U_\alpha\}$  and a partition of unity on $\{U_\alpha\}$ .

  \bthm[Stokes' theorem]\index{Stokes' theorem}
  Let $M$ be an oriented manifold with boundary of dimension $n$. Then if $\omega \in \Omega^{n - 1}(M)$ has compact support, then
  \[
    \int_M d \omega = \int_{\partial M}\omega.
  \]
  In particular, if $M$ has no boundary, then
  \[
    \int_M d \omega = 0~.
  \]
\ethm



\end{document}
