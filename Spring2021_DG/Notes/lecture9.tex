\documentclass[geometry-lectures-21.tex]{subfiles}

\begin{document}

\section{Vector bundles and Principal \texorpdfstring{$G$}{G}-bundles}\label{sec:bundle}
We have learned tangent bundles, cotangent bundles and their tensor products. Generalizing these leads to a notion called vector bundles where a fiber is a vector space, namely $\bR^r$. Further generalizations will lead to fiber bundles where a fiber is a general manifold. Among them, principal $G$-bundles play a distinctive role where a fiber is a Lie group $G$ and transition funtions take the value on $G$.  Remarkably, the notion of vector bundles and principal $G$-bundles is indispensable for the description of non-Abelian gauge theories. I would recomend you to \cite{morita2001geometry,Bott:1982} for this subject.



\subsection{Vector bundles}
The notion of vector bundles was introduced by Whitney.

\bdefn[Vector bundle]\index{vector bundle}
  A \textbf{vector bundle} of rank $r$ on $M$ is a smooth manifold $E$ with a smooth \term{projection} $\pi: E \to M$ such that
  \begin{enumerate}
    \item For each $p \in M$, the fiber $\pi^{-1}(p) = E_p$ is an $r$-dimensional vector space,
    \item For all $p \in M$, there is an open $U \subseteq M$ containing $p$ and a diffeomorphism
      \[
        t: E_U = \pi^{-1}(U) \to U \times \bR^r
      \]
      such that
      \[
        \begin{tikzcd}
          E_U \ar[r, "t"] \ar[d, "\pi"] & U \times \bR^r \ar[dl, "p_1"]\\
          U
        \end{tikzcd}
      \]
      commutes, and the induced map $E_q \to \{q\} \times \bR^r$ is a linear isomorphism for all $q \in U$.

      We call $t$ a \term{trivialization} of $E$ over $U$; call $E$ the \term{total space}; call $M$ the \term{base space}. Also, for each $q \in M$, the vector space $E_q = \pi^{-1}(\{q\})$ is called the \term{fiber} over $q$. If $r=1$, it is called \term{line bundle}.
  \end{enumerate}
\edefn


\begin{figure}[ht]\centering
\includegraphics[width=5cm]{pictures/Mobius_strip_illus}
\end{figure}


\bdefn[Transition function]\index{transition function}
  Suppose that $t_\alpha: E|_{U_\alpha} \to U_\alpha \times \bR^r$ and $t_\beta: E|_{U_\beta} \to U_\beta \times \bR^r$ are trivializations of $E$. Then
  \[
    t_\alpha \circ t_\beta^{-1} : (U_\alpha \cap U_\beta) \times \bR^r \to (U_\alpha \cap U_\beta) \times \bR^r
  \]
  is fiberwise linear, i.e.
  \[
    t_\alpha \circ t_\beta^{-1}(q, v) = (q, g_{\alpha\beta}(q) v),
  \]
  where $g_{\alpha\beta}(q)$ is in $\GL(r,\bR)$.

  In fact, $g_{\alpha\beta}: U_\alpha \cap U_\beta \to \GL(r,\bR)$ is smooth. Then $g_{\alpha\beta}$ is known as the \term{transition function} from $\beta$ to $\alpha$.
\edefn
We have the following equalities for transition functions:
\bea\label{transition}
&(1)\quad g_{\alpha\alpha} = \id \cr
&(2)\quad g_{\alpha\beta} = g_{\beta\alpha}^{-1}\cr
&(3)\quad g_{\alpha\beta}g_{\beta\gamma}  g_{\gamma\alpha}=1~, \textrm{ which is called {\bf cocycle condition}. }
\eea

On the other hand, given an open cover $\{U_\alpha\}$ of open sets of $M$, suppose we have transition functions $g_{\alpha\beta}$ which satisfy all the above properties. Then, we can glue $U_\alpha\times \bR^n$ and $U_\beta\times \bR^n$ by the transition functions $g_{\alpha\beta}$ and construct a bundle $E\to M$.

\bexample
We construct a line bundle $L$ over the real projective space  $\bR P^n$ as follows. First we consider the direct product $\bR P^n\times \bR^{n+1}$. An arbitrary point $\ell$ in $\bR P^n$ can be regarded as a line through the origin in $\bR^{n+1}$. We construct a line bundle by
\bea
L&=\{(\ell,y )\in \bR P^n\times \bR^{n+1}\mid y \in \ell \}\cr
&=\{ ([x^0;\cdots;x^n], y)\in \bR P^n\times \bR^{n+1}|\ y=\lambda \cdot (x^0,\cdots,x^n)  ~, \quad \lambda\in\bR\} ~.
\eea
This line bundle is called \textbf{Hopf line bundle} or \textbf{tautological line bundle}.
For an open subset $U_i=\{\ell =[x^0; x^1;\cdots; x^n]|x_i\neq 0\}$, a trivialization map is given by
$$
t_i:\pi^{-1}(U_i)\to U_i\times \bR~; ~ (\ell ,y)\mapsto y^i~.
$$
Then, the transition map is given by
$$
t_j\circ t_i^{-1}: (U_i\cap U_j)\times \bR \to (U_i\cap U_j)\times \bR~;~ (\ell, \eta) \mapsto \left(\ell , \frac{x^j}{x^i}\eta \right)
$$
It is easy to see that the transition functions obey \eqref{transition} above.
\eexample



\begin{figure}[ht]\centering
\includegraphics[width=7cm]{pictures/fig_vector_bundle}
\end{figure}

\bdefn[Section]\index{section}
  A \textbf{section} of a vector bundle $\pi: E \to M$ is a map $s: M \to E$ such that $\pi \circ s = \id$. In other words, $s(p) \in E_p$ for each $p\in M$. We denote a set of sections by $\Gamma(M,E)$.
\edefn







  \bdefn[Bundle map]\label{Bundle-map}
We can  consider about maps between vector bundles.  Let $E \to M$ and $E' \to M'$ be vector bundles. A \textbf{bundle map} from $E$ to $E'$ is a pair of smooth maps $(F: E \to E', f: M \to M')$ such that the following diagram commutes:
  \[
    \begin{tikzcd}
      E \ar[d] \ar[r, "F"] & E' \ar[d]\\
      M \ar[r, "f"] & M'
    \end{tikzcd}.
  \]
  i.e.\ such that $F_p: E_p \to E'_{f(p)}$ is linear for each $p$.
\edefn

Two vector bundles $\left(E_{i}, \pi_{i}, M\right), i=1,2$, over the same base $M$ are said to be \textbf{isomorphic} if there is a bundle map $E_{1} \rightarrow E_{2}$ over the identity map of $M$. In this case, we write $E_{1} \cong E_{2} .$

In fact, every operation on vector spaces can be performed on vector bundles, by doing it on each fiber.
\bdefn[Whitney sum of vector bundles]\index{vector bundle!Whitney sum}\index{Whitney sum of vector bundles}
  Let $\pi: E \to M$ and $\rho: F \to M$ be vector bundles. The \textbf{Whitney sum} is given by
  \[
    E \oplus F = \{(e, f)\in E \times F: \pi(e) = \rho(f)\}.
  \]
  This has a natural map $\pi \oplus \rho: E \oplus F \to M$ given by $(\pi \oplus \rho)(e, f) = \pi(e) = \rho(f)$. This is again a vector bundle, with $(E \oplus F)_p = E_p \oplus F_p$ and again local trivializations of $E$ and $F$ induce one for $E \oplus F$.
\edefn
Tensor products can be defined similarly.

\bdefn[Tensor product of vector bundles]
  Given two vector bundles $E, F$ over $M$, we can construct $E \otimes F$ similarly with fibers $(E \otimes F)|_p = E|_p \otimes F|_p$.
\edefn


Similarly, we can construct the alternating product of vector bundles $\Lambda^n E$. Finally, we have the \textbf{dual} vector bundle.

\bdefn[Dual vector bundle]
  Given a vector bundle $E \to M$, we define the \textbf{dual vector bundle} by
  \[
    E^* = \bigcup_{p \in M} (E_p)^*.
  \]
  Suppose again that $t_\alpha: E|_{U_\alpha} \to U_\alpha \times \bR^n$ is a local trivialization. Taking the dual of this map gives
  \[
    t_\alpha^*: U_\alpha \times (\bR^n)^* \to E|_{U_\alpha}^*.
  \]
  since taking the dual reverses the direction of the map. We pick an isomorphism $(\bR^n)^* \to \bR$ once and for all, and then reverse the above isomorphism to get a map
  \[
    E|_{U_\alpha}^* \to U_\alpha \times \bR^n.
  \]
  This gives a local trivialization.
\edefn


One important operation we can do on vector bundles is \textbf{pullback}:
\bdefn[Pullback of vector bundles]
  Let $\pi: E \to M$ be a vector bundle, and $f: N \to M$ a map. We define the \textbf{pullback}
  \[
    f^* E = \{(y, e) \in N \times E: f(y) = \pi(e)\}.
  \]
  This has a map $f^*\pi: f^*E \to N$ given by projecting to the first coordinate. The vector space structure on each fiber is given by the identification $(f^*E)_y = E_{f(y)}$. It is a little exercise in topology to show that the local trivializations of $\pi: E \to M$ induce local trivializations of $f^*\pi: f^* E \to N$.
\edefn



\bdefn[Subbundles and quotient bundles]
Let $\pi: E \rightarrow M$ be a vector bundle over $M .$ A vector bundle $\pi: F \rightarrow M$ is called a \textbf{subbundle} if $F$ is a submanifold of $E$ such that, for each point $p \in M$, the fiber $F_{p}$ is a vector subspace of the fiber $E_{p}$ of $E$.

Given a subbundle $F$ of $E$, consider the quotient subspace $E_{p} / F_{p}$ for each point $p \in M$, and set
$$
E / F=\bigcup_{p \in M} E_{p} / F_{p}
$$
We can verify that the natural projection $\pi: E / F \rightarrow M$ is a vector bundle, called the \textbf{quotient bundle} of $E$ by $F$.
\edefn

\bexample  If $f: M \hookrightarrow N$ is an embedding, $TM\to M$ is a subbundle of the pull-back bundle $f^{*} T N\to M$. The quotient bundle $f^{*} T N / T M=N M$ is a vector bundle over $M$ called the \textbf{normal bundle} of $M$.
\eexample


One can introduce a metric $g$ on fibers of a vector bundle $E$. Namely, we have a non-degenerate symmetric positive-definite 2-form on a fiber $E_p$
$$
g_p:E_p\times E_p\to \bR~,
$$
and $g_p$ is differentiable in terms of $x$. If it is the tangent bundle $TM$, it is a Riemannian metric. Given a trivialization $\pi^{-1}(U_\alpha)\to U_\alpha \times \bR^r$, we can take an orthonormal frame $(e_1,\cdots,e_r)$ for $e_i\in\Gamma(U,E)$ with respect to $g$. Then, the transition function takes the value at $\textrm{O}(r)$
$$
g_{\alpha\beta}:U_\alpha\cap U_\beta \to \textrm{O}(r)~.
$$
We can further generalize that the transition function takes the value at an arbitrary Lie group $G$ with a representation $\rho: G \to  \GL(V)$.
\bdefn[$G$-bundle]\index{$G$-bundle}
  Let $V$ be a vector space, $G$ a Lie group, and $\rho: G \to \GL(V)$ a representation. Then a $G$-bundle $\pi:E\to M$ consists of the following data:
  \begin{enumerate}
    \item For each $p\in M$, the fiber is $\pi^{-1}(p)\cong V$.
    \item  One can take a trivializing cover $\{U_\alpha\}$ with transition functions $g_{\alpha\beta}:(U_\alpha \cap U_\beta) \times V \to (U_\alpha \cap U_\beta) \times V$.
    \item The transition functions are constructed by maps $g_{\alpha\beta}: U_{\alpha} \cap U_\beta: \to G$ satisfying the cocycle conditions with the representation $\rho$ such that $g_{\alpha\beta} = \rho \circ g_{\alpha\beta}$.
  \end{enumerate}
\edefn


\subsection{Principal $G$-bundles}

We have studied vector bundles where a fiber is a vector space. We can further generalize it to a \textbf{fiber bundle} where a fiber is a general manifold $F$ and a transition function is given by a diffeomorphism of $F$. Among fiber bundles, principal $G$-bundles play an important role in physics.




\bdefn[Principal $G$-bundle]\index{principal $G$-bundle}
 Let $G$ be a Lie group, and $M$ a manifold. A \term{principal $G$-bundle} is a smooth manifold $P$ with a projection $\pi: P \to M$ such that a fiber is $\pi^{-1}(\{x\}) \cong G$ for each $x \in M$.
 More precisely, we are given an open cover $\{U_\alpha\}$ of $M$ and diffeomorphisms
 \[
  \begin{tikzcd}
     \pi^{-1}(U) \ar[r, "t"] \ar[d, "\pi"] & U \times G \ar[dl, "p_1"]\\
     U
    \end{tikzcd} \]
such that the transition functions
 \[
  t_\alpha \circ t_\beta^{-1}: (U_\alpha \cap U_\beta) \times G \to (U_\alpha \cap U_\beta) \times G
 \]
 is of the form
 \[
  (q, g) \mapsto (q, g_{\alpha\beta}(q) \cdot g)
 \]
 for some $g_{\alpha\beta}: U_\alpha \cap U_\beta \to G$ where $G$ is called the \term{structure group}. Transition funcitons $\{g_{\alpha\beta}\}$ obey the three conditions \eqref{transition}.
\edefn

A bundle map for two principal $G$-bundles is defined in a similar manner to Definition \ref{Bundle-map}, and a bundle map over the identity map of $M$ defines the isomorphism between two principal $G$-bundles.



For $g\in G$ we can define the right action $R_g$ on the total space $P$
$$
R_gP \to P; u\mapsto ug
$$
where each fiber onto itself. For a principal bundle to be trivial it is necessary and sufficient that it admits a section. Namely, if there is a section $s:M\to P$, one can have a trivialization by setting $s(p)=e$ and the other points can be specified in $M\times G$ by the right $G$-action.


Given a principle $G$-bundle and a representation $\rho:G\to \GL(V)$ for a vector space $V$, we can construct \term{associated vector bundle}
$$
E=P\times_\rho V
$$
as follow. Let us consider the direct product $P\times V$ and $G$ action
$$
(u,y)\mapsto (us,\rho(s)^{-1}y) \qquad \textrm{for}  \quad s\in G
$$
We define the associated bundle as the quotient space $P\times_\rho V:=(P\times V)/G$. Conversely, given a $G$-bundle $E\to M$ with fiber $V$, there is a canonical way of producing a principal $G$-bundle by using transition functions.



\subsection{Connections and curvatures}
In Riemannian geometry \S\ref{sec:Riemann}, we have learned Levi-Civita connections and Riemann curvature. Even in vector bundles and principal $G$-bundles, we can introduce connections and curvatures.

\bdefn[Connection]\index{connection}
 A \term{connection} \index{$\nabla$} in a vector bundle $\pi:E\to M$ is a bilinear map
 $$\nabla:\mathfrak{X}(M)\times \Gamma(E) \to \Gamma(E);(X,s)\mapsto \nabla_X s$$
 satisfying
 \begin{enumerate}
  \item $ \nabla_{fX}(s) = f\nabla_{X}(s) $

  \item Leibnitz property: $ \nabla_X (fs) = (X f) s + f (\nabla_X s)$
 \end{enumerate}
 for all $s \in \Gamma(E)$ and $f \in C^\infty(M)$.
\edefn

If a vector bundle $E$ has a metric $g$, a connection $\nabla$ is \term{compatible} with the metric if
$$
X g(s_1,s_2)= g(\nabla_X s_1,s_2)+g( s_1,\nabla_X s_2)
$$
for any $X\in \mathfrak{X}(M)$ and $s_1,s_2 \in \Gamma(E)$.

\bdefn[Curvature]\index{Curvature}
 A \term{curvature} \index{$\nabla$} in a vector bundle $\pi:E\to M$ is a trilinear map
 $$F:\mathfrak{X}(M)\times \mathfrak{X}(M)\times \Gamma(E) \to \Gamma(E);(X,Y,s)\mapsto F(X,Y) s$$
 defined by
 $$
 F(X,Y)s=\left[ \nabla_X\nabla_Y-\nabla_Y\nabla_X-\nabla_{[X,Y]}\right]s~.
 $$
 It has the following properties
  \begin{enumerate}
  \item $ F(X,Y)s =-F(Y,X)s $
  \item $F(fX,gY)(hs) = fghF(X,Y)s $ for $f,g,h\in C^\infty(M)$
 \end{enumerate}
 for all $s \in \Gamma(E)$.
\edefn


For a certain open subset of $M$, we can take a frame $s_1,\cdots,s_r \in \Gamma(\pi^{-1}(U))$. For any vector field $X$ on $U$, the connection can be locally written as
$$
\nabla_X s_j=\sum_{i=1}^rs_iA^i{}_j(X)
$$
where $A^i{}_j\in \Omega^1(U,\mathfrak{gl}(r,\bR))$ (1-form on $U$ taking its value on $\mathfrak{gl}(r,\bR)$) is called \term{connection form}. We now look at the curvature $R$ from differential forms.
$$
F(X,Y)s_j=\sum_{i=1}^r s_iF^i{}_j (X,Y)
$$
where $F^i{}_j\in \Omega^2(U,\mathfrak{gl}(r,\bR))$ (2-form on $U$ taking its value on $\mathfrak{gl}(r,\bR)$) is called \term{curvature form}. They are related by the following equation:
$$
F=dA+A\wedge A~.
$$
The curvature form satisfies the Bianchi identity
\be\label{Bianchi}
dF-F\wedge A+A\wedge F=0~.
\ee

More explicitly, in physics, we write a section $s=\sum_{j=1}^rv^j(x)s_j$ on $U$ so that
$$
\nabla_{\frac{\partial}{\partial x^\mu}} s=\sum_{i=1}^rs_i\left[\frac{\partial}{\partial x^\mu} v^i(x) + (A_{\mu}){}^i{}_j v^j(x) \right]~.
$$
Also, the curvature can be written in terms of local coordinates
$$
F\left(\frac{\partial}{\partial x^\mu},\frac{\partial}{\partial x^\nu}\right)=F_{\mu\nu}=\frac{\partial A_\nu}{\partial x^\mu}-\frac{\partial A_\mu}{\partial x^\nu} +[ A_\mu, A_\nu]~.
$$
In the case of Maxwell $\U(1)$ theory, the last term vanishes because it is a commutative group.



It is useful to know how the connection transforms under a change of local trivialization. Given a transition function $g_{\alpha\beta}: U_\alpha \cap U_\beta \to GL(r,\bR)$, the gauge fields on $U_\alpha$ and $U_\beta$ are related by
\be\label{gauge-trans}
 A_\beta = g_{\alpha \beta}^{-1} A_\alpha g_{\alpha\beta} + g_{\alpha\beta}^{-1} d (g_{\alpha\beta})~,
\ee
and the curvature forms are related by
\be\label{gauge-trans2}
F_\beta=g_{\alpha \beta}^{-1} F_\alpha g_{\alpha\beta}
\ee

In a similar manner, one can construct connections and curvatures for $G$-bundle where we have to replace the first term of \eqref{gauge-trans} by the adjoint representation of $G$ on $\mathfrak{g}$, and the second of \eqref{gauge-trans} by the Maurer--Cartan form \eqref{MC-form}.

In particular, for the Maxwell theory, the gauge transformation can be written as $g_{\alpha\beta}=e^{i\lambda_{\alpha\beta}(x)}$ so that
$$
A_\beta=A_\alpha + i d\lambda_{\alpha\beta}
$$
and the curvature form stays invariant.



\subsubsection*{Parallel transport and holonomy group}
Given a connection $\nabla$ on a vector bundle $E$, on can define \term{horizontal} directions in the space $\Gamma(E)$ of sections. A section $s\in \Gamma(E)$ is \term{parallel} along a path $\gamma:I\to (M)$
$$
\nabla_{\dot \gamma(t)} s=0 \quad \textrm{for} \quad t\in I~.
$$
In terms of local coordinates, it can be written as
$$
\frac{ds_i}{dt}+\sum_{j=1}^r  s_j (A_\mu)^j{}_i \frac{dx^\mu}{dt}=0~.
$$
A theorem of ordinary differential equations tells us that given an initial data $s(t=0)\in E_{\gamma(0)}$, one can do parallel transform along $\gamma(t)$ so that we have a map
$$
E_{\gamma(0)} \ni s(t=0) \to s(t=1)\in E_{\gamma(1)}~.
$$
In particular, if we consider a curve $p=\gamma(0)=\gamma(1)$, we obtain a map $\tau_\gamma:E_p\to E_p$. For given two curves $\gamma_1$ and $\gamma_2$, we can have a multiplication
$$
\tau_{\gamma_1\circ \gamma_2}=\tau_{\gamma_1}\circ \tau_{\gamma_2}
$$
and the inverse is defined by
$$
\tau_{\gamma^{-1}}=\tau_{\gamma}^{-1}~,
$$
so that it forms a group called \term{holonomy group}. In the case of $\U(1)$, this is the origin of the \term{Aharonov-Bohm effect}.

\begin{figure}[ht]\centering
\includegraphics[width=10cm]{pictures/abm}
\end{figure}
\begin{figure}[ht]\centering
\includegraphics[width=7cm]{pictures/Aharonov-Bohm}
\caption{The Aharonov-Bohm solenoid effect takes place when the wave function of a charged particle passing around a long solenoid experiences a phase shift as a result of the enclosed magnetic field, despite the magnetic field being negligible in particle trajectories.}
\end{figure}

\subsubsection*{Levi-Civita connections}
We can consider the tangent bundle $TM$ as a vector bundle and its metric $g$ is indeed a Riemannian metric. As we have seen in \S\ref{sec:Riemann}, there is the unique natural connection called \term{Levi-Civita connections} in $TM$. Let $X_1, \cdots, X_n$ be an orthonormal frame vector field on an open set $U\subset M$ and their dual $e^1,\cdots,e^n\in \Omega^1(U)$. For the Levi-Civita connection $\nabla$, we can write
\begin{align}
\nabla_{X_j}X_i&=\sum_k\mathbf{\Gamma}^k_{ij}X_k~,\cr
R(X_i,X_j)X_k&=\sum_{l}\mathbf{R}^{l}{}_{kij}X_l~.\nonumber
\end{align}

We can define connection one-form and curvature two-form taking their values on $\mathfrak{so}(n,\bR)$:
\be\label{Riemann-2-form}
\omega^k{}_j=\sum_k\mathbf{\Gamma}^k_{ij} e^i~, \quad \Omega^l{}_k=\sum_{l}\mathbf{R}^{l}{}_{kij} e^i\wedge e^j~.
\ee
They satisfy the following conditions
 \begin{align}
de^i &=-\sum_j \omega^{i}{}_j\wedge e^j~,\cr
\Omega^i{}_j&=d\omega^i{}_j+\omega^i{}_k\wedge \omega^k{}_j~.\nonumber
 \end{align}








\subsubsection*{Ehresmann connections}
Similarly, one can construct the theory of connections on principal $G$-bundles, which are called \term{Ehresmann connections}. In principle, an Ehresmann connection determines the horizontal direction of a principal $G$-bundle.



\bdefn[Ehresmann connection]\index{Ehresmann}
 A \term{Ehresmann connection} $A\in \Omega^1(P,\frakg)$ on a principle $G$-bundle $\pi:P\to M$ is a one-form taking its value on $\frakg$, which satisfies the following conditions:
  \begin{enumerate}
  \item Given $X\in \frakg$, there is the corresponding vector field $\overline X$ on $P$
  $$
  \overline X_u=\frac{d}{dt}(u\cdot \exp t X)\Big|_u \quad \textrm{for} \quad u\in P~.
  $$
 Then, $A$ is subject to
 $$
 A(\overline X)=X \in \frakg~.
 $$
  \item For ${}^\forall g\in G$, under the right action $R_g$, it behaves as
  $$ R^\ast_g(A)=\textrm{ad}(g^{-1}) A~.$$
  In other words,
  $$ A_{ug}(R_g(Y))=\textrm{ad}(g^{-1})(A_u(Y)) ~, \quad \textrm{for} \quad Y\in \frakg ~.  $$
 \end{enumerate}
\edefn

For an open set $U_\alpha\subset M$, there is a local trivialization $t_{\alpha}: \pi^{-1}(U_\alpha)\to U_\alpha\times G$. Then, we have a section on $\pi^{-1}(U_\alpha)$
$$ \sigma_{U_\alpha}:U_\alpha\to P; x\mapsto t_{\alpha}^{-1}(x,e)~.$$
Then, if we define $A_\alpha=\sigma^*_{U_\alpha}(A)$, we have cocycle condition
$$
 A_\beta = g_{\alpha \beta}^{-1} A_\alpha g_{\alpha\beta} + g_{\alpha\beta}^{-1} d (g_{\alpha\beta})~, \quad \textrm{for} \quad U_\alpha\cap U_\beta~,
$$
where $g_{\alpha\beta}$ is the transition function on $U_\alpha\cap U_\beta$.

We denote the space of Ehresmann connections on a principal $G$-bundle $P$ by $\mathscr{A}_P$. The space $\mathscr{G}_P$ of gauge transformations is the section $\Gamma(M,G_P)$ of the bundle $G_P:=P\times_{Ad}G$. A gauge transformation $g\in \mathscr{G}_P$ acts on the space of connection $\mathscr{A}_P$ via
$$
\mathscr{G}_P\times \mathscr{A}_P\to \mathscr{A}_P; (g,A)\mapsto g^*(A)=\{g^*(A)_U,g^*(A)_V, \cdots\}
$$
where
$$
g^*(A)_U=g_U^{-1}d(g_U)+g_U^{-1}Ag_U~.
$$
If $A'=g^*(A)$ for $A,A'\in \mathscr{A}_P$, they are physically identical, and we denote the space of physically different connections by
$$
\mathscr{B}_P:=\mathscr{A}_P/\mathscr{G}_P~.
$$






\subsection{Yang-Mills theory}
Now we can describe non-Abelian gauge theory called \textbf{Yang-Mills theory} \cite{yang1954conservation}.
Let us consider a principal $G$-bundle or its associated $G$-bundle. In addition, let $A$ be a connection on it and $F$ be its curvature.
The classical \term{Yang-Mills action} can be written as
\begin{align}
 S_{YM}[A] &= \frac{1}{2g^2_{YM}} \int_M \Tr F\wedge \ast F\cr
 &= \frac{1}{2g^2_{YM}} \int_M \Tr \left(F_{\mu\nu} F^{\mu\nu}\right) \;\sqrt{g} d^d x,
\end{align}
where the curvature is 2-form taking its value on the Lie algebra $\mathfrak{g}$. The action is almost the same as that of the Maxwell theory. However, due to the gauge transformation \eqref{gauge-trans2} of the curvature form,
we need to take $\Tr$ over Lie algebra $\mathfrak{g}$ in order for the action to be gauge-invariant.
The parameter $g_{YM}$ is called the Yang-Mills coupling constant. For the flat space, we have $\sqrt{g} = 1$ so that we will drop this term.

For example, if $G = \SU(N)$, we can choose the basis of the Lie algebra $\mathfrak{su}(N)$
\[
 \Tr(T_a T_b) = \frac{1}{2} \delta_{ab},
\]
and on a local $U \subseteq M$, we have
\[
 S_{YM}[A] = \frac{1}{4 g^2_{YM}} \int F_{\mu\nu}^a F^{b, \mu\nu} \delta_{ab} \; d^d x,
\]
with $F_{\mu\nu}=\sum_a F_{\mu\nu}^a T_a$ and
\[
 F_{\mu\nu}^a = \partial_\mu A^a_\nu - \partial_\nu A_\mu^a + f_{bc}{}^a A_\mu^b A_\nu^c~.
\]
Thus, Yang-Mills theory is the natural generalization of Maxwell theory to the non-Abelian case.



At the level of the classical field equations, if we vary our connection by $A_\mu \mapsto A_\mu + \delta a_\mu$, where $\delta a$ is a matrix-valued $1$-form, then we have
\[
 \delta F_{\mu\nu} = \partial_{[\mu} \delta a_{\nu]} + [A_\mu, \delta a_\nu].
\]
The equation of motion can be obtained by taking the variation of the Yang-Mills action
\[
\delta S_{YM}[A] = \frac{1}{2g_{YM}^2} \int \Tr(\delta F_{\mu\nu}, F^{\mu\nu}) \;d^d x = \frac{1}{2g_{YM}^2} \int \Tr(\nabla_\mu \delta a_\nu, F^{\mu\nu})\;d^d x = 0.
\]
Therefore, the \term{Yang--Mills equation} is
\[
 \nabla^\mu F_{\mu\nu}: = \partial^\mu F_{\mu\nu} + [A^\mu, F_{\mu\nu}] = 0~,
\]
or we can write it without coordinates
$$
\delta_A F:=\ast d_A\ast F=\ast (d+A)\ast F=0~.
$$
Recall we also have the Bianchi identity \eqref{Bianchi} which can be expressed in terms of a local coordinate
\[
 \nabla_\mu F_{\nu\lambda} + \nabla_\nu F_{\lambda\mu} + \nabla_\lambda F_{\mu\nu} = 0~.
\]
\term{Unlike} Maxwell's equations, these are non-linear PDE's for $A$. We no longer have the principle of superposition, which is similar to Einstein's equation \eqref{Einstein-eq}.

The Yang-Mills path integral is expressed by
$$
Z_{YM}=\int_{\mathscr{A}/\mathscr{G}} DA~ \exp({iS_{YM}[A]})~.
$$
You can try to solve one of the seven Millennium Prize Problems: quantum Yang-Mills theory exists on $\mathbb {R} ^{4}$ and has a mass gap \cite{jaffe2006quantum}.



\end{document}
