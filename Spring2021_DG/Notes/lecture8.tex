\documentclass[geometry-lectures-21.tex]{subfiles}

\begin{document}




\section{Fundamental groups and Homotopy groups}




\subsection{Fundamental groups}


  A \textbf{path} in a topological space $M$ is a map $\gamma: I\to M$. If $\gamma(0) = x_0$ and $\gamma(1) = x_1$, we say $\gamma$ is a path from $x_0$ to $x_1$.  If $\gamma(0) = \gamma(1)$, then $\gamma$ is called a \textbf{loop} (based at $x_0$).  The idea of fundamental groups is to consider homotopy equivalence of loops at a fixed base point $x_0$

\bdefn[Fundamental group]
  Let $M$ be a topological space and $x_0 \in M$. The \textbf{fundamental group} of $M$ (based at $x_0$), denoted $\pi_1(M, x_0)$, is the set of homotopy classes of loops in $M$ based at $x_0$ (i.e.\ $\gamma(0) = \gamma(1) = x_0$). The group operations are defined as follows:

  We define a group multiplication by $[\gamma_0][\gamma_1] = [\gamma_0\cdot \gamma_1]$; inverses by $[\gamma]^{-1} = [\gamma^{-1}]$; and the identity as the constant map $e = [c_{x_0}]$.
\edefn



It is easy to see from the figure below that fundamental groups are independent of the choice of a base point. Therefore, it is often written as $\pi_1(M)$.
\begin{center}
  \begin{tikzpicture}
    \draw plot [smooth cycle] coordinates {(-2.4, -0.7) (0, -0.7) (1.4, -1) (2.6, -0.9) (2.8, 0.6) (0.6, 1.9) (-3.4, 0.8)};

    \node [circ] at (-2, 0.2) {};
    \node [anchor = south west] at (-2, 0.2) {$x_0$};
    \node [circ] at (1.5, 0.1) {};
    \node [above] at (1.5, 0.1) {$x_1$};

    \draw [->-=0.5] (-2.4, 0.2) circle [radius = 0.4];
    \draw [->-=0.5] (-2, 0.2) parabola bend (-0.25, 0) (1.5, 0.1) node [pos=0.5, below] {$u$};
    \draw [mred, ->-=0.3, ->-=0.7] (1.5, 0.1) -- (1.5, 0.2) parabola bend (-0.25, 0.1) (-2, 0.3) arc (14:346:0.4123) parabola bend (-0.25, -0.1) (1.5, 0) -- (1.5, 0.1);
  \end{tikzpicture}
\end{center}


Unlike homology groups, the fundamental groups are non-abelian groups in general.
\bexample
Let us consider the space  $M=\bR^2\backslash \{p_1,p_2\}$ which is obtained by removing two points from the plane.
The fundamental group of $M$ is the free group  $\pi_1(M) = F(\{ a, b\})$ with the two generators. See Definition \ref{def:free-group}. It is easy to understand the free group on two generators by the \textbf{Cayley graph} below.
\eexample
\begin{figure}[h]\centering
  \begin{minipage}[b]{0.4\textwidth}
 \raisebox{1cm}{ \begin{tikzpicture}
    \draw (-1.5, 0) circle [radius=1.5];
    \draw (1.5, 0) circle [radius=1.5];
    \draw (-1.5, 0) circle [radius=.1];
    \draw (1.5, 0) circle [radius=.1];
    \node [circ] {};
    \node [left] at (-3, 0) {$a$};
    \node [right] at (3, 0) {$b$};
    \node [left] at (-1.5, 0) {$p_1$};
    \node [right] at (1.5, 0) {$p_2$};
    \draw [->] (3, 0.1) -- (3, 0.11);
    \draw [->] (-3, 0.1) -- (-3, 0.11);
    \node [right] {$x_0$};
\end{tikzpicture}}
\end{minipage}
\hspace{1cm}
\begin{minipage}[b]{0.4\textwidth}
  \begin{tikzpicture}[scale=0.5]
    \draw (-3, 0) -- (3, 0);
    \draw (0, -3) -- (0, 3);
    \foreach \x/\y/\rot in {3/0/0, 0/3/90, -3/0/180, 0/-3/270} {
      \begin{scope}[shift={(\x, \y)}, scale=0.5, rotate=\rot]
        \draw (0, 0) -- (3, 0);
        \draw (0, 3) -- (0, -3);
        \foreach \x/\y/\rot in {3/0/0, 0/3/90, 0/-3/270} {
          \begin{scope}[shift={(\x, \y)}, scale=0.5, rotate=\rot]
            \draw (0, 0) -- (3, 0);
            \draw (0, 3) -- (0, -3);
            \foreach \x/\y/\rot in {3/0/0, 0/3/90, 0/-3/270} {
              \begin{scope}[shift={(\x, \y)}, scale=0.5, rotate=\rot]
                \draw (0, 0) -- (3, 0);
                \draw (0, 3) -- (0, -3);
                \foreach \x/\y/\rot in {3/0/0, 0/3/90, 0/-3/270} {
                  \begin{scope}[shift={(\x, \y)}, scale=0.5, rotate=\rot]
                    \draw (0, 0) -- (3, 0);
                    \draw (0, 3) -- (0, -3);
                    \foreach \x/\y/\rot in {3/0/0, 0/3/90, 0/-3/270} {
                      \begin{scope}[shift={(\x, \y)}, scale=0.5, rotate=\rot]
                        \draw (0, 0) -- (3, 0);
                        \draw (0, 3) -- (0, -3);
                      \end{scope}
                    }
                  \end{scope}
                }
              \end{scope}
            }
          \end{scope}
        }
      \end{scope}
    }
    \node [circ] {};
    \node [anchor = south west] {$\widetilde{x}_0$};
    \draw [->] (1.7, 0) -- (1.71, 0);
    \node [above] at (1.5, 0) {$a$};
    \draw [->] (-1.3, 0) -- (-1.29, 0);
    \node [above] at (-1.5, 0) {$a$};

    \draw [->] (0, 1.7) -- (0, 1.71);
    \node [left] at (0, 1.5) {$b$};
    \draw [->] (0, -1.3) -- (0, -1.29);
    \node [left] at (0, -1.5) {$b$};

    \draw [->] (3.9, 0) -- (4, 0);
    \node [above] at (3.75, 0) {$a$};

    \draw [->] (3, 0.9) -- (3, 1);
    \node [left] at (3, 0.75) {$b$};
    \draw [->] (3, -0.5) -- (3, -0.4);
    \node [left] at (3, -0.75) {$b$};
  \end{tikzpicture}
\end{minipage}
\end{figure}

Usually, a fundamental group is presented as $\langle S\mid R\rangle$ by generators $S$ and the relations $R$. See Definition \ref{def:presentation}. As it can be easily read off from Figure \ref{fig:torus}, the fundamental group of a torus is
$$
\pi_1(T^2)\cong \langle a,b\mid aba^{-1}b^{-1}\rangle~.
$$




  A \textbf{based space} is a pair $(M, x_0)$ of a topological space $M$ and a \textbf{base point} $x_0\in M$. A \textbf{map of based spaces}
  \[
    f: (M, x_0) \to (N, y_0)
  \]
  is a continuous map $f: M\to N$ such that $f(x_0) = y_0$. A based map $f$ induces the group homomorphism
  \[
    f_* : \pi_1(M, x_0) \to \pi_1(N, y_0),
  \]
  defined by $[\gamma] \mapsto [f\circ \gamma]$. Thus, the induced homomorphisms are natural under compositions: For $\begin{tikzcd} (M, x_0) \ar[r, "h"] & (N, y_0) \ar[r, "h"] & (L, z_0) \end{tikzcd}$, we have $(k\circ h)_\ast = k_\ast\circ h_\ast$
  Moreover, if two based maps $f_0, f_1$ are homotopic $f_0 \simeq f_1$, then $(f_0)_\ast=(f_1)_\ast$.





\bthm
  If $f: M\to N$ is a homotopy equivalence, and $x_0 \in M$, then the induced map
  \[
    f_*: \pi_1(M, x_0) \to \pi_1(N, f(x_0)).
  \]
  is an isomorphism.
\ethm

\bexample
Let us denote the space obtained by joining two copies of $S^1$ at one point by $S^1 \vee S^1$. Then, it is homotopic to $\bR^2\backslash \{p_1,p_2\}$ so that the fundamental group is $\pi_1(S^1\vee S^1)\cong F(\{ a,b\})$.
\eexample
% \begin{proof}
%   Let $g: Y\to M$ be a homotopy inverse so that $f\circ g\simeq \id_N$ and $g\circ f\simeq \id_M$.
%   \begin{center}
%     \begin{tikzpicture}
%       \draw plot [smooth cycle] coordinates {(-3.7, -1.2) (-3.7, 0) (-4, 0.7) (-3.9, 1.3) (-2.4, 1.4) (-1.8, 0.3) (-2.2, -1.7)};
%       \draw plot [smooth cycle] coordinates {(1.3, -1.2) (1.3, 0) (0.7, 0.7) (0.6, 1.3) (2.6, 1.4) (3, 0.3) (2.8, -1.7)};
%
%       \node [circ] at (-3, 0.7) {};
%       \node [above] at (-3, 0.7) {$x_0$};
%       \node [above] at (-3, 1.6) {$M$};
%
%       \node [above] at (1.8, 1.6) {$N$};
%
%       \node [circ] at (1.8, 0.8) {};
%       \node [below] at (1.8, 0.8) {$f(x_0)$};
%
%       \node [circ] at (-3, -0.8) {};
%       \node [below] at (-3, -0.8) {$g\circ f(x_0)$};
%
%       \draw [decoration={snake}, decorate] (-3, 0.7) -- (-3, -0.8) node [right, pos=0.5] {$\eta$};
%
%       \draw [->] (-1.6, 1) parabola bend (-0.7, 1.3) (0.2, 1.2);
%       \node [above] at (-0.7, 1.3) {$f$};
%
%       \draw [->] (1, -1.1) parabola bend (-0.3, -1.3) (-1.6, -1);
%       \node [below] at (-0.3, -1.3) {$g$};
%     \end{tikzpicture}
%   \end{center}
%   We have no guarantee that $g\circ f(x_0) = x_0$, but we know that our homotopy $H'$ gives us $\eta = H'(x_0, \ph): x_0 \leadsto g\circ f(x_0)$.
%
%   Applying our previous lemma with $\id_M$ for ``$f$'' and $g \circ f$ for ``$g$'', we get
%   \[
%     \eta_\# \circ (\id_M)_* = (g\circ f)_*
%   \]
%   Using the properties of the $_*$ operation, we get that
%   \[
%     g_*\circ f_* = \eta_\#.
%   \]
%   However, we know that $\eta_\#$ is an isomorphism. So $f_*$ is injective and $g_*$ is surjective.
%
%   Doing it the other way round with $f\circ g$ instead of $g\circ f$, we know that $g_*$ is injective and $f_*$ is surjective. So both of them are isomorphisms.
% \end{proof}


\bdefn[simply-connected space]
  A topological space $M$ is \textbf{simply-connected} if it is path connected and $\pi_1(M, x_0) \cong 1$ for any choice of $x_0 \in M$.
\edefn

\bexample
  Clearly, a point is simply-connected. Hence, any contractible space is simply-connected since it is homotopic to a point. For example, $\R^n$ is simply-connected for any $n$.
\eexample

\bexample
An $n$-sphere $S^n$ ($n>1$) is simply-connected.  If we remove a point $\{\infty\}$ from $S^n$, it is homeomorphic to $\bR^n$ which is contractible. (Therefore, $S^n$ is often written as $S^n=\bR^n\cup \{\infty\}$.) given a loop $\gamma:I \to S^n$, it is easy to find a homotopy $F:[0,1]\times I\to S^n$ such that $F|_{0\times I}=\gamma$ and $F|_{1\times I}$ is a constant map.
\eexample


\begin{figure}[h]\centering
    \includegraphics[width=5cm]{pictures/sphere}
\end{figure}









\subsubsection*{Covering space}

Intuitively, a \textbf{covering space} of $M$ is a pair $(\widetilde{M}, p: \widetilde{M} \to M)$, such that if we take any $x_0 \in M$, there is some neighborhood $U$ of $x$ such that the pre-image of the neighborhood is ``many copies'' of $U$.
\begin{center}
  \begin{tikzpicture}
    \draw plot [smooth cycle] coordinates {(-1.2, -0.7) (0, -0.7) (0.7, -1) (1.3, -0.9) (1.4, 0.4) (0.3, 1) (-1.7, 0.6)};

    \draw [fill=mblue] ellipse (0.6 and 0.3);
    \node at (0.6, 0.3) [anchor = south east] {$U$};

    \draw [densely dashed] (0, 1.5) -- (0, 0);
    \foreach \y in {1.5, 2, 2.5, 3} {
      \node [circ] at (0, \y) {};
      \pgfmathsetmacro\c{100 - \y * 20};
      \draw [densely dashed] (0, \y) -- (0, \y - 0.5);
      \draw [fill=mblue!\c!mred] (0, \y) ellipse (0.6 and 0.3);
      \node [circ] at (0, \y) {};
    }
    \node [circ] at (0, 0) {};
    \node [left] at (0, 0) {$x_0$};

    \draw [->] (2.3, 2.5) node [above] {$\widetilde{M}$}-- +(0, -2.3) node [pos=0.5, right] {$p$} node [below] {$M$};
  \end{tikzpicture}
\end{center}

\bdefn
A pair $(\widetilde{M}, p: \widetilde{M} \to M)$ is a \textbf{covering space} of $M$ if each $x\in M$ has a path-connected open neighborhood $U$ such that the restriction of $p$ to a path-connected component $V_\alpha$ of $p^{-1}(U)$ is $p|_{V_\alpha}:V_\alpha \xrightarrow{\sim} U$.
\edefn


\bexample
  Consider $p_n: S^1 \to S^1$ (for any $n \in \Z\setminus\{0\}$) defined by $z \mapsto z^n$. We can consider this as ``winding'' the circle $n$ times, or as the following covering map:
  \begin{center}
    \begin{tikzpicture}
      \draw [samples=80, domain=0:5] plot [smooth] ({0.6 * sin (360 * \x)}, {1 + 0.3 * cos (360 * \x) + \x / 2});
      \node [circ, mred] at (0, 1.3) {};
      \node [circ, mred] at (0, 3.8) {};

      \draw ellipse (0.6 and 0.3);

      \draw [->] (1.8, 2.7) node [above] {$S_1$}-- +(0, -2.3) node [pos=0.5, right] {$p_n$} node [below] {$S^1$};
    \end{tikzpicture}
  \end{center}
  where we join the two red dots together. The preimage of $1$ would be $n$ copies of $1$.
\eexample


\bdefn[Universal cover]
  A covering map $p: \widetilde{M} \to M$ is a \textbf{universal cover} if $\widetilde{M}$ is simply-connected.
\edefn



  If $p: \widetilde{M} \to M$ is a universal cover, then there is a bijection $\ell: \pi_1(X, x_0) \to p^{-1}(x_0)$. More precisely, the set of covering transformations $\{f:\widetilde{M}\to \widetilde{M}\mid  p\circ f=p\}$ forms a group and it is isomorphic to $\pi_1(M)$.

  \bexample
 $\exp: \bR\to S^1~; x \mapsto \exp(2\pi i x)$ is a universal covering of $S^1$, and the preimage of a point $1$ consist of $\bZ$ so that $\pi_1(S^1)\cong \bZ$.
  \eexample

\bexample
   The real projective space $\bR P^2$ is defined by $S^2/{\sim}$, where we identify every $x\sim -x$, i.e.\ every pair of antipodal point. In fact, the quotient map $p: S^2 \to \bR P^2$ is indeed a universal covering.  \begin{center}
    \begin{tikzpicture}
      \draw circle [radius=2];

      \draw [dashed] (2, 0) arc (0:180:2 and 0.5);
      \draw (-2, 0) arc (180:360:2 and 0.5);

      \draw [dashed, mred] (1, 1.1) node [circ] {} -- (-1, -1.1) node [circ] {};

      \draw [rotate around={-45:(1, 1.1)}, mblue, fill=mblue, fill opacity=0.5] (1, 1.1) ellipse (0.3 and 0.1) node [opacity=1, anchor = south west] {$U$};

      \draw [dashed, rotate around={-45:(-1, -1.1)}, mblue, fill=mblue, fill opacity=0.5] (-1, -1.1) ellipse (0.3 and 0.1) node [opacity=1, anchor = north east] {$U$};
    \end{tikzpicture}
  \end{center}

\eexample
\bexample
In fact, the Cayley graph is a universal covering of $S^1\vee S^1$.
\eexample



  In \S\ref{sec:Mayer-Vietoris}, we have seen that Mayer-Vietoris exact sequence allow us to compute the homology group of a manifold by decomposing it into two parts. Even for fundamental groups, there is a similar theorem, called the van Kampen theorem. In the following, you will find the simple version of the theorem. Due to the time constraint, I have to omit how to use this theorem.
  If you are interested in it, you can read the book of Hatcher \cite[\S1.2]{hatcher2005algebraic}.

  \bthm[van Kampen theorem]
  Let a manifold $M = U_1 \cup U_2$ be a union of open path-connected sets $U_i$, and let $U_1 \cap U_2$ be
  path-connected. We write the normal subgroup $N\triangleleft (\pi_1(U_1)\ast \pi_1(U_2))$ generated by $\{i_{1*}(\omega)i_{2*}(\omega)^{-1}\mid \omega\in \pi_1(U_1 \cap U_2)\}$ where $i_i:U_1 \cap U_2\hookrightarrow U_i$ are the inclusion maps. Then there is an isomorphism $$\pi_1(M)\cong(\pi_1(U_1) * \pi_1(U_2))/N ~.$$
  \ethm




  \bthm[Simple version of van Kampen theorem]
  If $M = U_1 \cup U_2$ with $U_i$ open and path-connected, and $U_1 \cap U_2$
  path-connected and simply-connected, then there is an isomorphism $\ \pi_1(U_1) * \pi_1(U_2) \cong \pi_1(M)$.
  \ethm

  \bexample
   Let us take $M=S^1\vee S^1$ and $U_i=S^1$ so that $U_1\cap U_2$ is the point, which is simply-connected. Therefore, the free group $F(\{a,b\})$ is the free product $F(\{a,b\})\cong \pi_1(S^1) * \pi_1(S^1) \cong \bZ\ast \bZ$ of $\bZ$.
  \eexample



Now we can understand the statement of the famous Poincar\'e ``conjecture'':
\begin{conjecture}
    Every simply-connected, closed $n$-dimensional manifold is homeomorphic to the $n$-sphere $S^n$.
\end{conjecture}
In fact, the Poincar\'e ``conjecture'' is no longer a conjecture.
For $n\ge5$, it was proven by Smale \cite{smale2007generalized}. Then, the case of $n=4$ was proven by Freedman \cite{freedman1982topology}. Perelman has proven the case of $n=3$ \cite{perelman2002entropy} by solving the geometrization conjecture of Thurston.


\subsection{Homotopy groups}

Let $I^n$ ($n\ge1$) denote the unit-cube $I\times \cdots \times I$,
The boundary $\partial I^n$ is the geometrical boundary of $I^n$.
As in the fundamental group, we assume here that we shall be concerned with continuous maps $\alpha : I^n \to M$, which map the boundary $\partial I^n$ to a point $x_0 \in M$. Since the boundary is mapped to a single point $x_0$. The map $\alpha$ is called an \textbf{$n$-loop} at $x_0$.

\bdefn[Homotopy class]
 Let $M$ be a topological space. The set of homotopy classes of $n$-loops ($n \ge 1$) at $x_0 \in M$ is denoted by $\pi_n(M, x_0)$ and called the $n$-th homotopy group at $x_0$. $\pi_n(M, x_0)$ is called the higher homotopy group if $n \ge 2$.
 \edefn

The product $\alpha * \beta$  just defined naturally induces a product of homotopy classes defined by
 $$[\alpha] * [\beta]=[\alpha * \beta] $$
\begin{figure}[h]\centering
\includegraphics[width=8cm]{pictures/homotopy-group}
\end{figure}



 Higher homotopy groups are always Abelian; for any $n$-loops $\alpha$ and $\beta$ at $x_0 \in M$, $[\alpha]$ and $[\beta]$  satisfy
$$[\alpha] * [\beta] = [\beta] * [\alpha].$$


\begin{figure}[h]\centering
\includegraphics[width=\textwidth]{pictures/homotopy-group2}
\end{figure}

\subsubsection*{Global $SU(2)$ anomaly}
There are various applications of homotopy groups to physics, but only one example is presented here.

Let us consider the path integral of a non-Abelian gauge theory with chiral fermion $\psi$ in a representation $R$ of $G$.
\begin{equation}
Z[A_\mu] = \int [{\cal D} \psi][{\cal D}\overline\psi] e^{-\int \overline\psi \gamma^\mu  D_\mu \psi}
\end{equation}
To perform functional integral over gauge fields $A_\mu$ consistently, we need to impose the equivalence
\begin{equation}
Z[A_\mu]= Z[A^g_\mu], \quad A^g_\mu = g^{-1} A_\mu g  + g^{-1} \partial_\mu g.
\end{equation} for any gauge transformation $g:\mathbb{R}^4\to G$. We will learn gauge transformations of non-Abelian gauge theory in \S\ref{sec:bundle}.


When we consider an odd number of chiral fermions in the doublet of $\SU(2)$ gauge group, there is a global anomaly \cite{Witten:1982fp}.
Since continuous deformations of $g$ do not change the phase of the measure $[{\cal D}\psi][{\cal D}\overline \psi]$, we need to consider maps $g:\mathbb{R}^4\to \SU(2)$ up to continuous deformations. Upon one-point compactifiction of $\bR^4$ to $S^4$,
they are characterized by $\pi_4(\SU(2))$, which is known
\begin{equation}
\pi_4(\SU(2))=\pi_4(S^3)=\mathbb{Z}_2~.
\end{equation} Let $g_0:\mathbb{R}^4\to \SU(2)$ be the gauge transformation corresponding to the nontrivial element in this $\bZ_2$.
It is known that the measure $[{\cal D}\psi][{\cal D}\overline \psi]$ gets a minus sign under this gauge transformation.
Thus, one cannot have an odd number of Weyl fermions in the doublet representation of gauge group $\SU(2)$.



\section{Lie groups and Lie algebras}


\bdefn[Lie group]\index{Lie group}
 A \textbf{Lie group} is a manifold $G$ with a group structure such that multiplication $m: G \times G \to G$ and inverse $i: G \to G$ are smooth maps. The \textbf{dimension} of a Lie group $G$ is the dimension of the underlying manifold.
\edefn

 For each $h \in G$, we define the \textbf{left and right translation maps}
 \begin{align}\label{left-right-action}
  L_h: G &\to G; g \mapsto hg~,\cr
  R_h: G &\to G; g \mapsto gh~.
 \end{align}
 These maps are indeed diffeomorphisms because they have smooth inverse $L_{h^{-1}}$ and $R_{h^{-1}}$ respectively.



Some of Lie groups will be given by subsets of the space $M_n(\bF)$ of $n\times n$ matrices where $\bF=\bR$ or $\bC$ specified by certain algebraic equations. For example,
\begin{itemize}
\item General linear group: $\GL(n, \bF)=\{A \in \mathrm{M}_n(\bF)| \det A \neq 0\}$
\item Special linear group: $\SL(n, \bF)=\{A \in \GL(n, \bF)| \det A=1\}$
\item Symplectic group $\mathrm { Sp } ( n , \bF )=\{ A \in \GL(2n, \bF)| A^ { \mathrm { T } } J A = J \ \textrm{where} \ J=\begin{pmatrix} 0&I_n\\-I_n&0\end{pmatrix}\}$
\item Orthogonal group $\mathrm{O}(n,\bF)= \{A \in \GL(n, \bF)| A^{T}A=I\}$
\item Special orthogonal group $\SO(n,\bF)=\{A \in \mathrm{O}(n,\bF)| \det A=1\}$
\item Unitary group $\U(n)=\{A \in \GL(n, \bC)| A^\dagger A=I\}$
\item Special unitary group $\SU(n) =\{A \in \U(n)| \det A=1\}$
\end{itemize}


\bdefn[Lie algebra]\index{Lie algebra}
 A \textbf{Lie algebra} $\mathfrak{g}$ is a vector space (over $\bF=\bR$ or $\bC$) with a \term{bracket}
 \[
  [\ph,\ph] : \mathfrak{g} \times \mathfrak{g} \to \mathfrak{g}
 \]
 satisfying
 \begin{enumerate}
   \item $[\alpha X + \beta Y, Z] = \alpha [X, Z] + \beta [Y, Z]$ for all $X, Y, Z \in \mathfrak{g}$ and $\alpha, \beta \in \bF$ \hfill(bilinear)
  \item $[X, Y] = -[Y, X]$ for all $X, Y \in \mathfrak{g}$ \hfill(antisymmetry)
  \item $[X, [Y, Z]] + [Y, [Z, X]] + [Z, [X, Y]] = 0$ for all $X, Y, Z \in \mathfrak{g}$.\hfill(Jacobi identity\index{Jacobi identity})
 \end{enumerate}
 Note that linearity in the second argument follows from linearity in the first argument and antisymmetry.
\edefn



We now try to get a Lie algebra from a Lie group $G$, by considering ${T}_e(G)$.
 The tangent space of a Lie group $G$ at the identity naturally admits a Lie bracket
 \[
  [\ph, \ph]: T_e G \times T_e G \to T_e G; (X,Y) \mapsto [X,Y]=XY-YX
 \]
 such that
 \[
  \mathfrak{g} = (T_e(G), [\ph, \ph])
 \]
 is a Lie algebra.

\bdefn[Lie algebra of a Lie group]\index{Lie algebra of a Lie group}
 Let $G$ be a Lie group. The \textbf{Lie algebra} of $G$, written $\mathfrak{g}$, is the tangent space $T_e G$ under the natural Lie bracket.
\edefn
The general convention is that if the name of a Lie group is denoted by capital letters, then the corresponding Lie algebra is the same name with fraktur font. For example, the Lie group of $\SL(n,\bC)$ is $\fraksl(n,\bC)$. The semisimple Lie algebras over $\bC$ have been classified by Wilhelm Killing and \'Elie Cartan around 1890. The classification assigns the types
$$\begin{aligned} A_{n} &=\mathfrak{s l}(n+1, \mathbb{C}) \\ B_{n} &=\mathfrak{s o}(2 n+1, \mathbb{C}) \\ C_{n} &=\mathfrak{s p}(n, \mathbb{C}) \\ D_{n} &=\mathfrak{s} \mathfrak{o}(2 n, \mathbb{C}) \end{aligned}~.$$
Contrasting with the classical Lie algebras listed above are the exceptional Lie algebras, $E_6$, $E_7$, $E_8$, $F_4$, and $G_2$, which share their abstract properties.
For the classification of the simple Lie algebras, we refer the reader to \cite{kirillov2008introduction}.



Given a finite-dimensional Lie algebra, we can pick a basis $B$ for $\mathfrak{g}$.
\be\label{Lie-basis}
 B = \{T_a: a = 1, \cdots, \dim \mathfrak{g}\}.
\ee
Then any $X \in \mathfrak{g}$ can be written as
\[
 X = X^a T_a = \sum_{a = 1}^n X^a T_a,
\]
where $X^a \in \bF$.

By linearity, the bracket of elements $X, Y \in \mathfrak{g}$ can be computed via
\[
 [X, Y] = X^a Y^b [T_a, T_b].
\]
In other words, the whole structure of the Lie algebra can be given by the bracket of basis vectors. We know that $[T_a, T_b]$ is again an element of $\mathfrak{g}$. So we can write
\[
 [T_a, T_b] = f_{ab}{}^c T_c,
\]
where $f_{ab}{}^c\in \bF$ are called the \textbf{structure constants}.
By the antisymmetry of the bracket, we know
 \[
  f_{ba}{}^c = -f_{ab}{}^c.
 \]
The Jacobi identity amounts to
 \[
  f_{ab}{}^c f_{cd}{}^e + f_{da}{}^c f_{cb}{}^e + f_{bd}{}^c f_{ca}{}^e = 0.
 \]




\bexample
 Take $G = \SO(3,\bR)$. Then $\so(3,\bR)$ is the space of $3 \times 3$ real anti-symmetric matrices, which one can manually check are generated by
 \[
  {T}_1 =
  \begin{pmatrix}
   0 & 0 & 0\\
   0 & 0 & -1\\
   0 & 1 & 0
  \end{pmatrix},\quad
  {T}_2 =
  \begin{pmatrix}
   0 & 0 & 1\\
   0 & 0 & 0\\
   -1 & 0 & 0
  \end{pmatrix},\quad
  {T}_3 =
  \begin{pmatrix}
   0 & -1 & 0\\
   1 & 0 & 0\\
   0 & 0 & 0
  \end{pmatrix}
 \]
 We then have
 \[
  ({T}_a)_{bc} = -\varepsilon_{abc}.
 \]
 Then the structure constants are $f_{ab}{}^{c} = \varepsilon_{abc}$.
\eexample



Given a vector $X\in \mathfrak{g} $ in the tangent space of the identity $e$, one can generate the vector field by push-forward by the left translation $L_g$. Let us denote the corresponding vector field by $X$ too. Since $(L_g)_*X=X$, it is called a \term{left-invariant vector field}. The flow generated by the vector field $X$ is called \textbf{exponential map}, which can be expressed as a matrix
 \[
  \exp(tX) = \sum_{\ell = 0}^\infty \frac{1}{\ell!} (tX)^\ell~.
 \]
Therefore, for any matrix Lie group $G$, the exponential map defines a map $\exp:\mathfrak{g} \to G$.


\subsubsection*{Maurer-Cartan form}


Given a Lie algebra $\frakg$, it is also natural to think about its dual space $\frakg^*$. This can be identified with the set of all left invariant 1-forms $\omega$ on $G$ such that $L_g^*\omega =\omega$. Note that $\omega(X)$, $\omega(Y)$ are constant over $G$ for $\omega\in \frakg^*$ and $X,Y\in \frakg$. Therefore, we have $Y(\omega(X))=0=X(\omega(Y))$ so that \eqref{d} reduces to
$$
d\omega(X,Y)=-\frac12\omega([X,Y])~.
$$
Therefore, if we take the basis $\omega^1,\cdots,\omega^{\dim \frakg}$ dual to \eqref{Lie-basis}, we can write it as
$$
d\omega^c=-\frac12 \sum_{a,b} f_{ab}{}^{c}~\omega^a\wedge \omega^b~.
$$
Moreover, let $\omega\in \Omega^1(G;\frakg)$ be $\frakg$-valued 1-form on $G$ such that $\omega(X)=X$ for $X\in \frakg$. Using the above basis, it is describe as
\be\label{MC-form}
\omega = \sum_a \omega^a T_a~,
\ee
which is called \term{Maurer-Cartan form}. Then, the equation above has the following form
$$
d\omega = -\frac12 [\omega,\omega]~,
$$
which is called the \term{Maurer-Cartan equation}.


\subsubsection*{Adjoint action}
The \textbf{adjoint action} $G\ \rotatebox[origin=c]{-90}{$\circlearrowright$}\ G$ is defined by
\begin{equation}
\mathrm{Ad}: G \times G \rightarrow G; \quad(g, h) \mapsto g h g^{-1}~.
\end{equation}
The \textbf{adjoint action} $G \ \rotatebox[origin=c]{-90}{$\circlearrowright$}\ \frakg$ is defined by
\begin{equation}
\text { Ad: } G \times \mathfrak{g} \rightarrow \mathfrak{g};\quad(g, X) \mapsto g X g^{-1}~.
\end{equation}
This has the dual representation $G \ \rotatebox[origin=c]{-90}{$\circlearrowright$}\ \frakg^*$ called the \textbf{coadjoint action}
\begin{equation}\label{coadjoint}
\mathrm{Ad}^{*}: G \times \mathfrak{g}^{*} \rightarrow \mathfrak{g}^{*}
\end{equation}
defined by
\begin{equation}
(\operatorname{Ad}_{g}^{*} \omega)(X)=\omega (\operatorname{Ad}_{g^{-1}} X)~.
\end{equation}
These representations have infinitesimal versions (derivations)
\begin{align}
&\text { ad: } \mathfrak{g} \times \mathfrak{g} \rightarrow \mathfrak{g}, \quad(X, Y) \mapsto \operatorname{ad}_{X}(Y)=[X, Y]\cr
&\mathrm{ad}^{*}: \mathfrak{g} \times \mathfrak{g}^{*} \rightarrow \mathfrak{g}^{*} \qquad \mathrm{ad}_{Y}^{*}(\omega)(X)=\omega(-\operatorname{ad}_{Y} X)~, \quad X, Y \in \mathfrak{g}, \quad \omega \in \mathfrak{g}^{*}
\end{align}


\subsubsection*{$G$-action}
In general, a left $G$-action on a space $M$ is defined by $$G \times M \rightarrow M; \ (g, p) \mapsto g p$$ such that $g\cdot (h \cdot p)=(g h) \cdot p$, $e\cdot  p=p$.
A right $G$-action $$M \times G \rightarrow M;\ (p, g) \mapsto p g$$  is similarly defined as $ (ph)g = p(hg)$. When $M=G$, \eqref{left-right-action} define the left and right $G$-action on itself.

For a closed subgroup $H\subset G$, the quotient space $G / H$ becomes a manifold called the \textbf{homogenous space} and it receives the left $G$ action. The projection $\pi: G \rightarrow$ $G / H$ is a smooth map between manifolds. Moreover, the tangent space at $H$ is identified with $$T_{H}(G / H)\cong\mathfrak{g} / \mathfrak{h}~.$$

When a Lie group $G$ acts on a manifold $M$, $O_{G}(p):=G p=$ $\{g p \mid g \in G\} \subset M$ is  called the \textbf{$G$-orbit} of $p \in M$. If $O_{G}(p)=M$, the $G$-action on $M$ is \textbf{transitive}. A subgroup $G_{p}:=\{g \in G \mid g p=p\}$ is indeed a closed subgroup called the \textbf{stabilizer (isotropy) subgroup} of $p$. As a result, we can a diffeomorphism $$G / G_{p} \simeq M~.$$


\bexample
The $\SO(n+1)$-orbit of a unit vector $\mathbf{e}_1={ }^{t}(1,0, \ldots, 0) \in \mathbb{R}^{n+1}$ is an $n$-sphere $S^n$ and its stabilizer subgroup is $O_{\SO(n+1)}(\mathbf{e}_1)=\SO(n)$. Therefore,
$$\OO(n+1) /\OO(n) \simeq \SO(n+1) / \SO(n) \simeq S^{n}$$
Note that $\OO(n)=\SO(n) \sqcup \operatorname{diag}(-1,1, \ldots, 1) \SO(n)$.
\eexample


\bexample\label{projective-coset}
The $n$-dimensional real projective space is $\RP^{n} \cong S^n/\bZ_2$, and $\OO(1)\cong \bZ_2$.
$$
\RP^{n} \simeq \OO(n+1) /(\OO(1) \times \OO(n))
$$
In fact, $\OO(1) \times \OO(n)$ is the stabilizer subgroup of a line $\ell\in \RP^n$. Similarly,
the $n$-dimensional complex projective space is
$$
\CP^{n} \simeq \U(n+1) /(\U(1) \times \U(n))
$$
\eexample



\end{document}
