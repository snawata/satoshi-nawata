\documentclass[geometry-lectures-21.tex]{subfiles}

\begin{document}




\subsection{Principal $G$-bundles}

We have studied vector bundles where a fiber is a vector space. We can further generalize it to a \textbf{fiber bundle} where a fiber is a general manifold $F$ and a transition function is given by a diffeomorphism of $F$. Among fiber bundles, principal $G$-bundles play an important role in physics.




\bdefn[Principal $G$-bundle]\index{principal $G$-bundle}
 Let $G$ be a Lie group, and $M$ a manifold. A \term{principal $G$-bundle} is a smooth manifold $P$ with a projection $\pi: P \to M$ such that a fiber is $\pi^{-1}(\{x\}) \cong G$ for each $x \in M$.
 More precisely, we are given an open cover $\{U_\alpha\}$ of $M$ and diffeomorphisms
 \[
  \begin{tikzcd}
     \pi^{-1}(U) \ar[r, "t"] \ar[d, "\pi"] & U \times G \ar[dl, "p_1"]\\
     U
    \end{tikzcd} \]
such that the transition functions
 \[
  t_\alpha \circ t_\beta^{-1}: (U_\alpha \cap U_\beta) \times G \to (U_\alpha \cap U_\beta) \times G
 \]
 is of the form
 \[
  (x, g) \mapsto (x, t_{\alpha\beta}(x) \cdot g)
 \]
 for some $t_{\alpha\beta}: U_\alpha \cap U_\beta \to G$ where $G$ is called the \term{structure group}.
\edefn

For $g\in G$ we can define the right action $R_g$ on the total space $P$
$$
R_gP \to P; u\mapsto ug
$$
where each fiber onto itself.

Given a principle $G$-bundle and a representation $\rho:G\to \mathfrak{gl}(V)$ for a vector space $V$, we can construct \term{associated vector bundle}
$$
E=P\times_\rho V
$$
as follow. Let us consider the direct product $P\times V$ and $G$ action
$$
(u,y)\mapsto (us,\rho(s)^{-1}y) \qquad \textrm{for}  \quad s\in G
$$
We define the associated bundle as the quotient space $P\times_\rho V:=(P\times V)/G$. Conversely, given a $G$-bundle $E\to M$ with fiber $V$, there is a canonical way of producing a principal $G$-bundle by using transition functions.


\subsection{Connections and curvatures}
In Riemannian geometry \S\ref{sec:Riemann}, we have learned Levi-Civita connections and Riemann curvature. Even in vector bundles and principal $G$-bundles, we can introduce connections and curvatures.

\bdefn[Connection]\index{connection}
 A \term{connection} \index{$\nabla$} in a vector bundle $\pi:E\to M$ is a bilinear map
 $$\nabla:\mathfrak{X}(M)\times \Gamma(E) \to \Gamma(E);(X,s)\mapsto \nabla_X s$$
 satisfying
 \begin{enumerate}
  \item $ \nabla_{fX}(s) = f\nabla_{X}(s) $

  \item Leibnitz property: $ \nabla_X (fs) = (X f) s + f (\nabla_X s)$
 \end{enumerate}
 for all $s \in \Gamma(E)$ and $f \in C^\infty(M)$.
\edefn

If a vector bundle $E$ has a metric $g$, a connection $\nabla$ is \term{compatible} with the metric if
$$
X g(s_1,s_2)= g(\nabla_X s_1,s_2)+g( s_1,\nabla_X s_2)
$$
for any $X\in \mathfrak{X}(M)$ and $s_1,s_2 \in \Gamma(E)$.

\bdefn[Curvature]\index{Curvature}
 A \term{curvature} \index{$\nabla$} in a vector bundle $\pi:E\to M$ is a trilinear map
 $$F:\mathfrak{X}(M)\times \mathfrak{X}(M)\times \Gamma(E) \to \Gamma(E);(X,Y,s)\mapsto F(X,Y) s$$
 defined by
 $$
 F(X,Y)s=\frac12\left[ \nabla_X\nabla_Y-\nabla_Y\nabla_X-\nabla_{[X,Y]}\right]s~.
 $$
 It has the following properties
  \begin{enumerate}
  \item $ F(X,Y)s =-F(Y,X)s $
  \item $F(fX,gY)(hs) = fghF(X,Y)s $ for $f,g,h\in C^\infty(M)$
 \end{enumerate}
 for all $s \in \Gamma(E)$.
\edefn


For a certain open subset of $M$, we can take a frame $s_1,\cdots,s_r \in \Gamma(\pi^{-1}(U))$. For any vector field $X$ on $U$, the connection can be locally written as
$$
\nabla_X s_j=\sum_{i=1}^rA^i{}_j(X)s_i
$$
where $A^i{}_j\in \Omega^1(U,\mathfrak{gl}(r,\bR))$ (1-form on $U$ taking its value on $\mathfrak{gl}(r,\bR)$) is called \term{connection form}. We now look at the curvature $R$ from differential forms.
$$
F(X,Y)s_j=\sum_{i=1}^r F^i{}_j (X,Y)s_i
$$
where $F^i{}_j\in \Omega^2(U,\mathfrak{gl}(r,\bR))$ (2-form on $U$ taking its value on $\mathfrak{gl}(r,\bR)$) is called \term{curvature form}. They are related by the following equation:
$$
F=dA+A\wedge A~.
$$
The curvature form satisfies the Bianchi identity
\be\label{Bianchi}
dF-F\wedge A+A\wedge F=0~.
\ee

More explicitly, in physics, we write a section $s=\sum_{j=1}^rv^j(x)s_j$ on $U$ so that
$$
\nabla_{\frac{\partial}{\partial x^\mu}} s=\sum_{i=1}^r\left[\frac{\partial}{\partial x^\mu} v^i(x) + (A_{\mu}){}^i{}_j v^j(x) \right]s_i
$$
Also, the curvature can be written in terms of local coordinates
$$
F\left(\frac{\partial}{\partial x^\mu},\frac{\partial}{\partial x^\nu}\right)=F_{\mu\nu}=\frac{\partial A_\nu}{\partial x^\mu}-\frac{\partial A_\mu}{\partial x^\nu} +[ A_\mu, A_\nu]~.
$$
In the case of Maxwell $\U(1)$ theory, the last term vanishes because it is a commutative group.



It is useful to know how the connection transforms under a change of local trivialization. Given a transition function $g_{\alpha\beta}: U_\alpha \cap U_\beta \to GL(r,\bR)$, the gauge fields on $U_\alpha$ and $U_\beta$ are related by
\be\label{gauge-trans}
 A_\beta = g_{\alpha \beta}^{-1} A_\alpha g_{\alpha\beta} + g_{\alpha\beta}^{-1} d (g_{\alpha\beta})~,
\ee
and the curvature forms are related by
\be\label{gauge-trans2}
F_\beta=g_{\alpha \beta}^{-1} F_\alpha g_{\alpha\beta}
\ee

In a similar manner, one can construct connections and curvatures for $G$-bundle where we have to replace the first term of \eqref{gauge-trans} by the adjoint representation of $G$ on $\mathfrak{g}$, and the second of \eqref{gauge-trans} by the Maurer--Cartan form \eqref{MC-form}.

In particular, for the Maxwell theory, the gauge transformation can be written as $g_{\alpha\beta}=e^{i\lambda_{\alpha\beta}(x)}$ so that
$$
A_\beta=A_\alpha + i d\lambda_{\alpha\beta}
$$
and the curvature form stays invariant.



\subsubsection*{Parallel transport and holonomy group}
Given a connection $\nabla$ on a vector bundle $E$, on can define \term{horizontal} directions in the space $\Gamma(E)$ of sections. A section $s\in \Gamma(E)$ is \term{parallel} along a path $\gamma:I\to (M)$
$$
\nabla_{\dot \gamma(t)} s=0 \quad \textrm{for} \quad t\in I~.
$$
In terms of local coordinates, it can be written as
$$
\frac{ds_i}{dt}+\sum_{j=1}^r (A_\mu)^j{}_i \frac{dx^\mu}{dt} s_j=0~.
$$
A theorem of ordinary differential equations tells us that given an initial data $s(t=0)\in E_{\gamma(0)}$, one can do parallel transform along $\gamma(t)$ so that we have a map
$$
E_{\gamma(0)} \ni s(t=0) \to s(t=1)\in E_{\gamma(1)}~.
$$
In particular, if we consider a curve $p=\gamma(0)=\gamma(1)$, we obtain a map $\tau_\gamma:E_p\to E_p$. For given two curves $\gamma_1$ and $\gamma_2$, we can have a multiplication
$$
\tau_{\gamma_1\circ \gamma_2}=\tau_{\gamma_1}\circ \tau_{\gamma_2}
$$
and the inverse is defined by
$$
\tau_{\gamma^{-1}}=\tau_{\gamma}^{-1}~,
$$
so that it forms a group called \term{holonomy group}. In the case of $\U(1)$, this is the origin of the \term{Aharonov-Bohm effect}.

\begin{figure}[ht]\centering
\includegraphics[width=10cm]{pictures/abm}
\end{figure}
\begin{figure}[ht]\centering
\includegraphics[width=7cm]{pictures/Aharonov-Bohm}
\caption{The Aharonov-Bohm solenoid effect takes place when the wave function of a charged particle passing around a long solenoid experiences a phase shift as a result of the enclosed magnetic field, despite the magnetic field being negligible in particle trajectories.}
\end{figure}

\subsubsection*{Levi-Civita connections}
We can consider the tangent bundle $TM$ as a vector bundle and its metric $g$ is indeed a Riemannian metric. As we have seen in \S\ref{sec:Riemann}, there is the unique natural connection called \term{Levi-Civita connections} in $TM$. Let $X_1, \cdots, X_n$ be an orthonormal frame vector field on an open set $U\subset M$ and their dual $e^1,\cdots,e^n\in \Omega^1(U)$. For the Levi-Civita connection $\nabla$, we can write
\begin{align}
\nabla_{X_j}X_i&=\sum_k\mathbf{\Gamma}^k_{ij}X_k~,\cr
R(X_i,X_j)X_k&=\sum_{l}\mathbf{R}^{l}{}_{kij}X_l~.\nonumber
\end{align}

We can define connection one-form and curvature two-form taking their values on $\mathfrak{so}(n,\bR)$:
\be\label{Riemann-2-form}
\omega^k{}_j=\sum_k\mathbf{\Gamma}^k_{ij} e^i~, \quad \Omega^l_i=\sum_{l}\mathbf{R}^{l}{}_{ijk} e^j\wedge e^k~.
\ee
They satisfy the following conditions
 \begin{align}
de^i &=-\sum_j \omega^{i}{}_j\wedge e^j~,\cr
\Omega^i{}_j&=d\omega^i{}_j+\omega^i{}_k\wedge \omega^k{}_j~.\nonumber
 \end{align}








\subsubsection*{Ehresmann connections}
Similarly, one can construct the theory of connections on principal $G$-bundles, which are called \term{Ehresmann connections}. In principle, an Ehresmann connection determines the horizontal direction of a principal $G$-bundle.



\bdefn[Ehresmann connection]\index{Ehresmann}
 A \term{Ehresmann connection} $A$ on a principle $G$-bundle $\pi:P\to M$ is a one-form taking its value on $\frakg$, which satisfies the following conditions:
  \begin{enumerate}
  \item Given $X\in \frakg$, there is the corresponding vector field $\overline X$ on $P$
  $$
  \overline X_u=\frac{d}{dt}(u\cdot \exp t X)\Big|_u \quad \textrm{for} \quad u\in P~.
  $$
 Then, $A$ is subject to
 $$
 A(\overline X)=X \in \frakg~.
 $$
  \item For ${}^\forall g\in G$, under the right action $R_g$, it behaves as
  $$ R^\ast_g(A)=\textrm{ad}(g^{-1}) A~.$$
  In other words,
  $$ A_{ug}(R_g(Y))=\textrm{ad}(g^{-1})(A_u(Y)) ~, \quad \textrm{for} \quad Y\in \frakg ~.  $$
 \end{enumerate}
\edefn

For an open set $U_\alpha\subset M$, there is a local trivialization $\psi_{U_\alpha}: \pi^{-1}(U_\alpha)\to U_\alpha\times G$. Then, we have a section on $\pi^{-1}(U_\alpha)$
$$ \sigma_{U_\alpha}:U_\alpha\to P; x\mapsto \psi_{U_\alpha}^{-1}(x,e)~.$$
Then, if we define $A_\alpha=\sigma^*_{U_\alpha}(A)$, we have cocycle condition
$$
 A_\beta = g_{\alpha \beta}^{-1} A_\alpha g_{\alpha\beta} + g_{\alpha\beta}^{-1} d (g_{\alpha\beta})~, \quad \textrm{for} \quad U_\alpha\cap U_\beta~,
$$
where $g_{\alpha\beta}$ is the transition function on $U_\alpha\cap U_\beta$.

We denote the space of Ehresmann connections on a principal $G$-bundle $P$ by $\mathscr{A}_P$. The space $\mathscr{G}_P$ of gauge transformations is the section $\Gamma(M,G_P)$ of the bundle $G_P:=P\times_{Ad}G$. A gauge transformation $g\in \mathscr{G}_P$ acts on the space of connection $\mathscr{A}_P$ via
$$
\mathscr{G}_P\times \mathscr{A}_P\to \mathscr{A}_P; (g,A)\mapsto g^*(A)=\{g^*(A)_U,g^*(A)_V, \cdots\}
$$
where
$$
g^*(A)_U=g_U^{-1}d(g_U)+g_U^{-1}Ag_U~.
$$
If $A'=g^*(A)$ for $A,A'\in \mathscr{A}_P$, they are physically identical, and we denote the space of physically different connections by
$$
\mathscr{B}_P:=\mathscr{A}_P/\mathscr{G}_P~.
$$






\subsection{Yang-Mills theory}
Now we can describe non-Abelian gauge theory called \textbf{Yang-Mills theory} \cite{yang1954conservation}.
Let us consider a principal $G$-bundle or its associated $G$-bundle. In addition, let $A$ be a connection on it and $F$ be its curvature.
The classical \term{Yang-Mills action} can be written as
\begin{align}
 S_{YM}[A] &= \frac{1}{2g^2_{YM}} \int_M \Tr F\wedge \ast F\cr
 &= \frac{1}{2g^2_{YM}} \int_M \Tr \left(F_{\mu\nu} F^{\mu\nu}\right) \;\sqrt{g} d^d x,
\end{align}
where the curvature is 2-form taking its value on the Lie algebra $\mathfrak{g}$. The action is almost the same as that of the Maxwell theory. However, due to the gauge transformation \eqref{gauge-trans2} of the curvature form,
we need to take $\Tr$ over Lie algebra $\mathfrak{g}$ in order for the action to be gauge-invariant.
The parameter $g_{YM}$ is called the Yang-Mills coupling constant. For the flat space, we have $\sqrt{g} = 1$ so that we will drop this term.

For example, if $G = \SU(N)$, we can choose the basis of the Lie algebra $\mathfrak{su}(N)$
\[
 \Tr(T_a T_b) = \frac{1}{2} \delta_{ab},
\]
and on a local $U \subseteq M$, we have
\[
 S_{YM}[A] = \frac{1}{4 g^2_{YM}} \int F_{\mu\nu}^a F^{b, \mu\nu} \delta_{ab} \; d^d x,
\]
with $F_{\mu\nu}=\sum_a F_{\mu\nu}^a T_a$ and
\[
 F_{\mu\nu}^a = \partial_\mu A^a_\nu - \partial_\nu A_\mu^a + f_{bc}{}^a A_\mu^b A_\nu^c~.
\]
Thus, Yang-Mills theory is the natural generalization of Maxwell theory to the non-Abelian case.



At the level of the classical field equations, if we vary our connection by $A_\mu \mapsto A_\mu + \delta a_\mu$, where $\delta a$ is a matrix-valued $1$-form, then we have
\[
 \delta F_{\mu\nu} = \partial_{[\mu} \delta a_{\nu]} + [A_\mu, \delta a_\nu].
\]
The equation of motion can be obtained by taking the variation of the Yang-Mills action
\[
\delta S_{YM}[A] = \frac{1}{2g_{YM}^2} \int \Tr(\delta F_{\mu\nu}, F^{\mu\nu}) \;d^d x = \frac{1}{2g_{YM}^2} \int \Tr(\nabla_\mu \delta a_\nu, F^{\mu\nu})\;d^d x = 0.
\]
Therefore, the \term{Yang--Mills equation} is
\[
 \nabla^\mu F_{\mu\nu}: = \partial^\mu F_{\mu\nu} + [A^\mu, F_{\mu\nu}] = 0~,
\]
or we can write it without coordinates
$$
\delta_A F:=\ast d_A\ast F=\ast (d+A)\ast F=0~.
$$
Recall we also have the Bianchi identity \eqref{Bianchi} which can be expressed in terms of a local coordinate
\[
 \nabla_\mu F_{\nu\lambda} + \nabla_\nu F_{\lambda\mu} + \nabla_\lambda F_{\mu\nu} = 0~.
\]
\term{Unlike} Maxwell's equations, these are non-linear PDE's for $A$. We no longer have the principle of superposition, which is similar to Einstein's equation \eqref{Einstein-eq}.

The Yang-Mills path integral is expressed by
$$
Z_{YM}=\int_{\mathscr{A}/\mathscr{G}} DA~ \exp({iS_{YM}[A]})~.
$$
You can try to solve one of the seven Millennium Prize Problems: quantum Yang-Mills theory exists on $\mathbb {R} ^{4}$ and has a mass gap \cite{jaffe2006quantum}.



\end{document}
