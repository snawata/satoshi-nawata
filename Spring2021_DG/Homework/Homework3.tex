\documentclass[12pt,a4paper]{article}


\usepackage{macros}

\begin{document}\thispagestyle{empty}

\centerline{\Large \bf Homework 3: Due at class on March 25}




%
%
% \vspace{.5cm}
% \noindent \textbf{1}.  For a positive integer $d$, we define a map $f^{(d)}:\bC\to \bC; ~z\mapsto z^d$. Let $z=x+iy$ \((x, y \in \mathbb{R}),\) and consider \(f^{(d)}\) as a function of \(x\) and \(y .\) Compute the Jacobian matrix of \(f^{(d)}\).


\vspace{.5cm}

\noindent \textbf{1}.
Let $\varphi_t:M\to M$ be the flow generated by a vector field $X\in \mathfrak{X}(M)$. Then, show that for another vector field $Y\in \mathfrak{X}(M)$
$$
[X,Y]=\lim_{t\to 0}\frac{(\varphi_{-t})_*Y-Y}{t}
$$

\vspace{.5cm}

\noindent \textbf{2}.  Given a vector field $X\in \mathfrak{X}(M)$, the Lie derivative
$$
L_X:\Omega^k(M)\to \Omega^k(M)
$$
can be defined by
$$
L_X(\omega)=\lim_{t\to0}\frac{\varphi_t^*\omega-\omega}{t}
$$
where  $\varphi_t:M\to M$ be the flow  as above. Show that it satisfies
$$
L_X\omega(X_1,\cdots,X_k)=X\omega(X_1,\cdots,X_k)-\sum_{I=1}^k\omega(X_1,\cdots, [X,X_i],\cdots,X_k)
$$
for $X_1,\cdots,X_k\in \mathfrak{X}(M)$.


\vspace{.5cm}
\noindent \textbf{3}. Define an $n$-form $\omega$ on the space $\bR^{n+1}\backslash\{0\}$, obtained from $\bR^{n+1}$ by removing the origin, by
$$
\omega=\frac{1}{|x|^{n+1}}\sum_{i=0}^{n}(-1)^{i}x^idx^0\wedge \cdots \wedge \widehat{dx^i} \wedge \cdots \wedge dx^{n}~.
$$
Prove that $d\omega=0$.
Let $\iota:S^n\hookrightarrow \bR^{n+1}$ be the unit sphere in $\bR^{n+1}$. Show that $\iota^* \omega$ is the generator of $n$-the de Rham cohomology $H^n_{dR}(S^n)$ of the $n$-sphere. In fact, the de Rham cohomology of $S^n$ is
$$
H^k_{dR}(S^n)=\left\{\begin{array}{l} \bR \quad k=0,~n \\ 0 ~ \quad  \textrm{otherwise}\end{array}\right.~.
$$
In the case of $n=2$, evaluate the integral
$$
\int_{S^2}\iota^*\omega~.
$$



 \vspace{.5cm}
\noindent \textbf{4}. Let $T^2=S^1\times S^1$ be the torus. Using the formula for de Rham cohomology  of a product space,
$$
H_{dR}^{k}(M\times N)\cong \bigoplus_{k=p+q} H_{dR}^p(M)\otimes H_{dR}^q(N)~,
$$
find de Rham cohomology  $H_{dR}^*(T^2)$. Find the Poincare dual of each generator of  $H_{dR}^*(T^2)$.


 \vspace{.5cm}
\noindent \textbf{5}.
The Maxwell equations are written as
\begin{align}
\mathbf {\nabla }\cdot \mathbf {E}= \rho ~, &  \qquad    \mathbf {\nabla }\times \mathbf {B}= \mathbf {J}+\dfrac {\partial \mathbf {E}}{\partial t}  ~,\cr
   \mathbf {\nabla }\cdot \mathbf {B}=0 ~, &\qquad \mathbf {\nabla }\times \mathbf {E}=-\dfrac {\partial \mathbf {B}}{\partial t}~.\nonumber
\end{align}
Let us write the gauge potential
$$
A=A_{\mu }dx^{\mu }=\phi dt+A_{1}dx^{1}+A_{2}dx^{2}+A_{3}dx^{3}
$$
and
the current
$$
J=J_{\mu }dx^{\mu }=\rho dt+J_{1}dx^{1}+J_{2}dx^{2}+J_{3}dx^{3}~.
$$
Then, the field strength can be written as $F=dA$. Show that the Maxwell equations are equivalent to the following equations
$$
dF=0 ~,\qquad \delta F= - j~.
$$
Find the equation of motion for the following action
$$
S=-\frac{1}{2}\int F\wedge \ast F - \int  A\wedge \ast J~.
$$
 Discuss this for both a positive definite metric and a Lorentzian signature metrics.






\vspace{.5cm}

\noindent \textbf{6}. (This is a bonus problem with extra 3 points which is NOT mandatory.) Find the preimage of a smooth map $f:\textrm{SO}(3)\to \textrm{SO}(3);g\mapsto g^2$ and the rank of $f_\ast$ at every point of $\textrm{SO}(3)$. [Hint: An element $g\neq 1\in \textrm{SO}(3)$ is an rotation along a certain axis in $\bR^3$.]



\end{document}
