\documentclass[12pt,a4paper]{article}


\usepackage{macros}


\begin{document}\thispagestyle{empty}

\centerline{\Large \bf Homework 10: Due at class on May 16}

 \vspace{.5cm}
\noindent 1. Suppose that we have a connection $\nabla$ on a rank-$r$ vector bundle $\pi:E\to M$. Suppose that we have local trivializations of $E$ over two open sets $U_\alpha$ and $U_\beta$ in $M$. On these open sets, the connection can be written in terms of one-form $A_\alpha$ and $A_\beta$ taking value on $\mathfrak{gl}(r,\bR)$. Given a transition function $g_{\alpha\beta}: U_\alpha \cap U_\beta \to \GL(r,\bR)$, show that the connections are related by
$$
  A_\beta = g_{\alpha \beta}^{-1} A_\alpha g_{\alpha\beta} + g_{\alpha\beta}^{-1} d (g_{\alpha\beta})~,
$$
and the curvature forms are related by
$$
F_\beta=g_{\alpha \beta}^{-1} F_\alpha g_{\alpha\beta}
$$
on $U_\alpha\cap U_\beta$.


 \vspace{.5cm}
\noindent 2. Prove that the Bianchi identity
$$
dF+A\wedge F -F\wedge A=0~.
$$

\vspace{.5cm}
\noindent 3. Show that $\U(n)$ is a direct product $\SU(n)\times \U(1)$ as a manifold, but not as a group.

 \vspace{.5cm}
\noindent 4. \textbf{Hopf fibration and Dirac monopole}

We define $S^{2n+1}=\{(z_0,\cdots,z_n)\in  \bC^{n+1}| \sum_{i=0}^n|z_i|^2=1 \}$ and let
the complex projective space $\bC P^n$  be the quotient space $S^{2n+1}/\sim$ where we define an equivalent relation $(z_0,\cdots,z_n)\sim \lambda (z_0,\cdots,z_n)$ with $\lambda \in \U(1)$. We denote a point of $\bC P^n$ by $[z_0;\cdots;z_n]$.

When $n=1$, we can consider that $S^3$ is the principal $\U(1)$-bundle over $\bC P^1$
\begin{equation}\begin{tikzcd}
S^1 \arrow[r]
& S^3 \arrow[d, "\pi" ] \\
&\bC P^1
\end{tikzcd}\qquad.\nonumber
\end{equation}
 First of all, show that $\bC P^1$  can be identified with $S^2\subset \bR^3$ by a map
$$
[z_0;z_1]\mapsto (z_0 \overline z_1 + \overline z_0 z_1 , -i (z_0 \overline z_1 - \overline z_0 z_1) , |z_0 |^2 - |z_1 |^2 )~.
$$
Let us define trivializations of this bundle over $U_N=S^2\backslash \{\textrm{north  pole}\}$ and $U_S=S^2\backslash \{\textrm{south pole}\}$ as
\begin{align}\nonumber
\pi^{-1}(U_N)\to U_N\times S^1&;(z_0;z_1)\mapsto  \left( [z_0;z_1],\frac{z_0}{|z_0|}\right) \cr
\pi^{-1}(U_S)\to U_S\times S^1&;(z_0;z_1)\mapsto \left( [z_0;z_1],\frac{z_1}{|z_1|}\right) ~.
\end{align}
Find the transition function between these trivializations. Let us consider embedding
\begin{align}\nonumber
f_N:U_N\hookrightarrow \pi^{-1}(U_N)&: (\theta,\phi) \mapsto  \left( \cos \frac\theta2,\sin \frac\theta2 e^{-i\phi}\right) \cr
f_S:U_S\hookrightarrow \pi^{-1}(U_S)&: (\theta,\phi) \mapsto  \left( \cos \frac\theta2 e^{i\phi},\sin \frac\theta2 \right) ~,
\end{align}
where $(\theta,\phi)$ is the standard polar coordinate of $S^2$. Defining a connection $$A=i~ \textrm{Im}(\overline z_0dz_0 +\overline z_1dz_1 )$$ on $S^3\subset\bC^2$, show that
\begin{align}
f^*_N A&=-\frac i2 (1-\cos \theta)d\phi   \quad \textrm{on} \quad U_N\cr
f^*_S A&=\frac i2 (1+\cos \theta)d\phi  \qquad \textrm{on} \quad U_S ~. \nonumber
\end{align}
Show that they are related by gauge transformations and its curvature $F$ is well-defined on $S^2$. This connection provides \textbf{Dirac magnetic monopole}
$$
\frac{i}{2\pi }\int_{S^2} F=1~.
$$


 \vspace{.5cm}
\noindent 5.
Let us consider the $\SO(2)$ subgroup of $\SO(3)$ as
$$
\SO(3)\supset\SO(2)=\left(
\begin{array}{ccc}
\cos\theta&-\sin \theta&0\\
\sin \theta&\cos\theta&0\\
0&0&1\\
\end{array}
\right) ~.
$$
Show that $S^2\cong \SO(3)/\SO(2)$ and $\SO(3)$ can be considered as $S^1$ bundle over $S^2$. Construct local trivializations of this bundle over $U_N=S^2\backslash \{\textrm{north  pole}\}$ and $U_S=S^2\backslash \{\textrm{south pole}\}$, and find the corresponding transition function at the equator. Construct a $\U(1)$-connection on this bundle and find its curvature.


\end{document}
