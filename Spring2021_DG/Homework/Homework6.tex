\documentclass[12pt,a4paper]{article}


\usepackage{macros}
\begin{document}\thispagestyle{empty}

\centerline{\Large \bf Homework 7: Due at class on April 18}

 \vspace{.5cm}
\noindent 1. The Euler characteristics in the lecture note is defined by
$$
\chi(M)=\sum_{i\ge 0 } (-1)^i \dim H_i(K;\bR)~.
$$
Show that it is indeed equal to
$$
\chi(M)=\sum_{i\ge 0 } (-1)^i \dim C_i(K;\bR)
$$
given a triangulation $|K|\to M$.



 \vspace{.5cm}
\noindent 2. Show that the Euler characteristics of an odd-dimensional oriented closed manifold is zero.
%Show that $2 \chi(M)=\chi(\partial M)$ for any compact orientable 3-manifold $M$ with boundary $\partial M$.



 \vspace{.5cm}
\noindent 3. Find the integer-valued homology group $H_\ell(\Sigma_g;\bZ)$ of a Riemann surface $\Sigma_g$ of genus $g$. Compute their Euler characteristics.


 \vspace{.5cm}
\noindent 4. Find both the integer-valued $H_\ell(M;\bZ)$ and the real-valued $H_\ell(M;\bR)$ homology groups of both $M=\bR P^2$ and $M=$ Klein bottle. Compute their Euler characteristics.


 \vspace{.5cm}
\noindent 5. Let us construct a 3-dimensional complex \(K\) from \(n\) tetrahedra $T_{1}, \cdots, T_{n}$ by the following two steps. First we arrange the
tetrahedra in a cyclic pattern as in the figure, so that each \(T_{i}\) shares a common vertical face with its two neighbors \(T_{i-1}\)
and \(T_{i+1},\) subscripts being taken mod \(n .\) Then we identify the bottom face of \(T_{i}\) with the top face of \(T_{i+1}\) for each \(i\). Compute the  homology groups of $K$.
\begin{figure}[h]\centering
    \includegraphics[width=5cm]{lens}
\end{figure}

\vspace{.5cm}
\noindent 6. (This is a bonus problem with extra 3 points which is NOT mandatory.)
Let us define a manifold as
$$
T_{-1}=\frac{S^{2} \times I}{(x, 0) \sim(-x, 1)}
$$
where the top and bottom of $S^2$ are identified by the antipodal map $x\mapsto -x$. Compute homology groups of $T_{-1}$.



%
% \vspace{.5cm}
%\noindent 5.  \textbf{Fundamental theorem of algebra}
%
%We define $f:\bC\to \bC$  by
%$f(z) = z^n +a_1z^{n-1} +\cdots+a_n $ for $n\ge 1$. In addition, by writing $z=x+iy$, we define one-form
%$$
%\omega=\frac{-ydx}{x^2+y^2}+\frac{xdy}{x^2+y^2}~.
%$$
%Then, show that
%$$
%\frac{1}{2\pi} \int_{C_R} f^*\omega=n~,
%$$
%where $C_R$ is the circle with sufficiently large radius $R$. If there were no zero points $f(z)=0$ inside $C_R$, show that
%$$
%\frac{1}{2\pi} \int_{C_R} f^*\omega=0~
%$$
%by using the Stokes theorem.
%

\end{document}
