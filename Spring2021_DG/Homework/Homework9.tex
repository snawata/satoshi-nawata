\documentclass[12pt,a4paper]{article}


\usepackage{macros}

\begin{document}\thispagestyle{empty}

\centerline{\Large \bf Homework 9: Due at class on May 9}





 \vspace{.5cm}
\noindent 1.  Derive \textbf{real} dimensions of the following Lie groups
\begin{itemize}
\item Real special linear group: $\SL(n, \bR) = \{A \in \GL(n, \bR)| \det A = 1\}$
\item Symplectic group $\mathrm { Sp } ( n,\bR)=\{ A \in \GL(2n, \bR)| A^ { \mathrm { T } } J A = J \ \mathrm{where} \  J=\begin{pmatrix} 0&I_n\\-I_n&0\end{pmatrix}\}$
\item Special unitary group $\SU(n) = \{A \in \GL(n,\bC)|A^\dagger A=1,\ \det A = 1\}$
\item Special orthogonal group $\SO(n) = \{A \in \GL(n,\bR)|A^TA=1,\ \det A = 1\}$
\end{itemize}
%



 \vspace{.5cm}
\noindent 2.  Write down the definitions of the following Lie algebras:
 $\fraksl(n,\bR)$, $\mathfrak{sp}(n,\bR)$, $\mathfrak{su}(n)$, and $\frakso(n)$.




\vspace{.5cm}
\noindent 3. Show that there are matrices \(A, B \in \frakgl(n,\bC)\) such that $$e^A e^B \neq e^{A+B}~.$$ Modify this equation in such a way that the equality holds for those matrices $A,B$.



 \vspace{.5cm}
\noindent 4.
Let us define
$$
\sigma_{\mu} \equiv(\mathbf{1}, \vec{\sigma})
$$
where $\sigma_{i}$ are the Pauli matrices. Compute that $\operatorname{det} X$ where
$$
X:=x^{\mu} \sigma_{\mu}=\left(\begin{array}{cc}
t+z & x-i y \\
x+i y & t-z
\end{array}\right)~.
$$
Give the definition of the Lorentz group $\SO(1,3)$. $\SO(1,3)$ is indeed not connected.
Let us denote a subgroup of Lorentz group by $\mathrm{SO}^{+}(1,3)$ that satisfies the following two properties:
$$
\operatorname{det} \Lambda^\mu{}_\nu=1~, \qquad
\Lambda^{t}{}_{t} \geq 1  \qquad \Lambda^\mu{}_\nu\in \SO(1,3)~.
$$
Show that this subgroup $\SO^{+}(1,3)$ is isomorphic to $\SL(2,\bC)/\{\pm\mathrm{Id}\}$. Derive the fundamental group $\pi_1(\SO^{+}(1,3))$.



\vspace{.5cm}
\noindent 5. We have seen that $\mathrm{PSL}(2,\bR)=\SL(2,\bR)/\{\pm\mathrm{Id}\}$ acts on the upper half plane ($\textbf{H},\frac{dzd\bar z}{(\Im~z)^2}$) as an isometry group:
$$
\mathrm{PSL}(2,\bR) \times \textbf{H} \to \textbf{H};\begin{pmatrix}
a&b\\
c&d\\
\end{pmatrix} \cdot z\mapsto \frac{az+b}{cz+d}~.
$$
Show that this action is transitive and find the stabilizer subgroup of the point $i\in  \textbf{H}$. Find the fundamental group of $\mathrm{PSL}(2,\bR)$.






 \vspace{.5cm}
\noindent 6.
Show that there is a line (rank one vector) bundle $L$ over $S^2$
such that $TS^2 \oplus L$ is trivial where $TS^2$ is the tangent bundle of $S^2$.





\end{document}
