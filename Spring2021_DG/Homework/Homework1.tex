\documentclass[12pt,a4paper]{article}


\usepackage{macros}

\begin{document}\thispagestyle{empty}

\centerline{\Large \bf Homework 1: Due at class on March 11}

\vspace{1cm}


\noindent \textbf{1}. Euler Characteristic
\begin{enumerate}

\item Find the Euler characteristic of a surface of genus $g$ (Figure 1).

\item Find the Euler characteristic of a Klein bottle (Figure 2).

\item Find the area of the following triangle (Figure 4) bounded by great circles on a 2-sphere with radius one.

\item Generalize it to the area of $n$-gon on a 2-sphere with radius one. (An example of a pentagon is drawn in Figure 5.)

\item Prove the Euler characteristic of a 2-sphere is equal to two by using a cell decomposition (Figure 3).
\end{enumerate}


\begin{figure}[h]
  \begin{minipage}[b]{5cm}\centering
 \includegraphics[width=5cm]{genus-g}
\caption{A surface of genus $g$}
\end{minipage}
  \begin{minipage}[b]{5cm}\centering
 \includegraphics[width=2.5cm]{Klein}
 \caption{A Klein bottle}
\end{minipage}
  \begin{minipage}[b]{5cm}\centering
 \includegraphics[width=3cm]{decomposition}
\caption{cell decomposition of a 2-sphere }
\end{minipage}
\end{figure}

\begin{figure}[h]
  \begin{minipage}[b]{8cm}\centering
 \includegraphics[width=5cm]{triangle}
\caption{A triangle on a 2-sphere}
\end{minipage}
  \begin{minipage}[b]{8cm}\centering
 \includegraphics[width=5cm]{n-gon}
 \caption{A pentagon on a 2-sphere}
\end{minipage}
\end{figure}




\vspace{.5cm}
\noindent \textbf{2}.
 For elements $x=(x^0,\cdots,x^n)$ and   $y=(y^0,\cdots,y^n)$ of $\mathbb{R}^{n+1}\backslash  \{0\}$ , we define an equivalence relation $x \sim y$ by
$$
x=\alpha y  \qquad \forall\alpha \in \mathbb{R}~.
$$
Let us define $\mathbb{R}P^n$ by $(\mathbb{R}^{n+1}\backslash \{0\})/\sim$. Show that $\mathbb{R}P^n$ is a manifold and $\bR P^1$ is diffeomorphic to $S^1$. The space is called a real projective space.




\vspace{.5cm}

\noindent \textbf{3}.  For elements $x=(x^0,\cdots,x^n)$ and   $y=(y^0,\cdots,y^n)$ of $\mathbb{C}^{n+1}\backslash \{0\}$ , we define an equivalence relation $x \sim y$ by
$$
x=\alpha y  \qquad \forall\alpha \in \mathbb{C}~.
$$
Let us define $\mathbb{C}P^n$ by $(\mathbb{C}^{n+1}\backslash\{0\})/\sim$. Show that $\mathbb{C}P^n$ is a manifold and
$\bC P^1$ is diffeomorphic to $S^2$. The space is called a complex projective space.


\vspace{.5cm}

\noindent \textbf{4}.
 Let $M_n(\mathbb{R})$ and $M_n(\mathbb{C})$ denote the set of all $n\times n$ matrices over $\bR$ and $\bC$, respectively. We define
\begin{align}
SU(2)&=\{A\in M_2(\bC)| \ A^\dagger A=\mathrm{Id} , \ \det A=1 \}\cr
SO(3)&=\{A\in M_3(\bR)| \ A^T A=\mathrm{Id} , \ \det A=1 \}~.\nonumber
\end{align}


5.1 Construct a double covering (2-to-1) map $SU(2)\to SO(3)$.

5.2 Show that $SU(2)$ is diffeomorphic to $S^3$ and $SO(3)$ is diffeomorphic to $\bR P^3$.









\end{document}
