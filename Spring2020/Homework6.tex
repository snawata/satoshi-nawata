\documentclass[12pt,a4paper]{article}
\usepackage{hyperref} % Use the Charter font for the document text
%\usepackage[UTF8]{ctex}
\usepackage{fullpage}
\usepackage{amsfonts,amssymb,amsmath}

\usepackage{physics}
\usepackage{epsfig}
\usepackage{amsmath}
\usepackage{amssymb}
\usepackage{amsthm}
\usepackage{indentfirst}
\usepackage{xspace}
\usepackage{multirow}
\usepackage{hyperref}
\usepackage{xcolor}
\usepackage{verbatim}
\usepackage{subfigure}
\usepackage{mathrsfs}
\usepackage{bbm}


%\hypersetup{colorlinks=true,urlcolor=darkred,linkcolor=darkred,citecolor=darkred}
%\usepackage{verbatim}  
\usepackage[letterpaper,margin=0.9in,headheight=15pt]{geometry}
\usepackage{mathpazo}
\usepackage{authblk}
\usepackage{empheq}
\usepackage{feynmp}
\usepackage{graphicx}
\usepackage[matrix,arrow]{xy}
\usepackage{young}
\usepackage[vcentermath]{youngtab}
\usepackage{slashed}
%\usepackage{fontds}
%
\usepackage{bbm}
\usepackage{youngtab}
\usepackage{rotfloat}
\usepackage{stmaryrd}
\usepackage{amsfonts,amssymb,amsmath}
\usepackage{tikz-cd}
\usepackage{thmtools}
\usepackage{dashrule}
\usepackage[missing=]{gitinfo2}
\usepackage{fancyhdr}
\usepackage{mdframed}
\usepackage{booktabs}
\usepackage{subfiles}
\usepackage{simplewick}

\usepackage[utf8]{inputenc}


%%%%%%%%%%%% math fonts %%%%%%%%%%%%%%%%%%%%%%%%%%%%%%%%%%%%%
%
%---------- mathbb font --------------------------------
%

\newcommand{\bA}{\ensuremath{\mathbb{A}}}
\newcommand{\bB}{\ensuremath{\mathbb{B}}}
\newcommand{\bC}{\ensuremath{\mathbb{C}}}
\newcommand{\bD}{\ensuremath{\mathbb{D}}}
\newcommand{\bE}{\ensuremath{\mathbb{E}}}
\newcommand{\bF}{\ensuremath{\mathbb{F}}}
\newcommand{\bG}{\ensuremath{\mathbb{G}}}
\newcommand{\bH}{\ensuremath{\mathbb{H}}}
\newcommand{\bI}{\ensuremath{\mathbb{I}}}
\newcommand{\bJ}{\ensuremath{\mathbb{J}}}
\newcommand{\bK}{\ensuremath{\mathbb{K}}}
\newcommand{\bL}{\ensuremath{\mathbb{L}}}
\newcommand{\bM}{\ensuremath{\mathbb{M}}}
\newcommand{\bN}{\ensuremath{\mathbb{N}}}
\newcommand{\bO}{\ensuremath{\mathbb{O}}}
\newcommand{\bP}{\ensuremath{\mathbb{P}}}
\newcommand{\bQ}{\ensuremath{\mathbb{Q}}}
\newcommand{\bR}{\ensuremath{\mathbb{R}}}
\newcommand{\bS}{\ensuremath{\mathbb{S}}}
\newcommand{\bT}{\ensuremath{\mathbb{T}}}
\newcommand{\bU}{\ensuremath{\mathbb{U}}}
\newcommand{\bV}{\ensuremath{\mathbb{V}}}
\newcommand{\bW}{\ensuremath{\mathbb{W}}}
\newcommand{\bX}{\ensuremath{\mathbb{X}}}
\newcommand{\bY}{\ensuremath{\mathbb{Y}}}
\newcommand{\bZ}{\ensuremath{\mathbb{Z}}}



%
%---------- mathbf font --------------------------------
%


\newcommand{\bfA}{\ensuremath{\mathbf{A}}}
\newcommand{\bfB}{\ensuremath{\mathbf{B}}}
\newcommand{\bfC}{\ensuremath{\mathbf{C}}}
\newcommand{\bfD}{\ensuremath{\mathbf{D}}}
\newcommand{\bfE}{\ensuremath{\mathbf{E}}}
\newcommand{\bfF}{\ensuremath{\mathbf{F}}}
\newcommand{\bfG}{\ensuremath{\mathbf{G}}}
\newcommand{\bfH}{\ensuremath{\mathbf{H}}}
\newcommand{\bfI}{\ensuremath{\mathbf{I}}}
\newcommand{\bfJ}{\ensuremath{\mathbf{J}}}
\newcommand{\bfK}{\ensuremath{\mathbf{K}}}
\newcommand{\bfL}{\ensuremath{\mathbf{L}}}
\newcommand{\bfM}{\ensuremath{\mathbf{M}}}
\newcommand{\bfN}{\ensuremath{\mathbf{N}}}
\newcommand{\bfO}{\ensuremath{\mathbf{O}}}
\newcommand{\bfP}{\ensuremath{\mathbf{P}}}
\newcommand{\bfQ}{\ensuremath{\mathbf{Q}}}
\newcommand{\bfR}{\ensuremath{\mathbf{R}}}
\newcommand{\bfS}{\ensuremath{\mathbf{S}}}
\newcommand{\bfT}{\ensuremath{\mathbf{T}}}
\newcommand{\bfU}{\ensuremath{\mathbf{U}}}
\newcommand{\bfV}{\ensuremath{\mathbf{V}}}
\newcommand{\bfW}{\ensuremath{\mathbf{W}}}
\newcommand{\bfX}{\ensuremath{\mathbf{X}}}
\newcommand{\bfY}{\ensuremath{\mathbf{Y}}}
\newcommand{\bfZ}{\ensuremath{\mathbf{Z}}}




%
%---------- mathcal font -----------------------------
%

\newcommand{\scA}{\ensuremath{\mathscr{A}}}
\newcommand{\scB}{\ensuremath{\mathscr{B}}}
\newcommand{\scC}{\ensuremath{\mathscr{C}}}
\newcommand{\scD}{\ensuremath{\mathscr{D}}}
\newcommand{\scE}{\ensuremath{\mathscr{E}}}
\newcommand{\scF}{\ensuremath{\mathscr{F}}}
\newcommand{\scG}{\ensuremath{\mathscr{G}}}
\newcommand{\scH}{\ensuremath{\mathscr{H}}}
\newcommand{\scI}{\ensuremath{\mathscr{I}}}
\newcommand{\scJ}{\ensuremath{\mathscr{J}}}
\newcommand{\scK}{\ensuremath{\mathscr{K}}}
\newcommand{\scL}{\ensuremath{\mathscr{L}}}
\newcommand{\scM}{\ensuremath{\mathscr{M}}}
\newcommand{\scN}{\ensuremath{\mathscr{N}}}
\newcommand{\scO}{\ensuremath{\mathscr{O}}}
\newcommand{\scP}{\ensuremath{\mathscr{P}}}
\newcommand{\scQ}{\ensuremath{\mathscr{Q}}}
\newcommand{\scR}{\ensuremath{\mathscr{R}}}
\newcommand{\scS}{\ensuremath{\mathscr{S}}}
\newcommand{\scT}{\ensuremath{\mathscr{T}}}
\newcommand{\scU}{\ensuremath{\mathscr{U}}}
\newcommand{\scV}{\ensuremath{\mathscr{V}}}
\newcommand{\scW}{\ensuremath{\mathscr{W}}}
\newcommand{\scX}{\ensuremath{\mathscr{X}}}
\newcommand{\scY}{\ensuremath{\mathscr{Y}}}
\newcommand{\scZ}{\ensuremath{\mathscr{Z}}}

%
%---------- mathfrak font -----------------------------
%

\newcommand{\frakA}{\ensuremath{\mathfrak{A}}}
\newcommand{\frakB}{\ensuremath{\mathfrak{B}}}
\newcommand{\frakC}{\ensuremath{\mathfrak{C}}}
\newcommand{\frakD}{\ensuremath{\mathfrak{D}}}
\newcommand{\frakE}{\ensuremath{\mathfrak{E}}}
\newcommand{\frakF}{\ensuremath{\mathfrak{F}}}
\newcommand{\frakG}{\ensuremath{\mathfrak{G}}}
\newcommand{\frakH}{\ensuremath{\mathfrak{H}}}
\newcommand{\frakI}{\ensuremath{\mathfrak{I}}}
\newcommand{\frakJ}{\ensuremath{\mathfrak{J}}}
\newcommand{\frakK}{\ensuremath{\mathfrak{K}}}
\newcommand{\frakL}{\ensuremath{\mathfrak{L}}}
\newcommand{\frakM}{\ensuremath{\mathfrak{M}}}
\newcommand{\frakN}{\ensuremath{\mathfrak{N}}}
\newcommand{\frakO}{\ensuremath{\mathfrak{O}}}
\newcommand{\frakP}{\ensuremath{\mathfrak{P}}}
\newcommand{\frakQ}{\ensuremath{\mathfrak{Q}}}
\newcommand{\frakR}{\ensuremath{\mathfrak{R}}}
\newcommand{\frakS}{\ensuremath{\mathfrak{S}}}
\newcommand{\frakT}{\ensuremath{\mathfrak{T}}}
\newcommand{\frakU}{\ensuremath{\mathfrak{U}}}
\newcommand{\frakV}{\ensuremath{\mathfrak{V}}}
\newcommand{\frakW}{\ensuremath{\mathfrak{W}}}
\newcommand{\frakX}{\ensuremath{\mathfrak{X}}}
\newcommand{\frakY}{\ensuremath{\mathfrak{Y}}}
\newcommand{\frakZ}{\ensuremath{\mathfrak{Z}}}
\newcommand{\fraka}{\ensuremath{\mathfrak{a}}}
\newcommand{\frakb}{\ensuremath{\mathfrak{b}}}
\newcommand{\frakc}{\ensuremath{\mathfrak{c}}}
\newcommand{\frakd}{\ensuremath{\mathfrak{d}}}
\newcommand{\frake}{\ensuremath{\mathfrak{e}}}
\newcommand{\frakf}{\ensuremath{\mathfrak{f}}}
\newcommand{\frakg}{\ensuremath{\mathfrak{g}}}
\newcommand{\frakh}{\ensuremath{\mathfrak{h}}}
\newcommand{\fraki}{\ensuremath{\mathfrak{i}}}
\newcommand{\frakj}{\ensuremath{\mathfrak{j}}}
\newcommand{\frakk}{\ensuremath{\mathfrak{k}}}
\newcommand{\frakl}{\ensuremath{\mathfrak{l}}}
\newcommand{\frakm}{\ensuremath{\mathfrak{m}}}
\newcommand{\frakn}{\ensuremath{\mathfrak{n}}}
\newcommand{\frako}{\ensuremath{\mathfrak{o}}}
\newcommand{\frakp}{\ensuremath{\mathfrak{p}}}
\newcommand{\frakq}{\ensuremath{\mathfrak{q}}}
\newcommand{\frakr}{\ensuremath{\mathfrak{r}}}
\newcommand{\fraks}{\ensuremath{\mathfrak{s}}}
\newcommand{\frakt}{\ensuremath{\mathfrak{t}}}
\newcommand{\fraku}{\ensuremath{\mathfrak{u}}}
\newcommand{\frakv}{\ensuremath{\mathfrak{v}}}
\newcommand{\frakw}{\ensuremath{\mathfrak{w}}}
\newcommand{\frakx}{\ensuremath{\mathfrak{x}}}
\newcommand{\fraky}{\ensuremath{\mathfrak{y}}}
\newcommand{\frakz}{\ensuremath{\mathfrak{z}}}
\newcommand{\fraksl}{\ensuremath{\mathfrak{sl}}}
\newcommand{\frakso}{\ensuremath{\mathfrak{so}}}
\newcommand{\fraksp}{\ensuremath{\mathfrak{sp}}}

%%%%%%%%%%%%  Calligraphic, Roman and Maths integers %%%%%%%%%%%%%%%%%%

\newcommand{\cA}{\mathcal{A}}
\newcommand{\cB}{\mathcal{B}}
\newcommand{\cC}{\mathcal{C}}
\newcommand{\cD}{\mathcal{D}}
\newcommand{\cE}{\mathcal{E}}
\newcommand{\cF}{\mathcal{F}}
\newcommand{\cG}{\mathcal{G}}
\newcommand{\cH}{\mathcal{H}}
\newcommand{\cI}{\mathcal{I}}
\newcommand{\cJ}{\mathcal{J}}
\newcommand{\cK}{\mathcal{K}}
\newcommand{\cL}{\mathcal{L}}
\newcommand{\cM}{\mathcal{M}}
\newcommand{\cN}{\mathcal{N}}
\newcommand{\cO}{\mathcal{O}}
\newcommand{\cP}{\mathcal{P}}
\newcommand{\cQ}{\mathcal{Q}}
\newcommand{\cS}{\mathcal{S}}
\newcommand{\cU}{\mathcal{U}}
\newcommand{\cX}{\mathcal{X}}
\newcommand{\cY}{\mathcal{Y}}
\newcommand{\cV}{\mathcal{V}}
\newcommand{\cW}{\mathcal{W}}
\newcommand{\cR}{\mathcal{R}}
\newcommand{\cT}{\mathcal{T}}
\newcommand{\cZ}{\mathcal{Z}}


%%%%%%%%%%%% mathsf%%%%%%%%%%%%%%%%%%


\newcommand{\sfA}{\ensuremath{\mathsf{A}}}
\newcommand{\sfB}{\ensuremath{\mathsf{B}}}
\newcommand{\sfC}{\ensuremath{\mathsf{C}}}
\newcommand{\sfD}{\ensuremath{\mathsf{D}}}
\newcommand{\sfE}{\ensuremath{\mathsf{E}}}
\newcommand{\sfF}{\ensuremath{\mathsf{F}}}
\newcommand{\sfG}{\ensuremath{\mathsf{G}}}
\newcommand{\sfH}{\ensuremath{\mathsf{H}}}
\newcommand{\sfJ}{\ensuremath{\mathsf{J}}}
\newcommand{\sfK}{\ensuremath{\mathsf{K}}}
\newcommand{\sfL}{\ensuremath{\mathsf{L}}}
\newcommand{\sfM}{\ensuremath{\mathsf{M}}}
\newcommand{\sfN}{\ensuremath{\mathsf{N}}}
\newcommand{\sfO}{\ensuremath{\mathsf{O}}}
\newcommand{\sfP}{\ensuremath{\mathsf{P}}}
\newcommand{\sfQ}{\ensuremath{\mathsf{Q}}}
\newcommand{\sfS}{\ensuremath{\mathsf{S}}}
\newcommand{\sfU}{\ensuremath{\mathsf{U}}}
\newcommand{\sfX}{\ensuremath{\mathsf{X}}}
\newcommand{\sfY}{\ensuremath{\mathsf{Y}}}
\newcommand{\sfW}{\ensuremath{\mathsf{W}}}
\newcommand{\sfR}{\ensuremath{\mathsf{R}}}
\newcommand{\sfT}{\ensuremath{\mathsf{T}}}
\newcommand{\sfZ}{\ensuremath{\mathsf{Z}}}

%%%%%%%%%%%%  Special letters for Lie groups %%%%%%%%%%%%%%%%%%

\newcommand{\biA}{{\mathbi{A}}}
\newcommand{\biB}{{\mathbi{B}}}
\newcommand{\biC}{{\mathbi{C}}}
\newcommand{\biD}{{\mathbi{D}}}
\newcommand{\biE}{{\mathbi{E}}}
\newcommand{\biF}{{\mathbi{F}}}
\newcommand{\biG}{{\mathbi{G}}}
\newcommand{\biH}{{\mathbi{H}}}
\newcommand{\biI}{{\mathbi{I}}}
\newcommand{\biJ}{{\mathbi{J}}}
\newcommand{\biK}{{\mathbi{K}}}
\newcommand{\biL}{{\mathbi{L}}}
\newcommand{\biM}{{\mathbi{M}}}
\newcommand{\biN}{{\mathbi{N}}}
\newcommand{\biO}{{\mathbi{O}}}
\newcommand{\biP}{{\mathbi{P}}}
\newcommand{\biQ}{{\mathbi{Q}}}
\newcommand{\biS}{{\mathbi{S}}}
\newcommand{\biU}{{\mathbi{U}}}
\newcommand{\biX}{{\mathbi{X}}}
\newcommand{\biY}{{\mathbi{Y}}}
\newcommand{\biV}{{\mathbi{V}}}
\newcommand{\biW}{{\mathbi{W}}}
\newcommand{\biR}{{\mathbi{R}}}
\newcommand{\biT}{{\mathbi{T}}}
\newcommand{\biZ}{{\mathbi{Z}}}




%%%%%%%%%%%%%%%%%%%%%%%%%%%%%%%%%%%%%%%%%%%%%%%%%%%%%%%%%%%%%%%%
\newcommand{\SU}{\mathrm{SU}}
\newcommand{\SO}{\mathrm{SO}}
\newcommand{\SL}{\mathrm{SL}}
\newcommand{\Sp}{\mathrm{Sp}}
\newcommand{\U}{\mathrm{U}}
\newcommand{\ul}{\mathrm{u}}
\newcommand{\Spin}{\mathrm{Spin}}
\newcommand{\Pin}{\mathrm{Pin}}
\newcommand{\PSL}{\mathrm{PSL}}
%%%%%%%%%%%%%%%%%%%%%%%%%%%%%%%%%%%%%%%%%%%%%%%%%%%%%%%%%%%%%%%%




\def \be  {\begin{equation}}
\def \ee  {\end{equation}}
\def \bea {\begin{equation}\begin{aligned}}
\def \eea {\end{aligned}\end{equation}}
\def \ba  {\begin{eqnarray}}
\def \ea  {\end{eqnarray}}

\begin{document}\thispagestyle{empty}

\centerline{\Large \bf Homework 6: Due at class on May 25}


\section{Higgs mechanism in electroweak theory}

The Glashow-Weinberg-Salam theory is the part of the Standard Model (SM) of particle physics which describes the electroweak interactions by a non-Abelian gauge theory with the gauge group $\mathrm{SU}(2)_L \times \mathrm{U}(1)_{Y}$.  In a one-family approximation, the SM has the following particle content:
\begin{equation}\nonumber
\begin{array}{cccccc}
\hline & E_{L}=\left(\begin{array}{l}
\nu_{e} \cr
e^{-}
\end{array}\right)_{L} & e_{R} & \phi=\left(\begin{array}{c}
\phi^{+} \cr
\phi^{0}
\end{array}\right) & T^{a} W_{\mu}^{a} & B_{\mu} \cr
\hline \text { Hypercharge } Y & -\frac12& -1 & +\frac12 & 0 & 0 \cr
\mathrm{SU}(2)_L \text { rep. } & \mathbf{2} & \mathbf{1} & \mathbf{2} & \mathbf{3} & \mathbf{1} \cr
\text { Lorentz rep. } & (\frac12,0) & (0,\frac12) & (0,0) & (\frac12,\frac12) & (\frac12,\frac12) \cr
\hline
\end{array}
\end{equation}
where $E_L, e_R$ contain Dirac spinors and the superscripts in the Higgs doublet denote electromagnetic charges. The corresponding Lagrangian is given by
\begin{equation}\label{Lagrangian}
\begin{aligned}
\cL=&\overline{E}_{L}(i \slashed D) E_{L}+\overline{e}_{R}(i \slashed D) e_{R} -\frac{1}{4} W_{\mu \nu}^{a} W^{\mu \nu a}-\frac{1}{4} B_{\mu \nu} B^{\mu \nu} \cr
&+\left(D_{\mu} \phi\right)^{\dagger}\left(D^{\mu} \phi\right)-\mu^{2} \phi^{\dagger} \phi-\lambda\left(\phi^{\dagger} \phi\right)^{2} -\left(\overline{E}_{L} \phi e_R+\overline{e_R} \phi^{\dagger} E_L\right)
\end{aligned}
\end{equation}
with
\be\label{covariant}
\begin{aligned}
    D_{\mu}&=\partial_{\mu}-i g W_{\mu}^{a} T^{a}-i g^{\prime} Y B_{\mu} \cr
B_{\mu \nu} &=\partial_{\mu} B_{\nu}-\partial_{\nu} B_{\mu}, \quad W_{\mu \nu}^{a}=\partial_{\mu} W_{\nu}^{a}-\partial_{\nu} W_{\mu}^{a}+g \epsilon^{a b c} W_{\mu}^{b} W_{\nu}^{c}
\end{aligned}
\ee
\subsection{}
Write down how the covariant derivative \eqref{covariant} acts on the left- and right-handed leptons doublets and on the Higgs-doublet.

\subsection{}Show that the Lagrangian \eqref{Lagrangian} is Lorentz invariant and gauge invariant as well.

\subsection{}For the Higgs mechanism to work we need $\mu^{2}<0$.  For which value of $|\phi|$ does the Higgs potential obtain a minimum? By an $\mathrm{SU}(2)_L$ rotation we can choose the vacuum expectation value (VEV) of the Higgs field to be of the form $\langle\phi\rangle=\frac{1}{\sqrt{2}}(0, v)^{T}$.  This leads to a redefinition of the excitation modes of the Higgs fields,
$$
\phi(x)=\exp \left\{\frac{\mathrm{i}}{v} \xi^{a}(x) T^{a}\right\}\left(\begin{array}{c}
0 \cr
\frac{1}{\sqrt{2}}(v+\eta(x))
\end{array}\right)
$$
with $\xi^{a}(x)$ and $\eta(x)$ being real fields and $T^{a}$ the generators of $\mathrm{SU}(2)_L$.  Now we apply an $\mathrm{SU}(2)_L$ gauge transformation such that the angular excitations $\xi^{a}(x)$ vanish. This gauge transformation is called \emph{unitary gauge}. Show that the Higgs potential in the unitary gauge is given by
$$
V(\phi)=-\mu^{2} \eta^{2}(x)+\lambda v \eta^{3}(x)+\frac{\lambda}{4} \eta^{4}(x)
$$
What is the mass of the $\eta$ field? Compare the degrees of freedom (DOF) in the Higgs sector to the situation before symmetry breakdown.

\subsection{}Consider the kinetic energy terms of the Higgs field in \eqref{Lagrangian}. Show that
\bea\label{3}
\left(D_{\mu} \phi\right)^{\dagger}\left(D^{\mu} \phi\right)=\frac{1}{2} \partial_{\mu} \eta \partial^{\mu} \eta &+\frac{1}{4} g^{2}(v+\eta)^{2} W_{\mu}^{-} W^{+\mu} \cr
&+\frac{1}{8}(v+\eta)^{2}(W_{\mu}^{3}\quad B_{\mu}) \left(\begin{array}{cc}
g^{2} & -g^{\prime} g \cr
-g^{\prime} g & g^{\prime 2}
\end{array}\right)\left(\begin{array}{c}
W^{3 \mu} \cr
B^{\mu}
\end{array}\right)
\eea
with $W^{\pm \mu}:=\frac{1}{\sqrt{2}}\left(W^{1 \mu} \mp \mathrm{i} W^{2 \mu}\right)$



\subsection{}The masses of the gauge bosons are given by the terms that are quadratic in the fields, e.g. $\frac{1}{4} g^{2} v^{2} W_{\mu}^{-} W^{+\mu}=m_{W}^{2} W_{\mu}^{-} W^{+\mu},$ where $m_{W}=\frac{1}{2} v g$.  However, to see the masses of
$W_{\mu}^{3}$ and $B_{\mu}$ one has to diagonalize the matrix in \eqref{3}:
$$
\frac{1}{8}(W_{\mu}^{3}\quad B_{\mu}) \mathcal{O}^{T} \mathcal{O}\left(\begin{array}{cc}
g^{2} & -g^{\prime} g \cr
-g^{\prime} g & g^{\prime 2}
\end{array}\right) \mathcal{O}^{T} \mathcal{O}\left(\begin{array}{c}
W^{3 \mu} \cr
B^{\mu}
\end{array}\right)=\frac12(\begin{array}{cc}
Z_{\mu} & A_{\mu}
\end{array})\left(\begin{array}{cc}
m_{Z}^{2} & 0 \cr
0 & m_{A}^{2}
\end{array}\right)\left(\begin{array}{c}
Z^{\mu} \cr
A^{\mu}
\end{array}\right)
$$
Determine this orthogonal matrix $\mathcal{O}$ by computing the corresponding eigenvalues and eigenvectors. What are the masses of the $Z_{\mu}$ and $A_{\mu}$ fields? Compare the DOF in the gauge sector to the situation before the symmetry breakdown. What can you say about the total amount of DOF?

\subsection{}As you know, an orthogonal $2 \times 2$ matrix can be written as
$$
\mathcal{O}=\left(\begin{array}{cc}
\cos \theta_{W} & -\sin \theta_{W} \cr
\sin \theta_{W} & \cos \theta_{W}
\end{array}\right)
$$
Write $\cos \theta_{W}$ in terms of $g^{\prime}$ and $g$.  Show for the ratio of the $W$ - and $Z$ -boson masses
$$
\frac{m_{W}}{m_{Z}}=\cos \theta_{W}
$$
The angle $\theta_{W}$ is sometimes called Weinberg angle or weak mixing angle.

\subsection{}Finally, consider the covariant derivative \eqref{covariant}. Substitute the fields $B_{\mu}$ and $W_{\mu}^{a}$ by $W_{\mu}^{\pm}, Z_{\mu}$ and $A_{\mu}$ and show
$$
D_{\mu}=\partial_{\mu}-i \frac{g}{\sqrt{2}}\left(\begin{array}{ccc}
0 & W_{\mu}^{+} \cr
W_{\mu}^{-} & 0
\end{array}\right)-i \frac{1}{\sqrt{g^{2}+g^{\prime 2}}} Z_{\mu}\left(g^{2} T^{3}-g^{\prime 2} Y\right)
-i eQ A_{\mu}
$$
where we have defined the electric charge $e=\frac{g^{\prime} g}{\sqrt{g^{\prime 2}+g^{2}}}$ and $Q:=T^{3}+Y$



\subsection{}Consider the following terms of the Lagrangian:
$$
\mathcal{L} \supset \overline{E}_{L}(i \slashed D) E_{L}+\overline{e}_{R}(i \slashed D) e_{R}
$$
Find the interaction terms of the fermions with the gauge bosons. For the weak interaction, analyze its V-A structure $\frac{1}{2}\left(c_{V}+c_{A} \gamma^{5}\right)$.  Draw the corresponding Feynman diagrams (Note: use $i \mathcal{L},$ drop all fields and you get the vertex factor).


\subsection{}
 Using the unitary gauge, insert the shifted Higgs field
$$
\phi(x)=\left(\begin{array}{c}
0 \cr
\frac{1}{\sqrt{2}}(v+\eta(x))
\end{array}\right)
$$
into the Yukawa couplings of the Lagrangian:
$$
\mathcal{L} \supset-G_{e}\left(\overline{E}_{L} \phi e_R+\overline{e_R} \phi^{\dagger} E_L\right)
$$
Show that the mass of the electron is $m_{e}=\frac{G_{e} v}{\sqrt{2}}$.



\section{Anomaly cancellation in Standard Model}
 A condition for a QFT to be well-defined is the cancellation of anomalies associated with local symmetries. Show that the SM is anomaly free and
$$
\mathcal{A}^{a b c}=\operatorname{tr}\left[t^{a}\{t^{b}, t^{c}\}\right]=0
$$
\subsection{}
List the different triangle diagrams that exist in terms of group structure. Using the fact that $\SU(3)_{C}$ is a vector-like interaction and properties of $\mathrm{SU}(2)_L$, argue that the only nontrivial contributions are those with an odd number of hypercharge insertions. Show that some relations are equivalent to requiring cancellation of electric charge inside each generation.

\subsection{}
Are baryon number $(B)$ and lepton number $(L)$ currents anomaly free?



\end{document}
