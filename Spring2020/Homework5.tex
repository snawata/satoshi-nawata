\documentclass[12pt,a4paper]{article}
\usepackage{hyperref} % Use the Charter font for the document text
%\usepackage[UTF8]{ctex}
\usepackage{fullpage}
\usepackage{amsfonts,amssymb,amsmath}

\usepackage{physics}
\usepackage{epsfig}
\usepackage{amsmath}
\usepackage{amssymb}
\usepackage{amsthm}
\usepackage{indentfirst}
\usepackage{xspace}
\usepackage{multirow}
\usepackage{hyperref}
\usepackage{xcolor}
\usepackage{verbatim}
\usepackage{subfigure}
\usepackage{mathrsfs}
\usepackage{bbm}


%\hypersetup{colorlinks=true,urlcolor=darkred,linkcolor=darkred,citecolor=darkred}
%\usepackage{verbatim}  
\usepackage[letterpaper,margin=0.9in,headheight=15pt]{geometry}
\usepackage{mathpazo}
\usepackage{authblk}
\usepackage{empheq}
\usepackage{feynmp}
\usepackage{graphicx}
\usepackage[matrix,arrow]{xy}
\usepackage{young}
\usepackage[vcentermath]{youngtab}
\usepackage{slashed}
%\usepackage{fontds}
%
\usepackage{bbm}
\usepackage{youngtab}
\usepackage{rotfloat}
\usepackage{stmaryrd}
\usepackage{amsfonts,amssymb,amsmath}
\usepackage{tikz-cd}
\usepackage{thmtools}
\usepackage{dashrule}
\usepackage[missing=]{gitinfo2}
\usepackage{fancyhdr}
\usepackage{mdframed}
\usepackage{booktabs}
\usepackage{subfiles}
\usepackage{simplewick}

\usepackage[utf8]{inputenc}


%%%%%%%%%%%% math fonts %%%%%%%%%%%%%%%%%%%%%%%%%%%%%%%%%%%%%
%
%---------- mathbb font --------------------------------
%

\newcommand{\bA}{\ensuremath{\mathbb{A}}}
\newcommand{\bB}{\ensuremath{\mathbb{B}}}
\newcommand{\bC}{\ensuremath{\mathbb{C}}}
\newcommand{\bD}{\ensuremath{\mathbb{D}}}
\newcommand{\bE}{\ensuremath{\mathbb{E}}}
\newcommand{\bF}{\ensuremath{\mathbb{F}}}
\newcommand{\bG}{\ensuremath{\mathbb{G}}}
\newcommand{\bH}{\ensuremath{\mathbb{H}}}
\newcommand{\bI}{\ensuremath{\mathbb{I}}}
\newcommand{\bJ}{\ensuremath{\mathbb{J}}}
\newcommand{\bK}{\ensuremath{\mathbb{K}}}
\newcommand{\bL}{\ensuremath{\mathbb{L}}}
\newcommand{\bM}{\ensuremath{\mathbb{M}}}
\newcommand{\bN}{\ensuremath{\mathbb{N}}}
\newcommand{\bO}{\ensuremath{\mathbb{O}}}
\newcommand{\bP}{\ensuremath{\mathbb{P}}}
\newcommand{\bQ}{\ensuremath{\mathbb{Q}}}
\newcommand{\bR}{\ensuremath{\mathbb{R}}}
\newcommand{\bS}{\ensuremath{\mathbb{S}}}
\newcommand{\bT}{\ensuremath{\mathbb{T}}}
\newcommand{\bU}{\ensuremath{\mathbb{U}}}
\newcommand{\bV}{\ensuremath{\mathbb{V}}}
\newcommand{\bW}{\ensuremath{\mathbb{W}}}
\newcommand{\bX}{\ensuremath{\mathbb{X}}}
\newcommand{\bY}{\ensuremath{\mathbb{Y}}}
\newcommand{\bZ}{\ensuremath{\mathbb{Z}}}



%
%---------- mathbf font --------------------------------
%


\newcommand{\bfA}{\ensuremath{\mathbf{A}}}
\newcommand{\bfB}{\ensuremath{\mathbf{B}}}
\newcommand{\bfC}{\ensuremath{\mathbf{C}}}
\newcommand{\bfD}{\ensuremath{\mathbf{D}}}
\newcommand{\bfE}{\ensuremath{\mathbf{E}}}
\newcommand{\bfF}{\ensuremath{\mathbf{F}}}
\newcommand{\bfG}{\ensuremath{\mathbf{G}}}
\newcommand{\bfH}{\ensuremath{\mathbf{H}}}
\newcommand{\bfI}{\ensuremath{\mathbf{I}}}
\newcommand{\bfJ}{\ensuremath{\mathbf{J}}}
\newcommand{\bfK}{\ensuremath{\mathbf{K}}}
\newcommand{\bfL}{\ensuremath{\mathbf{L}}}
\newcommand{\bfM}{\ensuremath{\mathbf{M}}}
\newcommand{\bfN}{\ensuremath{\mathbf{N}}}
\newcommand{\bfO}{\ensuremath{\mathbf{O}}}
\newcommand{\bfP}{\ensuremath{\mathbf{P}}}
\newcommand{\bfQ}{\ensuremath{\mathbf{Q}}}
\newcommand{\bfR}{\ensuremath{\mathbf{R}}}
\newcommand{\bfS}{\ensuremath{\mathbf{S}}}
\newcommand{\bfT}{\ensuremath{\mathbf{T}}}
\newcommand{\bfU}{\ensuremath{\mathbf{U}}}
\newcommand{\bfV}{\ensuremath{\mathbf{V}}}
\newcommand{\bfW}{\ensuremath{\mathbf{W}}}
\newcommand{\bfX}{\ensuremath{\mathbf{X}}}
\newcommand{\bfY}{\ensuremath{\mathbf{Y}}}
\newcommand{\bfZ}{\ensuremath{\mathbf{Z}}}




%
%---------- mathcal font -----------------------------
%

\newcommand{\scA}{\ensuremath{\mathscr{A}}}
\newcommand{\scB}{\ensuremath{\mathscr{B}}}
\newcommand{\scC}{\ensuremath{\mathscr{C}}}
\newcommand{\scD}{\ensuremath{\mathscr{D}}}
\newcommand{\scE}{\ensuremath{\mathscr{E}}}
\newcommand{\scF}{\ensuremath{\mathscr{F}}}
\newcommand{\scG}{\ensuremath{\mathscr{G}}}
\newcommand{\scH}{\ensuremath{\mathscr{H}}}
\newcommand{\scI}{\ensuremath{\mathscr{I}}}
\newcommand{\scJ}{\ensuremath{\mathscr{J}}}
\newcommand{\scK}{\ensuremath{\mathscr{K}}}
\newcommand{\scL}{\ensuremath{\mathscr{L}}}
\newcommand{\scM}{\ensuremath{\mathscr{M}}}
\newcommand{\scN}{\ensuremath{\mathscr{N}}}
\newcommand{\scO}{\ensuremath{\mathscr{O}}}
\newcommand{\scP}{\ensuremath{\mathscr{P}}}
\newcommand{\scQ}{\ensuremath{\mathscr{Q}}}
\newcommand{\scR}{\ensuremath{\mathscr{R}}}
\newcommand{\scS}{\ensuremath{\mathscr{S}}}
\newcommand{\scT}{\ensuremath{\mathscr{T}}}
\newcommand{\scU}{\ensuremath{\mathscr{U}}}
\newcommand{\scV}{\ensuremath{\mathscr{V}}}
\newcommand{\scW}{\ensuremath{\mathscr{W}}}
\newcommand{\scX}{\ensuremath{\mathscr{X}}}
\newcommand{\scY}{\ensuremath{\mathscr{Y}}}
\newcommand{\scZ}{\ensuremath{\mathscr{Z}}}

%
%---------- mathfrak font -----------------------------
%

\newcommand{\frakA}{\ensuremath{\mathfrak{A}}}
\newcommand{\frakB}{\ensuremath{\mathfrak{B}}}
\newcommand{\frakC}{\ensuremath{\mathfrak{C}}}
\newcommand{\frakD}{\ensuremath{\mathfrak{D}}}
\newcommand{\frakE}{\ensuremath{\mathfrak{E}}}
\newcommand{\frakF}{\ensuremath{\mathfrak{F}}}
\newcommand{\frakG}{\ensuremath{\mathfrak{G}}}
\newcommand{\frakH}{\ensuremath{\mathfrak{H}}}
\newcommand{\frakI}{\ensuremath{\mathfrak{I}}}
\newcommand{\frakJ}{\ensuremath{\mathfrak{J}}}
\newcommand{\frakK}{\ensuremath{\mathfrak{K}}}
\newcommand{\frakL}{\ensuremath{\mathfrak{L}}}
\newcommand{\frakM}{\ensuremath{\mathfrak{M}}}
\newcommand{\frakN}{\ensuremath{\mathfrak{N}}}
\newcommand{\frakO}{\ensuremath{\mathfrak{O}}}
\newcommand{\frakP}{\ensuremath{\mathfrak{P}}}
\newcommand{\frakQ}{\ensuremath{\mathfrak{Q}}}
\newcommand{\frakR}{\ensuremath{\mathfrak{R}}}
\newcommand{\frakS}{\ensuremath{\mathfrak{S}}}
\newcommand{\frakT}{\ensuremath{\mathfrak{T}}}
\newcommand{\frakU}{\ensuremath{\mathfrak{U}}}
\newcommand{\frakV}{\ensuremath{\mathfrak{V}}}
\newcommand{\frakW}{\ensuremath{\mathfrak{W}}}
\newcommand{\frakX}{\ensuremath{\mathfrak{X}}}
\newcommand{\frakY}{\ensuremath{\mathfrak{Y}}}
\newcommand{\frakZ}{\ensuremath{\mathfrak{Z}}}
\newcommand{\fraka}{\ensuremath{\mathfrak{a}}}
\newcommand{\frakb}{\ensuremath{\mathfrak{b}}}
\newcommand{\frakc}{\ensuremath{\mathfrak{c}}}
\newcommand{\frakd}{\ensuremath{\mathfrak{d}}}
\newcommand{\frake}{\ensuremath{\mathfrak{e}}}
\newcommand{\frakf}{\ensuremath{\mathfrak{f}}}
\newcommand{\frakg}{\ensuremath{\mathfrak{g}}}
\newcommand{\frakh}{\ensuremath{\mathfrak{h}}}
\newcommand{\fraki}{\ensuremath{\mathfrak{i}}}
\newcommand{\frakj}{\ensuremath{\mathfrak{j}}}
\newcommand{\frakk}{\ensuremath{\mathfrak{k}}}
\newcommand{\frakl}{\ensuremath{\mathfrak{l}}}
\newcommand{\frakm}{\ensuremath{\mathfrak{m}}}
\newcommand{\frakn}{\ensuremath{\mathfrak{n}}}
\newcommand{\frako}{\ensuremath{\mathfrak{o}}}
\newcommand{\frakp}{\ensuremath{\mathfrak{p}}}
\newcommand{\frakq}{\ensuremath{\mathfrak{q}}}
\newcommand{\frakr}{\ensuremath{\mathfrak{r}}}
\newcommand{\fraks}{\ensuremath{\mathfrak{s}}}
\newcommand{\frakt}{\ensuremath{\mathfrak{t}}}
\newcommand{\fraku}{\ensuremath{\mathfrak{u}}}
\newcommand{\frakv}{\ensuremath{\mathfrak{v}}}
\newcommand{\frakw}{\ensuremath{\mathfrak{w}}}
\newcommand{\frakx}{\ensuremath{\mathfrak{x}}}
\newcommand{\fraky}{\ensuremath{\mathfrak{y}}}
\newcommand{\frakz}{\ensuremath{\mathfrak{z}}}
\newcommand{\fraksl}{\ensuremath{\mathfrak{sl}}}
\newcommand{\frakso}{\ensuremath{\mathfrak{so}}}
\newcommand{\fraksp}{\ensuremath{\mathfrak{sp}}}

%%%%%%%%%%%%  Calligraphic, Roman and Maths integers %%%%%%%%%%%%%%%%%%

\newcommand{\cA}{\mathcal{A}}
\newcommand{\cB}{\mathcal{B}}
\newcommand{\cC}{\mathcal{C}}
\newcommand{\cD}{\mathcal{D}}
\newcommand{\cE}{\mathcal{E}}
\newcommand{\cF}{\mathcal{F}}
\newcommand{\cG}{\mathcal{G}}
\newcommand{\cH}{\mathcal{H}}
\newcommand{\cI}{\mathcal{I}}
\newcommand{\cJ}{\mathcal{J}}
\newcommand{\cK}{\mathcal{K}}
\newcommand{\cL}{\mathcal{L}}
\newcommand{\cM}{\mathcal{M}}
\newcommand{\cN}{\mathcal{N}}
\newcommand{\cO}{\mathcal{O}}
\newcommand{\cP}{\mathcal{P}}
\newcommand{\cQ}{\mathcal{Q}}
\newcommand{\cS}{\mathcal{S}}
\newcommand{\cU}{\mathcal{U}}
\newcommand{\cX}{\mathcal{X}}
\newcommand{\cY}{\mathcal{Y}}
\newcommand{\cV}{\mathcal{V}}
\newcommand{\cW}{\mathcal{W}}
\newcommand{\cR}{\mathcal{R}}
\newcommand{\cT}{\mathcal{T}}
\newcommand{\cZ}{\mathcal{Z}}


%%%%%%%%%%%% mathsf%%%%%%%%%%%%%%%%%%


\newcommand{\sfA}{\ensuremath{\mathsf{A}}}
\newcommand{\sfB}{\ensuremath{\mathsf{B}}}
\newcommand{\sfC}{\ensuremath{\mathsf{C}}}
\newcommand{\sfD}{\ensuremath{\mathsf{D}}}
\newcommand{\sfE}{\ensuremath{\mathsf{E}}}
\newcommand{\sfF}{\ensuremath{\mathsf{F}}}
\newcommand{\sfG}{\ensuremath{\mathsf{G}}}
\newcommand{\sfH}{\ensuremath{\mathsf{H}}}
\newcommand{\sfJ}{\ensuremath{\mathsf{J}}}
\newcommand{\sfK}{\ensuremath{\mathsf{K}}}
\newcommand{\sfL}{\ensuremath{\mathsf{L}}}
\newcommand{\sfM}{\ensuremath{\mathsf{M}}}
\newcommand{\sfN}{\ensuremath{\mathsf{N}}}
\newcommand{\sfO}{\ensuremath{\mathsf{O}}}
\newcommand{\sfP}{\ensuremath{\mathsf{P}}}
\newcommand{\sfQ}{\ensuremath{\mathsf{Q}}}
\newcommand{\sfS}{\ensuremath{\mathsf{S}}}
\newcommand{\sfU}{\ensuremath{\mathsf{U}}}
\newcommand{\sfX}{\ensuremath{\mathsf{X}}}
\newcommand{\sfY}{\ensuremath{\mathsf{Y}}}
\newcommand{\sfW}{\ensuremath{\mathsf{W}}}
\newcommand{\sfR}{\ensuremath{\mathsf{R}}}
\newcommand{\sfT}{\ensuremath{\mathsf{T}}}
\newcommand{\sfZ}{\ensuremath{\mathsf{Z}}}

%%%%%%%%%%%%  Special letters for Lie groups %%%%%%%%%%%%%%%%%%

\newcommand{\biA}{{\mathbi{A}}}
\newcommand{\biB}{{\mathbi{B}}}
\newcommand{\biC}{{\mathbi{C}}}
\newcommand{\biD}{{\mathbi{D}}}
\newcommand{\biE}{{\mathbi{E}}}
\newcommand{\biF}{{\mathbi{F}}}
\newcommand{\biG}{{\mathbi{G}}}
\newcommand{\biH}{{\mathbi{H}}}
\newcommand{\biI}{{\mathbi{I}}}
\newcommand{\biJ}{{\mathbi{J}}}
\newcommand{\biK}{{\mathbi{K}}}
\newcommand{\biL}{{\mathbi{L}}}
\newcommand{\biM}{{\mathbi{M}}}
\newcommand{\biN}{{\mathbi{N}}}
\newcommand{\biO}{{\mathbi{O}}}
\newcommand{\biP}{{\mathbi{P}}}
\newcommand{\biQ}{{\mathbi{Q}}}
\newcommand{\biS}{{\mathbi{S}}}
\newcommand{\biU}{{\mathbi{U}}}
\newcommand{\biX}{{\mathbi{X}}}
\newcommand{\biY}{{\mathbi{Y}}}
\newcommand{\biV}{{\mathbi{V}}}
\newcommand{\biW}{{\mathbi{W}}}
\newcommand{\biR}{{\mathbi{R}}}
\newcommand{\biT}{{\mathbi{T}}}
\newcommand{\biZ}{{\mathbi{Z}}}




%%%%%%%%%%%%%%%%%%%%%%%%%%%%%%%%%%%%%%%%%%%%%%%%%%%%%%%%%%%%%%%%
\newcommand{\SU}{\mathrm{SU}}
\newcommand{\SO}{\mathrm{SO}}
\newcommand{\SL}{\mathrm{SL}}
\newcommand{\Sp}{\mathrm{Sp}}
\newcommand{\U}{\mathrm{U}}
\newcommand{\ul}{\mathrm{u}}
\newcommand{\Spin}{\mathrm{Spin}}
\newcommand{\Pin}{\mathrm{Pin}}
\newcommand{\PSL}{\mathrm{PSL}}
%%%%%%%%%%%%%%%%%%%%%%%%%%%%%%%%%%%%%%%%%%%%%%%%%%%%%%%%%%%%%%%%




\def \be  {\begin{equation}}
\def \ee  {\end{equation}}
\def \bea {\begin{equation}\begin{aligned}}
\def \eea {\end{aligned}\end{equation}}
\def \ba  {\begin{eqnarray}}
\def \ea  {\end{eqnarray}}

\begin{document}\thispagestyle{empty}

\centerline{\Large \bf Homework 5: Due at class on May 11}


\section{Anomaly from triangle diagrams}
In the lecture, we have derived the chiral anomaly a \'{l}a Fujikawa. In this exercise, we will take another approach to explore the chiral anomaly and compute the decay width of the neutral pion.
\begin{figure}[ht]
    \centering
    \includegraphics[width=0.7\textwidth]{triangle.pdf}\label{triangle}
    \caption{Diagrams contributing to the two-photon matrix element of the divergence of the axial vector current.}
\end{figure}

% In a classical field theory with vanishing quark masses we have a chiral symmetry implying that the axial current is conserved, $\partial_{\mu} j^{5 \mu}=0$. This symmetry forbids the neutral pion to decay into two photons, $\pi_{0} \rightarrow 2\gamma$. However, experiment has shown that the favored decay channel of the neutral pion is the one into two photons. Therefore we must go beyond the classical level and we find that at the quantum level chiral symmetry is broken,
% i.e. $\partial_{\mu} j^{5 \mu} \neq 0$. This situation in which a classical symmetry gets broken after quantization is called an anomaly. In this exercise we want to explore the chiral anomaly and compute the decay width of the neutral pion.

\subsection{}
Using the Feynman rules you have witnessed so far, write down the amplitude for the process in Figure \ref{triangle} (left). Insert for the axial vector current $\gamma^{\mu} \gamma^{5}$. Make sure you arrive at
\begin{equation}\label{eq.(1)}
\raisebox{-.7cm}{\includegraphics[width=2cm]{triangle2}}=(-1)(-i e)^{2} \int \frac{d^{4} \ell}{(2 \pi)^{4}} \operatorname{tr}\left[\gamma^{\mu} \gamma^{5} \frac{i(\slashed\ell-\slashed k)}{(\ell-k)^{2}} \gamma^{\lambda} \frac{i \slashed \ell}{\ell^{2}} \gamma^{\nu} \frac{i(\slashed \ell+\slashed p)}{(\ell+p)^{2}}\right]
\end{equation}

\subsection{}
Contracting \eqref{eq.(1)} with $iq_{\mu}$ and shifting the internal momentum $\ell\to \ell+k$,  arrive at
\begin{equation}\label{eq.(3)}
iq_\mu \raisebox{-.7cm}{\includegraphics[width=2cm]{triangle2}}=e^{2} \int \frac{d^{4} \ell}{(2 \pi)^{4}} \operatorname{tr}\left[\gamma^{5} \frac{\slashed\ell}{\ell^{2}} \gamma^{\lambda} \frac{(\slashed \ell+\slashed k)}{(\ell+k)^{2}} \gamma^{\nu}-\gamma^{5} \frac{\slashed\ell}{\ell^{2}} \gamma^{\nu} \frac{(\slashed \ell+\slashed p)}{(\ell+p)^{2}} \gamma^{\lambda}\right]
\end{equation}



\subsection{}
As evident, \eqref{eq.(3)} is antisymmetric under the interchange of $\left(p, \nu\right)$ and $\left(k, \lambda\right)$. Consequently, the contribution from the second diagram in Figure \ref{triangle} (right) is precisely canceled and one would expect the amplitude to vanish. However, since we have shifted the integration variable, we need further treatment! To do so we use a procedure called dimensional regularization. The basic idea is to assume that the loop momentum $\ell$ has higher dimensional components while the external momenta $p$ and $k$ remain 4 -dimensional. We introduce
\begin{equation}
\ell=\ell_{\|}+\ell_{\perp}
\end{equation}
with $\ell_{\|}$ being the 4D Minkowski part and $\ell_{\perp}$ being the higher dimensional remnant.
Using the the fact that $\gamma^{5}$ commutes with $\gamma^{\mu}$ in the extra-dimensions (i.e. $\mu>3$ ) show that the divergence of \eqref{eq.(1)} gets an additional contribution.
\begin{equation}
iq_\mu \raisebox{-.7cm}{\includegraphics[width=2cm]{triangle2}}=e^{2} \int \frac{d^{4} \ell}{(2 \pi)^{4}} \operatorname{tr}\left[-2 \gamma^{5} \slashed\ell_{\perp} \frac{(\slashed\ell-\slashed k)}{(\ell-k)^{2}} \gamma^{\lambda} \frac{\slashed\ell}{\ell^{2}} \gamma^{\nu} \frac{(\slashed \ell+\slashed p)}{(\ell+p)^{2}}\right]
\end{equation}


\subsection{}
As usual, introducing Feynman parameters and integrate over the loop momentum and the Feynman parameters, show that the divergence of the amplitude is,
\begin{equation}\label{eq.(7)}
iq_\mu \raisebox{-.7cm}{\includegraphics[width=2cm]{triangle2}}
=\frac{e^{2}}{4 \pi^{2}} \epsilon^{\alpha \lambda \beta \nu} k_{\alpha} p_{\beta}
\end{equation}
Obviously \eqref{eq.(7)} is symmetric under the interchange of $\left(p, \nu\right)$ and $\left(k, \lambda\right)$ and therefore the second diagram in Figure \ref{triangle} (left) gives an equal contribution.



\subsection{}
It can be shown that the current $J^\mu_5$ associated to a broken symmetry has a non-zero probability to create a Goldstone bosons out of the vacuum. In fact, pions can be identified with the Goldstone boson of the chiral symmetry breaking so that
the chiral anomaly implies that the neutral pion can decay into into two photons.
To obtain the amplitude for the decay $\pi^{0} \rightarrow 2 \gamma$ from the triangle diagrams Figure \ref{triangle}, we can use our result
\eqref{eq.(7)}. We make the ansatz
$$
\mathrm{i} \Pi=\mathrm{i} \frac{e^{2}}{4 \pi^{2}} \frac{1}{f_{\pi}} \varepsilon_{\nu}^{(i) *} \varepsilon_{\lambda}^{(j) *} \epsilon^{\nu \lambda \alpha \beta} k_{\alpha} p_{ \beta}
$$
where $\varepsilon^{(i) *}$ are the polarization vectors and $f_{\pi}$ denotes the pion decay constant which parameterizes the QCD effects.
This allows us to compute the width of the pion decay:
$$
\Gamma\left(\pi^{0} \rightarrow 2 \gamma\right)=\frac{1}{2 m_{\pi}} \frac{1}{8 \pi} \frac{1}{2} \sum_{\text {polarization }}\left|\Pi\left(\pi^{0} \rightarrow 2 \gamma\right)\right|^{2}
$$
where $m_{\pi}$ denotes the mass of the neutral pion and the factor $1 / 2$ is due to phase space of identical particles. Useful data: $m_{\pi}=134.976 \mathrm{MeV}$ and $f_{\pi}=93 \mathrm{MeV}$. Compare your result to the experimentally measured value $\Gamma^{(\exp )}=7.93 \mathrm{eV}$ and/or the mean life-time $\tau^{(e x p)}=8.4 \times 10^{-17} \mathrm{s}$
% Hint: Go to the rest frame of the pion and use momentum conservation and Lorentz invariance to determine the form of $p_{2, \alpha}$ and $p_{1, \beta}$. Remember that the photon has two polarization states which are orthogonal to its momentum. Recall: $\alpha=\frac{e^{2}}{4 \pi} \simeq \frac{1}{137}$






\section{Spontaneous symmetry breaking in linear sigma model}



As an application of spontaneous symmetry breaking, we want to have a look at the linear sigma model which consists of $N$ real scalar fields with the Lagrangian
$$
\mathcal{L}=\frac{1}{2} \partial_{\mu} \phi^{i} \partial^{\mu} \phi^{i}-\frac{1}{2} \mu^{2} \phi^{i} \phi^{i}-\frac{\lambda}{4}\left(\phi^{i} \phi^{i}\right)^{2}, \quad \text { sum over } i=1, \ldots, N
$$
with $\lambda>0$ and $\mu^{2}<0$



\subsection{} Let us find the symmetry group of the Lagrangian: We transform the fields $\phi^{i} \mapsto R^{i j} \phi^{j}$. What kind of matrices $R$ are allowed such that $\mathcal{L}$ remains invariant?

\subsection{}  Find the minimum $\phi_{0}^{i}$ of the potential. You will find that the minimum is any
$\phi_{0}^{i}$ that fulfills
$$
\sum_{i} \phi_{0}^{i} \phi_{0}^{i}=-\frac{\mu^{2}}{\lambda}
$$
This condition determines only the length of the "vector" $\phi_{0}^{i}$. We choose coordinates such that $\phi_{0}^{i}$ points into the $N$-th direction:
$$
\phi_{0}^{i}(x)=(0,0, \ldots, 0, v), \quad v=\sqrt{-\frac{\mu^{2}}{\lambda}}
$$


\subsection{} Now we break the symmetry by defining a set of shifted fields
$$
\phi^{i}(x):=\left(\pi^{k}(x), v+\eta(x)\right), \quad k=1, . ., N-1
$$
Rewrite the Lagrangian in terms of the $\pi$ and $\eta$ fields.


\subsection{} After spontaneous symmetry breaking, how many massive and massless fields are there? What is the symmetry of the new Lagrangian? Compare your result to Goldstone's Theorem which says that for every spontaneously broken continuous symmetry, the theory must contain a massless particle.






\section{Abelian Higgs model}
The Abelian Higgs Model is a $U(1)$ gauge theory coupled to a Klein-Gordon field with a symmetry breaking potential. The Lagrangian is given
by $$\mathcal{L}=-\frac{1}{4} F_{\mu \nu} F^{\mu \nu}+\left(D_{\mu} \phi\right)^{*} D^{\mu} \phi-V(\phi)$$
with $D_{\mu} \phi=\left(\partial_{\mu}+i e A_{\mu}\right) \phi$ and the same potential term as in the Abelian Sigma Model.

\subsection{}
We expand the field $\phi$ around its minimum as follows:
$$
\phi=\frac1{\sqrt{2}}\left(v+H\right) e^{i \theta(x) /v}
$$
Taking the unitary gauge, write down the entire Lagrangian density in terms of the gauge field $$B_{\mu}:=A_\mu+\frac1e \frac{\partial_\mu \theta}{v}$$ and the "Higgs field" $H$ and identify their masses.

\subsection{}
Show that the propagator of the gauge field is given by
$$
\frac{i}{q^{2}-m^{2}}\left(-g_{\mu \nu}+\frac{q_{\mu} q_{\nu}}{m^{2}}\right)
$$




\end{document}
