\documentclass[12pt,a4paper]{article}
\usepackage{hyperref} % Use the Charter font for the document text
%\usepackage[UTF8]{ctex}
\usepackage{fullpage}
\usepackage{amsfonts,amssymb,amsmath}

\usepackage{physics}
\usepackage{epsfig}
\usepackage{amsmath}
\usepackage{amssymb}
\usepackage{amsthm}
\usepackage{indentfirst}
\usepackage{xspace}
\usepackage{multirow}
\usepackage{hyperref}
\usepackage{xcolor}
\usepackage{verbatim}
\usepackage{subfigure}
\usepackage{mathrsfs}
\usepackage{bbm}


%\hypersetup{colorlinks=true,urlcolor=darkred,linkcolor=darkred,citecolor=darkred}
%\usepackage{verbatim}
\usepackage[letterpaper,margin=0.9in,headheight=15pt]{geometry}
\usepackage{mathpazo}
\usepackage{authblk}
\usepackage{empheq}
\usepackage{feynmp}
\usepackage{graphicx}
\usepackage[matrix,arrow]{xy}
\usepackage{young}
\usepackage[vcentermath]{youngtab}
\usepackage{slashed}
%\usepackage{fontds}
%
\usepackage{bbm}
\usepackage{youngtab}
\usepackage{rotfloat}
\usepackage{stmaryrd}
\usepackage{amsfonts,amssymb,amsmath}
\usepackage{tikz-cd}
\usepackage{thmtools}
\usepackage{dashrule}
\usepackage[missing=]{gitinfo2}
\usepackage{fancyhdr}
\usepackage{mdframed}
\usepackage{booktabs}
\usepackage{subfiles}
\usepackage{simplewick}

\usepackage[utf8]{inputenc}


%%%%%%%%%%%% math fonts %%%%%%%%%%%%%%%%%%%%%%%%%%%%%%%%%%%%%
%
%---------- mathbb font --------------------------------
%

\newcommand{\bA}{\ensuremath{\mathbb{A}}}
\newcommand{\bB}{\ensuremath{\mathbb{B}}}
\newcommand{\bC}{\ensuremath{\mathbb{C}}}
\newcommand{\bD}{\ensuremath{\mathbb{D}}}
\newcommand{\bE}{\ensuremath{\mathbb{E}}}
\newcommand{\bF}{\ensuremath{\mathbb{F}}}
\newcommand{\bG}{\ensuremath{\mathbb{G}}}
\newcommand{\bH}{\ensuremath{\mathbb{H}}}
\newcommand{\bI}{\ensuremath{\mathbb{I}}}
\newcommand{\bJ}{\ensuremath{\mathbb{J}}}
\newcommand{\bK}{\ensuremath{\mathbb{K}}}
\newcommand{\bL}{\ensuremath{\mathbb{L}}}
\newcommand{\bM}{\ensuremath{\mathbb{M}}}
\newcommand{\bN}{\ensuremath{\mathbb{N}}}
\newcommand{\bO}{\ensuremath{\mathbb{O}}}
\newcommand{\bP}{\ensuremath{\mathbb{P}}}
\newcommand{\bQ}{\ensuremath{\mathbb{Q}}}
\newcommand{\bR}{\ensuremath{\mathbb{R}}}
\newcommand{\bS}{\ensuremath{\mathbb{S}}}
\newcommand{\bT}{\ensuremath{\mathbb{T}}}
\newcommand{\bU}{\ensuremath{\mathbb{U}}}
\newcommand{\bV}{\ensuremath{\mathbb{V}}}
\newcommand{\bW}{\ensuremath{\mathbb{W}}}
\newcommand{\bX}{\ensuremath{\mathbb{X}}}
\newcommand{\bY}{\ensuremath{\mathbb{Y}}}
\newcommand{\bZ}{\ensuremath{\mathbb{Z}}}



%
%---------- mathbf font --------------------------------
%


\newcommand{\bfA}{\ensuremath{\mathbf{A}}}
\newcommand{\bfB}{\ensuremath{\mathbf{B}}}
\newcommand{\bfC}{\ensuremath{\mathbf{C}}}
\newcommand{\bfD}{\ensuremath{\mathbf{D}}}
\newcommand{\bfE}{\ensuremath{\mathbf{E}}}
\newcommand{\bfF}{\ensuremath{\mathbf{F}}}
\newcommand{\bfG}{\ensuremath{\mathbf{G}}}
\newcommand{\bfH}{\ensuremath{\mathbf{H}}}
\newcommand{\bfI}{\ensuremath{\mathbf{I}}}
\newcommand{\bfJ}{\ensuremath{\mathbf{J}}}
\newcommand{\bfK}{\ensuremath{\mathbf{K}}}
\newcommand{\bfL}{\ensuremath{\mathbf{L}}}
\newcommand{\bfM}{\ensuremath{\mathbf{M}}}
\newcommand{\bfN}{\ensuremath{\mathbf{N}}}
\newcommand{\bfO}{\ensuremath{\mathbf{O}}}
\newcommand{\bfP}{\ensuremath{\mathbf{P}}}
\newcommand{\bfQ}{\ensuremath{\mathbf{Q}}}
\newcommand{\bfR}{\ensuremath{\mathbf{R}}}
\newcommand{\bfS}{\ensuremath{\mathbf{S}}}
\newcommand{\bfT}{\ensuremath{\mathbf{T}}}
\newcommand{\bfU}{\ensuremath{\mathbf{U}}}
\newcommand{\bfV}{\ensuremath{\mathbf{V}}}
\newcommand{\bfW}{\ensuremath{\mathbf{W}}}
\newcommand{\bfX}{\ensuremath{\mathbf{X}}}
\newcommand{\bfY}{\ensuremath{\mathbf{Y}}}
\newcommand{\bfZ}{\ensuremath{\mathbf{Z}}}




%
%---------- mathcal font -----------------------------
%

\newcommand{\scA}{\ensuremath{\mathscr{A}}}
\newcommand{\scB}{\ensuremath{\mathscr{B}}}
\newcommand{\scC}{\ensuremath{\mathscr{C}}}
\newcommand{\scD}{\ensuremath{\mathscr{D}}}
\newcommand{\scE}{\ensuremath{\mathscr{E}}}
\newcommand{\scF}{\ensuremath{\mathscr{F}}}
\newcommand{\scG}{\ensuremath{\mathscr{G}}}
\newcommand{\scH}{\ensuremath{\mathscr{H}}}
\newcommand{\scI}{\ensuremath{\mathscr{I}}}
\newcommand{\scJ}{\ensuremath{\mathscr{J}}}
\newcommand{\scK}{\ensuremath{\mathscr{K}}}
\newcommand{\scL}{\ensuremath{\mathscr{L}}}
\newcommand{\scM}{\ensuremath{\mathscr{M}}}
\newcommand{\scN}{\ensuremath{\mathscr{N}}}
\newcommand{\scO}{\ensuremath{\mathscr{O}}}
\newcommand{\scP}{\ensuremath{\mathscr{P}}}
\newcommand{\scQ}{\ensuremath{\mathscr{Q}}}
\newcommand{\scR}{\ensuremath{\mathscr{R}}}
\newcommand{\scS}{\ensuremath{\mathscr{S}}}
\newcommand{\scT}{\ensuremath{\mathscr{T}}}
\newcommand{\scU}{\ensuremath{\mathscr{U}}}
\newcommand{\scV}{\ensuremath{\mathscr{V}}}
\newcommand{\scW}{\ensuremath{\mathscr{W}}}
\newcommand{\scX}{\ensuremath{\mathscr{X}}}
\newcommand{\scY}{\ensuremath{\mathscr{Y}}}
\newcommand{\scZ}{\ensuremath{\mathscr{Z}}}

%
%---------- mathfrak font -----------------------------
%

\newcommand{\frakA}{\ensuremath{\mathfrak{A}}}
\newcommand{\frakB}{\ensuremath{\mathfrak{B}}}
\newcommand{\frakC}{\ensuremath{\mathfrak{C}}}
\newcommand{\frakD}{\ensuremath{\mathfrak{D}}}
\newcommand{\frakE}{\ensuremath{\mathfrak{E}}}
\newcommand{\frakF}{\ensuremath{\mathfrak{F}}}
\newcommand{\frakG}{\ensuremath{\mathfrak{G}}}
\newcommand{\frakH}{\ensuremath{\mathfrak{H}}}
\newcommand{\frakI}{\ensuremath{\mathfrak{I}}}
\newcommand{\frakJ}{\ensuremath{\mathfrak{J}}}
\newcommand{\frakK}{\ensuremath{\mathfrak{K}}}
\newcommand{\frakL}{\ensuremath{\mathfrak{L}}}
\newcommand{\frakM}{\ensuremath{\mathfrak{M}}}
\newcommand{\frakN}{\ensuremath{\mathfrak{N}}}
\newcommand{\frakO}{\ensuremath{\mathfrak{O}}}
\newcommand{\frakP}{\ensuremath{\mathfrak{P}}}
\newcommand{\frakQ}{\ensuremath{\mathfrak{Q}}}
\newcommand{\frakR}{\ensuremath{\mathfrak{R}}}
\newcommand{\frakS}{\ensuremath{\mathfrak{S}}}
\newcommand{\frakT}{\ensuremath{\mathfrak{T}}}
\newcommand{\frakU}{\ensuremath{\mathfrak{U}}}
\newcommand{\frakV}{\ensuremath{\mathfrak{V}}}
\newcommand{\frakW}{\ensuremath{\mathfrak{W}}}
\newcommand{\frakX}{\ensuremath{\mathfrak{X}}}
\newcommand{\frakY}{\ensuremath{\mathfrak{Y}}}
\newcommand{\frakZ}{\ensuremath{\mathfrak{Z}}}
\newcommand{\fraka}{\ensuremath{\mathfrak{a}}}
\newcommand{\frakb}{\ensuremath{\mathfrak{b}}}
\newcommand{\frakc}{\ensuremath{\mathfrak{c}}}
\newcommand{\frakd}{\ensuremath{\mathfrak{d}}}
\newcommand{\frake}{\ensuremath{\mathfrak{e}}}
\newcommand{\frakf}{\ensuremath{\mathfrak{f}}}
\newcommand{\frakg}{\ensuremath{\mathfrak{g}}}
\newcommand{\frakh}{\ensuremath{\mathfrak{h}}}
\newcommand{\fraki}{\ensuremath{\mathfrak{i}}}
\newcommand{\frakj}{\ensuremath{\mathfrak{j}}}
\newcommand{\frakk}{\ensuremath{\mathfrak{k}}}
\newcommand{\frakl}{\ensuremath{\mathfrak{l}}}
\newcommand{\frakm}{\ensuremath{\mathfrak{m}}}
\newcommand{\frakn}{\ensuremath{\mathfrak{n}}}
\newcommand{\frako}{\ensuremath{\mathfrak{o}}}
\newcommand{\frakp}{\ensuremath{\mathfrak{p}}}
\newcommand{\frakq}{\ensuremath{\mathfrak{q}}}
\newcommand{\frakr}{\ensuremath{\mathfrak{r}}}
\newcommand{\fraks}{\ensuremath{\mathfrak{s}}}
\newcommand{\frakt}{\ensuremath{\mathfrak{t}}}
\newcommand{\fraku}{\ensuremath{\mathfrak{u}}}
\newcommand{\frakv}{\ensuremath{\mathfrak{v}}}
\newcommand{\frakw}{\ensuremath{\mathfrak{w}}}
\newcommand{\frakx}{\ensuremath{\mathfrak{x}}}
\newcommand{\fraky}{\ensuremath{\mathfrak{y}}}
\newcommand{\frakz}{\ensuremath{\mathfrak{z}}}
\newcommand{\fraksl}{\ensuremath{\mathfrak{sl}}}
\newcommand{\frakso}{\ensuremath{\mathfrak{so}}}
\newcommand{\fraksp}{\ensuremath{\mathfrak{sp}}}

%%%%%%%%%%%%  Calligraphic, Roman and Maths integers %%%%%%%%%%%%%%%%%%

\newcommand{\cA}{\mathcal{A}}
\newcommand{\cB}{\mathcal{B}}
\newcommand{\cC}{\mathcal{C}}
\newcommand{\cD}{\mathcal{D}}
\newcommand{\cE}{\mathcal{E}}
\newcommand{\cF}{\mathcal{F}}
\newcommand{\cG}{\mathcal{G}}
\newcommand{\cH}{\mathcal{H}}
\newcommand{\cI}{\mathcal{I}}
\newcommand{\cJ}{\mathcal{J}}
\newcommand{\cK}{\mathcal{K}}
\newcommand{\cL}{\mathcal{L}}
\newcommand{\cM}{\mathcal{M}}
\newcommand{\cN}{\mathcal{N}}
\newcommand{\cO}{\mathcal{O}}
\newcommand{\cP}{\mathcal{P}}
\newcommand{\cQ}{\mathcal{Q}}
\newcommand{\cS}{\mathcal{S}}
\newcommand{\cU}{\mathcal{U}}
\newcommand{\cX}{\mathcal{X}}
\newcommand{\cY}{\mathcal{Y}}
\newcommand{\cV}{\mathcal{V}}
\newcommand{\cW}{\mathcal{W}}
\newcommand{\cR}{\mathcal{R}}
\newcommand{\cT}{\mathcal{T}}
\newcommand{\cZ}{\mathcal{Z}}


%%%%%%%%%%%% mathsf%%%%%%%%%%%%%%%%%%


\newcommand{\sfA}{\ensuremath{\mathsf{A}}}
\newcommand{\sfB}{\ensuremath{\mathsf{B}}}
\newcommand{\sfC}{\ensuremath{\mathsf{C}}}
\newcommand{\sfD}{\ensuremath{\mathsf{D}}}
\newcommand{\sfE}{\ensuremath{\mathsf{E}}}
\newcommand{\sfF}{\ensuremath{\mathsf{F}}}
\newcommand{\sfG}{\ensuremath{\mathsf{G}}}
\newcommand{\sfH}{\ensuremath{\mathsf{H}}}
\newcommand{\sfJ}{\ensuremath{\mathsf{J}}}
\newcommand{\sfK}{\ensuremath{\mathsf{K}}}
\newcommand{\sfL}{\ensuremath{\mathsf{L}}}
\newcommand{\sfM}{\ensuremath{\mathsf{M}}}
\newcommand{\sfN}{\ensuremath{\mathsf{N}}}
\newcommand{\sfO}{\ensuremath{\mathsf{O}}}
\newcommand{\sfP}{\ensuremath{\mathsf{P}}}
\newcommand{\sfQ}{\ensuremath{\mathsf{Q}}}
\newcommand{\sfS}{\ensuremath{\mathsf{S}}}
\newcommand{\sfU}{\ensuremath{\mathsf{U}}}
\newcommand{\sfX}{\ensuremath{\mathsf{X}}}
\newcommand{\sfY}{\ensuremath{\mathsf{Y}}}
\newcommand{\sfW}{\ensuremath{\mathsf{W}}}
\newcommand{\sfR}{\ensuremath{\mathsf{R}}}
\newcommand{\sfT}{\ensuremath{\mathsf{T}}}
\newcommand{\sfZ}{\ensuremath{\mathsf{Z}}}

%%%%%%%%%%%%  Special letters for Lie groups %%%%%%%%%%%%%%%%%%

\newcommand{\biA}{{\mathbi{A}}}
\newcommand{\biB}{{\mathbi{B}}}
\newcommand{\biC}{{\mathbi{C}}}
\newcommand{\biD}{{\mathbi{D}}}
\newcommand{\biE}{{\mathbi{E}}}
\newcommand{\biF}{{\mathbi{F}}}
\newcommand{\biG}{{\mathbi{G}}}
\newcommand{\biH}{{\mathbi{H}}}
\newcommand{\biI}{{\mathbi{I}}}
\newcommand{\biJ}{{\mathbi{J}}}
\newcommand{\biK}{{\mathbi{K}}}
\newcommand{\biL}{{\mathbi{L}}}
\newcommand{\biM}{{\mathbi{M}}}
\newcommand{\biN}{{\mathbi{N}}}
\newcommand{\biO}{{\mathbi{O}}}
\newcommand{\biP}{{\mathbi{P}}}
\newcommand{\biQ}{{\mathbi{Q}}}
\newcommand{\biS}{{\mathbi{S}}}
\newcommand{\biU}{{\mathbi{U}}}
\newcommand{\biX}{{\mathbi{X}}}
\newcommand{\biY}{{\mathbi{Y}}}
\newcommand{\biV}{{\mathbi{V}}}
\newcommand{\biW}{{\mathbi{W}}}
\newcommand{\biR}{{\mathbi{R}}}
\newcommand{\biT}{{\mathbi{T}}}
\newcommand{\biZ}{{\mathbi{Z}}}




%%%%%%%%%%%%%%%%%%%%%%%%%%%%%%%%%%%%%%%%%%%%%%%%%%%%%%%%%%%%%%%%
\newcommand{\SU}{\mathrm{SU}}
\newcommand{\SO}{\mathrm{SO}}
\newcommand{\SL}{\mathrm{SL}}
\newcommand{\Sp}{\mathrm{Sp}}
\newcommand{\U}{\mathrm{U}}
\newcommand{\ul}{\mathrm{u}}
\newcommand{\Spin}{\mathrm{Spin}}
\newcommand{\Pin}{\mathrm{Pin}}
\newcommand{\PSL}{\mathrm{PSL}}
%%%%%%%%%%%%%%%%%%%%%%%%%%%%%%%%%%%%%%%%%%%%%%%%%%%%%%%%%%%%%%%%




\def \be  {\begin{equation}}
\def \ee  {\end{equation}}
\def \bea {\begin{equation}\begin{aligned}}
\def \eea {\end{aligned}\end{equation}}
\def \ba  {\begin{eqnarray}}
\def \ea  {\end{eqnarray}}

\begin{document}\thispagestyle{empty}

\centerline{\Large \bf Homework 1: Due at class on March 9}
\section{Path integrals}
\subsection{free particle}
The Lagrangian of a free particle is given by
\begin{equation}
L(\dot{x}, x):=\frac{m}{2} \dot{x}^{2}~.
\end{equation}
Show that the propagator is given by
$$
\left\langle x_{f}\left|e^{-i H T}\right| x_{i}\right\rangle =\sqrt{\frac{m}{2 \pi \mathrm{i}  T}} \exp \left\{\frac{\mathrm{i} m}{2  T}\left(x_f-x_i\right)^{2}\right\}~,
$$
and they obey the covolution
$$
\int_{-\infty}^{\infty} \mathrm{d} x ~K\left(x_f, x, t_f-t\right) K\left(x, x_i, t-t_i\right)=K\left(x_f, x_i, t_f-t_i\right)~.
$$


\subsection{harmonic oscillator}
The Lagrangian of a harmonic oscillator with mass \(m>0\)
and frequency \(\omega>0\) is given by
\begin{equation}
L(\dot{x}, x):=\frac{m}{2} \dot{x}^{2}-\frac{m}{2} \omega^{2} x^{2}~.
\end{equation}
Show that the propagator is give by
\begin{equation}
\begin{aligned}
\left\langle x_{f}\left|e^{-i H T}\right| x_{i}\right\rangle &=\sqrt{\frac{m \omega}{2 \pi i \sin (\omega T)}} ~e^{i S_{\mathrm{cl}}} \\
S_{\mathrm{cl}} &=\frac{m \omega}{2} \frac{1}{\sin (\omega T)}\left[\cos (\omega T)\left(x_{f}^{2}+x_{i}^{2}\right)-2 x_{i} x_{f}\right]
\end{aligned}
\end{equation}
Note that the eigenfunctions of the harmonic oscillator is expressed by the Hermite polynomials $H_{n}(x)$, and they satisfy
$$\frac{1}{\sqrt{1-\rho^{2}}} \exp \left\{-\frac{\rho^{2}\left(x^{2}+y^{2}\right)-2 \rho x y}{1-\rho^{2}}\right\}=\sum_{n=0}^{\infty} \frac{(\rho / 2)^{n}}{n !} H_{n}(x) H_{n}(y)~.$$


\section{Peskin-Schroeder: Problem 9.2}
%
% \section{Functional determinant}
% The partition function of the fermion coupled to the background gauge field is given by
% $$
% Z\left( \eta, \overline{\eta}\right)=\int \mathcal{D}\psi \mathcal{D}\overline{\psi} e^{i \int d^{4} x\left(\overline{\psi}(i \slashed D-m) \psi+\overline{\eta} \psi+\overline{\psi} \eta\right)}~.
% $$
% Show that
% $$
% \log Z\left( \eta, \overline{\eta}\right)=\log\det(i \slashed D-m) - \int d^{4} x \overline{\eta} \frac{1}{i \slashed D-m} \eta
% $$
% and argue that they can be diagramatically interpreted as
% $$
% \log Z\left( \eta, \overline{\eta}\right)=1+\sum \raisebox{-1cm}{\includegraphics[width=3cm]{circle-photon.pdf}} - \sum \raisebox{-.5cm}{\includegraphics[width=4cm]{line-photon.pdf}}
% $$


%
%
% \section{Effective action}
% \subsection{}
% Let us consider the action of the scalar field with a potential $V(\phi)$
% $$S[\phi]=\int \mathrm{d}^{4} x\left[\frac{1}{2}\left(\partial_{\mu} x\right)^{2}-\frac{1}{2} m^{2} \phi^{2}-V(\phi)\right]~,$$
% and its partition function
% $$
% Z[J]=\int \mathcal{D} \phi e^{i S[\phi]}
% $$
% By expanding around the classical configuration (saddle point)
% $$\left.\frac{\delta S[\phi]}{\delta \phi}\right|_{\phi=\phi_{0}}=0~,$$
% show that the correction of the Gaussian fluctuation (1-loop) can be read off
% \begin{equation}
% Z[J]=\exp \left\{iS[\phi_{0}]-\frac{1}{2} \operatorname{Tr}[\log \hat{K}]\right\}
% \end{equation}
% where
% $$
% \hat{K}=-\left(\Box+m^{2}+V'(\phi)\right)~.
% $$
%
% \subsection{}
% For the $\phi^4$-theory with $V(\phi)=\frac{\lambda}{4 !} \phi^{4}$, the 1PI effective action \(\Gamma\left[\phi_{c}\right]=W[J]-\int d^{4} x J(x) \phi_{c}(x)\) admits the expansion
% $$
% \Gamma[\phi_c]=\Gamma_0[\phi_c]+(-i \lambda) \Gamma_1[\phi_c]+(-i \lambda)^{2}\Gamma_2[\phi_c]+\cdots
% $$
% where $\Gamma_0[\phi_c]=S[\phi_c]$ is the classical action at $\phi=\phi_c$.
% Determine $\Gamma_1[\phi_c]$ and find its diagramatical expression.


\section{Generating functional for QED}
The generating functional for QED is given by
$$
Z\left(J_{\mu}, \eta, \overline{\eta}\right)=\int \mathcal{D}A_{\mu} \mathcal{D}\psi \mathcal{D}\overline{\psi} e^{i \int d^{4} x\left(\mathcal{L}_{Q E D}+\mathcal{L}_{G F}+J^{\mu} A_{\mu}+\overline{\eta} \psi+\overline{\psi} \eta\right)}
$$
where
$$
{\mathcal{L}_{Q E D}=-\frac{1}{4} F_{\mu \nu} F^{\mu \nu}+\overline{\psi}(i \slashed D-m) \psi} \qquad
{\mathcal{L}_{G F}=-\frac{1}{2 \xi}(\partial \cdot A)^{2}}~.
$$
\subsection{} Determine \(Z_{0}\left[J^{\mu}, \eta, \overline{\eta}\right]\) so that
$$
Z\left[J^{\mu}, \eta, \overline{\eta}\right]=\exp \left\{(- e) \int d^{4} x \frac{\delta}{\delta \eta_{\alpha}(x)}\left(\gamma^{\mu}\right)_{\alpha \beta} \frac{\delta}{\delta \overline{\eta}_{\beta}(x)} \frac{\delta}{\delta J_{\mu}(x)}\right\} Z_{0}\left[J^{\mu}, \eta, \overline{\eta}\right]
$$

\subsection{} We perform the perturbative expansion
$$
Z=Z_{0}\left[1+(-i e) Z_{1}+(-i e)^{2} Z_{2}+\cdots\right]
$$
where we have subtracted the vacuum-vacuum amplitudes in \(Z_{i}\), namely  \(Z[0]=1\). Express $Z_{1}$ and $Z_{2}$ by using Feynman Diagrams.

%




\end{document}
