\documentclass[12pt,a4paper]{article}
\usepackage{hyperref} % Use the Charter font for the document text
%\usepackage[UTF8]{ctex}
\usepackage{fullpage}
\usepackage{amsfonts,amssymb,amsmath}


\newcommand{\bA}{\ensuremath{\mathbb{A}}}
\newcommand{\bB}{\ensuremath{\mathbb{B}}}
\newcommand{\bC}{\ensuremath{\mathbb{C}}}
\newcommand{\bD}{\ensuremath{\mathbb{D}}}
\newcommand{\bE}{\ensuremath{\mathbb{E}}}
\newcommand{\bF}{\ensuremath{\mathbb{F}}}
\newcommand{\bG}{\ensuremath{\mathbb{G}}}
\newcommand{\bH}{\ensuremath{\mathbb{H}}}
\newcommand{\bI}{\ensuremath{\mathbb{I}}}
\newcommand{\bJ}{\ensuremath{\mathbb{J}}}
\newcommand{\bK}{\ensuremath{\mathbb{K}}}
\newcommand{\bL}{\ensuremath{\mathbb{L}}}
\newcommand{\bM}{\ensuremath{\mathbb{M}}}
\newcommand{\bN}{\ensuremath{\mathbb{N}}}
\newcommand{\bO}{\ensuremath{\mathbb{O}}}
\newcommand{\bP}{\ensuremath{\mathbb{P}}}
\newcommand{\bQ}{\ensuremath{\mathbb{Q}}}
\newcommand{\bR}{\ensuremath{\mathbb{R}}}
\newcommand{\bS}{\ensuremath{\mathbb{S}}}
\newcommand{\bT}{\ensuremath{\mathbb{T}}}
\newcommand{\bU}{\ensuremath{\mathbb{U}}}
\newcommand{\bV}{\ensuremath{\mathbb{V}}}
\newcommand{\bW}{\ensuremath{\mathbb{W}}}
\newcommand{\bX}{\ensuremath{\mathbb{X}}}
\newcommand{\bY}{\ensuremath{\mathbb{Y}}}
\newcommand{\bZ}{\ensuremath{\mathbb{Z}}}



\newtheorem{lemma}{Lemma}[section]
\newtheorem{conjecture}[lemma]{Conjecture}
\newtheorem{corollary}[lemma]{Corollary}
\newtheorem{theorem}[lemma]{Theorem}
\newtheorem{definition}[lemma]{Definition}
\newtheorem{question}[lemma]{Question}
\newtheorem{proposition}[lemma]{Proposition}

\usepackage{graphicx}


\begin{document}\thispagestyle{empty}

\centerline{\Large \bf Homework 6: Due at class on April 12}

 \vspace{.5cm}
\noindent 1. Let $T^2=S^1\times S^1$ be the torus. Using the formula for de Rham cohomology  of a product space,
$$
H_{dR}^{k}(M\times N)\cong \bigoplus_{k=p+q} H_{dR}^p(M)\otimes H_{dR}^q(N)~,
$$
find de Rham cohomology  $H_{dR}^*(T^2)$. Find the Poincare dual of each generator of  $H_{dR}^*(T^2)$.


\vspace{.5cm}
\noindent 2. Derive all the curvature identities in (6.5) of the lecture note.






\vspace{.5cm}
\noindent 3. Let $S^2 \subset \bR^3$ be the 2-sphere with unit radius and the metric on $S^2$ is induced from the standard metric of $\bR^3$ as in Homework 5. Find geodesics on $S^2$ and compute its Riemann, Ricci and scalar curvature.



\vspace{.5cm}
\noindent 4.  Let $\mathbf{H}=\{(x,y)|y>0\}$ be the upper half plane and the metric is given by
\begin{equation}\label{UHP}
ds^2=\frac{dx^2+dy^2}{y^2}~.
\end{equation}
Find geodesics on $\mathbf{H}$ and compute its Riemann, Ricci and scalar curvature.



 \vspace{.5cm}
\noindent 5. Do parallel transport of a vector along a triangle $\Delta$PQR on a unit sphere (Figure below) and find the angle difference when it comes back. Note that the sphere has the standard metric and we consider the parallel transport with respect to the Levi-Civita connection. Compare it with the area of the triangle. Do the same exercise for a triangle on the upper half plane $\mathbf{H}$ with the metric \eqref{UHP}. Describe the difference between the sphere and the upper half plane.




\begin{figure}[h]\centering
 \includegraphics[width=5cm]{triangle}
\caption{A triangle on a 2-sphere}
\end{figure}





\vspace{.5cm}
\noindent 6. The \textbf{M\"obius transformation} of the upper half plane $\mathbf{H}=\{z=x+iy~|~y>0\}$  is a rational function of the form
$$f(z) = \frac{a z + b}{c z + d}~,$$
where $ad-bc=1$ with $a,b,c,d\in \bR$.  If $f_1$ and $f_2$ are M\"obius transformations, prove
that $f_1 \circ f_2$ is also a M\"obius transformation. Show that this is an isometry for the metric \eqref{UHP}.


%
%
%
%
% \vspace{.5cm}
% % \noindent 7. 
% 7.1 From the action, derive the geodesic equation
% 7.2 From the Einstein-Hilbert action, derive the Einstein equations.
% %
%





\end{document}
