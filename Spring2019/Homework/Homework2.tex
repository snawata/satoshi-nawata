\documentclass[12pt,a4paper]{article}
\usepackage{hyperref} % Use the Charter font for the document text
%\usepackage[UTF8]{ctex}
\usepackage{fullpage}
\usepackage{amsfonts,amssymb,amsmath}


\newcommand{\bA}{\ensuremath{\mathbb{A}}}
\newcommand{\bB}{\ensuremath{\mathbb{B}}}
\newcommand{\bC}{\ensuremath{\mathbb{C}}}
\newcommand{\bD}{\ensuremath{\mathbb{D}}}
\newcommand{\bE}{\ensuremath{\mathbb{E}}}
\newcommand{\bF}{\ensuremath{\mathbb{F}}}
\newcommand{\bG}{\ensuremath{\mathbb{G}}}
\newcommand{\bH}{\ensuremath{\mathbb{H}}}
\newcommand{\bI}{\ensuremath{\mathbb{I}}}
\newcommand{\bJ}{\ensuremath{\mathbb{J}}}
\newcommand{\bK}{\ensuremath{\mathbb{K}}}
\newcommand{\bL}{\ensuremath{\mathbb{L}}}
\newcommand{\bM}{\ensuremath{\mathbb{M}}}
\newcommand{\bN}{\ensuremath{\mathbb{N}}}
\newcommand{\bO}{\ensuremath{\mathbb{O}}}
\newcommand{\bP}{\ensuremath{\mathbb{P}}}
\newcommand{\bQ}{\ensuremath{\mathbb{Q}}}
\newcommand{\bR}{\ensuremath{\mathbb{R}}}
\newcommand{\bS}{\ensuremath{\mathbb{S}}}
\newcommand{\bT}{\ensuremath{\mathbb{T}}}
\newcommand{\bU}{\ensuremath{\mathbb{U}}}
\newcommand{\bV}{\ensuremath{\mathbb{V}}}
\newcommand{\bW}{\ensuremath{\mathbb{W}}}
\newcommand{\bX}{\ensuremath{\mathbb{X}}}
\newcommand{\bY}{\ensuremath{\mathbb{Y}}}
\newcommand{\bZ}{\ensuremath{\mathbb{Z}}}



\newtheorem{lemma}{Lemma}[section]
\newtheorem{conjecture}[lemma]{Conjecture}
\newtheorem{corollary}[lemma]{Corollary}
\newtheorem{theorem}[lemma]{Theorem}
\newtheorem{definition}[lemma]{Definition}
\newtheorem{question}[lemma]{Question}
\newtheorem{proposition}[lemma]{Proposition}

\usepackage{graphicx}


\begin{document}\thispagestyle{empty}

\centerline{\Large \bf Homework 2: Due at class on March 15}
\vspace{.5cm}

\noindent 1. For elements $x=(x^0,\cdots,x^n)$ and   $y=(y^0,\cdots,y^n)$ of $\mathbb{R}^{n+1}\backslash  \{0\}$ , we define an equivalence relation $x \sim y$ by
$$
x=\alpha y  \qquad \alpha \in \mathbb{R}~.
$$
Let us define $\mathbb{R}P^n$ by $(\mathbb{R}^{n+1}\backslash \{0\})/\sim$. Show that $\mathbb{R}P^n$ is a manifold and it is orientable if and only if $n$ is odd. The space is called a real projective space.
\vspace{.5cm}

\noindent 2. Let $(x,y)$ be the Cartesian coordinate of $\bR^2$ and $(r,\theta)$ be the polar coordinate of $\bR^2$. Write a vector field $X$ in terms of the Cartesian coordinate that generate a flow $\varphi_t:\bR^2\to\bR^2$
$$
\left(\begin{array}{c}
x\\y
\end{array}\right) \to\left(\begin{array}{cc}
\cos t& -\sin t\\
\sin t&\cos t
\end{array}\right) \left(\begin{array}{c}
x\\y
\end{array}\right)~.
$$
This is the rotation in $\bR^2$. In addition, draw the schematic picture of the vector field $X$.

\vspace{.5cm}

\noindent 3. Let $M_n(\mathbb{R})$ and $M_n(\mathbb{C})$ denote the set of all $n\times n$ matrices over $\bR$ and $\bC$, respectively. We define
\begin{align}
SU(2)&=\{A\in M_2(\bC)| \ A^\dagger A=\mathrm{Id} , \ \det A=1 \}\cr
SO(3)&=\{A\in M_3(\bR)| \ A^T A=\mathrm{Id} , \ \det A=1 \}~.\nonumber
\end{align}


3.1 Construct a double covering (2-to-1) map $SU(2)\to SO(3)$.

3.2 Show that $SU(2)$ is diffeomorphic to $S^3$ and $SO(3)$ is diffeomorphic to $\bR P^3$.
\vspace{.5cm}

\noindent 4. Let $e$ be the identity element of $SO(3)$. Show that the tangent space $T_e SO(3)$ at $e$ is spanned by tangent vectors of curves in $SO(3)$
$$
\exp(t J_i)=1+tJ_i+\frac12 (tJ_i)^2 +\cdots
$$
at $t=0$  where $J_i$ $(i=x,y,z)$ are defined by
$$
J_x=\left(
\begin{array}{ccc}
0&0&0\\
0&0&-1\\
0&1&0\\
\end{array}
\right) \qquad
J_x=\left(
\begin{array}{ccc}
0&0&1\\
0&0&0\\
-1&0&0\\
\end{array}
\right) \qquad
J_x=\left(
\begin{array}{ccc}
0&-1&0\\
1&0&0\\
0&0&0\\
\end{array}
\right) ~.
$$
Let us define the commutator by $[X, Y ] = XY - Y X$. Then, show that
$$
[J_x,J_y]=J_z~, \qquad [J_y,J_z]=J_x~, \qquad [J_z,J_x]=J_y~.
$$

\vspace{.5cm}

\noindent 5. Show that the tangent space $T_e SU(2)$ is spanned by $i\sigma_x$, $i\sigma_y$ and $i\sigma_z$ (the Pauli matrices by $i$).

\vspace{.5cm}
\noindent 6. Write down vector fields that generate the rotation along $x$-, $y$-, $z$-axis in $\bR^3$. Find the commutation relations of these vector fields. Compare the theory of angular momenta in quantum mechanics.

%\begin{definition}
%A group is a set of $G$ with a multiplication $\times $
%\begin{itemize}\setlength{\parskip}{-0.1cm}
%\item (Closure) $a,b\in G \rightarrow a\times b\in G$
%\item (Associativity)  For $a,b,c\in G$, $(a\times b)\times c=a\times (b\times c)$
%\item (Identity element) There exists an element $e$ in $G$ such that, for every element $a$ in $G$, the equation $e\times a = a\times e = a$ holds. Such an element is unique (see below), and thus one speaks of the identity element.
%\item (Inverse element) For each $a$ in $G$, there exists an element $b$ in $G$, commonly denoted $a^{-1}$, such that $a \times  b = b \times  a = e$, where e is the identity element.
%\end{itemize}
%\end{definition}
%
%\begin{definition}
%A group $G$ is called Lie group if it is a smooth manifold so that the group operations
%\begin{align}
%&m:G\times G \to G;\quad (a,b) \to a\times b \cr
%&\mathrm{inv}:G\to G;\quad a\to a^{-1}\nonumber
%\end{align}
%are smooth.
%\end{definition}
%
%\noindent 4. (Optional) Show that the following Lie groups are manifold and find their dimensions
%\begin{itemize}
%\item General linear group: $GL(n, \bC) = \{A \in M_n(\bC)| \det A \neq 0\}$
%\item Special linear group: $SL(n, \bC) = \{A \in GL(n, \bC)| \det A = 1\}$
%\item Unitary group $U(n) = \{A \in GL(n, \bC)| AA^\dagger = I\}$
%\item Special unitary group $SU(n) = U(n) \cap SL(n, \bC)$
%\end{itemize}

\end{document}
