\documentclass[geometry-lectures-19.tex]{subfiles}

\begin{document}




\section{Lie groups and Lie algebras}


\bdefn[Lie group]\index{Lie group}
  A \textbf{Lie group} is a manifold $G$ with a group structure such that multiplication $m: G \times G \to G$ and inverse $i: G \to G$ are smooth maps. The \textbf{dimension} of a Lie group $G$ is the dimension of the underlying manifold.
\edefn

  For each $h \in G$, we define the \textbf{left and right translation maps}
  \begin{align*}
    L_h: G &\to G;  g \mapsto hg~,\\
    R_h: G &\to G; g \mapsto gh~.
  \end{align*}
  These maps are indeed diffeomorphisms because they have smooth inverse $L_{h^{-1}}$ and $R_{h^{-1}}$ respectively.



Some of Lie groups will be given by subsets of the space $M_n(\bF)$ of  $n\times n$ matrices where $\bF=\bR$ or $\bC$ specified by certain algebraic equations. For example,
\begin{itemize}
\item  General linear group: $\GL(n, \bF)=\{A \in \mathrm{M}_n(\bF)| \det A \neq 0\}$
\item  Special linear group: $\SL(n, \bF)=\{A \in \GL(n, \bF)| \det A=1\}$
\item Symplectic group $\mathrm { Sp } ( n , \bF )=\{ A \in \GL(2n, \bF)| A^ { \mathrm { T } } J A = J \ \textrm{where} \  J=\begin{pmatrix} 0&I_n\\-I_n&0\end{pmatrix}\}$
\item  Orthogonal group $\mathrm{O}(n,\bF)= \{A \in \GL(n, \bF)| A^{T}A=I\}$
\item Special orthogonal group $\SO(n,\bF)=\{A \in \mathrm{O}(n,\bF)| \det A=1\}$
\item Unitary group $\U(n)=\{A \in \GL(n, \bC)| A^\dagger A=I\}$
\item Special unitary group $\SU(n) =\{A \in \U(n)| \det A=1\}$
\end{itemize}


\bdefn[Lie algebra]\index{Lie algebra}
  A \textbf{Lie algebra} $\mathfrak{g}$ is a vector space (over $\bF=\bR$ or $\bC$) with a \term{bracket}
  \[
    [\ph,\ph] : \mathfrak{g} \times \mathfrak{g} \to \mathfrak{g}
  \]
  satisfying
  \begin{enumerate}
      \item $[\alpha X + \beta Y, Z] = \alpha [X, Z] + \beta [Y, Z]$ for all $X, Y, Z \in \mathfrak{g}$ and $\alpha, \beta \in \bF$ \hfill(bilinearity)
    \item $[X, Y] = -[Y, X]$ for all $X, Y \in \mathfrak{g}$ \hfill(antisymmetry)
    \item $[X, [Y, Z]] + [Y, [Z, X]] + [Z, [X, Y]] = 0$ for all $X, Y, Z \in \mathfrak{g}$.\hfill(Jacobi identity\index{Jacobi identity})
  \end{enumerate}
  Note that linearity in the second argument follows from linearity in the first argument and antisymmetry.
\edefn



We now try to get a Lie algebra from a Lie group $G$, by considering ${T}_e(G)$.
  The tangent space of a Lie group $G$ at the identity naturally admits a Lie bracket
  \[
    [\ph, \ph]: T_e G \times T_e G \to T_e G; (X,Y) \mapsto [X,Y]=XY-YX
  \]
  such that
  \[
    \mathfrak{g} = (T_e(G), [\ph, \ph])
  \]
  is a Lie algebra.

\bdefn[Lie algebra of a Lie group]\index{Lie algebra of a Lie group}
  Let $G$ be a Lie group. The \textbf{Lie algebra} of $G$, written $\mathfrak{g}$, is the tangent space $T_e G$ under the natural Lie bracket.
\edefn
The general convention is that if the name of a Lie group is denoted by capital letters, then the corresponding Lie algebra is the same name with fraktur font. For example, the Lie group of $\SL(n,\bC)$ is $\fraksl(n,\bC)$. The semisimple Lie algebras over $\bC$ have been classified by  Wilhelm Killing and \'Elie Cartan around 1890. The classification assigns the types
$$\begin{aligned} A_{n} &=\mathfrak{s l}(n+1, \mathbb{C}) \\ B_{n} &=\mathfrak{s o}(2 n+1, \mathbb{C}) \\ C_{n} &=\mathfrak{s p}(n, \mathbb{C}) \\ D_{n} &=\mathfrak{s} \mathfrak{o}(2 n, \mathbb{C}) \end{aligned}~.$$
Contrasting with the classical Lie algebras listed above are the exceptional Lie algebras, $E_6$, $E_7$, $E_8$, $F_4$, and  $G_2$,  which share their abstract properties.
For the classification of the simple Lie algebras, we refer the reader to \cite{kirillov2008introduction}.



Given a finite-dimensional Lie algebra, we can pick a basis $B$ for $\mathfrak{g}$.
\be\label{Lie-basis}
  B = \{T_a: a = 1, \cdots, \dim \mathfrak{g}\}.
\ee
Then any $X \in \mathfrak{g}$ can be written as
\[
  X = X^a T_a = \sum_{a = 1}^n X^a T_a,
\]
where $X^a \in \bF$.

By linearity, the bracket of elements $X, Y \in \mathfrak{g}$ can be computed via
\[
  [X, Y] = X^a Y^b [T_a, T_b].
\]
In other words, the whole structure of the Lie algebra can be given by the bracket of basis vectors. We know that $[T_a, T_b]$ is again an element of $\mathfrak{g}$. So we can write
\[
  [T_a, T_b] = f_{ab}{}^c T_c,
\]
where $f_{ab}{}^c\in \bF$ are called the \textbf{structure constants}.
By the antisymmetry of the bracket, we know
  \[
    f_{ba}{}^c = -f_{ab}{}^c.
  \]
The Jacobi identity amounts to
  \[
    f_{ab}{}^c f_{cd}{}^e + f_{da}{}^c f_{cb}{}^e + f_{bd}{}^c f_{ca}{}^e = 0.
  \]




\bexample
  Take $G = \SO(3,\bR)$. Then $\so(3,\bR)$ is the space of $3 \times 3$ real anti-symmetric matrices, which one can manually check are generated by
  \[
    {T}_1 =
    \begin{pmatrix}
      0 & 0 & 0\\
      0 & 0 & -1\\
      0 & 1 & 0
    \end{pmatrix},\quad
    {T}_2 =
    \begin{pmatrix}
      0 & 0 & 1\\
      0 & 0 & 0\\
      -1 & 0 & 0
    \end{pmatrix},\quad
    {T}_3 =
    \begin{pmatrix}
      0 & -1 & 0\\
      1 & 0 & 0\\
      0 & 0 & 0
    \end{pmatrix}
  \]
  We then have
  \[
    ({T}_a)_{bc} = -\varepsilon_{abc}.
  \]
  Then the structure constants are  $f_{ab}{}^{c} = \varepsilon_{abc}$.
\eexample



Given a vector $X\in \mathfrak{g} $ in the tangent space of the identity $e$, one can generate the vector field by pushing-forward by the left translation $L_g$. Let us denote the corresponding vector field by $X$ too. Since $(L_g)_*X=X$, it is called a \term{left-invariant vector field}.  The flow generated by the vector field $X$ is called \textbf{exponential map}, which can be expressed as a matrix
  \[
    \exp(tX) = \sum_{\ell = 0}^\infty \frac{1}{\ell!} (tX)^\ell~.
  \]
Therefore, for any matrix Lie group $G$, the exponential map defines a map $\exp:\mathfrak{g}  \to G$.

Given a Lie algebra $\frakg$, it is also natural to think about its dual space $\frakg^*$. This can be identified with the set of all left invariant 1-forms $\omega$ on $G$ such that $L_g^*\omega =\omega$. Note that $\omega(X)$, $\omega(Y)$ are constant over $G$ for $\omega\in \frakg^*$ and $X,Y\in \frakg$. Therefore, we have $Y(\omega(X))=0=X(\omega(Y))$ so that \eqref{d} reduces to
$$
d\omega(X,Y)=-\frac12\omega([X,Y])~.
$$
Therefore, if we take the basis $\omega_1,\cdots,\omega_{\dim \frakg}$ dual to \eqref{Lie-basis}, we can write it as
$$
d\omega_i=-\frac12 \sum_{j,k} f_{jk}{}_{i}~\omega_j\wedge \omega_k~.
$$
Moreover, let $\omega\in \Omega^1(G;\frakg)$ be $\frakg$-valued 1-form on $G$ such that $\omega(X)=A$ for $A\in \frakg$. Using the above basis, it is describe as
\be\label{MC-form}
\omega = \sum_i \omega_i T^i~,
\ee
which is called \term{Maurer-Cartan form}. Then, the equation above has the following form
$$
d\omega = -\frac12 [\omega,\omega]~,
$$
which is called the \term{Maurer-Cartan equation}.

\section{Vector bundles and Principal \texorpdfstring{$G$}{G}-bundles}\label{sec:bundle}

\subsection{Vector bundles}
We have learnt tangent bundles, cotangent bundles and their tensor products. Generalizing these leads to a notion called vector bundles. The notion of vector bundles was introduced by Whitney. Remarkably, the notion of vector bundles is indispensable for description of non-Abelian gauge theories.

\bdefn[Vector bundle]\index{vector bundle}
  A \textbf{vector bundle} of rank $r$ on $M$ is a smooth manifold $E$ with a smooth \term{projection} $\pi: E \to M$ such that
  \begin{enumerate}
    \item For each $p \in M$, the fiber $\pi^{-1}(p) = E_p$ is an $r$-dimensional vector space,
    \item For all $p \in M$, there is an open $U \subseteq M$ containing $p$ and a diffeomorphism
      \[
        t: E_U = \pi^{-1}(U) \to U \times \bR^r
      \]
      such that
      \[
        \begin{tikzcd}
          E_U \ar[r, "t"] \ar[d, "\pi"] & U \times \bR^r \ar[dl, "p_1"]\\
          U
        \end{tikzcd}
      \]
      commutes, and the induced map $E_q \to \{q\} \times \bR^r$ is a linear isomorphism for all $q \in U$.

      We call $t$ a \term{trivialization} of $E$ over $U$; call $E$ the \term{total space}; call $M$ the \term{base space}. Also, for each $q \in M$, the vector space $E_q = \pi^{-1}(\{q\})$ is called the \term{fiber} over $q$. If $r=1$, it is called \term{line bundle}.
  \end{enumerate}
\edefn


\begin{figure}[ht]\centering
\includegraphics[width=5cm]{pictures/Mobius_strip_illus}
\end{figure}


\bdefn[Transition function]\index{transition function}
  Suppose that $t_\alpha: E|_{U_\alpha} \to U_\alpha \times \bR^r$ and $t_\beta: E|_{U_\beta} \to U_\beta \times \bR^r$ are trivializations of $E$. Then
  \[
    t_\alpha \circ t_\beta^{-1} : (U_\alpha \cap U_\beta) \times \bR^r \to (U_\alpha \cap U_\beta) \times \bR^r
  \]
  is fiberwise linear, i.e.
  \[
    t_\alpha \circ t_\beta^{-1}(q, v) = (q, \varphi_{\alpha\beta}(q) v),
  \]
  where $\varphi_{\alpha\beta}(q)$ is in $\GL(r,\bR)$.

  In fact, $\varphi_{\alpha\beta}: U_\alpha \cap U_\beta \to \GL(r,\bR)$ is smooth. Then $\varphi_{\alpha\beta}$ is known as the \term{transition function} from $\beta$ to $\alpha$.
\edefn
We have the following equalities for transition functions:

(1) $\varphi_{\alpha\alpha} = \id$

(2) $\varphi_{\alpha\beta} = \varphi_{\beta\alpha}^{-1}$

(3) $\varphi_{\alpha\beta}\varphi_{\beta\gamma} = \varphi_{\alpha\gamma}$, which is called \term{cocycle condition}.



On the other hand, given an open cover $\{U_\alpha\}$ of open sets of $M$, suppose we have transition functions $\varphi_{\alpha\beta}$ which satisfy all the above properties. Then, we can glue $U_\alpha\times \bR^n$ and $U_\beta\times \bR^n$ by the transition functions $\varphi_{\alpha\beta}$ and construct a bundle $E\to M$.

\bexample
We construct a line bundle $L$ over the real projective space  $\bR P^n$ as follows. First we consider the direct product $\bR P^n\times \bR^{n+1}$. An arbitrary point $\ell$ in $\bR P^n$ can be regarded as a line through the origin in $\bR^{n+1}$. We construct a line bundle by
\bea
L&=\{(\ell,y )\in \bR P^n\times \bR^{n+1}\mid y \in \ell \}\cr
&=\{ ([x^0;\cdots;x^n], y)\in \bR P^n\times \bR^{n+1}|\ y=\lambda \cdot (x^0,\cdots,x^n)  ~, \quad \lambda\in\bR\} ~.
\eea
This line bundle is called \textbf{Hopf line bundle} or \textbf{tautological line bundle}.
For an open subset $U_i=\{\ell =[x^0; x^1;\cdots; x^n]|x_i\neq 0\}$, a trivialization map is given by
$$
t_i:\pi^{-1}(U_i)\to U_i\times \bR~; ~ (\ell ,y)\mapsto y^i~.
$$
Then, the transition map is given by
$$
t_j\circ t_i^{-1}: (U_i\cap U_j)\times \bR \to (U_i\cap U_j)\times \bR~;~ (\ell, \eta) \mapsto \left(\ell , \frac{x^j}{x^i}\eta \right)
$$
It is easy to see that the transition functions obey (1),(2),(3) above.
\eexample



\begin{figure}[ht]\centering
\includegraphics[width=7cm]{pictures/fig_vector_bundle}
\end{figure}

\bdefn[Section]\index{section}
  A \textbf{section} of a vector bundle $\pi: E \to M$ is a map $s: M \to E$ such that $\pi \circ s = \id$. In other words, $s(p) \in E_p$ for each $p\in M$. We denote a set of sections by $\Gamma(M,E)$.
\edefn







  \bdefn[Bundle map]\index{Bundle map}
We can  consider about maps between vector bundles.  Let $E \to M$ and $E' \to M'$ be vector bundles. A \textbf{bundle map} from $E$ to $E'$ is a pair of smooth maps $(F: E \to E', f: M \to M')$ such that the following diagram commutes:
  \[
    \begin{tikzcd}
      E \ar[d] \ar[r, "F"] & E' \ar[d]\\
      M \ar[r, "f"] & M'
    \end{tikzcd}.
  \]
  i.e.\ such that $F_p: E_p \to E'_{f(p)}$ is linear for each $p$.
\edefn




In fact, every operation on vector spaces can be performed on vector bundles, by doing it on each fiber.
\bdefn[Whitney sum of vector bundles]\index{vector bundle!Whitney sum}\index{Whitney sum of vector bundles}
  Let $\pi: E \to M$ and $\rho: F \to M$ be vector bundles. The \textbf{Whitney sum} is given by
  \[
    E \oplus F = \{(e, f)\in E \times F: \pi(e) = \rho(f)\}.
  \]
  This has a natural map $\pi \oplus \rho: E \oplus F \to M$ given by $(\pi \oplus \rho)(e, f) = \pi(e) = \rho(f)$. This is again a vector bundle, with $(E \oplus F)_x = E_x \oplus F_x$ and again local trivializations of $E$ and $F$ induce one for $E \oplus F$.
\edefn
Tensor products can be defined similarly.

\bdefn[Tensor product of vector bundles]\index{tensor product!vector bundle}\index{vector bundle!tensor product}
  Given two vector bundles $E, F$ over $M$, we can construct $E \otimes F$ similarly with fibers $(E \otimes F)|_p = E|_p \otimes F|_p$.
\edefn

Similarly, we can construct the alternating product of vector bundles $\Lambda^n E$. Finally, we have the \textbf{dual} vector bundle.

\bdefn[Dual vector bundle]\index{vector bundle!dual}\index{dual!vector bundle}
  Given a vector bundle $E \to M$, we define the \textbf{dual vector bundle} by
  \[
    E^* = \bigcup_{p \in M} (E_p)^*.
  \]
  Suppose again that $t_\alpha: E|_{U_\alpha} \to U_\alpha \times \bR^n$ is a local trivialization. Taking the dual of this map gives
  \[
    t_\alpha^*: U_\alpha \times (\bR^n)^* \to E|_{U_\alpha}^*.
  \]
  since taking the dual reverses the direction of the map. We pick an isomorphism $(\bR^n)^* \to \bR$ once and for all, and then reverse the above isomorphism to get a map
  \[
    E|_{U_\alpha}^* \to U_\alpha \times \bR^n.
  \]
  This gives a local trivialization.
\edefn

One important operation we can do on vector bundles is \textbf{pullback}:
\bdefn[Pullback of vector bundles]\index{pullback!vector bundle}\index{vector bundle!pullback}
  Let $\pi: E \to M$ be a vector bundle, and $f: N \to M$ a map. We define the \textbf{pullback}
  \[
    f^* E = \{(y, e) \in N \times E: f(y) = \pi(e)\}.
  \]
  This has a map $f^*\pi: f^*E \to N$ given by projecting to the first coordinate. The vector space structure on each fiber is given by the identification $(f^*E)_y = E_{f(y)}$. It is a little exercise in topology to show that the local trivializations of $\pi: E \to M$ induce local trivializations of $f^*\pi: f^* E \to N$.
\edefn


One can introduce a metric $g$ on fibers of a vector bundle $E$. Namely, we have a non-degenerate symmetric positive-definite 2-form on a fiber $E_x$
$$
g_x:E_x\times E_x\to \bR~,
$$
and $g_x$ is differentiable in terms of $x$. If it is the tangent bundle $TM$, it is a Riemannian metric. Given a trivialization $\pi^{-1}(U_\alpha)\to U_\alpha \times \bR^r$, we can take an orthonormal frame $(e_1,\cdots,e_r)$ for $e_i\in\Gamma(U,E)$ with respect to $g$. Then, the transition function takes the value at $\textrm{O}(r)$
$$
\varphi_{\alpha\beta}:U_\alpha\cap U_\beta \to \textrm{O}(r)~.
$$
We can further generalize that the transition function takes the value at an arbitrary Lie group $G$ with a representation $\rho: G \to  \GL(V)$.
\bdefn[$G$-bundle]\index{$G$-bundle}
  Let $V$ be a vector space, $G$ a Lie group, and $\rho: G \to \GL(V)$ a representation. Then a $G$-bundle $\pi:E\to M$ consists of the following data:
  \begin{enumerate}
    \item For each $p\in M$, the fiber is $\pi^{-1}(p)\cong V$.
    \item  One can take a trivializing cover $\{U_\alpha\}$ with transition functions $t_{\alpha\beta}:(U_\alpha \cap U_\beta) \times V \to (U_\alpha \cap U_\beta) \times V$.
    \item The transition functions are constructed by maps $\varphi_{\alpha\beta}: U_{\alpha} \cap U_\beta: \to G$ satisfying the cocycle conditions with the representation $\rho$ such that $t_{\alpha\beta} = \rho \circ \varphi_{\alpha\beta}$.
  \end{enumerate}
\edefn



\end{document}
