\documentclass[geometry-lectures-19.tex]{subfiles}

\begin{document}





\section{Characteristic classes}

As we (will) see in homework, principal $U(1)$-bundles on $S^2$ are classified by the monopole number $n$ which tells us how the $U(1)$ fibers over the upper hemisphere and the lower hemisphere are glued together. The generalization of this notion for a vector bundle $\pi:E\to M$ leads to a \term{characteristic class} associated to a cohomology class of $M$. Characteristic classes was introduced to extract topological information of a base manifold of vector bundles or principal $G$-bundles from curvature forms. This is called \term{Chern-Weil theory}, on which \cite[\S6]{morita2001geometry} is a wonderful exposition.




Characteristic classes are constructed as \term{invariant polynomials} of the curvature $F = dA+A\wedge A$. As we see in \eqref{gauge-trans2}, $F$ transforms as $F\to g^{-1}Fg$ under the gauge transformation where $g\in \mathscr{G}_E$. To construct characteristic classes, we need to introduce invariant polynomials $P(X)$ of matrixes, that is invariant under the conjugation, $P(g^{-1}F g) = P(F)$. Examples of invariant polynomials are $\Tr F^k$ ($k = 1,2,\cdots$) and $\det F$. In fact, we can use these to construct nice bases of invariant polynomials as follows:


(1) $\sigma_k(F)$ defined by
$$\det(1 + tF) = 1 + t\sigma_1(F) + t^2\sigma_2(F) + \cdots + t^r\sigma_r(F)~.$$

(2) $s_k(F)$ defined by
 $$s_k(F) = \Tr F^k~, \qquad (k = 1, \cdots, r)~.$$
They are related to each other by Newton's formula,
$$s_1 = \sigma_1~ , \quad  s_2 = \sigma_1^2 -\sigma_2 ~, \quad  s_3 = \sigma_1^3 - 3 \sigma_1 \sigma_2 + 3  \sigma_3 ~, \cdots$$
We can also express the invariant polynomials in terms of eigenvalues. If $F$ is a hermitian matrix, we can diagonalize it with eigenvalues $x_1,\cdots,x_k$. Then,
$$\prod_{k=1}^r (1 + tx_k) = 1 + t\sigma_1(x) + \cdots + t^r\sigma_r(x). $$
Similarly,
$s_k(F) = \sum_{j=1}^r (x_j)^k$.


If $P_k(F)$ is an invariant polynomial of degree $k$, we can use the curvature 2-form $F$ to define a $2k$-form $P_k(F)$ so that it is invariant under the gauge transformation, $F\to g^{-1}F g$. It is also a closed form because the Bianchi identity
$$
dF+[A\wedge F]=0
$$
tells us
\begin{align}\nonumber
d\; \Tr F^k &= \Tr (dF\;F^{k-1} +F\;dF \;F^{k-1} +\cdots + F^{k-1}dF ) \cr
&= -\Tr (([A\wedge F])F^{k-1} +\cdots+F^{k-1}([A\wedge F])  ) \cr
&=-\Tr(A\wedge F^k -F^k\wedge A)=0
\end{align}
Thus, we find $P_k(F)\in H^{2k}(M)$.

Moreover, $P_k(F)$ is invariant under continuous deformation of the gauge field $A$ as an element of $H^{2k}(M)$. Suppose that we change $A \to A + \eta$ with $\eta$ being an infinitesimal one-form.
Under this deformation, $F$ changes by $\delta F = d\eta + A\wedge \eta+\eta\wedge A$. Therefore,
\begin{align}
\delta \Tr F^k &=\Tr( (d\eta + A\wedge \eta+\eta\wedge A)F^{k-1} +\cdots+F^{k-1}(d\eta + A\wedge \eta+\eta\wedge A)  )\cr
&=k\Tr  ((d\eta + A\wedge \eta+\eta\wedge A)F^{k-1}) \cr
&=k\Tr ( d\eta F^{k-1} -\eta dFF^{k- 2} -\cdots-\eta F^{k- 2}dF  )\cr
&= k \;d\Tr ( \eta F^{k-1} )~.\nonumber
\end{align}
Since both $\eta$ and $F$ transform homogeneously under the gauge transformation, $\Tr(\eta F^{k-1})$ is a well-defined ($2k-1$)-form. Thus, under any infinitesimal deformation, $P_k(F)$ changes by an exact form. Thus, $P_k(F)$ depends only on the type of the bundle $E$ and not on a choice of the connection $A$ on $E$. For a more complete proof, please refer to \cite[Proposition 5.28]{morita2001geometry}.

\subsection{Pontryagin classes}
Let us consider a vector bundle $\pi:E\to M$ of rank $r$. One can always put a metric $g$ on $E$ and consider a connection $A$ which is compatible with the metric $g$. Therefore, we can consider the curvature form $F$ takes its value on $\mathfrak{so}(r)$. Namely it is an anti-symmetric $(F^i{}_j+F^j{}_i=0)$ 2-form so that $\Tr F^k=0$ if $k$ is odd. Therefore, for  the invariant polynomial $P_k$ of odd degree, we have
$$
P_k(F)=0~.
$$
As a result, the \term{Pointryagin class} is defined as
$$
p_k(E):=\frac1{(2\pi)^{2k}}\sigma_{2k}(F)\in H^{4k}(M;\bR)
$$
They may be written as
$$p(E)=\det \left(1+\frac{1}{2\pi} F \right)=1+p_1(E)+p_2(E)+\cdots+p_{[r/2]}(E)~.$$



\bthm[Hirzebruch signature theorem]\label{Hirzebruch}
Let $M$ be an oriented compact 4-dimensional manifold. Since the Hodge star $\ast:H^2(M)\to H^2(M)$ satisfies $\ast^2=1$ on $H^2(M)$, we can decompose it into $H^2(M)=H^2_+(M)\oplus H^2_-(M)$ with eigenvalues $\pm1$ of $\ast$. Let us define the signature of $M$ by
$$
\tau(M)=\dim H^2_+(M) - \dim H^2_-(M)~.
$$
Then, the signature can be expressed by
$$
\tau(M)=\frac13\int_M p_1(TM)~.
$$
\ethm






\subsection{Chern classes}
Now let us consider complex vector bundle $\pi:E\to M$ of rank $r$.  Similarly, one can always put a Hermitian metric $g$ on $E$ and consider a connection $A$ which is compatible with the metric $g$. Then, the curvature form $F$ takes its value $\mathfrak{u}(r)$. Namely it is a skew-Hermitian $(F^i{}_j+\overline F^j{}_i=0)$ 2-form so that $\Tr \left(\frac{F}{2\pi i} \right)^k$ is a real $2k$-form.

Then, \term{Chern class} is defined as
$$c_k(E):=\left(\frac{-1}{2\pi i}\right)^k\sigma_k(F) \in H^{2k}(M;\bR)~.$$
This can be written as
$$c(E)=\det  \left(1 - \frac{1}{2\pi i} F \right)  = 1 + c_1 (E)+ c_2 (E)+ \cdots +c_r(E)~.$$
  For example, we can explicitly write
$$ c_1=\frac{-1}{2\pi i} \Tr F~, \quad c_2=-\frac1{8\pi^2} (\Tr F\wedge \Tr  F-\Tr  F\wedge F), \cdots $$
  If the structure group is in $\SU(r)\in\U(r)$, we have a trivial first Chern class $c_1 = 0$ because $\mathfrak{su}(r)$ is traceless.



Instead of $\sigma_k$, we can use $s_k$ for invariant polynomials, which defines the \term{Chern characters}
$$ch_k(E) = \frac{1}{k!}\Tr \left(-\frac{F}{2\pi i}\right)^k  \in H^{2k}(M) ~. $$
We can also write it as
$$ch(E)=1 +ch_1 (E)+\cdots=\Tr\exp  \left(-\frac{F}{2\pi i}\right) ~.$$


\subsection*{Some properties}



The Pontryagin class and Chern class are related by
\be\label{p-c}
p_k(E)=(-1)^kc_{2k}(E\otimes\bC)\in H^{4k}(M;\bR)
\ee
where $E\otimes \bC$ is the complexification of a real bundle $E\to M$. Since $\sigma_k=0$, we have
\be\label{odd-vanish}
c_k(E\otimes \bC)=0~, \quad \textrm{for} \ k \  \textrm{odd}
\ee

   One of the important properties of the Pontryagin and Chern classes is that it behaves nicely when we take a direct sum $E_1 \oplus E_2$ of vector bundles $E_1, E_2$ as,
\begin{align}\nonumber
p(E_1 \oplus E_2) &= p(E_1) \wedge p(E_2)~,\cr
c(E_1 \oplus E_2) &= c(E_1) \wedge c(E_2)~.
\end{align}
On the other hand, it does not behave nicely under the direct product $E_1 \otimes E_2$.




The Chern characters behave nicely under both the direct sum and direct product as,
\begin{align}\nonumber
ch(E_1 \oplus E_2) = ch(E_1) + ch(E_2)~,\cr
ch(E_1 \otimes E_2) = ch(E_1) \wedge ch(E_2)~.
\end{align}
This property plays an important role in \term{K-theory}.

\subsection{Euler class}
Let us turn to a real vector bundle $\pi:E\to M$ of even rank $2r$. In this case, in addition to $\Tr$ and $\det$, we can consider one more way to construct an invariant polynomial, which is called the \term{Pfaffian},
$$
Pf(F) = \frac{1}{2^r r!} \sum_{\sigma\in S_{2r}}\textrm{sgn}(\sigma)~ F_{\sigma(1)\sigma(2)}F_{\sigma(3)\sigma(4)} \cdots F_{\sigma(2r-1)\sigma(2r)}~.
$$
Note that, for antisymmetric matrices, the Pfaffian is a square root of the determinant,
$\det F = Pf(F)^2$.
If $F$ is real and anti-symmetric, we can block diagonalize it by $\SO(2r)$ as
\be\label{anti-symm}
F=\left( \begin{array}{ccccccc}
 0& x_1 &0& 0& \cdots & 0& 0\\
-x_1& 0& 0& 0&\cdots &0& 0\\
 0& 0& 0& x_2 &&& \\
0 &0& -x_2& 0 &&&\\
\cdot&\cdot&&&\cdots&\cdot&\cdot\\
0&0 &&&\cdots& 0 &x_r\\
0 &0&&&\cdots&  -x_r&0\\
\end{array}\right)
\ee
 We can then write the Pfaffian as
$$Pf(F) = \prod_{i=1}^r x_k.$$
Under the conjugation $F \to g^TFg$, the Pfaffian transforms as $Pf( g^T Fg) = \det g \cdot Pf(F)$.

Thus, if $g\in\SO(2r)$, the Pfaffian is invariant.
We can now define the \term{Euler class} by
$$e(E) = \frac{1}{(2\pi)^r}Pf(E)\in H^{2r}(M;\bR)~.$$
In fact, the Euler class can be understood as the square root of the highest Pontryagin class $p_r(E)$
$$p_r(E) = e(E)^2~.$$
In particular, let us consider the tangent bundle $TM$ of an closed  orientable Riemannian manifold $(M,g)$ of dimensions
 $n = 2r$. Then $TM$ is an $\SO(2r)$ bundle, and its Euler form can be written in terms of the Riemann curvatures
\begin{align}\nonumber
n=2:&\quad e(TM)= \frac{1}{4\pi} \epsilon_{ab}\Omega^{ab}~, \cr
n=4:&\quad e(TM)= \frac{1}{32\pi^2} \epsilon_{abcd}\Omega^{ab}\wedge \Omega^{cd}~,
\end{align}
where the Riemann curvatures are regarded as the 2-forms as in \eqref{Riemann-2-form}
$$\Omega^a{}_b =\textbf{R}_{cd}{}^a{}_b e^c\wedge e^d=R_{\mu\nu}{}^a{}_b dx^\mu\wedge dx^\nu~.$$
The integral of the Euler class $e(TM)$ over $M$ is indeed equal to the Euler characteristic
$$\chi(M) = \int_M  e(TM)~,$$
which can be considered as the higher dimensional version of the Gauss-Bonnet theorem \eqref{Gauss-Bonnet}.


 \subsection{Todd, $L$- and $\widehat A$-classes}

Here we introduce other characteristic classes, such as \term{Todd classes, Hirzebruch $L$-class}, and \term{$\widehat A$-class} \cite{Eguchi:1980jx}. These classes are just defined by different basis of invariant polynomials. However, there are very important since these classes will show up in the index theorem.

To describe Todd class, let $x_1,\cdots,x_k$ be eigenvalues of curvature form $\frac{-F}{2\pi i}$ of a complex vector bundle. For example, the total Chern classes can
  be expressed as
$$ c(F)=\det \left(1-\frac{F}{2\pi i}\right)= \prod_{i=1}^r(1+x_k).$$
It is worth mentioning that the right-hand side takes the form $\prod_k c(L_k)$, where $L_k$ is a line bundle with a curvature given by $x_k$ and $c(L_k) = 1+x_k$. Thus, as far as the Chern classes are concerned, the vector bundle $E$ behaves like a Whitney (direct) sum of the line bundles $L_1 \oplus L_2 \oplus \cdots\oplus L_k$ although they are not isomorphic as bundles. This phenomenon is called the \term{splitting principle}, and $x_i$ are called \term{Chern roots}.
Using this notation, the Todd class is defined by,
$$Td(E) =\prod_k\frac{ x_k}{1-e^{-x_k}}=1+\frac12c_1+\frac1{12}(c_2+c_1^2)+\cdots~,$$

Furthermore, for a real vector bundle $E$, \eqref{odd-vanish} tells you that the Chern roots
of its complexification \(E\otimes \bC\) come in opposite pairs \(x,-x\). Moreover, the relation \eqref{p-c} tells us that Pontryagin classes of $E$ are written as
$$p(E)=\prod_{j}\left(1+x_{j}^{2}\right)$$
where $x_j$ are Chern roots of \(E\otimes \bC\) with the positive sign in each pair. With this notation, the Hirzebruch $L$-classes for the real vector bundle $E$ are defined by
$$ L(E)=\prod_{k} \frac{ x_k}{\tanh{x_k}}=1+\frac13 p_1+\frac1{45}(7p_2-p_1^2)+\cdots~.$$
Indeed Hirzebruch has introduced this class for the signature theorem \ref{Hirzebruch} of a $4k$-dimensional closed oriented manifolds.
The $\widehat A$-classes for the real vector bundle $E$ are defined by,
$$ \widehat A(E)=\prod_{k} \frac{ x_k/2}{\sinh{x_k/2}}=1+\frac{1}{24}p_1+\frac{1}{5760}(7p_1^2-4p_2)+\cdots~,$$
which appears in the index theorem of a Dirac operator.
\end{document}
