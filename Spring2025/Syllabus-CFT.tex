\documentclass[a4paper,11pt]{article}
\usepackage[top=3cm, bottom=3cm, left=2cm, right=2cm]{geometry}
\usepackage{CJK}
\usepackage{hyperref}
\usepackage{fullpage}
\usepackage{graphicx}

\begin{document}\thispagestyle{empty}
\begin{CJK}{UTF8}{gbsn}

\centerline{\Large \bf Syllabus}

\begin{description}
\item{\bf Course name:} Conformal Field Theory (PHYS130111.01)
\item{\bf Instructor:} Satoshi Nawata, Physics S422, Jiangwan \href{mailto:snawata@fudan.edu.cn}{snawata@fudan.edu.cn}
\item{\bf Teaching Assistant:} Junkang Huang  \href{mailto:22210190012@m.fudan.edu.cn}{22210190012@m.fudan.edu.cn}
\item{\bf Hours:} Wednesday 13:30-16:10
\item{\bf Place:} H3308
\item{\bf Office hour:} Flexible, but please text me in advance.
\item{\bf Prerequisites:} Linear algebra, Calculus, Classical mechanics

\item{\bf About the course:}
Conformal Field Theory (CFT) is a fundamental framework in theoretical physics, with applications spanning quantum field theory, string theory, integrable systems, and condensed matter physics. This course provides a thorough introduction to the foundational principles and techniques of CFT, offering students a firm grounding in its essential concepts while exploring its diverse applications across dimensions.

The first half studies 2d CFT, starting with the basics of conformal symmetry. We delve into the representation theory of conformal algebras, the operator product expansion, and the structure of correlation functions. Topics include the Virasoro algebra, minimal models, and renormalization group flow, all introduced with a focus on their fundamental aspects. Students will also engage with advanced concepts such as the Zamolodchikov ccc-theorem and the connections between CFT and Landau-Ginzburg models.

The second part of the course focuses on CFT in higher dimensions, higher-dimensional algebra and and its pivotal role in quantum field theory and statistical mechanics. These core ideas are further explored through practical examples, such as the Wilson-Fisher fixed point and the construction of conformal blocks.



\item{\bf Main Content:}
 Introduces foundational principles and techniques of CFT with applications in quantum field theory, statistical mechanics, and condensed matter physics.
Explores 2D CFT, focusing on:
    \begin{itemize}
        \item Conformal symmetry in quantum field theory and statistical mechanics.
        \item Virasoro algebra and minimal models.
        \item Renormalization group flow.
        \item Advanced concepts such as the Zamolodchikov \(c\)-theorem and connections to Landau-Ginzburg models.
    \end{itemize}
    Covers higher-dimensional CFT with topics like:
    \begin{itemize}
        \item Representation theory of conformal algebra.
        \item Operator product expansions and correlation functions.
        \item Examples including the Wilson-Fisher fixed point and conformal blocks.
    \end{itemize}


\item{\bf Lecture Notes:}
There will be no required textbook for this course. I will be writing a new lecture note from scratch, and it will be available at Overleaf. The previous lecture materials will not be used.

\item{\bf Supplementary Textbooks:}
\begin{itemize}
    \item Y. Nakayama, \emph{Scale Invariance vs Conformal Invariance}, Phys. Rept. 569 (2015), 1–93, \href{https://arxiv.org/abs/1302.0884}{arXiv:1302.0884 [hep-th]}.
    \item S. Rychkov, \emph{EPFL Lectures on Conformal Field Theory in \(D \geq 3\) Dimensions}, Springer, 2016, \href{https://arxiv.org/abs/1601.05000}{arXiv:1601.05000 [hep-th]}.
    \item D. Simmons-Duffin, \emph{The Conformal Bootstrap}, TASI 2015 Lecture Notes, \href{https://arxiv.org/abs/1602.07982}{arXiv:1602.07982 [hep-th]}.
    \item Fudan Lectures on 2D Conformal Field Theory by S. Nawata, R. Tao, and D. Yokoyama, \href{https://arxiv.org/abs/2208.05180}{arXiv:2208.05180}.
\end{itemize}


\item{\bf Grading:}
Grades will be based on a combination of homework assignments and a final exam:
\begin{itemize}
\item Homework Assignments: 60\% (6 sets, given every other week)
\item Final Exam: 40\%
\end{itemize}

%\item{\bf Course Goals:}
%By the end of the course, students will:
%\begin{itemize}
%\item Gain an intuitive understanding of the relationship between physical systems and their underlying geometrical and topological structures.
%\item Develop a working knowledge of how mathematical tools like differential geometry and topology are applied in classical mechanics, electrodynamics, and general relativity.
%\item Be able to apply these concepts to solve problems in physics, ranging from the behavior of classical systems to the structure of spacetime.
%\end{itemize}
%
%\item{\bf Office Hours:}
%I encourage students to contact me via email to schedule office hours for questions or discussions on course content or homework.

\end{description}

\end{CJK}

\end{document}
