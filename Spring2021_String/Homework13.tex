\documentclass[12pt,a4paper]{article}
%\usepackage{hyperref} % Use the Charter font for the document text
%\usepackage[UTF8]{ctex}
\usepackage{macros}


%
%---------- mathfrak font -----------------------------
%

\begin{document}\thispagestyle{empty}

\centerline{\Large \bf Homework 13 (Due at class on June 13)}


\section{Surface gravity}
Section 6, Problem 4 \cite{wald2010general}.


\section{Cardy formula}
\subsection{Free boson}
Let us consider the torus partition function of bosonic string theory
\bea\label{partition}
     Z(\beta) &=  {1 \over q} \prod_{s=1}^\infty {1\over(1 -
  q^s)^{24}}  \cr
   &\equiv \sum_{n=-1}^{\infty} d(n)q^{n} \qquad q := e^{-\beta}~.
\eea
As explained in the lecture, $1/\b$ can be interpreted as temperature.
The degeneracy $d(N)$ can be obtained from the  partition function by the inverse Laplace transform
\begin{equation}\label{omega}
    d (N) = {1\over 2\pi i} \int d\beta e^{\beta N} Z(\beta).
\end{equation}
We would like to evaluate this integral (\ref{omega}) for large $N$ which
corresponds to large worldsheet energy.

For large $N$, we  expect
that the integral receives most of its contributions from high
temperature or small $\beta$ region of the integrand.  To evaluate the small $\beta$ contribution, it is convenient to use S-dual frame
$$
    Z(\beta)  = \left(\frac{\beta}{2\pi}\right)^{12} Z ({4\pi^2 \over \beta}).
$$
Using this property, show that \eqref{omega} for large $N$ can be approximated as
$$
    d (N) \sim {1\over 2\pi i} \int \left({\beta \over
2\pi}\right)^{12} e^{\beta N + {4\pi^2 \over \beta}} d\beta.
$$
Using the saddle point analysis,  show that  the leading
asymptotic expression for the number of states is
$$
    d (N) \sim \exp{(4\pi \sqrt{N})}~.
$$


\subsection{D1-D5-P system}
In the lecture, the torus partition function of the long string in the D1-D5-P system is given by
$$
Z = \textrm{const} \left[ \prod_{m=1}^\infty {\frac{1+q^{m}}{1-q^{m}}}
\right]^{4} \equiv \sum \Omega(n) q^{n} \,,
$$
Note that the partition function can be written as
$$
Z=\Bigg[\frac{\vartheta_2(\tau)}{\eta(\tau)^3}\Bigg]^2
$$
upto some constant.
Using the modular property of this partition function, show that the number of states at large D-brane charges is
$$
\Omega(n=Q_1 Q_5 Q_P) \sim(Q_1 Q_5 Q_P)^{ -\frac74} \exp\Big(2\pi \sqrt{Q_1 Q_5 Q_P}\Big) ~.
$$






\bibliography{string-lecture}
\bibliographystyle{halpha}



\end{document}
