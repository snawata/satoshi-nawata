\documentclass[12pt,a4paper]{article}
%\usepackage{hyperref} % Use the Charter font for the document text
%\usepackage[UTF8]{ctex}
\usepackage{macros}




\begin{document}\thispagestyle{empty}

\centerline{\Large \bf Homework 4: Due at class on April 2}

\section{$bc$ ghost CFT}


\subsection{ energy-momentum tensor}

Given the $bc$ ghost action (Euclidian signature)
$$ S_{\rm gh} = \frac{1}{2\pi}\int d^2\sigma \sqrt{h}\ b^{ab}\nabla_a c_b~,$$
calculate the stress tensor for the $bc$ ghosts by
$$
T_{ab}=-\frac{4\pi}{\sqrt{h}}\frac{\d S}{\d h^{ab}}
$$
Note that the covariant derivative $\nabla^\alpha$ contains the Christoffel symbol  and $b_{ab}$ is symmetric traceless. Show that it becomes
\be T^{\rm gh}(z) =- 2:b(z)\partial c(z): + :c(z)\partial b(z):\label{ST}\ee
in the conformally flat metric.


\subsection{ $TT$ OPE}
Using the energy-momentum \eqref{ST},  derive the $TT$ OPE in the $bc$ ghost CFT
\begin{align}
T(z)\,T(w) &= \frac{-13}{(z-w)^4}+\frac{2T(w)}{(z-w)^2} + \frac{\partial T(w)}{z-w} +\ldots\nonumber
\end{align}

%\subsection{1.3 BRST transformation}
%Using the explicit form of the  BRST current,
%\begin{align}
%j_B = c(z)T^X(z) + : b(z)c(z)\partial c(z) : +\frac32:\partial^2c(z):~.\nonumber
%\end{align}
%show that the BRST transformations of the fields are
%\begin{align}\nonumber
%&\delta_B X^\mu = i \epsilon ( c \partial + \bar{c} \bar{\partial} ) X^\mu \,,
%\cr
%&\delta_B c=  i\epsilon  c \partial c  \qquad \delta_B \bar c=  i\epsilon \bar{c} \bar{\partial} \bar c \,,
%\\
%&\delta_B b =  i \epsilon ( T^X + T^{\rm gh} ) \qquad \delta_B \bar b =  i \epsilon ( \overline T^X + \overline T^{\rm gh} )\,. \nonumber
%\end{align}
\subsection{ BRST charge}
Using the explicit form of the  BRST current
\begin{align}
j_B = c(z)T^X(z) + : b(z)c(z)\partial c(z) : +\frac32:\partial^2c(z):~,\nonumber
\end{align}
express the BRST charge in terms of the $X^{\mu}$ Virasoro
operators and the ghost oscillators as
$$
Q_B = \sum_n c_n (L^X_{-n}-\d_{n,0}) + \sum_{m,n} \frac{m-n}{2} : c_m c_n
b_{-m-n} :  \,.\\
$$
Show that the OPE between two BRST currents is given by
$$
j_B(z)j_B(w)=-\frac{c^X -18}{ 2(z - w)^3} c\partial c(w)- \frac{c^X -18}{4(z - w)^2} c\partial^2c(w)- \frac{c^X -26}{12(z - w)} c\partial^3c(w)+\cdots,
$$
where $c^X$ is the central charge of the $X^\mu$ bosonic string theory.
Use this OPE to determine the anticommutator of the BRST charge with itself.
For what value of $c^X$ does this vanish?






\subsection{ $Tj_B$ OPE}
Show that the OPE between the total energy momentum tensor $T = T^X + T^{\rm gh}$ and $j_B$ is given by
$$T(z)j_B(w)=\frac{ c^X -26}{2(z - w)^4} c(w)+ \frac{ j_B(w)}{(z - w)^2}+ \frac{ \partial j_B(w)}{z - w}+\cdots~.$$
What does the result imply for $j_B$?

\section{$\b\g$ ghost CFT}
Now let us consider the same action as the $bc$ ghost system
$$
S=\frac{1}{2\pi}\int d^2z \beta\overline \partial \gamma +\overline\beta\partial \overline\gamma~,
$$
but now $\beta$ and $\gamma$ are bosonic fields. Hence, their OPEs are (pay attention to sign)
$$
  \g(z)\b(w)=-\b(z)\g(w) =\frac{1}{z - w}+\cdots
$$
If $\beta$ and $\g$ are primary fields of weights $(\lambda,0)$ and $(1-\lambda,0)$ respectively, the form of the stress energy tensor (holomorphic part) is
$$
T(z) =: (\partial \b) \g : -\lambda\, \partial: \b\g :
$$
Calculate the the $TT$ OPE to determine the central charge of the CFT in terms of $\lambda$.
%
%
%
% \section{linear fractional transformations}
%
% Let us consider the Riemann sphere $S^2  = \bC \cup \{\infty\}$. The action of $SL(2, \bC)$ defined by
% $$z  \mapsto w = \frac{az + b}{ cz + d}~,\qquad \begin{pmatrix}a&b\\c&d\end{pmatrix}\in SL(2, \bC)~,$$
% maps the Riemann sphere onto itself. These transformations are called linear fractional  transformations.
%
% \vspace{.3cm}
% \noindent $\bullet$ Given three points $z_1,z_2,z_3$, find a linear fractional transformation which maps the points
% to $0, 1, \infty$.
%
% \vspace{.3cm}
% \noindent $\bullet$ Given four points $z_1,z_2,z_3,z_4$, their \textbf{cross ratio} is defined by
% $$
% [z_1,z_2,z_3,z_4]= {\frac  {(z_{1}-z_{3})(z_{2}-z_{4})}{(z_{2}-z_{3})(z_{1}-z_{4})}}~.
% $$
% Show that the cross ratio is preserved by any linear fractional  transformation
% $$
% [z_1,z_2,z_3,z_4]=[w_1,w_2,w_3,w_4] ~.
% $$

\end{document}
