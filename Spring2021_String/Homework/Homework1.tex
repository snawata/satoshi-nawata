\documentclass[12pt,a4paper]{article}
%\usepackage{hyperref} % Use the Charter font for the document text
% %\usepackage[UTF8]{ctex}
% \usepackage{jheppub}
\usepackage{macros}
%%% Yokoyama def %%%
\providecommand{\vcentcolon}{\mathrel{\mathop{:}}}
%%%%%%%%%%%%%%%%%%%%
\usepackage{graphicx}


\begin{document}\thispagestyle{empty}

\centerline{\Large \bf Homework 1: Due at class on Mar 12}


\section{Canonical quantization of free scalar}

Using the mode expansions in the lecture note, derive from the canonical commutation relation for  $X^\mu$ and $\Pi_\mu$
%
\begin{align}
&[X^\mu(\sigma,\tau),\Pi_\nu(\sigma^\prime,\tau)]=i\delta(\sigma-\sigma^\prime)
\,\delta^\mu_{\ \nu} &\ , \cr
&[X^\mu(\sigma,\tau),X^\nu(\sigma^\prime,\tau)] = [\Pi_\mu
(\sigma,\tau),\Pi_\nu(\sigma^\prime,\tau)] = 0&\ .\nonumber\end{align}
commutation relations for the Fourier modes $x^\mu$, $p^\mu$, ${\alpha}_n^\mu$ and $\wt {\alpha}_n^\mu$
%
\be [x^\mu,p_\nu]=i\delta^\mu_{\ \nu} \ \ \ {\rm and}\ \ \
{[}{\alpha}_n^\mu,\alpha_m^\nu{]}=
{[}\wt {\alpha}_n^\mu,\widetilde{\alpha}_m^\nu{]}
=n\, \eta^{\mu\nu}\delta_{n+m,\,0}\ ,\nonumber\ee



\section{}

Problem 2.3 in Becker-Becker-Schwarz.





\section{Open string spectra}
An open string has boundaries so that one needs to impose a boundary condition. There are two boundary conditions one can impose:
\begin{itemize}
\item \textbf{Neumann boundary condition}:  $\partial_\sigma X^\mu = 0 \ \ \ \ {\rm at}\ \sigma=0,\pi$
\item \textbf{Dirichlet boundary condition}:  $X^\mu = c^\mu$ (constant) {at} $\sigma=0,\pi$
\end{itemize}
\begin{figure}[h]
\centering \includegraphics[width=10cm]{open-boundary}
\caption{Dirichlet (left) and Neumann (right) boundary conditions}
\end{figure}
Like the close string, we take the mode expansion for the open string $X^\mu = X^\mu_L(\sigma^+)
+X^\mu_R(\sigma^-)$ by
%
\begin{align}
X^\mu_L(\sigma^+) &= \frac12 x^\mu +  \a' p^\mu \,\sigma^+ + i\sqrt{\frac{\a'}{2}}\sum_{n\neq 0}
\frac{1}{n}\,\wt{\alpha}_n^\mu\, e^{-in\sigma^+} \ ,\cr
X^\mu_R(\sigma^-) &= \frac12 x^\mu +  \a' p^\mu \,\sigma^- + i\sqrt{\frac{\a'}{2}}\sum_{n\neq 0}
\frac{1}{n}\,{\alpha}_n^\mu\, e^{-in\sigma^-} \ .\label{openmode}\end{align}
%
Note that the second term differs from the closed string by factor of 2. Show that the boundary conditions impose the following requirements
\begin{itemize}
\item Neumann boundary condition  requires $ \alpha_n^a = \wt{\alpha}_n^a$.
\item Dirichlet boundary condition requires  $x^I=c^I,\quad p^I=0,\quad \alpha^I_n = -\wt{\alpha}_n^I$.
\end{itemize}
Actually, this is an essence of the previous probelm.


Now let us study open string mass spectrum in the quantum theory. In the case of open strings, we can define the momentum $\a_0^\mu=\sqrt{2\a'}\,p^\mu$. Show that the light-cone gauge quantization for \eqref{openmode} gives
$$
 2\alpha^-_n = \sqrt{\frac{1}{2\a'}}\frac{1}{p^+}\sum_{m=-\infty}^{\infty}\sum_{i=1}^{D-2} \alpha_{n-m}^i\alpha^i_m~.
$$
Check that $n=0$ can be read off
$$
M^2 = 2p^+p^- - \sum_{i=1}^{D-2}p^ip^i = \frac{1}{\a'}\left(\sum_{n>0}\sum_{i=1}^{D-2} \alpha_{-n}^i\alpha_n^i + \frac{D-2}{2}\sum_{n>0} n\right)~,
$$
and explain the reason why there is the difference of factor 4 from the closed string. Again, in $D=26$,  the open string mass spectrum becomes
$$
M^2=\frac1{\a'}(N-1)~.
$$
so that there is the tachyon for $N=0$. The massless states
$$ \alpha_{-1}^i |0;k\rangle\ \ \ \ \ i=1,\ldots,D-2$$
for $N=1$ correspond to a vector boson.

\end{document}
