\documentclass[12pt,a4paper]{article}
%\usepackage{hyperref} % Use the Charter font for the document text
%\usepackage[UTF8]{ctex}
\usepackage{macros}





\begin{document}\thispagestyle{empty}

\centerline{\Large \bf Homework 10 (Due at class on May 23)}


\section{ Fermionization}

\subsection{}
At the special radius $R =\sqrt{\frac{\a'}{2}}$ of the circle compactification, one can redefine a bosonic field $H(z):=\sqrt{\frac{2}{\a'}}X(z)$ so that the OPE are given by
$$H(z)H(0) \sim - \ln z~.$$
Let us also consider two Majorana-Weyl fermions $\psi^1, \psi^2$ with OPE
$$\psi^i(z)\psi^j(0) \sim \frac{\d^{ij}}{z}~.$$
We can define the complex fermion $$\psi(z) = 2^{-1/2} (\psi^1(z) +i\psi^2(z))\,,\qquad \overline\psi(z) = 2^{-1/2} (\psi^1(z) -i\psi^2(z))~.$$
Show the equivalence of operators in boson and fermion
$$:e^{iH} :\cong\psi\,,\quad :e^{-iH} :\cong\overline \psi\,,\quad i\partial H \cong \psi\overline \psi\,,\quad T_H \cong T_\psi~,$$
by calculating the OPEs of operators in both theories and comparing.





\subsection{}
At the special radius $R =\sqrt{\frac{\a'}{2}}$, show that the torus partition function can be written as
$$
Z^{25}=\frac{1}{2|\eta(q)|^2}\left[\Big| \sum_n q^{n^2/2} \Big|^2+\Big| \sum_n (-1)^n q^{n^2/2} \Big|^2+\Big|\sum_n q^{\frac12(n + 1 / 2)^2}\Big|^2\right]\,.
$$
    Verify that this is the torus partition function of a free complex fermion with all the possible boundary conditions (AA, AP, PA, PP).  This is called the \textbf{diagonal modular invariant} partition function.

%
% \section{Check the central charge}
% The world-sheet action of Heterotic string theory is given by
% \begin{align}
%  S^{\textrm{m}} &= \frac{1}{4\pi} \int d^2 z\ \Big( \frac{2}{\alpha'} \partial X^\mu  \overline\partial X_\mu+\psi^\mu\overline\partial\psi_\mu+\overline \lambda^A\partial\overline\lambda_A\Big)\cr
% S^{\textrm{gh}}&=\frac{1}{2\pi}\int d^2z \ (b\overline \partial c+\bar b \partial \bar c+\beta\overline \partial \g)\nonumber
% \end{align}
% where $\mu$ are 10-dimensional indices and the right-moving sector is supersymmetric. In the lecture, I have explained that if there are 32 left-moving fermions $\overline\lambda^A$, there is no Weyl-anomaly $c^{\textrm{tot}}=0$. Explain in detail the value of each contribution $c^X,c^\psi,c^{bc},c^{\beta\gamma},c^{\lambda}$ in the lecture note.
%
%



\section{Toroidal compactifications}

Let us consider the toroidal compactification in the presence of $B$-field
$$
S=-\frac{1}{4\pi\a'}\int d^2\sigma(\d_{IJ}\eta^{\a\b}+ B_{IJ}\e^{\a\b})\partial_\a X^I\partial_\b X^J~,
$$
where $\e^{01} =-1$ and $\eta^{\a\b}=\textrm{diag}(-1,1)$. The bosonic field is quantized as in the lecture note
\begin{align}
X_R^I(z)&=x^I-i\sqrt{\a'\over2}p_R^I
\ln z+i\sqrt{\a'\over2}\sum_{m\neq0}{\alpha^I_m\over mz^m}, \nonumber\\
\overline X_L^I(\bz)&={\overline x}^I-i\sqrt{\a'\over2}{p^I_L}
\ln \bz+i\sqrt{\a'\over2}\sum_{m\neq0}{{\overline\alpha}^I_m\over m\bz^m}.\nonumber
\end{align}
where the bosonic field $X ^I(z,\bz)=X_R^I(z)+\overline X_L^I(\bz)$ is periodically identified on the lattice $W^I\in \Lambda$
\be\label{bc}
X ^I( \s+2\pi,\t)=X ^I( \s,\t)+2\pi W^I~.
\ee




\subsection{} Show that  the center of mass momentum
$$
\pi_I:=\int_0^{2\pi} d\sigma ~ \Pi_I~,\qquad  \Pi_I=\frac{\d S}{\d \dot{X}^I}~,
$$
together with \eqref{bc} imply that
$$
(p_I)_{L,R} :=\d_{IJ}p^J_{L,R} = \sqrt{\frac{\a'}{2}}\Big( \pi_I  \pm \frac1{\a'}(\d_{IJ} \mp B_{IJ})W^J \Big)~.
$$
Here left-modes get the upper sign and right-modes get the lower sign whenever you find the notation $\pm$ and $\mp$.



\subsection{}  It is $\pi_I$ that generates translations and it must therefore lie on the lattice $\Lambda^*_D$, $\pi_I=e^{*i}_Im_i$ for $m_i\in\bZ$. Writing $\mathbf{g}=g_{ij} = e^I_i \d_{IJ }e^J_j $, $\mathbf{b} = b_{ij} = e^I_i B_{IJ} e^J_j$,
show that the mass formula and the level-matching condition  for closed strings are modified to (a matrix notation is employed in the expression below)
\begin{align}\label{mass-level}
\a' M^2=&{\a'}\bfm^T\bfg^{-1}\bfm+\frac{1}{\a'}\bfn^T(\bfg-\bfb\bfg^{-1}\bfb)\bfn+2\bfn^T\bfb\bfg^{-1}\bfm+2(N +\overline N -2)\cr
N-\overline N=&{\boldsymbol \pi}\cdot \bfW=\bfm^T \bfn~.
\end{align}


\subsection{}
Use \eqref{mass-level} to show that the spectrum is invariant under the map
$$
\bfm\leftrightarrow\bfn~,\quad \a' \bfg^{-1}\leftrightarrow \frac1{\a'} (\bfg-\bfb\bfg^{-1}\bfb)~, \quad  \bfb\bfg^{-1}\leftrightarrow -\bfg^{-1} \bfb~.
$$
The second and the third are indeed equivalent to
$$
\frac{1}{\a'}(\bfg+\bfb)\leftrightarrow \a'(\bfg+\bfb)^{-1}~,
$$
and this is the T-duality in the presence of $B$-field.


\subsection{} A non-trivial instructive example is the 2-torus $T^2= \bR^2/\Lambda$ where $\bfe_1,\bfe_2$ are the two generators of $\Lambda$ and its metric is $g_{ij}=\bfe_i\cdot \bfe_j$.
The torus has one K\"ahler modulus $\sqrt{\det g_{ij}}$ and one complex structure modulus
$$
\tau=\frac{|\bfe_2|}{|\bfe_1|}e^{i \,\textrm{angle}(\bfe_1,\bfe_2)}=\frac{g_{12}+i \sqrt{\det g_{ij} }}{g_{11}}=\tau_1+i\tau_2~.
$$
In the presence of $B$-field, the K\"ahler modulus  is complexified
$$
\omega=\frac{1}{\a'}(B+i \sqrt{\det g_{ij}})=\omega_1+i\omega_2~.
$$
This leads to the expression
$$
g_{ij}=\a'\frac{\omega_2}{\tau_2}\begin{pmatrix} 1&\tau_1\\ \tau_1 & |\tau|^2\end{pmatrix}~.
$$
Using these expressions, show that the momentum contribution to the mass formula is given by
\begin{align}
\bfp_L^2&=\frac{1}{2\omega_2\tau_2}|m_2-\t m_1+\overline{\omega}(n^1 + \t n^2)|^2  \cr
\bfp_R^2&=\frac{1}{2\omega_2\tau_2}|m_2-\t m_1+{\omega}(n^1 + \t n^2)|^2~.\nonumber
\end{align}
Show that, if we transform the momentum and winding numbers appropriately, the string spectrum is then invariant under the following two $\SL(2,\bZ)$ transformations


\noindent $\bullet$ T-dualities
$$
\omega\to\frac{a\omega+b}{c\omega+d}
$$
$\bullet$ Torus modular transformations
$$
\tau\to\frac{a\tau+b}{c\tau+d}
$$
$\bullet$  \textbf{mirror symmetry} (symmetry between complexified K\"ahler modulus and complex structure modulus)
$$
\omega\leftrightarrow \tau
$$
How do the momentum and winding numbers transform?


%Let us now figure out what is the moduli space of such
%a compactification.
%Any nontrivial $O(D,D)$ rotation
%would map one even self-dual Lorentzian
%lattice into a different one.
%The converse is a mathematical fact:
%any two even self-dual Lorentzian
%lattices are related by some $O(D,D)$
%rotation.  Therefore the space of such lattices is simply $O(D,D)$.
%However, not all of them correspond to different compactifications.
%The spectrum for the $(26-D)$-dimensional theory is determined
%by $p_L^2$ and $p_R^2$.  They are left invariant by
%$O(D)\times O(D)$, the maximal compact subgroup of $O(D,D)$,
%acting independently on the left and right momenta respectively.
%Therefore
%the space of vacua is \textbf{locally} $O(D,D) / (O(D)\times O(D))$,
%of dimension $D^2$.



\end{document}
