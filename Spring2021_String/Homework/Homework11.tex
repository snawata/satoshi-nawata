\documentclass[12pt,a4paper]{article}

\usepackage{macros}
\begin{document}


\centerline{\Large \bf  Homework 11: Due at class on May 30}




\section{Electromagnetic duality}

\subsection{Differential form}


An $n$-form field is defined by
\begin{align}
 A_n = \frac{1}{n!} A_{\mu_1,\mu_2,\cdots,\mu_n} dx^{\mu_1} \wedge dx^{\mu_2} \wedge \cdots \wedge dx^{\mu_n} \ .
\end{align}
Hodge dual in a $D$-dimensional curved space is defined by
\begin{align}
 * A_n = \frac{\sqrt{|G|}}{n! (D-n)!} {\epsilon_{\mu_1,\cdots,\mu_{D-n}}}^{\mu_{D-n+1},\cdots,\mu_{D}} A_{\mu_{D-n+1},\cdots,\mu_D} dx^{\mu_1} \wedge \cdots \wedge dx^{\mu_{D-n}} \ ,
\end{align}
where $G$ is the determinant of a metric $G_{MN}$, and
$\epsilon_{\mu_1,\cdots,\mu_D}$ is the totally anti-symmetric tensor and normalized as $\epsilon_{0,1,\cdots,D-1} = 1$.
\begin{itemize}
 \item Confirm that
\begin{align}
 \frac{1}{2g^2} \int {F_2}^2 \equiv \frac{1}{2g^2} \int F_2 \wedge * F_2
 = \frac{1}{4g^2} \int \sqrt{|G|} d^Dx\ F_{\mu\nu} F^{\mu\nu} \ .
\end{align}
(For example, compare the coefficients of $F_{01}F^{01}$ term in both side.
If it is still difficult consider $D=4$.)
\end{itemize}
Comment: this is the convention used in the lecture for kinetic terms of anti-symmetric tensors in SUGRA.


\subsection{ Electromagnetic duality}

Let us consider a following action
\begin{align}
 S = -\frac{1}{2g^2} \int F_{n+1} \wedge * F_{n+1} - \int f_{D-n-1} \wedge \left( F_{n+1} -d A_n \right) \ ,
\end{align}
where $F_{n+1}$ and $A_n$ are independent each other.
\begin{itemize}
 \item Consider an equation of motion for $f_{D-n-1}$ and derive the solution for the E.O.M.
       Then, what is the physical meaning of $F_{n+1}$ and $S$ with the solution ?
 \item Consider an E.O.M. for $F_{n+1}$ and derive the solution.
       Rewrite the action $S$ in terms of $f_{D-n-1}$ using the solution.
 \item Consider an equation of motion for $A_n$ with the action $S(f_{D-n-1})$ and derive the solution for the E.O.M.
       Then, what is the physical meaning of $f_{D-n-1}$ and $S(f_{D-n-1})$ with the solution ?
 \item Define $\wt S = (\mathrm{sign})S$, where you should properly choose the $(\mathrm{sign})$ so that
       the kinetic term has the usual sign. If we write $\wt S \equiv -\frac{1}{2\wt g^2} \int \wt F_{D-n-1} \wedge * \wt F_{D-n-1}$,
       what is the value of $\wt g$ in terms of $g$ ?
\end{itemize}
(Again, if you are confused with the differential form convention get back to normal convention and consider $D=4$.)




\section{Dirac monopole}



\subsection{Wu-Yang monopole}

Let us consider so called Dirac monopole, which is a field configuration of $\mathbf B$
that satisfies
\begin{align}
 \nabla \cdot \mathbf B = q_m \delta^3 (\mathbf r) \ ,
 \label{eq:1}
\end{align}
where $\delta^3 (\mathbf r)$ is a three-dimensional delta function and $\mathbf r$ is a three-dimensional space point vector.
\begin{itemize}
 \item Derive a solution of $\mathbf B$.
 \item Explain that $\mathbf B$ cannot be expressed by space components of a gauge field $\mathbf A$ globally.
\end{itemize}
Consider the polar coordinate $(r,\theta,\phi)$ and a following gauge field
\begin{align}
 \mathbf A^N &= \frac{q_m(1-\cos\theta)}{4\pi r\sin\theta} \mathbf e_\phi  ,  \\
 \mathbf A^S &= -\frac{q_m(1+\cos\theta)}{4\pi r\sin\theta} \mathbf e_\phi  ,
\end{align}
where $\mathbf A^N$ is defined in a region $r\neq 0$ and $\theta\neq \pi$,
and $\mathbf A^S$ is defined in a region $r\neq 0$ and $\theta\neq 0$.
(Comments: The singular lines are called Dirac string. Those solutions are called Wu-Yang monopole.
The point is that a gauge field $A_\mu$ is a section of fiber bundle and is not necessarily defined globally.)
\begin{itemize}
 \item Show that $\mathbf A^N-\mathbf A^S$ is a gauge transformation in the region $r\neq 0$ and $\theta\neq 0, \pi$.
 \item Show that both $\mathbf A^N$ and $\mathbf A^S$ lead the $\mathbf B$ derived above in the region where they are defined.
 \item Show that magnetic flux $\Phi$ from the monopole is $q_m$, using $\mathbf A^N$ and $\mathbf A^S$.
\end{itemize}
The statements so far are based on vector analysis.
More properly, a gauge field is a 1-form $A_1$ and magnetic flux density is a 2-form $B_2$.
Let us rewrite the statements in terms of differential forms.
\begin{itemize}
 \item Explain that Eq.~(\ref{eq:1}) can be expressed as $dF_2 = q_m \delta_3(\mathbf r)$,
       where $\delta_3(\mathbf r)$ is a 3-form delta function:
       $\delta_3(\mathbf r) = \delta^3(\mathbf r) d^3\mathbf r \left(= \delta(x)\delta(y)\delta(z) dx\wedge dy\wedge dz \right)$.
 \item Explain the Wu-Yang monopole can be written by
 \begin{align}
  A_N = \frac{q_m}{4\pi}(1-\cos\theta) d\phi \ , \qquad A_S = -\frac{q_m}{4\pi}(1+\cos\theta) d\phi \ .
 \end{align}
 \item Show that $A_N-A_S$ is a gauge transformation.
 \item Show that magnetic flux $\Phi$ from the monopole is $q_m$, using $A_N$ and $A_S$ in terms of differential forms.
\end{itemize}


\subsection{Dirac quantization condition}
The previous argument in differential forms can be easily generalized to higher dimension.
Let us consider a $(p+1)$-form gauge field $C_{p+1}$, which satisfies following equation of motion.
\begin{align}
 \frac{1}{2\kappa^2} dG_{p+2} = q_m \delta_{p+3}(M_{D-p-3}) \ ,
\end{align}
where $G_{p+2}$ is a field strength of $C_{p+1}$, $\delta_{p+3}(M_{D-p-3})$ is a $(p+3)$-form delta function,
and $M_{D-p-3}$ is a world-sheet of an object that is magnetically coupled to $C_{p+1}$ (Consider $D=4$ and $p=0$, which is the monopole case).
\begin{itemize}
 \item (Optional) Explain the origin of $2\kappa^2$.
 \item Show that $G_{p+2}$ integrated over a sphere enclosing the magnetic object is $2\kappa^2 q_m$.
\end{itemize}


Let us consider an object that is electrically coupled to $C_{p+1}$, whose coupling is
\begin{align}
 S_E = q_e \int_{E_{p+1}} C_{p+1} = q_e \int_{\wh E_{p+2}} G_{p+2} \equiv S_E(\wh E_{p+2}) \ ,
\end{align}
where $E_{p+1}$ is a world-sheet that starts from $t=-\infty$ and ends at $t=\infty$
or starts at $\mathbf x_0$ and ends at $\mathbf x_0$ (closed path), and $\wh E_{p+2}$ is a manifold whose boundary is $E_{p+1}$.
\begin{itemize}
 \item Since a choice of $\wh E_{p+2}$ is arbitrary one can consider a deformation of $\delta \wh E_{p+2} = \wh E_{p+2}^N -\wh E_{p+2}^S$,
       and then, $e^{iS_E(\delta\wh E_{p+2})} =1$. Show that this leads Dirac quantization condition.
\end{itemize}
% Classically the world-sheet is fixed, however, quantum mechanically any world-sheet has to be considered.
% Therefore, if there exist the magnetic object we have to consider two world-sheet $E_{p+1}^N$ and $E_{p+1}^S$
% such that $E_{p+1}^N -E_{p+1}^S$ enclose the magnetic object, or a closed path enclosing the magnetic object.
% \begin{itemize}
%  \item Show that $e^{iS_E}$
% \end{itemize}
% \begin{itemize}
%  \item (Optional) Show that gauge invariance of the action assure that a deformation of the world-sheet $E_{p+1}$ does not change the action
%        up to $2\pi n$ ($n\in \ZZ$).
%  \item Assume that a deformation of $E_{p+1}$ enclose the magnetic object and the deformation does not change the action,
%        and then, show that it leads Dirac quantization condition $2\kappa^2 q_e q_m \in 2\pi \ZZ$.
% \end{itemize}





\section{ $SL(2,\RR)$ invariance of type IIB SUGRA}


\begin{itemize}
 \item Show that the $SL(2,\RR)$ invariance of the type IIB SUGRA.
\end{itemize}




\section{ Kaluza-Klein theory}

Let us consider a $D+1$-dimensional real scalar Kaluza-Klein theory:
\begin{align}
 \cL^{(D+1)} = -\frac{1}{2} G^{MN} \partial_M \phi \partial_N \phi \ .
\end{align}
The metric is given by
\begin{align}
 ds^2 = G_{MN} dx^M dx^N = g_{\mu\nu} dx^\mu dx^\nu + e^{2\sigma} \left( dy +A_\mu dx^\mu \right)^2 \ ,
 \label{eq:2}
\end{align}
where $M,N$ run from $0$ to $D$, $\mu,\nu$ run from $0$ to $D-1$,
and $y \equiv x^D$, which is compactified to $S^1$ with radius $r$.
\begin{itemize}
 \item Rewrite the Lagrangian in terms of $\mu,\nu$, and $y$, rather than $M,N$.
 \item Calculate the determinant of $G_{MN}$ (i.e. $G \equiv \det G_{MN}$) and express it in terms of $g \equiv \det g_{\mu\nu}$.
 \item Assume that the fields $(g_{\mu\nu}, A_\mu, \sigma)$ are independent of $y$, except $\phi$,
       which can be expanded as $\phi(\mathbf x,y) = \sum_{n \in \ZZ} \phi_n(\mathbf x)e^{iny/r},\ (\phi_n^*(\mathbf x)=\phi_{-n}(\mathbf x))$.
       Write down the effective Lagrangian of $D$-dimensional theory:
 \begin{align}
  \sqrt{-g}\cL^{(D)} = \int dy \sqrt{-G} \cL^{(D+1)} \ .
 \end{align}
 \item Derive the masses of the fields $\phi_n$, and their coupling constants to the KK-gauge field $A_\mu$.
\end{itemize}


Let us consider a gravi-dilaton KK-theory:
\begin{align}
 S_{EH}^{(D+1)} \simeq \frac{1}{2\kappa^2} \int d^{D+1}x \sqrt{-G} e^{-2\Phi} \Big(R +\cdots \Big) \ ,
\end{align}
which, with the same metric Eq.~(\ref{eq:2}), reduces to
\begin{align}
 S_{EH}^{(D)} \simeq \frac{2\pi L}{2\kappa^2} \int d^{D}x \sqrt{-g} e^{-2\Phi} \Big(R +\cdots \Big) \ ,
\end{align}
where we defined $L = r e^{-\sigma}$, and $\cdots$ includes the dilaton kinetic term and other terms that are not important here.
\begin{itemize}
 \item Define an effective coupling $\frac{1}{2\kappa_\mathrm{eff}^2} = \frac{2\pi L}{2\kappa^2}e^{-2\langle\Phi\rangle}$,
       and compare it with that of T-dual theory.
       $\frac{1}{2\kappa_\mathrm{eff}^2}$ should be the same after the T-dual.
       Derive the value of $\langle\wt \Phi\rangle$ in terms of $\langle\Phi\rangle$,
       where $\langle\wt \Phi\rangle$ is a dilaton vev of the T-dual theory.
\end{itemize}


\end{document}
