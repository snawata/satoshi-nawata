\documentclass[String-lecture-21.tex]{subfiles}

\begin{document}\section{String dualities}\label{sec:dualities}




We have introduced superstring theories of five types: Type IIA, IIB, I, and Heterotic $\SO(32)$ and $E_8\times E_8$.
In fact, the seminal paper \cite{Witten:1995ex} reveals that these string theories are related by dualities, which led to the second string revolution. Indeed, we have already learned

\vspace{.3cm}
\noindent $\bullet$ Type IIA and IIB are T-dual to each other

\vspace{.3cm}
\noindent $\bullet$ Type I is the orientifold projection of Type IIB

\vspace{.3cm}
\noindent In this section, we will learn string dualities extensively studied in the second string revolution after \cite{Witten:1995ex}.

\vspace{.3cm}
\noindent $\bullet$  Type IIB has $\SL(2,\bZ)$ symmetry so that it is self-dual under S-duality

\vspace{.3cm}
\noindent $\bullet$  The strong coupling regime of Type IIA is described by M-theory on $S^1$


\vspace{.3cm}
\noindent $\bullet$  Heterotic $\SO(32)$ and $E_8\times E_8$ are T-dual to each other

\vspace{.3cm}
\noindent $\bullet$  Heterotic $\SO(32)$ is S-dual to Type I

\vspace{.3cm}
\noindent $\bullet$ The strong coupling regime of  Heterotic $E_8\times E_8$ is described by M-theory on $S^1/\bZ_2$

\vspace{.3cm}
\noindent $\bullet$  Heterotic string on $T^4$ is dual to Type IIA on K3

\begin{figure}[ht]\centering
\includegraphics[width=\textwidth]{picture/duality-web}
\caption{Duality web of string theory}
\end{figure}

Even for these dualities, we can cover only key points in this section. More details can be found in  \cite{Polchinski,BBS}. Moreover, we just see the tip of the iceberg, and there are much more string dualities. Thus, we refer to good reviews  \cite{Aspinwall:1996mn,Forste:1996yd,Ooguri:1996ik,Polchinski:1996nb,Polchinski:1996na,Townsend:1996xj,Schwarz:1996bh,Sen:1996yy,Dijkgraaf:1997ip,Vafa:1997pm,Sen:1998kr,Johnson:2000ch} written during the second string revolution for this rich subject. All in all, these dualities tell us that quantum strings somehow see geometry from drastically different viewpoints. I hope you will get some feeling of it in this section.




\subsection{S-duality of Type IIB supergravity}\label{sec:Sdual}

The electromagnetic duality seen in \eqref{Maxwell} and \eqref{Maxwell2} is a basic example of duality. The generalization to the Yang-Mills theory is proposed by Goddard-Nuyts-Olive \cite{Goddard:1976qe} and Montonen-Olive \cite{Montonen:1977sn}. The Montonen-Olive duality exchanges the Yang-Mills coupling by
\be
g_{\textrm{YM}}\leftrightarrow 1/g_{\textrm{YM}}~,
\ee
and it also exchanges a fundamental particle into a soliton. Thus, this duality is also called the \textbf{strong-weak duality} or \textbf{$S$-duality}.
Although standard techniques in quantum field theory cannot be applied to the strong coupling regime, the $S$-duality provides new insights into non-perturbative dynamics in the strong-coupling regime. One of the reasons why string theory sheds new light on physical theories is that the strong-weak dualities show up in string theory \cite{Sen:1994fa,Duff:1994zt}, which often have geometric origins.



In fact, Type IIB string theory enjoys the $S$-duality. To see it,
let us rewrite the action by rescaling the metric $G_{E,\mu\nu} = e^{-\Phi/2} G_{\mu\nu}$, and by introducing
\be\label{axiodilaton}\tau = C_{(0)} +ie^{-\Phi}~,\ee
and
\begin{align*}
 \mathbb M = \frac{1}{\Im \tau}
 \begin{pmatrix}
  |\tau|^2 & -\Re \tau \\ -\Re \tau & 1
 \end{pmatrix} \ , \qquad
 \mathbb F_{(3)} =
 \begin{pmatrix}
  H_{(3)} \\ G_{(3)}
 \end{pmatrix} \ ,
\end{align*}
Then, the action \label{IIB-supergravity} is rewritten as
\begin{align*}
 S_\mathrm{B} =& \frac{1}{2\kappa_{10}^2} \int d^{10}x \sqrt{ -G_E} \left[
 R_E - \frac{\partial_\mu \tau \partial^\mu \ol\tau}{2(\Im \tau)^2}
 -\frac{1}{2} \mathbb F_{(3)} \cdot \mathbb M \cdot \mathbb F_{(3)}-\frac{1}{4} \wt G_{(5)}^2
 \right]\cr
&  -\frac{1}{4\kappa_{10}^2} \int C_{(4)} \wedge \mathbb F_{(3)}^T \wedge \epsilon \mathbb F_{(3)} \ ,
\end{align*}
where $\epsilon = \begin{pmatrix} 0 & 1 \\ -1 & 0 \end{pmatrix}$. Remarkably,
this action is invariant under the $\SL(2,\mathbb R)$ transformation:
\begin{align*}
 \tau' = \frac{a\tau +b}{c\tau +d} \ , \quad
 \mathbb M' = (\Lambda^{-1})^T \mathbb M \Lambda^{-1} \ , \quad
 \mathbb F_{(3)}' = \Lambda \mathbb F_{(3)} \ , \quad
 \Lambda =
 \begin{pmatrix}
  d & c \\ b & a
 \end{pmatrix}\in \SL(2,\bR)\ .
\end{align*}
On the other hand, the metric $G_E$ and $C_{(4)}$ are invariant under the transformation.

Let us consider its physical implications.
Since  $C_{(4)}$ is invariant, a D$3$-brane is invariant. On the other hand, since the  2-form fields $B_{(2)}$ and $C_{(2)}$ are transformed by $\SL(2,\bR)$, the extended objects coupled to these fields are transformed accordingly. Namely, F$1$ and D$1$-branes are electrically coupled to the 2-form fields whereas
NS$5$ and D$5$-branes are magnetically coupled to them, respectively.
Thus, they are transformed under $\SL(2,\bR)$ as
\be  \begin{pmatrix}
  \mathrm{F1}' & \mathrm{D1}'
 \end{pmatrix} =
 \begin{pmatrix}
  \mathrm{F1} & \mathrm{D1}
 \end{pmatrix} \Lambda^{-1} ~,\qquad
\begin{pmatrix}
 \mathrm{NS5}' \\ \mathrm{D5}'
\end{pmatrix} = \Lambda
\begin{pmatrix}
 \mathrm{NS5} \\ \mathrm{D5}
\end{pmatrix} \ ,\qquad \textrm{as}\quad    \begin{pmatrix}
  B_{(2)}' \\ C_{(2)}'
  \end{pmatrix} = \Lambda
  \begin{pmatrix}
  B_{(2)} \\ C_{(2)}
 \end{pmatrix}\nonumber
\ee
Due to the Dirac quantization condition, the electric and the magnetic charges of D-branes
must be integers so that the true symmetry in Type IIB string theory is indeed $\SL(2,\ZZ)$ \cite{Hull:1994ys}.

In particular, if we choose \[\Lambda = S =
\begin{pmatrix}
 0 & 1 \\ -1 & 0
\end{pmatrix}~,\]
the fields transform as
$\tau \leftrightarrow -1/\tau$, and the extended objects are exchanged as \[\mathrm{F1} \leftrightarrow \mathrm{D1}~, \qquad \mathrm{NS5} \leftrightarrow \mathrm{D5}~.\]
Assuming that $\langle C_0 \rangle = 0$, this results in the S-duality of the string coupling constant, $g_s \leftrightarrow 1/g_s$.
Other elements of $\SL(2,\ZZ)$ lead to infinitely many bound states of F$1$ and D$1$: $(p,q)$-string,
and those of NS$5$ and D$5$: $(p,q)$ 5-branes.

The combination \eqref{axiodilaton} of the fields behaves as a complex structure of a torus and the theory enjoys the modular transformations
\eqref{eq:ModularTra} and \eqref{eq:ModularTra2} of a torus. Hence, a $D=12$ dimensional theory, called \textbf{F-theory}, is proposed in \cite{Vafa:1996xn,Morrison:1996na,Morrison:1996pp} by considering a torus fibration over ten-dimensional spacetime of Type IIB theory, which can be interpreted as the geometric origin of the field combination \eqref{axiodilaton}.


\subsection{M-theory}

We would also like to understand the strong coupling region of Type IIA theory.
As we obtain Type IIA supergravity by compactification of the $D=11$ supergravity on $S^1$, the radius is proportional to the string coupling constant  \eqref{Mcircle-radius} in the unit of the string length.
This suggests that in the strong coupling region $g_s\gg1$ of Type IIA string theory, another spacetime direction
emerges (dimensional decompactification).
Although the strong coupling behavior may look rather bizarre, it turns out that a number of phenomena in string theory can be explained ``more naturally'' from the  $D=11$ strongly-coupled Type IIA ``string theory''.
Witten calls the $D=11$ ``string theory'' \textbf{M-theory} where M stands for Magic, Mystery, or Membrane according to him.
% Some people also include Matrix, etc. so why not come up with your own M!


Since the $D=11$ supergravity is endowed with the 3-form fields $M_{(3)}$,
there must be corresponding objects coupled to the field electrically and magnetically.
They are called \textbf{M2-brane} and \textbf{M5-brane},
which are $(1+2)$- and $(1+5)$-dimensional objects, respectively.
Let us assume that they have the following tensions and charges
\begin{align*}
 T_{M2} = \mu_{M2} = \frac{2\pi}{(2\pi \ell_p)^3} \ , \qquad
 T_{M5} = \mu_{M5} = \frac{2\pi}{(2\pi \ell_p)^6} \ .
\end{align*}

The compactification of M-theory on a circle $S^1$ leads to
the IIA superstring theory. Since it is compactified on $S^1$, the momenta of a particle along $S^1$ are quantized as $n/R$ $(n\in \bZ)$, and it can be identified with D0-branes. An M2-brane on $S^1$ becomes a fundamental string whereas an M2-brane in $D=10$ reduces to a D2-brane. Similarly, an M5-brane on $S^1$ becomes a D4-brane whereas an M5-brane in $D=10$ reduces to an NS5-brane. A D6-brane arises geometrically as a Kaluza-Klein monopole, which we do not deal with in this lecture note. (See \cite[\S15.2]{johnson2002d} for instance.)
Let us see this by comparing the branes and their tensions in Table~\ref{table:tension}. One can convince oneself that the tensions of extended objects perfectly agree. Note that the tensions and charges of D-branes in Type IIA theory are related by the string coupling constant
\be
g_sT_{\textrm{D}p}^\textrm{eff}=\mu_p~.
\ee
In fact, charges of a D$p$ and D$(6-p)$-brane can be read off from Table~\ref{table:tension}, and it is easy to check that they are subject to the Dirac quantization condition \eqref{Dirac-quantization}. We will come back to this relation in \S\ref{sec:DBI}.


Note that M-theory is \emph{not} even defined in a sense that we do not know how to quantize
the M-branes. For instance, the world-volume theory on M5-branes is endowed with $\operatorname{OSp}(2,6|2)$ superconformal symmetry, called 6d $\cN=(2,0)$ superconformal field theory \cite{Witten:1995zh}. The field contents are as follows
\begin{itemize}\setlength{\parskip}{0.0cm}
 \item 2-form tensor field $B$ with self-dual field strength $dB=H=-* H$
\item spinors $\psi_{\alpha,a}$ with $\psi_{\alpha,a}=J_{\alpha \beta} J_{a b} \bar{\psi}^{\beta,b}$ ($\alpha,a=1,2,3,4$)
\item 5 scalars $\phi_{i}$ $(i=1,\ldots,5)$
\end{itemize}
The source for the tensor field is an M2-brane ending on the M5-brane, called self-dual string, $*dH=J$. However,
we do not know how to write down Lagrangian of the theory for nonabelian cases because of the self-dual 3-form $H$:
\begin{equation}
  \frac{1}{g^2}\int d^6x~ H\wedge *H = - \frac{1}{g^2}\int d^6x ~H\wedge H=0~.
\end{equation}
It is believed that 6d $\cN=(2,0)$ SCFTs are classified by $A_{n}, D_{n}, E_{6,7,8}$ root system.

Nevertheless, the existence of such theories tells us a lot.
Especially, even though no effective description of M$5$-branes in flat space is known,
M$5$-branes wrapped on manifolds give rise to surprising dualities between
a $d$-dim topological theory and $(6-d)$-dim supersymmetric gauge theories.
A salient example is the AGT relation \cite{Alday:2009aq} where M$5$-branes are wrapped on Riemann surface. Also, the highest dimension of SCFTs is $D=6$ \cite{Nahm:1977tg}, and string theory is indispensable for the study of 6d SCFTs. The reader is referred to \cite{Heckman:2018jxk} for this subject.



\begin{table}[htbp]
 \begin{center}
\begin{tabular}{c|cccccc}
 Dimension & $0$ & $1$ & $2$ & $4$ & $5$ & $6$ \\
 \hline
 M on $S^1$ & KK-mom. & M2/$S^1$ & M2 & M5/$S^1$ & M5 & KK-mono. \\
 & $\frac{1}{R}$ & $\frac{2\pi\cdot2\pi R}{(2\pi \ell_p)^3}$ & $\frac{2\pi}{(2\pi \ell_p)^3}$ & $\frac{2\pi\cdot2\pi R}{(2\pi \ell_p)^6}$ & $\frac{2\pi}{(2\pi \ell_p)^6}$ & $\frac{2\pi(2\pi R)^2}{(2\pi \ell_p)^9}$ \\
 \hline
 IIA & D$0$ & F$1$ & D$2$ & D$4$ & NS$5$ & D$6$ \\
 & $\frac{2\pi}{g_s (2\pi \ell_s)}$ & $\frac{2\pi}{(2\pi \ell_s)^2}$ & $\frac{2\pi}{g_s (2\pi \ell_s)^3}$ & $\frac{2\pi}{g_s (2\pi \ell_s)^5}$ & $\frac{2\pi}{g_s^2 (2\pi \ell_s)^6}$ & $\frac{2\pi}{g_s (2\pi \ell_s)^7}$ \\
\end{tabular}
\end{center}
\caption{Wrapped/unwrapped M-branes and the corresponding extended objects in Type IIA theory
with their effective tensions.}
\label{table:tension}
\end{table}


\subsection{Heterotic T-duality} \label{sec:HeteroticT}

Let us consider the T-duality in Heterotic strings on a circle $S^1$ in \cite{Narain:1985jj,Narain:1986am,Ginsparg:1986bx}. In the bosonic construction, the bosonic left-moving sector is compactified on an even self-dual Euclidean lattice of 16-dimensions. There are only two such lattices: the weight lattice $\G_{\SO(32)}$ of $\SO(32)$ and the root lattice $\G_{E_8}\oplus \G_{E_8}$ of $E_8\times E_8$ as we have seen in \S\ref{sec:Heterotic-bosonic}.


One may also describe the compactification on a circle $S^1$ in terms of lattices. As we have seen, the left-moving and right-moving momenta compactified boson takes the value on the lattice $\Gamma^{1,1}$ of the Lorentzian signature. Hence, the circle compactification results in adding ($\oplus$) the lattice $\Gamma^{1,1}$ to the original lattice.



It is a useful mathematical fact that for Lorentzian
lattices, there is
only unique even unimodular Lorentzian lattice for each rank. Therefore, the theorem implies
\[\G_{\SO(32)}\oplus \Gamma^{1,1} \cong\G^{1,17}
\cong  \G_{E_8}\oplus \G_{E_8}\oplus \Gamma^{1,1}~.\]
Together with the metric $G$ and the $B$-field,
they parameterize the
moduli space
\begin{equation}\label{eq:Mhettoroidal}
{\cal M}=\left. \frac{\mathrm{O}(1,17)}{\mathrm{O}(1)\times \mathrm{O}(17)}\right/ \mathrm{O}(1,17; \bZ) ,
\end{equation}
where $\mathrm{O}(1,17; \bZ)$ is the T-duality group. It is called \textbf{Narain moduli space}.
Different points in the moduli space correspond to physically distinct compactifications, e.g. the gauge groups can be different, although always of rank 18. At generic points, it is $\U(1)^{18}$ that corresponds to the fact that Wilson loops generically break the 10d gauge group to $\U(1)^{18}$.



However, there are special subspaces of the moduli space where it is enhanced.
This moduli space has exactly two asymptotic boundary points, one associated to the
decomposition $\G^{1,17}\cong \G_{E_8}\oplus \G_{E_8}\oplus\G^{1,1}$,
and the other to the decomposition $\G^{1,17}
\cong \G_{\SO(32)}\oplus\G^{1,1}$.
Therefore, these boundary points are associated to Type HE and HO string, or large and small radii.
T-duality will relate these boundary points.



In fact, starting from either Heterotic theory, there is a simple choice of
Wilson line which breaks the gauge group to
$\SO(16)\times \SO(16) \times \U(1)\times  \U(1)$.   If we leave this
group unbroken, then the only remaining parameter is the radius. An analysis of the massive states shows
that if we map $R\to 1/R$ while exchanging KK momenta
and winding modes, then the two Heterotic theories are exchanged \cite[\S11.6]{Polchinski}.



More generally, upon a compactification of Heterotic strings on a $D$-dimensional torus $T^D$, momenta take the values on an even self-dual Lorentzian lattice $\G^{D,D+16}$. Therefore, the Narain moduli space becomes
\be\label{Narain}
{\cal M}=\left. \frac{\mathrm{O}(D,D+16)}{\mathrm{O}(D)\times \mathrm{O}(D+16)}\right/ \mathrm{O}\left(D,D+16; \bZ\right)~.
\ee
which is  $D(D+16)$-dimensional.





\subsection{S--duality between Type I and Heterotic \texorpdfstring{$\SO(32)$}{SO(32)} }\label{sec:typeI-HO}


Now let us see S-duality between Type I theory  and Heterotic  $\SO(32)$ theory  \cite{Witten:1995ex,Dabholkar:1995ep,Hull:1995nu,Polchinski:1995df} from low-energy effective actions.



One can obtain Type I supergravity action from Type IIB by setting to zero the IIB fields $C_0$, $B_{(2)}$, and $C_4$ that are removed by the $\O$ projection. In addition, we include $\SO(32)$ gauge fields with appropriate dilaton dependence
\begin{align}\label{TypeI}
S_{\textrm{I}}&=S_{\textrm{grav}}+S_{\textrm{YM}}\cr
S_{\textrm{grav}}&= \frac{1}{2\kappa_{10}^2} \int d^{10}x \sqrt{ -G}   \,\left[
e^{-2\Phi}( R +4 \partial_\mu \Phi \partial^\mu \Phi )-\frac{1}{2}| \wt G_{(3)}|^{2} \right] \cr
S_{\textrm{YM}}&=- \frac{1}{2g_{10}^2} \int d^{10}x \sqrt{ -G}\,  e^{-\Phi} \Tr  |F_{(2)}|^2
\end{align}
where $F_{(2)}$ is the $\SO(32)$ field strength and the trace is in the adjoint representation. Here $G_{(3)}$ is the field strength of the R-R 2-form $C_{(2)}$
\[
\wt G_{(3)}=dC_{(2)}-\frac{\kappa_{10}^2}{g_{10}^2}\omega_3
\]
with  the Chern-Simons 3-form
\[
\omega_3=\Tr \Big(AdA-\frac{2i}3 A^3\Big)~.
\]
The gauge coupling constant $g_{10}$ and the gravitational constant $\kappa_{10}$ are related by $\kappa_{10}^2/g_{10}^2 = \a'/4$, which is determined by anomaly cancellation. Under the gauge transformation $\delta A=d\lambda-i[A,\lambda]$, the Chern-Simons term transforms as
\[
\delta \omega_3=d \Tr(\lambda A)
\]
Hence, it comes with
\[
\d C_{(2)} = \frac{\kappa_{10}^2}{g_{10}^2} \Tr(\lambda dA)~.
\]

Heterotic strings have the same supersymmetry as Type I string and so we expect the same action. However, in the absence of open strings or R-R fields the dilaton dependence should be $e^{-2\Phi}$ throughout:
\begin{align}\label{Het}
S_{\textrm{Het}}&=S_{\textrm{grav}}+S_{\textrm{YM}}\cr
S_{\textrm{grav}}&= \frac{1}{2\kappa_{10}^2} \int d^{10}x \sqrt{ -G}   \,e^{-2\Phi}\left[
 R +4 \partial_\mu \Phi \partial^\mu \Phi -\frac{1}{2}| \wt H_{(3)}|^{2} \right] \cr
S_{\textrm{YM}}&=- \frac{1}{2g_{10}^2} \int d^{10}x \sqrt{ -G}\,  e^{-2\Phi} \Tr  |F_{(2)}|^2
\end{align}
where the 3-form $\wt H_3$ is the field strength of the $B$-field equipped with Chern-Simons form
\[
\wt H_{(3)}=dB_{(2)}-\frac{\kappa_{10}^2}{g_{10}^2}\omega_3~.
\]

Indeed the low-energy effective actions of Type I \eqref{TypeI} and Heterotic $\SO(32)$ \eqref{Het} are related by the following field definitions (Homework)
\begin{align}
G_{\m\n}^{I} = e^{-\Phi^{H}} G_{\m\n}^{H} ~,&\qquad  \Phi^I = -\Phi^{H} \cr
\wt G_{(3)}^I = \wt  H_{(3)}^{H}~ ,&\qquad  A^I = A^{H} ~.
\end{align}
 Recalling that the vacuum expectation value of the dilaton is the string coupling $g_s=e^{\Phi}$, we see that the strong coupling limit of one theory is related to the weak coupling limit of the other theory and vice versa.

In Type I theory, D1-branes and
D5-branes are electrically and magnetically charged under $C_{(2)}$, respectively. In Heterotic $\SO(32)$ theory, fundamental strings and NS5-branes are electrically and magnetically charged under $B_{(2)}$, respectively. The S-duality
maps them as \cite{Polchinski:1995df}
\begin{table}[ht]\centering
\begin{tabular}{ccc}
Type I&$\leftrightarrow$& Heterotic SO(32)\\ \hline
D1-branes&$\leftrightarrow$&F-strings\\
D5-branes&$\leftrightarrow$&NS5-branes
\end{tabular}\end{table}

One can provide another evidence of this duality by looking at the massless spectrum. We have seen that Heterotic $\SO(32)$ has massless fields:
\begin{enumerate}\setlength{\parskip}{-0.1cm}
\item $\bf 8_v$ of SO(8): bosonic right-moving  $X^i(z)$
\item $\bf 8_c$ of SO(8):   fermionic right-moving $\psi^i(z)$
\item $\bf 32$ of SO(32):   left-moving Majorana-Weyl fermion $\wt\lambda^a(\bar z)$
\end{enumerate}
Correspondingly, one can see the massless BPS excitations from D1-strings stretched in the $x_1$-direction in Type I theory (Homework):
\begin{enumerate}\setlength{\parskip}{-0.1cm}
\item $\bf 8_v$ of SO(8): normal bosonic excitations of D1-D1 strings
\item $\bf 8_c$ of SO(8):  right-moving fermionic  excitations of D1-D1 strings
\item $\bf 32$ of SO(32):   left-moving fermionic  excitations of D1-D9 strings
\end{enumerate}

Further evidence of
this duality has been assembled by comparing tensions, $F_{(2)}^4$ interactions, and so on \cite[\S14.3]{Polchinski}.


\subsection{Heterotic \texorpdfstring{$E_8\protect\times E_8$}{E8xE8} string from M-theory}

Now we shall consider the strong-coupling behavior of Heterotic $E_8\times E_8$ theory. Taking T-duality and S-duality, Heterotic $E_8\times E_8$ theory is dual to Type I theory. As seen in \S\ref{sec:TypeI'}, the T-dual to Type I theory is the Type I' theory, which is Type IIA theory on a line segment $S^1/\bZ_2$ where O8${}^-$-planes sit at the two ends, and $(16+16)$ D8-branes are distributed on  $S^1/\bZ_2$. In the strong coupling regime, the M-theory circle will emerge, and it is described as M-theory on $S^1\times S^1/\bZ_2$.


\begin{figure}[ht]\centering
\includegraphics[width=15cm]{picture/HE-duality}
\caption{Duality web for Heterotic M-theory}
\end{figure}


Interestingly enough, the relative position of O8${}^-$-planes and D8-branes in Type I' string theory may be adjusted. This freedom goes away in the M-theory limit; the D8-branes have to be stuck at the O8${}^-$-planes, and they become the domain walls of M-theory, which are called \textbf{Ho\v{r}ava-Witten domain wall} or \textbf{M9-branes} \cite{Horava:1995qa,Horava:1996ma}.


Its low-energy effective description is given by $D=11$ supergravity on $S^1/\bZ_2$ which gives rise to gravitational anomaly \cite{AlvarezGaume:1983ig}. In order to cancel such
anomaly, non-Abelian gauge fields have to be present at the boundaries
in order to employ a Green-Schwarz mechanism \cite{Green:1984sg}. (Homework) This mechanism that bulk anomaly cancels with boundary anomaly is called \textbf{anomaly inflow}. Indeed the low-energy effective theory at the Ho\v{r}ava-Witten domain wall is described by 10d $\cN=1$ SYM with $E_8$ gauge group that cancels anomaly.



As in Type IIA, the distance between the two boundaries is related to Heterotic coupling $R = g_{\rm het}^{\frac{3}{2}}\ell_p$.
Hence, the line segment $S^1/\bZ_2$ shrinks at the weak coupling regime, leading to Heterotic $E_8\times E_8$ string theory.
The reason to pick $E_8 \times E_8$ is that the anomalies must be canceled on both boundaries, and there is no way to distribute $\SO(32)$ between two boundaries (it's a simple group with no factors). Using the previous terminology Heterotic $E_8\times E_8$
string theory can be viewed as M-theory compactified
on $S_1/\bZ_2$. This setup is called \textbf{Ho\v{r}ava-Witten M-theory}  or \textbf{Heterotic M-theory}  \cite{Horava:1995qa,Horava:1996ma}.







\begin{figure}[ht]\centering
\includegraphics[width=13cm]{picture/Horava-Witten}
\caption{Low-energy effective description of Heterotic M-theory. Ho\v{r}ava-Witten domain walls at the two boundaries give rise to $\cN=1$ SYM with $E_8$ gauge group and they cancel bulk anomaly.}
\end{figure}


\subsection{Duality between Heterotic on \texorpdfstring{$T^4$}{T4} and Type IIA on K3}
Let us now see one more non-trivial duality. Although we have studied only toroidal compactifications, we have seen a rich web of dualities. In string theory, a theory is consistent if we compactify it on a Calabi-Yau manifold. Since there are wide varieties of Calabi-Yau manifolds, string dualities involving them are much richer. It has still been an active research area both in physics and mathematics. Here, we deal with the next simplest Calabi-Yau manifold called \textbf{K3 surface}.

\subsubsection*{K3 surface}

A K3 surface is a resolution of $T^4/\bZ_2$.  We write a 4-torus as
\[
T^4=\bR^4/\bZ^4=\{\bfx=(x_1,x_2,x_3,x_4)\in\bR^4| x_i\sim x_i+1\}
\]
and  the $\bZ_2$ action is a reflection $x_i\rightarrow -x_i$.
Note that this action has $2^4=16$ fixed points given by choice of midpoints or the origin in any of the four $x_i$.
Thus, the resulting space $T^4/\bZ_2$ is singular at any of these 16 fixed points. The neighborhood of a singular point is indeed a cone of $\bR P^3$.
To make it smooth,  let us consider the set of vectors of length $\le 1$ in the tangent bundle of $TS^2$
\[
V=\{ (v_1,v_2)\in S^2\times T_{v_1}S^2| ~|v_2|\le 1\}~.
\]
Then the boundary of $V$  is $\partial V=\bR P^3$ so that you can replace the neighborhood of each singular point by $V$. Since $V$ is a smooth manifold, the resulting space is smooth, and it is a K3 surface. This smoothing procedure is called \textbf{resolution} or \textbf{blow-up}.

Although the construction of a K3 surface is rather simple, its geometry is surprisingly fertile. First of all, it is a Calabi-Yau manifold, namely a Ricci-flat K\"ahler manifold.
In real four dimensions, there are only two topologically equivalent compact closed Calabi-Yau manifolds, $T^4$ and K3.  Moreover, it is a hyper-K\"ahler manifold. (Let us not go into detail about hyper-K\"ahler manifolds.)

Let us now briefly look at the topological property of K3 surfaces. The resolution of the 16 singular points provides 16 elements of $H^2(K3,\bZ)$ in addition to $6={}_4C_{2}$ tori in $T^4$. Therefore, we have $H^2(K3;\bZ)\cong \bZ^{22}$. Moreover, the Hodge diamond as a complex manifold turns out to be
\[
\begin{array}{ccccc}
            &              &  h^{0,0} &            &            \\
            & h^{1,0}\! &              &\! h^{0,1}\!&            \\
  h^{2,0}\! & \!             & h^{1,1} & \!            &\! h^{0,2}\\
            & h^{2,1} &              &h^{1,2}&             \\
             &              &h^{2,2}&              &
\end{array} =
\begin{array}{cccccc}
\ &\ &\ 1\ &\ &\\
\ &\ 0\ &\ & \ 0\ &\\
\ 1\ &\ &\ 20\ &\ &\ 1\\
\ &\ 0\ &\ &\ 0\ &\\
\ &\ &\ 1\ &\ &
\end{array}\  .
\]
Since it is a real 4-dimensional manifold, one can consider the intersection matrix of rank 22
\[
Q(\a_i,\a_j) =\a_i \cap  \a_j \qquad \a_i\in H_2(K3;\bZ)
\]
In fact, in a certain nice basis,  the intersection  matrix can be written as follows
\[
Q(\a_i,\a_j) \sim 2(-E_8)\oplus 3\begin{pmatrix} 0&1\\1&0\end{pmatrix}
\]
where $-E_8$ denotes the $8\times8$ matrix given by minus the Cartan
matrix of the Lie algebra $E_8$. Hence, we may decompose
\[
  H^2(K3,\bR) = H^+\oplus H^-,
\]
where $H^\pm$ represents the cohomology of the space of
(anti-)self-dual 2-forms. We then see that
\[
  \dim H^+=3,\quad\dim H^-=19~.
\]

The moduli
space of non-trivial metric deformations on a K3
is 58-dimensional and given by the coset space \cite{Aspinwall:1996mn}
\[
{\cal M}_{\textrm{K3}}\ =\  \bR^+\ \times\
\left.\frac{\mathrm{O}(3,19)}{\mathrm{O}(3)\times \mathrm{O}(19)}\right/ \mathrm{O}(3,19,\bZ)\ ,
\]
where the second factor is the Teichm\"uller space
for Ricci-flat metrics of
volume one on a K3 surface and
the first factor is associated with the
size of the K3.


This is not the end of the story if we consider string propagation
on $K3$. For each element of $H_2(K3,\bZ)$, we can turn
on the $B$-field. Because of $H_2(K3,\bZ)=\bZ^{22}$, we have 22 additional real parameters and that makes the total
dimension of moduli space $58+22=80$.  It turns out that this moduli
space is isomorphic to
\begin{equation}
{\cal M}_{\textrm{K3}}^{\textrm{stringy}} =
\left. \frac{\mathrm{O}(4,20)}{\mathrm{O}(4)\times \mathrm{O}(20)}\right/
\mathrm{O}(4,20,\bZ)\ .
\end{equation}
Substituting $D=4$ into \eqref{Narain}, one can see that this is exactly the same as the Narain moduli space for Heterotic string on $T^4$!



\subsubsection*{Heterotic on $T^4$/Type IIA on K3}
The relations between Heterotic on $T^4$ and Type IIA on K3 can be seen by comparing the effective actions in $D= 6$. On the Heterotic side, for generic Wilson lines the $E_8 \times  E_8$ or $\SO(32)$ gauge symmetry is broken to $\U(1)^{16}$. Including the KK gauge bosons from $T^4$ compactification, the gauge group becomes $\U(1)^{24}$ and the effective 6d supergravity action of Heterotic string is
\[
S_{\textrm{Het}}= \frac{1}{2\kappa_{6}^2} \int d^{6}x \sqrt{ -G}   \,e^{-2\Phi}\left[
 R +4 \partial_\mu \Phi \partial^\mu \Phi -\frac{1}{2}| \wt H_{(3)}|^{2}- \frac{\kappa_{6}^2}{2g_{6}^2} \sum_{I=1}^{24} |F_{(2)}^I|^2 \right] ~.
\]

Type IIA superstring theory compactified on K3 breaks a half of supersymmetries so that there are 16 supercharges as in Heterotic string. It also gives rise to $\U(1)^{24}$ gauge fields which, via KK-reduction, all arise from the R-R sector. One comes from the one-form $C_{(1)}$  with indices in the 6d non-compact spacetime, and another one from the three-form $C_{(3)}$, which is Hodge-dual to a massless vector in 6d. Because of $H_2(K3,\bZ)=\bZ^{22}$, the three-form $C_{(3)}$ on two-cycles of K3 gives 22 vectors. (Thus, we have $1+1+22=24$ $\U(1)$ gauge fields.) As a result, the effective Type IIA action compactified on K3 is
\[
S_{\textrm{IIA}}= \frac{1}{2\kappa_{6}^2} \int d^{6}x \sqrt{ -G}   \left[e^{-2\Phi}\Big(
 R +4 \partial_\mu \Phi \partial^\mu \Phi -\frac{1}{2}| \wt H_{(3)}|^{2} \Big)- \frac{\kappa_{6}^2}{2g_{6}^2} \sum_{I=1}^{24} |F_{(2)}^I|^2 \right] ~.
\]

 It is straightforward to show that the two actions are equivalent via the following field redefinition
 \begin{align}\nonumber
\Phi^{\textrm{H}}=-\Phi^{\textrm{IIA}}~,&\qquad G^{\textrm{H}}=e^{-2\Phi^{\textrm{IIA}}}G^{\textrm{IIA}}\cr
A^{\textrm{H}}=A^{\textrm{IIA}}~,&\qquad \wt H^{\textrm{H}}=e^{-2\Phi^{\textrm{IIA}}}\ast \wt H^{\textrm{IIA}}~.
\end{align}





\subsubsection*{More dualities}



Of course, what we have glimpsed are merely a few representative examples of string dualities. For instance,  compactifications of M-theory on tori with orbifold actions, one can find many dualities.
These dualities have been discovered at by using similar arguments above. Some
examples of such conjectured dualities are given
below \cite{Dasgupta:1995zm,Witten:1995em,Sen:1996zq}:
\begin{eqnarray}
\hbox{M-theory on} && \nonumber \\
\textrm{K3} \qquad  &\leftrightarrow& \qquad\hbox{Heterotic/Type I on $T^3$} \nonumber
\\
T^5/\bZ_2 \qquad
&\leftrightarrow&\qquad \hbox{IIB on K3} \nonumber \\
T^8/\bZ_2 \qquad &\leftrightarrow&\qquad \hbox{Type I/Heterotic on $T^7$}
\nonumber \\
T^9/\bZ_2 \qquad &\leftrightarrow&\qquad \hbox{Type IIB on
$T^8/\bZ_2$}\nonumber
\end{eqnarray}
In each case, $\bZ_2$ acts as reversing the sign of all the coordinates of $T^D$.
These dualities have been checked by taking Type IIA limit of M-theory.




\end{document}
