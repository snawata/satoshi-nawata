\documentclass[String-lecture-21.tex]{subfiles}


\begin{document}
\section{Black holes in string theory}\label{sec:BH}


In the previous section, we have seen that D-branes are dynamical objects and D-branes can end on others forming bound states. Moreover, they were ideally suited for studying black holes.


A large number of D-branes is heavy enough  to produce a black hole by wrapping a cycle in a compact manifold.  There is a large degeneracy due to open strings attaching to D-branes, which gives
a statistical interpretation of the thermodynamic entropy. This leads to a precise microscopic accounting for the Bekenstein-Hawking entropy of the supersymmetric black holes, as shown by Strominger-Vafa \cite{Strominger:1996sh}.


The study of black holes in string theory by using D-branes has led to the celebrated AdS/CFT correspondence \cite{Maldacena:1997re}. (See Maldacena's Ph.D. thesis \cite{Maldacena:1996ky} for instance.)



First let us briefly summarize the basics of black holes in general relativity and the laws of black hole thermodynamics studied in the early 70s \cite{Bekenstein:1972tm,Bekenstein:1973ur,Bardeen:1973gs,Hawking:1974sw}. For more detail, I refer to a wonderful lecture note \cite{Townsend:1997ku}. Also I refer the reader to \cite{Weinstein:2021dop} for a historical account of black hole entropy.


\subsection{Black holes}

To begin with, we consider the Einstein-Maxwell action
\begin{equation}\label{EMaction}
{\frac1{16} \pi} \int  d^4x  \sqrt{g} \Bigl(\frac1G R-  F_{\mu\nu}F^{\mu\nu}\Bigr)~,
\end{equation}
where $G$ is Newton's constant. In this subsection, we shall review black hole solutions to the action \eqref{EMaction} and see that they are characterized by mass $M$, charge $Q$ and angular momentum $J$.


\subsubsection*{Schwarzschild metric}

If there is no electromagnetic fields $F=0$ in the action \eqref{EMaction}, the equation of motion is $R_{\mu\nu}-{\frac12}g_{\mu\nu} = 0$, which has a spherically symmetric, static solution
\[
ds^2 \equiv g_{\mu\nu} dx^\mu dx^\nu =  - \Bigl(1 - \frac{2GM}{r}\Bigr) dt^2
+ \Bigl(1 - \frac{2GM}{r}\Bigr)^{-1} dr^2 + r^2 d \Omega^2,
\]
where $t$ is the time, $r$ is the radial coordinate, and $d\Omega$
is the canonical metric of a 2-sphere.
This metric describes the spacetime outside a gravitationally collapsed
non-rotating star with zero electric charge, called \textbf{Schwarzschild metric}.
It is well-known that the \textbf{event horizon} appears at
\[
g^{rr} = 0 \,,
\]
and  the sphere $r = 2GM$ is indeed the {event horizon} of the
Schwarzschild black hole with mass $M$.

It turns out that much of the interesting physics having to
do with the quantum properties of black holes comes from the
region near the event horizon.
To examine the region \emph{near $r=2GM$}, we analytically continued to the Euclidean metric $t= -it_E$, and we set
\[
r-2GM=\frac{x^2}{8GM}~.
\]
Then, the metric near the event horizon $r=2GM$
\[
ds^2_{\textrm{E}} \approx  (\kappa x)^2dt_E^2+dx^2
+\frac{1}{4\kappa^2}d\Omega^2~,
\]
where $\kappa=\frac1{4GM}$ is called the \textbf{surface gravity} because it is indeed the acceleration of a static
particle near the horizon as measured at spatial infinity. Note that the surface gravity is defined by using Killing vector at the horizon, precisely speaking \cite{Townsend:1997ku}.
The first part of the metric is just $\bR^2$ with polar coordinates if we make the
{periodic identification}
\[
t_E \sim t_E +\frac{2\pi}{\kappa}~.
\]
Using the relation between Euclidean periodicity and temperature,
we can deduce \textbf{Hawking temperature} of the Schwarzschild black hole
\begin{equation}\label{hawktemp}
k_B T_H = \frac{\hbar \kappa}{2 \pi c}=\frac{\hbar c^3}{8\pi GM}~.
\end{equation}
Here we restore the Boltzman constant $k_B$, and the speed of light $c$.
This is a very heuristic way to introduce the Hawking temperature which was not originally found in this way.



\subsubsection*{Reissner-Nordstr\"om  black hole}

The most general static, spherically symmetric, charged  solution
of the Einstein-Maxwell theory (\ref{EMaction}) is
\begin{equation}\label{rn}
ds^2 = -\left(1 - \frac{2GM}{r} + \frac{G Q^2}{r^2}\right) dt^2 +
\left(1 - \frac{2GM}{r} + \frac{G Q^2}{r^2}\right)^{-1} dr^2 + r^2 d
\Omega^2,
\end{equation}
with the electromagnetic field strength
\[
F_{tr} = \frac{Q}{r^2}~.
\]
This solution is called the \textbf{
Reissner-Nordstr\"om (RN) black hole} with mass $M$ and charge $Q$.
{}From the metric \eqref{rn} we see that there are two event horizon for this solution where
$g^{rr} =0$ at
\[
r_\pm = GM \pm \sqrt{(GM)^2 - GQ^2}~.
\]
Thus, $r_+$ defines the outer horizon of the black hole and $r_-$
defines the inner horizon of the black hole. The area of the black
hole is  $4\pi r_+^2$. It turns out that the Hawking temperature of the RN black hole is
%
\[
T_H = \frac{\sqrt{(GM)^2-GQ^2}}{2\pi G
\left(GM+\sqrt{(GM)^2-GQ^2}\right)^2} \,.
\]
%

For a physically sensible definition of temperature, the mass must satisfy the bound $GM^2 \geq Q^2$, and  the two horizons
coincide $r_+ = r_- = GM$ when  this bound is saturated.
In this case, the temperature of the black hole is zero and it is called an \textbf{extremal black hole}.


\subsubsection*{Kerr-Newman black hole}

If we relax the static condition, black holes can have angular momentum. Hence, general stationary solutions, called \textbf{Kerr-Newman black holes}, to the action \eqref{EMaction}
are described with three parameters. In \textbf{Boyer-Linquist
coordinates}, the KN metric is
\bea\nonumber
ds^2 & =  -\frac{ \left(\Delta -a^2\sin^2\theta\right)}{\Sigma}dt^2 - 2 a
\sin^2\theta \frac{ \left(r^2+a^2-\Delta\right)}{\Sigma}dt\,d\phi \\
 &  +\left( \frac{ \left(r^2+a^2\right)^2-\Delta a^2\sin^2\theta}{\Sigma}
\right)\sin^2\theta d\phi^2 +\frac{\Sigma}{\Delta}dr^2+\Sigma d\theta^2
\eea
where
\bea\nonumber
\Sigma & =  r^2+a^2\cos^2\theta \cr
\Delta & =  r^2-2Mr+a^2+e^2~. \eea
The three parameters are $M$, $a$, and $e$.  It can be shown that
\[
a=\frac{J}{M}
\]
where $J$ is the total angular momentum, while
\[
e = \sqrt{ Q^2+P^2}
\]
where $Q$ and $P$ are the electric and magnetic (monopole) charges,
respectively.  The Maxwell 1-form of the KN solution is
\[
A_\mu dx^\mu= \frac{ Qr\left(dt-a\sin^2\theta d\phi\right)-P\cos\theta
\left[a dt-\left(r^2+a^2\right)d\phi\right] }{\Sigma} ~.
\]




\subsection{Black hole thermodynamics and Bekenstein-Hawking entropy}

Classically, a stationary black hole is characterized by its mass $M$, angular momentum $J$, and charge $Q$. This is called a black hole no hair theorem. However, in \cite{Bekenstein:1972tm}, Bekenstein asks an incisive question: if we treat a black hole as a purely geometric object, by throwing a package of entropy, a cup of tea, into a black hole, the total entropy of the world outside would seem to decrease. This contradicts the second law of thermodynamics which states the total entropy never decreases.  To save the second law of thermodynamics, this suggests that the black hole must have entropy. The question is how to characterize the entropy if a black hole has. The hit was hidden in the area theorem of black holes \cite{Hawking:1971tu,Hawking:1971vc}, stating that the total area of the black hole horizons never decreases in any process. For example, two Schwarzschild black holes with masses $M_1$ and $M_2$ can merge into a bigger black hole of mass $M=M_1+M_2$. Since the area is proportional to the square of the mass, this is consistent with the area theorem, namely $(M_1 + M_2)^2 \geq M^2_1 + M^2_2$. On the other hand, the opposite process where a bigger black hole splits into two is never allowed by this theorem. Motivated by the area theorem, Bekenstein proposed in \cite{Bekenstein:1972tm} that a black hole has entropy proportional to its area.


Soon after that, Bardeen, Carter and Hawking point out similarities between the laws of black hole mechanics and the laws of thermodynamics in \cite{Bardeen:1973gs}. More concretely, they find the laws of corresponding to the three laws of thermodynamics.
\begin{enumerate}
\item[{(0)}]
Zeroth Law: In thermodynamics, the zeroth law states  that the temperature $T$ of a thermal equilibrium object is constant throughout the body. Correspondingly, for a stationary black hole, its surface gravity $\kappa=1/4GM$ is constant over the event horizon.

\item[{(1)}]
First Law: The first law of thermodynamics states that energy is conserved, and the variation of energy is given by
\be dE = TdS + \mu dQ + \Omega dJ \ee
where $E$ is the energy, $Q$ is the charge with chemical potential $\mu$ and $J$ is the angular momentum with chemical potential
$\Omega$ in the system.
Correspondingly, for a black hole, the variation of its mass is given by
\be dM = \frac{\kappa}{8\pi G} dA + \mu dQ + \Omega dJ \ee
where $A$ is the area of the horizon,  and $\kappa$ is the surface gravity, $\mu$ is the chemical potential conjugate to $Q$, and $\Omega$ is the angular velocity conjugate to $J$.

\item[{(2)}]
Second Law: The second law of thermodynamics states that the total entropy $S$ never decreases, $\delta S \geq 0$.  Correspondingly, for a black hole, the area theorem states that the total area of a black hole in any process never decreases, $\delta A \geq 0$.
\end{enumerate}



\begin{table}[ht]
\centering
\begin{tabular}{c|c}
\hline
\textbf{Laws of thermodynamics} & \textbf{Laws of black hole mechanics}\\
 \hline
 Temperature is constant & Surface gravity is constant \\
 throughout a body at equilibrium. &  on the event horizon.\\
 $T$=constant. & $\kappa$ =constant.\\
 \hline
Energy is conserved.& Energy is conserved. \\
$dE = T dS + \mu dQ + \Omega dJ. $& $dM = \frac{\kappa}{8\pi G} dA + \mu dQ + \Omega dJ . $\\
 \hline
 Entropy never decreases.  & Area never decreases.\\
 $\delta S \geq 0$. & $ \delta A \geq 0 $. \\
 \hline
\end{tabular}
\caption{\small{Laws of black hole thermodynamics}}
\label{blackholelaws}
\end{table}



This result can be understood as one of the highlights of general relativity. Classically, a black hole is a not only geometric but also thermodynamic object.
If a black hole has energy $E$ and entropy $S$, then it must also have temperature $T$ given by
\[
\frac{1}{T} = \frac{\partial S}{\partial E}.
\]
For example, for a Schwarzschild black hole, the area and the entropy are proportional to $M^2$. Hence, we can derive
\[
\frac{1}{T} = \frac{\partial S}{\partial M} \sim \frac{\partial M^2
}{
\partial M} \sim M.
\]
Therefore, black hole temperature is inversely proportional to mass $M$. The smaller a black hole is, the hotter it is!
Moreover, if the black hole has temperature, it must thermally radiate like any hot body. The understanding of the thermal properties of black holes requires treatment beyond classical general relativity.



Hawking has applied techniques of quantum field theories on a curved background to the near-horizon region of a black hole and showed that a black hole indeed radiates \cite{Hawking:1974sw}.
This can be intuitively understood as follows: in a quantum theory, particle-antiparticle creations constantly occur in the vacuum. Around the horizon, after pairs are created, antiparticles fall into a black hole due to the gravitational attraction whereas particles escape to the infinity.  Although we do not deal with Hawking's calculation unfortunately (see \cite{Townsend:1997ku}), it indeed justifies this picture. Moreover, it revealed that the spectrum emitted by the black hole is precisely subject to the thermal radiation with temperature \eqref{hawktemp}. Indeed, a black hole is not black at quantum level.
Hence, we can treat a black hole as a thermal object, and the analogy of the laws in Table \ref{blackholelaws} can be understood as the natural consequence of the laws of thermodynamics. As a result,
the formula for the Hawking temperature \eqref{hawktemp} and the first law of thermodynamics
\[
c^2 dM = T_HdS = \frac{\kappa c^2}{8\pi G} dA,
\]
lead to the precise relation between entropy and the area of the black hole:
\be \label{BH-entropy}
 S = \frac{k_B c^3 A}{4G\hbar} \, .
\ee
This is a universal result for any black hole, and this remarkable relation between the thermodynamic properties
of a black hole and its geometric properties is called the celebrated \textbf{Bekenstein-Hawking entropy formula}.
This formula involves all four fundamental constants of nature; $(G,c,k_B,\hbar)$. Also, this is the first place where the Newton constant $G$ meets with the Planck constant $\hbar$.
Thus, this formula shows a deep connection between black hole geometry, thermodynamics and quantum mechanics.








For ordinary objects, Boltzmann has given the statistical interpretation of the thermodynamic entropy of a system.
We fix the macroscopic parameters (e.g. total electric charge, energy etc.) and count the number $\O$ of quantum states, known as microstates, each of which has the same values for the macroscopic parameters, and the entropy is expressed as
\[
S = k_B \log \O~.
\]
Since the Bekenstein-Hawking entropy \eqref{BH-entropy} behaves as the ordinary thermodynamic entropy in every aspect, it is therefore natural to ask
whether the black hole entropy admits a statistical interpretation in the same way.



Furthermore, one of the most dramatic results of Hawking's work was the implication that black holes are associated with information loss. Physically speaking, we can associate information with pure states in quantum mechanics. If we throw in a pure quantum state, say, the s-wave to a black hole, then it eventually comes out as a thermal (mixed) state.  Thus the net result of this
process is the evolution of a pure quantum state into a mixed state, which violates the law (unitarity) of quantum mechanics. This is called the \textbf{information paradox} \cite{Hawking:1976ra}.
This is because Hawking's calculation is based on the semi-classical analysis, namely, we fix the background and quantize particles. In fact, the information paradox stems from the absence of such a microscopic description of gravity.


In order to investigate the microscopic description of black hole entropy, we need quantum theory of gravity. This is precisely what string theorists have attempted to do and have been partially successful.










\subsection{Black holes in string theory}
In string theory on a $d$-dimensional compact manifold, branes can be wrapped in a cycle of the compact manifold and it looks like a point-like object in $(10-d)$-dimensional spacetime. In the regime where supergravity approximation is valid, configurations of this kind give rise to black hole solutions of the corresponding low-energy supergravity theory. Moreover, if a brane configuration preserves supersymmetry, then the corresponding solution will be an extremal supersymmetric black hole. Extremal black holes are interesting because they are stable against Hawking radiation and nevertheless have a large entropy.
On the other hand, configurations without supersymmetry yield non-extremal black holes.




In general, the regime of the parameter space in which supergravity is valid is different from the regime in which the microstates counting can be performed. Thus, even if we know that a given brane configuration becomes a black hole when we go from weak to strong coupling, it is generally difficult to extract microscopic information of the black hole from the brane configuration.


For supersymmetric black holes, however, one can count the number of states at weak coupling and extrapolate the result to the black hole phase due to the BPS property. We will see that in this way, one derives the Bekenstein-Hawking entropy formula (including the precise numerical coefficient) for a 5d supersymmetric black hole \cite{Strominger:1996sh}. (For more detail, I refer to \cite{David:2002wn,Dabholkar:2012zz}.)


\subsubsection*{D1-D5-P brane system}

\begin{figure}[ht]\centering
\includegraphics[width=13cm]{picture/D1D5}
\caption{}
\end{figure}


Here we follow the example treated in \cite{Callan:1996dv}, called the D1-D5-P brane system on $\bR^{1,4}\times T^5$.
Let us consider Type IIB compactified on a five-torus $T^5=T^4\times S^1$, which spans the $(x_5 \cdots x_9)$ coordinates, with $Q_1$ D1-branes and $Q_5$ D5-branes in the following configuration.
\begin{table}[ht]\centering\begin{tabular}{c|cccccccccc}
 & 0 & 1 & 2 & 3 & 4 & 5 & 6 & 7 & 8 & 9\tabularnewline
\hline
$Q_1$ D1 & $\times$ &  &  &  & &  &  &  &  & $\times$\tabularnewline
$Q_5$  D5& $\times$ &  &  &  &  & $\times$ & $\times$ & $\times$ & $\times$ & $\times$\tabularnewline
$Q_P$ mom &  &  &  &  & &  &  &  &  & $\rightsquigarrow$\tabularnewline
\end{tabular}\end{table}
We consider that the volume of  $T^4$ is $(2\pi)^4V$ and the radius of $S^1$ is $R$.
Here we also assume that there is an excitation by open strings carrying momenta $Q_P/R$ in the $x_9$-direction.
This system preserves 4 real supercharges since each constituent breaks a half of supersymmetry.




\subsubsection*{Black hole in 5d supergravity}

If there are large enough D-brane charges $(Q_1,Q_5,Q_P)$ and the five-torus is sufficiently small, the configuration produces a 5d black hole. We would like to compute the Bekenstein-Hawking entropy \eqref{BH-entropy} of the black hole by evaluating the area of the event horizon.
When 5d supergravity analysis is valid, this brane system gives rise to a $1/8$-BPS black hole configuration. Ignoring the R-R field and the $B$-field configuration, the 5d Einstein frame metric of this solution then becomes
\bea\nonumber
ds_5^2 = &\!
-\lambda(r)^{-\frac23} dt^2 + \lambda(r)^{\frac13}
\left[ dr^2 + r^2 d\Omega_3^2 \right] \,,
\eea
%
where the harmonic functions are
%
\[
\lambda(r)=H_1(r)H_5(r) K(r)  = \Bigl(1 + {\frac{r_1^2}{r^2}} \Bigr)\Bigl( 1 + {\frac{r_5^2}{r^2}}\Bigr)\Bigl(  1+{\frac{r_m^2}{r^2}}\Big )\,,
\]
%
with
%
\[
r_1^2 = {\frac{g_s Q_1\ell_s^6}{V}} \,, \qquad
r_5^2 = g_s Q_5 \ell_s^2 \, \qquad
r_m^2 = {\frac{g_s^2 Q_P \ell_s^8}{R^2 V}} \,.
\]
%
Let us briefly evaluate the validity of the supergravity analysis. In order for the $\alpha^\prime$ corrections to geometry to be small, the radius parameters have to be large
with respect to the string unit, $r_{1,5,m}\!\gg\!\ell_s$. Since we assume $V^{1/4}$, $R$ are an order of the string length, this implies
\be \label{sugra-limit}
g_s Q_1\gg1~, \qquad g_s Q_5\gg1~, \qquad g_s^2 Q_P\gg1~.
\ee
To suppress string loop corrections, we need $g_s$ to be small (but finite) so that the D-brane charges must be sufficiently large for supergravity analysis.


It turns out that the surface gravity and therefore the Hawking temperature of this black hole is zero, $T_{{H}}=0$, as expected.  The metric shows that the event horizon is located at $r=0$ and
the Bekenstein-Hawking entropy \eqref{BH-entropy} is
%
\bea\label{sugra-result}
S_{\textrm{macro}} &= {\displaystyle{
 {\frac{A}{4 G_5}}  = {\frac{1}{4G_5}}2
\pi^2\left[r^2\lambda(r)^{\frac13}\right]^{\frac32} }}
\ \ {\rm{at\ }} r=0 \,\cr
&= {\displaystyle{
 {2\frac{\pi^2}{4\left[\pi g_s^2\ell_s^8/(4VR)\right]}}
  \left(r_1 r_5 r_m\right)^{\frac12}
= {\frac{2\pi{}VR}{g_s^2\ell_s^8}}
\left(
{\frac{g_s{}Q_1\ell_s^6}{V}}\,g_s{}Q_5\ell_s^2\,
{\frac{g_s^2{}Q_P\ell_s^8}{R^2V}} \right)^{\frac12} }} \cr
 &=  2\pi\sqrt{Q_1 Q_5 Q_P} \,,
\eea
where we use  $G_5=\frac{G_{10}}{(2\pi)^5VR}$ and $16\pi G_{10}=(2\pi)^7g_2^2\ell_s^8$.
Notice that it is also independent of $R$ and of $V$ while the ADM mass depends on $R$, $V$ explicitly.
\[
M = {\frac{Q_P}{R}} + {\frac{Q_1R}{g_s\ell_s^2}}
+ {\frac{Q_5RV}{g_s\ell_s^6}}\,.
\]



%--------------------------------------------------------------------+
\subsubsection*{Counting microstates}


The next step is to identify the degeneracy of open string states of the D1-D5-P system, which can be analyzed at the limit opposite to \eqref{sugra-limit}, i.e.
\be\label{CFT-regime}
g_s Q_1\ll1~, \qquad g_s Q_5\ll1~, \qquad g_s^2 Q_P\ll1~.
\ee
Further simplification can be made by taking the limit that the
volume of $T^4$ is small as compared to the radius of the circle $S^1$,
%
\[
V^{\frac14}\ll R \,.
\]
%
In this limit,  the theory on the D-branes is an effective $2d$ theory on $(x_1,x_9)$-direction. Moreover,  the smeared D1-branes plus D5-branes have a
symmetry group
$SO(1,1)\!\times\!\SO(4)_\parallel\!\times\!\SO(4)_\perp$ where $\SO(4)_\parallel\!\times\!\SO(4)_\perp$ becomes $R$-symmetry of the $2d$ theory which we call $\cN=(4,4)$ 2d CFT. In the supersymmetric configuration, the left-movers are in their ground states so that we count excited right-movers.

Because the D1-branes are instantons in the D5-brane theory, the
low-energy theory of interest is in fact a $\sigma$-model on the
moduli space of instantons
\[{\cal{M}}=\textrm{Sym}^{Q_1Q_5}(T^4)=(T^{4})^{Q_1Q_5}/S_{Q_1Q_5}~.\]
The central charge of this 2d CFT is
\be \label{D1D5Pc} c=n_{\rm
bose}{+}{\frac12}n_{\rm fermi}=6Q_1Q_5~.\ee
Roughly, this central charge $c$ can be thought of as coming from having $Q_1Q_5$ 1-5
strings that can move in the 4 directions of $T^4$. Although this orbifold theory has many twisted sectors, the special point of the moduli space corresponds to a single string winding $Q_1Q_5$ times. It turns out that counting the excitations of this \textbf{long string} is only relevant in the limit of large D-brane charges.  For this long string, the level-matching condition is
\[N - \overline N =  \frac{Q_P}{R} W  ~, \quad W=Q_1Q_5~, \quad \to \quad N= \frac{Q_PQ_1Q_5}{R}\]
where the left-movers are in the ground states $\overline N=0$.

If $N_m^i$ and $n^i_m$ denote occupation numbers of the four transverse compact bosonic and fermionic oscillators, respectively, then evaluation of $N$ gives
\be\label{occupation}
nW=\sum_{i=1}^4\sum_{m=1}^\infty m(N_m^i+n^i_m)
\ee
The degeneracy $\O(Q_1, Q_5, Q_P)$ is then given by  the number of choices for $N_m^i$ and $n^i_m$ subject to \eqref{occupation}.

The partition function of this system is the partition function for
4 bosons and an equal number of fermions
%
\[
Z = \left[\prod_{m=1}^\infty {\frac{1+q^{m}}{1-q^{m}}}
\right]^{4} \equiv \sum \Omega(Q_1, Q_5, Q_P) q^{N} \,,
\]
%
where $\Omega(Q_1,Q_5,Q_P)$ is the degeneracy of states at the level $N= \frac{Q_PQ_1Q_5}{R}$. The Cardy formula \cite{Cardy:1986ie} can be applied in the regime in which the KK momenta $Q_P\gg Q_1Q_5$ much larger than the
central charge \eqref{D1D5Pc} while assuming \eqref{CFT-regime}:
%
\[
\Omega(Q_1,Q_5,Q_P)\!\sim\! \exp\left(2\pi\sqrt{Q_1Q_5Q_P} \right)
= \exp\left(2\pi\sqrt{\frac{c}{6}\,Q_P}\right)
\]
%
Therefore the microscopic D-brane statistical entropy is
%
\[
S_{\rm{micro}} = \log\left(\Omega(Q_1,Q_5,Q_P)\right) = 2\pi\sqrt{Q_1Q_5Q_P}
\,.
\]
%
This agrees exactly with the black hole result \eqref{sugra-result}!




\begin{figure}[ht]\centering
\includegraphics[width=16cm]{picture/protected}\caption{}
\end{figure}



\subsubsection*{More results}

By coupling the low energy degrees of freedom in the D1-D5-p system to supergravity modes (therefore perturbing the extremal condition), one can also compute the rate of Hawking radiation from a black hole that agrees precisely with the Hawking calculation. Thus this provides a microscopic explanation of Hawking radiation. (See \cite[\S8]{David:2002wn}.)

In fact, vigorous research in the last decade has shown that one can show the exact match between macroscopic and microscopic calculations of black hole entropy even in finite D-brane charges.

Moreover, a generalization of Bekenstein-Hawking entropy has been proposed in \cite{Ryu:2006bv} that connects quantum theory of gravity and quantum information theory.  A recent study has clearly suggested that quantum entanglement must have something to do with quantum physics of spacetime.
\end{document}
