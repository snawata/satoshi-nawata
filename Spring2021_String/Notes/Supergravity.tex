\documentclass[String-lecture-21.tex]{subfiles}
\begin{document}
\section{Supergravity}\label{sec:supergravity}

String theory includes massless states as well as massive states.
However, in the energy regions much lower than $1/\ell_s$, we do not see the stringy massive states
because the mass $M^2 \sim \frac{1}{\alpha'} \sim \frac{1}{\ell_s^2}$ is assumed to be very heavy.
Therefore, in the low-energy region, we can describe the theory by an \textbf{effective theory},
which only contains the massless particles (lightest states).
The effective theory, of course, does not contain all the information of the original theory,
however, it does give us some information about the original theory.


For the bosonic string theory, the low-energy effective action for the massless fields is given in \eqref{NLSM-general}.
Below we deal with its supersymmetric versions, type \textbf{IIA/IIB supergravity},
that are low-energy effective theories of IIA/IIB superstring theory. In \S\ref{sec:typeI-HO}, we will deal with the low-energy effective actions of Type I and Heterotic string theory. The theory of supergravity is very broad, and we refer to \cite{freedman2012supergravity} for more detail.


\subsection{Local supersymmetry}

Before going to Type IIA/IIB supergravity,
let us see the basic idea of supergravity.


Roughly speaking, a global supersymmetry algebra takes the form
\be
\{\e_2 Q, \overline \e_1 \overline Q\}=i \overline\e_1 \Gamma^\mu \e_2 P_\mu~.
\ee
(For instance, see \eqref{susy-algebra}.) The key idea in supergravity is that supersymmetry holds locally so that the spinor parameters $\epsilon$ are arbitrary functions of the spacetime coordinates $\epsilon \rightarrow \epsilon(x)$. The supersymmetry algebra will then involve local translation parameters $\bar{\epsilon}_{1} \gamma^{\mu} \epsilon_{2}$ which must be viewed as diffeomorphisms. Thus local supersymmetry requires gravity. Note that we use $x^\mu$ for local coordinates of the spacetime instead of $X^\mu$, which has been adopted in the previous sections.


First of all, in order to define spinors in a curved space,
we need to introduce vielbeins $e_\mu^a(x^\rho)$. The metric can be the orthonormal one at one point by coordinate transformations, and  vielbeins transforms a local coordinate $x^\mu$
into the orthonormal coordinate at the point:
\begin{align*}
 x^a = e^a_\mu x^\mu \ , \quad x^\mu = e^\mu_a x^a \ ,
\end{align*}
which is defined through spacetime metric $G_{MN}$ by
\begin{align*}
 G_{\mu\nu}(x^\rho) = \eta_{ab} e_\mu^a(x^\rho) e_\nu^b(x^\rho)  \ .
\end{align*}
Now we can use spinor representations and gamma matrices thanks to the vielbeins:
\begin{align*}
 \{\Gamma^a,\Gamma^b\} = 2\eta^{ab} \ .
\end{align*}



If we include gravity in a theory, the translational invariance is promoted to invariance under general coordinate transformations.
Similarly, global Lorentz symmetry is promoted to local Lorentz symmetry (this is done by the vielbeins):
\begin{align*}
 \Lambda^a_{\ b} e_a^\mu(x^\rho) e_\nu^b(x^\rho) \equiv \Lambda^\mu_{\ \mu}(x^\rho) \ .
\end{align*}
Gravity can be interpreted as the gauge field for general coordinate transformations.
Since the gauge transformation of a spin-1 gauge field can be generally written as
\begin{align*}
 \delta A_\mu = D_\mu \lambda \ ,
\end{align*}
the gauge transformation of the gravity field will take the form
\begin{align}
 \delta e_\mu^a = D_\mu \lambda^a \ ,
\end{align}
where $\lambda^a$ is a general coordinate transformation, and $D_\mu$ is a covariant derivative.
Similarly, we need to introduce a gauge field for local supersymmetry in supergravity, and the corresponding gauge field is a \textbf{gravitino}. (Recall that gravitino appears as a massless field in superstring theory.)
Then, the supersymmetry transformations for the gravitino and vielbeins are given by
\begin{align}\label{local-susy}
 \delta \psi_\mu^\alpha = D_\mu \epsilon^\alpha \ ,\qquad \delta_\epsilon e^a_\mu = i\ol\epsilon \Gamma^a \psi_\mu~.
\end{align}
Then, the square of the supersymmetry transformation is as we want:
\bea
\left[\delta_{1}, \delta_{2}\right] e_{\mu}^{a}=&iD_\mu ( \bar{\epsilon}_{1} \Gamma^{a} \epsilon_{2})  \cr
\left[\delta_{1}, \delta_{2}\right] \psi_{\mu}=&i\bar{\epsilon}_{1} \Gamma^{\rho} \epsilon_{2}(D_{\rho} \psi_{\mu}-D_\mu \psi_\rho)~.
\eea

Supergravity theory always includes the two gauge fields, $e^a_\mu$ and $\psi^\alpha_\mu$,
and the action is given by
\begin{align*}
 S = \frac{1}{2\kappa_D^2}\int d^D x\ e\ \left[ R -2i \psi_\mu \Gamma^{MNP} D_\nu \psi_P \right] \ ,
\end{align*}
where $e = \det e_\mu^a$. It is straightforward to check that the action is invariant under local supersymmetry transformation \eqref{local-susy}. This action can be regarded as the supersymmetric version of the Einstein-Hilbert action.



\subsection{\texorpdfstring{$D=11$}{D=11} supergravity}

Supersymmetry puts a strong constraint on the spacetime dimension.
If we limit ourselves to consider fields up to spin-$2$,
then it is known that the highest dimension is $D=11$ \cite{Nahm:1977tg}.
Roughly speaking, this is because the degrees of freedom for fermions grow exponentially as $2^{[D/2]}$ whereas
those of bosons grow by the power law of $D$
(for instance, $\frac{(D-1)(D-2)}{2}-1$ for gravitons).
Therefore, to balance fermions and bosons with spin less than two, we cannot go arbitrarily higher.

The $D=11$ supergravity \cite{Cremmer:1978km} consists of three fields;
\begin{itemize}
  \item one is the graviton $G_{MN}$ (44 states)
\item three-form field $M_{(3)}$ (84 states)
  \item    gravitino $\psi_M$ (128 states)
\end{itemize}
We can see that the numbers of fermions and bosons are balanced.
Although the existence of fermions is crucial,
we write the bosonic part of the action of the $D=11$ supergravity:
\begin{align}
 2\kappa_{11}^2 S_{11} = \int d^{11}x \sqrt{ -G} \left[R -\frac{1}{2} K_{(4)}^{2}\right]
 -\frac{1}{6}\int d^{11}x\ M_{(3)} \wedge K_{(4)} \wedge K_{(4)} \ ,\label{11dsugra}
\end{align}
where $K_{(4)} = d M_{(3)}$ is the field strength,
and $K_{(4)}^{2} = K_{(4)} \wedge * K_{(4)}$. The fermionic part of the action follows from supersymmetry in principle.
Note that there is only one parameter $\kappa_{11}$ defined as
\be\label{11dcoupling}\frac{1}{2\kappa_{11}^2} = \frac{2\pi}{(2\pi \ell_p)^9}~,\ee
which is written in terms of Planck length
\[\ell_{\mathrm{P}}=\sqrt{\frac{\hbar G}{c^{3}}}~.\]

The critical dimension of superstring theory is $D=10$. Nonetheless, Type IIA $D=10$ supergravity can be obtained from the dimensional reduction of the $D=11$ supergravity on $S^1$, as we see below. Furthermore, the $D=11$ supergravity gives an important clue to the strong coupling behavior of superstring theory.
%
% You may wonder why we looked into such a non-interesting theory
% in a sense that what we want is $10$ dimension, rather than $11$ dimension.
% One reason is that the interesting $10$d supergravity can be derived by
% dimensional reduction from the $11$ supergravity.
% Other reason, which is rather surprising, will be clear later.


\subsection{\texorpdfstring{$D=10$}{D=10} IIA supergravity}

Now we compactify the $D=11$ supergravity on $S^1$ of radius $R$. In the following, $C_{(p+1)}$ is the R-R $(p+1)$-form and $B_{(2)}$ is the $B$-field. Also, $G_{(p+2)}$ and $H_{(3)}$ are their field strengths. On the reduction,
the three-form field $M_{(3)}$ decomposes into the massless fields of Type IIA string theory in Table \ref{tab:masslessII}:
 \be M_{(3)} = C_{(3)} +B_{(2)} \wedge d\theta \ , \quad K_{(4)} = \wh G_{(4)} +H_{(3)} \wedge (d\theta +C_{(1)}) \ ,\ee
 where
 \be
\wh G_{(4)} = G_{(4)} -C_{(1)} \wedge H_{(3)}~.
\ee
Like \eqref{KK-metric}, the metric takes the form
\begin{align*}
 ds_{11}^2 = G_{MN} dx^M dx^N = G_{\mu\nu} dx^\mu dx^\nu +R^2 (d\theta+C_{(1)})^2 \ ,
\end{align*}
where the compactified direction is denoted as $\theta$.



Now we rewrite the action \eqref{11dsugra} in terms of the massless fields of Type IIA theory. However, we want to obtain the action of the form \eqref{bosonic-NS} in the NS-NS sector. For this purpose, we consider the radius of the circle depends on the spacetime coordinate and it can be written as the Dilaton fields as
\be\label{11d-radius}
R=\ell_p e^{\frac23\Phi}~.
\ee
Then, the action \eqref{11dsugra} is written as
\begin{align}\label{IIA-SUGRA}
 &S_\mathrm{A,NS} = \frac{1}{2\kappa_{10}^2} \int d^{10}x \sqrt{ -G} e^{-2\Phi} \left[
 R +4 \partial_\mu \Phi \partial^\mu \Phi -\frac{1}{2} H_{(3)}^{2} \right] \ ,  \cr
 &S_\mathrm{A,R} = -\frac{1}{4\kappa_{10}^2} \int d^{10}x \sqrt{ -G} \left[
G_{(2)}^{2} +\wh G_{(4)}^{2} \right] \ ,  \cr
 &S_\mathrm{A,CS} = -\frac{1}{4\kappa_{10}^2} \int B_{(2)} \wedge G_{(4)} \wedge G_{(4)} \ ,
\end{align}
where the coupling constants are related by
\be\label{kappa-11-10}
 \frac{2\pi R}{2\kappa_{11}^2} = \frac{e^{-2\Phi}}{2\kappa_{10}^2}~.
 \ee
Since it is the low-energy effective action of Type IIA string theory, the dimensional analysis of the coupling constant yields
\be
\frac{1}{2\kappa_{10}^2} = \frac{2\pi}{(2\pi \ell_s)^8}~.
\ee
Recalling that an expectation value of the dilaton is the string coupling $g_s=e^\Phi$, \eqref{11d-radius} gives
\be
 \left(\frac{R}{\ell_p}\right)^3 = g_s^2~. \ee
Also,
\eqref{kappa-11-10} provides the relation
\be
\frac{R}{\ell_p^3} = \frac{1}{\ell_s^2}\ee
Combining these two relations, we have
\be\label{Mcircle-radius} R = g_s \ell_s~.\ee


\subsection{\texorpdfstring{$D=10$}{D=10} IIB supergravity}\label{sec:IIB-SUGRA}

As seen in \S\ref{sec:Tdual}, Type IIA and IIB are T-dual to each other so that the action of Type IIB supergravity is consistent with the T-duality. In fact, the action takes a very similar form
\begin{align}\label{IIB-SUGRA}
 &S_\mathrm{B,NS} = \frac{1}{2\kappa_{10}^2} \int d^{10}x \sqrt{ -G}  e^{-2\Phi} \left[
 R +4 \partial_\mu \Phi \partial^\mu \Phi -\frac{1}{2} H_{(3)}^{2} \right] \ ,  \cr
 &S_\mathrm{B,R} = -\frac{1}{4\kappa_{10}^2} \int d^{10}x \sqrt{ -G} \left[
 G_{(1)}^{2}+ \wh G_{(3)}^{2} +\frac{1}{2} \wh G_{(5)}^{2} \right] \ ,  \cr
 &S_\mathrm{B,CS} = -\frac{1}{4\kappa_{10}^2} \int C_{(4)} \wedge H_{(3)} \wedge G_{(3)} \ ,
\end{align}
where
\bea
\wh G_{(3)} &= G_{(3)} -C_{(0)} H_{(3)}~,\cr
\wh G_{(5)} &=G_{(5)} -\frac{1}{2}C_{(2)} \wedge H_{(3)} +\frac{1}{2} B_{(2)} \wedge G_{(3)}~.
\eea
Note that the action for the NS sector is the same as that of Type IIA.

As in \eqref{SD-5form}, Type IIB supergravity has the self-dual 5-form $G_{(5)}$. In the notation here,
the self-dual condition is \[* \wh G_{(5)} = \wh G_{(5)}~,\]
which does not follow from the action.
In fact, if the field satisfies the self-dual condition, its kinetic action becomes trivial
\[
\int \wh G_{(5)}\wedge \ast \wh G_{(5)}=0~.
\]
Therefore, the self-duality condition needs to be imposed by hand.


\end{document}
