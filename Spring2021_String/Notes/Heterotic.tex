\documentclass[String-lecture-21.tex]{subfiles}

\begin{document}\section{Heterotic string theories}\label{sec:Heterotic}


We are ready to learn \textbf{Heterotic string theories} \cite{Gross:1984dd,Gross:1985fr,Gross:1985rr}. Heterotic string is a hybrid construction of the left-moving sector of the 26-dimensional bosonic string and the right-moving sector of 10-dimensional superstring. The 16 extra bosons of the left-movers are compactified on particular 16-dimensional tori, leading to $\SO(32)$ or $E_8\times E_8$. Since 16-dimensional tori have very special properties, we can also describe the left-movers in terms of 32 free fermions whose current algebra is associated to either  $\SO(32)$ or $E_8\times E_8$ with level $k=1$.  This hybridization of two different kinds of modes has been referred to as \textbf{heterosis}.


\subsection{Bosonic construction}

\subsubsection*{Toroidal compactifications}
We have learned the $S^1$ compactification in \S\ref{sec:Tdual} so that we now generalize our analysis to the compactification of bosonic string theory on a $D$-dimensional torus $T^D$. The resulting theory is effectively $(26-D)$-dimensional. The torus is defined by identifying points in the $D$-dimensional internal space as follows (compact dimensions are labeled with capital letters):
\be \label {torusequiv}
	X^I \sim X^I + 2 \pi e^I_i w^i =X^I+2\pi W^I~, \qquad \textrm{for} \quad w^i \in \bZ~.
	\ee
The $\mathbf{e}_i= \{e_i^I\}$  ($i= 1\cdots D$) are $D$ linear independent vectors called \textbf{vielbein} which generate a $D$-dimensional lattice $\Lambda$. In addition, the vielbein brings the metric into the standard Euclidean form:
\begin{align}
	G_{ij} =\bfe_i\cdot\bfe_j= e^I_i e^J_j \delta_{IJ}, \qquad 	X^I \equiv e^I_i X^i\end{align}
The torus on which we compactify is obtained by dividing $\bR^D$ by $\Lambda$:
\[T^D = \frac{{\bR}^D }{2\pi \Lambda}~.\]
The momentum $p^I$ conjugate to the coordinates $X^I$ on the torus
 is quantized as $\bfp \cdot \bfW \in\bZ$.	Therefore, the momentum $p$ takes its value on the dual lattice $\Lambda^*$
\[	\Lambda^* \equiv \{ e^{*Ii} n_i; \quad n_i \in \bZ \}, \qquad
		 G^{i j}=\bfe^{*i}\cdot\bfe^{*j}= e_I^{*i} e_J^{*j}\delta^{IJ}. \]

\begin{figure}[ht]\centering
\includegraphics[width=12cm]{picture/lattice}
\caption{Lattice and its dual lattice}
\end{figure}
The condition which a closed string in the compact directions $X ^I(z,\bz)=X_R^I(z)+\overline X_L^I(\bz)$ obey is
\[
X ^I( \s+2\pi,\t)=X ^I( \s,\t)+2\pi W^I
\]
so that $W^I$ are analogues of winding numbers. We express the mode expansion for the compact direction as follows:
\begin{align}
X_R^I(z)&=x^I-i\sqrt{\frac{\ap}{2}}p_R^I
\ln z+i\sqrt{\frac{\ap}{2}}\sum_{m\neq0}\frac{\alpha^I_m}{mz^m}, \nonumber\\
\overline X_L^I(\bz)&={\overline x}^I-i\sqrt{\frac{\ap}{2}}{p^I_L}
\ln \bz+i\sqrt{\frac{\ap}{2}}\sum_{m\neq0}\frac{{\overline\alpha}^I_m}{m\bz^m}.\nonumber
\end{align}
where the zero modes are
\begin{align}
\bfp_L := p^I_L &=\frac{1}{\sqrt{2}} \left[\sqrt{\a'}p^I + \frac {W^I}{\sqrt{\a'}}\right] =\frac{1}{\sqrt{2}}\left[ \sqrt{\ap}e^{* I i} n_i + \frac {e^I_i} {\sqrt{\ap}}  w^i\right]~,\cr
\bfp_R := p^I_R &=\frac{1}{\sqrt{2}} \left[\sqrt{\a'}p^I - \frac {W^I}{\sqrt{\a'}}\right] =\frac{1}{\sqrt{2}}\left[ \sqrt{\ap}e^{* I i} n_i - \frac{e^I_i} {\sqrt{\ap}}  w^i\right]~.
\end{align}
The mass formula and the level matching condition are now
\begin{align}
\ap M^2&=2(N+\overline N-2) + (\ap p_Ip^I+\frac1{\ap} W_IW^I)\cr
&=2(N+\overline N-2) +(\ap n_in_jG^{ij}+\frac1{\ap} w^iw^jG_{ij})\cr
N-\overline N&=p_I W^I=n_iw^i
\end{align}
As we have seen before,  the expressions for $p_L$ and $p_R$ suggest \textbf{T-duality} between the
winding number $W^I$ and the momentum $p^I$.  In fact, \textbf{T-duality} is equivalence between a pair of compactification lattices $\bfe_i$ and
$\bfe'_i$ that are related as $\sqrt{\ap}\bfe'_i = \frac{\bfe^{*i}}{\sqrt{\ap}}$.  These two
compactifications give the same spectrum since
their allowed values of the momenta are related as
\be \label {tdualp}
		\bfp_L \leftrightarrow \bfp_L';\qquad \bfp_R \leftrightarrow -\bfp_R'\ee
by interchanging the labels $n_i$ and $w^i$.


Now let us combine the zero modes into the $(D+D)$-dimensional vectors $\bfP=(\bfp_L,\bfp_R)$.
This construction treats $\Lambda$ and $\Lambda^*$ on equal footing
as
\[
\bfP=\bfE^{* i} n_i + \bfE_j w^j,
\]
where
\[
	\bfE_j = \frac1{\sqrt{\ap}}\left( \bfe_j,  - \bfe_j  \right)~, \quad
	\bfE^{* i} = \sqrt{\ap}\left(\bfe^{*i}, \bfe^{*i}  \right).
\]
Note that the length of the lattice is normalized by the string length $\sqrt{\ap}=\ell_s$.
Hence $\bfP$ takes value in a $(D+D)$-dimensional lattice $\G_{D,D}$
spanned by $\{\bfE^{* i}\}$ and $\{\bfE_j\}$  that satisfies the following properties:
\begin{itemize}\setlength{\parskip}{-0.1cm}
\item \textbf{Lorentzian} if the signature of the metric $G$ is $((+1)^D,(-1)^D)$,
\item \textbf{integral} if $v\cdot w \in \bZ$ for all $v,w\in \G_{D,D}$,
\item \textbf{even} if $\G_{D,D}$ is integral and $v^2$ is even for all $v\in\G_{D,D}$,
\item \textbf{self-dual} if $\G_{D,D} =(\G_{D,D})^*$,
\item \textbf{unimodular} if $\textrm{Vol}(\G_{D,D}) =  | \det G| = 1$.
\end{itemize}
In fact, the metric of this lattice is defined by
\[
\bfP\cdot\bfP'=(\bfp_L\cdot\bfp'_L-\bfp_R\cdot\bfp'_R)=n_iw'^{i}+n'_iw^{i}
\]
so that it is Lorentzian. Because of ${\bf P\cdot P}\in2\bZ$, it is even. The self-dual property will be shown in Homework. The unimodular property $\textrm{Vol}(\G_{D,D})=\textrm{Vol}(\G_{D,D})=1$ immediately follows from  the self-dual property.
The lattice $\G_{D,D}$  in the torus compactification of the string is called \textbf{Narain lattice}.


 The partition function of the bosonic string compactified on a torus $T^D$ is easy to write down:
\[	Z^{\textrm{bos}}_{\G_{D,D}} = \frac 1 {\t_2^{(24-D)/2}| {\eta (q)}|^{48}}
		\sum_{(\bfp_R, \bfp_L) \in \G_{D,D}} q^{\frac{1}{2} \bfp_R^2}
								\bar q^{\frac{1}{2} \bfp_L^2}
\]
where $| {\eta (q)}|^{48}$ is the bosonic oscillator contribution and $\t_2^{(24-D)/2}$ comes from
the integral of non-compact momenta.
This is easy to generalize to Type II string compactified on $T^D$
\[
Z^{\textrm{Type II}}_{\G_{D,D}} = \frac 1 {\t_2^{(8-D)/2}| {\eta (q)}|^{24}} \frac14\Big| -\vartheta_2^4(\tau) + \vartheta_3^4(\tau) - \vartheta_4^4(\tau)\Big|^2
		\sum_{(\bfp_R, \bfp_L) \in \G_{D,D}} q^{\frac{1}{2} \bfp_R^2}
								\bar q^{\frac{1}{2} \bfp_L^2}
\]
which vanishes by virtue of the Jacobi-Riemann identity.

\subsubsection*{Heterotic strings}\label{sec:Heterotic-bosonic}

After we learn about toroidal compactifications, we are ready to introduce the $D=10$ Heterotic string.
As mentioned at the beginning, Heterotic string is a combination of the left-moving sector of the 26-dimensional bosonic string and the right-moving sector of the 10-dimensional superstring. The left-moving bosonic string is compactified on a 16-dimensional torus so that the momenta of the additional chiral bosons $X^I(\bar z)$ takes value on 16-dimensional lattice  $\G_{16}$, \textbf{i.e} $\bfp_L\in\G_{16}$.
Hence, the partition function of Heterotic string can be written as
\begin{equation}\label{het-pf}
	Z^{\textrm{het}}(\tau )  = \frac 1 {\t_2^{4}\eta (q)^{12}\eta (\bar q)^{24}} \Big( -\vartheta_2^4(\tau) + \vartheta_3^4(\tau) - \vartheta_4^4(\tau)\Big)
		\sum_{\bfp_L \in \G_{16}} 	\bar q^{\frac{1}{2} \bfp_L^2}
\end{equation}
Here $\eta (q)^{8}\eta (\bar q)^{24}$ is the bosonic oscillator contribution, the $\t_2^{4}$ factor arises from the zero modes of the noncompact transverse coordinates and $\vartheta^4_i/\eta (q)^{4}$ comes from the world-sheet fermions. The most interesting part of this partition function is the lattice sum
\[
P(\tau):=\sum_{\bfp_L \in \G_{16}} 	\bar q^{\frac{1}{2} \bfp_L^2}
\]
Since the partition function \eqref{het-pf} should be invariant under the modular transformation $\SL(2,\bZ)$, the modular transformation \eqref{S-T} of $\eta$ and $\vartheta_i$ tell us that
\[
T:P(\t+1)=P(\t)~,\qquad S:P(-1/\t)=\t^8P(\t)~.
\]
The invariance under T-transformation clearly demands that $\bfp_L^2\in 2\bZ$ so that $\G_{16}$ must be \textbf{even}.
 For the $S$-transformation, we
make use of the Poisson resummation formula
\[	\sum_{\bfp\in\Lambda}e^{-\pi\a(\bfp+\bfx)^2+2\pi i \bfy\cdot (\bfp+\bfx)}=\frac{1}{\textrm{Vol}(\Lambda) \a^{\dim \Lambda /2}}\sum_{\bfq\in\Lambda^*}e^{-2\pi i \bfq\cdot \bfx-\frac{\pi}{\a}(\bfy+\bfq)^2}	\]
which amounts to
\[
P(-1/\t)=\frac{\t^8}{\textrm{Vol}(\Lambda)}\sum_{ \bfp_L\in(\G_{16})^*} 	\bar q^{\frac{1}{2} \bfp_L^2}~.
\]
This requires that the lattice $\G_{16}$ is \textbf{self-dual}, \textit{i.e.} $(\G_{16})^*=\G_{16}$ so that  $\textrm{Vol}(\Lambda)=1$.



It turns out that there are only two even self-dual Euclidean lattices in 16 dimensions
\begin{itemize}\setlength{\parskip}{-0.1cm}
\item the root lattice of $E_8\times E_8$
\item the weight lattice of $\Spin(32)/\bZ_2$
\end{itemize}
The metric $G_{ij}$ of the root lattice of $E_8$ is the Cartan matrix of $E_8$ \footnote{Unfortunately, we do not have time to talk about exceptional Lie algebras or classification of semi-simple Lie algebras \cite{kirillov2008introduction}. In particular, if you want to get some intuition of weight and root lattices, see \cite[Fig 7.3, Fig 8.1, Fig 8.2]{kirillov2008introduction} for $A_2$. If you want to understand the structure $E_8$ related to string theory, we refer to \cite[\S 6]{GSW}.}:
\begin{equation}\nonumber
\left (
\begin{smallmatrix}
 2 & -1 &  0 &  0 &  0 &  0 &  0 & 0 \\
-1 &  2 & -1&  0 &  0 &  0 &  0 & 0 \\
 0 & -1 &  2 & -1 &  0 &  0 &  0 & 0 \\
 0 &  0 & -1 &  2 & -1 &  0 &  0 & 0 \\
 0 &  0 &  0 & -1 &  2 & -1 &  0 & -1 \\
 0 &  0 &  0 &  0 & -1 &  2 & -1 & 0 \\
 0 &  0 &  0 &  0 &  0 & -1 &  2 & 0 \\
 0 &  0 & 0 &  0 &  -1 &  0 &  0 & 2
\end{smallmatrix}\right ). 		\qquad\qquad	 \raisebox{-.5cm}{\includegraphics[width=5cm]{picture/Dynkin_diagram_E8}}
\end{equation}


Let us read off massless fields in Heterotic string theory more carefully.
As usual, there is the tachyonic vacuum of the bosonic string. At the massless level, we have oscillator excitations $\overline \a_{-1}^\mu|0\rangle$, $\overline \a_{-1}^I|0\rangle$ in the left-moving sector. The former transform like spacetime vectors while the internal oscillator excitations correspond to the left-moving part of the Abelian $\U(1)^{16}$ gauge boson. They form the \textbf{Cartan subalgebra} of $E_8 \times E_8$ or $\SO(32)$. Both the root lattice of $E_8\times E_8$ and the weight lattice of $\Spin(32)/\bZ_2$  contain 480 vectors of (length)$^2=2$ and generate the 496-dimensional non-Abelian gauge bosons of these groups. Remarkably, although Heterotic strings are closed strings, gauge fields show up thanks to the extra 16-dimensional tours! This can be also understood as a novel \textbf{stringy effect} and gauge groups are restricted only to either $E_8 \times E_8$ or $\SO(32)$ in order for the theory to be consistent. Moreover, we have seen that $\SO(32)$ gauge group appears in Type I string theory. As we will see in \S\ref{sec:dualities}, this is not coincidental because Type I and Heterotic $\SO(32)$ are related by \textbf{S-duality}.



As a result, the massless spectra of Heterotic string are as follows

\vspace{.3cm}
\noindent $\bullet$ Gravitons, $B$-fields, dilaton in $D=10$
\[
\psi^\mu_{-\frac12}|0\rangle_{\textrm{NS}}\otimes\overline\alpha_{-1}^\nu|0\rangle
\]
$\bullet$  their supersymmetric partners, gravitino and dilatino
\[
|\mathbf{s}\rangle_{\textrm{R}}\otimes\overline\alpha_{-1}^\nu|0\rangle
\]
$\bullet$ 496 gauge bosons of  $E_8\times E_8$ or $\SO(32)$
\[
\psi^\mu_{-\frac12}|0\rangle_{\textrm{NS}}\otimes\overline\alpha_{-1}^I|0\rangle~,\qquad \psi^\mu_{-\frac12}|0\rangle_{\textrm{NS}}\otimes|\bfp_L^2=2\rangle
\]
$\bullet$  496 supersymmetric partners, gaugini
\[
|\mathbf{s}\rangle_{\textrm{R}}\otimes\overline\alpha_{-1}^I|0\rangle~,\qquad |\mathbf{s}\rangle_{\textrm{R}}\otimes|\bfp_L^2=2\rangle
\]
Indeed, Heterotic string theory is $D=10$ $\cN=1$ supergravity coupled to $D=10$ $\cN=1$  $E_8\times E_8$ or $\SO(32)$  super-Yang-Mills theory so that it has 16 real supersymmetric charges.


\subsection{Fermionic construction}

The 16 bosonic fields compactified on the self-dual lattice can be described by fermionic fields, which is called \textbf{fermionization}. Therefore, we will describe fermionic construction of Heterotic string theory next.

The world-sheet action of Heterotic string theory is given by
\begin{align}
 S^{\textrm{m}} &= \frac{1}{4\pi} \int d^2 z\ \Big( \frac{2}{\alpha'} \partial X^\mu  \overline\partial X_\mu+\psi^\mu\overline\partial\psi_\mu+\overline \lambda^A\partial\overline\lambda_A\Big)\cr
S^{\textrm{gh}}&=\frac{1}{2\pi}\int d^2z \ (b\overline \partial c+\bar b \partial \bar c+\beta\overline \partial \g)
\end{align}
where $\mu$ are 10-dimensional indices and the right-moving sector is supersymmetric. In order for the theory to be Weyl-anomaly free,  the central charge
\[
c^{\textrm{tot}}=c^X+c^\psi+c^{bc}+c^{\beta\gamma}+c^{\lambda}=10+\frac52-26+\frac{11}{2}+c^{\lambda}=c^{\lambda}-8
\]
should vanish. Since each Majorana-Weyl anti-chiral fermion $\overline\lambda^A$ contributes $\frac14$ to the central charge, we need 32 left-moving fermions $\overline\lambda^A$ in the action.


It turns out that  there are two possible boundary conditions on the left-moving
fermions $\overline\lambda_A$ which give rise to fully consistent string theories. If we impose the same boundary condition to all, it leads to $\SO(32)$ gauge group. On the other hand, if we impose one boundary condition to a half and the other boundary condition to the other half, we obtain $E_8\times E_8$ gauge group.

\subsubsection*{Heterotic $\SO(32)$ (HO)}
For Heterotic $\SO(32)$, we impose the same boundary condition to all the left-moving fermions as
\begin{align}
\overline \lambda^A(t,\sigma+2\pi)=+\overline \lambda^A
  (t,\sigma) &\qquad\qquad\textrm{ R: periodic
on cylinder}\cr
\overline \lambda^A(t,\sigma+2\pi)=-\overline \lambda^A
  (t,\sigma) &\qquad\qquad\textrm{ NS:
anti-periodic on cylinder}
\end{align}
so that there is a global symmetry $\SO(32)$ that rotates  $\overline\lambda^A$ ($A=1,\ldots,32$). In order for the theory to be consistent, we have to impose GSO projection on the left-moving sector. In HO theory, we pick only states with odd fermionic numbers in NS sector and those with even fermionic number
\[
P_{\textrm{NS}}^{\textrm{HO}}:=\frac{1-(-1)^F}{2}
\]
whereas we keep only the states with even fermion number
\[
P_{\textrm{R}}^{\textrm{HO}}:=\frac{1+(-1)^F}{2}~.
\]
In addition, we have to impose the level matching condition
\be\label{level-matching2}
N-a=\overline  N-\overline a
\ee
where the normal ordering constants in the left-moving sector are
\[
{\overline a}_{\textrm{NS}}=\frac{8}{24}+\frac{32}{48}=1~,\qquad {\overline a}_{\textrm{R}}=\frac{8}{24}-\frac{32}{24}=-1~.
\]
Here the first term comes from the left-moving bosonic field $\overline X^i$ whereas the second term depends on the boundary condition \eqref{RNS-normalordering} of $\overline\lambda^A$. Hence, the R sector contains only massive states. Contrary to the supersymmetric right-mover,  the Tachyon state $ |0\rangle_{\textrm{NS}}$ in the NS sector is preserved under the GSO projection. However, there is no corresponding state in the right-moving sector so that it does not obey the level matching condition \eqref{level-matching2}.  As a result, the left-moving Tachyon is not included in the spectrum. Then, the first excited states after the GSO projection in the NS sector are
\begin{align}
\overline \a^i_{-1} |0\rangle_{\textrm{NS}} \,&,\qquad (\mathbf{8_v},\mathbf{1}) \cr
\overline \lambda^A_{-1/2} \overline \lambda^B_{-1/2} |0\rangle_{\textrm{NS}}\,&, \qquad  (\mathbf{1},\textbf{adj}) \nonumber
\end{align}
where the bold letters are the representations of $\SO(8)\times \SO(32)$. The adjoint representation $\textbf{adj}$ of $\SO(32)$ is the antisymmetric tensor with dimension $32 \times 31/2 = \textbf{496}$. The following table shows the massless spectrum of HO where the first row represents the $D=10$ $\cN=1$ supergravity multiplet whereas the second row shows  $\cN = 1$ gauge multiplet in the adjoint of $\SO(32)$ as we have seen in the bosonic construction.
\begin{table}[ht] \centering
\begin{tabular}{ |c|c|c| }
 \hline
Left$\backslash $Right & ${\bf 8_v}$ & ${\bf 8_c}$ \\ \hline
 $ (\mathbf{8_v},\mathbf{1})$ & ${\bf 1} \oplus {\bf 28}  \oplus {\bf 35}$ & ${\bf 8_s}\oplus{\bf 56_c}$ \\
  & $\phi \ \ B_{\mu\nu} \ \ G_{\mu\nu}$ & $\lambda^+ \ \ \psi^-_m$ \\ \hline
  $(\mathbf{1},\textbf{496})$ & $\SO(32)$ gauge boson &  $\SO(32)$ gaugini \\
    & $A^\mu_{[A,B]}$ & $\eta_{[A,B]}$ \\
 \hline
\end{tabular}
\end{table}

\subsubsection*{Heterotic $E_8\times E_8$ (HE)}
The second Heterotic string theory is obtained by dividing the $\overline \lambda^A$ into two sets of 16 with independent boundary conditions,
\[
\overline\lambda^A(t,\sigma+2\pi)=\left\{ \begin{matrix}\e_1 \overline\lambda^A(t,\sigma) & \quad~ A=1,\ldots,16\\ \e_2 \overline\lambda^A(t,\sigma) & \qquad A=17,\ldots,32  \end{matrix}\right.
\]
where $\e_i=\pm1$. Therefore, in the left-moving sector,  we need to take the following boundary conditions into account
\[
(\textrm{NS}_1,\textrm{NS}_2)~,\quad (\textrm{R}_1,\textrm{NS}_2)~,\quad (\textrm{NS}_1,\textrm{R}_2)~,\quad (\textrm{R}_1,\textrm{R}_2)~.
\]
Consequently, the global symmetry is broken to $\SO(16)_1\times\SO(16)_2$. The GSO projection is imposed to the two sets of left-movers independently:
\[
P_{\textrm{NS}_i}^{\textrm{HE}}:=\frac{1-(-1)^F}{2}
~\qquad
P_{\textrm{R}_i}^{\textrm{HE}}:=\frac{1+(-1)^F}{2}~.
\]
We also apply for the level-matching condition \eqref{level-matching2}. The normal ordering constant in each boundary condition is
\[
{\overline a}_{\textrm{NS}_1,\textrm{NS}_2}=1~,\qquad {\overline a}_{\textrm{R}_1,\textrm{NS}_2}={\overline a}_{\textrm{NS}_1,\textrm{R}_2}=\frac{8}{24}+\frac{16}{48}-\frac{16}{24}=0~,\qquad {\overline a}_{\textrm{R}_1,\textrm{R}_2}=-1~.
\]
Again, $ (\textrm{R}_1,\textrm{R}_2)$ boundary condition has only massive states. Although the Tachyon state $ |0\rangle_{\textrm{NS}_1,\textrm{NS}_2}$ in the NS sector is preserved under the GSO projection, it does not obey the level-matching condition  \eqref{level-matching2} so that it is not present in the spectrum. Then, the massless states are
\begin{align}
\overline \a^i_{-1} |0\rangle_{\textrm{NS}_1,\textrm{NS}_2} \,&,\qquad (\mathbf{8_v},\mathbf{1},\mathbf{1}) \cr
\overline \lambda^A_{-1/2} \overline \lambda^B_{-1/2} |0\rangle_{\textrm{NS}_1,\textrm{NS}_2}\,&, \qquad  (\mathbf{1},\textbf{adj},\mathbf{1}) \ \textrm{or} \ (\mathbf{1},\mathbf{1},\textbf{adj})  \label{massless2}
\end{align}
where the bold letters are the representations of $\SO(8)\times \SO(16)_1\times \SO(16)_2$. Note that the GSO projection requires either $1\le A,B\le16$ or $17\le A,B\le32$ in \eqref{massless2}. The adjoint representation of $\SO(16)$ is of $16\times15/2=\textbf{120}$ dimensions.

In $ (\textrm{R}_1,\textrm{NS}_2)$ and $(\textrm{NS}_1,\textrm{R}_2)$, the ground states are massless since the normal ordering constant is zero. Since the 16 $\overline \lambda_0^A$ zero modes form 8 raising and 8 lowering operators
\[
\overline \lambda_0^{K\pm} = 2^{-1/2}(\overline \lambda_0^{2K-1} \pm i\overline \lambda_0^{2K})~ , \qquad  K = 1,\ldots, 8 \ \textrm{or} \ K = 9,\ldots, 16~,
\]
the $2^{8}=\textbf{256}$-dimensional spinor representation of $\SO(16)$ becomes massless.
However, the GSO projection picks positive chirality $\textbf{128}$ out of $\textbf{256}=\textbf{128}+\textbf{128}'$ in the Ramond sector. Hence, the ground states are $(\mathbf{1},\textbf{128},\mathbf{1})$ and $(\mathbf{1},\mathbf{1},\textbf{128})$ under $\SO(8)\times \SO(16)_1\times \SO(16)_2$ in $ (\textrm{R}_1,\textrm{NS}_2)$ and $(\textrm{NS}_1,\textrm{R}_2)$, respectively .

All in all, the left-moving massless states form the representations of $\SO(8)\times \SO(16)_1\times \SO(16)_2$
\[
\bf (8_v,1,1) + (1,120,1) + (1,1,120) + (1,128,1) + (1,1,128)
\]
This spectrum strongly suggests that gauge symmetry is enhanced $\SO(16)\to E_8$ because $E_8$ has dimension $\bf 120+128=248$ which is also the dimension of the adjoint representation $E_8$. In fact, $E_8$ has an $\SO(16)$ subgroup under which the $E_8$ adjoint $\bf 248$ transforms as $\bf 120 + 128$. Hence, the massless spectrum is the  $D=10$ $\cN=1$ supergravity multiplet plus an  $\cN=1$ $E_8 \times E_8$ gauge multiplet. Even in fermionic construction, we reproduce the 496-dimensional adjoint representations of both $\SO(32)$ and $E_8\times E_8$ gauge groups.


\subsubsection*{No D-branes in  Heterotic strings}
We have seen that D-branes are charged to R-R fields in Type II theories. However, there is no R-R field in Heterotic string theories because there is only world-sheet supersymmetry in the right-moving sector. In other words, although the R-R $(p+2)$-form field strength $G$ in Type II theories can be expressed as
\[
G=\overline \psi^L \G^{\mu_1\cdots \mu_{p+2}}\psi^R~,
\]
there is no $\psi^L$ in Heterotic string theories.
Hence, there is no D-brane in Heterotic string theories. Consequently, Heterotic string theories are the theories of closed strings\footnote{However, Polchinski argues in \cite{Polchinski:2005bg} that there exist open Heterotic strings.}. However, apart from the fundamental strings, there are extended objects, \textbf{NS5-branes or Heterotic fivebranes}, in Heterotic string theories and they are magnetically charged under the $B$-field.
\end{document}
