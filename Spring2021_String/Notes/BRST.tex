\documentclass[String-lecture-21.tex]{subfiles}

\begin{document}
\section{BRST quantization}\label{sec:BRST}


\subsection{Quantization via path integral}\label{sec:ghost}

We have been studying bosonic string theory, but there are several caveats.

\vspace{.3cm}
\noindent$\bullet$ In the light-cone quantization \eqref{lcg}, Lorentz invariance is not manifest.  Can we quantize strings in a way that is manifestly Lorentz invariant?

\vspace{.3cm}

\noindent$\bullet$ We have seen that the Weyl symmetry of the string sigma model is anomalous \eqref{Weyl-anomaly} on a curved world-sheet. How can the bosonic string be anomaly-free?


\vspace{.3cm}

\noindent$\bullet$ Although we learned that string amplitude is expressed via Feynman path integral
\eqref{amplitude},
we do not know how to perform this path integral. In particular, the path integral is endowed with huge gauge symmetries, world-sheet diffeomorphism and Weyl symmetry. How can we treat integration measure and fix gauge in the path integral?

\vspace{.3cm}


To answer these questions, we will study the quantization procedure via path integral, which is often called \textbf{modern covariant quantization}. This method uses  the analog of the
Faddeev-Popov method of gauge theories. Furthermore, the physical state condition is imposed via the BRST symmetry.





After gauge-fixing the reparametrization and Weyl symmetry, the integral over $h_{ab}$ turns into a path-integral over ghost CFT, which has $c^{g}=-26$. Therefore, in order for the theory to be Weyl anomaly-free, the matter part of the theory has to be a CFT with $c^X=26$.

\subsubsection*{Faddeev-Popov gauge fixing}



\begin{figure}[ht]\centering
\includegraphics[width=9cm]{picture/gphase}
\caption{We need to fix a gauge in the field configuration space. Physically inequivalent configurations are depicted by the dotted line.}\label{fig:gauge}
\end{figure}

The integration measure can be written as $[\cD X_\mu][\cD h_{ab}]_{g,n}$ where the scalar fields and the metric on 2d surfaces with genus $g$ and $n$ marked points are integrated out. However, if there is no anomaly, this integral has the world-sheet diffeomorphisms and Weyl symmetry under which the field configurations are transformed as
\bea\label{global}
X_\mu^\zeta(\wt\s)=&X_\mu^(\s) \cr
h_{ab}^\zeta(\wt\s)=&e^{2\omega(\s)}  \frac{\partial \s^c}{\partial \wt\s^a}\frac{\partial \s^d}{\partial \wt\s^b} h_{cd}(\s)~,
\eea
where $\zeta$ indicates Diff$\times$Weyl.
These symmetries are redundant and integrating along these directions just gives rise to the volume of the symmetry group. (In Figure \ref{fig:gauge}, the solid arrows schematically draw gauge redundancy and  the dotted line shows physically distinct configurations.) Thus, we need to carry out gauge-fixing. Thankfully, there is a
standard method to fix gauge, introduced by Faddeev and Popov. A basic idea is to insert the identity of the following form in the path integral:
\be1= \int {\cal D}\zeta\ \delta(h-\hat{h}{}^{\,\zeta}) \det \left(\frac{\delta \hat{h}{}^{\,\zeta} }{\delta \zeta}\right)\label{dfp}\ee
where the Jacobian
factor  $\det \left(\frac{\delta \hat{h}{}^{\,\zeta} }{\delta \zeta}\right)$ is called \textbf{Faddeev-Popov determinant} and we denote it by $\Delta_{FP}[\hat{h}]$. Since ${\cal D}\zeta$ is a gauge-invariant measure, $\Delta_{FP}[\hat{h}]$ is independent of $\zeta$ so that it can be factored out of the integral above. The insertion of this identity into the path integral fixes the metric as $\hat{h}$ because of the delta functional. Because $\int {\cal D}\zeta$ integral only contributes an infinite multiplicative factor, we can discard this integral. Therefore, after Faddeev-Popov gauge fixing, the
form of the path integral can be schematically written as
\be Z[\hat{h}] = \int {\cal D}X\ \Delta_{FP}[\hat{h}]\,e^{-S_{\s}[X,\hat{h}]}\label{fullpf}\ee
We should make several remarks about this procedure.

\vspace{.3cm}
\noindent$\bullet$ First, the conformal transformations are residual gauge symmetries not fixed above. We have to throw away these residual gauge symmetries in the path integral in order to avoid over-counting. Indeed we will be careful to fix this extra residual gauge freedom when computing string amplitudes.

\vspace{.3cm}
\noindent$\bullet$ Second, there are caveats related to global properties of the world-sheet Riemann surface $\Sigma_{g,n}$. In fact, metrics on a Riemann surface encode the information of ``shape'' of the Riemann surface $\Sigma_{g,n}$ called \textbf{complex moduli}, which is not accounted for by local gauge transformations $\zeta$. The space which parametrizes ``shape'' of the Riemann surface is called \textbf{moduli space of Riemann surface} $\Sigma_{g,n}$ which is $(6g-6+2n)$-dim$_\bR$:
\[
\cM_{g,n}:=\frac{\left[\cD h_{ab}\right]_{g,n}}{\textrm{Diff}\times\textrm{Weyl}}
\]
Therefore, the path integral involves integrals over $\cM_{g,n}$ as well. At this moment, we postpone both the issues and will come back to them in \S\ref{sec:amplitude} on string amplitudes.

\vspace{.3cm}
Now let us take an infinitesimal version of \eqref{global} where
a Weyl transformation is parameterized by $\omega(\sigma)$ and an
infinitesimal diffeomorphism by $\delta\sigma^\alpha = \e^\alpha(\sigma)$. Subsequently, the change of the metric under a gauge transformation is read off
%
\be\label{gauge-orbit}\delta \hat{h}_{ab}  = 2\omega\hat{h}_{ab} + \nabla_a \e_b + \nabla_b \e_a:=2\wt \omega \hat{h}_{ab} +(P\cdot \e)_{ab}~,\ee
%
where we decompose it into
\begin{align}
(P\cdot \e)_{ab}&=\nabla_a \e_b + \nabla_b \e_a-h_{ab}(\nabla\cdot \e) \cr
\wt \omega&=\omega+\frac12 (\nabla\cdot \e) ~.
\end{align}

Indeed, the operator $P$ maps vectors $\e_a$ to symmetric traceless two-tensors $(P\cdot \e)_{ab}$. Thus, the Faddeev-Popov determinant can be written as
\[
\Delta_{FP}[\hat{h}]=\det \frac{\delta(P\cdot \e,\wt\omega)}{\delta(\e,\omega)}=\det \left| \begin{matrix}P&0\\\ast &1
\end{matrix}
\right|=\det P~.
\]

To compute $\det P$, we use \textbf{Faddeev-Popov ghosts}, which can be understood as an infinite-dimensional version of the following integral.
Given a matrix $M_{ij}$, its determinant can be expressed as a Grassmann integral
\[
\int \prod_{i=1}^nd\psi_i d\theta_i  \exp(\theta_i M_{ij}\psi_j) = \det M~.
\]
where $\theta, \psi$ are Grassmann variables. Accordingly, we introduce anti-commuting fermionic fields, $c^a$ (ghosts) and $b_{ab}$ (anti-ghost) where $b_{ab}$ transforms as a symmetric traceless tensor  and $c^a$ as a vector. Then, we can express
\[
\Delta_{FP}[\hat{h}] =\int{\cal D} c{\cal D} b\, \exp\left(\frac{i}{2\pi}\int d^2\sigma \sqrt{-\hat{h}}\, b^{ab} (P\cdot c)_{ab}\right):= \int{\cal D}c{\cal D}b\ \exp[i S_{g}]~,
\]
where the ghost action can be written as
\be\label{ghost1} S_{g}=\frac{1}{2 \pi} \int d^{2} \sigma \sqrt{-\hat{h}} b^{a b} \nabla_{a} c_{b} ~.\ee
Even though the $bc$ ghost fields were introduced to fix a gauge, they look like dynamical fields with the action above. Consequently, the Faddeev-Popov gauge fixing results in a fermionic 2d CFT, usually called \textbf{$bc$ ghost CFT}.

Let us make some remarks about the equation of motion of $S_{g} $

\vspace{.3cm}
\noindent$\bullet$ The equation of motion for $c_a$ is given by
\be\label{CKV}
P\cdot c = \nabla_a c_b +\nabla_b c_a
-h_{ab} (\nabla \cdot c) = 0~.
\ee
Therefore the solutions for $c$ are in one-to-one correspondence with the \textbf{conformal Killing vectors}, which are the generators of the residual symmetry.

\vspace{.3cm}
\noindent$\bullet$ The equation of motion for $b_{ab}$ is
\be\label{metric-moduli} \nabla_a b^{ab}=0 ~.\ee
We will understand the geometric meaning of these equations when discussing the moduli
space of Riemann surfaces.

\vspace{.3cm}
To understand the properties of the $bc$ ghost CFT, it is convenient to use the Euclidean signature so that we will perform Wick rotation in what follows. Then, the factor of $i$ in the action disappears.
The expression for the full partition function \eqref{fullpf} is
%
\be Z[\hat{h}]=\int {\cal D}X{\cal D}c{\cal D}b\ \exp\left(-S_{\s}[X,\hat{h}] -
S_{g}[b,c,\hat{h}]\right)~.\label{total}\ee
%









\subsubsection*{$bc$ ghost CFT}

Now let us study the $bc$ ghost CFT more in detail. For this purpose, we pick the conformal gauge \eqref{conformal-gauge} for a world-sheet metric. Using \eqref{conformal-gauge2}, the ghost action can be written as
\begin{align}
S_{g}& = \frac{1}{2\pi}\int d^2z\ \left( b_{zz}\nabla_{\bz}c^z + b_{\bz\bz}\nabla_z c^{\bz}\right)\cr
&=\frac{1}{2\pi} \int d^2z\ b_{zz}\,\partial_{\bz}c^z +
b_{\bz\bz}\,\partial_zc^{\bz}\end{align}
For the sake of simplicity, let us define
%
\begin{align} b= b_{zz}~,\qquad \bar{b}=b_{\bar{z}\bar{z}}~,\qquad
c=c^z~,\qquad \bar{c}=c^{\bar{z}}~.\nonumber\end{align}
%
Then, the action simplifies to
\be\label{ghost-action}
S_{g} = \frac{1}{2\pi}\int d^2z\ \left(b\,\overline \partial c + \bar{b}\,\partial\bar{c}
\right)~,\ee
which gives the equations of motion
\be \label{ghost-holo}\overline \partial b = \partial\bar{b} = \overline \partial c = \partial\bar{c}=0\ee
Thus, we see that $b$ and $c$ are holomorphic fields, while $\bar{b}$ and $\bar{c}$ are
anti-holomorphic.





To compute the OPEs of the $bc$ ghost fields, we use the path integral techniques  in \S\ref{sec:OPE}:
\be
0=\int \mathcal{D} c \mathcal{D} b \frac{\delta}{\delta b(z)}\left[e^{-s_{g}} b(w)\right]=\int \mathcal{D} c \mathcal{D} b s^{-s_{g}}\left[-\frac{1}{2 \pi} \bar{\partial} c(z) b(w)+\delta(z-w)\right]~,\nonumber
\ee
which tells us that
\be \overline \partial c(z)\,b(w) = 2\pi \,\delta(z-w)~.\nonumber\ee
We can perform a similar computation for $c(z)$, which yields
\be \overline \partial b(z)\,c(w) = 2\pi\,\delta(z-w)~.\nonumber\ee
We can integrate both of these equations using \eqref{eq:deltaFunc}.  Then, we obtain the $bc$ OPE
\begin{align} b(z)\,c(w) &= \frac{1}{z-w} + \ldots\nonumber\\ c(w)\,b(z) &= \frac{1}{w-z} + \ldots\nonumber\end{align}
In fact, the second equation follows from the first equation and Fermi statistics. Hence, the
OPEs of $b(z)\,b(w)$ and $c(z)\,c(w)$ are trivial for the obvious reason.


In any CFT, it is of most importance to find the form of the energy-momentum tensor.
The energy-momentum tensor is obtained via Noether's theorem with respect to world-sheet transformations $\delta z = \e(z)$, under which
\[\delta b = (\e\partial+2(\partial\e))b~,\qquad \d c = (\e\partial-(\partial\e))c~.\]
Indeed both $b$ and $c$ are primary fields with conformal dimensions $h=2$ and $h=-1$, respectively, which can be easily seen from their index structure $b_{zz}$ and $c^z$.
From these rules, one can deduce the form of the energy-momentum tensor
\be\label{Tgh} T^{g}(z) =- 2:b(z)\partial c(z): + :c(z)\partial b(z):~.\ee
In fact, this form can be obtained from the first principle, namely the variation of the action under the metric (Exercise).



The OPEs of $b$ and $c$ with the stress tensor are
\begin{align}
&T^{g}(z) c(w)=-\frac{c(w)}{(z-w)^{2}}+\frac{\partial c(w)}{z-w}+\ldots \cr
&T^{g}(z) b(w)=\frac{2 b(w)}{(z-w)^{2}}+\frac{\partial b(w)}{z-w}+\ldots
\nonumber\end{align}
so that $b$ and $c$ are primary fields of conformal dimension $2$ and $-1$, respectively. Consequently, they admit the mode expansions
\[
b(z)=\sum_{m\in\bZ}\frac{b_m}{z^{m+2}}\qquad c(z)=\sum_{m\in\bZ}\frac{c_m}{z^{m-1}}~.
\]
The ghost OPEs give the commutation relations
\be\label{ghost-comm}\left\{b_{m},c_{n}\right\}=\delta_{m+n,0}~, \quad\left\{c_{m}, c_{n}\right\}=\left\{b_{m}, b_{n}\right\}=0~.\ee
Also, the Virasoro generators of the $bc$ ghost are expressed as
\be \label{L-bc}
L^{g}_m=\sum_{n\in\bZ}(2m-n):b_nc_{m-n}:+a^g\delta_{m,0}~.
\ee
The constant $a^g$ in the zero mode $L_0^{g}$ is determined by the Casimir energy $\frac{c^g}{24}$ in \eqref{Casimir2} and the commutation relation of the $bc$ modes $\frac1{12}=-\sum_{n>0}n$
\[
a^g=\frac{-26}{24}+\frac1{12}=-1~.
\]
This is consistent with the commutation relation $L_0=[L_1,L_{-1}]$ (\cite[(2.7.20)]{Polchinski}).

Finally, we can compute the $TT$ OPE  (Exercise)
%
\be T^{g}(z)\,T^{g}(w) = \frac{-13}{(z-w)^4}+\frac{2T^{g}(w)}{(z-w)^2} + \frac{\partial T^{g}(w)}{z-w} +\ldots\nonumber\ee
%
Now one can read off the  central charge of the $bc$ ghost system, which is
%
\be c^{g}=2(-13)=-26 ~.\nonumber\ee

We learned that the Weyl symmetry is anomalous unless $c=0$.
Since the Weyl symmetry is a gauge symmetry, the theory must be Weyl anomaly-free. Since the total central charge of the string sigma model and ghost theory \eqref{total} is given by $c=c^X +c^{g}$, the dimension of the target space must be $D=26$.
Again, we obtain the critical dimension of the bosonic string theory!




\subsection{BRST quantization}

In 4d Yang-Mills theory, the Lagrangian with Faddeev-Popov ghosts has the continuous symmetry, called \textbf{BRST symmetry} (Becchi-Rouet-Stora-Tyupin). The BRST symmetry is generated by a nilpotent charge $Q_B$ $(Q_B^2=0)$ that commutes with the Hamiltonian. The nilpotency of the BRST charge has substantial
consequences. The BRST transformations come from the gauge symmetry, all physical states must be BRST-invariant. Hence, we require a physical state to be annihilated by $Q_B$
\[
Q_{B} |\textrm{phys}\rangle=0~.
\]
However, one can always add a state of the form $Q_{B} | \chi \rangle$ since this state will be annihilated by $Q_B$ because of the nilpotent BRST charge. However, this state is orthogonal to all physical states, and therefore it is a \textbf{null state}.
Thus, two states related by
\[ | \psi'\rangle = |\psi\rangle + Q_B |\chi\rangle \]
have the same inner products with all the physical states and they are thus indistinguishable.
This is the remnant in the gauge-fixed version of the original
gauge symmetry. As a result, the Hilbert
space of physical states is isomorphic to the $Q_B$-cohomology, i.e.
\be \label{BRST-quantization}
\cH^{\textrm{phys}}\cong\frac{\cH^{Q_B\textrm{-closed}}}{\cH^{Q_B\textrm{-exact}}}~.
\ee
This convariant way of determining the physical Hilbert space with ghosts is known as \textbf{BRST quantization}.

We are now ready to apply this formalism to the bosonic string \cite[\S4]{Polchinski}. Combining \eqref{free-scalar} and \eqref{ghost-action}, the action we consider is
\[
S_X+S_{g}= \frac{1}{2\pi}\int d^2z\ \left( \frac1{\alpha'}\partial X^\mu  \ol\partial X_\mu+b\,\overline \partial c + \bar{b}\,\partial\bar{c}\right)~.
\]
This is invariant under the following BRST transformations:
\begin{align}\nonumber
&\delta_B X^\mu = i \epsilon ( c \partial + \bar{c} \bar{\partial} ) X^\mu \,,
\cr
&\delta_B c=  i\epsilon  c \partial c  \qquad \delta_B \bar c=  i\epsilon \bar{c} \bar{\partial} \bar c \,,
\\
&\delta_B b =  i \epsilon ( T^X + T^{g} ) \qquad \delta_B \bar b =  i \epsilon ( \overline T^X + \overline T^{g} )\,, \label{BRST-trans}
\end{align}
where the explicit forms of the energy-momentum tensors can be found in \eqref{TX} and \eqref{Tgh}. Note that we impose \eqref{ghost-holo} here.  This exhibits typical features of the BRST transformation: a bosonic field is transformed into a fermionic field and vice versa, and the ghost field $b$ is transformed into the energy-momentum tensor.
Noether's theorem tells us that there is a classical current associated to the BRST symmetry, and the holomorphic part of the BRST current takes the form (Exercise)
\begin{align}\label{jB}
j_B &= c(z)T^X(z) + \frac12 : c(z)T^{g}(z): +\frac32:\partial^2c(z):\\
&= c(z)T^X(z) + : b(z)c(z)\partial c(z) : +\frac32:\partial^2c(z):~.\nonumber
\end{align}
and the BRST charge is defined by
\[ Q_B = \oint \frac{dz}{2\pi i} ~ j_B \,.\]
It follows from \eqref{BRST-trans} that
\be\label{Q-b}
\{Q_B,b(z) \}=( T^X + T^{g} ) \to \{Q_B,b_m\}=L_m^X+L_m^{g}~.
\ee
Computing $j_Bj_B$ OPE, one can convince oneself that the BRST charge is
nilpotent  $Q_B^2=0$ if and only if  $D=26$ (Exercise).
Therefore, $j_B$ is a primary field with $h=1$, and $Q_B$ is a conserved charge at quantum level only when $D=26$.
Furthermore,
we can express the BRST charge in terms of the $X^{\mu}$ Virasoro
operators and the ghost oscillators as
\be
Q_B = \sum_n c_n (L^X_{-n}-\d_{n,0}) + \sum_{m,n} \frac{m-n}{2} : c_m c_n
b_{-m-n} :  \,,\\
\label{27}\ee
where the normal ordering constant comes from $\{Q_B,b_0\}=L_0^X+L_0^{g}$.
In the case of closed strings, there is the anti-holomorphic part $\bar Q_B$, and the total
BRST charge is $Q_B+\bar Q_B$.

We will find the physical open string states in the BRST context.
According to our previous discussion, the physical states will have
to be annihilated by the BRST charge, and not be of the form
$Q_B|\phantom{\chi}\rangle$.

First, we have to describe our extended Hilbert space which includes
the ghosts.
As far as the $X^{\mu}$ oscillators are concerned, the situation is
the same as in \eqref{bosonic-Hilb}, so we need to consider only the ghost Hilbert space.
Indeed the ghost commutation relations \eqref{ghost-comm} generate a two-state spin system $|\uparrow\rangle, |\downarrow\rangle$ where
\bea\label{ghost-Hilb}
&b_0 | \downarrow \rangle = 0\,, \qquad b_0 |\uparrow\rangle =
|
\downarrow
\rangle\,, \cr
&c_0 | \uparrow \rangle
= 0\,, \qquad c_0 | \downarrow\rangle = |
\uparrow
\rangle \,,\cr
&b_{n>0}|\uparrow\rangle=b_{n>0}|\downarrow\rangle=c_{n>0}|\uparrow\rangle=c_{n>0}|\downarrow\rangle=0~.
\eea


The full Hilbert space will be a tensor product of the two  $|k,\uparrow\rangle, |k,\downarrow\rangle$ by acting creation operators where $|k\rangle=|0;k\rangle$ denotes the vacuum of the matter theory.
From the light-cone quantization, we know that there is only one vacuum called Tachyon. Therefore, we have to pick the ghost vacuum among the two spin states.
For this purpose, we further impose one more condition, namely
\be
b_0|{\rm phys}\rangle =0\,.
\label{25}\ee
This is sometimes called the \textbf{Siegel gauge} \cite[\S3.2]{GSW}.
Under this condition, \eqref{ghost-Hilb} tells us that the correct ghost vacuum is
$|\downarrow\rangle$.
We can now create states from this vacuum by acting with the negative
modes
of the ghosts $b_m,c_n$. Note that $c_0 | \downarrow\rangle = |
\uparrow
\rangle$ does not satisfy the
Siegel condition \eqref{25}.
Also, \eqref{25} yields the condition
\be\label{L0-annihilate} 0=\{Q_B,b_0\}|{\rm phys}\rangle=(L_0^X+L_0^{g})|{\rm phys}\rangle ~.\ee
Therefore, if a physical state  is at level $N$ with momentum $k$, $|{\rm phys}\rangle=|N;k\rangle$, then we have
\be \label{k_N}
k^2 = \frac{1-N}{\alpha'}
\ee
which is consistent with the light-cone gauge.


Now, let us impose the physical condition \eqref{BRST-quantization} for open-string states level by level. To this end, let us explicitly write the open-string mode expansions of the zero modes of the Virasoro generators
\begin{align}\label{open-Virasoro}
L^X_0&=\alpha^{\prime} p^{2}+\sum_{n=1}^{\infty} \alpha_{-n} \cdot \alpha_{n}=\a'p^2+\a_{-1}\cdot\a_1+\cdots~,\cr
L^{g}_0&=-1\sum_{n \in \bZ} n: b_{-n} c_{n}:=-1+b_{-1}c_{1}+c_{-1}b_{1}+\cdots~,
\end{align}
(Compare with \eqref{Virasoro} and \eqref{L-bc}.)
At level zero, there is only the vacuum state and the BRST quantization leads to
$|k,\downarrow\rangle$
\[
 Q_B | k,\downarrow \rangle = ( L^X_0 - 1 ) c_0 | k,\downarrow\rangle= ( L^X_0 - 1 ) | k,\uparrow\rangle=0\,.
\]
and it is not $Q_B$-exact.


At the first level, the possible operators that can act on the vacuum $| k,\downarrow\rangle$ are $\alpha^\mu_{-1}$,
$b_{-1}$ and $c_{-1}$. The most general state of this form is then
\be
|\psi\rangle = ( \zeta \cdot \alpha_{-1} + \beta b_{-1} + \gamma c_{-1} )
| k,\downarrow\rangle \,, \label{n=1}
\ee
which has 28 parameters: a 26-dimensional vector $\zeta_\mu$ and two more constants $\beta$, $\gamma$. First we note that \eqref{k_N} yields the massless
condition  $k^2 =0$. In addition, the BRST condition demands
\[
 0 = Q_B |\psi\rangle = 2 ( (k\cdot \zeta) c_{-1} + \beta
k\cdot \alpha_{-1} ) |k,\downarrow \rangle \,.
\]
This only holds if $k\cdot\zeta=0$ and
$\beta=0$. Therefore the BRST condition removes the unphysical anti-ghost excitations as well as all polarizations that are not orthogonal to the momentum, thereby eliminating two out of the $26+2$ original states. Hence, there are only 26 parameters left.

Next, we have to make sure that this state is not $Q_B$-exact: a general state $|\chi\rangle$
is of the same form as \eqref{n=1}, but with parameters
$\zeta'^{\mu}$, $\b'$ and $\g'$. Thus, the most general $Q_B$-exact state at
this level
with $k^2=0$ will be
\[ Q_B |\chi \rangle = 2 ( k\cdot \zeta' c_{-1} + \beta'
k\cdot\alpha_{-1} ) | k,\downarrow\rangle \,. \]
This means that the $c_{-1}$ part in \eqref{n=1} is BRST-exact and that
the polarization has the equivalence relation $\zeta_\mu \sim
\zeta_\mu + 2 \beta' k_\mu $. This leaves us with the 24 physical
degrees of freedom we expect for a massless vector particle in 26 dimensions. In sum, the physical state at level one is
 \[
 \{|\zeta;k\rangle ~; \quad k\cdot\zeta=0\}/~ \zeta_\mu \sim
\zeta_\mu + 2 \beta' k_\mu ~.
 \]

The same procedure can be followed for the higher levels of open string states. It can be proved that $\cH_{\textrm{light-cone}}$ is isomorphic to the $Q_B$-cohomology, and the inner product is positive-definite. In the
case
of the closed string, we have to use $Q_B+\bar Q_B$ for the BRST quantization.
\end{document}
