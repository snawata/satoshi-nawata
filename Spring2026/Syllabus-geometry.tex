

\documentclass[a4paper,11pt]{article}
\usepackage[top=3cm, bottom=3cm, left=2cm, right=2cm]{geometry}
\usepackage{CJK}

\usepackage{hyperref} % Use the Charter font for the document text
%\usepackage[UTF8]{ctex}
\usepackage{fullpage}

\usepackage{graphicx}


\begin{document}\thispagestyle{empty}
\begin{CJK}{UTF8}{gbsn}

\centerline{\Large \bf Syllabus}

\begin{description}
\item{\bf Course name:} Differential geometry and topology in physics (PHYS130115.01, PHYS630056)
\item{\bf Instructor:} Satoshi Nawata, Physics S422, Jiangwan \href{mailto:snawata@fudan.edu.cn}{snawata@fudan.edu.cn}
\item{\bf Teaching Assistant:} 
\item{\bf Hours:}  9:55 -- 12:30 Monday
\item{\bf Place:} TBA
\item{\bf Office hour:} Whenever, but email me beforehand.
\item{\bf Prerequisites:} Linear algebra, Calculus, Classical mechanics
\item{\bf About the course:}


In this course, I will explain basics of differential geometry and topology and their applications to physics. The theory of differential forms is indispensable for the description of Maxwell's electro-magnetic theory. Einstein's general relativity has been established based on Riemannian geometry. Non-perturbative quantum effects like Aharonov-Bohm effect, Dirac monopoles, Berry phase, quantum Hall effects, instantons, and anomaly have deep connections to vector bundles and characteristic classes. Therefore, this course introduces to basic concepts of differential geometry and topology. In addition, I will explain many aspects of physics from the viewpoint of geometry and topology. 

I have already taught this course four times. Basically, I will follow the lecture notes I have already prepared, but I will include new examples and more connnections to physics along the way. 

\item{\bf Main content:}
\begin{itemize}
\item differential manifolds, Riemannian manifold, complex manifolds
\item theory of differential forms, harmonic forms
\item vector bundles, connections, curvature, characteristic classes
\item homology, cohomology, fundamental groups, homotopy groups
\item relations to physics such as electromagnetism, general relativity, quantum physics,  condensed matter physics
\end{itemize}

\item{\bf Main textbook:}

My own lecture notes.

\item{\bf Supplementary textbooks:}

物理学家用微分几何(第2版) 侯伯元

Nash Sen, Topology and Geometry for Physicists

Shigeyuki Morita, Geometry of differential forms

Mikio Nakahara, Geometry, Topology and Physics





\item{\bf Grade evaluation:} Grade will be determined based on 6 sets of homework (50\%) given every other week and the final test (50\%).


\end{description}


\end{CJK}


\end{document}
