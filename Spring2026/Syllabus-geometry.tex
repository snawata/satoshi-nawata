

\documentclass[a4paper,11pt]{article}
\usepackage[top=3cm, bottom=3cm, left=2cm, right=2cm]{geometry}
\usepackage{CJK}

\usepackage{hyperref} % Use the Charter font for the document text
%\usepackage[UTF8]{ctex}
\usepackage{fullpage}

\usepackage{graphicx}


\begin{document}\thispagestyle{empty}
\begin{CJK}{UTF8}{gbsn}

\centerline{\Large \bf Syllabus}

\begin{description}
\item{\bf Course name:} Differential geometry and topology in physics (PHYS130115.01, PHYS630056)
\item{\bf Instructor:} Satoshi Nawata, Physics S422, Jiangwan \href{mailto:snawata@fudan.edu.cn}{snawata@fudan.edu.cn}
\item{\bf Teaching Assistant:} 
\item{\bf Hours:}  9:55 -- 12:30 Monday
\item{\bf Place:} TBA
\item{\bf Office hour:} Whenever, but email me beforehand.
\item{\bf Prerequisites:} Linear algebra, Calculus, Classical mechanics
\item{\bf About the course:}

\item{\bf About the course:}

This course introduces the basic ideas of differential geometry and topology, with an emphasis on their roles in modern physics. Differential forms provide the natural language for Maxwell’s theory of electromagnetism, while Einstein’s general relativity is founded on Riemannian geometry. Moreover, many genuinely quantum and non-perturbative phenomena—such as the Aharonov–Bohm effect, Dirac monopoles, the Berry phase, the quantum Hall effect, instantons, and anomalies—are deeply intertwined with the theory of vector bundles and characteristic classes.

Motivated by these examples, the course develops the fundamental concepts of differential geometry and topology in a way that is directly connected to physical intuition. Throughout the lectures, I will repeatedly return to concrete physical systems and reinterpret them from a geometric and topological viewpoint.

I have taught this course four times previously. This year, I will mainly follow the lecture notes I have already prepared, while enriching them with new examples and making even more explicit connections to physics along the way.

\item{\bf Main content:}
\begin{itemize}
\item differential manifolds, Riemannian manifold, complex manifolds
\item theory of differential forms, harmonic forms
\item vector bundles, connections, curvature, characteristic classes
\item homology, cohomology, fundamental groups, homotopy groups
\item relations to physics such as electromagnetism, general relativity, quantum physics,  condensed matter physics
\end{itemize}

\item{\bf Main textbook:}

My own \href{https://www.overleaf.com/read/ctbckvdrgctf}{lecture notes}.

\item{\bf Supplementary textbooks:}

物理学家用微分几何(第2版) 侯伯元

Nash Sen, Topology and Geometry for Physicists

Shigeyuki Morita, Geometry of differential forms

Mikio Nakahara, Geometry, Topology and Physics





\item{\bf Grade evaluation:} Grade will be determined based on 6 sets of homework (50\%) given every other week and the final test (50\%).


\end{description}


\end{CJK}


\end{document}
