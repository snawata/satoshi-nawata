\documentclass[12pt,a4paper]{article}



\usepackage{macros}


\begin{document}\thispagestyle{empty}

\centerline{\Large \bf Homework 2: Due at class on March 12}



\vspace{.5cm}
\noindent \textbf{1}.  
We construct vector fields on  an $n$-sphere  $S^n \subset \mathbb{R}^{n+1}$ in Example 2.11 of the lecture note.
$$
Y=\sum Y^i \frac{\partial}{\partial x^i}
$$
where
$$
Y^i=\left\{\begin{array}{ll}
\left(-x^1, x^0,-x^3, x^2, \cdots,-x^{2 k+1}, x^{2 k}\right) & n=2 k+1 \\
\left(-x^1, x^0,-x^3, x^2, \cdots,-x^{2 k+1}, x^{2 k}, 0\right) & n=2 k+2
\end{array} .\right.
$$
For $n=1,2,3$, write explicitly the flows generated by this vector field.



%\vspace{.5cm}

%
%\noindent \textbf{2}. Let $(x,y)$ be the Cartesian coordinate of $\bR^2$ and $(r,\theta)$ be the polar coordinate of $\bR^2$. Write a vector field $X$ in terms of the Cartesian coordinate that generate a flow $\varphi_t:\bR^2\to\bR^2$
%$$
%\left(\begin{array}{c}
%x\\y
%\end{array}\right) \to\left(\begin{array}{cc}
%\cos t& -\sin t\\
%\sin t&\cos t
%\end{array}\right) \left(\begin{array}{c}
%x\\y
%\end{array}\right)~.
%$$
%This is the rotation in $\bR^2$. In addition, draw the schematic picture of the vector field $X$.



\vspace{.5cm}

\noindent \textbf{2}.  Write down vector fields that generate the rotation along $x$-, $y$-, $z$-axis in $\bR^3$. Find the commutation relations of these vector fields. Compare the theory of angular momenta in quantum mechanics.



%
% \vspace{.5cm}
% \noindent \textbf{3}.  For a positive integer $d$, we define a map $f^{(d)}:\bC\to \bC; ~z\mapsto z^d$. Let $z=x+iy$ \((x, y \in \mathbb{R}),\) and consider \(f^{(d)}\) as a function of \(x\) and \(y .\) Compute the Jacobian matrix of \(f^{(d)}\).



%
% \vspace{.5cm}
%
% \noindent \textbf{3}.   Are there zeros of the vector fields in Example 3.11 of the lecture note?
%

%
% \vspace{.5cm}
%
% \noindent \textbf{5}.    Construct a smooth vector field on $S^2$ which vanishes only at one point explicitly in terms of local coordinates.
%
%

%
% \vspace{.5cm}
%
% \noindent \textbf{6}.
% Show that the Lie bracket satsfies the Jacobi identity.
% Show that, for $X_1,X_2\in \mathfrak{X}(M)$ and $f\in C^\infty(M)$,
% $$
% [fX_1,X_2]=f[X_1,X_2]-X_2(f) X_1~,\qquad [X_1,fX_2]=f[X_1,X_2]+X_1(f) X_2~.
% $$
%



\vspace{.5cm}
\noindent \textbf{3}. Let $e$ be the identity element of $SO(3)$. Show that the tangent space $T_e SO(3)$ at $e$ is spanned by tangent vectors of curves in $SO(3)$
$$
\exp(t J_i)=1+tJ_i+\frac12 (tJ_i)^2 +\cdots
$$
at $t=0$  where $J_i$ $(i=x,y,z)$ are defined by
$$
J_x=\left(
\begin{array}{ccc}
0&0&0\\
0&0&-1\\
0&1&0\\
\end{array}
\right) \qquad
J_x=\left(
\begin{array}{ccc}
0&0&1\\
0&0&0\\
-1&0&0\\
\end{array}
\right) \qquad
J_x=\left(
\begin{array}{ccc}
0&-1&0\\
1&0&0\\
0&0&0\\
\end{array}
\right) ~.
$$
Let us define the commutator by $[X, Y ] = XY - Y X$. Then, show that
$$
[J_x,J_y]=J_z~, \qquad [J_y,J_z]=J_x~, \qquad [J_z,J_x]=J_y~.
$$



\vspace{.5cm}
\noindent \textbf{4}. Show that the tangent space $T_e SU(2)$ is spanned by $i\sigma_x$, $i\sigma_y$ and $i\sigma_z$ (the Pauli matrices by $i$).


\vspace{.5cm}

\noindent \textbf{5}.
Show that $\bR P^n$ is non-orientable for even $n$. In addition, construct an example of unorientable manifolds except the M\"obius strip and even-dimensional real projective space.


\end{document}
