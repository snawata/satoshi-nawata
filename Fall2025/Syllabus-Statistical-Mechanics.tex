\documentclass[a4paper,11pt]{article}
\usepackage[top=3cm, bottom=3cm, left=2cm, right=2cm]{geometry}
\usepackage{CJK}
\usepackage{hyperref}
\usepackage{fullpage}
\usepackage{graphicx}
\usepackage{enumitem}

\begin{document}\thispagestyle{empty}
\begin{CJK}{UTF8}{gbsn}

\centerline{\Large \bf Syllabus}

\begin{description}
\item{\bf Course name:} Statistical Mechanics II (PHYS40011)
\item{\bf Instructor:} Satoshi Nawata, Physics S422, Jiangwan \href{mailto:snawata@fudan.edu.cn}{snawata@fudan.edu.cn}
\item{\bf Teaching Assistant:} Qituan Zhang  \href{mailto:24210190037@m.fudan.edu.cn}{24210190037@m.fudan.edu.cn},  \\ 
. \hspace{7em}  Shutong Zhuang \href{mailto:22110190097@m.fudan.edu.cn}{22110190097@m.fudan.edu.cn}
\item{\bf Hours:} Monday 13:30-16:10
\item{\bf Place:} H3108
\item{\bf Office hour:} Flexible, but please text me in advance.
\item{\bf Prerequisites:} Thermodynamics, Statistical Mechanics I, Quantum Mechanics

\item{\bf About the course:}

This course offers a comprehensive introduction to quantum many-body systems, serving as a bridge between statistical mechanics and modern condensed matter physics.  After reviewing the basics of Statistical Mechanics I, we delve into advanced concepts such as the density matrix formalism and Landau's Fermi liquid theory, before focusing on two central topics:

1. Superconductivity (BCS Theory) – We develop the microscopic theory of superconductivity, starting from Cooper pairing and mean-field approximation to derive the Bogoliubov-de Gennes equations and gap equation. Key phenomena such as the Meissner effect, flux quantization, and thermodynamic properties (specific heat, quasiparticle excitations) are analyzed in depth.  

2. Quantum Spin Systems – We explore low-dimensional magnetism, particularly 1D antiferromagnetic spin chains and the Haldane conjecture. Topics include exact solutions via the Bethe ansatz, valence-bond states, the Lieb-Schultz-Mattis theorem, and the profound distinction between integer and half-odd-integer spin chains. 


\item{\bf Course Schedule:}

\begin{enumerate}[label=Week \arabic*:, leftmargin=*]
    \item Review of Thermodynamics and Basics of Equilibrium Statistical Mechanics
    \item Quantum Mechanics and Second Quantization
    \item Statistical Mechanics of Ideal Gases
    \item Density Matrices
    \item Hartree--Fock Equations and Landau's Fermi-Liquid Theory
    \item Attractive Interaction and Bound States
    \item Mean-Field Equations of Superconductivity
    \item BCS Theory
    \item Superfluidity, Meissner Effect, and Flux Quantization
    \item Introduction to Quantum Spin System
    \item 1D Antiferromagnetic Spin Chains
    \item Theoretical Understanding of the Haldane Conjecture
    \item Valence-Bond States
    \item Hidden Structures in Haldane Gap Systems
\end{enumerate}


\item{\bf Lecture Notes:}
To help students concentrate fully during class and minimize the need for note-taking, I will prepare detailed lecture notes and provide recording videos for all sessions. 


The part of superconductivity will primarily follow the treatment in \cite{kita2015}, while the part of quantum spin systems  will be based on \cite{tasaki2020}. 


\item{\bf Grading:}
Grades will be based on a combination of homework assignments and a final exam:
\begin{itemize}
\item Homework Assignments: 50\% (5 sets, given every other week)
\item Final Exam: 50\%
\end{itemize}


\begin{thebibliography}{9}

\bibitem{pathria2011} 
Pathria, R. K., \& Beale, P. D. (2011). 
\textit{Statistical Mechanics} (3rd ed.). 
Elsevier. 
(Note: The 2nd edition was published in 1996; 3rd edition is the most current) 
ISBN: 978-0123821881.

\bibitem{kardar2007}
Kardar, M. (2007).
\textit{Statistical Physics of Particles}.
Cambridge University Press.
ISBN: 978-0521873420.


\bibitem{coleman2015}
Coleman, P. (2015).  
\textit{Introduction to Many-Body Physics}.  
Cambridge University Press.  
ISBN: 978-0521864886.


\bibitem{kita2015}
Kita, T. (2015).
\textit{Statistical Mechanics of Superconductivity}.
Springer.
DOI: \url{https://doi.org/10.1007/978-4-431-55405-9}
ISBN: 978-4431554042 (Hardcover), 978-4431554059 (eBook).

\bibitem{tasaki2020}
Tasaki, H. (2020).
\textit{Physics and Mathematics of Quantum Many-Body Systems}.
Springer.
Series: Graduate Texts in Physics
ISBN: 978-3-030-41264-8 (Hardcover)
eISBN: 978-3-030-41265-5 (eBook)
DOI: \url{https://doi.org/10.1007/978-3-030-41265-5}.

\bibitem{sachdev2011}
Sachdev, S. (2011).
\textit{Quantum Phase Transitions} (2nd ed.).
Cambridge University Press.
ISBN: 978-0-521-51468-2 (Hardback)
ISBN: 978-0-521-73264-8 (Paperback)
DOI: \url{https://doi.org/10.1017/CBO9780511973765}.

\end{thebibliography}


%\item{\bf Course Goals:}
%By the end of the course, students will:
%\begin{itemize}
%\item Gain an intuitive understanding of the relationship between physical systems and their underlying geometrical and topological structures.
%\item Develop a working knowledge of how mathematical tools like differential geometry and topology are applied in classical mechanics, electrodynamics, and general relativity.
%\item Be able to apply these concepts to solve problems in physics, ranging from the behavior of classical systems to the structure of spacetime.
%\end{itemize}
%
%\item{\bf Office Hours:}
%I encourage students to contact me via email to schedule office hours for questions or discussions on course content or homework.

\end{description}

\end{CJK}

\end{document}
