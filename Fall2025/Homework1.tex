\documentclass[12pt,a4paper]{article}
\pdfoutput=1

\usepackage{macros}
%%% Yokoyama def %%%
\providecommand{\vcentcolon}{\mathrel{\mathop{:}}}
%%%%%%%%%%%%%%%%%%%%
\usepackage{graphicx}
\usepackage{exercise}


\begin{document}\thispagestyle{empty}

\centerline{\Large \bf Homework 1: Due on Sep 28}
\bigskip
\textbf{Caution:} 
When solving homework problems, it is important to show your derivation at each step. Nowadays, many online tools make it easy to find answers, but the primary goal of these assignments is to deepen your understanding through hands-on problem-solving. By working through the calculations yourself, you engage more deeply with the material, making the learning process more meaningful, rather than simply copying answers from external sources.

\subsection*{Thermodynamics}

Suppose that there are $n$ moles of a gas that obey van der Waals' equation,
for which the heat capacity $C_V$ is known experimentally to be temperature
independent.
\begin{enumerate}
\item Write expressions for the infinitesimal increments in entropy $dS$ and
internal energy $dU$ based on 
\begin{equation}
\begin{gathered}
\mathrm{d} S=\frac{C_V}{T} \mathrm{~d} T+\left(\frac{\partial P}{\partial T}\right)_V \mathrm{~d} V, \\
\mathrm{~d} U=C_V \mathrm{~d} T+\left[T\left(\frac{\partial P}{\partial T}\right)_V-P\right] \mathrm{d} V .
\end{gathered}
\end{equation}
\item Show that $C_V$ does not depend on volume $V$.
\item Obtain expressions for the entropy $S$ and the internal energy $U$.
\item Show that the quantity $T\,(V-nb)^{nR/C_V}$ does not change in reversible
adiabatic processes.
\item Calculate the temperature change during an adiabatic free expansion from volume $V_1$ to volume $V_2$.
\end{enumerate}



\subsection*{Bosonic Bogoliubov diagonalization on a 1D lattice}
Consider a one–dimensional bosonic chain with $N$ sites, lattice spacing $a$, and
periodic boundary conditions. Let $a_j$ and $a_j^\dagger$ be bosonic operators at
site $j$ satisfying $[a_i,a_j^\dagger]=\delta_{ij}$, and let the $j$th site be at
$R_j=ja$. The Hamiltonian is
\begin{equation}
H=\sum_{j}\Bigl\{J_1\bigl(a_{j+1}^\dagger a_j+\text{H.c.}\bigr)
      +J_2\bigl(a_{j+1}^\dagger a_j^\dagger+\text{H.c.}\bigr)\Bigr\},
\end{equation}
where ``H.c.'' denotes the Hermitian conjugate and $J_{1,2}\in\mathbb R$.

\begin{enumerate}
\item \textbf{Fourier transform.}
Adopt the momentum representation
\begin{equation}
a_j=\frac{1}{\sqrt{N}}\sum_{q} e^{iqR_j}\,a_q, \qquad
q=\frac{2\pi}{Na}\,n,\; n=0,1,\dots,N-1,
\end{equation}
and we consider them in the first Brillouin zone in what follows:
\be 
q \in\left(-\frac{\pi}{a}, \frac{\pi}{a}\right]
\ee
Rewrite $H$ in the quadratic (BdG) form in the momentum space
\bea 
H&=\sum_{q}A_q\,a_q^\dagger a_q
+\sum_{q>0} B_q\bigl(a_q^\dagger a_{-q}^\dagger+a_{-q}a_{q}\bigr)\cr 
&=\sum_{q>0}A_q \left(a_q^{\dagger} a_q+a_{-q}^{\dagger} a_{-q}\right)+B_q\bigl(a_q^\dagger a_{-q}^\dagger+a_{-q}a_{q}\bigr)
\eea
Determine $A_q$ and $B_q$ explicitly in terms of $J_1,J_2$, and $q$.

\item \textbf{Bogoliubov transformation.}
Diagonalize $H$ via the canonical (bosonic) transformation
\begin{equation}
b_q=u_q\,a_q+v_q\,e^{-i\phi_q}\,a_{-q}^\dagger,
\qquad u_q^{\,2}-v_q^{\,2}=1,
\end{equation}
choosing the phase $\phi_q$ so that the anomalous terms $b_q b_{-q}$ and
$b_q^\dagger b_{-q}^\dagger$ vanish. Show that your choice leads to a real,
positive quasiparticle spectrum and express $u_q,v_q$ through a single
parameter $\theta_q$ (e.g.\ $u_q=\cosh\theta_q$, $v_q=\sinh\theta_q$) with a
condition for $\theta_q$ such as $\tanh(2\theta_q)=B_q/A_q$.

\item \textbf{Dispersion and stability.}
Obtain the diagonal form
\be 
H=\sum_{q>0}\Bigl[E_q\left(b_q^{\dagger} b_q+b_{-q}^{\dagger} b_{-q}\right) + E_0\Bigr],
\ee 
and derive the dispersion $E_q$ and the vacuum shift $E_0$ in closed form. State the \emph{stability
condition} for the quadratic bosonic Hamiltonian (i.e.\ the requirement on
$J_1,J_2,$ and $q$ ensuring $E_q\in\mathbb R_{\ge0}$ for all $q$). What happens when $J_1=J_2$?




% \item[\emph{Hint.}]
% With real $J_{1,2}$, a convenient choice is $\phi_q=\arg B_q$ (for this
% model, $\phi_q=qa$), so that $B_q e^{i\phi_q}=|B_q|$ is real and positive.
\end{enumerate}



\subsection*{Two Fermions (Bogoliubov)}
Let $a_1,a_2$ be two fermionic modes with
\[
\{a_i,a_j^\dagger\}=\delta_{ij},\qquad \{a_i,a_j\}=\{a_i^\dagger,a_j^\dagger\}=0 .
\]
Consider the quadratic Hamiltonian
\begin{equation}
H=\epsilon\bigl(a_1^\dagger a_1-a_2 a_2^\dagger\bigr)
+\Delta\bigl(a_1^\dagger a_2^\dagger+\text{H.c.}\bigr),
\label{eq:H_original}
\end{equation}
with real parameters $\epsilon,\Delta$.

\begin{enumerate}
\item \textbf{Canonical transformation.}
Show that the Bogoliubov transformation
\begin{align}
c_1 &= u\,a_1+v\,a_2^\dagger, \\
c_2 &= u\,a_2 - v\,a_1^\dagger,
\end{align}
with real $u,v$ preserves the canonical anticommutation relations
\(\{c_i,c_j^\dagger\}=\delta_{ij}\) and \(\{c_i,c_j\}=0\) \emph{iff}
\[
u^2+v^2=1 .
\]
% (\emph{Hint:} Write $u=\cos\theta$, $v=\sin\theta$.)

\item \textbf{Diagonalization.}
Using the result of (1), show that for a suitable choice of $\theta$
the Hamiltonian \eqref{eq:H_original} can be written as
\begin{equation}
H=E\bigl(c_1^\dagger c_1+c_2^\dagger c_2-1\bigr),
\label{eq:H_diag}
\end{equation}
and determine $E$ and $u,v$.

\item \textbf{Ground-state energy.}
What is the ground-state energy $E_0$ of \eqref{eq:H_diag}?  
(State it in terms of $\epsilon$ and $\Delta$.)

\item \textbf{Ground state in the \(\boldsymbol{a}\)-basis.}
Let $\ket{0_a}$ be the vacuum annihilated by $a_{1,2}$:
$a_{1,2}\ket{0_a}=0$.  Let $\ket{\mathrm{GS}}$ be the vacuum annihilated by $c_{1,2}$: $c_{1,2}\ket{\mathrm{GS}}=0$. Find the relation between  $\ket{\mathrm{GS}}$ and $\ket{0}$.  Verify your state minimizes \eqref{eq:H_original}.

% \item[\emph{Optional.}] 
% Allow complex $u,v$ and show that the most general
% canonical transform requires $|u|^2+|v|^2=1$ and a relative phase fixed by the
% phase of $\Delta$.
\end{enumerate}


% \subsection*{Second quantization}
% Consider a system of fermions or bosons created by the field $\psi^\dagger(\mathbf r)$
% interacting under the potential
% \begin{equation}
% V(r)=
% \begin{cases}
% U, & r<R,\\
% 0, & r>R .
% \end{cases}
% \tag{3.154}
% \end{equation}

% \begin{enumerate}
% \item Write the interaction in second-quantized form.
% \item Switch to the momentum basis, where
% \begin{equation}
% \psi(\mathbf r) = \int \!\frac{d^3 k}{(2\pi)^3}\, c_{\mathbf k}\, e^{i\mathbf k\cdot \mathbf r}.
% \end{equation}
% Verify that
% \begin{equation}
% [c_{\mathbf k}, c^{\dagger}_{\mathbf k'}]_{\pm}
% = (2\pi)^3 \delta^{(3)}(\mathbf k-\mathbf k'),
% \end{equation}
% and write the interaction in this new basis. Sketch the form of the interaction in momentum space.
% \end{enumerate}




\end{document}
