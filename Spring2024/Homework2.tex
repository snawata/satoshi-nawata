\documentclass[12pt,a4paper]{article}
%\usepackage{hyperref} % Use the Charter font for the document text
%\usepackage[UTF8]{ctex}
\usepackage{macros}


%%%%%%%%%%%%%%%%%%%%


\begin{document}\thispagestyle{empty}

\centerline{\Large \bf Homework 2: Due on Mar 24}


%
% \section{Prob. 2}
% \begin{enumerate}
%  \item Start from Green's theorem in Cartesian coordinate and show that the complex version is given by the form given in the lecture.
%  \item Confirm that $\partial\ol\partial \log |z|^2 = 2\pi \delta^2(z)$.
%  \item Show that $\oint \frac{d\ol z}{2\pi i} \frac{1}{\ol z} =-1$.
% \end{enumerate}
%
%
% \section{}
% You do not have to consider space-time indices in this problem.
% \begin{enumerate}
%  \item Express $\nord{X_1X_2X_3X_4}$ in terms of normal products.
%  \item Using the previous result express $X_1X_2X_3X_4$ in terms of normal-ordered products, and confirm that it can be expressed in the form of
%        \begin{align}
%          f[X] = \exp \left[ \frac{1}{2} \int d^2z d^2w\ G(z,w) \frac{\delta}{\delta X(z)} \frac{\delta}{\delta X(w)} \right]
%  \nord{f[X]} \ .  \nonumber
%        \end{align}
% % \item Express $X_1 \partial X_2 \partial^2 X_3$ in terms of normal-ordered products.
% \end{enumerate}

%
% \section{}
% Let us define $T(z) = -\frac{1}{\alpha'}\nord{\partial X^\mu \partial X_\mu}$,
% and $j(z) = a_\mu \partial X^\mu$.
% Extract divergent terms from $j(z)X^\nu(w)$, and $T(z)T(w)$.

% %
% \subsection*{Prob. 5}
% \begin{enumerate}
% \item Consider the example given in the lecture $\delta X = \epsilon g(z)$.
%       Derive $\delta f[x]$ for $f[x]=\partial X(w)$ and $f[x]= \ e^{ik\cdot X(w)}$ using the Ward identity.
%
% \item Derive the Noether currents for $\delta X = \epsilon \partial X +\ol\epsilon \ol\partial X$.
%
%
%\section{Vertex operator and OPE}
%Show that $:e^{ikX}:$ is a primary field of weight $h = \bar h = \a'k^2/4$ in the free scalar theory. In addition, show that $\partial^n X$ ($n\ge 2$) is not a primary field.

\section{Virasoro algebra}
From the OPE of the stress-energy tensor, derive the Virasoro algebra:
$$
T(z)\,T(w) = \frac{c/2}{(z-w)^4} + \frac{2T(w)}{(z-w)^2} +
\frac{\partial T(w)}{z-w} + \ldots
$$
$$\longrightarrow\quad [L_m,L_n] = (m-n)L_{m+n} + \frac{c}{12}m(m^2-1)\delta_{m+n,0}~.$$

\section{Witt algebra}
A general infinitesimal holomorphic map can be expressed as
$$z'=z-\e(z)=z-\sum_{n\in\bZ}\e_n z^{n+1} ~,$$
with the infinitesimal parameters $\e_n$, and therefore one can define generators of the transformation by $\ell_n = -z^{n+1}\frac\partial{\partial z}$.
Show that they satisfy the Witt algebra
$$ [\ell_m,\ell_n] = (m-n)\ell_{m+n}~,$$
so that the Virasoro algebra is the central extension of the Witt algebra.



\section{Linear fractional transformations}

Let us consider the Riemann sphere $S^2  = \bC \cup \{\infty\}$. The action of $SL(2, \bC)$ defined by
$$z  \mapsto w = \frac{az + b}{ cz + d}~,\qquad \begin{pmatrix}a&b\\c&d\end{pmatrix}\in SL(2, \bC)~,$$
maps the Riemann sphere onto itself. These transformations are called linear fractional  transformations.

\vspace{.3cm}
\noindent $\bullet$ Given three points $z_1,z_2,z_3$, find a linear fractional transformation which maps the points
to $0, 1, \infty$.

\vspace{.3cm}
\noindent $\bullet$ Given four points $z_1,z_2,z_3,z_4$, their \textbf{cross ratio} is defined by
$$
[z_1,z_2,z_3,z_4]= {\frac  {(z_{1}-z_{3})(z_{2}-z_{4})}{(z_{2}-z_{3})(z_{1}-z_{4})}}~.
$$
Show that the cross ratio is preserved by any linear fractional  transformation
$$
[z_1,z_2,z_3,z_4]=[w_1,w_2,w_3,w_4] ~.
$$



\section{2-point and 3-point function of primary fields}
\subsection{2-point function}

Let us determine the form of the 2-point function of chiral primary operators $\phi_i(z_i)$ with weight $h_i$ ($i=1,2$).  The 2-point function is invariant under the translation $z\to z+a$ of the coordinate so that it is a function $g(z_1-z_2)$ of their relative coordinate $z_1-z_2$.

Using the property of chiral primary fields under the scaling $z\to \lambda z$, show that the function is of the form
$$
g(z_1-z_2) =\frac{d_{12}}{(z_1-z_2)^{h_1+h_2}}~.
$$
Furthermore, show that  $h_1$ must be equal to $h_2$ by using the property under the transformations $z\to -1/z$.


\subsection{3-point function}

The translation invariance tells us that the 3-point function is also a function $g(z_{12},z_{23},z_{31})$ where $z_{ij}=z_i-z_j$. Applying the same argument above, derive the form of the 3-point function
$$
\langle \phi_1(z_1)\phi_2(z_2)\phi_2(z_3)\rangle =\frac{C_{123}}{(z_{12})^{h_1 +h_2 -h_3}(z_{23})^{h_2 +h_3 -h_1}(z_{31})^{h_3 +h_1 -h_2}}~.
$$






\section{Derivation}
The derive (4.5) from (4.4) in the latest lecture note. Namely, start with 
$$
\delta_\epsilon T(z)=\delta_\epsilon T_{z z}-2 k\left(\partial \partial \delta_\epsilon \omega-2 \partial \omega \partial \delta_\epsilon \omega\right) .
$$
Using (finite) transformations,
$$
\begin{aligned}
& z \rightarrow \widetilde{z}=z-\epsilon(z), \\
& T_{z z} \rightarrow \widetilde{T}_{\tilde{z} \tilde{z}}=\left(\partial_z \widetilde{z}\right)^{-2} T_{z z}, \\
& \omega(z) \rightarrow \widetilde{\omega}(\tilde{z})=\omega(z)-\frac{1}{2} \log \left|\partial_z \tilde{z}\right|^2,
\end{aligned}
$$
Show
$$
\delta_\epsilon T(z)=\epsilon(z) \partial T(z)+2 \partial \epsilon(z) T(z)-k \partial^3 \epsilon(z) .
$$


\end{document}
