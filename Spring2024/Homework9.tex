\documentclass[12pt,a4paper]{article}
%\usepackage{hyperref} % Use the Charter font for the document text
%\usepackage[UTF8]{ctex}
\usepackage{macros}

\begin{document}\thispagestyle{empty}

\centerline{\Large \bf Homework 9: Due at class on May 12}




% \section{1 Cylinder partition function and torus partition function}

% The torus partition function of bosonic string is given by
% \begin{align}
%  A_{0,T} = \frac{iV_{26}}{(2\pi l_s)^{26}} \int_F \frac{d^2\tau}{2\tau_2^2}
%  \tau_2^{-12} |\eta(\tau)|^{-48} \ ,
% \end{align}
% where $F$ is the fundamental region of the torus moduli space.
% \begin{itemize}
%  \item Consider low energy region in $F$ and show that $\tau_1$ integration ($-\frac{1}{2} \le \tau_1 \le \frac{1}{2}$) leads level matching condition.
%  \item Again, consider low energy region.
%        Show that
%        it can be identified with the cylinder partition function up to constant.
%  \item Explain why an open string short 1-loop corresponds to
%        a closed string long propergation.
% \end{itemize}


\section{ Contribution from symplectic representation}


Consider
\begin{align}
 P \Lambda^T P = \pm \Lambda
\end{align}
for $P = i
\begin{pmatrix}
 0 & -\mathbb{1}_{k\times k} \\
 \mathbb{1}_{k\times k} & 0
\end{pmatrix}$
where $n=2k$.
Derive the dimension(degrees of freedom) of $\Lambda$ for $+$ and $-$, respectively,
and show that $\tr[\Omega_\Lambda] = -n$.


\section{ Orientation flip in superstring}


Consider the orientation flip operator $\Omega$ for world-sheet fermions of closed string.
\begin{itemize}
 \item Define the action of the orientation operator on the fermions
       $\psi^\mu(t,\sigma)$, $\overline \psi^\mu(t,\sigma)$ in NS- and R-sector properly,
       and derive the action on their modes.
       (Hint: the orientation flip is define for $c$ as $\Omega: c(t,\sigma) \to -\ol c(t,2\pi-\sigma)$.
       The minus sign in from of $\ol c$ is coming from relative phase of overall coefficient of the mode expansion in cylinder:
       $c = i \sum c_n e^{in(it+\sigma)}$ and $\ol c = -i \sum \ol c_n e^{in(it+\sigma)}$.
 \item Consider IIB RR-fields as in Prob. 2 of Homework 8, and show that only the 2-form
       RR-field survives under the $\Omega$ projection $\frac{1+\Omega}{2}$.
       (Hint: you have to consider field strength of $n$-form RR-fields because RR-fields themselves
       are not physical degrees of freedom.
       Note that RR-fields are real valued object so its complex conjugate is itself.
       Also note that the complex conjugate of fermions gives minus sign
       $(\psi^\dagger \chi)^\dagger =-\chi^\dagger\psi$ due to their statistics.
       You can assume that gamma matrices are invariant under $\Omega$.
       This is because zero modes are identical in L and R so $\Gamma$ is actually sum of L and R, i.e. $\Gamma = \Gamma^L +\Gamma^R$,
       which is manifestly invariant under $\Omega$.)
\end{itemize}


\section{SO(32) in Type I}

In the lecture, we evaluate the cylinder, Klein bottle and M\"obius strip amplitude in unoriented bosonic string. Evaluate the cylinder, Klein bottle and M\"obius strip amplitude in the RR-sector of superstring theory. Show the divergence is absent only when the gauge group is SO(32).




\end{document}
