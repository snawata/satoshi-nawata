\documentclass[12pt,a4paper]{article}
%\usepackage{hyperref} % Use the Charter font for the document text
% %\usepackage[UTF8]{ctex}
% \usepackage{jheppub}
\usepackage{macros}
%%% Yokoyama def %%%
\providecommand{\vcentcolon}{\mathrel{\mathop{:}}}
%%%%%%%%%%%%%%%%%%%%
\usepackage{graphicx}


\begin{document}\thispagestyle{empty}

\centerline{\Large \bf Homework 1: Due at class on Mar 17}

%
%\section{Canonical quantization of free scalar}
%
%Using the mode expansions in the lecture note, derive from the canonical commutation relation for  $X^\mu$ and $\Pi_\mu$
%%
%\begin{align}
%&[X^\mu(\sigma,\tau),\Pi_\nu(\sigma^\prime,\tau)]=i\delta(\sigma-\sigma^\prime)
%\,\delta^\mu_{\ \nu} &\ , \cr
%&[X^\mu(\sigma,\tau),X^\nu(\sigma^\prime,\tau)] = [\Pi_\mu
%(\sigma,\tau),\Pi_\nu(\sigma^\prime,\tau)] = 0&\ .\nonumber\end{align}
%commutation relations for the Fourier modes $x^\mu$, $p^\mu$, ${\alpha}_n^\mu$ and $\wt {\alpha}_n^\mu$
%%
%\be [x^\mu,p_\nu]=i\delta^\mu_{\ \nu} \ \ \ {\rm and}\ \ \
%{[}{\alpha}_n^\mu,\alpha_m^\nu{]}=
%{[}\wt {\alpha}_n^\mu,\widetilde{\alpha}_m^\nu{]}
%=n\, \eta^{\mu\nu}\delta_{n+m,\,0}\ ,\nonumber\ee
%
%
%
%\section{}
%
%Problem 2.3 in Becker-Becker-Schwarz.





\section{Open string spectra}
An open string has boundaries so that one needs to impose a boundary condition. There are two boundary conditions one can impose:
\begin{itemize}
\item \textbf{Neumann boundary condition}:  $\partial_\sigma X^\mu = 0 \ \ \ \ {\rm at}\ \sigma=0,\pi$
\item \textbf{Dirichlet boundary condition}:  $X^\mu = c^\mu$ (constant) {at} $\sigma=0,\pi$
\end{itemize}
\begin{figure}[h]
\centering \includegraphics[width=10cm]{open-boundary}
\caption{Dirichlet (left) and Neumann (right) boundary conditions}
\end{figure}
Like the close string, we take the mode expansion for the open string $X^\mu = X^\mu_L(\sigma^+)
+X^\mu_R(\sigma^-)$ by
%
\begin{align}
X^\mu_L(\sigma^+) &= \frac12 x^\mu +  \a' p^\mu \,\sigma^+ + i\sqrt{\frac{\a'}{2}}\sum_{n\neq 0}
\frac{1}{n}\,\overline{\alpha}_n^\mu\, e^{-in\sigma^+} \ ,\cr
X^\mu_R(\sigma^-) &= \frac12 x^\mu +  \a' p^\mu \,\sigma^- + i\sqrt{\frac{\a'}{2}}\sum_{n\neq 0}
\frac{1}{n}\,{\alpha}_n^\mu\, e^{-in\sigma^-} \ .\label{openmode}\end{align}
%
Note that the second term differs from the closed string by factor of 2. Show that the boundary conditions impose the following requirements
\begin{itemize}
\item Neumann boundary condition  requires $ \alpha_n^a = \overline{\alpha}_n^a$.
\item Dirichlet boundary condition requires  $x^I=c^I,\quad p^I=0,\quad \alpha^I_n = -\overline{\alpha}_n^I$.
\end{itemize}
Actually, this is an essence of the previous probelm.


Now let us study open string mass spectrum in the quantum theory. In the case of open strings, we can define the momentum $\a_0^\mu=\sqrt{2\a'}\,p^\mu$. Show that the light-cone gauge quantization for \eqref{openmode} gives
$$
 2\alpha^-_n = \sqrt{\frac{1}{2\a'}}\frac{1}{p^+}\sum_{m=-\infty}^{\infty}\sum_{i=1}^{D-2} \alpha_{n-m}^i\alpha^i_m~.
$$
Check that $n=0$ can be read off
$$
M^2 = 2p^+p^- - \sum_{i=1}^{D-2}p^ip^i = \frac{1}{\a'}\left(\sum_{n>0}\sum_{i=1}^{D-2} \alpha_{-n}^i\alpha_n^i + \frac{D-2}{2}\sum_{n>0} n\right)~.
$$
%    
%    \section{Prob. 2}
%    \begin{enumerate}
%    \item Start from Green's theorem in Cartesian coordinate and show that the complex version is given by the form given in the lecture.
%    \item Confirm that $\partial\ol\partial \log |z|^2 = 2\pi \delta^2(z)$.
%    \item Show that $\oint \frac{d\ol z}{2\pi i} \frac{1}{\ol z} =-1$.
%    \end{enumerate}
%    
%    
%    \section{}
%    You do not have to consider space-time indices in this problem.
%    \begin{enumerate}
%    \item Express $\nord{X_1X_2X_3X_4}$ in terms of normal products.
%    \item Using the previous result express $X_1X_2X_3X_4$ in terms of normal-ordered products, and confirm that it can be expressed in the form of
%        \begin{align}
%          f[X] = \exp \left[ \frac{1}{2} \int d^2z d^2w\ G(z,w) \frac{\delta}{\delta X(z)} \frac{\delta}{\delta X(w)} \right]
%    \nord{f[X]} \ .  \nonumber
%        \end{align}
%    % \item Express $X_1 \partial X_2 \partial^2 X_3$ in terms of normal-ordered products.
%    \end{enumerate}



\section{Vertex operator and OPE}
\begin{enumerate}
\item  Let us define $T(z) = -\frac{1}{\alpha'}\nord{\partial X^\mu \partial X_\mu}$,
 and $j(z) = a_\mu \partial X^\mu$ in the free scalar theory. Compute the OPE $j(z)X^\nu(w)$, and $T(z)T(w)$.
\item Show that $:e^{ikX}:$ is a primary field of weight $h = \bar h = \a'k^2/4$ in the free scalar theory. In addition, show that $\partial^n X$ ($n\ge 2$) is not a primary field.
\end{enumerate}

\end{document}
