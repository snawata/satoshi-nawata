\documentclass[12pt,a4paper]{article}

\usepackage{macros}
\begin{document}


\centerline{\Large \bf  Homework 11: Due at class on May 26}



\section{Supersymmetric transformation in SUGRA}

Show the action 
\begin{align}
 S = \frac{1}{2\kappa_D^2}\int d^D x\ e\ \left[ R -2i \psi_\mu \Gamma^{\m\n\rho} D_\nu \psi_\rho\right] \ ,
\end{align}
is invariant under the supersymmetric transformation:
\begin{align}\label{local-susy}
 \delta \psi_\mu^\alpha = D_\mu \epsilon^\alpha \ ,\qquad \delta_\epsilon e^a_\mu = i\ol\epsilon \Gamma^a \psi_\mu~.
\end{align}



\section{S-duality in IIB SUGRA}
Show that the actions (12.16) and (13.3) are equivalent.
Show that the action (13.3) is invariant under $\mathrm{SL}(2, \mathbb{R})$-transformations.


\section{S-duality between Type I and Heterotic SO(32)}
In the lecture, the low-energy effective action of Type I string is given by (13.4).
Also, the low-energy effective action of
Heterotic $\SO(32)$ is given by (13.5).

\begin{itemize}
\item[(i)] Explain why the Yang-Mills action $S_{\textrm{YM}}$ in Type I  (13.4) has $e^{-\Phi}$ whereas $S_{\textrm{YM}}$ in Heterotic $\SO(32)$   (13.5) has $e^{-2\Phi}$.
\item[(ii)]  Show that Type I  (13.4) and Heterotic SO(32)  (13.5) actions are related by the following the field definitions
\begin{align}
G_{\m\n}^{I} \leftrightarrow e^{-\Phi^{H}} G_{\m\n}^{H} ~,&\qquad  \Phi^I \leftrightarrow -\Phi^{H} \cr
\wt G_{(3)}^I \leftrightarrow \wt  H_{(3)}^{H}~ ,&\qquad  A^I \leftrightarrow A^{H} ~.\nonumber
\end{align}
\end{itemize}


\bigskip
(The equation numbers are all in the lecture note.)


\end{document}
