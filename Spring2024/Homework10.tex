\documentclass[12pt,a4paper]{article}
%\usepackage{hyperref} % Use the Charter font for the document text
%\usepackage[UTF8]{ctex}
\usepackage{macros}





\begin{document}\thispagestyle{empty}

\centerline{\Large \bf Homework 10 (Due at class on May 19)}
\section{Poisson Resummation Formula}

The Poisson resummation formula allows us to express sums over a lattice as sums over its dual lattice, a powerful tool in understanding the transformation properties of partition functions under $S$-modular transformations. This problem set explores the proof of this formula and its applications to the behavior of the Dedekind $\eta$-function under modular transformations.

\begin{enumerate}
\item[(i)] Given a function $f(x)$ which approaches zero suitably fast for $x \rightarrow \pm \infty$, i.e. to be precise we consider $f(x)$ to be a Schwartz function. Then the Fourier transformed function $\hat{f}(p)$ is given by
$$
\hat{f}(p)=\int_{-\infty}^{\infty} f(x) e^{-2 \pi i x p} d x .
$$

Prove the Poisson resummation formula for a Schwartz function $f$
$$
\sum_{n \in \mathbb{Z}} f(n)=\sum_{n \in \mathbb{Z}} \hat{f}(n) .
$$

Hint: What are the Fourier coefficients of the periodic function $F(x)=\sum_{n \in \mathbb{Z}} f(x+n)$?
\item[(ii)] Use the Poisson resummation formula to derive a summation identity for the function
$$
f(x)=e^{-a x^2+b x}, \quad \operatorname{Re}(a)>0 .
$$
\item[(iii)]  Use the Jacobi triple product identity
$$
\prod_{n=1}^{\infty}\left(1-q^n\right)\left(1+q^{n-1 / 2} y\right)\left(1+q^{n-1 / 2} y^{-1}\right)=\sum_{n \in \mathbb{Z}} q^{\frac{1}{2} n^2} y^n, \quad|q|<1 y \neq 0,
$$
to derive the infinite sum formula
$$
\eta(q)=\sum_{n \in \mathbb{Z}}(-1)^n q^{\frac{3}{2}\left(n-\frac{1}{6}\right)^2}
$$
for the Dedekind eta function. Apply this result to determine the behavior of the Dedekind eta function $\eta(\tau)$ with respect to modular transformations. 

Hint: In order to find the modular formula $\eta(-1 / \tau)=\sqrt{-i \tau} \eta(\tau)$, split the infinite sum into three suitable infinite sums after Poisson resummation.

\item[(iv)] Show that $\tau_2^{1/2} |\eta(q)|^2$ is invariant under modular transformation.  

\item[(v)] Extend the Poisson resummation formula to a function $f$ on $\mathbb{R}^n$ and a lattice $\Lambda$ in $\mathbb{R}^n$:
  \[
  \sum_{\lambda \in \Lambda} f(\lambda) = \sum_{\lambda^* \in \Lambda^*} \hat{f}(\lambda^*),
  \]
  where $\Lambda^*$ is the dual lattice. Define $\hat{f}$ in this context.

\end{enumerate}


\section{Even, Unimodular Lattices and T-duality}

In the context of toroidal compactifications of string theory and the heterotic string, a significant subsector is represented by left-moving and right-moving bosons with a non-trivial zero-mode sector. To facilitate a deeper understanding of such conformal field theories (CFTs), we explore a CFT comprising $N$ left-moving and $M$ right-moving bosons. The zero modes $\left(p_L, p_R\right)$ of these bosons are valued in a lattice $\Gamma_{N, M} \subset \mathbb{R}^{N+M}$. This lattice is equipped with a Lorentzian metric defined as $\left\|\left(p_L, p_R\right)\right\|^2 = p_L^2 - p_R^2$ and is integral, meaning that all scalar products between lattice points yield integers. We define the rank of the lattice as $N+M$ and its signature as $N-M$. A lattice is termed even (type II) if $\|x\|^2$ is even for all $x \in \Gamma_{N, M}$, and odd (type I) otherwise. A unimodular, or self-dual, lattice has a unit cell volume of 1. We assume that $\Gamma_{N, M}$ is both even and unimodular.

\begin{itemize}
    \item[(i)] Calculate the partition function $Z(\tau, \bar{\tau})$ of this CFT.
    \item[(ii)] Demonstrate that $Z(\tau, \bar{\tau})$ is modular invariant. \textit{Hint: Use Poisson resummation.}
    \item[(iii)] Construct an example of an even, unimodular Lorentzian lattice $\Gamma_{1,1}$. Discuss whether a Euclidean lattice $\Gamma_{2,0}$ of this type exists and justify your answer.
    
    \item[(iv)] Consider the positive, even, unimodular root-lattice of $\mathrm{E}_8$, generated by basis vectors $e_i \in \mathbb{R}^8, i=1,2, \ldots, 8$, with the Cartan matrix $C_{ij}=e_i \cdot e_j$,
    \[
    C_{ij} = \left(\begin{array}{cccccccc}
    2 & -1 & 0 & 0 & 0 & 0 & 0 & 0 \\
    -1 & 2 & -1 & 0 & 0 & 0 & 0 & 0 \\
    0 & -1 & 2 & -1 & 0 & 0 & 0 & -1 \\
    0 & 0 & -1 & 2 & -1 & 0 & 0 & 0 \\
    0 & 0 & 0 & -1 & 2 & -1 & 0 & 0 \\
    0 & 0 & 0 & 0 & -1 & 2 & -1 & 0 \\
    0 & 0 & 0 & 0 & 0 & -1 & 2 & 0 \\
    0 & 0 & -1 & 0 & 0 & 0 & 0 & 2
    \end{array}\right).
    \]
    Verify that this lattice is even and unimodular and identify a set of vectors $e_i$ that realize the above Cartan matrix.
    \item[(v)] The orthogonal group $\mathrm{O}(N, M; \mathbb{R})$ acts on lattices of rank $N+M$ and signature $N-M$. Verify that the properties of being even and unimodular are invariant under this transformation, and argue that the partition function $Z(\tau, \bar{\tau})$ remains unchanged when replacing the lattice $\Gamma_{N, M}$ with $\Lambda \cdot \Gamma_{N, M}$, where $\Lambda \in \mathrm{O}(N, M; \mathbb{Z})$ is a Lorentz transformation and a lattice automorphism. Such transformations are known as $T$-duality.
    
    \item[(vi)] The Hasse-Minkowski classification of unimodular, indefinite lattices states that, up to isomorphism, there is exactly one lattice for a given rank, signature and type (see for example J.-P. Serre, A course in arithmetics, theorem V.2.2.5 and theorem V.2.2.6). Use this result to argue why the moduli space $\mathcal{M}_{N, M}$ of physically inequivalent lattices is given by $(N, M>0)$ 
    \[
    \mathcal{M}_{N, M} = \frac{\mathcal{M}_{N, M}^0}{\mathrm{O}(N, M; \mathbb{Z})}, \quad \mathcal{M}_{N, M}^0 = \frac{\mathrm{O}(N, M; \mathbb{R})}{(\mathrm{O}(N; \mathbb{R}) \times \mathrm{O}(M; \mathbb{R}))}~.
    \]
Find its dimension. Further, describe the moduli spaces $\mathcal{M}_{1,1}^0$ and $\mathcal{M}_{1,1}$.
\end{itemize}


\end{document}