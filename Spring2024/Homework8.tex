\documentclass[12pt,a4paper]{article}
%\usepackage{hyperref} % Use the Charter font for the document text
%\usepackage[UTF8]{ctex}

\usepackage{macros}


\begin{document}\thispagestyle{empty}

\centerline{\Large \bf Homework 8 (Due at class on May 5)}


\section{Bosonic string on a circle}


\subsection{}
The torus partition function of the bosonic string on a circle $S^1$ of radius $R$ is given by
\begin{align}
Z^{25} &= \Tr~ q^{L_0 -  1 / 24} \bar q^{\overline L_0 - 1 / 24} ,	\cr
	&=  \left| \eta(q)\right| ^ {-2}
		\sum_{n, w} q^{\frac {\a'}{4} p_R^2}\bar q ^ {\frac {\a'}{4} p_L^2}~. \label{torus-PF}
\end{align}
where $n$ are KK momenta and $w$ are winding numbers.
If we include the non-compact space $\bR^{1,24}$, we have to multiply the partition function of  the non-compact direction
$$
Z^{1,24} =\textrm{const}\times |\eta(q)|^{-46}
$$
By expanding out the Dedekind $\eta$-functions in $Z^{1,24}Z^{25}$, show that each term means the right hand sides of the mass formula and the level matching condition:
\begin{align}
M^2&=\frac{n^2}{R^2}+\frac{w^2R^2}{\a'^2}+\frac2{\a'}(N+\overline N-2)\cr
nw&=N-\overline N~.\nonumber
\end{align}

\subsection{}
By using the Poisson resummation formula,
$$
\sum_{n\in \bZ}\exp(-\pi a n^2+2\pi i b n)=a^{-1/2}\sum_{m\in\bZ}\exp\Big[-\frac{\pi(m-b)^2}{a}\Big]~,
$$
show that \eqref{torus-PF} is modular-invariant.

\subsection{}
Show that the partition function \eqref{torus-PF} of the theory at the self-dual radius $R=\sqrt{\a'}$ can be written as
$$
Z^{25}=|\chi_1(q)|^2+|\chi_2(q)|^2\,, \qquad  \textrm{where} \quad \chi_1= \frac 1 {\eta} \sum_n q^{n^2}
\quad \chi_2=\frac 1 {\eta} \sum_n q^{(n + 1 / 2)^2}
$$
The $\chi_i$ are the characters of the $\SU(2)$ affine Lie algebra with level $k=1$. By expanding this expression out find the massless states from above.

\subsection{}
Show that the currents in the bosonic string theory defined by
$$
j^\pm(z)=j^1(z)\pm i \, j^2(z):=e^{\pm 2iX^{25}(z)/\sqrt{\a'}} \qquad j^3(z):=i\, \partial X^{25}(z)/\sqrt{\a'}~,
$$
satisfy the OPEs
$$
j^a(z) j^b (0) \sim \frac { \delta^{ab} } {2z^2}
		+ \frac {i {\epsilon^{abc}} j^c(0)} {z}~.
$$
From the OPEs,
show that the oscillator modes of the currents
$$
j^a(z)=\sum_{m\in\bZ}\frac{j^a_m}{z^{m+1}}~,
$$
satisfy
$$
[j_m^a,j^b_n]=\frac  m2\delta_{m+n,0}\delta^{ab} + i \e^{abc} j^c_{m+n}~.
$$
This infinite-dimensional algebra is called the \textbf{$\SU(2)$ affine Lie
algebra with level $k=1$}. (Check that the zero modes satisfy the $\SU(2)$ Lie algebra.)

\section{R-R field strengths and T-duality in Type II}
Let
$$\{\G^\m,\G^\n\} = 2\eta^{\m\n}\qquad  \m= 0,\cdots,9 $$
be the Clifford algebra of SO(1,9) gamma matrices.  The gamma matrices have the
following hermiticity property,
$$
(\G^\mu)^\dagger=-\G^0\G^\m(\G^0)^{-1}~.
$$
By using the chirality operator
$\G_{11}=\G^0\G^1\cdots \G^9$,
chiral spinors are  defined by
\be\label{chirality}
\G_{11} \psi_\pm= \pm \psi_{\pm}~.
\ee
Show that
$$ \overline \psi_\pm\G_{11}=\mp \overline \psi_\pm$$
where  $\overline \psi_\pm = \psi_\pm^\dagger \G^0 $.

We define the R-R field strengths $G^{\m_1\cdots\m_{p+2}}$ as spinor bilinears
\be\label{FS}
\textrm{IIA}:~ \overline\psi_-^L\G^{\m_1\cdots\m_{p+2}}\psi_+^R~,\qquad\qquad \textrm{IIB}:~ \overline\psi_+^L\G^{\m_1\cdots\m_{p+2}}\psi_+^R~,
\ee
 where $\psi^R$ ($\psi^L$) comes from the right (left) movers
and
 $$\G^{\m_1\cdots\m_{p+2}}=\G^{[\m_1}\cdots\G^{\m_{p+2}]}  $$
 is the antisymmetric product of ($p+2$) gamma matrices.
Using the chirality \eqref{chirality} of the spinors, determine for which values of $p$ the R-R field strengths \eqref{FS} are non-zero.



In the lecture, we learn that T-duality of the 9th direction in Type II theory acts the left-moving fermion mode
 $$\psi_n^9 \to -\psi^{\prime 9}_n~,\qquad n\in\bZ~.$$
The action of duality on the spinor fields is of the form
\be\label{t-dual}
\bar\psi^L \to \bar\psi^L\beta_9 ~,\qquad
\psi^R  \to \psi^R
\ee
where $\beta_9 =\Gamma_{11} \Gamma^9$. Show that
$$
\{\beta_9, \G^9\}=0~, \qquad [\beta_9,\G^\mu]=0~, \quad \textrm{for}\quad \mu\neq9~.
$$
Using the effect of \eqref{t-dual} on the R-R field strengths \eqref{FS}, show that T-duality transforms the R-R field strengths in IIA to those in IIB, and vice versa.

%\section{D-branes in Type II and T-duality}
%Two D-branes intersect orthogonally over a $p$-brane if they share $p$ directions with the remaining directions wrapping different directions. For example a D5-brane extending in the directions $x^0,x^1,\cdots,x^5$ and a D3-brane extending in $x^0,x^1,x^2,x^6$ intersect orthogonally over an 2-brane $x^0,x^1,x^2$.
%\begin{table}[h]\centering
%\begin{tabular}{c|ccccccccccc}
%&0&1&2&3&4&5&6&7&8&9\\\hline
%D3&$\times$&$\times$&$\times$&&&&$\times$&&&\\
%D5&$\times$&$\times$&$\times$&$\times$&$\times$&$\times$
%\end{tabular}
%\end{table}
%
%\noindent In such cases we can divide the spacetime directions into 4 sets, {NN, ND, DN, DD} according to whether the coordinate $X^\mu$ has Neumann (N) or Dirichlet (D) boundary conditions on the first or second brane. In the example of the D3-D5 system for a string stretching from the D3-brane to the D5-brane: NN = $\{x^0,x^1,x^2\}$, ND=$\{x^6\}$, DN = $\{x^3,x^4,x^5\}$,  DD = $\{x^7,x^8,x^9\}$.
%
%
%\subsection{} Show that the numbers (\#NN+\#DD) and (\#ND+\#DN) are invariant under T-duality, where \#NN is the number of NN directions, etc.
%
%\subsection{} List all orthogonal intersections in IIB string theory that have (\#ND+\#DN) = 4 and contain at least one D3-brane. Show that all these configurations are T-dual to the following D1-D5 configuration:
%\begin{table}[h]\centering
%\begin{tabular}{c|ccccccccccc}
%&0&1&2&3&4&5&6&7&8&9\\\hline
%D1&$\times$&$\times$&&&&\\
%D5&$\times$&$\times$&$\times$&$\times$&$\times$&$\times$
%\end{tabular}
%\end{table}

\end{document}
