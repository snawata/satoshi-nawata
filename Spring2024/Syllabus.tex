\documentclass[12pt,a4paper]{article}
\usepackage{hyperref} % Use the Charter font for the document text
%\usepackage[UTF8]{ctex}
\usepackage{fullpage}
\usepackage{color}

\usepackage{graphicx}


\begin{document}\thispagestyle{empty}

\centerline{\Large \bf Syllabus}

\begin{description}
\item{\bf Course name:} Introduction to String Theory (PHYS630049)
\item{\bf Instructor:} Satoshi Nawata, Physics S422, Jiangwan \href{mailto:snawata@fudan.edu.cn}{snawata@fudan.edu.cn}
\item{\bf Hours:} 9:55 -- 12:30 Friday
\item{\bf Place:} JB103
\item{\bf Office hour:} Whenever, but email us beforehand
\item{\bf Prerequisites:}

Quantum Mechanics, Electrodynamics, General Relativity,

Basic knowledge of quantum field theory.
\item{\bf About the course:}


The basic idea of String Theory is that elementary particles are excitations of a string. Each quantum excitation of the string behaves like an elementary particle, and closed strings have a massless spin-2 particle in their spectrum, which made string theory a promising candidate for a quantum theory of gravity. It has also been remarkably successful as a theoretical framework capable of influencing other fields of physics and mathematics.

This course will introduce to basics of string theory.
String theory is very broad and it is still rapidly developing. Therefore, it is impossible to cover all the topics within one semester. However, we will study selected advanced topics too.
First we will learn quantization of bosonic strings and superstrings in
RNS formalism, two-dimensional conformal field theories, supergravity, string scattering
amplitudes, and D-branes. We also introduce the
web of dualities, black holes in string theory
and AdS/CFT correspondence.



Undergraduate students are also very welcome to audit this course.




\item{\bf Main content:}
\begin{itemize}
\item bosonic strings
\item superstrings
\item string duality and branes
\item black holes, AdS/CFT, selected advanced topics.
\end{itemize}



\item{\bf Main references:} Polchinski, String Theory I, II

It is not good idea to try to read through Polchinski in one semester. There are more concise and elementary introductions to string theory. I will use combinations of these references.

Lecture notes by Kazuo Hosomichi, available on \href{https://sites.google.com/view/kazuohosomichi/home/string-theory}{\color{blue} his website}.

David Tong, String theory \href{http://arxiv.org/abs/0908.0333}{\color{blue} [arXiv:0908.0333]}

\item{\bf Supplementary references:}

There are too many references on string theory. I just list the famous books on string theory.
\begin{itemize}
\item Green, Schwarz and Witten, Superstring Theory I, II

\item Becker, Becker, Schwarz, String Theory and M-Theory: A Modern Introduction

\item  Zwiebach, A First Course in String Theory
\end{itemize}


For more information, you can refer to \href{https://www.stringwiki.org/w/index.php?title=String_Theory_Wiki}{\color{blue} this website}.

\item{\bf Grading:} Grade will be determined based on homework given every week.


\end{description}



\end{document}
